\chapter{Una parola è già abbastanza}

\begin{openingQuote}
  \centering

  Al Saṅgha occidentale da poco giunto in Inghilterra, 1979.
\end{openingQuote}

Tutto quello che insegnerete non potrà andare al di là di \emph{sīla},
\emph{samādhi} e \emph{paññā} o, per usare un'altra classificazione
usuale, di moralità, meditazione e generosità. La gente è piuttosto
complicata. Dovete guardare le persone alle quali insegnate e capirle.
Siccome sono complicate, dovete dar loro cose alle quali possano
relazionarsi. Non va bene dire solo «~Lasciar andare, lasciar andare!~»
Per il momento, mettetelo da parte. È come parlare alla gente anziana in
Thailandia. Se parlate bruscamente si risentiranno. Se lo faccio io,
d'accordo. Se lo sentono da me, a loro piace, altrimenti si arrabbiano.
Potete essere in grado di parlare bene, ma questo ancora non significa
che siete abili. Giusto, Sumedho? È così, non è vero?

Ajahn Sumedho: È così. Alcuni \emph{bhikkhu} dicono la verità, ma non lo
fanno in modo abile, e i laici non vogliono ascoltare. Non posseggono
mezzi abili.

Ajahn Chah: Giusto. Non hanno una ``tecnica''. Non hanno la tecnica del
parlare. È come costruire. Posso costruire delle cose, ma non avere la
tecnica per farne di belle e che durino nel tempo. Posso parlare, tutti
possono parlare, ma è necessario possedere mezzi abili per sapere cosa
sia appropriato dire. Allora anche una sola parola può essere di
beneficio. Altrimenti con le parole puoi causare problemi.

Ad esempio, qui la gente ha imparato un sacco di cose. Non esaltate la
vostra via: «~La mia via è giusta! La tua è sbagliata!~» Non fatelo! E
non cercate nemmeno di essere profondi. Potreste condurre la gente alla
follia. Dite solamente: «~Non scartare le altre vie che hai imparato;
per il momento, però, mettile da parte per favore e concentrati sul modo
in cui stiamo praticando ora.~» La consapevolezza del respiro è ad
esempio una cosa che si può insegnare a tutti. Insegnate a concentrarsi
sul respiro che entra ed esce. Limitatevi a insegnare in questo modo, e
lasciate che la gente arrivi a comprenderlo. Quando diventate abili
nell'insegnare una cosa, la vostra abilità si svilupperà da sé e sarete
in grado di insegnarne altre. Comprendendo per bene una cosa, la gente
ne può capire molte. Succede da sé. Se però cercate di insegnare molte
cose, non ne comprenderà davvero nemmeno una. Se invece ne evidenziate
bene una, allora molte saranno le cose che potranno conoscere con
chiarezza.

Come quei cristiani che sono venuti oggi. Dicevano una sola cosa, una
cosa sola ma molto significativa: «~Un giorno ci rincontreremo tutti nel
posto della Verità Ultima.~» Questa sola affermazione era sufficiente.
Erano le parole di una persona saggia. Non importa quale genere di
Dhamma impariamo, se non realizziamo nei nostri cuori la Verità Ultima,
il \emph{paramatthadhamma}, non arriveremo alla Realizzazione.

Ad esempio, Sumedho potrebbe insegnarmi. Io devo poi impadronirmi di
quella conoscenza e cercare di metterla in pratica. Quando Sumedho
m'insegna, io capisco, ma non è una comprensione reale o profonda,
perché non ho ancora praticato. Quando pratico e realizzo il frutto
della pratica, arriverò al punto e conoscerò il reale significato di
tale frutto. Allora potrò dire che conosco Sumedho. Vedrò Sumedho in
quel punto. Quel punto è Sumedho. Poiché egli insegna quel punto, questo
è Sumedho. Quando insegno sul Buddha, è così. Dico che il Buddha è quel
punto. Il Buddha non è negli insegnamenti. Quando la gente lo sente,
sbigottisce: «~Il Buddha non insegnò queste cose?~» Si, lo fece, ma per
parlare della Verità Ultima. La gente non può ancora riuscire a capirlo.

Per far riflettere le persone che sono venute oggi ho parlato in questo
modo. Questa mela è una cosa che potete vedere con i vostri occhi. Il
sapore della mela non lo potete conoscere guardandola. La mela, però, la
vedete. Sentivo che erano in grado di capirlo. Il sapore non potete
vederlo, ma è lì. Quando lo conoscerete? Quando prenderete la mela e la
mangerete. Il Dhamma che insegniamo è come la mela. La gente ascolta, ma
non conosce veramente il sapore della mela. Quando praticano, allora
possono conoscerlo. Il sapore della mela non può essere percepito con
gli occhi, e la Verità del Dhamma non può essere appresa con le
orecchie. C'è una conoscenza, è vero, ma non raggiunge la realtà. C'è
bisogno di metterla in pratica. Allora sorge la saggezza e si conosce
direttamente la Verità Ultima. È qui che si vede il Buddha. Questo è il
Dhamma profondo. L'ho paragonato a una mela e l'ho offerto a quel gruppo
di cristiani affinché ascoltassero e ci pensassero su.

Quel discorso era un po' ``salato''.\footnote{Con il senso di ``duro'',
  ``diretto''.} Salato è buono. Dolce è buono. Aspro è buono. Molti modi
differenti di insegnare sono buoni. Bene, se avete qualcosa da dire,
chiunque di voi si senta per favore libero di parlare. Presto non avrete
l'opportunità di discutere sulle cose. Sumedho è a corto di cose da
dire, probabilmente.

\begin{quote}
A.S.: Sono stufo di spiegare cose alla gente.

A.C.: Non esserlo. Non puoi essere stufo.

A.S.: Sì, ho chiuso.

A.C.: L'insegnante in capo non può permettersi di essere stufo. C'è un
sacco di gente che cerca di raggiungere il \emph{Nibbāna}, e loro
dipendono da te. A volte insegnare riesce facile. Altre volte non sai
cosa dire. Mancano le parole, non viene fuori niente. Oppure è solo che
non vuoi parlare? È un buon addestramento per te.

A.S.: La gente qui intorno è piuttosto buona. Non è violenta, meschina o
molesta. Non siamo sgraditi ai sacerdoti cattolici e la gente chiede
cose che riguardano Dio. Vogliono sapere cos'è Dio e cos'è il
\emph{Nibbāna}. Alcuni pensano addirittura che il buddhismo insegni il
nichilismo e che voglia distruggere il mondo.

A.C.: Questo significa che la loro comprensione non è completa o matura.
Temono che tutto finisca, che il mondo giunga alla fine. Concepiscono il
Dhamma come vuoto e nichilistico e perciò sono sfiduciati. La loro
strada conduce solo alle lacrime. Non avete visto cosa succede quando la
gente teme la ``vacuità''? I capifamiglia cercano di aumentare i loro
possedimenti e di controllarli, come topi. Questo forse li protegge dal
senso di vuoto dell'esistenza? Anche loro finiranno su una pira
funeraria, e tutto andrà perduto. Mentre sono in vita, però, cercano
d'aggrapparsi alle cose, tutti i giorni, con il timore che andranno
perdute, cercando di evitare la vacuità. Soffrono così tanto? Certo,
soffrono davvero. Questo significa non comprendere la reale mancanza di
sostanzialità delle cose, la loro vacuità. Non comprendendolo, la gente
è infelice.

Siccome le persone non osservano se stesse, non capiscono veramente cosa
succede nella vita. Come fermare quest'illusione? La gente crede che
``questo sono io'', che ``questo è mio''. Se parli del non-sé, se dici
che io e mio non esistono, la persone sono pronte a discutere fino al
giorno della loro morte. Perfino il Buddha, dopo aver ottenuto la
Conoscenza, si sentì affaticato a pensarci. Appena ottenne
l'Illuminazione pensò che sarebbe stato estremamente difficile spiegare
la via agli altri. Alla fine, però, comprese che questo atteggiamento
non era giusto. Se non insegniamo a queste persone, a chi insegneremo?
Questa è la mia domanda, questa è la domanda che ero solito porre a me
stesso quando ero stufo e non volevo più insegnare: a chi insegneremo,
se non insegniamo agli illusi? Davvero non c'è posto nel quale andare.
Quando siamo stufi e vogliamo scappare dai discepoli e vivere soli,
siamo degli illusi.
\end{quote}

Un \emph{bhikkhu}: Potremmo essere \emph{Paccekabuddha}.\footnote{\emph{Paccekabuddha}:
  Una persona che, come il Buddha, ha conseguito il Risveglio senza
  beneficiare dell'insegnamento di un maestro, ma che non possiede
  sufficienti \emph{pāramī} per insegnare agli altri la pratica che
  conduce all'Illuminazione e, dopo averla realizzata, vive in
  solitudine.}

A.C.: È una buona cosa. Ma non è davvero giusto essere dei
\emph{Paccekabuddha} se volete solo fuggire dalle situazioni.

A.S.: Solo vivendo naturalmente, in un ambiente semplice, potremmo
essere dei \emph{Paccekabuddha}. Oggigiorno però è impossibile.
L'ambiente in cui viviamo non ce lo consente. Dobbiamo vivere come
monaci.

A.C.: A volte dovete vivere prima in una situazione come quella in cui
vi trovate ora, con alcuni fastidi. Per spiegarlo in modo semplice,
alcune volte sarete un Buddha onnisciente (\emph{sabaññū}), altre volte
un \emph{pacceka}. Dipende dalle condizioni. Parlare di questo tipo di
esseri è parlare della mente. Non è che uno nasce \emph{pacceka}. Questo
significa ``spiegare per mezzo della personificazione degli stati
mentali'' (\emph{puggalādhitthāna}). Un \emph{pacceka} dimora in modo
equanime senza insegnare. Da questo non deriva grande beneficio. Ma
quando si è in grado d'insegnare agli altri, ecco che si manifesta un
Buddha onnisciente. Sono solo metafore. Non siate nulla! Non siate
proprio nulla! Essere un Buddha è un fardello. Essere un \emph{pacceka}
è un fardello. Non desiderate essere. «~Io sono il monaco Sumedho.~»
«~Io sono il monaco ānando.~» In questo modo c'è sofferenza, credere che
voi esistiate così. ``Sumedho'' è solo una convenzione. Capite? Credere
di esistere realmente porta sofferenza. Se c'è Sumedho, quando qualcuno
lo critica Sumedho si arrabbia. ānando si arrabbia. Questo è quel che
succede quando queste cose le ritenete reali. Se non c'è nessun ānando o
nessun Sumedho, allora lì non c'è nessuno. Nessuno risponde al telefono.
Suona, suona, ma nessuno alza la cornetta. Non diventate nulla. Se
nessuno diventa qualcosa, non c'è sofferenza.

Se crediamo di essere qualcosa o qualcuno, tutte le volte che il
telefono suona alziamo la cornetta, e siamo coinvolti. Come possiamo
liberarci da questo? Dobbiamo osservare con chiarezza e sviluppare la
saggezza, così che non c'è nessun ānando, nessun Sumedho ad alzare la
cornetta. Se siete ānando o Sumedho e rispondete al telefono, sarete
coinvolti nella sofferenza. Così, non siate Sumedho. Non siate ānando.
Riconoscete questi nomi come esistenti solo al livello della
convenzione. Se qualcuno dice che siete buoni, non siatelo. Non pensate:
«~Sono buono.~» Se qualcuno dice che siete cattivi, non pensate «~Sono
cattivo.~» Non cercate di essere qualcosa. Conoscete quel che sta
avvenendo, ma non attaccatevi neanche alla conoscenza.

La gente non ci riesce. Non capisce di cosa si tratti. Quando sentono
queste cose, si confondono e non sanno cosa fare. Tempo fa ho utilizzato
l'analogia del piano superiore e di quello inferiore. Se scendete dal
piano di sopra, siete al piano di sotto e vedete il piano di sotto. Se
andate di nuovo al piano di sopra, vedete il piano di sopra. Lo spazio
fra i due, quello in mezzo, non lo vedete. Questo significa che il
\emph{Nibbāna} non si vede. Vediamo le forme degli oggetti fisici, ma
non vediamo che ci aggrappiamo, ci aggrappiamo al piano di sopra e a
quello di sotto, al divenire e alla nascita. Divenire in continuazione.
Il luogo senza divenire è vuoto. Quando cerchiamo di insegnare alle
persone il posto che è vuoto, dicono solo: «~Lì non c'è niente.~» Non
capiscono. È difficile. Per capirlo si deve praticare veramente.

Fin dal giorno della nostra nascita abbiamo fatto affidamento sul
divenire, sull'attaccamento a noi stessi. Quando qualcuno parla di
non-sé, è troppo strano. Non riusciamo a modificare le nostre percezioni
con facilità. Così, è necessario fare in modo che la mente veda mediante
la pratica, e allora potremo crederci: «~Oh! È vero!~»

Le persone provano felicità quando pensano: «~Questo è mio, questo è
mio!~» Però, quando ciò che è ``mio'' andrà perduto, piangeranno. Questo
è il sentiero che fa nascere la sofferenza. Possiamo vederlo. Se non c'è
alcun ``mio'' o ``io'', mentre siamo in vita possiamo usare le cose
senza attaccarci a esse come se fossero nostre. Se vanno perdute o si
rompono, è semplicemente un fatto naturale; non le vediamo come nostre o
di chiunque altro, e non abbiamo la concezione del sé o dell'altro da
sé. Non stiamo parlando di un folle, ma di una persona diligente, che
conosce davvero ciò che è utile, e in molti modi differenti. Quando però
gli altri lo guarderanno e tenteranno di comprenderlo, vedranno un
pazzo.

Quando Sumedho guarda i laici, li considera ignoranti, bambini. Quando i
laici guardano Sumedho, pensano che sia uscito di senno. Voi non vi
interessate affatto alle cose per le quali i laici vivono. Per dirla in
altro modo, un \emph{arahant}\footnote{\emph{Arahant}: Letteralmente, un
  ``Meritevole''; una persona la cui mente è libera dalle contaminazioni
  (\emph{kilesa}). È anche un titolo del Buddha e il livello più alto
  dei suoi Nobili Discepoli.} e uno squilibrato sono simili. Pensateci.
Se la gente guarda un \emph{arahant}, pensa che sia folle. Se gli
imprecate contro, lui non se ne cura. Qualsiasi cosa gli diciate, non
reagisce, come un folle. È folle, ma ha la consapevolezza. Uno che è
folle davvero può anche non arrabbiarsi quando inveite contro di lui, ma
è perché non sa cosa stia succedendo. Uno che guardi un \emph{arahant} e
un matto potrebbe considerarli identici. Sul gradino più basso c'è un
pazzo, su quello più alto di tutti c'è un \emph{arahant}. Se guardate le
manifestazioni esteriori, il più alto e il più basso sono simili, ma la
loro consapevolezza è del tutto diversa, come il senso che attribuiscono
alle cose.

Pensateci. Quando qualcuno dice qualcosa che dovrebbe farvi arrabbiare e
voi lasciate andare, la gente può pensare che siete matti. Perciò,
quando insegnate questo genere di cose, gli altri con capiscono con
facilità. Hanno bisogno di interiorizzarle per comprenderle davvero. Ad
esempio, in questo paese la gente ama la bellezza. Non vogliono
ascoltarvi se vi limitate a dire: «~No, queste cose in realtà non sono
belle.~» Se parlate di ``invecchiare'' non sono contenti, e neanche
della ``morte'' vogliono sentir parlare. Significa che non sono pronti
per comprendere. Se non vi credono, non pensiate che sia una loro colpa.
È come cercare di fare un baratto, di dare loro qualcosa di nuovo per
rimpiazzare ciò che hanno, ma senza che loro riescano a vedere alcun
valore in quello che state offrendo. Se ciò che avete fosse
evidentemente di grandissimo valore, certamente lo accetterebbero. Però,
come mai ora non vi credono? Non avete sufficiente saggezza. Non
arrabbiatevi: «~Che cosa c'è che non va in te, sei fuori di testa?~» Non
fatelo. Dovete prima di tutto insegnare a voi stessi, impiantare la
verità del Dhamma in voi stessi e sviluppare il giusto modo per
presentarla agli altri, e allora l'accetteranno.

A volte gli \emph{ajahn} insegnano ai discepoli, ma i discepoli non
credono a quello che dicono gli \emph{ajahn}. Questo potrebbe irritarvi,
ma invece d'irritarvi è meglio che individuiate la ragione per cui non
vi credono: quel che offrite ha per loro poco valore. Se offrite
qualcosa di maggior valore rispetto a ciò che hanno, certamente lo
vorranno. È in questo modo che dovreste pensare, quando state per
arrabbiarvi con i vostri discepoli, così potrete bloccare la vostra
rabbia. Non è affatto divertente arrabbiarsi. Per fare in modo che i
suoi discepoli capissero il Dhamma, il Buddha insegnò un solo sentiero,
ma con varie caratteristiche. Non utilizzò una sola forma di
insegnamento, né presentò il Dhamma nello stesso modo a tutti. Però,
insegnò con il solo scopo di far trascendere la sofferenza. Tutti i
generi di meditazione che insegnò avevano quest'unico obiettivo.

Già c'è molto nella vita delle persone in Europa. Se cercate di mettere
addosso alla gente qualcosa di grande e complicato, potrebbe essere
troppo. Che cosa dovreste fare, allora? Avete dei suggerimenti? Se
qualcuno ha qualcosa da dire, questo è il momento di parlare. Non avrete
di nuovo questa opportunità. Se non avete niente su cui discutere, se
avete esaurito i vostri dubbi, potrei pensare che siate dei
\emph{Paccekabuddha}. In futuro qualcuno di voi sarà un insegnante di
Dhamma. Insegnerete agli altri. Quando insegnate agli altri, insegnate
anche a voi stessi. Chi è d'accordo? La vostra abilità e la vostra
saggezza aumenteranno. La vostra contemplazione aumenterà. Quando ad
esempio insegnate una cosa a qualcuno per la prima volta, iniziate a
domandarvi perché è così, qual è il significato di quella cosa.
Cominciate a pensare così e poi vorrete contemplare, per scoprire quale
sia il reale significato di essa. Così, insegnando agli altri, insegnate
anche a voi stessi. Se avrete consapevolezza, se starete praticando la
meditazione, sarà così. Non pensiate che state insegnando solo agli
altri. Tenete in considerazione l'idea che state insegnando anche a voi
stessi. Allora non c'è perdita.

A.S.: Sembra che nel mondo la gente stia diventando tutta uguale. Le
idee di classe sociale e di casta vanno scomparendo e modificandosi.
Alcuni credono nell'astrologia e dicono che, entro pochi anni, si
verificheranno dei disastri naturali che causeranno molta sofferenza.
Non so se sia vero, ma pensano che si tratti di una cosa che va oltre le
nostre capacità di controllo, perché la nostra vita è troppo distante
dalla natura e il nostro benessere dipende dalle macchine. Dicono che ci
saranno molti cambiamenti nella natura, terremoti ad esempio, che
nessuno può prevedere.

A.C.: Parlano per far soffrire la gente.

A.S.: Giusto. Se non abbiamo consapevolezza, possiamo soffrire davvero
per queste cose.

A.C.: Il Buddha insegnò in relazione al presente. Non ci consigliò di
preoccuparci di quel che potrebbe accadere entro due o tre anni. In
Thailandia la gente viene da me e dice: «~Oh, Luang Por, stanno
arrivando i comunisti, che cosa faremo?~»\footnote{In quegli anni nei
  paesi confinanti si stavano affermando i regimi comunisti, e i
  thailandesi si sentivano minacciati da un'invasione.} Io chiedo:
«~Dove sono questi comunisti?~» «~Beh, arriveranno da un giorno
all'altro~», rispondono. I comunisti ci sono da quando siamo nati. E qui
mi fermo. I ``comunisti'' vengono eliminati da un atteggiamento che
riconosce che di ostacoli e di difficoltà ce ne sono sempre. In questo
modo non siamo distratti. Parlare di quel che succederà tra quattro o
cinque anni è guardare troppo lontano. Dicono: «~Tra due o tre anni la
Thailandia sarà comunista!~» È da quando sono nato che vi è la
percezione che ci siano dei ``comunisti'' e, perciò, è come se avessi
sempre discusso con loro. La gente però non capisce di cosa stia
parlando.

È la verità. L'astrologia può forse dire che cosa succederà entro due
anni. Quando però parliamo del presente, non sa che fare. Il buddhismo
si occupa delle cose nel momento presente e a prepararci per bene a
tutto quel che può succedere. Qualsiasi cosa possa succedere nel mondo,
non dobbiamo preoccuparcene troppo. Noi pratichiamo solo per sviluppare
la saggezza nel presente e per fare quello che è necessario ora, non
domani. È meglio, no? Possiamo metterci ad aspettare un terremoto che
potrebbe giungere fra tre o quattro anni, ma in effetti le cose tremano
adesso. Gli Stati Uniti stanno realmente tremando. La mente delle
persone è così agitata, questo è il terremoto, proprio qui e ora. La
gente però non se ne accorge. Grandi terremoti accadono di rado, ma la
terra delle nostre menti sta sempre tremando, ogni giorno, ogni momento.
Durante la mia vita non ho mai sperimentato un grande terremoto, ma
questo genere di tremori si verifica sempre, facendoci agitare e
turbinare ovunque. Questo il Buddha voleva che guardassimo. Forse, però,
non è quel che la gente vuole sentire.

Le cose succedono perché ci sono delle cause. Cessano in quanto sono le
cause a cessare. Non c'è bisogno di preoccuparci delle previsioni
astrologiche, anche se a tutti piace porsi questo tipo di domande.
Possiamo solo sapere cosa sta succedendo ora. In Thailandia i funzionari
del governo vengono da me e dicono: «~Il paese sarà tutto comunista. Che
faremo se succede?~» «~Siamo nati. Che facciamo? Non ho riflettuto molto
su questo problema. Ho sempre pensato che i ``comunisti'' mi sarebbero
stati alle calcagna, fin dal giorno in cui sono nato.~» Quando rispondo
così, non hanno nulla da dire. Questo li blocca.

La gente può parlare dei pericoli che potrebbero essere causati dalla
conquista comunista, ma il Buddha ci insegnò a prepararci proprio ora a
essere consapevoli e a contemplare i pericoli innati nella vita. Questo
è il grande problema. Non siate distratti! Se fate affidamento su quello
che l'astrologia dice che succederà entro un paio d'anni, non arrivate
al nocciolo della questione. Se fate affidamento sulla ``buddhologia''
non dovete rimuginare sul passato né preoccuparvi del futuro, ma
guardare il presente. Le cause nascono nel presente, perciò osservatele
nel presente.

Chi dice queste cose sta solo insegnando agli altri a soffrire. Se
qualcuno però parla come io sto facendo ora, la gente dice che è folle.
Nel passato vi è sempre stato del movimento, ma avveniva sempre un po'
per volta, così che non era possibile notarlo. Ad esempio: Sumedho,
appena sei nato eri così grande? È il risultato del movimento e del
cambiamento. Il cambiamento è bene? Certo che lo è. Se non ci fosse
stato movimento o cambiamento, non saresti mai cresciuto. Non c'è
bisogno d'aver paura delle trasformazioni naturali. Se contemplate il
Dhamma, non so a cos'altro possiate aver bisogno di pensare. Se qualcuno
predice cosa avverrà entro qualche anno, non è che prima di fare
qualcosa possiamo solo stare ad aspettar di vedere quel che succede. Non
possiamo vivere così. Qualsiasi cosa si debba fare, va fatta adesso,
senza aspettare che succeda qualcosa di particolare.

Di questi tempi, nel mondo la popolazione è costantemente in movimento.
I quattro elementi sono in movimento. Terra, acqua, fuoco e aria sono in
movimento. Le persone però non vedono che la terra si sta muovendo.
Guardano solo la terra esterna e non vedono alcun movimento. In questo
mondo, se nel futuro la gente si sposerà e resterà assieme più di un
anno o due, gli altri penseranno che c'è qualcosa che in loro non va. La
norma sarà pochi mesi. Le cose sono in costante movimento. È la mente
delle persone che è in movimento. Non avete bisogno di guardare
l'astrologia. Guardate la buddhologia e potete comprenderlo.

«~Luang Por, dove andrete se arrivano i comunisti?~» Dov'è che si può
andare? Siamo nati e dobbiamo avere a che fare con vecchiaia, malattia e
morte. Dove possiamo andare? Dobbiamo restare proprio qui e affrontare
queste cose. Se i comunisti vinceranno, si resterà in Thailandia ad
affrontare la situazione. Non dovranno mangiare riso anche i comunisti?
Perché mai si è allora così spaventati? Se continuate a preoccuparvi di
cosa potrebbe accadere nel futuro, non la smetterete mai. Ci saranno
solo costante confusione e continue congetture. Sumedho, sai cosa
succederà tra due o tre anni? Ci sarà un grande terremoto? Quando la
gente viene a chiedervi queste cose, potete rispondere che non c'è
bisogno di guardare cose lontane, che non si possono conoscere con
certezza; parlate loro dei movimenti e dei tremori che si verificano
sempre, delle trasformazioni che vi hanno consentito di crescere e di
diventare come siete ora.

Il modo di pensare delle persone è che, dopo essere nate, non vogliono
morire. Vi sembra giusto? È come versare acqua in un bicchiere senza
volere che si riempia. Se continuate a versare acqua, non potete
aspettarvi che non si riempia. Così pensa la gente: nasce, ma non vuole
morire. È un modo di pensare corretto? Prendetelo in considerazione. Se
la gente nascesse ma non morisse, ciò porterebbe felicità? Se nessuno di
quelli che vengono al mondo morisse, le cose andrebbero ben peggio. Se
nessuno morisse mai, probabilmente finiremmo per mangiare escrementi!
Dove potremmo mai stare tutti quanti? Sarebbe come versare acqua in un
bicchiere senza mai smettere, ma continuare a volere che non si riempia.
Nasciamo, ma non vogliamo morire.

Come insegnò il Buddha, se davvero non vogliamo morire dovremmo
realizzare ``Ciò che non muore'' (\emph{amatadhamma}). Sapete cosa
significa \emph{amatadhamma}? È Ciò che non muore: sebbene si muoia, se
si ha saggezza è come non morire. Non morire, non essere nati. Così le
cose possono finire. Nascere e desiderare felicità e piacere senza
morire non è assolutamente una cosa corretta. È però quel che la gente
vuole, e così non c'è fine alla loro sofferenza. I praticanti del Dhamma
non soffrono. Bene, praticanti come i normali monaci soffrono ancora,
perché non hanno ancora percorso tutto il Sentiero della pratica. Non
hanno ancora realizzato l'\emph{amatadhamma}, e così soffrono ancora.
Sono ancora soggetti alla morte. L'\emph{amatadhamma} è Ciò che non
muore. Nati da un utero, possiamo evitare la morte? Se non si comprende
che non esiste alcun sé reale, non c'è modo di evitare la morte. Non è
che ``io'' muoio; è che i \emph{saṅkhāra}, seguendo la loro natura, sono
soggetti alla trasformazione. È difficile da capire. La gente non riesce
a pensare in questo modo. C'è bisogno di liberarsi dalla mondanità, come
ha fatto Sumedho. Dovete lasciare la vostra grande casa piena di
comodità e il mondo del progresso, come fece il Buddha. Fu lasciando il
suo palazzo e andando a vivere nella foresta che raggiunse tutto questo.
La vita fatta di piaceri e divertimenti nel palazzo non era la via
dell'Illuminazione. Chi è che vi parla di previsioni astrologiche?

A.S.: Un sacco di gente ne parla, spesso come se fosse solo un
passatempo o un interesse casuale.

A.C.: Se è davvero come dicono, che cosa dovrebbe allora fare la gente?
Stanno offrendo alle persone una via da percorrere? Dal mio punto di
vista, il Buddha insegnò con grande chiarezza. Disse che da quando
nasciamo le cose sulle quali non possiamo essere certi sono molte.
L'astrologia può parlare di mesi o di anni futuri, ma il Buddha indica
il momento della nascita. Predire il futuro può rendere la gente ansiosa
su quello che potrebbe succedere, ma la verità è che l'incertezza è
sempre con noi proprio a partire dal momento della nascita. La gente non
è propensa a credere a questo discorso, o no?

{[}Rivolgendosi a un laico{]} Se hai paura, pensa a questo. Supponi di
essere condannato per un crimine che prevede la pena capitale, e che
entro sette giorni avrà luogo l'esecuzione. Che cosa ti passerebbe per
la testa? Questa è la domanda che ti faccio. Se tra sette giorni ti
toccasse la pena capitale, che cosa faresti? Se ci pensi e fai un passo
in più, capirai che noi, proprio ora, siamo tutti soggetti alla pena
capitale, solo che non sappiamo quando succederà. Potrebbe avvenire
anche prima di sette giorni. Sei consapevole di essere soggetto a questa
sentenza di morte?

Qualora aveste violato una legge e foste giudicati a morte, sareste
certamente angosciati al massimo. La meditazione sulla morte ci rammenta
che la morte sta per prenderci e che potrebbe avvenire molto presto.
Però non ci pensate, e sentite che state vivendo nel benessere. Se ci
pensate, ciò indurrà devozione nei riguardi della pratica del Dhamma.
Perciò il Buddha ci insegnò a praticare regolarmente la contemplazione
della morte. Chi non lo fa, vive nella paura. Non conosce se stesso. Se
invece ve ne rammentate e siete consapevoli di voi stessi, questo
v'indurrà a praticare seriamente il Dhamma e a liberarvi da questa
paura.

Se siete consapevoli di questa sentenza di morte, vorrete trovare una
soluzione. In genere alla gente non piace sentire questo genere di cose.
Ciò non significa forse che è molto lontana dal vero Dhamma? Il Buddha
ci raccomandò di rammentarci della morte, ma questo discorso turba le
persone. È il kamma degli esseri. Hanno una certa qual conoscenza
di questo dato di fatto, ma essa non è ancora chiara.

