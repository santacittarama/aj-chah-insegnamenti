\chapter{Solo questo}

\begin{openingQuote}
  \centering

  Tratto da un discorso offerto in Inghilterra nel 1977\\
  a un praticante di~Dhamma.
\end{openingQuote}

Sai dove finirà? Oppure continuerai a studiare in questo modo? C'è un
punto d'arrivo? Va bene, ma si tratta di studi esteriori, non di studi
interiori. Per gli studi interiori devi studiare questi occhi, questi
orecchi, questo naso, questa lingua, questo corpo e questa mente. Questo
è il vero studio. Lo studio dei libri è soltanto studio esteriore, è
davvero difficile che finisca. Quando l'occhio vede una forma, che cosa
succede? Quando l'orecchio, il naso e la lingua sperimentano suoni,
odori e sapori, che cosa succede? Quando il corpo e la mente entrano in
rapporto con sensazioni tattili e stati mentali, quali sono le reazioni?
Ci sono ancora brama, avversione e illusione? Ci perdiamo nelle forme,
nei suoni, negli odori, nei sapori, nelle sensazioni tattili e negli
stati mentali? Questo è lo studio interiore. Ha un completamento.

Se studiamo ma non pratichiamo, non otterremo alcun risultato. È come un
uomo che alleva una mucca. Al mattino la porta al pascolo e la sera la
riporta nel recinto, ma non beve mai il latte. Studiare va bene, ma non
lasciate che succeda questo. Dovreste allevare la mucca e anche bere il
latte. Dovete studiare e anche praticare per avere ottimi risultati. Te
lo spiego in un altro modo. È come un uomo che alleva le galline, ma non
raccoglie le uova. Tutto quello che ricava dalle galline è lo sterco!
Dico questo alla gente che a casa alleva le galline. Fai attenzione a
non diventare così! Ciò significa che studiamo le Scritture, ma che non
sappiamo come lasciar andare le contaminazioni, non sappiamo come
``spingere'' fuori dalla nostra mente la brama, l'avversione e
l'illusione. Studiare senza praticare, senza questo ``rinunciare'', non
reca risultati. Per questa ragione faccio il paragone con chi alleva
galline e non raccoglie le uova, ma solo lo sterco. È la stessa cosa.

È per questa ragione che il Buddha voleva che studiassimo le Scritture e
poi rinunciassimo alle cattive azioni del corpo, della parola e della
mente. Sviluppare la bontà nelle nostre azioni, nella nostra parola e
nei nostri pensieri. Per mezzo delle nostre azioni, della nostra parola
e dei nostri pensieri si fruirà del reale valore del genere umano. Se
parliamo soltanto, senza agire di conseguenza, non c'è completezza. Se
compiamo buone azioni, ma la mente non è ancora buona, nemmeno in questo
caso c'è completezza. Il Buddha insegnò a sviluppare la bontà nel corpo,
nella parola e nella mente, a sviluppare azioni eccellenti, parole
eccellenti e pensieri eccellenti. Questo è il tesoro della specie umana.
Lo studio e la pratica devono essere buone entrambe.

Il Nobile Ottuplice Sentiero del Buddha, il Sentiero della pratica, ha
otto fattori. Questi otto fattori non sono nient'altro che proprio
questo corpo: due occhi, due orecchi, due narici, una lingua e un corpo.
Questo è il Sentiero. E la mente è chi percorre il Sentiero. È per
questo motivo che sia lo studio sia la pratica esistono nel nostro
corpo, nella nostra parola e nella nostra mente.

Hai mai visto Scritture che insegnino qualcosa che non sia in relazione
con il corpo, con la parola e con la mente? Le Scritture insegnano solo
in relazione a queste cose, a nient'altro. Le contaminazioni nascono
proprio qui. Se le conosci, muoiono proprio qui. Dovresti comprendere
che sia la pratica sia lo studio esistono proprio qui. Se studiamo solo
questo possiamo conoscere tutto. È come con le nostre parole: dire una
sola parola vera è meglio di un'intera vita di parole sbagliate.
Capisci? Chi studia e non pratica è come un mestolo in una zuppiera. È
nella zuppiera tutti i giorni, ma non conosce il sapore della zuppa. Se
non pratichi, anche se studi fino al giorno della tua morte non
conoscerai mai il sapore della libertà!

