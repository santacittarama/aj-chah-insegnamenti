\chapter{Tranquillità e visione profonda}

Calmare la mente significa trovare il giusto equilibrio. Se cercate di
forzarla in modo esagerato la mente va troppo in là, ma se non provate
con sufficiente energia essa non giunge fino al punto d'equilibrio, lo
manca. Di norma la mente non è immobile, si muove in continuazione.
Dobbiamo potenziarla. Rafforzare la mente e rafforzare il corpo non è la
stessa cosa. Per rafforzare il corpo dobbiamo esercitarlo, spronarlo, ma
rafforzare la mente significa renderla serena, non pensare a questo e a
quello. Per la maggior parte di noi la mente non è mai stata serena, non
ha mai avuto l'energia del \emph{samādhi}. Perciò dobbiamo mettere dei
paletti. Sediamo in meditazione e restiamo con ``Colui che Conosce''.

Se forziamo il nostro respiro a essere troppo lungo o troppo corto, non
siamo in equilibrio, la mente non diverrà serena. È come quando
cominciamo a usare una macchina da cucire a pedale. Prima di cucire,
all'inizio ci esercitiamo solo a usare il pedale per imparare la giusta
coordinazione. È simile a seguire il respiro. Non ci preoccupiamo di
quanto sia lungo o corto, debole o forte, prendiamo solo atto di com'è.
Lasciamo semplicemente che il respiro sia, che segua il suo corso
naturale.

Quando la nostra respirazione è equilibrata, la assumiamo come oggetto
di meditazione. Quando inspiriamo, l'inizio del respiro è sulla punta
del naso, la metà è nel petto e la fine è nell'addome. Questo è il
percorso del respiro. Quando espiriamo, l'inizio è nell'addome, la metà
è nel petto e la fine sulla punta del naso. Limitatevi a prendere atto
di questo percorso del respiro dalla punta del naso, al petto e
all'addome, poi dall'addome, al petto e alla punta del naso. Prendiamo
atto di questi tre punti per rendere la mente stabile, per limitare
l'attività mentale in modo che possano facilmente sorgere la presenza
mentale e la consapevolezza di sé.

Quando la nostra attenzione si è stabilizzata su questi tre punti,
possiamo lasciarli andare e osservare il respiro che entra ed esce,
concentrandoci solo sulla punta del naso o sul labbro superiore, dove
l'aria passa quando entra ed esce. Non dobbiamo seguire il respiro,
instauriamo solo la consapevolezza di fronte a noi, sulla punta del
naso, e osserviamo il respiro in questo solo punto. Entra, esce, entra,
esce. Non c'è bisogno di pensare a nulla di speciale, per ora basta
concentrarsi su questo semplice compito con costante presenza mentale.
Non c'è nient'altro da fare, solo inspirare ed espirare. La mente
diventa presto serena e il respiro sottile. La mente e il corpo
diventano leggeri. Questa è la giusta condizione per lavorare con la
meditazione.

Quando sediamo in meditazione la mente si affina, ma dovremmo cercare di
essere consapevoli, di conoscere ogni stato mentale. L'attività mentale
è lì, insieme alla tranquillità. Questo è \emph{vitakka}. \emph{Vitakka}
è l'azione di condurre la mente al tema della contemplazione. Se non c'è
molta consapevolezza, non ci sarà molto \emph{vitakka}. Segue poi
\emph{vicāra}, la contemplazione di ciò che avviene nella mente attorno
a questo tema. Di tanto in tanto possono sorgere varie impressioni
mentali deboli, ma quel che conta è la nostra consapevolezza: qualsiasi
cosa possa avvenire nella mente, la conosciamo in continuazione. Quando
andiamo più in profondità, siamo costantemente consapevoli dello stato
in cui verte la nostra meditazione, sappiamo se la mente si è
stabilizzata con fermezza. Sono allora presenti sia la concentrazione
sia la consapevolezza.

Avere una mente serena non significa che in essa non succede niente. Le
impressioni mentali sorgono. Ad esempio, quando parliamo del primo
livello di assorbimento meditativo, diciamo che ha cinque fattori.
Assieme a \emph{vitakka} e a \emph{vicāra}, con il tema della
contemplazione sorgono \emph{pīti} e poi \emph{sukha}. Queste quattro
cose si trovano tutte quante insieme nella mente che s'è stabilizzata
nella tranquillità. Sono come un solo stato. Il quinto fattore è
\emph{ekaggatā}, o unificazione mentale. Potreste chiedervi come
l'unificazione mentale possa essere lì insieme a tutti gli altri
fattori. Ciò si verifica perché si sono tutti unificati sul fondamento
della tranquillità. Tutti insieme sono detti stato di \emph{samādhi}.
Non si tratta di stati mentali comuni, sono fattori di assorbimento
meditativo. Ci sono queste cinque caratteristiche, ma esse non turbano
la tranquillità di fondo. V\emph{itakka} c'è, ma non disturba la mente.
Sorgono \emph{vicāra}, rapimento e felicità, ma non disturbano la mente.
La mente è una cosa sola con questi fattori. Così è il primo livello di
assorbimento meditativo.

Non è necessario chiamarli primo \emph{jhāna},\footnote{\emph{jāhna}.
  Assorbimento mentale; uno stato di forte concentrazione focalizzata.}
secondo \emph{jhāna}, terzo \emph{jhāna} e così via, chiamiamoli solo
``mente serena''. Quando la mente diverrà progressivamente più calma,
farà a meno di \emph{vitakka} e di \emph{vicāra}, e resteranno solo il
rapimento e la felicità. Perché la mente scarta \emph{vitakka} e
\emph{vicāra}? Perché quando la mente diviene più affinata, le attività
di \emph{vitakka} e di \emph{vicāra} sono troppo grossolane per
rimanere. A questo stadio, quando la mente tralascia \emph{vitakka} e
\emph{vicāra}, possono sorgere sensazioni di grande rapimento estatico e
anche sgorgare lacrime. Però, quando il \emph{samādhi} s'intensifica,
anche il rapimento viene eliminato, e restano solo la felicità e
l'unificazione mentale, fino a che pure la felicità va via e la mente
raggiunge il più alto affinamento possibile. Ci sono solo equanimità e
unificazione, tutto il resto viene lasciato indietro. La mente permane
immobile.

Questo può succedere quando la mente è serena. Non dovete pensarci
molto, succede da sé quando i fattori causali sono maturi. Questa è
detta energia di una mente serena. In questo stato la mente non è
assonnata. I cinque impedimenti -- desiderio sensoriale, avversione,
irrequietezza, noia e dubbio -- sono tutti fuggiti. Se però l'energia
mentale non è ancora forte e la consapevolezza è debole, saltuariamente
sorgeranno e si intrometteranno delle impressioni mentali. La mente è
serena, ma è come se nella calma ci fosse una ``nuvolosità''. Tuttavia
non si tratta di una specie di sonnolenza, e alcune impressioni si
manifesteranno. Forse si sentirà un suono oppure si vedrà un cane o
qualcos'altro. Non si tratta Ḍ iun qualcosa di chiaro e nemmeno di un
sogno. È perché questi cinque fattori si sono squilibrati e indeboliti.

A questi livelli di tranquillità la mente tende a ingannarvi. Quando la
mente è in questo stato, attraverso uno qualsiasi dei sensi talora
sorgono delle ``immagini'', e il meditante può anche non essere in grado
di dire con esattezza cosa stia succedendo. «~È un sogno? No, non è un
sogno.~» Queste impressioni sorgono da un tipo di tranquillità mediocre.
Se però la mente è davvero serena e chiara, non dubitiamo delle varie
impressioni mentali o delle immagini che sorgono. «~Sono forse svenuto?
Mi sto addormentando? Sto dormendo? Mi sono perso?~» Domande di questo
tipo non sorgono, perché esse caratterizzano una mente che sta ancora
dubitando. «~Sto dormendo o sono sveglio?~» In questo caso la mente è
confusa. Questa è una mente che si perde nei suoi stati. È come la luna
che va a finire dietro una nuvola. È ancora possibile vederla, ma le
nuvole la coprono e la rendono indistinta. Non è come quando la luna
emerge dalle nuvole chiara, netta e luminosa.

Quando la mente è serena, è presente a se stessa e dimora salda nella
consapevolezza di sé, e non ci saranno dubbi a riguardo dei vari
fenomeni nei quali si imbatte. La mente sarà davvero al di là degli
impedimenti. Conosceremo con chiarezza tutto ciò che sorge nella mente
per quello che è. Non dubiteremo perché la mente è chiara e luminosa. La
mente che raggiunge il \emph{samādhi} è così.

Per alcune persone è difficile entrare in \emph{samādhi} perché non
hanno le necessarie inclinazioni. Il \emph{samādhi} c'è, ma non è forte
o stabile. Si può ovviamente raggiungere la pace utilizzando la
saggezza, per mezzo della contemplazione e, vedendo la verità delle
cose, i problemi si risolvono. Questo è usare la saggezza invece
dell'energia del \emph{samādhi}. Per raggiungere la calma nella pratica
non è necessario sedere in meditazione. Ad esempio, basta che chiediate
a voi stessi: «~Ehi, che cos'è?~», e risolvere il vostro problema lì per
lì! Una persona che ha saggezza fa così. Forse non può raggiungere alti
livelli di \emph{samādhi}, anche se vi deve comunque essere una certa
qual concentrazione, sufficiente per coltivare la saggezza. È come la
differenza esistente tra il coltivare il riso e coltivare il granturco.
Per il sostentamento si può dipendere più dal riso o più dal mais. La
nostra pratica può essere così, possiamo dipendere più dalla saggezza
per risolvere i problemi. Quando vediamo la Verità, sorge la pace.

Le due vie non sono uguali. Alcuni hanno la visione profonda e una forte
saggezza, ma non hanno molto \emph{samādhi}. Quando siedono in
meditazione non sono tanto sereni. Tendono a pensare molto, a
contemplare questo e quello, fino a che, alla fine, contemplano la
felicità e la sofferenza e comprendono la verità di esse. Alcuni sono
più inclini a questo che al \emph{samādhi}. L'Illuminazione del Dhamma
può avvenire quando si è in piedi, quando si cammina, si sta seduti o
distesi. Per mezzo della comprensione, per mezzo della rinuncia,
raggiungono la pace. Raggiungono la pace per mezzo della conoscenza
della Verità e perché, andando al di là del dubbio, l'hanno vista da
soli. Altri hanno solo poca saggezza ma il loro \emph{samādhi} è davvero
potente. Possono entrare velocemente in un \emph{samādhi} davvero
profondo, ma senza molta saggezza non riescono a catturare le loro
contaminazioni. Non le conoscono. Non possono risolvere i loro problemi.
Indipendentemente dal metodo che utilizziamo, dobbiamo mettere da parte
il pensiero errato, deve restare solo la Retta Visione. Dobbiamo
sbarazzarci della confusione, deve restare solo la pace. Quale che sia
la via, raggiungiamo lo stesso posto. Queste sono le due facce della
pratica, ma tali due cose, calma e visione profonda, vanno insieme. Non
possiamo fare a meno di nessuna delle due. Devono stare insieme.

Ciò che esamina i vari fattori che sorgono durante la meditazione è
\emph{sati}, la consapevolezza. \emph{Sati} è una condizione che,
attraverso la pratica, può aiutare altri fattori a sorgere. \emph{Sati}
è vita. Tutte le volte che siamo privi di \emph{sati}, quando siamo
distratti, è come se fossimo morti. Se non abbiamo \emph{sati}, le
nostre parole e le nostre azioni non hanno senso. \emph{Sati} è
semplicemente rammemorazione. È la causa del sorgere della
consapevolezza di sé e della saggezza. Quali che siano le virtù che
abbiamo coltivato, esse sono imperfette se mancano di \emph{sati}. È ciò
che vigila su di noi mentre stiamo in piedi, quando camminiamo, mentre
siamo seduti e distesi. \emph{Sati} dovrebbe essere sempre presente,
anche quando non siamo più in \emph{samādhi}.

Qualsiasi cosa facciamo, prestiamo attenzione. Sorgerà un senso di
vergogna.\footnote{Si tratta di un senso di vergogna salutare, basato
  sulla conoscenza di causa ed effetto, piuttosto che di un emotivo
  senso di colpa.} Ci vergogniamo delle cose fatte in modo non corretto.
Quando la vergogna cresce, cresce anche il nostro raccoglimento. Quando
cresce il raccoglimento, scompare la distrazione. Questi fattori saranno
presenti perfino quando non saremo seduti in meditazione. E ciò sorge
perché coltiviamo \emph{sati}. Sviluppate \emph{sati}! Questa è la
qualità che vigila sul lavoro che svolgiamo nel momento presente. Ha un
valore reale. Dovremmo conoscere noi stessi sempre. Se conosciamo noi
stessi in questo modo, il giusto si distinguerà da sé dallo sbagliato,
il Sentiero diverrà chiaro, e si dissolveranno tutte le cause per la
vergogna. Sorgerà la saggezza.

Possiamo riassumere tutta la pratica in termini di moralità,
concentrazione e saggezza. Essere raccolti, controllati, questa è
moralità. Fissare con stabilità la mente entro i confini di questo
controllo è concentrazione. Saggezza è conoscenza completa e complessiva
nell'attività che stiamo svolgendo. In breve, la pratica è solo
moralità, concentrazione e saggezza o, in altre parole, il Sentiero. Non
c'è altra via.

