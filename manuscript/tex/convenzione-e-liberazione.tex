\chapter{Convenzione e Liberazione}

\begin{openingQuote}
  \centering

  È si tratta di un discorso informale pronunciato nel dialetto del nord-est della
  Thailandia, da una registrazione non si sa da chi realizzata.
\end{openingQuote}

Le cose di questo mondo sono mere convenzioni, prodotte da noi. Dopo
averle stabilite, ci perdiamo in esse e rifiutiamo di lasciarle andare,
facendo sorgere attaccamenti a modi di vedere e opinioni personali.
Questo attaccamento non ha mai termine, è il \emph{saṃsāra}, che scorre
senza fine. Non ha compimento. Allora, se conosciamo la realtà
convenzionale, conosceremo la Liberazione. Se conosciamo con chiarezza
la Liberazione, poi conosceremo la convenzione. Questo è conoscere il
Dhamma. Qui vi è compimento.

Prendete ad esempio le persone. In realtà, non hanno alcun nome, nel
mondo nasciamo nudi. I nostri nomi sorgono mediante convenzioni. L'ho
contemplato e ho visto che se non conoscete la verità di questa
convenzione, ciò può essere davvero nocivo. È semplicemente una cosa che
utilizziamo per convenienza. Senza di essa non potremmo comunicare, non
ci sarebbe niente da dire, nessun linguaggio.

Ho visto praticare la meditazione seduta in Occidente. Dopo essere stati
seduti insieme uomini e donne, gli occidentali quando si alzano vanno
talvolta a toccarsi l'un l'altro la testa!\footnote{In Thailandia
  toccare la testa a una persona è di solito considerato un insulto;
  come si vedrà appena più avanti, è però ritenuto di buon auspicio che
  a toccarla sia un monaco molto stimato.} Quando l'ho visto, ho
pensato: «~Eh, se ci attacchiamo alle convenzioni, proprio lì nascono le
contaminazioni.~» Se possiamo lasciar andare le convenzioni, rinunciare
alle nostre opinioni, siamo in pace.

Come i generali e i colonnelli, uomini di alto rango e con una posizione
sociale, che vengono a visitarmi. Quando arrivano, dicono: «~Per favore,
mi tocchi la testa.~» Se lo chiedono loro, non c'è nulla di sbagliato,
sono contenti di farsi toccare la testa. Ma se sfioraste appena la loro
testa in mezzo alla strada, sarebbe ben diverso! È a causa
dell'attaccamento. Per questo penso che lasciar andare sia davvero la
via per la pace. Toccare la testa è contro le nostre usanze, ma in
realtà è niente. Quando loro sono d'accordo a farsela toccare non c'è
nulla di sbagliato, proprio come quando si tocca un cavolo o una patata.

Accettare, rinunciare, lasciar andare: questa è la via della leggerezza.
Dovunque ci aggrappiamo, proprio lì vi è divenire e nascita. Proprio lì
sta il pericolo. Il Buddha insegnò a proposito delle convenzioni,
insegnò ad annullarle nel giusto modo, e raggiungere così la
Liberazione. Questa è la libertà: non aggrapparci alle convenzioni.
Tutte le cose in questo mondo hanno una realtà convenzionale. Non
dovremmo farci ingannare dalle convenzioni che noi stessi stabiliamo,
perché perdersi in esse conduce davvero alla sofferenza. Questo punto a
riguardo delle regole e delle convenzioni è della massima importanza.
Chi riesce ad andare oltre le regole e le convenzioni, va oltre la
sofferenza.

Ovviamente, le convenzioni sono una caratteristica del nostro mondo.
Prendete ad esempio il signor Boonmah; prima era uno dei tanti, ma
adesso è stato nominato Commissario del Distretto. È solo una
convenzione, ma dovremmo rispettarla. Fa parte del mondo della gente. Se
pensate: «~Oh, prima eravamo amici, era normale lavorare insieme dal
sarto~», e poi andate e gli date un colpetto sulla testa in pubblico, si
arrabbierà. Non lo riterrebbe giusto e si risentirebbe. Dovremmo perciò
seguire le convenzioni per evitare che sorga il risentimento.
Comprendere le convenzioni è utile. Vivere nel mondo è tutto qui.
Conoscere il tempo giusto e il posto giusto, conoscere le persone.

Perché è sbagliato andare contro le convenzioni? È sbagliato a causa
della gente. Dovreste essere intelligenti, conoscere sia la convenzione
sia la Liberazione. Conoscere il tempo giusto per ognuna di esse. Quando
sappiamo come usare regole e convenzioni con agio, allora siamo abili.
Se tentiamo di comportarci in accordo col più alto livello di realtà
nella situazione sbagliata, questo non va bene. Dove si sbaglia? Si
sbaglia con le contaminazioni della gente, ecco dove. Tutti hanno
contaminazioni. In una situazione ci comportiamo in un modo, in un'altra
situazione ci dobbiamo comportare in un altro. Dovremmo conoscere i pro
e i contro perché viviamo nelle convenzioni. I problemi esistono perché
la gente si attacca a esse. Se supponiamo che qualcosa esista, allora
esiste. È lì perché supponiamo che sia lì. Se però osserviamo da vicino,
in senso assoluto queste cose non esistono realmente.

Come spesso dico, prima eravamo laici e ora siamo monaci. Abbiamo
vissuto nella convenzione ``laico'' e ora viviamo nella convenzione
``monaco''. Siamo monaci per convenzione, non monaci a causa della
Liberazione. Inizialmente stabiliamo delle convenzioni come questa, ma
se riceviamo solo l'ordinazione monastica questo non significa che
abbiamo superato le contaminazioni. Se prendiamo una manciata di sabbia
e ci accordiamo di chiamarla sale, questo la rende sale? È sale, ma solo
per il nome, non in realtà. Non potreste usarla per cucinare. L'unico
uso è nell'ambito di quell'accordo, perché in realtà lì non c'è alcun
sale, solo sabbia. È sale unicamente perché supponiamo che tale sia.

Questa parola, ``Liberazione'', è essa stessa solo una convenzione, ma
si riferisce a ciò che è al di là delle convenzioni. Raggiunta la
Libertà, raggiunta la Liberazione, dobbiamo ancora usare la convenzione
di riferirci a essa come Liberazione. Se non avessimo convenzioni non
potremmo comunicare, questa è la loro utilità. Ad esempio, le persone
hanno nomi differenti, ma sono tutte persone allo stesso modo. Se non ci
fossero i nomi per differenziarle una dall'altra e volessimo chiamare
qualcuno che sta in piedi in mezzo alla folla, sarebbe inutile dire:
«~Ehi, Persona! Persona!~» Non potreste sapere chi vi risponderebbe,
perché sono tutte ``persone''. Però, se chiamate: «~Ehi, Mario!~» sarà
Mario a rispondervi, e non gli altri. I nomi servono a questo. Per mezzo
di essi possiamo comunicare, sono la base delle relazioni sociali.

È per questo motivo che dovreste conoscere sia la convenzione sia la
Liberazione. Le convenzioni hanno un'utilità, ma in realtà lì non c'è
nulla. Anche le persone non esistono. Sono soltanto degli insiemi di
elementi, nate da condizioni causali, cresciute in dipendenza di
condizioni, che esistono per un po' e poi scompaiono in modo naturale.
Nessuno può opporsi o controllare questo processo. Senza convenzioni,
però, non avremmo nulla da dire, non avremmo nomi, non vi sarebbe
pratica, né lavoro. Regole e convenzioni vengono stabilite per darci un
linguaggio e per facilitare le cose. Questo è tutto.

Prendiamo come esempio il denaro. Anticamente non c'erano monete o
banconote, esse non avevano valore. La gente usava barattare i beni, ma
le cose erano difficili da conservare e così inventarono il denaro, le
monete e le banconote. Forse in futuro un nuovo sovrano decreterà che
non dobbiamo usare valuta cartacea, ma che si dovrebbe usare cera,
facendola sciogliere e pressandola in blocchi. Diremo che questo è il
denaro e lo useremo in tutto il paese. Cera a parte, potrebbero perfino
decidere che lo sterco delle galline sia la valuta locale: tutte le
altre cose non potrebbero essere denaro, solo lo sterco delle galline!
Allora le persone combatterebbero e si ucciderebbero a vicenda per lo
sterco delle galline!

Così stanno le cose. Molti sono gli esempi per illustrare le
convenzioni. Ciò che utilizziamo come denaro è solo una convenzione da
noi istituita, e l'utilità di esso vale all'interno di quella
convenzione. Avendo deciso che debba essere denaro, diventa denaro. In
realtà, però, che cos'è il denaro? Nessuno può dirlo. Quando vi è un
comune accordo su qualcosa, ecco che arriva una convenzione per
soddisfare il bisogno. Così è il mondo.

Questa è la convenzione, ma far comprendere la Liberazione alla gente è
davvero difficile. Il nostro denaro, la nostra casa, la nostra famiglia,
i nostri figli e parenti sono solo convenzioni da noi inventate, ma in
realtà, viste alla luce del Dhamma, queste cose non ci appartengono. Nel
sentirlo forse non ci sentiamo molto bene, ma questa è la realtà. Queste
cose hanno valore solo grazie a convenzioni affermate. Se stabiliamo che
ciò non ha valore, allora non ha valore. Se stabiliamo che ha valore,
allora ha valore. Così stanno le cose. Nel mondo produciamo una
convenzione per soddisfare un bisogno.

Perfino questo corpo non è realmente nostro, noi supponiamo solo che sia
così. È davvero solo una nostra supposizione. Se cercate di trovare un
reale, un sostanziale sé nel corpo, non potete riuscirci. Semplicemente,
ci sono solo elementi che sono nati, continuano a vivere per un po' e
poi muoiono. Tutto è in questo modo. Non c'è nessuna reale, vera
sostanza in esso, ma è giusto che se ne faccia uso. È come una tazza. A
un certo punto quella tazza deve rompersi, ma fino a quando è ancora qui
potete usarla e prendervene cura per bene. È per voi uno strumento. Se
si rompe sono problemi e così, sebbene debba rompersi prima o poi,
dovreste cercare di fare tutto il possibile per preservarla.

Noi abbiamo i quattro beni di prima necessità\footnote{I quattro beni di
  prima necessità di supporto ai monaci sono l'abito, il cibo
  elemosinato, l'alloggio e le medicine.} che il Buddha ci insegnò a
contemplare continuamente. Sono dei beni indispensabili, dai quali un
monaco dipende per continuare la sua pratica. Per tutto il tempo della
vostra vita dovete dipendere da questi beni, ma cercate di comprenderli.
Non attaccatevi a essi, facendo sorgere la brama nella vostra mente.

Convenzione e Liberazione sono continuamente in rapporto, in questo
modo. Anche se noi usiamo una convenzione, non fate affidamento su di
essa come se fosse la verità. Se vi attaccate a essa, sorgerà la
sofferenza. Un buon esempio è la questione riguardante ciò che è giusto
e ciò che è sbagliato. Alcune persone vedono quel che è sbagliato come
se fosse giusto e quel che è giusto come se fosse sbagliato, ma in fin
dei conti chi sa davvero cos'è giusto e cos'è sbagliato? Non lo
sappiamo. Persone diverse fissano convenzioni differenti a proposito di
cos'è giusto e di cos'è sbagliato, ma il Buddha assunse come direttiva
la sofferenza. Se volete discutere in proposito, non vi sarà mai fine.
Uno dice ``giusto'', un altro dice ``sbagliato''. Uno dice
``sbagliato'', un altro dice ``giusto''. In verità, noi non conosciamo
affatto giusto e sbagliato. Su un livello utile e pratico, però,
possiamo dire che giusto è non nuocere a se stessi e agli altri. Questa
via realizza un fine per noi costruttivo.

Dopo tutto, regole, convenzioni e Liberazione sono semplicemente
\emph{dhamma}.\footnote{\emph{Dhamma}: Con la lettera minuscola,
  significa il fenomeno tanto fisico quanto mentale, oppure solo lo
  stato mentale, l'oggetto mentale, la caratteristica o la qualità.}
La Liberazione sta più in alto, ma tutte queste cose procedono tenendosi
per mano. Non vi è alcuna possibilità di garantire che qualcosa sia
senza dubbio in questo o in quel modo, perciò il Buddha disse di
lasciarla semplicemente così com'è. Lasciatela essere incerta. Per
quanto vi piaccia o dispiaccia, dovreste comprenderla come incerta.

Indipendentemente da tempo e spazio, l'intera pratica del Dhamma giunge
a compimento dove non c'è nulla. È il posto della resa, del vuoto, ove
deporre il fardello. Tutto termina qui. Non è come succede quando uno
dice: «~Perché la bandiera sventola? Io dico che è a causa del vento.~»
Un altro dice che è a causa della bandiera. Il primo replica che è per
il vento. Queste cose non finiscono mai! È come quel vecchio
indovinello: «~Che cosa viene prima, l'uovo o la gallina?~» Non c'è modo
di giungere a una conclusione, così è la natura. Diciamo che tutte
queste cose sono solo convenzioni, le stabiliamo noi stessi. Se le
conoscete con saggezza, allora conoscerete impermanenza, sofferenza e
non-sé. Questa è la prospettiva che conduce all'Illuminazione.

Addestrare la gente e insegnare a vari livelli di comprensione è davvero
difficile. Alcuni hanno idee precise; tu dici loro qualcosa, ma non ti
credono. Tu dici la verità, e loro dicono che non è vero. «~Io ho
ragione, tu hai torto.~» Queste cose non finiscono mai. Se non lasciate
andare, lì vi sarà sofferenza. Vi ho già raccontato dei quattro uomini
che vanno nella foresta. Sentono un ``coccodé''. Uno chiede: «~È un
gallo o una gallina?~» Gli altri tre insieme dicono: «~È una gallina.~»
L'altro però non è d'accordo, e insiste che sia un gallo: «~Una gallina
come potrebbe fare quel verso?~» Gli altri rispondono: «~Beh, ha un
becco, o no?~» Discutono e discutono fino allo sfinimento, arrabbiandosi
davvero, ma alla fine hanno tutti torto. Per quanto dicano che si tratta
di una gallina oppure di un gallo, si tratta solo di nomi. Noi fissiamo
tali convenzioni, affermando che un gallo è in questo modo e che una
gallina in quest'altro, che un gallo fa questo verso, una gallina
quest'altro, ed è così che restiamo bloccati nel mondo! Ricordatevene!
In effetti, se dite che non c'è né un gallo né una gallina, allora si
arriva alla fine della questione. Nell'ambito della realtà
convenzionale, da una parte c'è quello che è giusto, dall'altra quello
che è sbagliato, ma noi non saremo mai completamente d'accordo.
Discutere fino allo sfinimento non ha alcuna utilità.

Il Buddha insegnò a non attaccarsi. Come si pratica il non attaccamento?
Lo pratichiamo semplicemente rinunciando all'attaccamento, ma questo è
davvero difficile da capire. Ci vuole un'acuta saggezza per investigare
e comprenderlo a fondo, per raggiungere davvero il non attaccamento. Se
ci pensate, che le persone siano felici o tristi, contente o scontente,
non dipende dal fatto che abbiano poco o tanto, dipende dalla saggezza.
Ogni afflizione può essere trascesa solo mediante la saggezza, solo
vedendo la verità delle cose.

Per questo il Buddha ci esortò a investigare, a contemplare.
``Contemplazione'' significa solo cercare di risolvere questi problemi
correttamente. Questa è la nostra pratica. Come la nascita, la
vecchiaia, la malattia e la morte: si tratta degli eventi più naturali e
comuni. Il Buddha insegnò a contemplare la nascita, la vecchiaia, la
malattia e la morte, ma alcuni non lo capiscono. Dicono: «~Che cosa c'è
da contemplare?~» Sono nati ma non conoscono la nascita, moriranno ma
non conoscono la morte. Chi investiga queste cose continuamente, vedrà.
Avendo visto, gradualmente risolverà i suoi problemi. Anche se ha ancora
attaccamenti, ha la saggezza e vede che nascita, vecchiaia, malattia e
morte sono aspetti della natura, e sarà in grado di alleviare la propria
sofferenza.

Non c'è poi molto a fondamento del buddhismo, vi è solo la nascita e la
morte della sofferenza, ed è questo che il Buddha chiamò verità. La
nascita è sofferenza, la vecchiaia è sofferenza, la malattia è
sofferenza e la morte è sofferenza. La gente non vede questa sofferenza
come verità. Se conosciamo la verità, allora conosciamo la sofferenza.
L'orgoglio delle opinioni personali, le discussioni: sono cose che non
hanno fine. Per far riposare la nostra mente, per trovare la pace,
dovremmo contemplare il nostro passato, il presente e le cose che ci
riserva il futuro, come nascita, vecchiaia, malattia e morte. Possiamo
evitare di essere afflitti da queste cose? Anche se possiamo ancora
avere qualche preoccupazione, se investighiamo fino a conoscere in
accordo con la Verità, tutta la sofferenza cesserà, perché non ci
attaccheremo più alle cose.

