\chapter{Senza dimora}

\begin{openingQuote}
  \centering

  Un discorso per monaci, novizi e laici del Wat Pah Nanachat in visita al Wat
  Nong Pah Pong tenuto durante la Stagione delle Piogge del~1980.
\end{openingQuote}

Alcuni insegnamenti li ascoltiamo, ma non li comprendiamo davvero.
Pensiamo che non dovrebbero essere come sono, e così non li seguiamo,
anche se in verità tutti gli insegnamenti hanno una loro ragion
d'essere. Può forse sembrare che quelle cose non dovrebbero essere così,
ma non è vero. All'inizio, io non credetti nemmeno nella meditazione
seduta. Non riuscivo a capire di quale utilità potesse essere stare
seduti a occhi chiusi. E la meditazione camminata? Camminare da questo a
quell'albero, voltarsi per poi tornare indietro di nuovo. «~Perché
farlo?~» «~Che utilità ha tutto questo camminare?~» È così che pensavo,
ma in realtà la meditazione camminata e quella seduta sono di grande
utilità. Alcuni hanno delle tendenze che li inducono a preferire la
meditazione camminata, altri quella seduta, ma non si può fare a meno di
nessuna delle due. Le Scritture fanno riferimento alle quattro posture:
in piedi, camminare, seduti e distesi. Viviamo in queste quattro
posture. Possiamo preferire una all'altra, ma dobbiamo usarle tutte.

Le Scritture dicono di rendere uguali queste quattro posture, di
praticare allo stesso modo in tutte le posture. All'inizio non riuscivo
a immaginare cosa significava renderle uguali. Forse che si dorme per
due ore, poi si sta in piedi per due ore, poi si cammina per due ore \ldots{}
è forse così? Provai, ma non riuscivo, era impossibile! Non è questo che
significa rendere uguali le quattro posture. ``Rendere uguali le
posture'' si riferisce alla mente, alla nostra presenza mentale,
significa far sorgere la saggezza nella mente, illuminare la mente.
Questa nostra saggezza deve essere presente in tutte le posture,
dobbiamo conoscere, o comprendere, costantemente. Stando in piedi,
camminando, seduti e distesi, sappiamo che tutti gli stati mentali sono
impermanenti, insoddisfacenti e non-sé. In questo le posture possono
essere rese uguali, così è possibile. Che nella nostra mente sia
presente piacere o dispiacere, non dimentichiamo la nostra pratica,
siamo consapevoli.

Se la nostra pratica è costante, quando veniamo lodati si tratta solo di
una lode, se veniamo criticati è solo una critica. Non siamo contenti o
scontenti, non ci sentiamo né su né giù, restiamo proprio qui. Perché?
Perché in tutto questo vediamo il pericolo, ne comprendiamo gli effetti.
Siamo costantemente consapevoli del pericolo insito sia nella lode sia
nella critica. Normalmente, se siamo di buon umore anche la mente è a
proprio agio e le vediamo entrambe come se fossero uguali; se siamo di
cattivo umore va male anche per la mente, le consideriamo diverse, la
critica non ci piace. Se le cose stanno così, questa è una pratica non
uniforme.

Se la nostra costanza arriva anche solo fino al punto di conoscere i
nostri stati mentali e di sapere che ci stiamo aggrappando a essi, già
va meglio. Abbiamo presenza mentale, sappiamo cosa sta succedendo, ma
non riusciamo ancora a lasciar andare. Vediamo che ci stiamo aggrappando
al bene e al male, e lo sappiamo. Ci aggrappiamo al bene e sappiamo che
non è retta pratica, ma non riusciamo ancora a lasciar andare. Questo è
già dal cinquanta al settanta per cento di retta pratica. Non c'è ancora
Liberazione, ma sappiamo che potremmo lasciar andare, che sarebbe la via
per la pace. Continuiamo a vedere le conseguenze ugualmente dannose di
tutto ciò che ci piace e ci dispiace, della lode e della critica,
sempre. Quali che siano le condizioni, quest'attitudine della mente è
costante.

La gente del mondo, però, quando viene criticata o biasimata s'arrabbia
davvero. Se le persone sono lodate, questo le rallegra, dicono che è
bene e ne sono proprio felici. Se conosciamo la verità dei nostri vari
umori, se conosciamo le conseguenze dell'attaccamento a lode e critica,
come pure il pericolo dell'attaccamento a qualsiasi cosa, diventiamo
sensibili ai nostri stati mentali. Sapremo che attaccarsi a essi causa
davvero sofferenza. Questa sofferenza la vediamo, e vediamo che il
nostro attaccamento è la causa di quella sofferenza. Iniziamo a notare
le conseguenze dell'aggrapparsi, dell'attaccamento al bene e al male,
perché li abbiamo afferrati e abbiamo già visto il risultato: nessuna
reale felicità. Così, ora cerchiamo la strada per il lasciar andare.

Dov'è questa ``strada per il lasciar andare''? «~Non attaccarti a
niente~», diciamo noi buddhisti. «~Non attaccarti a niente!~» Lo
sentiamo dire in continuazione. Significa tenere, ma non attaccarsi.
Come questa torcia elettrica. Pensiamo: «~Che cos'è?~» Così la prendiamo
in mano -- «~Ah, è una torcia elettrica~» -- e la posiamo di nuovo.
Teniamo le cose in questo modo.

Se non tenessimo nulla, che cosa potremmo fare? Non potremmo fare la
meditazione camminata o qualsiasi altra cosa, è per questo che prima le
cose dobbiamo tenerle. È volizione, certo, questo è vero, ma in seguito
ciò condurrà alla \emph{pāramī},\footnote{\emph{Pāramī}: ``Perfezione''.
  Per l'elenco delle dieci relative qualità si veda il \emph{Glossario}, p. \pageref{glossary-parami}.}
alla virtù o perfezione. Come, ad esempio, il voler venire in questo
luogo. Il venerabile Jagaro\footnote{Allora il secondo abate,
  australiano, del Wat Pah Nanachat, che condusse il suo gruppo di
  monaci e laici a visitare Ajahn Chah.} è venuto al Wat Pah Pong. Prima
è stato necessario che volesse venire. Se non avesse sentito che voleva
venire, non sarebbe venuto. È lo stesso per tutti, sono venuti qui
perché lo hanno voluto.

Quando la volizione sorge, però, non attaccatevi a essa! Così venite, e
poi tornate a casa. Di cosa si tratta? Prima prendiamo, guardiamo e
vediamo -- «~Ah, è una torcia elettrica~» -- e poi la posiamo. Questo
significa ``tenere ma non attaccarsi'', lasciar andare. Conosciamo e poi
lasciamo andare. Per metterla semplicemente, diciamo così: «~Conosci,
poi lascia andare.~» Continua a conoscere e a lasciar andare. «~Dicono
che questo è bene, dicono che questo è male.~» Conosci, e poi lascia
andare. Bene e male, li conosciamo, ma li lasciamo andare. Non ci
attacchiamo come sciocchi alle cose, ma le ``teniamo'' con saggezza. Si
può praticare sempre in questa ``postura''. È così che dovete essere
costanti. Fate in modo che la mente conosca in questa maniera, lasciate
sorgere la saggezza. Quando la mente ha saggezza, che altro c'è da
cercare?

Dovremmo riflettere su cosa ci stiamo a fare qui. Per quale motivo
stiamo vivendo in questo luogo, a cosa stiamo lavorando? Nel mondo la
gente lavora per questa o quella ricompensa, ma noi monaci insegniamo
qualcosa di più profondo. Qualsiasi cosa facciamo, non chiediamo nulla
in cambio. Non lavoriamo per una ricompensa. La gente del mondo lavora
perché vuole questo o quello, perché vuole un qualche guadagno o altro
ancora, ma il Buddha insegnò a lavorare solo per lavorare. Non chiediamo
nient'altro. Se fate una cosa solo per ottenere qualcos'altro, ciò vi
causerà sofferenza. Sperimentatelo da voi stessi. Se volete rendere la
vostra mente serena e perciò sedete in meditazione cercando di renderla
serena, soffrirete! Provateci! La nostra via è più sottile. Noi
facciamo, e poi lasciamo andare. Fare, e poi lasciar andare.

Guardate il brāhmaṇo\footnote{\emph{Brāhmaṇo}: Membro della casta dei brāhmaṇi,
  ``sacerdote''; la casta dei brāhmaṇi in India ha per molto tempo
  ritenuto che, per nascita, i suoi componenti fossero degni del più
  alto rispetto; si veda \emph{brāhmaṇa}, nel \emph{Glossario}, p. \pageref{glossary-brahmana}.} che
compie sacrifici. Ha qualche desiderio nella mente, è per questo che
compie sacrifici. Queste sue azioni non l'aiuteranno a trascendere la
sofferenza perché sta agendo dietro la spinta d'un desiderio.
Inizialmente pratichiamo con qualche desiderio nella mente; continuiamo
a praticare, ma il nostro desiderio non lo realizziamo. Così,
pratichiamo fino a quando non raggiungiamo il punto in cui pratichiamo
per non ottenere nulla in cambio, pratichiamo per lasciar andare. Si
tratta di una cosa che dobbiamo vedere da noi stessi, è davvero
profonda. Forse pratichiamo perché vogliamo entrare nel
Nibbāna.\footnote{Nibbāna (sanscrito \emph{Nirvāṇa}): La
  Liberazione finale da ogni sofferenza, lo scopo della pratica
  buddhista.} E proprio così non otterremo il Nibbāna. È naturale
che si voglia la pace, anche se in verità non è corretto. Dobbiamo
praticare senza volere proprio nulla. Se non vogliamo proprio nulla,
cosa otterremo? Non otterremo nulla! Qualsiasi cosa si ottenga è causa
di sofferenza, così pratichiamo per non ottenere nulla.

Proprio questo significa ``rendere vuota la mente''. È vuota, ma è
ancora attiva. Questa vacuità è una cosa che di solito la gente non
capisce; solo chi la raggiunge capisce il suo reale valore. Non è la
vacuità del non avere nulla, è una vacuità interna alle cose che sono
qui. Come questa torcia elettrica: dovremmo vedere questa torcia
elettrica come vuota. A causa della torcia elettrica c'è vacuità. Non è
una vacuità in cui non c'è nulla, non è così. La gente che capisce in
questo modo, sbaglia completamente. Dovete comprendere la vacuità
interna alle cose che sono qui.

Chi sta praticando perché ha ancora una qualche idea di ottenere
qualcosa è come il brāhmaṇo che compie un sacrificio per esaudire un
desiderio. Come le persone che vengono a trovarmi per essere spruzzate
di ``acqua santa''. Quando chiedo loro: «~Perché volete quest'acqua
santa?~» «~Vogliamo vivere felici e nell'agio, e non ammalarci~»,
rispondono. Ecco! In questo modo non trascenderanno mai la sofferenza.

La via del mondo consiste nel fare le cose per una ragione, per una
ricompensa, ma nel buddhismo facciamo le cose senza l'idea di ottenere
qualcosa in cambio. Il mondo deve comprendere le cose in termini di
causa ed effetto, ma il Buddha ci insegna ad andare oltre e al di là di
causa ed effetto. La sua saggezza consisteva nell'andare oltre la causa,
al di là dell'effetto. Andare oltre la nascita e al di là della morte.
Andare oltre la felicità e al di là della sofferenza. Pensate a questo:
non v'è posto in cui stare. Le persone vivono in una ``casa''. Per
lasciare la casa e andare dove non c'è casa non sappiamo come fare,
perché siamo sempre vissuti con il divenire, con l'attaccamento. Se non
possiamo attaccarci non sappiamo che cosa fare.

È per questo motivo che la maggior parte della gente non vuole andare
nel Nibbāna. Perché lì non c'è niente, proprio niente. Guardate
qui, il tetto e il pavimento. L'estremo in alto è il tetto, un
``dimorare''. L'estremo in basso è il pavimento, un altro ``dimorare''.
Nello spazio vuoto tra il pavimento e il tetto, però, non vi è luogo sul
quale poggiare. Si potrebbe stare sul tetto, oppure sul pavimento, ma
non in quello spazio vuoto. Ove non c'è dimorare, là si trova la
vacuità, e il Nibbāna è questa vacuità.

Quando la gente sente queste cose, si tira un po' indietro, non ci vuole
andare nel Nibbāna. Le persone hanno paura che non vedranno più i
loro figli, i loro parenti. Questa è la ragione per cui quando
benediciamo i laici, diciamo: «~Che tu possa avere una lunga vita,
bellezza, felicità e forza.~» Ciò li rende davvero felici. Dicono tutti:
«~Sādhu!~»\footnote{Esclamazione utilizzata in varie occasioni
  cerimoniali del Theravāda, per significare ``bene'', ``meraviglioso'',
  ossia ringraziamento e, nello stesso tempo, plaudire e approvare
  quanto si è ricevuto in segno di merito o di benedizione.} A loro
queste cose piacciono. Se iniziate a parlare di vacuità, non la
vogliono, sono attaccati al dimorare. Avete però mai visto una persona
molto anziana con una bella carnagione? Avete mai visto un anziano con
molta forza, o molto felice? No, ma se diciamo «~Lunga vita, bellezza,
felicità e forza~» sono tutti molto compiaciuti. Tutti dicono:
«~Sādhu!~» È come il brāhmaṇo che celebra sacrifici per esaudire un
desiderio.

Nella nostra pratica non ``celebriamo sacrifici'', non pratichiamo per
ottenere qualcosa in cambio. Non vogliamo nulla. Se vogliamo qualcosa,
questo allora significa che lì c'è ancora qualcosa. Rendete solo la
mente serena, e avete fatto tutto. Se parlo in questo modo, però, forse
non vi sentite del tutto a vostro agio perché volete ``rinascere''.
Tutti voi praticanti laici dovreste stare vicini ai monaci e vedere la
loro pratica. Essere vicini ai monaci significa essere vicini al Buddha,
essere vicini al suo Dhamma. Il Buddha disse: «~ānanda, pratica molto,
sviluppa la tua pratica! Chiunque veda il Dhamma vede me, e chiunque
veda me vede il Dhamma.~»

Dov'è il Buddha? Possiamo pensare che il Buddha sia esistito e poi se ne
sia andato, ma il Buddha è il Dhamma, la Verità. Ad alcuni piace dire:
«~Oh, se fossi nato ai tempi del Buddha avrei raggiunto il
Nibbāna.~» Ecco, sono gli sciocchi a parlare in questo
modo. Il Buddha è ancora qui. Il Buddha è Verità. Indipendentemente da
chi nasca o muoia, la Verità è ancora qui. La Verità non se ne va mai
dal mondo, è sempre qui. Che un Buddha sia nato o no, che qualcuno lo
abbia conosciuto o no, la Verità è ancora qui. Noi dovremmo avvicinarci
al Buddha, dovremmo entrare in noi stessi e trovare il Dhamma. Quando
raggiungiamo il Dhamma, raggiungeremo il Buddha. Vedendo il Dhamma
vedremo il Buddha, e tutti i dubbi si dissolveranno.

Un paragone: è come Choo l'insegnante. Prima non era un insegnante, era
solo il signor Choo. Dopo aver studiato e frequentato le scuole
necessarie, è diventato un insegnante, noto come Choo l'insegnante.
Quando Choo l'insegnante muore, gli studi necessari per insegnare
rimarranno ancora, e chiunque li frequenterà sarà un insegnante. Quel
corso di studi per diventare un insegnante non scomparirà, proprio come
la Verità, quel conoscere che consentì al Buddha di diventare il Buddha.
Per questo il Buddha è ancora qui. Chiunque pratichi e veda il Dhamma,
vede il Buddha. Oggigiorno la gente fraintende tutto, non sa dove sia il
Buddha. Dice: «~Se fossi nato ai tempi del Buddha, sarei stato un suo
discepolo e sarei diventato un Illuminato.~» È solo follia.

Non arriviate a pensare che alla fine del Ritiro delle Piogge\footnote{L'annuale
  periodo di tempo di tre mesi, che in India corrisponde a quello dei
  primi tre mesi monsonici, durante i quali i monaci hanno la regola
  dell'obbligo di residenza in monastero, un periodo che
  tradizionalmente è dedicato a una formazione più intensiva.} lascerete
l'abito monastico. Non lo pensate nemmeno! In un solo istante un cattivo
pensiero può sorgere nella mente, e si può uccidere qualcuno. Allo
stesso modo, basta una frazione di secondo perché il bene baleni nella
mente, e già ci siete arrivati. Non pensiate che si debba essere monaci
ordinati da lungo tempo per essere in grado di meditare. La retta
pratica risiede nell'istante in cui produciamo kamma.\footnote{%
  \emph{Kamma}:
  Atto intenzionale compiuto per mezzo del corpo, della parola o della
  mente, il quale conduce sempre a un effetto (\emph{kamma-vipāka}).}
Un cattivo pensiero sorge in un baleno, e prima che ve ne rendiate conto
avrete commesso un'azione legata a un pesante kamma. Così, tutti
i discepoli del Buddha praticarono a lungo, ma raggiunsero
l'Illuminazione con un pensiero che durò un solo momento.

Non siate distratti, neanche nelle piccole cose. Sforzatevi duramente,
cercate di avvicinarvi ai monaci, contemplate le cose e allora
conoscerete i monaci. Bene, è sufficiente, eh? Deve essersi fatto tardi,
alcuni sono assonnati. Il Buddha disse di non insegnare il Dhamma alle
persone assonnate.

