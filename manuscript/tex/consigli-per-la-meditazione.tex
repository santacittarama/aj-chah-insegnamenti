\chapter{Consigli per la meditazione}

\begin{openingQuote}
  \centering

  Il discorso venne tenuto allo Hampstead Vihara a Londra, nel 1977.
\end{openingQuote}

Cercatori del bene che siete qui riuniti, per favore ascoltate con
serenità. Ascoltare il Dhamma con serenità significa ascoltare con mente
unificata, prestando attenzione a ciò che sentite, e poi lasciar andare.
Ascoltare il Dhamma è di grande beneficio. Mentre ascoltiamo il Dhamma
siamo incoraggiati ad assumere il \emph{samādhi} a fondamento sia del
corpo sia della mente, perché si tratta di un modo di praticare il
Dhamma. Ai tempi del Buddha la gente ascoltava i discorsi di Dhamma con
attenzione, con una mente che aspirava alla vera comprensione, e alcuni
realizzavano davvero il Dhamma mentre ascoltavano. Questo luogo si
presta bene alla pratica di meditazione. Dopo aver trascorso qui un paio
di notti, capisco che si tratta di un luogo importante. A un livello
esteriore è già un posto tranquillo, tutto quello che resta da fare è a
livello interiore, nel vostro cuore e nella vostra mente. Perciò chiedo
a tutti voi di fare uno sforzo, di prestare attenzione.

Perché vi siete riuniti qui per praticare la meditazione? Perché il
vostro cuore e la vostra mente non comprendono quel che dovrebbe essere
compreso. In altre parole, non conoscete le cose come sono in verità,
ciò che è. Non sapete quel che è sbagliato e quel che è giusto, ed è
questo a indurvi a soffrire e a dubitare. Perciò, la prima cosa che
dovete fare è sviluppare la serenità. La ragione per cui siete venuti
qui a sviluppare la serenità e il contenimento è che il vostro cuore e
la vostra mente non sono a proprio agio. Le vostre menti non sono
serene, contenute. Sono preda del dubbio e dell'agitazione. Questa è la
ragione per cui oggi siete venuti qui e ora state ascoltando il Dhamma.
Vorrei che vi concentraste e che ascoltaste con attenzione quel che
dico, e vi chiedo il permesso di parlare in modo franco, perché è così
che io sono. Per favore, capite che anche se parlo in modo forte, lo
faccio spinto dalla benevolenza. Vi chiedo di perdonarmi se tra quanto
vi dirò ci sarà qualcosa che vi turba, perché le usanze thailandesi e
quelle occidentali non sono le stesse. In realtà, parlare in modo un po'
forte può essere una cosa buona perché aiuta a spronare le persone che
potrebbero essersi assopite o addormentate. Invece di scuotersi per
ascoltare il Dhamma, potrebbero andare alla deriva verso l'indulgenza e
il risultato sarebbe che non riuscirebbero mai a comprendere nulla.

Sebbene possa sembrare che i modi di praticare sono molti, in realtà ce
n'è solo uno. Come avviene per gli alberi da frutto, è possibile avere
velocemente della frutta piantando una talea, ma l'albero non sarebbe
resistente e non vivrebbe a lungo. Un altro modo consiste nel far
crescere l'albero direttamente dal seme, ciò che produce un albero forte
e resistente. Con la pratica è la stessa cosa. Appena ho cominciato a
praticare, avevo delle difficoltà a comprenderla. Fino a quando non fui
in grado di capire in cosa consistesse la meditazione seduta, essa fu
per me un vero strazio e talora mi portò alle lacrime. A volte miravo
troppo in alto, altre volte non abbastanza in alto, non riuscivo a
trovare mai un punto d'equilibrio. Praticare in modo sereno significa
posizionare la mente né troppo in alto né troppo in basso, ma in un
punto d'equilibrio.

Capisco bene che ciò possa indurvi in confusione, perché provenite da
luoghi diversi e avete praticato in vari modi con differenti maestri.
Giunti a praticare qui, siete tormentati da ogni genere di dubbi. Un
maestro vi dice che dovete praticare in un modo, un altro vi dice che
dovreste farlo in un altro. Insicuri a proposito dell'essenza della
pratica, vi chiedete quale metodo usare. Il risultato è la confusione. I
maestri sono così numerosi e gli insegnamenti così tanti che nessuno sa
come armonizzare la propria pratica. Ne derivano numerosi dubbi e
incertezze. Dovreste perciò cercare di non pensare troppo. Se pensate,
fatelo con consapevolezza. Però, finora il vostro pensiero è stato privo
di consapevolezza. La prima cosa è rendere la vostra mente serena. Dove
c'è conoscenza non c'è bisogno del pensiero; al posto del pensiero
\mbox{sorgerà} la consapevolezza, ed essa si trasformerà a sua volta in
saggezza (\emph{paññā}). Però, il modo ordinario di pensare non è
saggezza, è solo un modo di vagare senza meta della mente che
inevitabilmente diventa agitazione. Questa non è saggezza.

A questo punto non c'è bisogno che pensiate. Avete già pensato molto a
casa vostra, vero? Serve solo ad agitare il cuore. Dovete instaurare un
po' di consapevolezza. Il pensiero ossessivo può perfino farvi piangere:
provateci. Perdervi in treni di pensieri non vi condurrà alla Verità,
non è saggezza. Il Buddha era una persona davvero saggia, aveva imparato
a fermare il pensiero. Allo stesso modo, voi state praticando qui per
fermare il pensiero e, così, arrivare alla pace. Quando si è già sereni
il pensiero non è necessario, e al suo posto sorgerà la saggezza.

Per meditare avete bisogno di pensare solo per decidere che ora è giunto
il momento di addestrare la mente, nient'altro. Non lasciate che la
mente scappi a destra o a sinistra, avanti o indietro, sopra o sotto.
Ora avete solo il compito di praticare la consapevolezza del respiro.
Fissate la vostra attenzione sul capo e spostatela giù, attraverso il
corpo, fino alla punta dei piedi, e poi di nuovo su, verso la sommità
del capo. Con la consapevolezza attraversate da cima a fondo tutto il
corpo, osservandolo con saggezza. Lo facciamo per conseguire una
conoscenza iniziale del modo in cui il corpo è. Poi cominciate la
meditazione, prendendo atto di quello che ora è il vostro unico compito,
osservare le inspirazioni e le espirazioni. Non forzate il respiro a
essere più lungo o corto del normale, fatelo solo continuare con
semplicità. Sul respiro non esercitate alcuna pressione, lasciatelo
fluire in modo uniforme, lasciando andare ogni inspirazione ed
espirazione. Dovete capire che, mentre lo fate, state sì lasciando
andare, ma dovrebbe esserci ancora la consapevolezza. Dovete sostenere
questa consapevolezza, mentre consentite al respiro di entrare e uscire
con agio. Sostenete anche la decisione che ora non avete altri compiti o
responsabilità. Di tanto in tanto possono sorgere pensieri su quello che
avverrà, su ciò che conoscerete o vedrete durante la meditazione ma,
appena sorgono, lasciateli solo cessare da sé, non preoccupatevene
eccessivamente.

Durante la meditazione non c'è bisogno di prestare attenzione alle
impressioni che provengono dai sensi. Ogni volta che la mente è affetta
da un impatto sensoriale, ovunque nella mente si verifichi
un'impressione o una sensazione, lasciatela andare. Non importa che le
sensazioni siano belle o brutte. Non è necessario fare qualcosa con
queste sensazioni, lasciatele solo scomparire e riconducete la vostra
attenzione sul respiro. Sostenete la consapevolezza del respiro che
entra ed esce. Se il respiro è troppo lungo o troppo corto non fatevene
un problema, osservatelo semplicemente senza cercare in alcun modo di
controllarlo o di trattenerlo. In altre parole, non attaccatevi.
Consentite al respiro di continuare così com'è, e la mente si calmerà.
Man mano che continuate, la mente poserà i suoi oggetti e si acquieterà,
il respiro diverrà sempre più leggero fino a che sarà così lieve da
sembrare di non esserci affatto. Sia il corpo sia la mente si sentiranno
leggeri ed energizzati. Tutto quel che rimarrà sarà una conoscenza
unificata. Si potrebbe dire che la mente è cambiata e ha raggiunto uno
stato di quiete.

Se la mente è agitata, instaurate la consapevolezza e inspirate
profondamente, fino a che non c'è più posto per altra aria, poi fatela
uscire tutta, finché non ne resta più. Fate seguire un'altra
inspirazione profonda, e di nuovo fate uscire l'aria. Andate avanti così
per due o tre volte, poi ripristinate la concentrazione. La mente
dovrebbe essere più calma. Quando le impressioni che provengono dai
sensi causano agitazione alla mente, ripetete questo procedimento.
Comportatevi allo stesso modo durante la meditazione camminata. Se
mentre camminate la mente si agita, fermatevi, calmate la mente,
ripristinate la consapevolezza con l'oggetto di meditazione e poi
continuate a camminare. La meditazione seduta e quella camminata sono
essenzialmente la stessa cosa, differiscono solo per la postura del
corpo.

A volte ci può essere il dubbio, e perciò dovete avere \emph{sati},
dovete essere ``Colui che Conosce'', dovete seguire ed esaminare
continuamente la mente agitata, qualsiasi forma assuma. Avere
\emph{sati} significa questo. \emph{Sati} vigila sulla mente e si prende
cura di essa. Dovete sostenere questa conoscenza e, indipendentemente
dalle condizioni in cui la mente si trova, non dovete distrarvi né
smarrirvi nei vostri pensieri. Il trucco consiste nell'avere
\emph{sati}, che prende il controllo e sorveglia la mente. Quando la
mente è unificata grazie a \emph{sati}, emergerà un nuovo tipo di
consapevolezza. La mente che ha sviluppato la calma è tenuta sotto
controllo da quella stessa calma, proprio come un pollo nella stia. Il
pollo non può andarsene in giro, perché deve muoversi solo all'interno
della gabbia. Cammina qui e là ma non si mette nei guai perché è
trattenuto dalla gabbia. Allo stesso modo, quando la mente ha
\emph{sati} ed è calma, vi è una consapevolezza che non causa problemi.
Nessun pensiero o sensazione che ha luogo all'interno della mente calma
è nocivo o induce turbamento.

Alcuni non vogliono sperimentare alcun pensiero o sensazione, ma questo
è troppo. Le sensazioni sorgono all'interno dello stato di calma. La
mente sperimenta nello stesso momento sia le sensazioni sia la calma,
senza esserne disturbata. Quando c'è questa calma, non ci sono
conseguenze dannose. I problemi arrivano quando il ``pollo'' esce dalla
``stia''. Ad esempio, può succedere che stiate osservando il respiro che
entra ed esce ma che vi dimentichiate di voi stessi, consentendo al
pensiero di vagare lontano dal respiro, verso casa, in giro per i negozi
o in numerosi altri posti. Magari passa perfino mezz'ora prima di
rendervi improvvisamente conto che state praticando la meditazione e di
rimproverarvi per la vostra mancanza di \emph{sati}. È qui che dovete
stare davvero attenti, perché questo è il momento in cui il pollo esce
dalla stia: la mente abbandona la sua calma di fondo.

Dovete stare attenti a sostenere questa consapevolezza con \emph{sati},
e cercare di far tornare indietro la mente. Sebbene io usi le parole
``far tornare indietro la mente'', nei fatti la mente non va proprio da
nessuna parte, è solo cambiato l'oggetto della consapevolezza. Dovete
far restare la mente proprio nel ``qui e ora''. Per tutto il tempo che
\emph{sati} ci sarà, ci sarà presenza mentale. Sembra che abbiate
riportato indietro la mente, ma in realtà la mente non è andata da
nessuna parte, è solamente cambiata un po'. Pare che sia andata qui e
là. Però, nei fatti, il cambiamento si verifica in un unico punto.
Quando si riacquista \emph{sati}, con la mente si torna in un attimo
senza essere stati riportati indietro da un qualsiasi altro luogo.

Ci devono essere sia \emph{sati} sia \emph{sampajañña}. \emph{Sati} è
rammemorazione e \emph{sampajañña} è consapevolezza di sé. In questo
momento siete chiaramente consapevoli del respiro. Questo esercizio
dell'osservazione del respiro aiuta \emph{sati} e \emph{sampajañña} a
svilupparsi in modo congiunto. Si dividono i compiti. Avere sia
\emph{sati} sia \emph{sampajañña} è come avere a disposizione due
manovali per sollevare una pesante asse di legno. Supponiamo che ci
siano due persone che cercano di sollevare alcune assi, ma queste sono
così pesanti che devono sforzarsi in modo intollerabile. Ecco che però
arriva una persona piena di buona volontà, che li vede e si affretta ad
aiutarli. Allo stesso modo, quando ci sono \emph{sati} e
\emph{sampajañña}, in loro aiuto sorgerà \emph{paññā}, la saggezza.
Tutti e tre si sosterranno a vicenda.

Con \emph{paññā} ci sarà comprensione degli oggetti sensoriali. Ad
esempio, quando durante la meditazione gli oggetti sensoriali vengono
sperimentati, ciò fa sorgere sensazioni e stati mentali. Potreste
cominciare a pensare a un amico, ma dovrebbe immediatamente intervenire
\emph{paññā} con un ``non importa'', ``fermati'', ``lascia perdere''.
Oppure, se ci sono pensieri a proposito di dove andrete domani, la
risposta potrebbe essere: «~Non m'interessa, non voglio occuparmi di
queste cose.~» Se magari cominciate a pensare ad altre persone, allora
dovreste dire a voi stessi: «~Non voglio farmi coinvolgere.~» «~Lascia
andare e basta.~» Oppure: «~È tutto incerto, non si tratta mai di una
cosa certa.~» È così che dovreste affrontare le cose durante la
meditazione, riconoscerle come «~non è sicuro, non è sicuro~»,
sostenendo questo genere di consapevolezza.

Dovete abbandonare ogni pensiero, il dialogo interiore e i dubbi. Non
fatevi catturare da queste cose durante la meditazione. Alla fine tutto
quel che resterà nella mente nella sua forma più pura saranno
\emph{sati}, \emph{sampajañña} e \emph{paññā}. Tutte le volte che
s'indeboliscono, sorgono dei dubbi. Provate però ad abbandonarli
immediatamente, devono restare solamente \emph{sati}, \emph{sampajañña}
e \emph{paññā}. Cercate di sviluppare \emph{sati} in questo modo, fino a
quando riuscite a mantenerla sempre. Riuscirete allora a comprendere
completamente \emph{sati}, \emph{sampajañña} e \emph{samādhi}.
Focalizzando l'attenzione sul punto in cui ci saranno \emph{sati},
\emph{sampajañña}, \emph{paññā} e \emph{samādhi}, tutte le volte che
proverete attrazione o repulsione per gli oggetti esteriori dei sensi,
sarete in grado di dire a voi stessi: «~Non è sicuro.~» In entrambi i
casi si tratta solo di impedimenti da spazzare via, fino a che la mente
è limpida. Tutto quel che dovrebbe restare è \emph{sati},
rammemorazione, \emph{sampajañña}, chiara consapevolezza,
\emph{samādhi}, mente stabile e incrollabile, e \emph{paññā}, perfetta
saggezza. Per il momento posso dirvi solo questo sulla meditazione.

Ora, per quanto concerne gli strumenti o sussidi per la pratica di
meditazione, nel vostro cuore dovrebbe esserci \emph{mettā}, gentilezza
amorevole. In altri termini, le qualità della generosità, della
gentilezza e della disponibilità. Esse dovrebbero essere sostenute in
quanto fondamenti della purezza mentale. Ad esempio, iniziate con l'atto
del donare per eliminare \emph{lobha}, l'avidità, e l'egoismo. Le
persone egoiste non sono felici. Anche se la gente tende a essere molto
egoista senza comprendere quale influsso eserciti su di loro questo modo
d'essere, l'egoismo induce una sensazione di scontentezza. Potete farne
esperienza in ogni momento, soprattutto quando siete affamati.
Supponiamo che otteniate alcune mele e che abbiate l'opportunità di
dividerle con un amico. Ci pensate per un po' e, anche se vi è
l'intenzione di dare, vorrete dare la mela più piccola. Dare quella
grande sarebbe, beh \ldots{} proprio un peccato. È dura pensare rettamente.
Dite al vostro amico di prenderne una, ma poi aggiungete: «~Prendi
questa!~» E gli date la mela più piccola. Si tratta di una forma di
egoismo di cui la gente di solito non si rende conto. Vi siete mai
comportati in questo modo?

Dovete davvero andare controcorrente per donare. Anche se vorreste
proprio dare solo la mela più piccola, dovete obbligarvi e dare quella
più grande. Appena l'avrete data al vostro amico, vi sentite bene,
interiormente. Addestrare la mente andando controcorrente in questo modo
richiede auto-disciplina: dovete conoscere come si dà e come si
rinuncia, senza consentire all'egoismo di incollarvisi addosso. Quando
avrete imparato a donare, se esitate ancora sul frutto da donare, mentre
starete decidendo sarete turbati e, anche se darete il frutto più
grande, resterà in voi una sensazione di riluttanza. Però, appena
deciderete con fermezza per quello più grande, la questione sarà
conclusa, finita. Questo è andare controcorrente nel modo giusto.

Comportandovi in questo modo otterrete la padronanza di voi stessi. Se
non riuscirete a farlo, sarete vittime di voi stessi e continuerete a
essere egoisti. In passato tutti siamo stati egoisti. È una
contaminazione che deve essere eliminata. Nelle Scritture in pāli,
donare è detto \emph{dāna}, che significa rendere felici gli altri. È
una di quelle condizioni che aiuta a purificare la mente dalle
contaminazioni. Rifletteteci su e sviluppatelo nella vostra pratica.

Potreste pensare che praticare in questo modo significa perseguitare se
stessi, ma non è così. In verità significa perseguitare la brama e le
contaminazioni. Se in voi sorgono delle contaminazioni, dovete fare
qualcosa per porvi rimedio. Le contaminazioni sono come un gatto
randagio. Se gli date tutto il cibo che vuole, vi girerà sempre attorno
alla ricerca di altro cibo. Se però smettete di nutrirlo, dopo un paio
di giorni la pianterà di girarvi attorno. Con le contaminazioni è lo
stesso, non verranno a disturbarvi, lasceranno in pace la vostra mente.
Perciò, invece di aver paura delle contaminazioni, fate in modo che le
contaminazioni abbiano paura di voi. Per far in modo che le
contaminazioni vi temano, dovete vedere il Dhamma dentro la vostra
mente.

Dov'è che sorge il Dhamma? Sorge con la nostra conoscenza e comprensione
in questo modo. Tutti sono in grado di conoscere e di comprendere il
Dhamma. Non si tratta di una cosa che deve essere cercata nei libri, non
dovete studiare molto per vederlo, basta riflettere in questo stesso
momento per riuscire a capire di cosa sto parlando. Tutti possono
vederlo perché si trova proprio nel nostro cuore. Tutti hanno
contaminazioni, vero? Se siete capaci di vederle, potete capire. In
passato vi siete presi cura delle vostre contaminazioni e le avete
coccolate, ma ora dovete conoscere le vostre contaminazioni e non dovete
consentire a esse di venire a infastidirvi.

Il successivo elemento costitutivo della pratica è \emph{sīla}, il
contenimento morale. \emph{Sīla} sorveglia e nutre la pratica nello
stesso modo in cui i genitori si prendono cura dei loro figli. Mantenere
il contenimento morale non significa solo evitare di nuocere agli altri,
ma anche aiutarli e incoraggiarli. Dovreste osservare almeno i Cinque
Precetti, ossia:

\begin{itemize}

\item Non solo non uccidere o non nuocere deliberatamente, ma effondere
  benevolenza nei riguardi di tutti gli esseri.

\item Essere onesti, astenersi dal violare i diritti degli altri, in altre
  parole non rubare.

\item Essere moderati nei rapporti sessuali. Nella vita di coppia esiste la
  struttura famigliare, che ruota attorno al marito e alla moglie. Conoscere il
  proprio marito o la propria moglie, conoscere la moderazione, conoscere i
  giusti limiti dell'attività sessuale. Alcuni non hanno limiti. Un marito o una
  moglie non bastano, ne hanno bisogno di due o tre. Secondo me, non si può del
  tutto accontentare neanche un solo partner e perciò averne due o tre significa
  solo indulgere alla propria insoddisfazione. Dovete cercare di purificare la
  mente e addestrarla a conoscere la moderazione. Conoscere la moderazione è
  vera purezza, senza di essa il vostro comportamento è privo di limiti. Quando
  mangiate del cibo delizioso, non soffermatevi troppo sul sapore che ha,
  pensate al vostro stomaco e tenete conto della quantità di cibo che vi è
  indispensabile. Se mangiate troppo avrete problemi, ed è per questo che dovete
  conoscere la moderazione.

\item Essere sinceri quando si parla, anche questo è uno strumento per sradicare
  le contaminazioni. Dovete essere sinceri e onesti, veritieri e retti.

\item Astenersi dall'assumere sostanze intossicanti. Dovete conoscere il
  contenimento e, preferibilmente, rinunciare del tutto a queste cose. La gente
  è già intossicata abbastanza dalla propria famiglia, dai propri parenti e
  amici, dai possessi materiali, dal benessere e da tutto il resto. È già
  abbastanza, non c'è bisogno di peggiorare le cose assumendo pure intossicanti.
  Queste cose servono solo a oscurare la mente. Coloro che ne assumono grandi
  quantità dovrebbero gradualmente diminuire l'uso di esse e infine rinunciarvi
  del tutto.

\end{itemize}

Dovrei forse chiedervi di perdonarmi, ma se parlo in questo modo è
perché mi preoccupo del vostro benessere, affinché possiate comprendere
ciò che è bene. Avete bisogno di sapere come stanno le cose. Cos'è che
vi opprime quotidianamente? Quali sono i comportamenti che causano
questa oppressione? Le buone azioni portano buoni effetti e quelle
cattive effetti cattivi. Queste sono le cause.

Allorché il contenimento morale sarà puro, vi sentirete onesti e gentili
nei riguardi degli altri. Ciò porterà con sé appagamento e libertà da
preoccupazioni e rimorsi. Non ci saranno quei rimorsi che provengono da
comportamenti aggressivi e offensivi. È una forma di felicità. È una
condizione quasi paradisiaca. Vi è benessere, mangiate e dormite a
vostro agio grazie alla felicità che sorge dal contenimento morale.
Questo è il risultato. Sostenere il contenimento morale è la causa. È un
principio della pratica di Dhamma: astenersi dalle cattive azioni in
modo che la bontà possa sorgere. Se il contenimento morale viene
sostenuto in questo modo, il male scomparirà e il bene sorgerà al suo
posto. È il risultato della retta pratica.

Questa però non è la fine della storia. Quando le persone hanno ottenuto
un po' di felicità, tendono a essere distratte e non vanno più avanti
nella pratica. Restano bloccate nella felicità. Non vogliono progredire
ulteriormente, preferiscono la felicità del ``paradiso''. Si sta bene,
ma non c'è reale comprensione. Dovete continuare a riflettere per
evitare le illusioni. Si tratta di una cosa transitoria, non dura per
sempre. Presto ve ne separerete. Non è cosa sicura. Quando la felicità
scompare, al suo posto sorge la sofferenza e sopraggiungono di nuovo le
lacrime. Perfino gli esseri celestiali finiscono per piangere e per
soffrire.

Perciò il Buddha ci insegnò a riflettere sugli svantaggi della felicità,
sul fatto che in essa vi è un aspetto insoddisfacente. Di solito, quando
si sperimenta una felicità di questo genere, non vi è una reale
comprensione di essa. La pace che è davvero certa e durevole è coperta
da questa ingannevole felicità. Questa felicità non è un genere di pace
certa e permanente, bensì, piuttosto, un tipo di contaminazione, un
raffinato tipo di contaminazione al quale ci attacchiamo. A tutti piace
essere felici. La felicità sorge a causa del fatto che qualcosa ci
piace. Non appena quel piacere si traforma in dispiacere, sorge la
sofferenza. Dobbiamo riflettere su questa felicità per comprenderne i
limiti e l'incertezza. Quando le cose cambiano, sorge la sofferenza.
Anche questa sofferenza è incerta. Non pensiate che sia stabile o
assoluta. Questa maniera di riflettere è detta \emph{ādīnavakathā},
riflessione sull'inadeguatezza e sui limiti del mondo dei fenomeni
condizionati. Ciò significa riflettere sulla felicità, invece di
accettarla come se avesse valore. Comprendendo che è incerta, non
dovreste attaccarvi saldamente a essa. Dovreste tenerla, ma poi
lasciarla andare, comprendendo sia i benefici sia i pericoli della
felicità. Per meditare abilmente dovete vedere gli svantaggi insiti
nella felicità. Riflettete in questo modo. Quando sorge la felicità,
contemplatela accuratamente fino a quando ne appaiono gli svantaggi.

Quando vedrete che le cose sono imperfette (\emph{dukkha}), il vostro
cuore perverrà a comprendere il \emph{nekkhammakathā}, la riflessione
sulla rinuncia. La mente si disinteresserà e cercherà una via d'uscita.
Il disinteresse proviene dall'aver compreso il modo in cui le forme sono
nella realtà, il modo in cui i sapori sono nella realtà, il modo in cui
sono veramente l'amore e l'odio. Con ``disinteresse'' intendiamo che non
c'è più il desiderio di aggrapparsi o di attaccarsi alle cose. C'è una
rinuncia all'attaccamento che ci consente di dimorare serenamente e di
osservare le cose con un'equanimità che è libera dall'attaccamento.
Questa è la pace che sorge dalla pratica.

