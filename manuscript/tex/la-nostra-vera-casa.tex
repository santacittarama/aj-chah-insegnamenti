\chapter{La nostra vera casa}

\begin{openingQuote}
  \centering

  Il discorso è rivolto a un'anziana discepola laica prossima alla morte.
\end{openingQuote}

Nella tua mente, decidi ora di ascoltare con rispetto il Dhamma. Mentre
parlo, stai attenta alle mie parole come se davanti a te sedesse il
Buddha. Chiudi gli occhi, fai in modo di sentirti a tuo agio, calma e
unifica la mente. Per mostrare il tuo rispetto all'Essere Compiutamente
Illuminato consenti con umiltà alla Triplice Gemma della saggezza, della
verità e della purezza di dimorare nel tuo cuore. Oggi non ho portato
nulla di concreto, di materiale da offrirti, solo il Dhamma, solo gli
insegnamenti del Buddha. Dovresti capire che perfino il Buddha,
nonostante tutte le virtù da Lui accumulate, non poté evitare la morte
fisica. Quando raggiunse la vecchiaia, abbandonò il corpo e lasciò
andare questo pesante fardello. Ora devi imparare ad accontentarti dei
numerosi anni durante i quali hai fatto affidamento sul corpo. Dovresti
avere la sensazione che è abbastanza.

È come per gli utensili domestici che hai usato per molto tempo, tazze,
piatti, piattini e così via. Erano puliti e lucenti quando erano nuovi
ma adesso, dopo averli usati a lungo, stanno cominciando a consumarsi.
Alcuni si sono già rotti, altri sono andati perduti, e quelli che sono
rimasti si stanno consumando, non hanno una forma stabile. Sono così a
causa della loro natura. Per il tuo corpo è la stessa cosa. È cambiato
in continuazione dal giorno in cui sei nata e, attraversando
fanciullezza e gioventù, ha ora raggiunto la vecchiaia. Devi accettarlo.
Il Buddha disse che i fenomeni condizionati, siano essi interni o
esterni, sono non-sé, la loro natura è il cambiamento. Contempla con
chiarezza questa verità.

Proprio questo aggregato di carne che è, qui, sulla via della
disgregazione, è realtà (\emph{saccadhamma}). Gli avvenimenti di questo
corpo sono realtà, costituiscono l'insegnamento senza tempo del Buddha.
Il Buddha ci insegnò a contemplarli e a fare i conti con la loro natura.
Dobbiamo avere la capacità di essere in pace con il corpo, non importa
in quali condizioni sia. Il Buddha insegnò che dovremmo avere la
certezza che è solo il corpo a essere incarcerato, che la mente non è
imprigionata insieme a esso. Adesso che il corpo si deteriora e consuma
con l'età, non resistere, ma nello stesso tempo non permettere alla tua
mente di deteriorarsi con esso. Mantieni la mente separata. Dai energia
alla mente, comprendendo la verità del modo in cui le cose sono. Il
Buddha insegnò che questa è la natura del corpo, non può essere in alcun
altro modo. Dopo essere nato, il corpo invecchia e si ammala, e alla
fine muore. Questa è una grande verità, della quale ora sei testimone.
Guarda con saggezza il corpo e comprendilo.

Se la tua casa è allagata o rasa al suolo dalle fiamme, quale che sia la
minaccia che incombe su di essa, fai in modo che riguardi solo la casa.
Se c'è un'inondazione, non consentire alla mente d'essere inondata. Se
c'è il fuoco, non consentire al cuore di bruciare. Lascia che sia solo
la casa, che sia solo ciò che è all'esterno a essere inondato o
bruciato. Per la tua mente questo è il momento di lasciar andare gli
attaccamenti.

È da molto che sei in vita. I tuoi occhi hanno visto ogni tipo di forme
e di colori, i tuoi orecchi hanno sentito tanti suoni, hai avuto un gran
numero di esperienze. E questo è tutto ciò che sono state: solo
esperienze. Hai mangiato cibi deliziosi, e tutti quei buoni sapori erano
solo buoni sapori, nulla di più. I cattivi sapori erano solo cattivi
sapori, tutto qui. Se gli occhi vedono una bella forma è tutto qui, una
bella forma. Una forma sgradevole è solo una forma sgradevole. Gli
orecchi sentono un suono incantevole e melodioso, non è nulla di più di
quel che è. Un suono stridente e non armonioso è solo quello che è.

Il Buddha disse che ricchi e poveri, giovani e anziani, esseri umani e
animali, non esiste creatura in questo mondo che possa conservarsi in
un'identica condizione per lungo tempo. Tutto sperimenta cambiamento e
privazione. È un dato di fatto della vita, non possiamo far nulla per
porvi rimedio. Il Buddha disse che, però, possiamo contemplare il corpo
e la mente per vedere che sono impersonali, che né l'uno né l'altra sono
``io'' o ``mio''. Hanno solo una realtà transitoria. Ti appartengono
solo nominalmente, come questa casa. Non puoi portarla con te ovunque.
Lo stesso vale per la tua salute, i tuoi possessi e la tua famiglia.
Sono tuoi solo nominalmente. In realtà non ti appartengono, appartengono
alla natura.

Questa verità non si applica solo a te, siamo tutti sulla stessa barca,
perfino il Buddha e i suoi discepoli illuminati. Sono diversi da noi
solo per una ragione, l'accettazione del modo in cui sono le cose.
Capirono che non poteva essere altrimenti.

Per questo motivo il Buddha ci insegnò a investigare ed esaminare il
corpo, dalla pianta dei piedi fino alla sommità della testa, e poi
viceversa, giù di nuovo verso la pianta dei piedi. Anche se dai solo
un'occhiata al corpo, che genere di cose vedi? C'è qualcosa
d'intrinsecamente pulito? Riesci a individuare qualche sostanza
durevole? Tutto il corpo degenera costantemente. Il Buddha ci insegnò a
capire che non ci appartiene. Per il corpo è naturale essere così,
perché tutti i fenomeni condizionati sono soggetti al cambiamento. In
quale altro modo vorresti che fosse? Nei fatti non c'è nulla di
sbagliato nel modo in cui il corpo è. Non è il corpo che causa
sofferenza, sono i pensieri errati. Quando le cose sono viste in modo
errato, è inevitabile che ci sia confusione.

È come l'acqua di un fiume. Fluisce naturalmente a valle, mai a monte. È
la natura. Se una persona si mettesse in piedi sulla riva di un fiume e
volesse che l'acqua fluisse a monte, sarebbe folle. Ovunque si trovasse
ad andare, questo suo pensiero malsano non gli consentirebbe di avere la
mente serena. Soffrirebbe a causa della sua errata visione, del suo
pensiero controcorrente. Se avesse Retta Visione vedrebbe che l'acqua
deve inevitabilmente scorrere a valle e, fino a che non riuscirà a
comprendere e ad accettare questo dato di fatto, quella persona sarà
sconcertata e frustrata.

Il fiume che scorre seguendo la pendenza è come il tuo corpo. Dopo
essere stato giovane, il corpo invecchia e serpeggia in direzione della
morte. Non desiderare che sia altrimenti, non è una cosa alla quale hai
il potere di porre rimedio. Il Buddha ci insegnò a vedere il modo in cui
sono le cose, e poi a lasciar andare il nostro attaccamento a esse.
Prendi questa sensazione del lasciar andare come rifugio.

Continua a meditare anche se ti senti stanca, se sei esausta. Lascia che
la tua mente rimanga col respiro. Fai qualche respiro profondo, e poi
fissa l'attenzione sul respiro, usando come mantra la parola
\emph{Bud-dho}. Rendi continua questa pratica. Più ti senti esausta, più
tenue e focalizzata dev'essere la tua concentrazione, per consentirti di
far fronte a ogni sensazione dolorosa che dovesse sorgere. Quando inizi
a sentirti affaticata arresta tutti i pensieri, lascia che la mente si
unifichi e poi torna alla consapevolezza del respiro. Continua con la
recitazione interiore, \emph{Bud-dho}, \emph{Bud-dho}. Lascia andare
tutte le cose esteriori. Non aggrapparti ai pensieri riguardanti i tuoi
figli e parenti, non aggrapparti a nulla, di qualsiasi cosa si tratti.
Lascia andare. Lascia che la mente si unifichi in un punto e che dimori
con compostezza con il respiro. Lascia che il respiro sia l'unico
oggetto di conoscenza. Concentrati finché la mente non diventa sempre
più sottile, finché le sensazioni diventano insignificanti, finché
giungono grande chiarezza e vigilanza interiori. Allora ogni sensazione
dolorosa cesserà gradualmente, da sé.

Infine osserverai il respiro come se si trattasse di parenti venuti a
farti visita. Quando se ne vanno, esci con loro per salutarli. Li guardi
fino a quando percorrono tutto il viale e sono fuori dalla tua vista, e
poi rientri in casa. Noi osserviamo il respiro in questo stesso modo. Se
il respiro è pesante sappiamo che è pesante, se è lieve sappiamo che è
lieve. Continuiamo a seguirlo man mano che diventa sempre più sottile,
nel contempo tenendo desta la mente. Col passare del tempo il respiro
scompare del tutto e resta solo una sensazione di vigilanza. Noi diciamo
che questo è incontrare il Buddha. Abbiamo quella chiara, vigile
consapevolezza chiamata \emph{Bud-dho}, Colui che Conosce, il
Risvegliato, il Radioso. Questo è incontrare il Buddha e dimorare con
Lui, con la conoscenza e con la chiarezza. Fu solo il Buddha storico a
morire. Il vero Buddha, il Buddha che è chiaro, radioso conoscere può
ancora essere sperimentato e raggiunto al giorno d'oggi. E se lo
raggiungiamo, il cuore è unificato.

Perciò deponi tutto, tutto a parte il conoscere. Se sorgono visioni o
suoni nella tua mente durante la meditazione, non farti ingannare.
Lasciali perdere. Non afferrare assolutamente nulla, resta solo con
questa consapevolezza unificata. Non preoccuparti del passato o del
futuro, resta immobile e raggiungerai il posto in cui non si va né
avanti né indietro, e neanche si resta fermi, dove non c'è nulla cui
aggrapparsi o attaccarsi. Perché? Perché non c'è alcun sé, non c'è
``io'' o ``mio''. Tutto è andato. Il Buddha insegnò a svuotare se stessi
di qualsiasi cosa in questo modo, senza portarsi dietro nulla; ci
insegnò a conoscere e, dopo aver conosciuto, a lasciar andare.

Realizzare il Dhamma, realizzare il Sentiero verso la libertà dal ciclo
della nascita e della morte, è un compito che ognuno di noi deve
svolgere da solo. Continua perciò a cercare di lasciar andare e di
comprendere gli insegnamenti. Impegnati nella contemplazione. Non
preoccuparti della tua famiglia. In questo momento sono così come sono,
in futuro saranno come te. Non c'è nessuno al mondo che possa sfuggire a
questo destino. Il Buddha insegnò a deporre le cose prive di essenza
durevole. Se deponi tutto vedrai la Verità, se non lo fai, non la
vedrai. Così stanno le cose. È così per tutti, nel mondo. Per questa
ragione non aggrapparti a nulla.

Se ti sorprendi a pensare va bene, sempreché tu lo faccia saggiamente.
Non pensare con stoltezza. Se pensi ai tuoi figli, pensa a loro con
saggezza, non in modo stolto. A qualsiasi cosa la mente si rivolga,
pensa con saggezza, sii consapevole della natura di ciò a cui pensi.
Conoscere una cosa con saggezza è lasciarla andare senza soffrire per
quella cosa. La mente è luminosa, gioiosa e in pace. Si tiene lontana
dalle distrazioni ed è unificata. Per ricevere aiuto e sostegno puoi
osservare il tuo respiro, proprio ora.

È tuo compito, non quello di qualcun altro. Lascia che gli altri
facciano il loro lavoro. Tu hai le tue responsabilità e i tuoi doveri,
non devi assumerti quelli della tua famiglia. Non farti carico di
nient'altro, lascia andare tutto. Questo lasciar andare renderà la mente
calma. Ora il tuo unico compito è concentrare la mente e condurla alla
pace. Lascia agli altri tutto il resto. Forme, suoni, odori, sapori \ldots{}
lascia che se ne occupino gli altri. Gettati tutto alle spalle e fai il
tuo lavoro, assolvi a questa tua responsabilità. Qualsiasi cosa sorga
nella tua mente, sia essa paura o dolore, timore della morte, ansia
riguardo agli altri o altro ancora, dille: «~Non disturbarmi, non sei
più una delle mie preoccupazioni.~» Attieniti a questo, quando vedi
sorgere quei ``dhamma''.

A cosa si riferisce la parola dhamma? Tutto è dhamma, non
c'è nulla che non sia dhamma. E il ``mondo''? Il mondo è proprio
quello stato mentale che in questo momento ti rende agitata. «~Cosa
faranno? Quando non ci sarò più, chi si occuperà di loro? Come
faranno?~» Proprio tutto questo è il ``mondo''. Anche il semplice
sorgere di un pensiero di timore per la morte o per il dolore è il
mondo. Getta via il mondo! Il mondo è così com'è. Se gli consenti di
dominare la mente, essa si oscurerà e non riuscirà a vedere se stessa.
Qualsiasi cosa appaia nella mente, dille solo: «~Non è affar mio, è
impermanente, insoddisfacente, privo di un sé.~»

Pensare che vorresti vivere a lungo ti farà soffrire. Non è però giusto
neanche pensare che vorresti morire subito o molto in fretta. È
sofferenza, non è vero? I fenomeni condizionati non ci appartengono,
seguono le loro leggi naturali. Non puoi fare nulla per cambiare il modo
in cui il corpo è. Puoi abbellirlo un po', renderlo attraente e pulito
per un po', come le ragazze che si mettono il rossetto e si fanno
crescere le unghie, ma quando la vecchiaia arriva, siamo tutti sulla
stessa barca. Il corpo è così, non puoi renderlo diverso. Quel che puoi
migliorare e abbellire è la mente.

Tutti possono costruire una casa di legno e mattoni, ma il Buddha
insegnò che questo tipo di casa non è la nostra vera casa, è nostra solo
nominalmente. È la casa nel mondo e segue le vie del mondo. La nostra
vera casa è la pace interiore. Una casa materiale, esteriore, può anche
essere bella, ma serena non lo è affatto. C'è prima questa
preoccupazione e poi quella, prima un'ansia poi un'altra. Per questo
diciamo che non è la nostra vera casa, è esterna a noi. Prima o poi
dovremo rinunciarvi. Non è un posto nel quale possiamo vivere
permanentemente, perché non ci appartiene davvero, appartiene al mondo.
Lo stesso avviene con il nostro corpo. Lo consideriamo come un sé, come
``io'' o ``mio'', ma nei fatti non è assolutamente così, è un'altra casa
nel mondo. Il tuo corpo ha seguito il suo corso naturale fin dalla
nascita, e ora è vecchio e malato, non puoi impedirgli di esserlo. È
così che stanno le cose. Volere che stiano diversamente sarebbe sciocco
come volere che un'anatra sia uguale a una gallina. Quando vedrai che è
impossibile -- che un'anatra deve essere un'anatra e una gallina deve
essere una gallina -- troverai coraggio ed energia. Per quanto tu voglia
che il corpo continui a durare, non lo farà. Il Buddha disse:

\emph{Aniccā vata saṅkhāra}

Tutti i fenomeni condizionati sono impermanenti

\emph{Uppāda-vaya-dhammino}

Soggetti a sorgere e a scomparire

\emph{Uppajjitvā nirujjhanti}

Dopo essere sorti, cessano

\emph{Tesam vūpasamo sukho}

Il loro placarsi è beatitudine.

La parola \emph{saṅkhāra} si riferisce a questo corpo e a questa mente.
I \emph{saṅkhāra} sono impermanenti e instabili. Dopo essere giunti
all'esistenza scompaiono, dopo essere sorti declinano, però tutti
vogliono che siano permanenti. È follia. Osserva il respiro. Prima
entra, poi esce, è la sua natura, è così che deve essere. Inspirazione
ed espirazione devono alternarsi, ci deve essere cambiamento. I fenomeni
condizionati esistono mediante il cambiamento. Non puoi impedirlo.
Pensa, potresti espirare senza inspirare? Ci si sentirebbe bene? Oppure,
potresti solo inspirare? Vogliamo che le cose siano permanenti, ma non
si può, è impossibile. Dopo che il respiro è entrato, deve uscire. Una
volta che è uscito, di nuovo rientra, ed è una cosa naturale, o no?
Essendo nati, invecchiamo e poi si muore, e questo è assolutamente
naturale e normale. È perché i fenomeni condizionati hanno fatto il loro
lavoro, è perché inspirazioni ed espirazioni si sono alternate in questo
modo che gli esseri umani sono qui ancora oggi.

Siamo morti da quando siamo nati. La nostra nascita e la nostra morte
sono una sola cosa. È come un albero: quando ci sono radici ci devono
essere i rami, quando ci sono i rami ci devono essere le radici. Non si
possono avere le une senza gli altri. È un po' strano notare quanto la
gente sia addolorata e perplessa in occasione di una morte e quanto sia
lieta e felice per una nascita. È un'illusione, nessuno vede mai queste
cose con chiarezza. Penso che se si voglia davvero piangere, sarebbe
meglio farlo quando qualcuno nasce. Nascere è morire, morire è nascere.
Il ramo è la radice, la radice è il ramo. Se si vuole piangere, che si
pianga alla radice, alla nascita. Guarda più da vicino: se non ci fosse
nascita non ci sarebbe morte. Riesci a capirlo?

Non ti preoccupare troppo per le cose, pensa solo: «~Questo è il modo in
cui sono le cose.~» Questo è il tuo lavoro, il tuo dovere. Ora nessuno
può aiutarti, non c'è nulla che la tua famiglia o le tue proprietà
possano fare per te. Quello che può aiutarti ora è la chiara
consapevolezza. Non esitare. Lascia andare. Getta via tutto. Anche se
non sei tu a lasciar andare, ad ogni modo tutto sta cominciando a
lasciarti. Riesci a vedere come tutte le parti del tuo corpo stanno
cercando di svignarsela? I tuoi capelli, ad esempio. Quando eri giovane,
erano folti e neri. Adesso stanno cadendo. Stanno andando via. I tuoi
occhi erano buoni e forti, ora sono deboli, la tua vista è sfuocata.
Quando i tuoi organi ne hanno abbastanza se ne vanno, questa non è la
loro casa. Quando eri bambina i tuoi denti erano sani e saldi, ora
traballano, oppure sono finti. I tuoi occhi, i tuoi orecchi, il tuo
naso, la tua lingua, tutto sta cercando di andare perché non è la sua
casa. I fenomeni condizionati non possono essere una casa permanente,
puoi restare con loro solo poco tempo, ma poi si deve andare. È come un
inquilino che sorveglia la sua piccola casa con gli occhi stanchi. I
suoi denti non sono poi così buoni, i suoi occhi non sono poi così
buoni, il suo corpo non è poi così in salute, tutto lo sta lasciando.

Non c'è bisogno che ti preoccupi di nulla, perché questa non è la tua
vera casa, è solo un ricovero temporaneo. Siccome sei venuta al mondo,
devi contemplarne la natura. Tutto quello che c'è si prepara a
scomparire. Guarda il tuo corpo. Vedi qualcosa che sia ancora nella sua
forma originaria? La tua pelle è com'era di solito? E i tuoi capelli?
Non sono gli stessi, vero? Dov'è andato tutto quanto? Questa è la
natura, il modo in cui sono le cose. Quando il tempo è finito, i
fenomeni condizionati vanno per la loro strada. In questo mondo non c'è
nulla su cui fare affidamento, si gira in tondo senza fine tra
turbamenti e problemi, piacere e dolore. Non c'è pace.

Quando non abbiamo una vera casa, siamo per strada come viaggiatori
senza meta, andiamo qua e là, ci fermiamo per un po' e poi si parte di
nuovo. Fino a che non torniamo nella nostra vera casa non ci sentiamo a
nostro agio, proprio come chi ha lasciato il proprio paese. Solo quando
torniamo a casa possiamo davvero rilassarci ed essere in pace.

In nessun posto al mondo può esservi vera pace. Il povero non ha pace e
nemmeno il ricco, gli adulti non hanno pace e nemmeno le persone molto
colte. Non c'è pace da nessuna parte, questa è la natura del mondo.
Quelli che hanno pochi possessi soffrono, e così pure chi ne ha molti. I
bambini, gli anziani, i giovani \ldots{} tutti soffrono. La sofferenza di
essere anziani, la sofferenza di essere giovani, la sofferenza di essere
benestanti e la sofferenza di essere poveri: non c'è altro che
sofferenza. Se contempli le cose in questo modo vedrai \emph{aniccā},
l'impermanenza, e \emph{dukkha}, l'insoddisfazione. Perché le cose sono
impermanenti e insoddisfacenti. Perché sono \emph{anattā}, non-sé.

Sia il tuo corpo che giace malato e dolorante sia la mente che è
consapevole della malattia e del dolore, sono detti dhamma.
Quello che è privo di forma, i pensieri, le sensazioni e le percezioni,
è \emph{nāma-dhamma}. Ciò che è tormentato dal dolore e dalla sofferenza
è \emph{rūpa-dhamma}. Quello che è materiale è dhamma e quello
che è immateriale è dhamma. Perciò noi viviamo con i
dhamma, nei dhamma e siamo dhamma. In verità non
c'è alcun sé, ci sono solamente dhamma che sorgono e svaniscono
in continuazione come è nella loro natura. Ogni momento siamo soggetti a
nascita e morte. Questo è il modo in cui sono le cose.

Quando pensiamo al Buddha, al modo veritiero in cui parlò, sentiamo
quanto Egli sia degno di reverenza e rispetto. Ogni volta che vediamo la
verità di qualcosa, vediamo i suoi insegnamenti, pure se non abbiamo mai
praticato il Dhamma. Però, anche se abbiamo una conoscenza degli
insegnamenti, li abbiamo studiati e praticati, non abbiamo una casa fino
a quando non vediamo la Verità.

Perciò, comprendi quel che ora ti dico. Tutte le persone, tutte le
creature si stanno preparando ad andarsene. Quando gli esseri hanno
vissuto per un tempo appropriato, devono andare per la loro strada.
Ricchi, poveri, giovani e vecchi, tutti devono sperimentare questo
cambiamento. Quando capirai che così è il mondo, sentirai che è un posto
faticoso. Quando vedrai che non c'è niente di reale o di sostanziale su
cui fare affidamento, proverai stanchezza e disincanto. Essere
disincantati non significa provare avversione. La mente è chiara. Vede
che non c'è nulla che possa essere fatto per porre rimedio a questo
stato di cose, si tratta solo del modo in cui è il mondo. Conoscendo in
questa maniera puoi lasciar andare gli attaccamenti. Puoi lasciar andare
con una mente che non è né felice né triste, ma in pace con i fenomeni
condizionati perché ha visto con saggezza la loro natura mutevole.
\emph{Aniccā vata saṅkhāra}: tutti i fenomeni condizionati sono
impermanenti.

Per dirlo in modo semplice, l'impermanenza è il Buddha. Se davvero
vediamo un fenomeno condizionato impermanente, vedremo quello che è
permanente. È permanente nel senso che il suo essere soggetto al
cambiamento è immutabile. Questa è la permanenza che caratterizza gli
esseri viventi. Vi è trasformazione continua, dalla fanciullezza alla
vecchiaia, e proprio quest'impermanenza, questa propensione al
cambiamento, è permanente e stabile. Se guardi così il tutto, il tuo
cuore sarà a proprio agio. Non sei solo tu a doverci passare, è così per
tutti.

Se consideri le cose in questo modo, le vedrai come tediose, e sorgerà
il disincanto. La tua gioia nel mondo dei piaceri dei sensi scomparirà.
Vedrai che se hai molti possessi, devi lasciare molto dietro di te. Se
hai poco, devi lasciare poco. La ricchezza è solo ricchezza, una vita
lunga è solo una vita lunga. Niente di speciale. Quello di importante
che dovremmo fare è, come insegnò il Buddha, costruire la nostra vera
casa, costruirla con il metodo che ti ho spiegato. Costruisci la tua
vera casa. Lascia andare. Lascia andare finché la mente raggiunge quella
pace che è libertà dall'andare avanti, libertà dall'andare indietro e
libertà pure dal fermarsi. Il piacere non è la tua casa, il dolore non è
la tua casa. Tanto il piacere quanto il dolore tramontano e passano.

Il grande Maestro vide che tutti i fenomeni condizionati sono
impermanenti e perciò ci insegnò a lasciar andare il nostro attaccamento
a essi. Quando raggiungeremo la fine della vita, in qualsiasi caso non
avremo scelta, non potremo portare nulla con noi. Non sarebbe meglio
lasciare le cose prima di allora? Sono solo un pesante fardello da
portare in giro, perché di questo carico non ce ne liberiamo adesso?
Perché preoccuparsi di trascinare queste cose? Lascia andare, rilassati,
e consenti alla tua famiglia di prendersi cura di te.

Chi assiste un malato cresce in bontà e virtù. Il malato, che dà agli
altri quest'opportunità, non dovrebbe rendere le cose difficili. Se c'è
dolore o qualche problema o altro ancora, lo dica e conservi la mente in
uno stato salutare. Chi assiste i genitori dovrebbe colmare la loro
mente con il calore e la gentilezza, senza lasciarsi catturare
dall'avversione. Questo è il momento in cui dovete ripagare il vostro
debito nei loro riguardi. Dalla vostra nascita fino alla vostra
fanciullezza, finché non siete cresciuti, siete stati dipendenti dai
vostri genitori. Se oggi siete qui, è grazie a vostra madre e a vostro
padre che vi hanno aiutato in moltissimi modi. Avete un grandissimo
debito di gratitudine.

Così, oggi, tutti voi, figli e parenti qui riuniti, osservate come
vostra madre sia diventata vostra figlia. Prima eravate voi i suoi
figli, adesso lei è diventata vostra figlia. È diventata sempre più
anziana, fino a diventare di nuovo una bambina. La sua memoria è andata,
i suoi occhi non vedono bene e il suo udito non è ottimo. Talvolta si
confonde con le parole. Non irritatevi. Anche voi che assistete la
malata dovete saper lasciar andare. Non attaccatevi alle cose,
permettetele di fare come vuole. A volte, quando un bambino
disobbedisce, i genitori gli consentono di fare a modo suo per mantenere
la pace, solo per farlo felice. Ora vostra madre è proprio come quel
bambino. I suoi ricordi e le sue percezioni sono confusi. A volte
confonde i vostri nomi o vi chiede di portarle una tazza quando vuole un
piatto. È normale, non arrabbiatevi per queste cose.

Consentite alla malata di rammentarsi della gentilezza di chi la assiste
e di sopportare con pazienza le sensazioni dolorose. Tu esercitati
mentalmente, non lasciare che la mente si disperda e si confonda, e non
rendere le cose difficili a chi si prende cura di te. Permetti a coloro
che ti assistono di colmare le loro menti di virtù e gentilezza. E voi
non siate avversi all'aspetto poco piacevole del vostro compito, quando
la ripulite dal muco e dagli altri umori, dall'urina e dagli escrementi.
Fate del vostro meglio. Tutti in famiglia diano una mano. È l'unica
madre che avete. Vi ha dato la vita, è stata la vostra insegnante, il
vostro dottore e la vostra nutrice, è stata tutto per voi. La bontà dei
genitori è in lei rappresentata dal fatto che vi abbia fatto crescere,
che abbia condiviso con voi i suoi beni e vi abbia reso suoi eredi.
Questa è la ragione per cui il Buddha insegnò le virtù \emph{kataññū} e
\emph{katavedī}:\footnote{\emph{Kataññu}: Questa parola significa
  letteralmente ``conoscere'', riconoscere ciò che è stato fatto a un
  qualcuno, ossia essere grati; viene spesso utilizzato insieme a
  \emph{katavedī}, per indicare gratitudine e coscienza dei benefici
  ricevuti.} conoscere il nostro debito di gratitudine e cercare di
ripagarlo. Questi due dhamma sono complementari. Se i nostri
genitori sono nel bisogno, malati o in difficoltà, allora facciamo del
nostro meglio per aiutarli. \emph{Kataññū-katavedī} è la virtù che
sostiene il mondo. Essa evita che le famiglie si disgreghino e le rende
stabili e armoniose.

Oggi, in questi momenti di malattia, vi ho portato il dono del Dhamma.
Non ho cose materiali da offrirvi, pare che ce ne siano già in
abbondanza in questa casa. Così vi porgo il Dhamma, qualcosa che ha
valore duraturo, che non sarete mai in grado di esaurire. Dopo averlo
ricevuto, potete passarlo a molti altri a vostro piacimento, non perderà
mai valore. Questa è la natura della Verità. Sono felice di essere stato
in grado di offrirvi questo dono del Dhamma e spero che vi dia la forza
per affrontare il vostro dolore.

