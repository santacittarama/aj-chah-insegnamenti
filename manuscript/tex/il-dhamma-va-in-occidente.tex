\chapter{Il Dhamma va in Occidente}

\emph{Domanda:} Un mio amico è andato a praticare con un maestro zen. Gli ha
chiesto: «~Che cosa stava facendo il Buddha mentre sedeva sotto l'albero
della Bodhi?~» Il maestro ha risposto: «~Stava praticando zazen!~» «~Non
ci credo~», ha replicato il mio amico. E il maestro: «~Cosa vuol dire
che non ci credi?~» Il mio amico: «~Ho fatto a Goenka\footnote{Goenka.
  Satya Narayan Goenka (1924-2013) è un rinomato insegnante laico della
  meditazione di tradizione birmana.} la stessa domanda e lui ha
risposto che il Buddha, quando era seduto sotto l'albero della Bodhi,
stava praticando \emph{vipassanā}! Ognuno dice che il Buddha stava
facendo quello che fanno loro.~»

\emph{Risposta:} Quando il Buddha sedeva all'aperto, stava sedendo sotto
l'albero della Bodhi. Non è così? Quando sedeva sotto un altro tipo di
albero, stava sedendo sotto l'albero della Bodhi. Non c'è niente di
errato in queste spiegazioni. ``Bodhi'' significa il Buddha stesso,
Colui che Conosce. Va bene parlare di stare seduti sotto l'albero della
Bodhi, ma sono molti anche gli uccelli che stanno sotto l'albero della
Bodhi. Tante persone stanno sedute sotto l'albero della Bodhi. Sono però
lontani da un tal genere di conoscenza, sono lontani dalla Verità. Sì,
``sotto l'albero della Bodhi'' possiamo dirlo. Le scimmie giocano
sull'albero della Bodhi. La gente siede sotto l'albero della Bodhi. Ma
questo non significa che abbiano una comprensione profonda. Quelli che
hanno una più profonda comprensione capiscono che il vero significato
dell'``albero della Bodhi'' è il Dhamma assoluto.

In questo senso, perciò, per noi è certamente un bene cercare di sedere
sotto l'albero della Bodhi. Possiamo essere il Buddha. Ma non c'è
necessità di discutere con altri in proposito. Quando uno dice che il
Buddha stava facendo un certo tipo di pratica sotto l'albero della Bodhi
e un altro lo contesta, non c'è bisogno di farsi coinvolgere. Dovremmo
guardare la questione da un punto di vista definitivo, quello della
realizzazione della Verità. Vi è pure l'idea convenzionale di ``albero
della Bodhi'', di cui la maggior parte delle persone parla, ma quando ci
sono due tipi di albero della Bodhi, la gente può finire per discutere
ed entrare in accese dispute, e così non c'è proprio più nessun albero
della Bodhi.

Stiamo parlando di \emph{paramatthadhamma},\footnote{\emph{Paramatthadhamma.}
  ``Verità o Realtà Ultima'', il Dhamma o i dhamma descritti in
  termini di significato ultimo, non di mera convenzione.} il livello
della Verità Ultima. In questo caso possiamo anche tentare di stare
sotto l'albero della Bodhi. Molto bene, allora saremo il Buddha. Non si
tratta di una cosa su cui discutere. Quando qualcuno dice che il Buddha
stava praticando un certo tipo di meditazione sotto l'albero della Bodhi
e qualcun altro risponde -- «~No, non è vero~» -- non c'è bisogno di
farsi coinvolgere. Stiamo mirando al \emph{paramatthadhamma}, il che
significa dimorare in totale consapevolezza. Questa verità ultima
pervade ogni cosa. Non importa se il Buddha stesse sedendo sotto
l'albero della Bodhi o svolgendo altre attività in ulteriori posture.
Questa è solo analisi intellettuale sviluppata dalla gente. Una persona
ha un modo di vedere, un'altra persona ha un'idea differente. Non
dobbiamo farci coinvolgere in dispute al riguardo.

Il Buddha entrò nel Nibbāna, che significa estinzione senza
residuo. Finito. Dire ``finito'' viene dalla conoscenza, la conoscenza
del modo in cui le cose realmente sono. Le cose giungono alla fine così,
e questo è \emph{paramatthadhamma}. Ci sono spiegazioni in accordo con
il livello della convenzione e altre in accordo con il livello della
Liberazione. Sono vere entrambe, ma le loro verità sono differenti. Ad
esempio, noi diciamo che tu sei una persona. Il Buddha dice: «~Non è
così, non esiste una persona.~» Dobbiamo catalogare i vari modi di
parlare e di spiegare in termini di convenzione e di Liberazione.

Possiamo spiegarlo in questo modo. In precedenza eri un bambino. Ora sei
cresciuto. Sei un'altra persona o la stessa di prima? Se sei la stessa
vecchia persona di prima, come hai fatto a diventare un adulto? Se sei
una persona nuova, da dove vieni fuori? Parlare di una vecchia e di una
nuova persona, però, non tocca il nocciolo della questione. Questo
illustra i limiti del linguaggio e della comprensione convenzionali. Se
c'è il ``grande'', allora c'è il ``piccolo''. Se c'è il piccolo, c'è il
grande. Possiamo parlare di piccolo e grande, giovane e vecchio, ma in
realtà queste cose in senso assoluto non ci sono. Non potete dire
davvero che qualcuno o qualcosa è grande. Il saggio non accetta queste
indicazioni come reali, ma quando la gente comune sente queste cose --
che ``grande'' non è davvero la verità e che ``piccolo'' non è davvero
la verità -- si confonde perché è attaccata ai concetti di grande e
piccolo.

Piantiamo un alberello e lo osserviamo mentre cresce. Dopo un anno è
alto un metro. Dopo un altro anno è alto due metri. Si tratta dello
stesso albero o di un albero differente? Se è lo stesso albero, com'è
che è diventato più grande? Se è un albero differente, com'è che è
cresciuto dall'albero piccolo? Dal punto di vista di chi è illuminato al
Dhamma e vede correttamente, non c'è nessun albero nuovo o vecchio,
nessun albero grande o piccolo. Uno guarda l'albero e pensa che sia
alto. Un altro dirà che non è alto. Ma non c'è alcun ``alto'' che esista
davvero in modo indipendente. Non puoi dire che qualcuno è grande e che
qualcun altro è piccolo, che uno è cresciuto e che un altro è giovane.
Le cose finiscono qui, e con esse i problemi. Se non ci annodiamo a
queste distinzioni convenzionali, non avremo dubbi a proposito della
pratica.

Ho sentito che alcuni venerano le loro divinità sacrificando animali.
Uccidono anatre, galline e mucche e le offrono ai loro dèi, pensando che
a loro farà piacere. Questa è errata comprensione. Pensano di ``fare
meriti'', ma è l'esatto opposto. In realtà stanno accumulando un
kamma molto negativo. Chi osservi tutto questo davvero con
attenzione, non la penserà in questo modo. Lo avete notato? Temo che la
gente in Thailandia stia per diventare così. Non investigano davvero.

\emph{D.:} È questo che significa \emph{vīmaṃsā}?\footnote{\emph{Vīmaṃsā.}
  Investigazione, indagine.}

\emph{R.:} Significa comprendere causa ed effetto.

\emph{D.:} Gli insegnamenti parlano di \emph{chanda}, aspirazione,
\emph{viriya}, sforzo, e \emph{citta}, mente; insieme a \emph{vīmaṃsā},
sono i quattro \emph{iddhipāda},\footnote{\emph{Iddhipāda.} Per un
  elenco degli \emph{iddhipāda} si veda il \emph{Glossario}, p. \pageref{glossary-iddhipada}.} le ``basi
per la realizzazione''.

\emph{R.:} Quando c'è appagamento, si è appagati per qualcosa di corretto? Lo
sforzo è corretto? \emph{Vīmaṃsā} deve essere presente con questi altri
fattori.

\emph{D.:} Sono cose diverse \emph{citta} e \emph{vīmaṃsā}?

\emph{R.:} \emph{Vīmaṃsā} è investigazione. Significa abilità o saggezza. È un
fattore della mente. Si può dire che \emph{chanda} è mente,
\emph{viriya} è mente, \emph{citta} è mente, \emph{vīmaṃsā} è mente.
Sono tutti aspetti della mente, possono essere tutti sintetizzati in
``mente'', ma qui sono distinti per sottolineare differenti fattori
della mente. Se c'è appagamento, potremmo non sapere se è giusto o
sbagliato. Se c'è sforzo, non sappiamo se è giusto o sbagliato. Quel che
chiamiamo mente è la vera mente? Ci deve essere \emph{vīmaṃsā} per
discernere queste cose. Quando investighiamo gli altri fattori con
saggio discernimento, la nostra pratica diventa gradualmente corretta e
comprendiamo il Dhamma.

Però, il Dhamma non arreca molto beneficio se non pratichiamo la
meditazione. Non sapremo di cosa si tratta in verità. Questi fattori
sono sempre presenti nella mente dei veri praticanti. Anche se vanno
fuori strada, se ne accorgono e sono in grado di correggersi. È per
questa ragione che il Sentiero della loro pratica è continuo.

La gente può guardarti e ritenere che il tuo modo di vivere e il tuo
interesse per il Dhamma non abbiano senso. Alcuni potrebbero dire che se
vuoi praticare il Dhamma dovresti farti ordinare monaco. In realtà, il
punto cruciale non è l'ordinazione monastica. È come si pratica. Come è
stato detto, ognuno dovrebbe essere il testimone di se stesso. Non
prendere gli altri come tuoi testimoni. Questo significa imparare ad
avere fiducia in se stessi. Allora non vi è perdita alcuna. La gente può
pensare che tu sia matto, ma non preoccupartene. Non sanno nulla del
Dhamma.

Le parole degli altri non possono valutare la tua pratica. E il Dhamma
non si comprende per quello che dicono gli altri. Intendo il vero
Dhamma. Gli insegnamenti che gli altri possono darti servono a indicare
il Sentiero, ma non è vera conoscenza. Quando le persone incontrano il
Dhamma, lo comprendono distintamente dentro se stessi. Per questo il
Buddha disse: «~Il \emph{Tathāgata} indica solamente la via.~» Quando
qualcuno viene ordinato, gli dico: «~La nostra responsabilità arriva
solo fin qui: gli \emph{ācariya}\footnote{\emph{Ācariya.} Insegnante,
  mentore, maestro.} hanno recitato i loro canti. Ti ho dato la
\emph{pabbajjā}\footnote{\emph{Pabbajjā.} Nei testi
  buddhisti in pāli indica il passaggio dalla vita laica a quella di
  monaco privo di dimora, e può essere reso con l'``abbandono'' della
  vita laica.} e i voti di ordinazione. Noi abbiamo fatto il nostro
lavoro. Il resto, praticare correttamente, dipende da~te.~»

Gli insegnamenti possono essere i più profondi, ma chi ascolta potrebbe
non capire. Non ti preoccupare. Non farti sconcertare dalla profondità o
dalla mancanza di profondità. Devi solo praticare con tutto il cuore, e
arriverai alla vera comprensione; la pratica ti porterà nello stesso
posto di cui parlano gli insegnamenti. Non fare affidamento sulle
percezioni delle persone ordinarie. Hai letto la storia dei ciechi e
dell'elefante? Ben illustra tutto questo. Supponiamo che ci siano un
elefante e un gruppo di ciechi che cercano di descriverlo. Uno tocca la
zampa e dice che è come una colonna. Un altro tocca l'orecchio e dice
che è come un ventaglio. «~No, non un ventaglio, è come una scopa~»,
dice un altro toccando la coda. Un altro ancora tocca la spalla, e dice
una cosa diversa da quello che hanno detto gli~altri.

È così. Non c'è soluzione, non c'è fine. Ogni cieco tocca una parte
dell'elefante e ha un'idea del tutto diversa di quello che è. Si tratta
però di un solo elefante, dello stesso elefante. Così è anche per la
pratica. Con un po' di comprensione o di esperienza, si hanno idee
limitate. Puoi andare da un maestro all'altro alla ricerca di
spiegazioni e di istruzioni, cercando di capire se stiano insegnando
correttamente o no, e come i loro insegnamenti siano in rapporto uno con
l'altro. Certi monaci vagano in continuazione con le loro ciotole per la
questua e i loro \emph{glot},\footnote{\emph{Glot} (in thailandese \thai{กลค}).
  Ombrello con una zanzariera tutt'intorno all'estremità, utilizzato sia
  per la meditazione sia come riparo dai monaci che intraprendono i
  \emph{dhutaṅga}; viene appeso ai rami degli alberi così da potercisi
  sedere sotto, al riparo dagli insetti; è un termine diverso rispetto a
  quello utilizzato per l'ombrello dei laici, \emph{rom} (in thailandese
  \thai{ร่ม}).} imparando da vari insegnanti. Cercano di giudicare e valutare,
così quando siedono per meditare sono costantemente confusi su cosa sia
giusto e cosa sia sbagliato. «~Questo insegnante ha detto questo, ma
quell'insegnante ha detto quello. Uno insegna in questo modo, ma i
metodi di quell'altro sono diversi. Non sembrano concordare.~» Ciò può
indurre molti dubbi. Potresti sentir dire che alcuni insegnanti sono
davvero bravi e così vai a ricevere insegnamenti da
\emph{ajahn}\footnote{\emph{Ajahn} (in thailandese, \thai{อาจารย์}).
  Il termine deriva da \emph{ācariya}, in pāli, letteralmente
  ``insegnante''; spesso viene utilizzato per un monaco o per una monaca
  con più di dieci anni di vita monastica.} thailandesi, da maestri zen
e da altri. Mi sembra che di insegnamenti ne hai probabilmente avuti
abbastanza, ma la tendenza è quella di volerne ascoltare sempre di più,
per paragonarli e ottenere l'unico risultato di essere in dubbio. Così,
ogni insegnante farà crescere ulteriormente la tua confusione.

C'è una storia su un asceta errante che, ai tempi del Buddha, si trovava
in questa stessa situazione. Si recò da un maestro dopo l'altro e
ascoltò le loro differenti spiegazioni e i loro metodi. Cercava di
imparare la meditazione, ma le sue perplessità aumentavano sempre più.
Infine, i suoi viaggi lo condussero dal maestro Gotama, ed egli
descrisse la sua situazione al Buddha.

«~Fare come hai fatto finora non porrà fine ai dubbi e alla
confusione~», gli disse il Buddha. «~Ora lascia andare il passato.
Qualsiasi cosa tu possa aver o non aver fatto, sia essa giusta o
sbagliata, lascia andare. Il futuro non è ancora arrivato. Non farti
domande, non chiederti come le cose potrebbero essere. Lascia andare
tutti questi pensieri inquieti, sono solo pensieri. Quando lasci andare
passato e futuro, guarda il presente. Così conoscerai il Dhamma.
Potresti conoscere le parole pronunciate da vari insegnanti, ma non
conosci ancora la tua mente. Il momento presente è vuoto. Osserva solo
il sorgere e il cessare dei \emph{saṅkhāra}. Vedi che sono impermanenti,
insoddisfacenti e privi di sé. Comprenderai con chiarezza che il passato
non c'è più e che il futuro non è ancora arrivato. Contemplando il
presente, comprenderai che è il risultato del passato. I risultati delle
azioni passate si vedono nel presente. Il futuro non è ancora giunto.
Qualsiasi cosa capiterà nel futuro sorgerà e passerà nel futuro. Non c'è
ragione di preoccuparsene ora, non è ancora successo. Perciò, contempla
nel presente. Il presente è la causa del futuro. Se vuoi un buon futuro,
crea il bene nel presente, aumentando la consapevolezza di ciò che fai
ora. Il futuro ne sarà il risultato. Il passato è la causa e il futuro è
il risultato del presente. Conoscendo il presente, si conosce il passato
e il futuro. Così si lascia andare passato e futuro, sapendo che sono
riuniti nel momento presente.~»

Dopo aver capito, quell'asceta errante decise di praticare nel modo
consigliato dal Buddha, lasciando andare tutto. Vedendo con sempre
maggior chiarezza, ottenne molti tipi di conoscenza, e con la saggezza
vide l'ordine naturale delle cose. I suoi dubbi finirono. Lasciò andare
il passato e il futuro, e tutto gli apparve nel presente. Ciò era
\emph{eko Dhamma}, il Dhamma unificato. Non gli fu più necessario
portare la ciotola per la questua su per le montagne e nelle foreste
alla ricerca della comprensione. Se si recava da qualche parte, vi
andava in modo naturale, non perché desiderasse qualcosa. Se restava,
restava in modo naturale, non perché lo desiderava. Praticando in questo
modo, si liberò dal dubbio. Non c'era nulla da aggiungere alla sua
pratica, e nulla da togliere. Dimorava nella pace, senza ansie per il
passato o per il futuro. Questa era la via insegnata dal Buddha. Non è
una storia su una cosa che avvenne molto tempo fa. Anche oggi, se
pratichiamo correttamente, possiamo ottenere la Realizzazione. Possiamo
conoscere il passato e il futuro perché sono riuniti in quest'unico
punto, il momento presente. Se guardiamo il passato, non conosceremo. Se
guardiamo il futuro, non conosceremo. Non è lì che si trova la Verità.
Essa esiste qui, nel presente.

Il Buddha disse: «~Sono diventato illuminato per mezzo dei miei stessi
sforzi, senza alcun maestro.~» Hai letto questa storia? Un asceta
errante di un'altra setta gli chiese: «~Chi è il tuo maestro?~» Il
Buddha rispose: «~Non ho alcun maestro, ho ottenuto l'Illuminazione da
solo.~» L'asceta scosse la testa e andò via. Pensò che il Buddha stesse
inventando tutto e non prestò attenzione alle sue parole. Pensò che non
era possibile raggiungere qualcosa senza un maestro e una guida.

È così. Studi con un maestro spirituale e lui ti dice di rinunciare ad
avidità e rabbia. Ti dice che sono dannose e che bisogna vincerle.
Allora puoi praticare e farlo. La vittoria sull'avidità e sulla rabbia,
però, non arriva perché te lo ha insegnato. Devi effettivamente
praticare e farlo. Attraverso la pratica pervieni a realizzare qualcosa
da te stesso. Vedi l'avidità nella tua mente e ci rinunci. Vedi la
rabbia nella tua mente e ci rinunci. L'insegnante non può vincerle al
posto tuo. Ti dice di vincerle, ma questo non avviene solo perché te lo
dice. Sei tu che pratichi e giungi alla Realizzazione. Comprendi queste
cose da te stesso. È come se il Buddha ti prendesse e ti portasse
all'inizio del Sentiero, dicendoti: «~Questo è il sentiero,
percorrilo.~» Non ti aiuta a camminare. Devi farlo da te. Quando
percorri il Sentiero e pratichi il Dhamma, incontri il vero Dhamma, che
va oltre tutto quello che possono spiegarti. Ognuno s'illumina da se
stesso, comprendendo il passato, il futuro e il presente, comprendendo
causa ed effetto. Allora i dubbi sono finiti.

Stiamo parlando di abbandono e di sviluppo, di rinunciare e di
coltivare. Però, quando il frutto della pratica è realizzato, non c'è
nulla da aggiungere e nulla da togliere. Il Buddha insegnò che è qui che
vogliamo arrivare, ma la gente non vuole fermarsi qui. I loro dubbi e
attaccamenti li tengono in movimento, continuano a confonderli e
impediscono loro di fermarsi. Così, quando uno è arrivato ma gli altri
sono altrove, costoro non saranno in grado di attribuire alcun senso a
ciò che egli potrebbe dire in proposito. Possono avere una qualche
comprensione intellettuale delle parole, ma questa non è reale
comprensione o conoscenza della Verità.

Di solito, quando ci esprimiamo sulla pratica parliamo di entrare e
uscire, di accrescere ciò che è positivo e di rimuovere ciò che è
negativo. Il risultato finale, però, è farla finita con tutto questo.
C'è la \emph{sekha-puggala},\footnote{\emph{Sekha.} Chi si sottopone
  all'addestramento spirituale; il termine si riferisce ai sette
  \emph{ariya-sāvaka} o \emph{ariya-puggala} che non sono ancora diventati
  \emph{arahant}.} la persona che necessita di addestrarsi, e c'è
l'\emph{asekha-puggala},\footnote{\emph{Asekha.} Una persona
  (\emph{puggala}) oltre l'addestramento, ossia un \emph{arahant}.} la
persona che non ha più bisogno di alcun addestramento. Questo è parlare
della mente, e quando la mente ha raggiunto questo livello di completa
realizzazione non c'è più niente da praticare. Perché? Perché questa
persona non ha più bisogno di utilizzare nessuna convenzione
dell'insegnamento e della pratica. Ha abbandonato le contaminazioni. La
\emph{sekha-puggala} deve addestrarsi nei vari passi del Sentiero,
dall'inizio fino al livello più alto. Quando uno ha completato tutto, si
parla di \emph{asekha}, per significare che non ha più necessità di
addestrarsi perché tutto è finito. Le cose in cui addestrarsi sono
finite. I dubbi sono finiti. Non ci sono qualità da sviluppare. Non ci
sono contaminazioni da rimuovere. Si dimora nella pace. Sia il bene sia
il male non hanno più effetti. Si è incrollabili indipendentemente da
chi si incontra. Questo è parlare della mente vuota. Ora ti sentirai
davvero confuso.

Non riesci proprio a capirlo. «~Se la mia mente è vuota, come posso
camminare?~» Proprio perché la mente è vuota. «~Se la mente è vuota,
come posso mangiare? Quando la mia mente sarà vuota, avrò il desiderio
di mangiare?~» Non è di grande beneficio parlare di vacuità in questo
modo. Se le persone non sono addestrate in modo opportuno, non sono in
grado di comprendere. Chi utilizza questi termini ha cercato dei modi
per darci una qualche impressione che possa aiutarci a comprendere la
Verità. Ad esempio, il Buddha disse che in realtà questi
\emph{saṅkhāra}, questi cumuli che portiamo con noi fin dalla nostra
nascita, non sono noi stessi e non ci appartengono. Perché affermò una
cosa del genere? Non c'è altro modo per formulare la Verità. Parlò in
questo modo per chi ha discernimento, affinché si possa crescere in
saggezza. È però una cosa da contemplare con accuratezza.

Quando alcuni sentono le parole ``niente mi appartiene'', traggono la
conclusione che dovrebbero liberarsi di tutti i loro possessi. Con una
comprensione solo superficiale, la gente inizierà ad argomentare su cosa
questo significhi e su come possa essere applicato. ``Questo non è il
mio sé'' non significa che dovreste porre fine alla vostra vita o
gettare quel che possedete. Significa che dovreste abbandonare
l'attaccamento. C'è il livello della realtà convenzionale e il livello
della realtà ultima: supposizione e Liberazione. Al livello della
convenzione, ci sono il signor A, la signora B, il signor L, la signora
N, e così via. Utilizziamo queste supposizioni in quanto utili per
comunicare nel mondo e per farlo funzionare. Il Buddha non insegnò che
non dovremmo utilizzare queste cose, ma che non dovremmo attaccarci a
esse. Dovremmo capire che sono vuote. È difficile parlarne. Dobbiamo
fare affidamento sulla pratica e acquisire comprensione per mezzo di
essa. Se vuoi ottenere conoscenza e comprensione studiando e chiedendo
agli altri, non comprenderai la Verità davvero. È un qualcosa che devi
vedere e conoscere da te stesso mediante la pratica. Volgiti
all'interno, per conoscere dentro te stesso. Non guardare sempre
all'esterno. Quando parliamo di pratica la gente diventa polemica. La
loro mente è pronta ad argomentare, perché ha imparato questo o
quell'approccio alla pratica ed è unilateralmente attaccata a ciò che ha
imparato. La verità non è stata compresa attraverso la pratica.

L'altro giorno hai osservato quei thailandesi che abbiamo incontrato?
Facevano domande irrilevanti: «~Perché non hai mangiato dalla tua
ciotola?~» Ho potuto notare che erano lontani dal Dhamma. Avevano avuto
un'educazione moderna e non ho voluto dire molto. Ho lasciato che fosse
il monaco statunitense a parlare con loro. Forse desideravano
ascoltarlo. Oggigiorno i thailandesi non sono molto interessati al
Dhamma e non lo comprendono. Perché lo dico? Se qualcuno non studia
qualcosa, è ignorante in materia. Loro hanno studiato altre cose, ma
sono ignoranti sul Dhamma. Ammetto di essere ignorante su quello che
loro hanno imparato. Il monaco occidentale ha studiato il Dhamma, perciò
può dire loro qualcosa in proposito.

Oggigiorno tra i thailandesi si interessano sempre meno all'ordinazione
monastica, allo studio e alla pratica del Dhamma. Non so quale sia la
ragione, se è perché sono occupati con il lavoro o perché la loro
nazione si sta sviluppando materialmente. In passato, quando qualcuno
aveva ricevuto l'ordinazione, restava almeno qualche anno, per quattro o
cinque Stagioni delle Piogge.\footnote{Vale a dire per quattro o cinque
  Ritiri delle Piogge.} Ora si resta una
settimana o due. Alcuni la ricevono al mattino e lasciano l'abito alla
sera. Questa è la direzione in cui si va. Un conoscente mi ha detto:
«~Se tutti ricevessero l'ordinazione monastica come piace a te, almeno
per qualche Stagione delle Piogge, nel mondo non ci sarebbe progresso.
Le famiglie non crescerebbero. Nessuno costruirebbe niente.~» Gli ho
risposto così: «~Il tuo pensiero è il pensiero di un lombrico. Un
lombrico vive nella terra. Mangia la terra, la terra è il suo cibo.
Mangiare e poi ancora mangiare. Inizia a preoccuparsi che sarà a corto
di sporcizia da mangiare anche se è circondato dalla sporcizia. L'intera
terra copre la sua testa, ma si preoccupa che sarà a corto di
sporcizia.~» Così pensa un lombrico. Le persone si preoccupano che se il
mondo non progredirà, giungerà alla fine. È il modo di vedere di un
lombrico. Non sono lombrichi, ma pensano come lombrichi. È l'errata
comprensione del regno animale. Sono davvero ignoranti.

C'è una storia che racconto spesso, su una tartaruga e un serpente. La
foresta era in fiamme e stavano cercando di fuggire. La tartaruga si
muoveva pesantemente e vide il serpente che strisciava. Ne ebbe pietà.
Perché? Il serpente non aveva zampe, e la tartaruga pensò che non
sarebbe stato in grado di scappare dal fuoco. Voleva aiutarlo. Quando
però le fiamme continuarono a diffondersi, il serpente fuggì con
facilità, mentre la tartaruga non riuscì a fare altrettanto anche se
aveva quattro zampe, e morì. Quella era l'ignoranza della tartaruga.
Pensava che se hai delle zampe, puoi muoverti. Se non hai le zampe, non
puoi andare da nessuna parte. Così si preoccupò del serpente. Pensò che
sarebbe morto perché non aveva le zampe. Ma il serpente non era
preoccupato, sapeva che poteva scappare con facilità dal pericolo.
Questo è un modo per parlare con chi ha le idee confuse. Hanno pietà di
te perché non sei come loro e non hai il loro modo di vedere e le loro
conoscenze. Chi è ignorante, allora? Io sono ignorante a mio modo, ci
sono cose che non so, sono ignorante in quell'ambito.

Cambiare situazione può indurre tranquillità. Io però non capivo quanto
fossi folle, quanto mi sbagliassi. Ogni volta che qualcosa disturbava la
mia mente, cercavo di allontanarmi, di scappare. Quel che stavo facendo
era scappare dalla pace. Fuggivo in continuazione dalla pace. Non volevo
vedere questo né sapere qualcosa in relazione a quell'altro. Non volevo
pensare o fare esperienza di varie cose. Non capivo che era una
contaminazione. Pensavo di aver solo bisogno di togliermi di mezzo e di
allontanarmi da persone e situazioni, in modo da non imbattermi in cose
che mi disturbavano o sentire parole che non mi piacevano. Più lontano
potevo andare meglio era. Passarono molti anni, e il corso naturale
degli eventi mi costrinse a cambiare prospettiva. Avendo ricevuto
l'ordinazione monastica da parecchio tempo, finii per avere un numero
sempre maggiore di discepoli, la gente che mi cercava aumentò. Vivere e
praticare nella foresta spingeva la gente a venire a porgermi omaggio.
Il numero dei discepoli cresceva e fui costretto ad affrontare la cosa.
Non potei più fuggire. I miei orecchi dovevano sentire suoni, i miei
occhi vedere. E fu allora, come \emph{ajahn}, che cominciai ad acquisire
più conoscenza. Tutto questo mi portò ad avere molta saggezza e a
lasciar andare. Mi rese molto più abile di prima.

Quando arrivava della sofferenza, andava bene, non ne aggiungevo altra
cercando di sfuggirla. In precedenza nella meditazione avevo desiderato
solo la tranquillità. Pensavo che l'ambiente esterno fosse utile solo
nella misura in cui potesse aiutarmi a raggiungere la tranquillità. Non
mi era venuto in mente che la Retta Visione sarebbe stata la causa per
realizzare la tranquillità. Dico spesso che ci sono due tipi di
tranquillità. Il saggio distingue la pace ottenuta per mezzo della
saggezza e la pace ottenuta per mezzo di \emph{samatha}. Nella pace
ottenuta per mezzo di \emph{samatha}, gli occhi devono essere lontani da
ciò che si vede, gli orecchi dai suoni, il naso dagli odori e così via.
Allora, senza sentire, senza sapere e così via, si può essere
tranquilli. Questo genere di pacificazione è buona, ma a suo modo. Ha
valore? Si, lo ha, ma non un valore supremo. Ha vita breve. Non ha un
fondamento affidabile. Quando i sensi incontrano oggetti spiacevoli, la
mente cambia perché non vuole che quelle cose ci siano. In questo modo
la mente deve sempre combattere con quegli oggetti, e non nasce alcuna
saggezza finché si pensa di non essere in pace a causa di quei fattori
esterni.

Se d'altra parte si decide di non fuggire e di guardare direttamente le
cose, si arriva a comprendere che la mancanza di tranquillità non è
dovuta a oggetti o situazioni esterne, bensì solo all'errata
comprensione. Lo insegno spesso ai miei discepoli. Dico così: «~Quando
nella vostra meditazione vi dedicate intensamente a cercare la
tranquillità, potete riuscire a trovare il posto più sereno e isolato,
ove non incontrate oggetti visivi o suoni, ove non succede nulla che vi
disturbi. Lì la mente può acquietarsi e calmarsi, perché non c'è nulla
che a provocarla. Quando fate questa esperienza, esaminate la vostra
mente per vedere quanta forza ha. Andate via da quel posto, e iniziate a
sperimentare i contatti sensoriali, e notate se diventate contenti o
scontenti, lieti o scoraggiati, e come la mente si turba. Allora
capirete che quel tipo di tranquillità non è genuino.~» Qualsiasi cosa
avvenga nell'ambito della tua esperienza è semplicemente quello che è.
Quando una cosa ci piace, decidiamo che è bene, e quando non ci piace
diciamo che non è bene. È solo la mente discriminante a dare un senso
agli oggetti esterni. Se lo comprendiamo, abbiamo una base per
investigare queste cose e vederle come veramente sono.

Quando durante la meditazione c'è tranquillità, un sacco di pensieri non
sono necessari. La sensibilità che nasce dalla mente tranquilla ha una
certa qual capacità di conoscere. Non è pensare, è \emph{dhammavicaya},
il fattore di investigazione del Dhamma. Questo genere di tranquillità
non è disturbato dall'esperienza e dal contatto sensoriale. Allora, ecco
la domanda: se c'è tranquillità, perché succede ancora qualcosa? Si
tratta di un qualcosa, però, che succede all'interno della
tranquillità, non in modo ordinario, non c'è quella solita afflizione
che ci fa vedere in quel qualcosa più di quanto in realtà ci sia. Quando
succede qualcosa all'interno della tranquillità, la mente lo sa con
estrema chiarezza. La saggezza è nata, e la mente contempla con sempre
maggior chiarezza. Vediamo il modo in cui le cose avvengono realmente.
Quando conosciamo la verità delle cose, la tranquillità diventa
onnicomprensiva. Quando gli occhi vedono delle forme o gli orecchi
sentono suoni, li riconosciamo per quel che sono. In questo tipo di
tranquillità, quando gli occhi vedono delle forme, la mente è serena.
Quando gli orecchi sentono suoni, la mente è serena. Non vacilla.
Qualsiasi cosa sperimenti, non è scossa.

Da dove viene questo tipo di tranquillità? Arriva da quell'altro tipo di
tranquillità, da quel \emph{samatha} privo di conoscenza. Questa è la
causa che le consente di arrivare. Si insegna che la saggezza proviene
dalla tranquillità. La conoscenza proviene dalla non conoscenza. La
mente giunge a conoscere da quello stato di non conoscenza, così,
imparando a contemplare. Ci saranno sia tranquillità sia saggezza.
Allora dovunque siamo, qualsiasi cosa facciamo, vediamo la verità delle
cose. Sappiamo che il sorgere e il cessare dell'esperienza nella mente è
solo così. A questo punto non c'è nient'altro da fare, niente da
correggere o da risolvere. Non ci sono più congetture. Non c'è più alcun
posto in cui andare, non c'è più fuga. Possiamo fuggire solo per mezzo
della saggezza, per mezzo della conoscenza delle cose così come sono,
trascendendole.

In passato, appena mi stabilii al Wat Pah Pong e la gente iniziò a
venire a trovarmi, alcuni discepoli dissero: «~Luang Por socializza
sempre con la gente. Questo non è più il posto giusto in cui stare.~»
Non era che io fossi andato a cercarla, quella gente. Avevamo fondato un
monastero e venivano a porgere omaggio al nostro modo di vita. Bene, non
potevo negare quel che stavano dicendo, ma in realtà stavo acquisendo
molta saggezza e pervenendo alla conoscenza di molte cose. I discepoli
però non lo sapevano. Potevano solo guardarmi e pensare che la mia
pratica stesse degenerando: stavano arrivando così tanta gente e così
tanti fastidi! Non avevo modo di convincerli del contrario, ma quando il
tempo passò, superai vari ostacoli e finalmente giunsi a capire che la
vera tranquillità nasce dalla Retta Visione. Se non abbiamo la Retta
Visione, non conta dove ci troviamo, non saremo in pace e la saggezza
non sorgerà.

La gente sta cercando di praticare qui, in Occidente. Non voglio
criticare nessuno, ma da quel che vedo, \emph{sīla}, la moralità, non è
ben sviluppata. Certo, si tratta di una convenzione. Si può cominciare a
praticare prima il \emph{samādhi}. È come camminare e imbattersi in un
lungo pezzo di legno. Lo si può prendere da un'estremità o dall'altra,
ma è lo stesso, unico pezzo di legno e, da qualsiasi estremità lo si
prenda, si può riuscire a muoverlo. Quando c'è una certa calma che
deriva dalla pratica del \emph{samādhi}, la mente può vedere le cose con
chiarezza. La saggezza cresce e vede quanto certi comportamenti possano
essere dannosi, e si diventerà moderati e cauti. Puoi muovere il pezzo
di legno da entrambe le estremità, ma il punto fondamentale è essere
fermamente determinati nella pratica. Se inizi con \emph{sīla}, questa
moderazione porterà alla calma. Si tratta del \emph{samādhi} che diviene
una causa per la saggezza. Quando c'è saggezza, ciò aiuta a sviluppare
ulteriormente il \emph{samādhi}, e il \emph{samādhi} continua ad
affinare \emph{sīla}. In realtà sono sinonimi, si sviluppano assieme.
Alla fine il risultato è che sono una sola cosa, sono inseparabili.

Non possiamo distinguere il \emph{samādhi} e classificarlo
separatamente. All'inizio, però, dobbiamo distinguere: c'è il livello
della convenzione e il livello della Liberazione. Al livello della
Liberazione non ci attacchiamo al bene e al male. Utilizzando la
convenzione, distinguiamo bene e male, e i vari aspetti della pratica. È
necessario farlo, ma non siamo ancora al livello supremo. Se
comprendiamo l'uso della convenzione, possiamo giungere a comprendere la
Liberazione. Allora possiamo capire i modi in cui i vari termini sono
utilizzati per condurre le persone verso la stessa cosa.

In quei giorni ho imparato a relazionarmi alla gente e a qualsiasi tipo
di situazione. Venendo a contatto con queste cose, ho dovuto rendere
stabile la mia mente. Facendo affidamento sulla saggezza sono stato in
grado di vedere con chiarezza e di dimorare nelle varie situazioni senza
essere influenzato da tutto ciò che incontravo. Qualsiasi cosa gli altri
potessero dire, non ne ero disturbato perché la mia certezza era
stabile. Quanti diverranno insegnanti necessitano di questa fermezza in
ciò che stanno facendo per non essere influenzati da quel che la gente
dice. C'è bisogno di una certa saggezza e, per quanto sia grande la
saggezza che ognuno ha, essa può sempre aumentare. Facciamo il punto su
tutti i nostri passati modi di essere, sul valore che hanno rivelato di
avere e continuiamo a fare pulizia.

Dovete davvero rendere salda la vostra mente. A volte il corpo o la
mente non si sentono a proprio agio. Succede quando viviamo insieme, è
una cosa naturale. Talora, ad esempio, dobbiamo misurarci con le
malattie. Mi è capitato parecchie volte. Come si dovrebbe affrontare
tutto questo? Bene, tutti vogliono vivere comodamente, avere buon cibo e
riposare molto. Ma non è sempre possibile. Non possiamo solo indulgere
ai nostri desideri.

In questo mondo, però, alcuni benefici li creiamo tramite i nostri
sforzi virtuosi. Generiamo benefici per noi stessi e per gli altri, per
questa vita e per la prossima. Questo è il risultato di quando si rende
la mente serena. Altrettanto vale per la mia visita qui in Inghilterra,
e negli Stati Uniti. Si tratta di una breve visita, ma cercherò di
aiutarvi come meglio posso e di offrire il mio insegnamento e la mia
guida. Qui ci sono sia Ajahn sia studenti, cercherò di aiutarli. Anche
se fino a ora non sono venuti dei monaci a vivere qui, tutto sommato va
bene. Questa visita può preparare la gente ad avere dei monaci qui. Se
verranno troppo presto, sarà difficile. La gente può acquisire poco a
poco familiarità con la pratica e con i modi di vivere del
\emph{bhikkhusaṅgha}.\footnote{\emph{Bhikkhusaṅgha.} La comunità dei
  monaci buddhisti.} Allora qui il \emph{sāsana}\footnote{\emph{Sāsana.}
  Insegnamento, dispensazione, dottrina ed eredità del Buddha; la scuola
  spirituale buddhista.} potrà fiorire. Per ora dovete cercare di
prendervi cura della vostra mente, cominciate a farlo.

