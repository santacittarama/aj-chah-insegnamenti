\chapter{Il Sentiero verso la pace}

Oggi il mio insegnamento è specificamente rivolto a voi monaci e novizi.
Perciò, per favore, fate in modo che il vostro cuore e la vostra mente
siano determinati ad ascoltare. Per noi non c'è nient'altro di cui
parlare se non della pratica del Dhamma-Vinaya.

Ognuno di voi dovrebbe comprendere con chiarezza che, siccome ha
ricevuto l'ordinazione da monaco o da novizio buddhista, dovrebbe
comportarsi in modo appropriato. Tutti noi abbiamo avuto esperienza
della vita laica, che è caratterizzata dalla confusione e dalla mancanza
di una formale pratica di Dhamma. Ora, siccome abbiamo assunto la forma
di un \emph{samaṇa} buddhista, nella nostra mente devono avvenire alcuni
cambiamenti essenziali per differenziarci dal modo di pensare dei laici.
Dobbiamo cercare di rendere il nostro linguaggio e le nostre azioni --
mangiare e bere, muoversi, andare e venire -- appropriati a chi ha
ricevuto l'ordinazione come ricercatore spirituale, indicato dal Buddha
come \emph{samaṇa}. Egli intendeva una persona calma e contenuta. In
precedenza, da laici, non comprendevamo cosa significasse essere un
\emph{samaṇa}, avere una sensazione di serenità e di contenimento.
Davamo piena licenza al nostro corpo e alla nostra mente di divertirsi e
giocare sotto l'influsso della brama e delle contaminazioni. Quando
sperimentavamo \emph{ārammaṇa} piacevoli, questi ci mettevano di buon
umore, mentre oggetti mentali spiacevoli ci mettevano di cattivo umore.
Così è, quando si è in potere degli oggetti mentali. Il Buddha disse che
chi è ancora sotto l'influsso degli oggetti mentali non si prende cura
di se stesso. È privo di un rifugio, di un posto nel quale dimorare
davvero, e perciò consente alla propria mente di seguire stati mentali
che indulgono ai sensi, vanno alla ricerca del piacere e restano
prigionieri della sofferenza, della pena, del lamento, dell'afflizione e
della disperazione. Non sanno come e quando fermarsi, e riflettere sulla
loro esperienza.

Nel buddhismo, quando riceviamo l'ordinazione e assumiamo il modo di
vivere di un \emph{samaṇa}, dobbiamo adattare il nostro aspetto alla
forma esteriore del \emph{samaṇa.} Ci rasiamo il capo e indossiamo
l'abito color ocra dei \emph{bhikkhu}, l'insegna degli Esseri Nobili, il
Buddha e gli \emph{arahant}. Siamo in debito con il Buddha per i
salutari fondamenti da Lui instaurati e tramandati, i quali ci
consentono di vivere come monaci con gli adeguati sostegni. I nostri
alloggi sono stati costruiti e offerti quale risultato delle benefiche
attività di coloro che nutrono fiducia nel Buddha e nei suoi
insegnamenti. Non abbiamo bisogno di prepararci il cibo perché traiamo
vantaggio dalle radici trasmesse dal Buddha. Allo stesso modo, abbiamo
ereditato le medicine, l'abito e tutti gli altri generi di prima
necessità che usiamo fin dai tempi del Buddha. Quando si riceve
l'ordinazione, a un livello convenzionale si è chiamati monaci e ci
viene dato il titolo di ``venerabile''.\footnote{Venerabile, in
  thailandese \emph{prah} (\thai{พระ}).} Assumere solo l'apparenza esteriore di
monaci, però, non ci rende davvero venerabili. Essere monaci a un
livello convenzionale significa che siamo monaci per la nostra
apparenza. Basta che ci rasiamo il capo e che indossiamo l'abito color
ocra per essere chiamati ``venerabili'', ma quel che è veramente degno
di venerazione non è ancora sorto dentro di noi. Siamo ancora
``venerabili'' solo di nome. È come quando si modella il cemento o si
fonde dell'ottone per farne un'immagine del Buddha. Si dice che è un
Buddha, ma non è davvero così. È solo metallo, legno, cera o pietra. La
realtà convenzionale è così.

Per noi è la stessa cosa. Quando riceviamo l'ordinazione monastica ci
viene dato il titolo di venerabile \emph{bhikkhu}, ma ciò non è
sufficiente a renderci venerabili. Al livello della realtà suprema -- in
altre parole, nella mente -- il termine non è applicabile. La nostra
mente e il nostro cuore non sono ancora stati del tutto perfezionati per
mezzo della pratica con qualità come \emph{mettā}, \emph{karuṇā},
\emph{muditā} e \emph{upekkhā}.\footnote{Sono le quattro dimore
  ``divine'' o ``sublimi'' (\emph{brahma-vihāra}) che si ottengono per
  mezzo dello sviluppo di, appunto, un'illimitata \emph{mettā}
  (benevolenza, gentilezza amorevole), \emph{karuṇā} (compassione),
  \emph{muditā} (gioia empatica e di apprezzamento) e \emph{upekkhā}
  (equanimità).} Non abbiamo ancora raggiunto la piena purezza
interiore. Avidità, odio e illusione ci stanno ancora sbarrando la
strada, non consentendo a ciò che è degno di venerazione di sorgere. La
nostra pratica deve cominciare distruggendo l'avidità, l'odio e
l'illusione, contaminazioni che in genere possono essere rinvenute
dentro ognuno di noi. Sono queste contaminazioni a mantenerci nel ciclo
del divenire e della nascita, e a impedirci di ottenere la pace della
mente. Avidità, odio e illusione impediscono al \emph{samaṇa} -- alla
pace -- di sorgere dentro di noi. Finché questa pace non sorge, non
siamo ancora \emph{samaṇa}. In altre parole, il nostro cuore non ha
sperimentato quella pace che è la libertà dall'influsso dell'avidità,
dell'odio e dell'illusione. La ragione per cui pratichiamo risiede
nell'intento di cancellare avidità, odio e illusione dal nostro cuore.
Solo quando queste contaminazioni sono state rimosse possiamo
raggiungere la purezza, che è davvero venerabile.

Interiorizzare quel che è venerabile all'interno del nostro cuore non
comporta solo lavorare con la mente, ma anche con il corpo e con la
parola. Sono cose che devono lavorare insieme. Prima di poter praticare
con il corpo e con la parola, dovete praticare con la mente. Se
praticherete solo con la mente, trascurando il corpo e la parola,
neanche questo funzionerà, ovviamente. Sono inseparabili. Praticare con
la mente fino a quando diventa liscia, affinata e bella è simile a
produrre una ben rifinita colonna o una bella tavola di legno. Prima di
ottenere una colonna liscia, verniciata e piacevole a vedersi dovete
innanzitutto abbattere un albero. Poi, prima di frazionarlo, segarlo e
lavorarlo, dovete tagliare le parti grezze, le radici e i rami.
Praticare con la mente è come lavorare con l'albero. Prima dovete
lavorare con le cose grezze. Dovete eliminare le parti grossolane.
Dovete eliminare le radici, eliminare la corteccia e tutto ciò che non è
bello a vedersi, per ottenere ciò che è bello e piacevole per gli occhi.
Dovete attraversare le cose grezze per raggiungere quelle levigate. La
pratica del Dhamma è uguale. Si mira a pacificare e a purificare la
mente, ma è difficile farlo. Dovete cominciare a praticare con le cose
esterne -- il corpo e la parola -- aprendovi la strada verso l'interno,
finché raggiungete ciò che è levigato, risplendente e bello. Lo potete
paragonare a un mobile finito, come quei tavoli e quelle sedie. Ora sono
belli, ma prima erano solo pezzi di legno grezzi con rami e foglie, che
abbiamo dovuto piallare e lavorare. Questo è il modo per ottenere mobili
belli, o una mente perfetta e pura.

Per questo motivo, il retto sentiero verso la pace, il Sentiero che il
Buddha impostò e che conduce alla pace della mente e alla pacificazione
delle contaminazioni, è \emph{sīla}, \emph{samādhi} e \emph{paññā}.
Questo è il Sentiero della pratica. È il Sentiero che conduce alla
purezza, e a realizzare e incarnare la qualità del \emph{samaṇa}. È la
via del completo abbandono dell'avidità, dell'odio e dell'illusione. Che
la guardiate interiormente o esteriormente, la pratica non è altro che
questo.

Questo modo di addestrare e di far maturare la mente -- che comporta la
recita dei canti, la meditazione, i discorsi di Dhamma e tutte le altre
componenti della pratica -- vi forzano ad andare controcorrente rispetto
alle contaminazioni. Dovete andare contro le tendenze della mente,
perché normalmente a noi piace prendere le cose alla leggera, essere
pigri ed evitare tutto quel che procura attriti o comporta sofferenze e
difficoltà. La mente non vuole fare sforzi o mettersi in gioco. Questa è
la ragione per cui dovete essere pronti a sopportare disagi e a
sforzarvi nella pratica. Dovete usare il Dhamma della sopportazione e
lottare davvero. Prima il vostro corpo era solo un mezzo per divertirsi
e, siccome avete sviluppato ogni genere di abitudini malsane, è
difficile iniziare a praticare. Prima non applicavate il contenimento
alle vostre parole, e così ora risulta difficile iniziare a contenerle.
È come per quel pezzo di legno, non importa quanto problematico o duro
possa sembrare. Prima di farlo diventare un tavolo o una sedia si
incontrano delle difficoltà. Non è questo il punto importante. Si tratta
soltanto di una cosa che dovete sperimentare lungo il cammino. Dovete
lavorare con il legno grezzo per arrivare a produrre dei mobili finiti.

Il Buddha insegnò che per tutti noi questo è il modo di praticare. Tutti
i suoi discepoli che portarono a termine il loro lavoro e divennero
compiutamente illuminati, prima di ricevere l'ordinazione monastica e di
praticare con Lui erano stati \emph{puthujjana}.\footnote{\emph{Puthujjana}:
  Una persona comune, ordinaria, non illuminata; un essere ``mondano''
  che non ha ancora realizzato alcuna Illuminazione.} Erano tutti stati
esseri ordinari non illuminati come noi, con braccia e gambe, occhi e
orecchi, avidità e collera, proprio come noi. Non avevano alcuna
caratteristica speciale che li rendeva particolarmente diversi da noi.
Così erano all'inizio sia il Buddha sia i suoi discepoli. Praticarono, e
dalla non-Illuminazione procedettero verso l'Illuminazione, dalla
bruttezza verso la bellezza e da quel che è quasi infruttuoso verso quel
che è grandemente benefico. Questo lavoro è proseguito attraverso le
successive generazioni fino ai giorni nostri. Sono i figli di gente
ordinaria -- contadini, commercianti e uomini d'affari -- che, dopo
essere stati intrappolati nei piaceri sensoriali del mondo, hanno
lasciato casa per l'ordinazione monastica. Quei monaci dei tempi del
Buddha erano in grado di praticare e di addestrarsi, e voi dovete capire
che avete lo stesso potenziale. Siete fatti dei cinque \emph{khandhā},
proprio come loro. Anche voi avete un corpo, sensazioni piacevoli e
spiacevoli, memoria e percezione, formazioni mentali e coscienza, come
pure una mente che vaga e prolifera. Potete essere coscienti del bene e
del male. È tutto uguale. Alla fine, quella combinazione di fenomeni
fisici e mentali presenti in ognuno di voi, in quanto individui,
differisce poco da quella dei monaci che praticarono e divennero
Illuminati sotto la guida del Buddha. Tutti cominciarono come esseri
ordinari e non Illuminati. Alcuni erano pure stati banditi e
delinquenti, mentre altri provenivano da un buon ambiente sociale. Non
erano diversi da noi. Il Buddha li ispirò a lasciare casa, a ricevere
l'ordinazione monastica e a praticare per la realizzazione di
\emph{magga} (il Nobile Ottuplice Sentiero) e di \emph{phala}\footnote{\emph{Phala}:
  Frutto. Più specificamente, la fruizione di uno dei quattro Sentieri
  della trascendenza o livelli dell'Illuminazione.} (la Fruizione), e
oggigiorno, allo stesso modo, persone come voi aspirano alla pratica di
\emph{sīla}, \emph{samādhi} e \emph{paññā}.

\emph{Sīla}, \emph{samādhi} e \emph{paññā} sono i nomi che vengono dati
ai differenti aspetti della pratica. Quando praticate con \emph{sīla},
\emph{samādhi} e \emph{paññā} significa che praticate con voi stessi. La
Retta Pratica ha luogo qui, dentro di voi. Retto \emph{sīla} esiste qui,
retto \emph{samādhi} esiste qui. Perché? Perché il vostro corpo è
proprio qui. La pratica di \emph{sīla} coinvolge ogni parte del corpo.
Il Buddha ci insegnò a prestare attenzione ad ogni azione del nostro
corpo. Il vostro corpo esiste qui! È proprio qui che avete delle braccia
e delle gambe. È qui che praticate \emph{sīla}. Se le vostre azioni
saranno in accordo o meno con \emph{sīla} e con il Dhamma, dipenderà da
come addestrerete il corpo. Praticare con la parola significa essere
consapevoli delle cose che dite. Comprende l'astensione dai modi errati
di parlare, per la precisione dalle parole che seminano discordia,
oppure da quelle volgari e, comunque, da quelle non indispensabili o
frivole. Le errate azioni del corpo includono l'uccisione di esseri
viventi, il furto e una cattiva condotta sessuale.

È facile snocciolare l'elenco dei tipi di comportamento errati così come
si trovano nei libri, ma quel che è importante è capire che
potenzialmente questi comportamenti sono tutti dentro di noi. Il vostro
corpo e il vostro linguaggio sono con voi, proprio qui e ora. Praticare
il contenimento morale significa aver cura di evitare le malsane azioni
di uccidere, rubare e comportarsi in modo sessualmente improprio. Il
Buddha ci insegnò a fare attenzione alle nostre azioni fin dal livello
più grossolano. Durante la vostra vita da laici la vostra condotta
morale potrebbe anche non essere stata molto accurata ed è possibile che
abbiate spesso trasgredito i precetti. Ad esempio, in passato potreste
aver ucciso animali con l'accetta o schiacciato degli insetti con un
pugno, o forse non aver prestato molta attenzione al vostro modo di
parlare. Errata parola significa mentire o esagerare la verità.
Linguaggio grossolano significa insultare o essere bruschi con gli altri
in continuazione. «~Sei una nullità.~» «~Sei un idiota.~» E così via.
Linguaggio frivolo significa chiacchierare senza motivo, in modo sciocco
e confuso, senza alcuno scopo e significato. Lo abbiamo fatto tutti.
Senza contenimento! In breve, mantenere \emph{sīla} significa
sorvegliarsi, sorvegliare le vostre azioni e le vostre parole.

Chi svolgerà questo compito di sorvegliare? Chi si assumerà la
responsabilità delle vostre azioni? Quando uccidete un animale, chi è
che lo sa? È la vostra mano a saperlo o è qualcun altro? Quando rubate
qualcosa che appartiene a un altro, chi è consapevole di questa azione?
È la vostra mano a saperlo? È qui che dovete sviluppare la vostra
consapevolezza. Prima di commettere qualche azione sessualmente
impropria, dov'è la vostra consapevolezza? È il vostro corpo a saperlo?
Chi è ``Colui che Conosce'' prima che mentiate, che giuriate o che
diciate qualcosa di frivolo? È la vostra bocca a essere consapevole di
quel che dice, oppure ``Colui che Conosce'' è nelle parole stesse?
Contemplate questo: chiunque sia, è ``Colui che Conosce'' che deve
assumersi la responsabilità del vostro \emph{sīla}. Conducete quella
consapevolezza a sorvegliare le vostre azioni e il vostro linguaggio.
Quell'atto di conoscere, quella consapevolezza è ciò di cui vi avvalete
per sorvegliare la vostra pratica. Per mantenere \emph{sīla} usate
quella parte della mente che indirizza le vostre azioni e vi induce a
fare il bene e il male. Catturate il colpevole e fatelo diventare un
poliziotto o un sindaco. Afferrate la mente ribelle e mettetela al
vostro servizio affinché si prenda la responsabilità di tutte le vostre
azioni e parole. Guardate tutto questo e contemplatelo. Il Buddha ci
insegnò a fare attenzione alle nostre azioni. Chi è che fa attenzione?
Il corpo non sa nulla, si limita a stare in piedi, ad andarsene in giro
e così via. Lo stesso vale per le mani, non sanno nulla. Prima che
tocchino o che afferrino qualcosa, ci deve essere qualcuno che
impartisca degli ordini. Quando prendono o posano le cose, ci deve
essere qualcuno che dice cosa devono fare. Le mani stesse non sono
consapevoli di nulla, ci deve essere qualcuno che impartisce degli
ordini. Lo stesso vale per la bocca. Qualsiasi cosa dica, se racconta il
vero o mente, se è maleducata o semina discordia, ci deve essere
qualcuno che la fa parlare.

La pratica comporta instaurare \emph{sati}, la consapevolezza,
all'interno di questo ``Colui che Conosce''. ``Colui che Conosce'' è
quell'intenzione della mente che in precedenza vi induceva a uccidere
esseri viventi, a rubare cose che appartengono ad altri, a indulgere al
sesso illecito, a mentire, a calunniare, a dire stupidaggini e cose
frivole, e a intraprendere ogni genere di comportamento privo di
moderazione. ``Colui che Conosce'' ci induce a parlare. Esiste
all'interno della mente. Focalizzate la vostra consapevolezza o
\emph{sati} -- quella costante rammemorazione -- su questo ``Colui che
Conosce''. Lasciate che sia la conoscenza a prendersi cura della vostra
pratica.

In fin dei conti, le indicazioni più importanti per la condotta morale
fissate dal Buddha furono: uccidere è male, è una trasgressione a
\emph{sīla}; rubare è una trasgressione; comportarsi in modo
sessualmente scorretto è una trasgressione; mentire è una trasgressione;
parlare in modo volgare e frivolo è una trasgressione. Sono tutte
trasgressioni a \emph{sīla}. Imprimete tutto questo nella vostra
memoria. È un codice di disciplina morale, così com'è stato voluto dal
Buddha, che vi incoraggia a fare attenzione a colui che sta dentro di
voi, al responsabile delle precedenti trasgressioni ai precetti morali.
Quello lì, il responsabile degli ordini di uccidere o di fare del male
agli altri, di rubare, di fare sesso illecito, di dire cose false o
malsane e dell'essere privi di moderazione in tutti i modi possibili:
cantare, ballare, festeggiare e fare stupidaggini. Colui che vi dava
ordini per indulgere a tutti questi comportamenti è quello stesso che
ora inducete a prendersi cura della mente. Usate \emph{sati}, la
consapevolezza, per far sì che la mente abbia rammemorazione del momento
presente e mantenga così la compostezza mentale. Fate in modo che la
mente si prenda cura di se stessa. Fatelo bene.

Se la mente è davvero in grado di prendersi cura di se stessa, non è
così difficile controllare il linguaggio e le azioni, visto che è la
mente a esserne il supervisore. Mantenere \emph{sīla}, in altre parole
prendersi cura delle vostre azioni e del vostro linguaggio, non è poi
una cosa tanto difficile. Sostenete la consapevolezza in ogni momento e
in ogni postura, in piedi, camminando, seduti o distesi. Insediate la
consapevolezza prima di compiere qualsiasi azione, di parlare o di
entrare in conversazione. Non è che prima parlate o fate qualcosa:
dovete prima insediare la consapevolezza e dopo potrete agire o parlare.
Dovete avere \emph{sati}, dovete avere rammemorazione prima di fare
qualsiasi cosa. Non importa cosa stiate per dire, prima di tutto dovete
avere rammemorazione nella mente. Praticate in questo modo finché lo
fate in modo fluido. Praticate per riuscire a tenere il passo con quello
che succede nella mente, fino a quando la consapevolezza non vi richiede
alcuno sforzo e prima di agire siete consapevoli, prima di parlare siete
consapevoli. Questo è il modo per instaurare la consapevolezza nel
cuore. È con ``Colui che Conosce'' che sorvegliate voi stessi, perché
tutte le vostre azioni nascono da lì.

È il luogo dal quale si originano tutte le intenzioni delle vostre
azioni, e questa è la ragione per cui la pratica non funzionerà se
cercate di far sì che sia qualcun altro a svolgere il lavoro. La mente
deve badare a se stessa. Se non riesce a prendersi cura di se stessa,
nessun altro può farlo. Per questa ragione il Buddha insegnò che
mantenere \emph{sīla} non è molto difficile, perché significa solo
sorvegliare la propria mente. Se la consapevolezza è completamente
insediata, tutte le volte che direte o farete qualcosa di nocivo per voi
stessi o per gli altri, lo saprete immediatamente. Sapete quello che è
giusto e quello che è sbagliato. E così che conservate \emph{sīla}.
Praticate con il corpo e con la parola dal livello più basilare.

Le vostre azioni e le vostre parole diventano aggraziate e piacevoli per
gli occhi e per le orecchie se le sorvegliate, mentre voi stessi vi
sentite bene e a vostro agio nella moderazione. Ogni vostro
comportamento e modo di essere, ogni parola e ogni movimento diventano
belli, perché fate attenzione a riflettere, a regolare e correggere il
vostro comportamento. Potete paragonare tutto questo con il luogo in cui
dimorate o con la sala per la meditazione. Se regolarmente pulite il
luogo in cui vivete e ve ne prendete cura, allora sia l'interno sia
l'area circostante saranno piacevoli da vedere e non un pugno
nell'occhio, disordinati, perché c'è qualcuno che bada a essi. Per le
vostre azioni e per le vostre parole è la stessa cosa. Se ve ne prendete
cura diventano belle, e impedite che sorga tutto ciò che è cattivo e
sporco.

\emph{Ādikalyāṇa}, \emph{majjhekalyāṇa}, \emph{pariyosānakalyāṇa}:~bello
all'inizio, bello nel mezzo, e bello alla fine, oppure armonioso
all'inizio, armonioso nel mezzo e armonioso alla fine. Che cosa
significa? Proprio che la pratica di \emph{sīla}, \emph{samādhi} e
\emph{paññā} è bella. La pratica è bella all'inizio. Se all'inizio è
bella, ne consegue che sarà bella nel mezzo. Se praticate la
consapevolezza e il contenimento fino a quando ciò vi risulterà agevole
e naturale -- in modo tale che la vigilanza sia costante -- la mente
diverrà stabile e risoluta nel praticare \emph{sīla} e il contenimento.
Farà continuamente attenzione alla pratica e diverrà perciò concentrata.
Quella caratteristica di essere stabile e irremovibile nella forma e
nella disciplina monastica, e incrollabile nella pratica della
consapevolezza e del contenimento può essere definita come
\emph{samādhi}.

\emph{Sīla} è quell'aspetto della pratica caratterizzato da un
contenimento continuo, allorché vi prendete costantemente cura delle
vostre azioni e delle vostre parole, e vi assumete la responsabilità di
ogni vostro comportamento esterno. La caratteristica di essere
incrollabili nella pratica della consapevolezza e del contenimento è
chiamata \emph{samādhi}. La mente è stabilmente concentrata in questa
pratica di \emph{sīla} e del contenimento. Essere stabilmente
concentrati nella pratica di \emph{sīla} significa che vi è uniformità e
continuità nella pratica della consapevolezza e del contenimento. Queste
sono le caratteristiche esteriori del \emph{samādhi} utilizzate nella
pratica del mantenimento di \emph{sīla}. Ovviamente esiste anche un lato
interiore e più profondo. È essenziale che sviluppiate e manteniate
\emph{sīla} e \emph{samādhi} fin dall'inizio, dovete farlo prima di
qualsiasi altra cosa.

Quando la mente sarà determinata nella pratica, e \emph{sīla} e
\emph{samādhi} saranno stabilmente insediati, sarete in grado di
investigare e di riflettere su quello che è salutare e quello che non è
salutare allorché sperimenterete vari oggetti mentali. Vi chiederete:
«~Questo è giusto?~» «~Quello è sbagliato?~» Quando la mente entra in
contatto con varie immagini, suoni, odori, sapori, sensazioni tattili o
pensieri, ``Colui che Conosce'' sorgerà e insedierà la consapevolezza
del piacere e del non piacere, della felicità e della sofferenza e dei
differenti generi di oggetti mentali che sperimentate. Giungerete a
capire con chiarezza, e vedrete molte e diverse cose. Se siete
consapevoli vedrete i vari oggetti mentali che passano nella mente e le
reazioni che a essi si sovrappongono quando li sperimentate. ``Colui che
Conosce'' li assumerà automaticamente come oggetti di contemplazione.
Quando la mente è vigile e la consapevolezza è stabilmente insediata,
noterete tutte le reazioni che si manifestano per mezzo del corpo, della
parola o della mente allorché si ha esperienza degli oggetti mentali.
\emph{Paññā} è quell'aspetto della mente che, all'interno del campo
della vostra consapevolezza, identifica e seleziona il bene e il male,
quello che è giusto e quello che è sbagliato in tutti gli oggetti
mentali. È \emph{paññā} al suo stadio iniziale, la sua maturazione è un
risultato della pratica. Tutti questi differenti aspetti della pratica
sorgono dall'interno della mente. Il Buddha definì queste
caratteristiche come \emph{sīla}, \emph{samādhi} e \emph{paññā}. Quando
si praticano all'inizio, sono così.

Andando avanti con la pratica, nella mente iniziano a sorgere nuovi
attaccamenti e altri tipi di illusione. Questo significa che cominciate
ad attaccarvi a ciò che è bene, a ciò che è salutare. Si inizia a temere
ogni difetto, qualsiasi errore della mente, si è ansiosi che possano
nuocere al \emph{samādhi}. Nello stesso tempo si comincia a essere
diligenti e a lavorare sodo, e ad amare e nutrire la pratica. Ogni volta
che la mente entra in contatto con gli oggetti mentali, si è timorosi e
tesi. Divenite anche consapevoli degli errori degli altri, pure della
più piccola cosa fatta in modo sbagliato. È perché vi preoccupate della
vostra pratica. Questo significa praticare \emph{sīla}, \emph{samādhi} e
\emph{paññā} a un certo livello -- quello esteriore -- basato sul fatto
che nel vostro modo di vedere si è instaurato una sintonia con la forma
e con i fondamenti della pratica prescritta dal Buddha. Infatti, sono
queste le radici della pratica ed è essenziale che si insedino nella
mente.

Continuate a praticare in questo modo il più possibile, fino a quando
arrivate al punto che, ovunque andiate, state costantemente a giudicare
e a individuare errori in tutti coloro che incontrate. Reagite
continuamente con attrazione e avversione al mondo che vi circonda,
siete colmi di ogni genere di incertezze e vi attaccate sempre a
opinioni riguardanti il modo giusto e il modo sbagliato di praticare. È
come se foste ossessionati dalla pratica. Ora però non dovete
preoccuparvene, a questo punto è meglio praticare troppo piuttosto che
troppo poco. Praticate molto e dedicatevi a sorvegliare il corpo, le
parole e la mente. In realtà, non è mai troppo. Questo è quel che si
dice praticare \emph{sīla} a un certo livello. Nei fatti \emph{sīla},
\emph{samādhi} e \emph{paññā} sono una cosa sola. Se la pratica di
\emph{sīla} a questo stadio la si dovesse descrivere in termini di
\emph{pāramī},\footnote{\emph{Pāramī}: ``Perfezione''. Per l'elenco
  completo delle dieci qualità spirituali, si veda il \emph{Glossario}, p. \pageref{glossary-parami}.}
si dovrebbe parlare di \emph{dāna}\footnote{\emph{Dāna}: L'atto di
  donare, liberalità, generosità; fare offerte, elemosine.}
\emph{pāramī}, o di \emph{sīla pāramī}, la perfezione spirituale del
contenimento morale. Questa è la pratica a un certo livello. Dopo
raggiunto questo stadio di sviluppo, potete intraprendere la pratica al
livello più profondo di \emph{dāna upapāramī}\footnote{\emph{Upapāramī}:
  Il termine si riferisce ugualmente alle Dieci Perfezioni o qualità
  spirituali, ma praticate a un livello più intenso e profondo;
  praticate al grado più alto, vengono chiamate \emph{paramattha
  pāramī}.} e \emph{sīla upapāramī}. Sorgono dalle stesse qualità
spirituali, ma la mente pratica a un livello più sottile. Per ottenere
quel che è sottile da ciò che è grezzo, semplicemente concentrate e
mettete a fuoco i vostri sforzi.

Quando raggiungerete questo fondamento della vostra pratica, nel cuore
si radicherà una sensazione di grande vergogna e timore di fare delle
cose sbagliate. Quale che sia il momento e il luogo, sia in pubblico che
in privato, questa paura di fare qualcosa di sbagliato sarà sempre nella
vostra mente. Avrete veramente paura di qualsiasi azione sbagliata. Si
tratta di una qualità della mente che conservate per ogni aspetto della
pratica. Il vostro oggetto mentale è la pratica della consapevolezza e
del contenimento nel corpo, nella parola e nella mente, nonché la
costante distinzione tra giusto e sbagliato. In questo modo divenite
concentrati, e mediante tale stabile e irremovibile ancoraggio a questo
modo di praticare è la mente stessa che diventa \emph{sīla},
\emph{samādhi} e \emph{paññā}: sono le caratteristiche della pratica
descritte negli insegnamenti tradizionali.

Man mano che continuate a sviluppare e a sostenere la pratica, queste
differenti caratteristiche e qualità si perfezionano insieme nella
mente. Ovviamente praticare \emph{sīla}, \emph{samādhi} e \emph{paññā} a
questo livello non è ancora sufficiente a produrre i fattori dei
\emph{jhāna},\footnote{\emph{Jhāna}: Assorbimento mentale; uno stato di
  forte concentrazione focalizzata su una singola sensazione fisica (che
  conduce a un \emph{rūpajhāna}), oppure su di una nozione mentale (che
  conduce a un \emph{arūpajhāna}).} la pratica è ancora troppo
grossolana. Tuttavia, la mente è già sufficientemente affinata, ma è
affinata da un punto di vista grossolano! Per una persona ordinaria non
illuminata che non si è mai presa cura della mente o che non ha
praticato molta meditazione e consapevolezza, già questo è una cosa
abbastanza affinata. È come quando un povero pensa che possedere due o
tre dollari sia molto, mentre per un milionario non è nulla. Così stanno
le cose. Qualche dollaro è molto quando si è a terra e a corto di
denaro, e allo stesso modo pure se nelle fasi iniziali della pratica
potreste essere in grado di lasciar andare solo le contaminazioni più
grossolane, questo può sembrare una cosa abbastanza profonda per chi non
è illuminato e non ha mai praticato o lasciato andare delle
contaminazioni in precedenza. A questo livello, si può provare una certa
soddisfazione quando si riesce a praticare al massimo delle proprie
capacità. Si tratta di una cosa che vedrete da voi stessi. Deve essere
sperimentata nella mente del praticante.

Se è così, significa che siete già sul Sentiero, che state cioè
praticando \emph{sīla}, \emph{samādhi} e \emph{paññā}. Devono essere
praticate insieme. Se uno di questi aspetti è manchevole, la pratica non
si svilupperà in modo corretto. Più \emph{sīla} si perfeziona, più la
mente diviene salda. Più la mente è salda, più coraggiosa diventa
\emph{paññā} e così via, ogni elemento della pratica supporta e
intensifica gli altri. Alla fine, siccome i tre aspetti della pratica
sono strettamente legati, potenzialmente i tre termini divengono
sinonimi. Quando praticate continuamente in questo modo, senza
affievolire i vostri sforzi, questa è \emph{sammā-paṭipadā}, retta
pratica.

Se state praticando in questo modo, siete entrati nel corretto Sentiero
della pratica. Vi siete incamminati proprio lungo il primo stadio del
Sentiero: è il livello più grossolano, è una cosa davvero difficile da
sostenere. Quando approfondirete e affinerete la pratica, \emph{sīla},
\emph{samādhi} e \emph{paññā} matureranno insieme a partire da questo
stesso punto, si affineranno partendo da questo stesso materiale grezzo.
È come per le nostre palme da cocco. La palma da cocco assorbe l'acqua
dalla terra e la spinge su per il tronco. Quando l'acqua raggiunge il
cocco è diventata pulita e dolce, sebbene provenga dalla normale acqua
del suolo. La palma da cocco si nutre di elementi essenzialmente
grossolani come la terra e l'acqua che vengono assorbite, purificate e
trasformate in qualcosa di molto più puro e dolce di quello che erano in
precedenza. Similmente, la pratica di \emph{sīla}, \emph{samādhi} e
\emph{paññā} -- in altre parole \emph{magga} -- ha un inizio grossolano,
ma il risultato dell'addestramento e dell'affinamento della mente per
mezzo della meditazione e della riflessione fa sì che la pratica stessa
divenga sempre più sottile.

Con il progressivo affinarsi della mente, la pratica della
consapevolezza si fa più focalizzata poiché si concentra su un'area
sempre più precisa. In realtà la pratica diventa più facile quando la
mente si volge sempre più all'interno per focalizzarsi su se stessa. Non
fate più grandi errori né vi comportate in modo palesemente sbagliato.
Ora, quando qualcosa eserciterà un influsso sulla mente, sorgeranno dei
dubbi. Ad esempio, se agire o parlare in una certa maniera sia giusto o
sbagliato. Semplicemente vi limitate ad arrestare la proliferazione
mentale e, intensificando gli sforzi nella pratica, continuate a
rivolgere l'attenzione sempre più in profondità verso l'interno. La
pratica del \emph{samādhi} gradualmente diventerà più stabile e
concentrata. La pratica di \emph{paññā} si intensificherà, e potrete
così vedere le cose con maggiore chiarezza e con sempre maggiore
facilità.

Ne risulterà che infine sarete in grado di vedere la mente e i suoi
oggetti, senza dover fare alcuna distinzione tra mente, corpo e parola.
Non avrete più alcun bisogno di separare nulla di tutto questo, che si
tratti della mente e del corpo, o della mente e dei suoi oggetti.
Vedrete che è la mente a impartire ordini al corpo. Il corpo deve
dipendere dalla mente prima di poter entrare in funzione. È ovvio che
sia la mente a essere in continuazione soggetta ai differenti oggetti
che la contattano e condizionano, prima che essa possa esercitare un
qualche effetto sul corpo. Quando continuate a rivolgere l'attenzione
verso l'interno e a riflettere sul Dhamma, la facoltà della saggezza
matura progressivamente, e alla fine non fate altro che contemplare la
mente e gli oggetti mentali. Questo significa che iniziate a
sperimentare il corpo, \emph{rūpadhamma} (materiale), come
\emph{arūpadhamma} (immateriale). Mediante la visione profonda non
andate più a tentoni, non siete più incerti nella vostra comprensione
del corpo e del modo in cui esso è. La mente sperimenta le
caratteristiche fisiche del corpo come \emph{arūpadhamma} -- oggetti
privi di forma -- che entrano in contatto con la mente. In definitiva,
contemplate solo la mente e gli oggetti mentali, quegli oggetti che
entrano nella vostra consapevolezza.

Ora, esaminando la vera natura della mente, potete osservare che nel suo
stato naturale essa non ha preoccupazioni o problemi che la sovrastano.
È come un pezzo di stoffa o una bandiera legata all'estremità di un
palo. Fino a quando la bandiera sta per conto suo, indisturbata, non le
succede nulla. La foglia di un albero è un altro esempio: di norma resta
quieta e imperturbata. Se si muove o svolazza è a causa del vento, di
una forza esterna. Di solito non accadono molte cose alle foglie,
restano ferme. Non vanno a cercarsi coinvolgimenti con qualcosa o
qualcuno. Quando cominciano a muoversi, ciò è dovuto all'influsso di
qualcosa di esterno, come il vento, che le fa muovere avanti e indietro.
Lo stesso avviene con la mente nel suo stato naturale. Nella mente non
esiste amore o odio, né essa cerca di per sé di criticare altre persone.
È indipendente, esiste in uno stato di purezza che è davvero
cristallino, radioso e privo di macchia. Nella sua condizione di
purezza, la mente è serena, priva di felicità o di sofferenza, non
sperimenta alcuna \emph{vedanā} (sensazione). È questo il vero stato
della mente.

Lo scopo della pratica, allora, consiste nel cercare interiormente,
nello scrutare e investigare fino a quando si raggiunge la mente
originaria. La mente originaria è conosciuta anche come mente pura. La
mente pura è la mente priva di attaccamento. Non viene influenzata dagli
oggetti mentali. In altre parole, non insegue i vari tipi di oggetti
mentali piacevoli e spiacevoli. È in un continuo stato di conoscenza e
di vigilanza, completamente consapevole di tutto quello che sperimenta.
Quando la mente è così, non vi è oggetto mentale sperimentato, piacevole
o spiacevole, che sia in grado di disturbarla. La mente non ``diventa''
nulla. In altri termini, niente la scuote. Perché? Perché c'è
consapevolezza. La mente conosce se stessa come pura. Si è evoluta per
conto suo, è del tutto indipendente. Ha raggiunto il suo stato
originario. Com'è possibile realizzare questo stato originario? Per
mezzo della facoltà della consapevolezza, riflettendo con saggezza e
vedendo che tutte le cose sono solo fenomeni condizionati che sorgono in
ragione dell'influsso esercitato dagli elementi, senza che ci sia alcun
essere individuale a controllarli.

Avviene così per la felicità e per l'infelicità che sperimentiamo.
Quando questi stati mentali sorgono, sono solo ``felicità'' e
``sofferenza''. Non c'è alcun proprietario della felicità. La mente non
è il proprietario della sofferenza. Gli stati mentali non appartengono
alla mente. Guardatelo voi stessi. In realtà queste cose non fanno parte
della mente, sono cose separate e distinte. La felicità è solo lo stato
mentale della felicità, la sofferenza è solo lo stato mentale della
sofferenza. Voi siete solo il conoscitore. In passato, a causa delle
radici dell'avidità, dell'odio e dell'illusione che già esistevano nella
mente, ogni volta che vi capitava di intravedere un oggetto mentale
minimamente piacevole o spiacevole, la mente reagiva all'istante: lo
afferravate e dovevate sperimentare o felicità o sofferenza. Stavate in
continuazione a indulgere agli stati di felicità e di sofferenza. È così
finché la mente non conosce se stessa, finché non è brillante e
luminosa. La mente non è libera. Su di essa influiscono tutti gli
oggetti mentali che sperimenta. In altre parole, essa è priva di
rifugio, non è in grado di fare affidamento su se stessa. Riceve
un'impressione mentale piacevole, e arriva il buon umore. La mente
dimentica se stessa.

La mente originaria è invece al di là del bene e del male. Questa è la
natura originaria della mente. Se siete felici per aver sperimentato un
oggetto mentale piacevole, è un'illusione. Se siete infelici per aver
sperimentato un oggetto mentale spiacevole, è un'illusione. Gli oggetti
mentali spiacevoli vi fanno soffrire e quelli piacevoli vi rendono
felici: questo è il mondo. Gli oggetti mentali sorgono insieme al mondo.
Sono il mondo. Fanno sorgere felicità e sofferenza, bene e male, e tutto
ciò che è soggetto all'impermanenza e all'incertezza. Quando ci si
separa dalla mente originaria, tutto diviene incerto. C'è solo nascere e
morire incessantemente, incertezza e apprensione, sofferenza e disagio,
senza alcuna possibilità di fermare tutto questo o di condurlo a
cessazione. Questo è \emph{vaṭṭa}.

Mediante saggia riflessione, potete constatare di essere soggetti a
vecchie abitudini e condizionamenti. La mente è in se stessa realmente
libera, ma dovete soffrire a causa dei vostri attaccamenti. Prendiamo ad
esempio la lode e il biasimo. Supponete che vi dicano che siete stupidi.
Perché soffrite? Perché vi sentite criticati. Vi riempite la mente con
questo pezzetto d'informazione che avete ``preso''. L'azione di
``prendere'', di accumulare e ricevere quella conoscenza senza piena
consapevolezza, fa sorgere un'esperienza che equivale a trafiggere se
stessi. Questo è \emph{upādāna}. Quando venite trafitti, c'è
\emph{bhava}. \emph{Bhava} è la causa per \emph{jāti} (nascita).
Allorché vi siete addestrati a non prestare alcuna attenzione né
attribuire importanza alle cose che la gente dice, considerandole solo
come suoni che entrano in contatto con i vostri orecchi, non ci sarà
alcuna forte reazione e, quando nulla si crea nella mente, non sarete
costretti a soffrire. Sarà come se a rimproverarvi fosse un cambogiano:
sentirete i suoni delle sue parole, ma resterebbero solo suoni, perché
non ne comprenderete il significato. Non sarete consapevoli di quello
che vi viene detto. La mente non riceverà le informazioni, sarà come
sentire solo dei suoni, e continuerete a sentirvi a vostro agio. Se
qualcuno vi criticasse in una lingua che non comprendete, sentireste
solo il suono della voce e restereste imperturbati. Non sareste
riassorbiti e feriti dal significato delle parole. Quando avete
praticato con la mente fino a questo livello, è più facile conoscere
momento dopo momento il sorgere e lo svanire della coscienza. Quando
riflettete in questo modo, penetrando sempre più a fondo
nell'interiorità, in modo graduale la mente si affina sempre più, va al
di là delle contaminazioni più grossolane.

\emph{Samādhi} significa mente concentrata con saldezza, e più praticate
più la mente diventa stabile. Più la mente è concentrata con saldezza,
più diventa risoluta nella pratica. Più contemplate, più siete
fiduciosi. La mente diviene davvero stabile, al punto che nulla in
assoluto può farla vacillare. Siete del tutto certi che non c'è alcun
oggetto mentale che abbia il potere di scuoterla. Gli oggetti mentali
sono oggetti mentali, la mente è la mente. La mente sperimenta stati
mentali buoni e cattivi, felicità e sofferenza, perché è ingannata dagli
oggetti mentali. Se non viene ingannata dagli oggetti mentali non c'è
sofferenza. La mente priva di illusioni non può essere scossa. Questa
condizione consiste in uno stato di consapevolezza nel quale tutte le
cose e tutti i fenomeni sono complessivamente visti come
\emph{dhātu}\footnote{\emph{Dhātu}: Elemento, proprietà. Terra (nel
  senso di solidità), acqua (liquidità), fuoco (calore) e vento
  (movimento).} che sorgono e svaniscono. Tutto qui. È possibile avere
questo genere di esperienza, e tuttavia non essere ancora in grado di
lasciar andare completamente. Che riusciate o meno a lasciar andare, non
siatene turbati. Prima di qualsiasi altra cosa, dovete almeno sviluppare
e sostenere questo livello di consapevolezza o di stabile determinazione
nella mente. Dovete continuare a esercitare pressione e a distruggere le
contaminazioni per mezzo di uno sforzo determinato, che penetra sempre
più a fondo nella pratica.

Quando si arriva a questo grado di discernimento del Dhamma, la mente si
ritrae a un livello di minor intensità della pratica, che il Buddha e le
successive scritture buddhiste descrivono come
\emph{gotrabhū-citta}.\footnote{\emph{Gotrabhū-citta}: ``Conoscenza del
  cambio di lignaggio'': intravedere il Nibbāna con la
  transizione dalla condizione di essere ordinario (\emph{puthujjana}) a
  quella di Nobile Persona (\emph{ariya}-\emph{puggala}).}
\emph{Gotrabhū-citta} significa che la mente ha sperimentato di essere
al di là dei confini dell'ordinaria mente umana. È la mente del
\emph{puthujjana}, l'ordinario individuo non illuminato, che fa
irruzione nel regno degli \emph{Ariya}, gli Esseri Nobili. Ovviamente,
questo fenomeno ha luogo ancora all'interno della mente di individui
ordinari non illuminati come noi. Il \emph{gotrabhū-puggala} è colui
che, dopo aver progredito nella pratica fino a sperimentare
temporaneamente il Nibbāna, si ritrae da questa esperienza e
continua a praticare ad un altro livello, in quanto non ha ancora del
tutto eliminato le contaminazioni. È come chi si trova ad attraversare
camminando un corso d'acqua, con un piede sulla sponda più vicina e
l'altro su quella più lontana. Sa per certo che ci sono due sponde, ma
non è in grado di attraversare la corrente e perciò fa un passo
indietro. La comprensione che esistono due sponde del corso d'acqua è
simile alla comprensione del \emph{Gotrabhū-puggala} o del
\emph{gotrabhū-citta}. Significa che conoscete il modo per andare al di
là delle contaminazioni, ma non siete ancora in grado di farlo, e perciò
fate un passo indietro. Quando conoscete da voi stessi che questo stato
esiste davvero, questa conoscenza resta costantemente con voi mentre
continuate a praticare la meditazione e a sviluppare la vostra
\emph{pāramī}. Siete certi sia della meta sia della via più diretta per
raggiungerla.

Parlando semplicemente, questo stato che è sorto è la mente stessa. Se
contemplate in accordo con la Verità del modo in cui sono le cose,
potete vedere che esiste solo un Sentiero e che è vostro compito
seguirlo. Significa che fin dall'inizio sapete che gli stati mentali
della felicità e della sofferenza non sono il sentiero da seguire. Si
tratta di una cosa che dovete conoscere da voi stessi. È la Verità del
modo in cui sono le cose. Se vi attaccate alla felicità, siete fuori dal
Sentiero, perché attaccarsi alla felicità causerà il sorgere della
sofferenza. Se vi attaccate alla tristezza, essa sarà una causa per il
sorgere della sofferenza. Lo comprendete, siete già consapevoli e avete
Retta Visione, ma nello stesso tempo non siete ancora in grado di
lasciar andare del tutto i vostri attaccamenti.

Qual è allora il modo corretto di praticare? Dovete percorrere la Via di
Mezzo, il che significa osservare l'andamento dei vari stati mentali di
felicità e sofferenza, e nello stesso tempo tenerli a distanza, lontani
da voi. Questo è il modo corretto di praticare, conservate la
consapevolezza e la presenza mentale anche se non siete ancora capaci di
lasciar andare. È il modo corretto, perché tutte le volte che la mente
si attacca a stati mentali di felicità e di sofferenza, la
consapevolezza dell'attaccamento è sempre presente. Ciò significa che
ogni qualvolta la mente si attacca agli stati di felicità, né la lodate
né vi attribuite valore, e ogni qualvolta si attacca allo stato di
sofferenza non la criticate. In questo modo potete realmente osservare
la mente così com'è. La felicità non va bene, la sofferenza non va bene.
Comprendete che nessuna delle due è il Retto Sentiero. Siete
consapevoli, la consapevolezza di entrambe viene mantenuta, ma non siete
ancora capaci di abbandonarle del tutto. Non siete in grado di lasciarle
cadere, ma potete esserne consapevoli. Instauratasi la consapevolezza,
non attribuite un indebito valore alla felicità o alla sofferenza. Non
date importanza a nessuna di queste due direzioni che la mente può
intraprendere, e non avete dubbi in merito. Sapete che seguire una di
queste strade non è il Retto Sentiero della pratica, e così assumete
sempre la Via di Mezzo dell'equanimità come oggetto mentale. Quando
praticherete fino a raggiungere il punto in cui la mente va al di là
della felicità e della sofferenza, dovrà necessariamente sorgere
l'equanimità quale sentiero da seguire, e vi dovrete muovere
gradualmente, poco a poco. Il cuore conosce la via da percorrere per
andare oltre le contaminazioni, non essendo però in grado di
trascenderle definitivamente, si ritrae e continua a praticare.

Tutte le volte che la felicità sorge e la mente vi si attacca, dovete
prendere questa felicità e contemplarla, e tutte le volte che si attacca
alla sofferenza, è questa sofferenza che dovete prendere e contemplare.
La mente raggiunge infine uno stadio di totale consapevolezza sia della
felicità sia della sofferenza. Ciò avverrà allorché essa sarà in grado
di mettere da parte la felicità e la sofferenza, il piacere e la
tristezza, e di mettere da parte tutto ciò che è il mondo per diventare
\emph{lokavidū}. Quando la mente -- ``Colui che Conosce'' -- potrà
lasciar andare, si assesterà in quel punto. Perché si assesta? Perché
avete praticato e seguito il Sentiero proprio fino a quel punto. Sapete
che quello che dovete fare è raggiungere la fine del Sentiero, ma non
siete ancora capaci di farlo. Quando la mente si attacca o alla felicità
o alla sofferenza, non siete tratti in inganno da esse e vi sforzate di
rimuovere l'attaccamento e di sradicarlo.

Questo è praticare al livello di uno \emph{yogāvacara}, colui che
percorre il Sentiero della pratica, che si sforza di eliminare le
contaminazioni senza aver ancor raggiunto la meta. Vi focalizzate su
queste condizioni e sul modo in cui stanno le cose momento dopo momento
nella vostra mente. Non è necessario che vi venga personalmente chiesto
quali siano i vostri stati mentali, né che facciate alcunché di
speciale. Quando c'è attaccamento a uno di questi due stati mentali, c'è
illusione. Si tratta di attaccamento al mondo. Significa essere bloccati
nel mondo. La felicità significa attaccamento al mondo, la sofferenza
significa attaccamento al mondo. Così è l'attaccamento mondano. Che
cos'è che crea o che fa sorgere il mondo? Il mondo viene creato e
fondato per mezzo dell'ignoranza. Questo avviene perché non siamo
consapevoli del fatto che la mente attribuisce importanza alle cose,
modellando e creando in continuazione \emph{saṅkhāra} (formazioni
mentali).

È a questo punto che la pratica diventa davvero interessante. Tutte le
volte che nella mente c'è attaccamento, continuate a colpire proprio
quel punto, senza lasciar andare. Se c'è attaccamento alla felicità,
continuate a martellarla, senza consentire alla mente di essere portata
via dallo stato mentale. Se la mente si attacca alla sofferenza, la
tenete, la affrontate davvero e la contemplate direttamente. Siete in
procinto di terminare il vostro lavoro. La mente non si lascia sfuggire
un solo oggetto mentale senza rifletterci sopra. Nulla può resistere
all'energia della vostra consapevolezza e della vostra saggezza. Anche
se la mente viene catturata da uno stato mentale non salutare, lo
riconoscete come non salutare e la mente non si distrae. È come quando
si cammina sulle spine. Ovviamente non si vuole camminarci sopra, si
cerca di evitarle, ma nonostante tutto a volte si mette un piede su una
di esse. Quando mettete un piede su una spina, vi sentite bene? Si prova
avversione. Quando conoscete il Sentiero della pratica, significa che
sapete che si tratta del mondo, che si tratta di sofferenza e di ciò che
ci lega a infiniti cicli di nascita e di morte. Benché lo sappiate non
siete capaci di smettere di camminare su queste spine. La mente continua
a seguire i vari stati di felicità e di tristezza, ma non indulge del
tutto a essi. Sostenete un continuo sforzo per distruggere qualsiasi
attaccamento nella mente, per distruggere e rimuovere dalla mente tutto
quello che è il mondo.

Dovete praticare proprio nel momento presente. Meditate proprio qui,
costruite la vostra \emph{pāramī} proprio qui. Questo è il cuore della
pratica, il cuore dei vostri sforzi. Portate avanti un dialogo
interiore, discutete e riflettete sul Dhamma dentro di voi. Si tratta di
una cosa che avviene proprio dentro la mente. Quando gli attaccamenti
mondani sono sradicati, la consapevolezza e la saggezza penetrano verso
l'interno instancabilmente, e ``Colui che Conosce'' sostiene la presenza
mentale con equanimità, consapevolezza e chiarezza, senza lasciarsi
coinvolgere o diventare schiavo di qualcosa o di qualcuno. Non farsi
coinvolgere dalle cose significa conoscere senza attaccamento, conoscere
mentre le cose vengono messe da parte e lasciate andare. Sperimentate
ancora la felicità, sperimentate ancora la sofferenza, sperimentate
ancora oggetti e stati mentali, ma non vi attaccate a essi.

Quando vedete le cose così come sono, conoscete la mente per quello che
è, e conoscete gli oggetti mentali per quello che sono. Conoscete la
mente come separata dagli oggetti mentali e gli oggetti mentali come
separati dalla mente. La mente è la mente, gli oggetti mentali sono gli
oggetti mentali. Quando questi due fenomeni li conoscete per quello che
sono, tutte le volte che entreranno in contatto reciproco ne sarete
consapevoli. Quando la mente sperimenterà gli oggetti mentali, ci sarà
consapevolezza. Il nostro Maestro, il Buddha, descrisse la pratica dello
\emph{yogāvacara} -- che è in grado di sostenere questa consapevolezza
in piedi, camminando, seduto o disteso -- come a ciclo continuo. È
\emph{sammā-paṭipadā}, retta pratica. In questo modo non vi dimenticate
di voi stessi, non siete distratti.

Non vi limitate a osservare le parti più grossolane della pratica, ma
esaminate anche la mente dall'interno, a un livello più sottile. Quello
che sta all'esterno lo mettete da parte. Da questo momento in poi
osservate solo il corpo e la mente, esaminate solo questa mente e gli
oggetti mentali che sorgono e svaniscono, e comprendete che tutto quel
che sorge deve cessare. Con il cessare c'è un ulteriore sorgere, nascere
e morire, morire e nascere, cessare seguito dal sorgere, sorgere seguito
dal cessare. Infine osservate solo l'atto della cessazione.
\emph{Khayavayam} significa degenerazione e cessazione. Degenerazione e
cessazione sono le strade naturali percorse dalla mente e dai suoi
oggetti, questo è \emph{khayavayam}. Quando la mente pratica e
sperimenta queste cose, non ha bisogno di seguire o di cercare altro:
resterà con consapevolezza al passo con le cose. Vedere è solo vedere.
Conoscere è solo conoscere. La mente e gli oggetti mentali sono solo
quello che sono. È questo il modo in cui sono le cose. La mente non ci
prolifera su, né vi aggiunge altro con le sue creazioni.

Non siate confusi o incerti a proposito della pratica. Non fatevi
catturare dai dubbi. Ciò vale anche per la pratica di \emph{sīla}. Come
ho già detto prima, dovete osservarla e contemplare per vedere se è
giusta o sbagliata. Dopo averla contemplata, lasciatela lì dov'è. Non
abbiate dubbi in merito. Praticare \emph{samādhi} è la stessa cosa.
Continuate a praticare, calmando la mente un po' per volta. Se iniziate
a pensare, non importa. Se non pensate, non importa. Lo cosa importante
è conseguire una comprensione della mente. Alcuni vogliono rendere la
mente serena, ma in realtà non sanno cosa sia la vera pace. Non
conoscono la mente serena. Ci sono due tipi di tranquillità: uno
proviene dal \emph{samādhi}, l'altro proviene da \emph{paññā}. La mente
che è serena grazie al \emph{samādhi} è ancora illusa. La pace
proveniente solo dalla pratica del \emph{samādhi} dipende dalla
separazione della mente dagli oggetti mentali. Quando essa non
sperimenta alcun oggetto mentale, ecco che c'è calma, e di conseguenza
ci si attacca alla felicità che arriva con quella calma.

Ovviamente, tutte le volte che c'è un impatto con i sensi, la mente cede
subito. Ha paura degli oggetti mentali. Ha paura della felicità e della
sofferenza, ha paura della lode e del biasimo, ha paura delle forme, dei
suoni, degli odori e dei sapori. Chi raggiunge la serenità solo grazie
al \emph{samādhi} ha paura di tutto e non vuole entrare in contatto con
niente e nessuno all'esterno. Chi pratica \emph{samādhi} in questo modo
vuole solo restarsene isolato in una caverna o da qualche altra parte,
dove poter sperimentare la beatitudine del \emph{samādhi} senza essere
costretto a uscire. Ovunque ci sia un posto tranquillo, ci sgattaiola
dentro e si nasconde. Questo genere di \emph{samādhi} comporta molta
sofferenza: per queste persone è difficile uscire fuori e stare insieme
ad altra gente. Non vogliono vedere forme né sentire suoni. Non vogliono
sperimentare assolutamente nulla! Devono vivere in un luogo che
mantengono particolarmente tranquillo proprio per questa ragione, un
luogo nel quale nessuno possa andare a disturbarli. Hanno necessità di
un ambiente davvero tranquillo.

Questo tipo di serenità non può funzionare. Se avete raggiunto il
necessario livello di calma, allora ritraetevi da esso. Il Buddha non
insegnò a praticare il \emph{samādhi} con illusione. Se state praticando
così, smettete. Se la mente ha conseguito la calma, utilizzatela come
fondamento per la contemplazione. Contemplate la pace della stessa
concentrazione e utilizzatela per entrare in contatto con la mente e
riflettere sui vari oggetti mentali che essa sperimenta. Utilizzate la
calma del \emph{samādhi} per contemplare immagini, suoni, odori, sapori,
sensazioni tattili e pensieri. Utilizzate questa calma per contemplare
le differenti parti del corpo, come i capelli, i peli, le unghie, i
denti, la pelle e così via. Contemplate le tre caratteristiche
\emph{aniccā} (impermanenza), \emph{dukkha} (sofferenza) e \emph{anattā}
(non-sé). Riflettete sul mondo intero. Quando avete contemplato a
sufficienza, potete ripristinare la calma del \emph{samādhi}. Potete
entrarvi per mezzo della meditazione seduta e, dopo aver ripristinato la
calma, continuate con la contemplazione. Utilizzate la condizione di
serenità per addestrare e purificare la mente. Utilizzatela per sfidare
la mente. Man mano che ottenete la conoscenza, usatela per combattere le
contaminazioni, per addestrare la mente. Se vi limitate a entrare in
\emph{samādhi} e a restare lì, non otterrete alcuna visione profonda.
State solo tranquillizzando la mente, questo è tutto. Se però usate la
mente calma per riflettere, a cominciare dalla vostra esperienza
esteriore, questa calma penetrerà per gradi sempre più a fondo
nell'interno, fino a quando la mente sperimenterà la pace più profonda
di ogni altra.

La pace che sorge grazie a \emph{paññā} è caratteristica, perché quando
la mente si ritrae dallo stato di calma, la presenza di \emph{paññā} non
le fa temere immagini, suoni, odori, sapori, sensazioni tattili e
pensieri. Significa che appena vi è contatto con i sensi, la mente è
immediatamente consapevole dell'oggetto mentale. Appena c'è un contatto
con i sensi, lo mettete da parte. Appena c'è un contatto con i sensi, la
consapevolezza è sufficientemente acuta per lasciar subito andare.
Questa è la pace che proviene da \emph{paññā}. Quando state praticando
con una mente così, essa diviene considerevolmente più sottile di quando
sviluppate solo il \emph{samādhi}. La mente diventa davvero potente, e
non cerca più di scappare via. Grazie a questa energia diventate
coraggiosi. Se nel passato avevate paura di sperimentare qualsiasi cosa,
ora conoscete gli oggetti mentali per quello che sono e di paura non ne
avete più. Conoscete la forza della vostra mente e siete impavidi.

Quando vedete una forma, la contemplate. Quando sentite un suono, lo
contemplate. Diventate esperti nella contemplazione degli oggetti
mentali. A fondarvi nella pratica è un nuovo coraggio, che prevale in
ogni circostanza. Che si tratti di immagini, di suoni o di odori, li
percepite e li lasciate andare man mano che si presentano. Di qualsiasi
cosa si tratti, riuscite a lasciar andare tutto. Vedete con chiarezza la
felicità, e la lasciate andare. Vedete con chiarezza la sofferenza, e la
lasciate andare. Ovunque le vediate, le lasciate andare proprio lì dove
si trovano. Non ci sono oggetti mentali in grado di fare presa sulla
mente. Li lasciate lì, e restate al sicuro nel luogo in cui dimorate
all'interno della vostra mente. Quando sperimentate le cose, le mettete
da parte. Quando sperimentate le cose, le osservate. Dopo averle
osservate, le lasciate andare. Tutti gli oggetti mentali perdono il loro
valore, non riescono più a governarvi. Questo è il potere della
\emph{vipassanā}. Quando queste caratteristiche sorgono all'interno
della mente di un praticante, è opportuno modificare il nome della
pratica in \emph{vipassanā}: chiara conoscenza in accordo con la Verità.
È tutto qui. Conoscenza in accordo con Verità del modo in cui sono le
cose. Questa è la pace al suo livello più alto, la pace della
\emph{vipassanā}. Sviluppare la pace solo per mezzo del \emph{samādhi} è
molto, molto difficile. Si è costantemente pietrificati.

Quando la mente ha raggiunto il suo massimo grado di calma, che cosa si
dovrebbe fare? Addestrarla. Praticate con essa. Usatela per contemplare.
Non fatevi spaventare dalle cose. Non attaccatevi. Sviluppare il
\emph{samādhi} solo per restare lì seduti e attaccarsi a stati di
beatitudine mentale non è il vero scopo della pratica. Dovete ritrarvi.
Il Buddha disse che questa è una battaglia che deve essere combattuta,
che non bisogna solo nascondersi in una trincea cercando di evitare i
proiettili del nemico. Quando è tempo di combattere, dovete realmente
uscire allo scoperto a fucile spianato. Alla fine dovete venir fuori da
quella trincea. Non potete restare lì a dormire, quando è tempo di
combattere. La pratica è così. Non potete consentire alla vostra mente
solo di nascondersi, di strisciare nell'ombra.

\emph{Sīla} e \emph{samādhi} sono il fondamento della pratica, ed è
essenziale svilupparli prima di ogni altra cosa. Dovete addestrare voi
stessi e investigare in accordo con il modello monastico e con i modi di
praticare che ci sono stati tramandati.

Sia come sia, vi ho descritto la pratica a grandi linee. In quanto
praticanti dovreste evitare di essere preda dei dubbi. Non dubitate del
modo di praticare. Quando c'è felicità, osservate la felicità. Quando
c'è sofferenza, osservate la sofferenza. Dopo aver instaurato la
consapevolezza, fate lo sforzo di distruggerle entrambe. Lasciatele
andare. Mettetele da parte. Conoscete gli oggetti mentali e continuate a
lasciar andare. Non importa se volete fare la meditazione seduta o
quella camminata. Se continuate a pensare non importa. La cosa
importante è sostenere momento dopo momento la consapevolezza della
mente. Se siete catturati dalla proliferazione mentale, mettete tutto
insieme e contemplatela nel suo insieme come se fosse una cosa sola e
interrompetela laddove essa sorge, dicendo: «~Tutti questi miei
pensieri, tutte queste mie idee e immaginazioni sono solo proliferazione
mentale e nulla di più. È tutto \emph{aniccā}, \emph{dukkha} e
\emph{anattā}. Nulla di tutto questo è certo.~» Disfatevene lì per lì.

