\chapter{Imparare ad ascoltare}

\begin{openingQuote}
  \centering

  Discorso offerto nel settembre del 1978 al Wat Pah Pong.
\end{openingQuote}

Una sera, durante una riunione informale presso il suo alloggio, Ajahn
Chah disse: «~Quando ascoltate il Dhamma, dovete aprire il vostro cuore
e raccogliervi nel centro di esso. Non cercate di accumulare quel che
sentite né sforzatevi di trattenerlo scrupolosamente nella vostra
memoria. Lasciate solo che il Dhamma fluisca nel vostro cuore e si
riveli, e mantenetevi costantemente aperti al flusso nel momento
presente. Ciò che è pronto per essere trattenuto, resterà. Avverrà da
sé. Succederà da sé, non per mezzo di un impegno forzato da parte
vostra.

Nella stessa maniera, quando esponete il Dhamma non dovete costringervi
in alcun modo. Il Dhamma deve fluire spontaneamente dal momento
presente, in accordo con le circostanze. Sapete, è strano, ma è così,
avviene da sé. Anche quando a volte la gente viene da me e davvero non
manifesta alcun desiderio di ascoltare il Dhamma. Il Dhamma inizia a
fluire all'esterno senza nessuno sforzo. Altre volte, la gente sembra
invece ascoltare volentieri. Chiedono un discorso anche in modo formale,
e poi, niente! Semplicemente non succede. Che si può fare? Non so
perché, ma so che è così che succede. È come se le persone avessero
diversi livelli di ricettività, quando però si è lì, allo stesso
livello, le cose succedono e basta.

Se dovete esporre il Dhamma, il modo migliore è non pensarci affatto.
Dimenticatevene e basta. Più ci pensate e cercate di pianificare, peggio
sarà. Però è difficile a farsi, non è vero? A volte mentre il discorso
fluisce senza intoppi, c'è una pausa, e qualcuno può farvi una domanda.
Improvvisamente, ecco che il discorso prende una direzione completamente
diversa. Sembra che ci sia una fonte illimitata, che non si può mai
esaurire.

Credo senza alcun dubbio che il Buddha avesse la capacità di conoscere
il temperamento e la ricettività degli altri esseri. Egli impiegò
proprio questo metodo dell'insegnamento spontaneo. Non è che avesse
bisogno di utilizzare alcun potere sovrumano, è che era sensibile alle
necessità della gente che stava attorno a lui e insegnava di
conseguenza. Un esempio indica la sua spontaneità. Dopo aver esposto il
Dhamma a un gruppo di suoi discepoli, chiese loro se in precedenza
avessero sentito questo insegnamento. Risposero di no. Allora andò
avanti, e disse che nemmeno Lui lo aveva mai sentito.

Continuate la vostra pratica, non importa cosa stiate facendo. Praticare
non dipende da alcuna postura, come stare seduti o camminare. Si tratta
piuttosto di essere di continuo mentalmente presenti al flusso della
vostra consapevolezza e delle vostre sensazioni. Non importa cosa stia
succedendo, raccoglietevi in voi stessi e siate sempre mentalmente
presenti e consapevoli di quel flusso.~»

Poi Ajahn Chah proseguì dicendo: «~La pratica non è andare avanti, ma
c'è movimento in avanti. Nello stesso tempo, non è andare indietro, ma
c'è movimento all'indietro. Infine, la pratica non è fermarsi e restare
fermi, ma c'è il fermarsi e lo stare fermi. Perciò, c'è movimento in
avanti e all'indietro come pure lo stare fermi, ma non potete dire che
si tratti di uno dei tre. La pratica infine giunge ad un punto nel quale
non c'è né movimento in avanti né movimento all'indietro, e nemmeno lo
stare fermi. E allora dov'è?~»

In un'altra occasione informale, disse: «~Per definire il buddhismo
senza troppe frasi e parole, possiamo dire semplicemente: ``Non
aggrapparti o attaccarti a nulla. Resta in armonia con il presente, con
le cose così come sono''.~»

