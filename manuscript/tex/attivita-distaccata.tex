\chapter{Attività distaccata}

Guardiamo l'esempio del Buddha. Egli fu esemplare sia nella pratica sia
nei metodi per insegnare ai discepoli. Il Buddha insegnò i princìpi
della pratica come mezzi abili per vincere la presunzione. Non poteva
praticare per noi. Dopo aver ascoltato l'insegnamento, dobbiamo
insegnare a noi stessi, praticare da soli. È a questo punto che
sorgeranno i risultati, non nel momento dell'insegnamento.

L'insegnamento del Buddha ci consente solo di avere un'iniziale
comprensione del Dhamma, ma il Dhamma non è ancora nei nostri cuori.
Perché? Perché non abbiamo ancora praticato, non abbiamo ancora
insegnato a noi stessi. Il Dhamma sorge dall'interno della pratica. Se
lo conoscete, lo conoscete attraverso la pratica. Se ne dubitate, ne
dubitate nella pratica. Gli insegnamenti del Maestro possono essere
veri, ma ascoltare solo il Dhamma non ci rende ancora in grado di
realizzarlo. L'insegnamento indica solamente la via per realizzare il
Dhamma. Per realizzarlo dobbiamo prendere questo insegnamento e
collocarlo nel nostro cuore. La parte che riguarda il corpo la
applichiamo al corpo, la parte che è per la parola la applichiamo alla
parola, e la parte che è per la mente la applichiamo alla mente. Ciò
significa che dopo aver ascoltato l'insegnamento dobbiamo insegnare a
noi stessi a conoscere quel Dhamma, a essere quel Dhamma.

Il Buddha disse che chi si limita a credere agli altri non è davvero
saggio. Un saggio pratica fino a quando è una cosa sola con il Dhamma,
fino a quando ha fiducia in se stesso, indipendentemente dagli altri.
Una volta, mentre il venerabile Sāriputta sedeva ai piedi del Buddha e
lo ascoltava rispettosamente esporre il Dhamma, il Buddha si rivolse a
lui e gli chiese: «~Sāriputta, credi in questo insegnamento?~» «~No, non
ancora~», rispose il venerabile Sāriputta. È un buon esempio. Il
venerabile Sāriputta ascoltò e ne prese atto. Egli prese semplicemente
atto di quell'insegnamento, perché non aveva ancora sviluppato una sua
propria comprensione in merito e, perciò, disse al Buddha che non
credeva ancora in quell'insegnamento, perché in realtà era così. Queste
parole suonano per lo più come scortesi, ma in verità Sāriputta non fu
scortese. Egli disse la verità, e il Buddha lo lodò per questo. «~Bene,
Sāriputta, bene. Un saggio non crede subito. Dovrebbe prima valutare,
poi credere.~» La convinzione a proposito d'una credenza può assumere
varie forme. Una forma si esprime in accordo con il Dhamma, mentre
un'altra forma è in contrasto con il Dhamma. Quest'ultima è la via della
distrazione, della comprensione avventata, è \emph{micchā-ditthi},
errata visione. Non si ascolta nessun altro.

Prendete come esempio il brahmano\footnote{\emph{Brahmano}: Membro della casta
  dei brahmani, ``sacerdote''; la casta dei brahmani in India ha per
  molto tempo ritenuto che, per nascita, i suoi componenti fossero degni
  del più alto rispetto; si veda \emph{brāhmaṇa}, nel \emph{Glossario}, p. \pageref{glossary-brahmana}.}
Dīghanakha. Questo brahmano credeva solamente in se stesso, non credeva
agli altri. Una volta, quando il Buddha stava riposando a Rājagaha,
Dīghanakha andò ad ascoltare il suo insegnamento. Si potrebbe anche dire
che Dīghanakha andò a insegnare al Buddha poiché aveva l'intenzione di
esporgli i suoi punti di vista. «~Sono dell'opinione che nessun punto di
vista mi convince.~» Questa era la sua opinione. Il Buddha l'ascoltò e
rispose: «~Brahmano, non ti convince neanche il tuo punto di vista.~»
Quando il Buddha gli rispose in questo modo, Dīghanakha fu sconcertato.
Non sapeva cosa dire. Il Buddha glielo spiegò in molti modi, fino a che
il brahmano comprese. Si fermò a riflettere e disse: «~Hmm, questo mio
punto di vista non è giusto.~»

Ascoltando la risposta del Buddha, il brahmano abbandonò le sue
presuntuose opinioni e comprese immediatamente la Verità. Cambiò proprio
lì per lì, come si capovolgerebbe una mano. Lodò l'insegnamento del
Buddha in questo modo: «~Ascoltando l'insegnamento del Beato, la mia
mente s'è illuminata, proprio come quando chi è nell'oscurità vede la
luce. La mia mente è come una bacinella capovolta che è stata
raddrizzata, come un uomo smarrito che ha ritrovato la strada.~» Allora
sorse nella sua mente una conoscenza, nella sua mente raddrizzata.
L'errata visione si dileguò e il suo posto venne occupato dalla Retta
Visione. Scomparve l'oscurità e sorse la luce. Il Buddha dichiarò che al
brahmano Dīghanakha s'era aperto l'Occhio del Dhamma. In precedenza
Dīghanakha era attaccato alle sue opinioni e non aveva intenzione di
cambiarle. Quando però ascoltò l'insegnamento del Buddha, la sua mente
vide la Verità, egli comprese che il suo attaccamento a quei modi di
vedere era sbagliato. Quando sorse la Retta Comprensione, fu in grado di
percepire come erronea la sua precedente conoscenza, e per questo
paragonò la sua esperienza a quella di una persona che vive
nell'oscurità e trova la luce. È così. Fu allora che il brahmano
Dīghanakha trascese la sua errata visione.

Ora siamo noi a dover cambiare in questo modo. Prima di abbandonare le
contaminazioni, dobbiamo cambiare la nostra prospettiva. Dobbiamo
iniziare a praticare correttamente e bene. In precedenza non abbiamo
praticato correttamente e bene, abbiamo pensato di essere nel giusto e
di avere ragione. Quando osserviamo la questione davvero a fondo,
capovolgiamo noi stessi, nello stesso modo in cui si capovolge una mano.
Questo significa che ``Colui che Conosce'', o la saggezza, sorge nella
mente, così che essa è in grado di vedere le cose in modo nuovo. Sorge
un nuovo tipo di consapevolezza.

È per questa ragione che i praticanti devono sviluppare questa
conoscenza che noi chiamiamo \emph{Buddho}, ``Colui che Conosce'', nella
loro mente. In precedenza Colui che Conosce non è lì, la nostra
conoscenza non è chiara, vera e completa. Per educare la mente la
conoscenza che abbiamo è troppo debole. La mente in seguito però cambia,
o si capovolge, a causa di questa consapevolezza chiamata ``saggezza'' o
``visione profonda'' che supera la nostra precedente consapevolezza.
Quel precedente ``Colui che Conosce'' non conosceva ancora completamente
e non era perciò in grado di condurci al nostro obiettivo.

Per questo il Buddha insegnò a guardare nell'interiorità,
\emph{opanayiko}.\footnote{\emph{Opanayiko}: ``Che conduce
  all'interno'', degno di essere realizzato e condotto all'interno
  della mente; un attributo del Dhamma.} Guardare interiormente, non
guardare all'esterno. Oppure, se guardate all'esterno, in seguito
guardate dentro per vedere, interiormente, causa ed effetto. Cercate la
verità in tutte le cose, perché gli oggetti esterni e quelli interni
influiscono vicendevolmente gli uni sugli altri. La nostra pratica
consiste nello sviluppare un certo tipo di consapevolezza, fino a che
essa non diventa più forte di quella precedente. Questo fa sì che nella
mente sorgano la saggezza e la visione profonda, rendendoci in grado di
conoscere con chiarezza il modo in cui lavora la mente, il linguaggio
della mente nonché le modalità e i mezzi mediante i quali agiscono le
contaminazioni.

Il Buddha, quando all'inizio lasciò la sua casa per cercare la
Liberazione, forse non era sicuro sul da farsi, proprio come noi. Cercò
di sviluppare la sua saggezza in molti modi. Andò alla ricerca di
insegnanti, come Uddaka Rāmaputta,\footnote{Uddaka Rāmaputta: Uno dei
  maestri del \emph{bodhisatta} durante la sua ricerca
  dell'Illuminazione.} per praticare la meditazione: gamba destra sulla
gamba sinistra, mano destra sulla mano sinistra, corpo eretto, occhi
chiusi e lasciar andare tutto finché fu in grado di raggiungere un alto
livello di assorbimento meditativo.\footnote{Il livello del ``nulla è'',
  uno degli assorbimenti meditativi nel ``senza-forma'', talora chiamato
  settimo \emph{jhāna}.} Però, quando uscì da quel \emph{samādhi} tornò
il suo vecchio modo di pensare ed egli vi si attaccò proprio come prima.
Notando questo, comprese che la saggezza non era ancora sorta. La sua
comprensione non aveva ancora penetrato la Verità, era ancora
incompleta, manchevole. Ciò nonostante, anche se ottenne una certa qual
comprensione -- non si trattava ancora del più alto genere di pratica --
lasciò quel posto per andare alla ricerca di un nuovo insegnante.

Il Buddha non biasimò il suo vecchio insegnante quando lo lasciò, si
comportò come fa l'ape, che prende il nettare dei fiori senza
danneggiare i petali. Continuò poi a praticare con ālāra
Kālāma\footnote{Ālāra Kālāma: Il maestro che insegnò al
  \emph{bodhisatta} la meditazione nella sfera del senza forma sulla
  base del ``nulla è'' quale più alta fruizione della vita santa.} e
raggiunse un livello ancora più alto di \emph{samādhi}, ma quando uscì
da quello stato, Bimba e Rāhula\footnote{Bimba, la principessa
  Yasodharā, era la moglie del Buddha; Rāhula, suo figlio.} tornarono di
nuovo nei suoi pensieri assieme ai vecchi ricordi e sentimenti. Aveva
ancora bramosie sensoriali e desideri. Riflettendo, vide che
interiormente non aveva ancora raggiunto il suo scopo, e così lasciò
pure quell'insegnante. Diede ascolto ai suoi maestri e fece del suo
meglio per seguire i loro insegnamenti. Passò continuamente in rassegna
i risultati della sua pratica. Non è che faceva delle cose e poi
interrompeva per fare qualcos'altro.

Dopo aver provato con le pratiche ascetiche, comprese che digiunare fino
a diventare uno scheletro coinvolge solo il corpo. Il corpo non conosce
nulla. Praticare in quel modo era come giustiziare un innocente,
ignorando il vero ladro. Quando il Buddha esaminò davvero la questione,
capì che la pratica non è un'occupazione del corpo, ma della mente. Il
Buddha aveva provato l'\emph{attakilamathānuyogo},
l'auto-mortificazione, e scoperto che essa si limitava al corpo. Infatti
tutti i Buddha sono illuminati nella mente.

In relazione sia al corpo sia alla mente, considerate tutto quanto come
transitorio, imperfetto e privo di un sé: \emph{aniccā, dukkha} e
\emph{anattā}. Sono solo fenomeni condizionati della natura. Sorgono in
dipendenza di fattori che li supportano, esistono per un po' e poi
cessano. Quando ci sono condizioni appropriate, sorgono nuovamente. Dopo
essere sorti, esistono per un po' e poi cessano di nuovo. Queste cose
non sono un ``sé'', un ``essere'', un ``noi'' o un ``loro''. Lì non c'è
nessuno, sono semplici sensazioni. La felicità non ha un sé intrinseco,
la sofferenza non ha un sé intrinseco. Non si riesce a trovare alcun sé,
ci sono solo fenomeni della natura che sorgono, esistono e cessano.
Attraversano questo costante ciclo di trasformazione.

Tutti gli esseri, inclusi gli esseri umani, tendono a considerare il
sorgere come se a sorgere fossero loro stessi, l'esistere come se a
esistere fossero loro stessi e il cessare come se a cessare fossero loro
stessi. Perciò si attaccano a tutto. Non vogliono che le cose siano nel
modo in cui sono o vogliono che non vadano diversamente. Ad esempio,
dopo che le cose sono sorte, non vogliono che cessino. Dopo aver
sperimentato la felicità, non vogliono la sofferenza. Se la sofferenza
sorge, vogliono che se ne vada il più in fretta possibile, e ancor
meglio sarebbe se non sorgesse affatto. Questo avviene perché
identificano il corpo e la mente con loro stessi o pensano che
appartenga a loro, e così pretendono che quelle cose seguano i loro
desideri.

Pensare in questo modo è come costruire un argine o una diga senza uno
sbocco per la fuoriuscita dell'acqua. Il risultato è che la diga crolla.
Altrettanto avviene con questo modo di pensare. Il Buddha comprese che
pensare così causa sofferenza. Vedendo questa causa, il Buddha vi
rinunciò. Questa è la nobile verità della causa della sofferenza. Le
verità della sofferenza, della sua causa, della sua cessazione e del
Sentiero che conduce a questa cessazione: proprio qui si blocca la
gente. Se le persone vogliono superare i loro dubbi, è a questo
proposito che lo devono fare. Se si comprende che queste cose sono solo
\emph{rūpa} e \emph{nāma}, ossia fenomeni materiali e mentali, diviene
ovvio che non sono un essere, una persona, un ``noi'' o un ``loro''.
Seguono solo le leggi della natura.

La nostra pratica consiste nel conoscere in questo modo. Non abbiamo il
potere di controllare davvero le cose, in realtà non le possediamo.
Cercare di controllarle causa sofferenza, perché non sono affatto sotto
il nostro controllo. Né il corpo né la mente sono ``sé'' o ``altro da
sé''. Se sappiamo questo, così com'è nella realtà, allora vediamo con
chiarezza. Vediamo la Verità, siamo tutt'uno con essa. È come vedere un
blocco di ferro incandescente che è stato arroventato in una fornace. È
rovente dappertutto. È incandescente se lo tocchiamo sopra, oppure sotto
o di lato. Non importa dove lo tocchiamo, è rovente. È così che dovremmo
vedere le cose.

Per lo più quando cominciamo a praticare vogliamo ottenere, raggiungere,
conoscere e vedere, ma non sappiamo ancora cosa stiamo per ottenere o
conoscere. Tempo fa la pratica di un mio discepolo era affetta da
confusione e dubbi. Continuò però a praticare e io continuai a dargli
istruzioni, fino a che non iniziò a trovare un po' di pace. Quando
infine ottenne un po' di calma, fu di nuovo catturato dai dubbi e disse:
«~Che cosa faccio ora?~» Ecco! La confusione sorge di nuovo. Dice che
vuole la pace, ma quando la ottiene non la vuole, e chiede cosa ci sia
ora da fare! Perciò, in questa pratica dobbiamo fare tutto con distacco.
Come con distacco? Ci distacchiamo vedendo le cose con chiarezza.
Conosciamo le caratteristiche del corpo e della mente per quello che
sono. Meditiamo per trovare la pace, ma nel farlo vediamo quello che
tranquillità non è. Ciò avviene perché il movimento è nella natura della
mente.

Quando pratichiamo il \emph{samādhi} fissiamo la nostra attenzione
sull'inspirazione e sull'espirazione, sulla punta del naso o sul labbro
superiore. Questo ``sollevare'' o ``elevare'' la mente per stabilizzarla
si chiama \emph{vitakka.}\footnote{\emph{Vitakka}: Applicazione
  dell'attenzione.} Quando abbiamo ``sollevato'' la mente e l'attenzione
è fissa su un oggetto, ciò si chiama \emph{vicāra},\footnote{\emph{Vicāra}:
  Mantenimento dell'attenzione.} contemplazione del respiro sulla punta
del naso. Questa caratteristica di \emph{vicāra} sarà ovviamente
mescolata ad altre sensazioni mentali, e si potrebbe pensare che la
nostra mente non sia serena, che non si calmerà, ma in realtà si tratta
solo del modo di lavorare di \emph{vicāra} nel suo fondersi con quelle
sensazioni. Ora, se tutto questo va in una direzione sbagliata, la
nostra mente perderà la concentrazione. Allora dovremo ricomporre la
nostra mente e riportarla sull'oggetto di concentrazione con
\emph{vitakka}. Appena avremo fissato la nostra attenzione, subentrerà
\emph{vicāra}, che si mescolerà con le varie sensazioni mentali.

Quando vediamo che ciò accade, la nostra mancanza di comprensione può
indurci a formulare questa domanda: «~Perché la mia mente vagava? Volevo
che fosse calma, perché non lo è stata?~» Ciò significa praticare con
attaccamento. In realtà, la mente sta solo seguendo la sua natura, ma
noi aggiungiamo a quest'attività il desiderio che la mente sia calma e
pensiamo: «~Perché non è calma?~» L'avversione sorge, e così la
aggiungiamo a tutto il resto, aumentando i nostri dubbi, aumentando la
nostra sofferenza e la nostra confusione. Perciò, se c'è \emph{vicāra},
riflettendo in questo modo sui vari accadimenti all'interno della mente,
dovremmo pensare con saggezza: «~Ah, la mente è semplicemente così.~»
Ecco, a parlare così è Colui che Conosce, che vi dice di vedere il modo
in cui sono le cose. La mente è semplicemente così. Con questo modo di
vedere la lasciamo andare ed essa si rasserena. Quando non è più
centrata, la riportiamo ancora una volta su \emph{vitakka}, e dopo un
po' vi è di nuovo la pace. \emph{Vitakka} e \emph{vicāra} lavorano
insieme in questo modo. Utilizziamo \emph{vicāra} per contemplare le
varie sensazioni che sorgono. Quando \emph{vicāra} si disperde
progressivamente, solleviamo di nuovo la nostra attenzione con
\emph{vitakka}.

L'importante a questo punto è che la nostra pratica sia svolta con
distacco. Vedendo il processo di \emph{vicāra} che interagisce con le
sensazioni mentali potremmo pensare che la mente è confusa e provare
avversione nei riguardi questo processo stesso. Ecco la causa. Non siamo
contenti solo perché vogliamo che la mente sia calma. Questa è la causa:
l'errata visione. Se correggiamo la nostra visione appena un po',
comprendendo che quest'attività è semplicemente nella natura della
mente, ciò è abbastanza per soggiogare la confusione. Questo è chiamato
lasciar andare. Ora, se non ci attacchiamo, se pratichiamo ``lasciando
andare'' -- distacco nell'attività e attività nel distacco -- se
impariamo a praticare in questo modo, \emph{vicāra} avrà sempre meno
cose con cui lavorare. Qualora la nostra mente cessasse di essere
disturbata, \emph{vicāra} tenderà verso la comprensione del Dhamma,
perché la mente ritorna verso la distrazione se non contempliamo il
Dhamma.

Così, c'è \emph{vitakka} e poi \emph{vicāra}, \emph{vitakka} e poi
\emph{vicāra}, \emph{vitakka} e poi \emph{vicāra} e così via, fino a
che, progressivamente, \emph{vicāra} diviene più sottile. Inizialmente
\emph{vicāra} non è ben organizzata. Quando comprendiamo che si tratta
solo della naturale attività della mente, non ci darà problemi a meno
che non ci attacchiamo a essa. È come l'acqua che scorre. Se siamo
ossessionati dalla domanda: «~Perché scorre?~», è ovvio che soffriremo.
Se comprendiamo che l'acqua scorre solo perché è nella sua natura,
allora non c'è sofferenza. \emph{Vicāra} è così. C'è \emph{vitakka} e
poi \emph{vicāra} che interagisce con le sensazioni mentali. Possiamo
assumere queste sensazioni come oggetto di meditazione e, notando tali
sensazioni, calmare la mente. Se conosciamo la natura della mente in
questo modo, allora lasciamo andare, proprio come lasciamo scorrere
l'acqua. \emph{Vicāra} diventa sempre più sottile. Forse la mente tende
a contemplare il corpo, oppure la morte per esempio, o anche altri temi
di Dhamma. Quando il tema della contemplazione è giusto, sorgerà una
sensazione di benessere. Che cos'è questo benessere? È
\emph{pīti},\footnote{\emph{Pīti}: Gioia. Il terzo fattore
  dell'assorbimento meditativo.} il rapimento. Sorge \emph{pīti}, il
benessere. Può manifestarsi con pelle d'oca, imperturbabilità o un senso
di leggerezza. La mente è rapita. Questo si chiama \emph{pīti}. C'è
anche piacere, \emph{sukha},\footnote{\emph{Sukha}: Piacere; benessere;
  soddisfazione, felicità; il quarto fattore dell'assorbimento
  meditativo.} l'andirivieni di varie sensazioni, e lo stato di
\emph{ekaggatārammaṇa},\footnote{\emph{Ekaggatā}: Unificazione mentale;
  il quinto fattore dell'assorbimento meditativo.} o di unificazione
mentale.

Ora, se parliamo nei termini del primo stadio di
concentrazione,\footnote{In termini dottrinali si parlerebbe di primo
  \emph{jhāna}.} esso deve essere così: \emph{vitakka}, \emph{vicāra},
\emph{pīti}, \emph{sukha}, \emph{ekaggatā}. Com'è allora il secondo
stadio? Quando la mente diviene più sottile, \emph{vitakka} e
\emph{vicāra} vengono eliminati in quanto al confronto più grossolani, e
restano solo \emph{pīti}, \emph{sukha} ed \emph{ekaggatā}. Si tratta di
una cosa che la mente fa da sé, non è necessario fare congetture,
conosciamo solo le cose così come sono. Quando la mente diventa ancor
più affinata, anche \emph{pīti} viene scartata, e rimangono solo
\emph{sukha} ed \emph{ekaggatā.} Dove va \emph{pīti}? Non va da nessuna
parte, è che la mente diventa progressivamente più sottile, ed elimina
le qualità troppo grossolane. Elimina tutto ciò che è troppo grossolano,
e continua a eliminare in questo modo fino a raggiungere il culmine
dell'affinamento, noto nei libri come quarto \emph{jhāna}, il più alto
livello di assorbimento mentale. La mente ha progressivamente eliminato
tutto ciò che è diventato troppo grossolano, finché rimangono solo
\emph{ekaggatā} e \emph{upekkhā},\footnote{\emph{Upekkhā}: Equanimità. È
  una delle quattro dimore divine (\emph{brahmavihāra}) e una delle
  Dieci Perfezioni (\emph{pāramī}).} l'equanimità. Oltre non c'è nulla,
questo è il limite.

La mente deve procedere in questo modo quando sta sviluppando gli stadi
del \emph{samādhi}, però -- vi prego -- intendiamoci bene sugli elementi
basilari della pratica. Vogliamo calmare la mente, ma essa non si
calmerà. Questo significa praticare con desiderio, ma non lo
comprendiamo. Desideriamo la calma. La mente è già disturbata, e noi
complichiamo ulteriormente le cose volendo renderla calma. Proprio
questo desiderio è la causa. Non comprendiamo che questo desiderio di
calmare la mente è \emph{tanhā}. Significa solo appesantire il fardello.
Più desideriamo calmare la mente più essa si turba, fino a che non
rinunciamo. Finiamo per combattere per tutto il tempo, ci sediamo e
combattiamo con noi stessi.

Perché? Perché non abbiamo riflettuto su come ci siamo predisposti
mentalmente. Le condizioni della mente sono semplicemente così come
sono. Qualsiasi cosa sorga, osservatela e basta. È solo la natura della
mente; non è nociva, sempre che se ne comprenda la natura. Non è
pericolosa, se vediamo la sua attività per quello che è. Per questo
pratichiamo con \emph{vitakka} e \emph{vicāra}, finché la mente inizia a
calmarsi e diviene meno impetuosa. Quando le sensazioni sorgono, noi le
contempliamo, ci mescoliamo con esse e giungiamo a conoscerle.
Ovviamente, di solito all'inizio si tende a combatterle, perché fin dal
principio siamo decisi a voler calmare la mente. Appena ci sediamo, i
pensieri arrivano a disturbarci. Appena scegliamo il nostro oggetto di
meditazione, la nostra attenzione vaga, la mente vaga seguendo ogni
pensiero, e pensa che tutti quei pensieri siano arrivati per
disturbarci, ma in effetti è proprio lì che sorge il problema, proprio
dal desiderio di calmare la mente.

Se comprendiamo che la mente si sta solo comportando in accordo con la
sua natura, che naturalmente va e viene in questo modo, e non ce ne
interessiamo più di tanto, possiamo capire che i suoi modi di essere
sono molto simili a quelli di un bambino. I bambini non sanno di
sbagliare, possono dire ogni genere di cose. Se li comprendiamo, allora
li lasciamo parlare, perché i bambini parlano naturalmente così. Quando
lasciamo andare in questo modo, non siamo ossessionati dai bambini.
Possiamo parlare indisturbati con i nostri ospiti, mentre il bimbo
chiacchiera e gioca. La mente è così. Non è dannosa a meno che non ci
aggrappiamo a essa e ne siamo ossessionati. Questa è la vera causa del
problema.

Quando sorge \emph{pīti}, si prova un piacere indescrivibile; possono
rendersene conto solo coloro che l'hanno sperimentato. Sorge
\emph{sukha}, il piacere, ed è presente pure la qualità
dell'unificazione mentale. Ci sono \emph{vitakka}, \emph{vicāra},
\emph{pīti}, \emph{sukha} ed \emph{ekaggatā}, possiamo vederle tutte
insieme nella mente, tutte e cinque queste qualità. Sarebbe difficile
rispondere se dovessero chiedere: «~Com'è quando \emph{vitakka} è lì,
com'è quando \emph{vicāra} è lì, com'è quando \emph{pīti} e \emph{sukha}
sono lì?~» Però, quando tutte queste qualità convergono nella mente,
com'è lo vedremo da noi stessi.

A questo punto la nostra pratica diventa qualcosa di particolare. Per
non perderci dobbiamo avere rammemorazione e consapevolezza di noi
stessi. Conoscere le cose per quello che sono. Questi sono stati
meditativi, potenzialità della mente. Non abbiate alcun dubbio a
riguardo della pratica. Anche se sprofondate nella terra o volate in
aria, o perfino se ``morite'' mentre sedete, non dubitatene. Quali che
siano le qualità della mente, restate solo con la conoscenza di esse.
Questo è il nostro fondamento: avere \emph{sati}, rammemorazione, e
\emph{sampajañña}, consapevolezza di sé, in piedi, camminando, seduti o
distesi. Qualsiasi cosa sorga, lasciate che sia, non attaccatevi a essa.
Che vi piaccia o che vi dispiaccia, felicità o sofferenza, dubbio o
certezza, contemplate con \emph{vicāra} e valutate i risultati di quelle
qualità. Non cercate di etichettare ogni cosa, conoscete solamente.
Notate che tutte le cose che sorgono nella mente sono solo sensazioni.
Sono transitorie. Sorgono, esistono e cessano. È tutto quel che c'è in
esse, non hanno alcun sé o sostanza, non sono ``noi'' né ``loro''. Non
vale la pena aggrapparsi a nessuna di esse.

Quando comprenderemo tutti i \emph{rūpa} e \emph{nāma} in questo modo
con saggezza, allora vedremo le solite vecchie strade che percorriamo da
sempre. Vedremo la transitorietà della mente, la transitorietà del
corpo, la transitorietà della felicità, della sofferenza, dell'amore e
dell'odio. Sono tutti impermanenti. Vedendolo, la mente prova
stanchezza, diviene stanca del corpo e della mente, stanca delle cose
che sorgono e cessano e della loro transitorietà. Quando la mente sarà
disincantata, cercherà una via d'uscita da tutto questo. Non vorrà più
rimanere bloccata nelle cose, vedrà l'inadeguatezza di questo mondo e
l'inadeguatezza della nascita. Se la mente conosce in questo modo,
ovunque si vada vediamo \emph{aniccā} (transitorietà), \emph{dukkha}
(imperfezione) e \emph{anattā} (non-sé). Non c'è più nulla cui
attaccarsi. Possiamo ascoltare l'insegnamento del Buddha sia seduti ai
piedi di un albero, sia sulla cima di una montagna sia in una valle.
Tutti gli alberi ci sembreranno la stessa cosa, tutti gli esseri saranno
una sola cosa, non ci sarà alcunché di speciale in nessuno di essi.
Sorgono, esistono per un po', invecchiano e poi muoiono, tutti quanti.

In questo modo vedremo il mondo con maggiore chiarezza, vedremo questo
corpo e questa mente con maggiore chiarezza. Risultano più chiari alla
luce della transitorietà, più chiari alla luce dell'imperfezione e più
chiari alla luce del non-sé. Se la gente si attacca alle cose, soffre. È
così che sorge la sofferenza. Se comprendiamo che il corpo e la mente
sono semplicemente nel modo in cui sono, la sofferenza non sorge, perché
non ci attacchiamo a essi. Ovunque andremo avremo saggezza. Perfino
quando vediamo un albero lo possiamo guardare con saggezza. Guardare
l'erba e i vari insetti nutrirà la riflessione. Quando tutto si riduce a
questo, ogni cosa va a finire dentro la stessa barca. Tutto è Dhamma,
tutto è invariabilmente transitorio. Questa è la Verità, questo è il
vero Dhamma, questo è certo. Com'è che è certo? È certo nel senso che il
mondo è in quel modo e non può essere altrimenti. In esso non c'è nulla
di più di questo. Se riusciamo a vedere in questo modo, il nostro
viaggio è terminato.

Nel buddhismo si dice che pensare di essere più stolti degli altri non è
giusto, pensare di essere uguali agli altri non è giusto e pensare di
essere meglio degli altri non è giusto: perché non c'è alcun ``noi''. È
così, dobbiamo sradicare la presunzione. Questo si chiama
\emph{lokavidū},\footnote{\emph{Lokavidū}: ``Conoscitore del mondo'', un
  epiteto del Buddha.} conoscere con chiarezza il mondo così com'è. Se
vediamo la Verità, la mente conoscerà del tutto se stessa e si
disgiungerà dalla causa della sofferenza. Quando non c'è più alcuna
causa, gli effetti non possono sorgere. Questo è il modo in cui dovrebbe
procedere la nostra pratica.

Le cose fondamentali che abbiamo necessità di sviluppare sono queste:
primo, essere retti e onesti; secondo, astenersi dalle azioni illecite;
terzo, avere nel nostro cuore la qualità dell'umiltà, essere distaccati
e accontentarci di poco. Se ci accontentiamo di poco in relazione alla
parola e a tutte le altre cose, vedremo noi stessi, non saremo attratti
dalle distrazioni. La mente sarà fondata in \emph{sīla}, \emph{samādhi}
e \emph{paññā}. È per questa ragione che i praticanti del Sentiero non
dovrebbero essere distratti. Anche se avete ragione, non siate
distratti. E se siete in torto, non siate distratti. Se le cose stanno
andando bene o se vi sentite felici, non siate distratti. Perché dico:
«~non siate distratti~»? Perché tutte queste cose sono incerte.
Consideratele come tali. Se vi sentite tranquilli, lasciate solo che la
pace sia. Potreste davvero desiderare d'indulgere a essa, ma dovreste
solo conoscerne la verità, che è la stessa delle cose spiacevoli.

Questa pratica della mente dipende da ogni individuo. L'insegnante
spiega solo il modo di addestrare la mente, perché la mente è interna a
ogni individuo. Siamo noi a sapere cosa c'è dentro, nessun altro può
conoscere bene come noi la nostra mente. La pratica richiede questo
genere di onestà. Praticate in modo giusto, non con poca convinzione.
Quando dico «~praticate in modo giusto~», significa che dovete stancarvi
fino a essere esausti? Non dovete arrivare al punto di essere esausti,
perché si pratica con la mente. Se lo sapete, conoscete la pratica. Non
avete bisogno di un sacco di cose. Usate solo le regole basilari della
pratica per riflettere interiormente su voi stessi.

Ora il tempo del Ritiro delle Piogge è per metà trascorso. Per la
maggior parte della gente dopo un po' è normale lasciare che la pratica
si allenti. Non sono costanti dall'inizio alla fine. Ciò indica che la
pratica non è ancora matura. Ad esempio, dopo aver deciso all'inizio del
ritiro di svolgere un certo tipo di pratica, quale che essa sia, si
dovrebbe adempiere a tale impegno. Per questi tre mesi praticate
costantemente. Dovete provarci tutti quanti. Quale che sia la pratica
che avete deciso di svolgere, prendete in considerazione quanto vi ho
detto e riflettete se si è allentata o meno. Se si è allentata, fate uno
sforzo per ripristinarla. Continuate a dare forma alla vostra pratica,
esattamente come quando pratichiamo la meditazione sul respiro. Quando
il respiro entra ed esce la mente si distrae. Allora fissate di nuovo la
vostra attenzione sul respiro. Quando la vostra attenzione vaga di
nuovo, la riportate indietro ancora una volta. È la stessa cosa. Sia in
relazione al corpo sia in relazione alla mente la pratica procede in
questo modo. Fate uno sforzo, per favore.

