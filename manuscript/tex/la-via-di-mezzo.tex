\chapter{La Via di Mezzo dentro di noi}

\begin{openingQuote}
  \centering

  Il discorso fu pronunciato nel 1970 per una riunione di monaci e
  laici, nel dialetto del nord-est della Thailandia.
\end{openingQuote}

Il buddhismo insegna ad abbandonare il male e a praticare il bene. Poi,
quando il male è stato abbandonato e il bene è impiantato, dobbiamo
lasciar andare sia il bene sia il male. Abbiamo già ascoltato abbastanza
sugli stati mentali salutari e su quelli non salutari per capire
qualcosa in proposito. Perciò vorrei parlare della Via di Mezzo, ossia
del sentiero per trascenderli entrambi. Tutti i discorsi di Dhamma e gli
insegnamenti del Buddha hanno un obbiettivo: mostrare la via per uscire
dalla sofferenza a coloro che non sono ancora riusciti ad abbandonarla.
Gli insegnamenti servono a darci la Retta Comprensione. Se non
comprendiamo rettamente, non possiamo giungere alla pace.

Tutti i Buddha, quando divennero illuminati e offrirono i loro primi
insegnamenti, evidenziarono questi due estremi: l'indulgenza al piacere
e l'indulgenza al dolore. Tali due tipi d'infatuazione rappresentano i
due poli opposti tra i quali sono costretti a oscillare, senza mai avere
pace, coloro che indulgono ai piaceri sensoriali. Si tratta delle vie
che ruotano intorno al \emph{saṃsāra}.\footnote{%
  \emph{Saṃsāra}: Flusso del Divenire o dell'Esistenza; un vagare
  perpetuo, il continuo processo del nascere, invecchiare e morire.}
L'Illuminato osservò che tutti gli esseri non riescono
mai a vedere la Via di Mezzo del Dhamma perché sono bloccati in questi
due estremi e, perciò, li evidenziò per mostrare la sofferenza che
entrambi implicano. Siccome siamo ancora bloccati poiché soggetti alla
volizione, viviamo continuamente sotto il loro dominio. Il Buddha
dichiarò che tali due vie non sono quelle del meditante, della pace, ma
le vie dell'intossicazione. Queste vie sono l'indulgenza al piacere e
l'indulgenza al dolore, o, per dirla con semplicità, la via
dell'indolenza e la via della tensione.

Se momento dopo momento investigate dentro di voi, vedrete che la via
della tensione è la collera, la via del dolore. Percorrere questa via
porta solo difficoltà e disagio. Se avete trasceso l'indulgenza al
piacere, ciò significa che avete trasceso la felicità. Felicità e
infelicità non sono stati mentali sereni. Il Buddha insegnò a lasciarli
andare entrambi. Questa è retta pratica. È la Via di Mezzo. Queste
parole, ``Via di Mezzo'', non si riferiscono al nostro corpo e al nostro
linguaggio, ma alla mente. Quando sorge un'impressione mentale sgradita,
essa influisce sulla mente e vi è confusione. Quando la mente è confusa,
quando è ``scossa'', questa non è la retta via. Quando sorge
un'impressione mentale gradita e la mente indulge al piacere, neanche
questa è la via.

Non vogliamo soffrire, vogliamo la felicità. Nei fatti, però, la
felicità non è altro che una sottile forma di sofferenza. La sofferenza
stessa è la forma grossolana. Potete paragonarle a un serpente. La testa
del serpente è l'infelicità, la coda del serpente è la felicità. La
testa è davvero pericolosa, ha denti velenosi. Se la toccate, il
serpente vi morderà immediatamente. Anche se lasciamo perdere la testa e
ci aggrappiamo alla coda, esso si volgerà e ci morderà ugualmente,
perché sia la testa sia la coda sono parte del serpente.

Allo stesso modo, tanto la felicità quanto l'infelicità, o il piacere e
la tristezza, sorgono dallo stesso genitore: la volizione. Così, quando
siamo felici la mente non è serena. Non lo è davvero! Quando ad esempio
otteniamo quel che ci piace, come la ricchezza, il prestigio, la lode o
la felicità, il risultato è che siamo soddisfatti. Nella mente alberga
però ancora qualche disagio, perché abbiamo timore di perderlo. Proprio
questo timore è una condizione di non serenità. In seguito potremmo
veramente perdere quella cosa e, allora, soffrire davvero. Perciò, se
non siete consapevoli, la sofferenza è imminente anche se siete felici.
È proprio come afferrare la coda del serpente: se non la lasciate andare
vi morderà. Che si tratti della coda del serpente o della sua testa,
ossia di condizioni piacevoli o spiacevoli, esse sono solo
caratteristiche della ``Ruota dell'Esistenza'', del cambiamento senza
fine.

Il Buddha stabilì che moralità, concentrazione e saggezza rappresentano
il sentiero per la pace, la via per l'Illuminazione. In verità, però,
queste cose non sono l'essenza del buddhismo. Sono solo il Sentiero. Il
Buddha le chiamò \emph{magga}, che significa ``Sentiero''. L'essenza del
buddhismo è la pace, quella pace che sorge dal conoscere veramente la
natura di tutte le cose. Se investighiamo attentamente, possiamo vedere
che la pace non è né la felicità né l'infelicità. Né l'una né l'altra
sono la verità.

La mente umana, la mente che il Buddha ci esortò a conoscere e
investigare, è una cosa che possiamo conoscere solamente dalla sua
attività. Non vi è niente che possa definire la vera ``mente
originaria'', non vi è niente che possiate utilizzare per conoscerla. Il
suo stato naturale è saldo, immobile. Quando sorge la felicità, succede
solo che questa mente si smarrisce in un'impressione mentale, vi è
movimento. Quando la mente si muove in questo modo, l'aggrapparsi e
l'attaccarsi alle cose giunge in essere.

Il Buddha ha già impostato il Sentiero della pratica nella sua
interezza, ma noi non lo abbiamo ancora praticato o, se lo abbiamo
praticato, lo abbiamo fatto solo a parole. La nostra mente e le nostre
parole non sono in armonia, indulgiamo solo a vuoti discorsi. Il
fondamento del buddhismo non è una cosa opinabile, di cui si possa
parlare. Il fondamento del buddhismo è la conoscenza completa della vera
realtà. Quando si conosce questa verità, allora non è necessario alcun
insegnamento. Se non la si conosce, anche se si ascolta l'insegnamento
non lo si sente davvero. Questa è la ragione per cui il Buddha disse:
«~l'Illuminato indica solo la via.~» Egli non può praticare per voi,
perché la Verità è un qualcosa che non si può esprimere a parole o
regalare.

Tutti gli insegnamenti sono solo similitudini e paragoni, mezzi per
aiutare la mente a vedere la Verità. Se la Verità non l'abbiamo vista,
allora siamo costretti a soffrire. Ad esempio, per far riferimento al
corpo utilizziamo comunemente il termine \emph{saṅkhāra}.\footnote{\emph{Saṅkhāra}:
  Formazione, fenomeno condizionato.} Chiunque può pronunciarlo, ma nei
fatti abbiamo dei problemi solo perché ignoriamo la verità di questi
\emph{saṅkhāra} e, perciò, ci aggrappiamo a essi. Siccome non conosciamo
la verità del corpo, soffriamo.

Facciamo un esempio. Supponiamo che un giorno stiate camminando per
recarvi al lavoro e, dall'altra parte della strada, un uomo vi insulti,
urlando parole ingiuriose. Non appena le sentite, la vostra mente
cambia, il suo stato non è più quello solito. Non vi sentite molto bene,
siete arrabbiati e feriti. Quell'uomo se ne va in giro insultandovi di
notte e di giorno. Ogni volta che sentite questi insulti provate rabbia
e, anche quando tornate a casa, siete ancora arrabbiati poiché volete
vendicarvi, rivalervi. Pochi giorni dopo, un altro uomo si avvicina alla
vostra casa e vi dice: «~Ehi! Quella persona che t'insulta è un folle,
un matto! Lo è da anni! Insulta tutti in quel modo. Nessuno tiene conto
di quel che dice.~» Non appena sentite queste cose, vi sentite
immediatamente sollevati. La rabbia e il dispiacere che avete covato per
tutti quei giorni svaniscono completamente. Perché? Perché ora conoscete
la verità. Prima non la conoscevate, pensavate che quell'uomo fosse
normale e, così, eravate in collera con lui. Pensare in quel modo vi
aveva fatto soffrire. Non appena scoprite la verità, tutto cambia. «~Oh,
è matto! Questo spiega ogni cosa!~»

Quando lo capite, vi sentite bene, perché siete voi stessi a conoscere
come stanno le cose. Avendo conosciuto, potete lasciar andare. Se non
conoscete la verità, è proprio lì che vi aggrappate. Quando pensavate
che l'uomo che vi aveva insultato fosse normale, avreste potuto
ucciderlo. Però, quando scoprite la verità, che è matto, vi sentite
molto meglio. Così è la conoscenza della verità. Chi vede il Dhamma ha
un'esperienza simile. Quando attaccamento, avversione e illusione
scompaiono, è in questo stesso modo che scompaiono. Per tutto il tempo
che ignoriamo queste cose, pensiamo: «~Che cosa posso fare? Provo così
tanta avidità e avversione.~» Questa non è chiara conoscenza. È
esattamente come quando pensiamo che un folle sia una persona sensata.
Quando finalmente vediamo che è matto, siamo liberi per sempre dalla
preoccupazione. Nessuno poteva mostrarvelo. Solo quando la mente vede da
sé può sradicare l'attaccamento, abbandonarlo.

Lo stesso avviene con questo corpo che chiamiamo \emph{saṅkhāra}. Benché
il Buddha abbia già spiegato che il corpo in quanto tale non è né
sostanziale né reale, tuttavia non siamo d'accordo e ci aggrappiamo
ostinatamente a esso. Se il corpo potesse parlare, tutti i giorni ci
direbbe: «~Lo sai, non mi possiedi.~» Nei fatti, ce lo dice in
continuazione, ma lo fa nel linguaggio del Dhamma e, perciò, non siamo
in grado di capirlo. Ad esempio, gli organi sensoriali -- l'occhio,
l'orecchio, il naso, la lingua e il corpo -- cambiano in continuazione,
ma non li ho visti nemmeno una volta chiederci il permesso di farlo!
Come quando abbiamo un mal di testa o un dolore allo stomaco: il corpo
non ci chiede mai prima il permesso, va avanti semplicemente, segue il
suo corso naturale. Ciò indica che il corpo non consente a nessuno di
possederlo, non ha un proprietario. Il Buddha lo descrisse come un
oggetto vuoto di sostanza.

Non comprendiamo il Dhamma e perciò non comprendiamo questi
\emph{saṅkhāra}. Li confondiamo con noi stessi, come se appartenessero a
noi o ad altri. Questo fa sorgere l'attaccamento. Quando l'attaccamento
sorge, segue il ``divenire''. Appena sorge il divenire, vi è nascita.
Appena vi è nascita, ecco allora invecchiamento, malattia, morte \ldots{} e
sorge l'intera massa della sofferenza.

Questo è il \emph{paṭiccasamuppāda}.\footnote{%
  \emph{Paṭiccasamuppāda}:
  Coproduzione condizionata, genesi interdipendente. Una tabella che
  descrive il modo in cui i cinque aggregati (\emph{khandhā}) e le sei
  basi sensoriali (\emph{āyatana}) interagiscono dopo il contatto
  (\emph{phassa}) con l'ignoranza (\emph{avijjā}) e con la brama
  (\emph{taṇhā}) per condurre alla tensione e alla sofferenza
  (\emph{dukkha}).}
Diciamo che l'ignoranza fa sorgere le attività
volitive, che esse fanno sorgere la coscienza, e così via. Tutte queste
cose sono solo eventi che accadono nella mente. Quando entriamo in
contatto con qualcosa che non ci piace, se non abbiamo presenza mentale
ecco l'ignoranza. La sofferenza sorge immediatamente. La mente, però,
attraversa questi cambiamenti così rapidamente che non riusciamo a stare
al passo con essi. È come cadere da un albero. Prima che ve ne rendiate
conto -- bum! -- siete a terra! In realtà siete caduti attraverso molti
rami, grandi e piccoli, ma non siete riusciti a contarli, né potevate
ricordarveli dopo che li avevate attraversavati. Solo una caduta, e poi
bum!

Così è il \emph{paṭiccasamuppāda}. Se lo suddividiamo come si fa nelle
Scritture, diciamo che l'ignoranza fa sorgere le attività volitive, le
attività volitive fanno sorgere la coscienza, la coscienza fa sorgere la
mente e la materia, la mente e la materia fanno sorgere le sei basi dei
sensi, le basi dei sensi fanno sorgere il contatto, il contatto fa
sorgere la sensazione, la sensazione fa sorgere la volizione, la
volizione fa sorgere l'attaccamento, l'attaccamento fa sorgere il
divenire, il divenire fa sorgere la nascita, la nascita fa sorgere la
vecchiaia, la malattia, la morte e ogni forma di sofferenza. Quando però
entrate in contatto con qualcosa che non vi piace, la sofferenza è
immediata! Quella sensazione di sofferenza è in realtà il risultato
dell'intera catena del \emph{paṭiccasamuppāda}. Questa è la ragione per
cui il Buddha esortò i suoi discepoli a investigare e conoscere appieno
le loro menti.

Le persone nascono nel mondo prive di nomi e, dopo che sono nate,
attribuiamo a esse dei nomi. È una convenzione. Attribuiamo alle persone
dei nomi per ragioni di utilità, affinché possano chiamarsi a vicenda.
Nelle Scritture avviene lo stesso. Separiamo ogni cosa con etichette per
rendere proficuo lo studio della realtà. Tutte le cose sono, allo stesso
modo, semplicemente \emph{saṅkhāra}. La natura originaria è
semplicemente quella di cose composte. Il Buddha disse che esse sono
impermanenti, insoddisfacenti e non-sé. Sono instabili. Non lo
comprendiamo con saldezza, la nostra comprensione non è retta e così
abbiamo un'errata visione. Questa errata visione è che i \emph{saṅkhāra}
sono noi stessi, noi siamo i \emph{saṅkhāra}, o che la felicità e
l'infelicità sono noi stessi, noi siamo la felicità e l'infelicità.
Vedere in questo modo non è piena, chiara comprensione della vera natura
delle cose. La verità è che non possiamo costringere tutte queste cose a
seguire i nostri desideri, esse seguono la via della natura.

Ecco un semplice paragone. Supponiamo che andiate a sedervi nel mezzo di
una superstrada, con automobili e autocarri che vi vengono contro. Non
potete arrabbiarvi con le automobili e urlare: «~Non passate qui! Non
passate qui!~» È una superstrada, non potete farlo. Che cosa potete
fare, allora? Via dalla strada! La strada è il luogo in cui passano le
automobili e se non volete che le automobili siano lì, soffrite.

Con i \emph{saṅkhāra} è la stessa cosa. Diciamo che ci disturbano, come
quando sediamo in meditazione e sentiamo un rumore. Pensiamo: «~Oh, quel
rumore mi sta disturbando.~» Se pensiamo che il rumore ci stia
disturbando, la conseguenza è che soffriamo. Se investighiamo un po' più
a fondo, vediamo che siamo noi ad andare a disturbare il rumore! Il
rumore è semplicemente rumore. Se comprendiamo questo, non c'è niente di
più nel rumore, lo lasciamo essere. Vediamo che il rumore è una cosa e
noi un'altra. Uno che pensa che il rumore vada a disturbarlo è uno che
non vede se stesso. Davvero non vede se stesso! Non appena vedete voi
stessi, allora siete a vostro agio. Un suono è solo un suono, perché mai
dovreste andare ad afferrarlo? Capite che in realtà eravate voi a uscire
da voi stessi per andare a disturbare il rumore.

Questa è reale conoscenza della verità. Vedete entrambi i lati e, così,
ottenete la pace. Se vedete un lato solo, vi è sofferenza. Non appena
vedete entrambi i lati, allora seguite la Via di Mezzo. Questa è la
retta pratica della mente. È questo che intendiamo quando parliamo di
raddrizzare la nostra comprensione. Allo stesso modo, la natura di tutti
i \emph{saṅkhāra} è impermanenza e morte, ma noi vogliamo afferrarli, ce
li portiamo dietro dappertutto e li bramiamo. Vogliamo che siano veri.
Vogliamo trovare la verità nelle cose che non sono vere. Ogni volta che
qualcuno pensa in questo modo e si attacca ai \emph{saṅkhāra}
identificandosi con essi, soffre.

La pratica del Dhamma non dipende dall'essere monaco, novizio o laico,
dipende dal raddrizzare la vostra comprensione. Se la nostra
comprensione è corretta, giungiamo alla pace. Che siate stati ordinati
monaci o meno è lo stesso, tutti hanno la possibilità di praticare il
Dhamma, di contemplarlo. Contempliamo tutti la stessa cosa. Se ottenete
la pace, è la stessa pace; è lo stesso Sentiero, i metodi sono gli
stessi. Per questo motivo il Buddha non fece differenza tra laici e
monaci, insegnò a tutti a praticare affinché conoscessero la verità dei
\emph{saṅkhāra}. Quando conosciamo questa verità, lasciamo andare. Se
conosciamo la verità, non ci sarà più divenire o nascita. E com'è che
non c'è più nascita? Non vi è più possibilità che la nascita abbia luogo
perché conosciamo appieno la verità dei \emph{saṅkhāra}. Se conosciamo
appieno la verità, allora vi è la pace. Avere o non avere, è lo stesso.
Guadagno e perdita sono una sola cosa. Questo ci insegnò a conoscere il
Buddha. Questa è la pace, una pace senza felicità e senza infelicità,
senza contentezza e senza dolore.

Dobbiamo comprendere che non vi è ragione per nascere. Nascere in che
modo? Nascere alla contentezza: quando otteniamo qualcosa che ci piace,
siamo contenti. Se non c'è attaccamento a quella contentezza, non c'è
nascita. Se vi è attaccamento, questo si chiama ``nascita''. Così, se
otteniamo qualcosa, non nasciamo alla contentezza. Se perdiamo qualcosa,
non nasciamo al dolore. Questo è il senza-nascita e il senza-morte.
Nascita e morte sono entrambe radicate nell'attaccamento e nella
predilezione per i \emph{saṅkhāra}. Per questo il Buddha disse: «~Per me
non vi è più divenire, la vita santa è compiuta, questa è la mia ultima
nascita.~» Ecco! Egli conobbe il non-nascere e il non-morire. Il Buddha
esortò continuamente i suoi discepoli a conoscere proprio questo. Questa
è retta pratica. Se non la conseguite, non conseguite la Via di Mezzo e
allora non trascenderete la sofferenza.

