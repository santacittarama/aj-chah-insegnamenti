\chapter{Soffrire in cammino}

Ai tempi del Buddha viveva un monaco che desiderava ardentemente trovare
la vera via per l'Illuminazione. Voleva sapere con certezza quale fosse
la via giusta e quale fosse la via sbagliata per addestrare la mente
durante la meditazione. Dopo aver deciso che vivere in un monastero con
un gran numero di monaci era fonte di confusione e di distrazioni, se ne
andò alla ricerca di posti tranquilli per meditare da solo. Vivendo da
solo, praticò in continuazione, a volte sperimentando periodi di pace
quando la sua mente si unificava nella concentrazione (\emph{samādhi}),
altre volte senza trovarne affatto. Nella sua meditazione non vi era
ancora nessuna vera certezza. A volte era molto diligente e si impegnava
a fondo, altre volte era pigro. Alla fine fu preda dei dubbi e dello
scetticismo perché mentre cercava il giusto modo di praticare non
otteneva risultati.

Allora in India c'erano molti maestri di meditazione, e al monaco capitò
di sentir parlare di un famoso maestro, ``Ajahn A'', che era molto
conosciuto e aveva fama di essere abile nell'impartire istruzioni per la
meditazione. Il monaco sedette e ci pensò a fondo, e decise che avrebbe
trovato questo famoso maestro e, qualora egli fosse veramente stato a
conoscenza del giusto modo per ottenere l'Illuminazione, si sarebbe
addestrato sotto la sua guida. Dopo aver ricevuto gli insegnamenti, il
monaco tornò di nuovo a meditare da solo e constatò che solo alcuni dei
nuovi insegnamenti concordavano con il suo modo di pensare, mentre altri
no. Constatò di essere ancora continuamente preda di dubbi e di
incertezze. Dopo un po' sentì parlare di un altro monaco famoso, ``Ajahn
B'', che aveva pure la reputazione di aver conseguito la completa
Illuminazione, oltre che di essere abile nella meditazione. Queste
notizie fecero solo aumentare ancor più i dubbi e le domande che aveva
in mente. Queste riflessioni, infine, lo spinsero ad andare alla ricerca
di questo nuovo insegnante. Fresco di insegnamenti, il monaco ancora una
volta se ne andò per praticare e meditare in solitudine. Paragonò tutti
gli insegnamenti che aveva ricevuto dall'ultimo insegnante con quelli
del primo, e constatò che non erano uguali. Confrontò i differenti modi
di essere e i comportamenti di ogni insegnante, e constatò anche in
questo caso che erano molto diversi. Confrontò tutto quel che aveva
imparato con i suoi modi di vedere a proposito della meditazione, e
constatò che non sembravano affatto adattarsi gli uni agli altri! Più li
paragonava, più dubitava.

Non molto tempo dopo, il monaco sentì voci concitate a proposito di
``Ajahn C'', un insegnante davvero saggio. La gente parlava così tanto
di questo insegnante che si sentì obbligato a cercarlo. Il monaco
desiderava ascoltare e mettere alla prova tutto quello che quest'altro
nuovo maestro gli avrebbe detto. Alcune delle cose che insegnava erano
uguali a quelle dei precedenti insegnanti, altre no. Il monaco continuò
a pensare e a confrontare, cercando di capire la ragione per cui un
maestro faceva le cose in una certa maniera e un altro le faceva in
un'altra. Nella sua mente rimuginava su tutte le informazioni accumulate
sui diversi punti di vista e modi di fare, ma quando le metteva insieme
con le sue idee, che erano del tutto differenti, finiva per non avere
alcun \emph{samādhi}. Più cercava di comprendere quel che ogni
insegnante faceva, più diventava inquieto e agitato. Consumò tutte le
sue energie e, esausto sia mentalmente sia fisicamente, fu completamente
sconfitto dal suo stesso continuo dubitare e speculare.

Più in là corse voce che nel mondo era apparso un insegnante
compiutamente illuminato di nome Gotama. La mente del monaco fu subito
sopraffatta dalla notizia e, ancor più velocemente del solito, si mise a
correre e a far ipotesi su questo maestro. Esattamente come le altre
volte, non riuscì a resistere al bisogno di vedere anche questo nuovo
insegnante e, così, andò a porgergli omaggio e ad ascoltarlo. Gotama il
Buddha espose il Dhamma, spiegando che in definitiva non è possibile
ottenere vera comprensione e trascendere il dubbio solo andando alla
ricerca e ricevendo gli insegnamenti dagli altri. Più si ascolta, più si
dubita. Più si ascolta, più ci si confonde. Il Buddha sottolineò che la
saggezza degli altri non può eliminare i nostri dubbi. Gli altri non
possono lasciar andare il dubbio al posto nostro. Tutto quello che un
insegnante può fare è spiegare il modo in cui i dubbi sorgono nella
mente e come riflettere su di essi, ma è a noi che tocca mettere le sue
parole in pratica, fino a quando si ottiene la visione profonda e si
conoscono le cose da se stessi. Il Buddha insegnò che il luogo della
pratica è dentro il nostro corpo. Forma, sensazione, percezione,
pensiero e coscienza sensoriale -- i cinque \emph{khandhā} -- sono i
nostri insegnanti. Già essi ci forniscono la base per la visione
profonda. Quel che ancora manca è fondarsi nella coltivazione mentale
(\emph{bhāvanā}) e nella saggia riflessione.

Il Buddha insegnò che l'unico modo per porre davvero fine al dubbio
consiste nella contemplazione del corpo e della mente: ``tutto qui''.
Abbandonate il passato, abbandonate il futuro, praticate la conoscenza e
il lasciar andare. Sostenete la conoscenza. Quando si è instaurata la
conoscenza, lasciate andare, ma non cercate di lasciar andare senza la
conoscenza. È la presenza di questa conoscenza che vi consente di
lasciar andare. Lasciate andare tutto quel che avete fatto in passato,
sia in bene che in male. Qualsiasi cosa abbiate fatto prima, lasciatela
andare, perché attaccarsi al passato non è di alcun beneficio. Il bene
che avete fatto era bene allora, il male che avete fatto era male
allora. Quello che era giusto era giusto. Ora potete mettere tutto da
parte, lasciatelo andare. Per accadere, gli eventi del futuro sono
ancora in attesa. Tutto quello che sorgerà e che cesserà in futuro non
ha ancora avuto luogo, perciò non attaccatevi troppo saldamente ai
pensieri su quel che potrà o non potrà succedere nel futuro. Siate
consapevoli di voi stessi e lasciate andare. Lasciate andare il passato.
Tutto quello che s'è verificato in passato è cessato. Perché trascorrere
molto tempo ad arrovellarsi? Se pensate a qualcosa che è avvenuto in
passato, lasciate andare anche i pensieri. Si tratta di un
\emph{dhamma}, di un fenomeno che è sorto nel passato. Dopo essere sorto
è cessato nel passato. Non ci sono ragioni nemmeno perché la mente
proliferi sul presente. Quando si è stabilita in voi la consapevolezza
di quel che state pensando, lasciate andare. Praticate il conoscere e il
lasciar andare.

Non è che non si dovrebbe sperimentare alcun pensiero né avere opinioni:
sperimentate pensieri e opinioni, e poi lasciateli andare, perché essi
sono già compiuti. Il futuro sta ancora davanti a voi, qualsiasi cosa
stia per sorgere nel futuro avrà anche termine nel futuro. Siate
consapevoli dei vostri pensieri sul futuro, e poi lasciateli andare.
Allo stesso modo, anche i vostri pensieri e le vostre opinioni in
relazione al passato sono incerti. Il futuro è del tutto incerto. Siate
consapevoli e poi lasciate andare, perché è incerto. Siate consapevoli
del momento presente, investigate quel che state facendo qui e ora. Non
c'è bisogno di guardare nulla che stia al di fuori di voi stessi.

Il Buddha non lodò coloro che investono tutta la loro fiducia e tutta la
loro fede in quello che dicono gli altri. Nemmeno lodò coloro che
restano intrappolati negli stati mentali positivi o in quelli negativi a
causa di quello che gli altri dicono e fanno. Quello che gli altri
dicono e fanno deve essere affar loro. Ne potete essere consapevoli, ma
poi lasciate andare. Anche se fanno una cosa giusta, capite che è giusta
per loro, ma se non allineate la vostra mente alla Retta Visione
(\emph{sammā-diṭṭhi}) non potrete mai sperimentare quello che è buono e
giusto per voi stessi, resterà tutto un qualcosa di esteriore. Tutti
quegli insegnanti stanno effettuando la loro pratica, giusta o sbagliata
che sia, da qualche altra parte, lontani da voi. Qualsiasi buona pratica
svolgano, non è in verità essa a cambiarvi. Se si tratta di pratica
corretta, è corretta per loro, non per voi. Tutto questo significa che
il Buddha insegnò che chi non riesce a coltivare da sé la mente e a
ottenere la visione profonda nella Verità, non merita di essere lodato.

Questo è l'insegnamento da sottolineare: il Dhamma è \emph{opanayiko},
deve essere ricondotto all'interno di noi stessi, in modo che la mente
conosca, comprenda e sperimenti i risultati dell'addestramento
all'interno di se stessa. Se la gente dice che state facendo la
meditazione in modo corretto, non credeteci subito e, allo stesso modo,
se dice che la state facendo in modo errato, non limitatevi ad accettare
ciò che dice finché non avete praticato davvero e non lo avete capito da
voi stessi. Anche altri insegnano il modo corretto che conduce
all'Illuminazione: si tratta pur sempre delle parole altrui. Dovete
prendere i loro insegnamenti e applicarli fino a quando sperimentate i
risultati da voi stessi proprio qui e ora. Questo significa che dovete
diventare testimoni di voi stessi, dovete essere in grado di confermare
gli effetti degli insegnamenti dall'interno della vostra stessa mente.

È come avviene per esempio con un frutto aspro. Immaginate che io vi
dica che un frutto sia aspro e che vi inviti ad assaggiarlo. Potreste
morderlo un po' per provarne l'asprezza. Alcuni mi prenderebbero
volentieri in parola se io dicessi loro che il frutto è aspro, ma se si
limitassero a credere che è aspro senza nemmeno assaggiarlo, quella
convinzione sarebbe inutile (\emph{mogha}), non avrebbe alcun reale
valore o significato. Se diceste che il frutto è aspro, significherebbe
che state solo facendo affidamento sulla mia percezione, tutto qui. Il
Buddha non lodò questa convinzione. Però, non dovreste neanche
rifiutarla solamente: investigatela. Voi stessi dovete assaggiare il
frutto, e sperimentandone nella realtà il sapore aspro diventate i
vostri testimoni interiori. Se qualcuno dice che è aspro, scoprite se è
davvero aspro o no mangiandolo. È come avere una doppia certezza: fate
affidamento sulla vostra stessa esperienza e anche su quel che gli altri
dicono. In questo modo potete veramente confidare nell'autenticità del
suo sapore aspro, avete un testimone che ne attesta la verità.

Il venerabile Ajahn Mun definì questo testimone interiore
\emph{sakkhibhūto}, un testimone diretto che sorge nella mente.
L'autenticità di qualsiasi conoscenza acquisita solo da quello che gli
altri dicono è priva di fondamento, è solo una verità provata da qualcun
altro: avete solo le parole di qualcun altro per affermare che il frutto
è aspro. Potreste dire che si tratta di una mezza verità, al cinquanta
per cento. Se però assaggiate il frutto e lo trovate aspro, siamo al
cento per cento, la verità è completa: ne avete la prova sia da quello
che dicono gli altri sia dalla vostra esperienza diretta. Questa è
verità completamente fondata al cento per cento. Questo è
\emph{sakkhibhūto,} il testimone interiore è sorto dentro di voi.

Per questa ragione il modo di addestrarsi è \emph{opanayiko}. Si dirige
l'attenzione verso l'interiorità, fino a che la visione profonda e la
comprensione divengono \emph{paccattaṃ}. La comprensione ottenuta
dall'ascolto e dall'osservazione degli altri è superficiale se
paragonata con la comprensione profonda che è \emph{paccattaṃ}, resta
all'esterno di \emph{paccattaṃ}. Una conoscenza di questo genere non
sorge dall'esame di se stessi, non è la vostra visione profonda, è
quella degli altri. Ciò non significa che si dovrebbe essere noncuranti
e sprezzanti nei riguardi di ogni insegnamento che ricevete da altre
fonti, ma che essi dovrebbero diventare oggetto di studio e di
investigazione. La prima volta che nei libri vi imbattete in qualche
aspetto dell'insegnamento e iniziate a comprenderlo, va bene crederci
fino a un certo punto, ma nello stesso tempo riconoscete di non aver
ancora addestrato la mente e sviluppato quella conoscenza per mezzo
della vostra stessa esperienza. È per questa ragione che non avete
ancora completamente sperimentato i benefici dell'insegnamento. È come
se il genuino valore della vostra comprensione fosse completo solo a
metà. È per questo che dovete coltivare la mente e consentire alla
vostra visione profonda di maturare, finché non penetrerete del tutto la
Verità. In questo modo la vostra conoscenza diventerà del tutto
compiuta. È allora che si va al di là del dubbio. Quando si ha una
completa visione profonda della Verità dall'interno della propria mente,
ogni incertezza a proposito della via per l'Illuminazione scompare del
tutto.

Quando diciamo di praticare con \emph{paccuppanna dhamma}, ciò significa
che qualsiasi fenomeno sorga all'improvviso nella mente lo dobbiamo
subito investigare e affrontare. La vostra consapevolezza deve essere
proprio lì. Siccome \emph{paccuppanna dhamma} è riferito all'esperienza
del momento presente, esso comprende sia la causa che l'effetto. Il
momento presente è saldamente radicato all'interno del processo di causa
ed effetto. Il modo in cui siete nel presente riflette le cause che
stanno nel passato, la vostra attuale esperienza ne è il risultato. Ogni
singola esperienza che avete avuto fino al momento presente è sorta da
cause passate. Si potrebbe ad esempio dire che uscire dalla vostra
capanna per la meditazione è una causa, e stare qui seduti è l'effetto.
Questa è la verità del modo in cui le cose sono, c'è una costante
successione di cause e di effetti. Così, quel che avete fatto in passato
è la causa, l'esperienza presente è il risultato. Allo stesso modo le
azioni del presente sono la causa per quello che sperimenterete nel
futuro. Stando seduti qui, proprio ora, state generando delle cause!
Cause passate vengono fruite nel presente, e questi risultati generano
cause che produrranno effetti nel futuro.

Quello che il Buddha comprese fu che si deve abbandonare sia il passato
sia il futuro. Quando diciamo ``abbandonare'', ciò non significa che
dovete letteralmente sbarazzarvene. Abbandonare significa focalizzare la
vostra consapevolezza e la vostra visione profonda proprio su questo
punto qui, ora, nel momento presente. Proprio qui si saldano assieme
passato e futuro. Il presente è sia il risultato del passato sia la
causa di quel che si trova più avanti, nel futuro. Dovreste perciò
abbandonare tanto la causa quanto l'effetto, e semplicemente dimorare
nel momento presente. Diciamo di abbandonarli, ma si tratta solo di
parole utilizzate per descrivere il modo di addestrare la mente. Sebbene
possiate lasciar andare il vostro attaccamento e abbandonare il passato
e il futuro, il naturale processo di causa ed effetto resta. Lo si
potrebbe definire come un punto a mezza strada, fa già parte del
processo di causa ed effetto. Il Buddha insegnò a osservare il momento
presente, nel quale si vede il continuo processo del sorgere e dello
svanire, seguito da ulteriore sorgere e svanire.

Qualsiasi cosa sorga nel momento presente è impermanente. Lo dico
spesso, ma la maggior parte delle persone non presta molta attenzione.
Sono riluttanti a usare questo semplice e piccolo insegnamento. Tutto
quel che è soggetto a sorgere è impermanente. È incerto. Questo è
veramente il modo più facile e in assoluto il meno complicato per
riflettere sulla Verità. Se non meditate su questo insegnamento, quando
le cose iniziano effettivamente a mostrarsi come incerte e mutevoli, non
sapete come rispondere con saggezza e avete la tendenza ad alterarvi e
agitarvi. Proprio l'investigazione di questa impermanenza vi conduce
alla visione profonda e alla comprensione di quello che è permanente.
Contemplando quel che è incerto, vedete ciò che è certo. Questo è il
modo in cui dovete spiegarlo per far capire alla gente la Verità. Però
c'è la tendenza a non capire e a trascorrere la maggior parte del tempo
smarriti, correndo di qua e di là. Se volete davvero sperimentare la
pace vera, dovete condurre la mente al punto in cui essa diviene del
tutto consapevole del momento presente. Che lì, nella mente, sorga
felicità o sofferenza, insegnate a voi stessi che si tratta di cose
transitorie. La parte della mente la quale rammemora che felicità e
sofferenza sono impermanenti è la saggezza del Buddha che sta dentro
ognuno di voi. Colui che riconosce l'incertezza dei fenomeni è il Dhamma
che sta dentro di voi.

Ciò che è Dhamma è il Buddha, ma la maggior parte delle persone non lo
capisce. Vedono il Dhamma là fuori, da qualche parte, come qualcosa di
esteriore, e il Buddha qui, come un'altra cosa. Se l'occhio della mente
vede tutti i fenomeni condizionati come incerti, allora tutti i problemi
che sorgono dall'attaccarsi alle cose e dall'attribuire a esse
un'importanza eccessiva scompariranno. In qualsiasi modo la guardiate,
questa Verità intrinseca è l'unica cosa davvero certa. Quando lo capite,
la mente invece di aggrapparsi e di attaccarsi, lascia andare. La causa
del problema -- l'attaccamento -- scompare, facendo sì che la mente
penetri la Verità e si fonda con il Dhamma. Non c'è niente di più
elevato o di più profondo da cercare che non sia la realizzazione di
questa Verità. In questo modo il Dhamma è uguale al Buddha, il Buddha è
uguale al Dhamma.

Questo insegnamento che tutti i fenomeni condizionati sono incerti e
soggetti al cambiamento è il Dhamma. Il Dhamma è l'essenza del Buddha,
non è nient'altro. Lo scopo della coltivazione della consapevolezza per
mezzo della continua recitazione di \emph{Buddho}, \emph{Buddho} --
``Colui che Conosce'' -- è vedere questa Verità. Quando la mente si
unifica per mezzo della recitazione di \emph{Buddho} viene supportato lo
sviluppo della visione profonda nelle Tre Caratteristiche
dell'impermanenza (\emph{aniccā}), della sofferenza (\emph{dukkha}) e
del non-sé (\emph{anattā}), e il chiarore della consapevolezza conduce a
vedere le cose come incerte e mutevoli. Se vedete con chiarezza e
direttamente, la mente lascia andare. Perciò, quando sperimentate un
qualsiasi genere di felicità, sapete che è incerta, e quando
sperimentate un qualsiasi genere di sofferenza, sapete che è incerta
allo stesso modo. Se andate a vivere da qualche parte sperando che sarà
meglio di dove già vi trovate, ricordate che non è sicuro che troviate
davvero quello che state cercando. Se pensate che la cosa migliore sia
stare qui, di nuovo, non è sicuro. Proprio questo è il punto. Con la
visione profonda, vedete che tutto è incerto, e perciò ovunque andiate a
praticare non dovrete soffrire. Quando volete stare qui, ci state.
Quando volete andare da qualche altra parte, andate senza crearvi
problemi. Ha termine tutto quel dubitare e vacillare a proposito di cosa
sia giusto fare. È il modo di addestrarsi fissando la consapevolezza
unicamente sul momento presente che pone fine ai dubbi.

Non preoccupatevi perciò del passato o del futuro. Il passato è già
cessato. Qualsiasi cosa sia avvenuta in passato ha già avuto luogo, è
andata, finita. Qualsiasi cosa stia per sorgere nel futuro finirà pure
nel futuro, lasciate andare anche questo. Perché preoccuparsene?
Osservate i fenomeni (\emph{dhamma}) che sorgono nel momento presente e
notate come sono mutevoli e inaffidabili. Quando \emph{Buddho} maturerà
e penetrerà più a fondo, otterrete una più profonda consapevolezza
dell'essenziale Verità che tutti i fenomeni condizionati sono per natura
impermanenti. È qui che la visione profonda diviene più intensa e
consente alla stabilità e alla tranquillità del \emph{samādhi} di
rafforzarsi e diventare più raffinata.

\emph{Samādhi} significa mente ferma e stabile, o mente calma. Ce ne
sono due tipi. Un tipo di calma proviene dalla pratica effettuata in un
luogo tranquillo, dove non ci sono immagini, suoni o altri impatti
sensoriali a disturbarvi. La mente che ha questo tipo di calma non è
ancora libera dalle contaminazioni (\emph{kilesa}).\footnote{\emph{kilesa}.
  Contaminazione; inquinante mentale; fattore mentale che oscura e
  contamina la mente.} Le contaminazioni sovrastano ancora la mente, ma
quando c'è calma durante il \emph{samādhi} sono sopite. È come l'acqua
stagnante, che è momentaneamente limpida quando tutto lo sporco e le
particelle di polvere si sono assestate sul fondo. Fino a quando i
sedimenti non vengono smossi, l'acqua resta limpida, ma appena qualcosa
la smuove, lo sporco torna su e l'acqua diventa di nuovo torbida. A voi
succede proprio la stessa cosa. Quando sentite un suono, vedete
un'immagine oppure la mente viene toccata da uno stato mentale, una
reazione di rifiuto rannuvola la mente. Se l'avversione non viene
stimolata vi sentite a vostro agio. Quel sentirsi a proprio agio
proviene però dall'attaccamento e dalle contaminazioni, non dalla
saggezza.

Supponiamo ad esempio che vogliate questo registratore. Finché il
desiderio resta inesaudito vi sentite insoddisfatti. Ovviamente, quando
andate fuori a cercarne uno per voi e lo trovate, vi sentite contenti e
soddisfatti, o no? Qualora vi attaccaste alla sensazione di
soddisfazione che è sorta in quanto siete riusciti a ottenere un
registratore, in realtà stareste creando le condizioni per una futura
sofferenza. Creereste condizioni per una futura sofferenza senza esserne
consapevoli. Ciò avviene perché la vostra sensazione di soddisfazione
dipende dal fatto che otteniate un registratore e così, finché non ne
avete uno, sperimentate della sofferenza. Quando acquistate un
registratore siete contenti e soddisfatti. Se però un ladro ve lo
rubasse, quella sensazione di soddisfazione sparirebbe insieme al
registratore, e voi cadreste di nuovo in uno stato di sofferenza. Così
è. Senza un registratore soffrite. Con un registratore siete felici, ma
se per una qualche ragione lo perdete, diventate tristi nuovamente. Va
sempre in questo modo. Questo è ciò che si intende con un \emph{samādhi}
che dipende da condizioni esterne tranquille. È incerto, come la
felicità sperimentata quando ottenete quel che volete. Quando alla fine
avete il registratore che cercavate, vi sentite benissimo. Qual è però
la vera causa di quella sensazione piacevole? Sorge perché il vostro
desiderio è stato soddisfatto. Questo è tutto. È tanto profonda quanto
la felicità che può raggiungere. È una felicità condizionata dalle
contaminazioni che controllano la mente. Di questo non siete nemmeno
consapevoli. In qualsiasi momento può arrivare qualcuno che vi ruba il
registratore e vi fa ricadere nella sofferenza.

Questo tipo di \emph{samādhi} vi garantisce perciò solo una serenità
temporanea. Dovete contemplare la natura della calma che sorge dalla
meditazione di tranquillità (\emph{samatha})\footnote{\emph{samatha}.
  Calma concentrata, tranquillità.} per capire del tutto la verità della
questione. Quel registratore che ottenete o qualsiasi altra cosa
possediate è destinata a deteriorarsi, a disgregarsi e infine a
scomparire. Avete qualcosa da perdere perché avete ottenuto un
registratore. Se non possedete un registratore, non lo potete perdere.
Nascita e morte sono la stessa cosa. Siccome c'è stata una nascita ci
deve essere l'esperienza della morte. Se non nasce nulla, non c'è nulla
che muore. Tutta quella gente che muore, deve pur essere nata, chi non
nasce non deve morire. Così stanno le cose. Essere in grado di
riflettere in questo modo significa che appena acquistate quel
registratore siete consapevoli della sua impermanenza: un giorno si
romperà o verrà rubato, e alla fine dovrà inevitabilmente cadere a pezzi
e disintegrarsi del tutto. Vedete la verità con saggezza, e comprendete
che la vera natura di quel registratore è impermanente. Se il
registratore si rompe o viene rubato davvero, si tratta solo di
manifestazioni dell'impermanenza. Se riuscite a vedere le cose nel modo
corretto, sarete in grado di usare il registratore senza soffrire.

Potete paragonare tutto questo a quando nella vita laica si avviano
degli affari. Se all'inizio avete bisogno di un prestito dalla banca per
cominciare l'attività, iniziate subito a sentirvi tesi. Soffrite perché
volete i soldi di qualcun altro. Cercare denaro è sia difficile sia
faticoso e così soffrite fino a quando riuscite a racimolarne un po'. Il
giorno che riuscite a ottenere il prestito dalla banca vi sembra
ovviamente di toccare la luna, ma quest'euforia non dura che poche ore,
perché in pochissimo tempo gli interessi sul prestito iniziano a
mangiarsi tutti i vostri guadagni. Con la stessa velocità di quando si
alza un dito, ecco che i vostri soldi sono drenati verso la banca per il
pagamento degli interessi. Non riuscite neanche a crederci! Ed eccovi
lì, seduti, a soffrire di nuovo. Riuscite a capirlo? Perché succede
questo? Quando non avevate denaro, soffrivate. Quando alla fine ne avete
ricevuto un po' pensate che i vostri problemi siano finiti, ma subito
gli interessi iniziano a mangiarsi i vostri fondi e voi soffrite ancora
di più. Così è.

Il Buddha insegnò che per praticare con queste cose bisogna osservare il
momento presente, e sviluppare la visione profonda nella natura
transitoria del corpo e della mente per vedere la Verità del Dhamma: che
i fenomeni condizionati semplicemente sorgono e svaniscono, nulla di
più. È la natura del corpo e della mente a essere in questo modo, e per
queste ragioni non bisogna attaccarsi o aggrapparsi con saldezza a essi.
Se si ha visione profonda dentro queste cose, il risultato è che sorge
la pace. Questa è una pace che proviene dal lasciar andare le
contaminazioni, sorge assieme al sorgere della saggezza. Che cosa causa
il sorgere della saggezza? Essa proviene dalla contemplazione delle Tre
Caratteristiche dell'impermanenza, della sofferenza e del non-sé, e ciò
conduce alla visione profonda nella Verità del modo in cui sono le cose.
Dovete vedere nella vostra mente la Verità con chiarezza e
inequivocabilmente. È l'unica maniera per ottenere davvero la saggezza.
Ci deve essere chiara visione continuamente. Vedete da voi stessi che
tutti gli oggetti mentali e gli stati mentali (\emph{ārammana}) che
sorgono nella coscienza svaniscono, e che dopo questa cessazione ne
sorgono altri. Dopo questo ulteriore sorgere c'è ulteriore cessazione.
Se ancora avete degli attaccamenti, la sofferenza deve sorgere di
momento in momento. Se invece state lasciando andare, non create alcuna
sofferenza. La mente che vede con chiarezza l'impermanenza dei fenomeni,
questo s'intende con \emph{sakkhibhūto}, il testimone interiore. La
mente è così saldamente assorta nella contemplazione che la visione
profonda si sostiene da sé. È per questo che tutti gli insegnamenti e
tutta la saggezza che ricevete dagli altri possono essere accettati solo
come verità parziali.

Una volta il Buddha tenne un discorso a un gruppo di monaci e poi chiese
al venerabile Sāriputta, che stava ascoltando: «~Sāriputta, credi a
quello che ti ho insegnato?~» E Sāriputta: «~Non ci credo ancora,
\emph{bhante}.~»\footnote{\begin{quote}
  \emph{bhante}. Epiteto, ``venerabile signore''; viene spesso
  utilizzato quando ci si rivolge a un monaco buddhista.
  \end{quote}} Il Buddha fu soddisfatto di questa risposta e continuò:
«~Bene, Sāriputta. Non dovresti credere con troppa facilità ad alcun
insegnamento degli altri. Un saggio deve contemplare tutto quello che
sente con accuratezza prima di accettarlo completamente. Prima di tutto
dovresti portare con te questo insegnamento e contemplarlo.~» Sebbene
avesse ricevuto un insegnamento dal Buddha stesso, il venerabile
Sāriputta non credette immediatamente a ogni parola. Faceva attenzione
al retto modo di addestrare la sua mente, portava gli insegnamenti con
sé al fine di investigarli ulteriormente. L'Insegnamento lo avrebbe
accettato se, dopo aver riflettuto sulla spiegazione della Verità
offerta dal Buddha, avesse constatato che essa stimolava il sorgere
della saggezza e che la visione profonda rasserenava la sua mente e la
unificava con il Dhamma, con la Verità. La comprensione che sorgeva
doveva far sì che il Dhamma si fissasse nella sua mente. Doveva
accordarsi con la Verità del modo in cui le cose sono. Il Buddha insegnò
ai suoi discepoli ad accettare un elemento di Dhamma solo se vedevano
che esso -- in base all'esperienza e alla comprensione sia propria che
altrui, e in linea con il modo in cui sono realmente le cose -- non
poteva essere messo in dubbio.

Alla fine l'importante è solo investigare la Verità. Non c'è bisogno di
guardare molto lontano, basta osservare che cosa sta avvenendo nel
momento presente. Osservate che cosa sta succedendo nella vostra mente.
Lasciate andare il passato. Lasciate andare il futuro. Siate consapevoli
solo del momento presente, e la saggezza sorgerà dall'investigazione e
dal vedere con chiarezza le caratteristiche dell'impermanenza, della
sofferenza e del non-sé. Se state camminando vedete che è impermanente,
se state seduti vedete che è impermanente, se siete distesi vedete che è
impermanente: qualsiasi cosa stiate facendo, queste caratteristiche si
manifestano in continuazione, perché questo è il modo in cui le cose
sono. Non cambia mai. Se coltivate la visione profonda fino a quando la
vostra visione delle cose è completamente e incrollabilmente in linea
con questa Verità, sarete a vostro agio con il mondo.

Porterà davvero serenità andare a vivere lassù, da qualche parte da soli
sulle montagne? Si tratta di un tipo di pace soltanto temporanea. Quando
sperimenterete più volte la fame e il corpo sentirà la mancanza del
nutrimento al quale è abituato, inizierete a stancarvi pure di questa
esperienza. Il corpo urlerà reclamando le sue vitamine, ma la gente che
vive sulle montagne e che vi offre il cibo in elemosina non sa poi molto
della giusta quantità di vitamine necessaria per una dieta equilibrata.
Probabilmente alla fine scenderete dalle montagne e tornerete qui in
monastero. Se vivrete a Bangkok forse vi lamenterete che la gente offre
troppo cibo e che starci è un peso, che comporta un sacco di fastidi, e
magari deciderete che è meglio andare a vivere in solitudine da qualche
parte nella foresta. In verità siete piuttosto sciocchi se pensate che
vivere da soli vi procuri sofferenza. Se pensate che vivere in una
comunità con tanta gente significhi molta sofferenza, siete ugualmente
sciocchi. È come lo sterco di gallina. Se state camminando per conto
vostro e avete con voi dello sterco di gallina, puzza. Se c'è un gruppo
di persone che se ne va in giro con dello sterco di gallina, puzza
ugualmente. Continuare a trascinarsi dietro ciò che è marcio e putrido
può diventare un'abitudine. Questo avviene perché avete ancora errata
visione. Chi però ha Retta Visione, sebbene possa avere assolutamente
ragione quando pensa che vivere in una grande comunità non dia molta
serenità, potrebbe essere comunque in grado di ricavare molta saggezza
dalla sua esperienza.

Per quanto mi concerne, insegnare a un gran numero di monaci, monache e
laici è stato fonte di saggezza. In passato erano pochi i monaci che
vivevano con me. Quando però iniziarono a venirmi a trovare più laici e
la comunità di monaci e monache crebbe, fui esposto a molti più problemi
perché ognuno aveva i propri pensieri, le proprie opinioni ed
esperienze. La mia pazienza, la mia sopportazione e la mia tolleranza
maturarono e si rafforzarono come se fossero condotte fino ai loro
stessi limiti. Se continuate a riflettere, tutte le esperienze di questo
genere vi possono essere di beneficio, ma se non comprendete la Verità
del modo in cui sono le cose, all'inizio potreste pensare che è meglio
vivere da soli. Poi, dopo un po', potreste annoiarvi e pensare che è
meglio vivere in una grande comunità. Oppure, ritenere che l'ideale sia
stare in un posto nel quale viene offerto solo poco cibo. Potreste anche
decidere che la cosa migliore di tutte sia avere a disposizione cibo in
abbondanza e che poco cibo non vada bene affatto, o anche cambiare idea
di nuovo e arrivare alla conclusione che troppo cibo è una brutta cosa.
Alla fine, la maggior parte della gente non fa altro che restare
intrappolata in punti di vista e opinioni perché non ha abbastanza
saggezza per decidere da sé.

Cercate perciò di vedere l'incertezza delle cose. Se vi trovate in una
grande comunità, è incerto. Se state vivendo con poche persone, anche
questa non è una cosa sicura. Non attaccatevi, non aggrappatevi a
opinioni riguardanti il modo in cui sono le cose. Sforzatevi di essere
consapevoli del momento presente. Investigate il corpo, penetrando
all'interno di esso sempre più in profondità. Il Buddha insegnò ai
monaci e alle monache a trovare un posto nel quale si sarebbero sentiti
a proprio agio. E lì, dove il cibo è idoneo, dove si può stare in
compagnia di amici spirituali praticanti (\emph{kalyāṇamitta})\footnote{\emph{kalyāṇamitta}.
  Amico spirituale, maestro che consiglia o insegna il Dhamma.} e
alloggiare adeguatamente, vivere e addestrarsi. Però, è in realtà
difficile trovare un posto nel quale tutte queste cose si realizzino e
si adattino alle nostre necessità. Così, nel contempo, Egli insegnò che
ovunque si vada a vivere è possibile andare incontro a disagi e dover
tollerare cose che non ci piacciono. Ad esempio, quanto è confortevole
questo monastero? Se i laici lo rendessero veramente confortevole, a
cosa somiglierebbe? Se tutti i giorni fossero al vostro servizio per
portarvi bevande calde o fresche a seconda dei vostri desideri, e anche
tutti i dolciumi che riuscite a mangiare. Se fossero sempre gentili e vi
lodassero in continuazione, dicendo solo e sempre cose belle. È questo
che significa avere un buon sostegno da parte dei laici, vero? Ad alcuni
monaci e ad alcune monache piace che così vadano le cose: «~I laici sono
davvero magnifici, qui ci si sente a proprio agio e si sta veramente
bene.~» In pochissimo tempo tutto l'addestramento alla consapevolezza e
alla visione profonda morirebbe. È così che succede.

Quel che è veramente confortevole e adatto alla meditazione può
significare cose differenti per persone diverse, ma quando sapete come
rendere la vostra mente soddisfatta di quel che avete, allora ovunque
andiate siete a vostro agio. Se dovete stare da qualche parte che forse
non corrisponde al luogo che preferite, sapete comunque come esserne
soddisfatti perché è lì che vi addestrate. Se arriva il momento di
andare da qualche altra parte, allora siete contenti di andare. Non
avete alcuna preoccupazione per queste cose esteriori. Se non si sa
molto, le cose possono essere difficili. Se si sa troppo, anche questo
può farvi soffrire molto. Tutto può essere fonte di disagio e di
sofferenza. Finché non avrete visione profonda sarete continuamente
catturati dagli stati mentali di soddisfazione e di insoddisfazione
causati dalle condizioni attorno a voi, e potenzialmente ogni minima
cosa potrà causarvi sofferenza. Ovunque andiate, il significato
dell'insegnamento del Buddha resta corretto, ma è il Dhamma che sta
nella vostra mente a non essere ancora corretto. Dove andrete mai per
trovare le condizioni giuste per praticare? Forse questo o quel monaco
ha capito bene e lavora davvero sodo con la pratica di meditazione, e
appena il pasto è terminato si affretta ad andarsene a meditare. Tutto
quel che fa è praticare per sviluppare il \emph{samādhi}. Si impegna
veramente e con serietà. O forse non così tanto. Non è possibile saperlo
davvero. Se praticate sinceramente e con tutto il cuore, è certo che
raggiungerete la pace mentale. Se gli altri addestrano se stessi
realmente con dedizione e sincerità, perché non sono ancora sereni? Qui
sta la verità della questione. Alla fine, il fatto che non siano sereni
mostra che dopo tutto non praticano poi tanto seriamente.

Quando riflettiamo sull'addestramento nel \emph{samādhi}, è importante
comprendere che virtù (\emph{sīla}), concentrazione (\emph{samādhi}) e
saggezza (\emph{paññā}) sono tutte quante le radici essenziali che
supportano il tutto. Si sostengono a vicenda, e ognuna di esse gioca un
ruolo indispensabile. Sono strumenti necessari per il progresso della
meditazione, ma spetta a ognuno di noi individuare l'abile modo di usare
questi strumenti. Chi ha molta saggezza può ottenere facilmente la
visione profonda. Altri che hanno poca saggezza possono ottenere la
visione profonda con difficoltà. Quelli che di saggezza non ne hanno
affatto non otterranno alcuna visione profonda. Due persone diverse
potrebbero coltivare la mente nello stesso modo, che però riescano o
meno a ottenere la visione profonda nel Dhamma dipende dalla quantità di
saggezza di ognuno di loro. Se andate a osservare maestri diversi e ad
addestrarvi con loro, dovete usare la saggezza per collocare nella
giusta prospettiva quel che vedete. Com'è che fa questo \emph{ajahn}?
Qual è il modo di insegnare di quell'altro \emph{ajahn}? Li osservate da
vicino, ma questo è tutto, non si va oltre. Si tratta solo di osservare
in superficie comportamenti e modi di fare le cose. Se osservate solo a
questo livello non smetterete mai di dubitare. Perché quell'insegnante
fa così? Perché questo insegnante fa cosà? Se in quel monastero
l'insegnante tiene molti discorsi di Dhamma, perché in questo monastero
ne offre così pochi? In quell'altro monastero ancora l'insegnante
addirittura non tiene alcun discorso! Se la mente prolifera senza fine
in paragoni e ipotesi sui vari maestri è solo follia. Finite unicamente
per invischiarvi in un gran pasticcio. Dovete rivolgere la vostra
attenzione verso l'interiorità e coltivare la mente da voi stessi. La
cosa corretta da fare è focalizzarsi interiormente proprio sul vostro
addestramento, perché è così che si sviluppa la Retta Pratica
(\emph{sammā-patipadā}). Osservate i vari maestri e imparate dai loro
esempi, ma poi dovete farlo da voi. Se contemplate a questo livello più
sottile, tutti i dubbi cesseranno.

C'era un monaco anziano che non trascorreva molto tempo a pensare e a
riflettere sulle cose. Non attribuiva molta importanza ai pensieri sul
passato o sul futuro perché non intendeva consentire alla sua attenzione
di allontanarsi dalla mente stessa. Osservava intensamente quello che
sorgeva nella sua consapevolezza nel momento presente. Osservando gli
atteggiamenti mutevoli e le differenti reazioni della mente a quel che
sperimentava, non attribuiva importanza né agli uni né alle altre, e
ripeteva a se stesso questo insegnamento: «~È incerto.~» «~Non è una
cosa sicura.~» Potete insegnare a voi stessi a vedere l'impermanenza in
questa maniera, non ci vorrà molto prima che otteniate la visione
profonda nel Dhamma.

Non è infatti necessario correre dietro alle proliferazioni mentali. Ci
si muove all'interno di un circuito chiuso, si gira in tondo. La mente
lavora in questo modo. È \emph{saṃsāra-vatta}, il ciclo senza fine della
nascita e della morte. La mente ne è completamente avvolta. Se cercaste
di inseguire la mente mentre gira in tondo, ce la fareste a
raggiungerla? Si muove così velocemente! Riuscireste anche solo a stare
al passo con essa? Provate a rincorrerla e vedete cosa succede. Quel che
dovete fare è restare fermi in un punto, e lasciare che la mente giri
per conto suo in questo circuito. Immaginate che la mente sia un
bambolotto meccanico in grado di andarsene in giro. Se iniziasse a
correre sempre di più fino a raggiungere la massima velocità, non
sareste in grado di correre tanto da stare al passo. In realtà, però,
non c'è bisogno di correre da nessuna parte. Potete limitarvi a restare
fermi in un posto e lasciare che sia il bambolotto a correre. Se restate
fermi al centro del circuito senza corrergli dietro, riuscite a vedere
il bambolotto tutte le volte che vi supera e completa un giro. Se
infatti tentate di corrergli dietro, quanto più cercate di inseguirlo e
di acchiapparlo, tanto più sarà in grado di schivarvi.

Per quanto concerne il \emph{tudong}, incoraggio e nello stesso tempo
scoraggio i monaci ad andarci. Se il praticante ha già un po' di
saggezza sul modo di addestrarsi, allora non ci sono problemi. Un monaco
che conoscevo non considerava necessario andare in \emph{tudong} nella
foresta, non pensava che il \emph{tudong} implicasse viaggiare in un
qualche posto. Dopo averci pensato, decise di restare ad addestrarsi in
monastero, facendo voto di intraprendere tre delle pratiche
\emph{dhutaṅga} e di mantenerle rigorosamente senza andare da nessuna
parte. Sentiva che non era necessario stancarsi camminando a lungo,
portandosi a tracolla il peso della sua ciotola, delle vesti e delle
altre cose indispensabili. Anche la sua scelta era giusta. Se però il
desiderio di vagare per colline e foreste in \emph{tudong} è forte, non
la si trova molto soddisfacente. Alla fine, se si ha chiara percezione
della Verità delle cose, basta ascoltare una parola d'insegnamento ed
essa vi condurrà a una penetrante visione profonda.

Posso offrirvi un altro esempio. Un giovane novizio che ho incontrato
una volta voleva praticare del tutto solo in un luogo di cremazione.
Siccome era poco più di un bambino, un adolescente appena, ero piuttosto
preoccupato del suo benessere e lo tenevo d'occhio per vedere come
andavano le cose. La mattina seguente sarebbe andato a fare la questua e
poi avrebbe portato lì il suo cibo, dove avrebbe consumato il suo pasto
in solitudine, attorniato dalle fosse in cui erano stati sepolti i
cadaveri che non erano stati arsi. Tutte le notti avrebbe dormito
completamente solo accanto ai resti dei defunti. Dopo essere stato nelle
vicinanze per circa una settimana, sono andato a controllare e a vedere
di persona come stava. Dall'esterno pareva essere a proprio agio. Gli
chiesi: «~Allora non hai paura a stare qui?~» «~No, non ne ho~», mi
rispose. «~Com'è che non hai paura?~» «~Mi sembra improbabile che qui ci
sia qualcosa di cui avere paura.~» Tutto quello di cui c'era bisogno era
questa semplice riflessione, e la mente smetteva di proliferare. Quel
novizio non aveva bisogno di pensare a cose che gli avrebbero solo
complicato la vita. Aveva una ``cura'' immediata. La sua paura svanì.
Dovreste cercare di meditare in questo modo.

Dico che qualsiasi cosa stiate facendo, se sostenete la consapevolezza
senza arrendervi, in piedi, camminando, che arriviate o che ve ne
andiate, il vostro \emph{samādhi} non degraderà. Non regredirà. Se c'è
troppo cibo dite che è sofferenza, che è solo un fastidio. Come mai
tutta questa agitazione? Se ce n'è troppo, prendetene solo un po' e
lasciate il resto a qualcun altro. Perché lo fate diventare un problema
così grande? Questo non porta serenità. Cos'è che non porta serenità?
Prendetene una piccola porzione e date via il resto. Se però siete
attaccati al cibo e vi sentite male a rinunciare in favore degli altri,
allora è ovvio che trovate difficili le cose. Se siete esigenti e volete
mangiare un po' di questo e un po' di quello, ma non tanto di
quell'altro ancora, vedrete che alla fine avrete preso così tanto cibo
da riempire la ciotola fino al punto che nulla più avrà comunque un buon
sapore. E così vi succede di attaccarvi all'opinione che è una
distrazione e un problema quando viene offerto molto cibo. Perché essere
distratti e agitati? Siete voi a consentire a voi stessi di agitarvi per
il cibo. Forse che il cibo stesso si distrae e si agita? È ridicolo. Vi
state agitando tutti per nulla.

Quando c'è molta gente che viene in monastero, dite che è un disturbo.
Dov'è il disturbo? In realtà, se la routine quotidiana e
l'addestramento abituale vengono rispettati, è piuttosto semplice. Non
dovete farne un grosso problema: andate a fare la questua, tornate
indietro e mangiate, svolgete tutte le attività e le faccende che è
necessario sbrigare addestrando voi stessi con consapevolezza, e andate
avanti. Fate in modo di essere certi di non tralasciare qualche aspetto
della routine monastica. Quando recitate i canti della sera, la
coltivazione della consapevolezza viene davvero meno? Se il solo
eseguire i canti del mattino e quelli della sera fa andare in pezzi la
vostra meditazione, questo significa che, comunque, non avete proprio
imparato a meditare. Prostrarsi, cantare le lodi del Buddha, del Dhamma
e del Saṅgha durante gli incontri giornalieri, come pure qualsiasi altra
cosa facciate, sono tutte quante attività estremamente salutari. Come
possono essere la causa della degenerazione del vostro \emph{samādhi}?
Se pensate che andare agli incontri giornalieri sia una distrazione,
questa cosa osservatela di nuovo. Non sono gli incontri a rappresentare
una distrazione e a essere sgradevoli, siete voi. Se consentite a
pensieri non salutari di agitarvi, allora tutto diventa una distrazione
e una cosa sgradevole. Anche se non andate agli incontri, finirete
comunque per essere distratti e agitati.

Dovete imparare come riflettere con saggezza e come conservare uno stato
mentale salutare. Tutti vengono catturati da questi stati mentali di
confusione e di agitazione, soprattutto chi ha cominciato da poco
l'addestramento. Quello che in realtà avviene, è che permettete alla
vostra mente di uscire all'esterno, di interferire con tutte queste cose
e di agitarsi. Quando venite ad addestrarvi in una comunità monastica,
limitatevi a prendere la decisione di restarci e di continuare a
praticare. Se gli altri si addestrano nel modo giusto o nel modo
sbagliato è affar loro. Continuate a impegnarvi nell'addestramento,
seguite i parametri monastici e aiutatevi reciprocamente con consigli
utili. Tutti coloro che non sono contenti di addestrarsi qui sono liberi
di andare altrove. Se volete restare qui, andate avanti e continuate a
praticare.

Sulla comunità ha un effetto estremamente benefico il fatto che nel
gruppo ci sia un monaco riservato e che si addestra con saldezza. Gli
altri monaci che gli stanno intorno inizieranno a notarlo e a prendere
esempio dal suo buon comportamento. Lo osserveranno e si chiederanno
come riesca a conservare una sensazione di benessere e di serenità
mentre si addestra alla consapevolezza. Il buon esempio offerto da quel
monaco è una delle cose più benefiche che egli possa fare per gli altri.
I giovani componenti della comunità monastica, addestrandosi con una
routine giornaliera e continuando a osservare le regole sul modo in cui
vanno fatte le cose, devono seguire la guida dei monaci più anziani e
continuare a impegnarsi nella routine. Quale che sia l'attività che
state svolgendo, quando è tempo di interromperla, interrompetela. Dite
cose appropriate e utili, e addestrate voi stessi ad astenervi da parole
inappropriate e dannose. Non consentite a questo genere di parole di
sgusciare fuori. Non c'è bisogno di prendere una gran quantità di cibo
al momento del pasto. Prendete solo poche cose e lasciate il resto.
Quando vedete che c'è molto cibo, la tendenza è a indulgere e cominciare
a scegliere un po' di questo e a provare un po' di quello, e così si
finisce col mangiare tutto quello che è stato offerto. «~Ajahn, per
favore prendi un po' di questo.~» «~Per favore, venerabile, prendi un
po' di quello.~» Quando sentite che vi invitano in questo modo, se non
state attenti la mente si agiterà. La cosa da fare è lasciar andare.
Perché farsi coinvolgere? Voi pensate che sia il cibo ad agitarvi, ma la
vera radice del problema sta nel fatto che lasciate uscire la mente
all'esterno e che essa resta invischiata nel cibo. Se riuscite a
riflettere e a capirlo, ciò dovrebbe semplificarvi molto la vita. Il
problema è che non avete sufficiente saggezza. Non avete sufficiente
visione profonda per vedere come funziona il processo di causa ed
effetto.

In realtà, quando in passato capitava che fossi in cammino, se era
indispensabile ero pronto a fermarmi nel monastero di un villaggio o di
una città.\footnote{In genere in Thailandia i monaci che vivono nel
  monastero di un villaggio o di una città trascorrono più tempo
  studiando la lingua pāli e i testi buddhisti piuttosto che
  addestrandosi nelle regole della disciplina o nella meditazione, che
  sono invece più praticate nella Tradizione della Foresta.} Durante i
vostri viaggi, quando siete soli e dovete attraversare varie comunità
monastiche con differenti standard di addestramento e di disciplina,
recitate questi versi sia per protezione sia come guida per la
riflessione: \emph{suddhi asuddhi paccattaṃ}, la purezza o la non
purezza della propria virtù ognuno la conosce da sé. Potreste finire per
dover fare affidamento unicamente sulla vostra stessa integrità.

Quando state viaggiando in posti nei quali non siete mai stati in
precedenza, potreste trovarvi nella necessità di scegliere dove
trascorrere la notte. Il Buddha insegnò che i monaci e le monache
dovrebbero vivere in luoghi sereni. Perciò, basandovi su quel che avete
a disposizione, dovreste cercare un posto tranquillo per restarvi e
meditare. Se non riuscite a trovare un posto davvero tranquillo, potete,
in seconda istanza, almeno trovarne uno nel quale siate in grado di
essere sereni interiormente. Se per un qualche motivo è necessario
rimanere in un posto, dovete imparare come viverci serenamente, senza
consentire alla brama (\emph{taṇhā}) di sopraffare la mente. Se poi
decidete di andarvene da quel monastero o da quella foresta, non
andatevene per brama. Allo stesso modo, se restate da qualche parte, non
restate per brama. Comprendete ciò che motiva i vostri pensieri e le
vostre azioni. È vero che il Buddha raccomandò ai monaci di condurre uno
stile di vita e di vivere in situazioni che favoriscano la tranquillità
e che siano adatte alla meditazione. Come affronterete le situazioni in
cui un posto tranquillo non riuscite a trovarlo? Alla fine tutto questo
potrebbe solo farvi diventare matti. Dove andrete? Restate proprio lì,
dove vi trovate. Rimanete fermi e imparate a vivere in pace. Addestrate
voi stessi finché siete in grado di restare e di meditare nel posto in
cui vi trovate. Il Buddha insegnò che dovreste conoscere e comprendere
il tempo giusto e il luogo opportuno sulla base delle condizioni
concrete. Non incoraggiò i monaci e la monache a vagare ovunque senza
scopo alcuno. Raccomandò indubbiamente di trovare un luogo adatto, ma se
ciò è impossibile potrebbe essere necessario trascorrere alcune
settimane o qualche mese in un posto che non è poi tanto tranquillo o
adatto. Che fareste allora? Forse morireste per lo shock!

Imparate perciò a conoscere la vostra mente e a conoscere le vostre
intenzioni. Alla fine viaggiare da un posto all'altro è solo questo.
Quando si va da qualche altra parte, si ha la tendenza ad aspettarsi che
lì ci siano cose dello stesso genere di quelle che ci si è lasciati alle
spalle, e si hanno in continuazione dubbi a proposito di quello che ci
attende nel posto in cui si andrà. Potreste trovarvi a prendere la
malaria o qualche altra spiacevole malattia ancor prima che riusciate a
rendervene conto, e a dover cercare un dottore che vi curi, vi dia
medicine e vi faccia iniezioni. In pochissimo tempo, la vostra mente
sarebbe più agitata e distratta che mai! In verità, il segreto per una
meditazione ben riuscita consiste nell'allineare al Dhamma il vostro
modo di vedere le cose. L'unica cosa importante è instaurare la Retta
Visione nella mente. Niente di complicato, solo questo. Dovete però
continuare a sforzarvi di investigare e di cercare la strada giusta per
voi. Ovviamente, questo comporta alcune difficoltà, perché non avete
ancora la maturità della saggezza e della comprensione.

Che cosa pensate di fare, allora? Provate ad andare in \emph{tudong} e
vedete cosa succede ... potreste anche stancarvi di andare in giro. Non
è una cosa sicura. Oppure, forse state pensando che, se vi dedicherete
davvero alla meditazione, non desidererete andare in \emph{tudong}
perché il tutto non vi sembrerà interessante. Però, anche questa
percezione è incerta. Potreste sentirvi completamente annoiati al solo
pensiero d'andare in \emph{tudong}, ma pure questa percezione può
cambiare, e potrebbe non trascorrere molto tempo prima che iniziate a
desiderare di uscire e di mettervi di nuovo in movimento. Oppure,
ancora, potreste stare fuori in \emph{tudong} per un tempo indefinito e
continuare a vagare da un posto all'altro senza limiti di tempo e senza
una destinazione fissa. Di nuovo, è incerto. È su questo che dovete
riflettere quando fate meditazione. Andate controcorrente rispetto ai
vostri desideri. Potreste attaccarvi o all'opinione che certamente
andrete in \emph{tudong} o all'opinione che certamente rimarrete fermi
in monastero ma, comunque sia, state rimanendo prigionieri
dell'illusione. Vi state attaccando nel modo sbagliato a modi fissi di
vedere. Andate a investigare questa cosa da voi stessi. Io l'ho già
contemplata nel corso della mia esperienza, e ve la sto spiegando così
com'è, nel modo più semplice e diretto possibile. Ascoltate perciò quel
che vi dico, e poi osservate e contemplate da voi stessi. Questo è
veramente il modo in cui stanno le cose. Alla fine sarete in grado di
vedere la verità di tutto questo da voi stessi. Quando avrete visione
profonda nella verità, qualsiasi decisione prendiate essa sarà
accompagnata da Retta Visione, e concorderà con il Dhamma.

Qualsiasi cosa decidiate di fare, andare in \emph{tudong} o restare in
monastero, dovete prima riflettere con saggezza. Non è che vi sia stato
proibito di andare nella foresta o di cercare un posto tranquillo per
meditare. Se vi mettete in cammino, fatelo davvero e camminate fino a
che non siete esausti, sul punto di cadere, mettetevi alla prova fino ai
limiti della vostra resistenza fisica e mentale. In passato, appena
intravedevo le montagne, mi sentivo euforico, mi sembrava che i miei
piedi non toccassero il suolo. Oggi, il mio corpo inizia a gemere appena
le intravedo e tutto quel che desidero fare è voltarmi e tornare in
monastero. Non c'è più alcun entusiasmo per tutto questo. Prima ero
davvero felice di vivere sulle montagne. Ho perfino pensato che lassù ci
avrei trascorso tutta la vita!

Il Buddha insegnò a essere consapevoli di quello che sorge nella mente
nel momento presente. Conoscere la Verità del modo in cui sono le cose
nel momento presente. Questi sono gli insegnamenti che Egli ci lasciò, e
sono corretti, ma sono i vostri pensieri e i vostri modi di vedere a non
essere ancora corretti e in linea con il Dhamma, e questa è la ragione
per cui continuate a soffrire. Provate ad andare in \emph{tudong}, se
questa vi sembra la cosa giusta da fare. Vedete com'è andarsene in giro
da un posto all'altro e come ciò influisca sulla mente.

Non voglio vietarvi di andare in \emph{tudong}, ma non voglio neanche
darvi il permesso di farlo. Capite cosa intendo? Non voglio né impedirvi
di andare né consentirvelo, voglio solo condividere con voi alcune mie
esperienze. Se andate in \emph{tudong}, il tempo usatelo a beneficio
della vostra meditazione. Non andatevene in giro come se foste dei
turisti, divertendovi a viaggiare qui e là. Di questi tempi sembra quasi
che un numero sempre maggiore di monaci e di monache vada in
\emph{tudong} per indulgere a un po' di godimento sensoriale e di
desiderio di avventura piuttosto che a reale beneficio
dell'addestramento spirituale. Se andate, fate allora davvero uno sforzo
sincero e utilizzate le pratiche \emph{dhutaṅga} per eliminare le
contaminazioni. Queste pratiche \emph{dhutaṅga} potete assumerle pure se
restate in monastero. Di questi tempi, quel che chiamano \emph{tudong}
ha più la tendenza a essere un periodo per la ricerca di eccitazione e
di stimoli, invece che di addestramento nelle tredici pratiche
\emph{dhutaṅga}. Se andate per questa ragione, quando parlate di
\emph{tudong} state solo mentendo a voi stessi. È un \emph{tudong}
immaginario. Nei fatti il \emph{tudong} può essere una cosa che sostiene
e intensifica la vostra meditazione. Se andate dovreste davvero farlo.
Contemplate quello che è il vero scopo e il vero significato del recarsi
in \emph{tudong}. Se andate, vi incoraggio a utilizzare questa
esperienza come un'opportunità per imparare e favorire la vostra
meditazione, non per perdere tempo. Non consentirò ai monaci di andare
se non sono ancora pronti, ma se qualcuno è sincero e seriamente
interessato alla pratica, non lo fermerò.

Quando state programmando di andare, vale la pena che poniate a voi
stessi queste domande, e prima di tutto che riflettiate su di esse.
Stare sulle montagne può essere un'esperienza utile, anch'io ero solito
farlo. Allora dovevo alzarmi molto presto al mattino perché le case
presso le quali andavo a elemosinare il cibo erano molto distanti.
Dovevo salire e scendere una montagna e a volte il cammino era così
lungo e arduo che non era possibile andare e tornare in tempo per
consumare prima di mezzogiorno il pasto nel luogo in cui ero accampato.
Se confrontate tutto questo con il modo in cui oggigiorno stanno le
cose, forse pensate che non sia in realtà necessario percorrere tragitti
così lunghi e costringersi a disagi tanto grandi. Difatti potrebbe
essere di maggior beneficio andare a elemosinare il cibo in uno dei
villaggi vicini a questo monastero, tornare per il pasto e avere molta
energia di riserva per impegnarsi ulteriormente nella pratica formale. È
così se vi state addestrando con sincerità, ma non è una cosa giusta se
state solo prendendo le cose alla leggera e dopo il pasto vi piace
tornare subito indietro nella vostra capanna per un pisolino. Nei giorni
in cui ero in \emph{tudong} dovevo lasciare il luogo in cui ero
accampato alle prime luci dell'alba e consumare molte delle mie energie
solo per camminare attraverso le montagne, e alla fine, il poco tempo a
disposizione mi costringeva a mangiare da qualche parte nel bel mezzo
della foresta, prima che riuscissi a tornare indietro. Pensandoci
adesso, mi chiedo se sia necessario sottoporsi a tutti questi disagi.
Sarebbe meglio trovare un posto per praticare nel quale le strade per
andare a elemosinare il cibo nel villaggio vicino non siano troppo
lunghe o difficili da percorrere, una cosa che vi consentirebbe di
risparmiare le vostre energie per la meditazione formale. Mentre voi
ripulite e sistemate tutto e tornate nella vostra capanna, pronti per
continuare a fare meditazione, quel monaco su per le montagne sarebbe
ancora bloccato nella foresta senza aver nemmeno cominciato a consumare
il suo pasto.

I punti di vista sul modo migliore di praticare possono essere diversi.
A volte, in realtà, si deve sperimentare un po' di difficoltà prima di
poter avere visione profonda nella sofferenza e conoscerla per quello
che è. Il \emph{tudong} può avere i suoi vantaggi, e io non critico né
quelli che restano in monastero né quelli che vanno in \emph{tudong}, se
il loro scopo è fare progressi nell'addestramento di se stessi. Non lodo
i monaci solo perché restano in monastero, e nemmeno lodo i monaci
unicamente perché vanno in \emph{tudong}. Coloro che meritano davvero di
essere lodati sono quelli con Retta Visione. Se restate in monastero,
dovrebbe essere per coltivare la mente. Se andate, dovrebbe essere per
coltivare la mente. La meditazione e l'addestramento vanno male quando
uscite con gli amici ai quali siete attaccati, quando siete interessati
solo a divertirvi insieme e a farvi coinvolgere in stolte occupazioni.

\textbf{Domande e risposte}

Avete da dire qualcosa sul modo di addestrarsi? Che ne pensate di quello
che vi ho detto? Cosa pensate di decidere di fare in futuro?

Un \emph{bhikkhu}: Vorrei qualche insegnamento sull'idoneità di
differenti oggetti di meditazione per vari temperamenti. Per molto tempo
ho cercato di calmare la mente focalizzando l'attenzione sul respiro
congiuntamente alla recitazione di \emph{Buddho}, ma non sono mai
diventato davvero sereno. Ho cercato di contemplare la morte, ma non mi
ha aiutato a calmare la mente. Neanche la riflessione sui cinque
aggregati (\emph{khandhā}) ha funzionato. Così, alla fine ho esaurito
tutta la mia saggezza.

Ajahn Chah: Lascia andare e basta! Se hai esaurito tutta la tua
saggezza, devi lasciar andare.

Un \emph{bhikkhu}: Appena inizio a sperimentare un po' di calma durante
la meditazione seduta, immediatamente saltano fuori numerosissimi
ricordi e pensieri che disturbano la mente.

Ajahn Chah: Proprio questo è il punto. È incerto. Insegna a te stesso
che non è certo. Sostieni questa riflessione sull'impermanenza quando
mediti. Ogni oggetto dei sensi e ogni stato mentale che sperimenti è
senza alcuna eccezione impermanente. tieni sempre presente questa
riflessione nella mente. Durante la meditazione rifletti sul fatto che
la mente distratta è una cosa incerta. Quando la mente diventa calma con
il \emph{samādhi}, anche questo è allo stesso modo incerto. La cosa che
dovrebbe veramente offrirti un sostegno è la riflessione
sull'impermanenza. Non attribuire troppa importanza a nient'altro. Non
lasciarti coinvolgere dalle cose che sorgono nella mente. Lascia andare.
Anche se sei sereno, non c'è bisogno di pensarci troppo su. Non prendere
la cosa troppo seriamente. E non prendere la cosa troppo seriamente
neanche se non sei sereno. \emph{Viññānaṃ aniccaṃ}: lo hai mai letto da
qualche parte? Significa che la coscienza sensoriale è impermanente. Lo
hai mai sentito prima? Come dovresti addestrarti in relazione a questa
verità? Come dovresti contemplare, quando constati che sia la mente
serena sia la mente agitata sono transitorie? La cosa importante è
sostenere la consapevolezza del modo in cui sono le cose. In altre
parole, conosci che sia la mente calma sia la mente distratta sono
incerte. Quando lo sai, come vedrai le cose? Quando questa comprensione
s'è impiantata nella mente, tutte le volte che sperimenti stati mentali
di serenità sai che sono transitori e anche quando sperimenti stati
mentali agitati sai che sono transitori. Sai come meditare con questo
genere di consapevolezza e visione profonda?

Un \emph{bhikkhu}: No, non lo so fare.

Ajahn Chah: Investiga l'impermanenza. Quanti giorni possono veramente
durare quegli stati mentali di serenità? La meditazione seduta con una
mente distratta è una cosa incerta. Quando la meditazione ha buoni
risultati e la mente entra in uno stato di calma, anche questa è una
cosa incerta. È così che arriva la visione profonda. Cosa ti resta per
attaccarti? Continua a seguire quel che avviene nella mente. Quando
investighi, continua a interrogarti e a pungolarti, scandagliando sempre
più in profondità la natura dell'impermanenza. Sostieni la tua
consapevolezza proprio su questo punto, non c'è bisogno di andare da
nessuna altra parte. In pochissimo tempo la mente si calmerà proprio
come volevi che facesse.

Praticare la meditazione con \emph{Buddho} non pacifica la mente o
praticare la consapevolezza del respiro non pacifica la mente perché ti
stai attaccando alla mente distratta. Quando reciti \emph{Buddho} o ti
concentri sul respiro e la mente non s'è ancora calmata, rifletti
sull'incertezza e non lasciarti troppo coinvolgere dal fatto che la
mente sia o non sia serena. Anche se entri in uno stato di tranquillità,
non lasciarti coinvolgere nemmeno da questo, perché ti può ingannare e
indurti ad attribuire troppo significato e importanza a questo stato
mentale. Devi usare un po' di saggezza quando hai a che fare con la
mente governata dall'illusione. Quando c'è calma riconosci semplicemente
questo dato di fatto e prendilo come un segnale che la meditazione sta
andando nella giusta direzione. Se la mente non è calma, semplicemente
riconosci la realtà, che la mente è confusa e distratta, ma non c'è
niente da guadagnare se si rifiuta di accettare la verità e si cerca di
combatterla. Quando la mente è serena, puoi essere consapevole che è
serena, ma ricorda a te stesso che qualsiasi stato di serenità è
incerto. Quando la mente è distratta, osserva l'assenza di pace e
riconosci che è solo questo: la mente distratta è soggetta al
cambiamento come quella serena.

Se instauri questo genere di visione profonda, l'attaccamento al senso
del sé collassa appena inizi a confrontarti con esso e a investigare.
Quando la mente è agitata, nel momento in cui cominci a riflettere
sull'incertezza di questo stato mentale, il senso del sé, che deriva
dall'attaccamento, inizia a sgonfiarsi. Si inclina da un lato come un
gommone forato. Quando l'aria esce, il gommone inizia a capovolgersi: il
senso del sé collassa in questo stesso modo. Provalo tu stesso. Il
problema sta nel fatto che di solito non si riesce ad acchiappare con
sufficiente velocità il pensiero illuso. Quando sorge, tutt'intorno ad
esso il senso del sé immediatamente genera agitazione mentale, ma appena
rifletti sulla sua natura mutevole l'attaccamento collassa.

Questa cosa cerca di osservarla da te stesso. Continua a interrogarti, e
a esaminare sempre più in profondità la natura dell'attaccamento. Di
solito non riesci a fermare e a interrogare l'agitazione mentale. Devi
essere paziente e procedere con cautela. Lascia che questa agitata
proliferazione segua il suo corso, poi continua a procedere con cautela,
lentamente. Sei abituato a non esaminarla, e perciò devi essere
determinato a focalizzare l'attenzione sull'agitazione mentale. Sii
saldo e non lasciarle spazio alcuno per restare nella mente. Però,
quando di solito vi parlo prorompete in lamenti di frustrazione:
«~Questo vecchio \emph{ajahn} parla sempre di impermanenza e della
natura mutevole delle cose.~» Fin dal primo momento non riuscite a
sopportare di sentirlo e volete solo scappare da qualche altra parte.
«~Luang Por ha solo questo insegnamento, che tutto è incerto.~» Se siete
davvero stanchi di questo insegnamento, dovreste andarvene e applicarvi
nella meditazione finché sviluppate una visione profonda sufficiente per
consentire alla vostra mente di avere qualche reale fiducia e certezza.
Andate avanti e provateci. In pochissimo tempo probabilmente tornerete
di nuovo qui! Cercate perciò di conservare nella vostra memoria e nel
vostro cuore questi insegnamenti. Poi andate avanti, e provate ad andare
in \emph{tudong}. Se non riuscite a comprendere e a vedere la Verità nel
modo che vi ho spiegato, poca sarà la pace che troverete. Ovunque siate,
dentro di voi non vi sentirete a vostro agio. Non sarete affatto in
grado di trovare da nessuna parte quello su cui potete veramente
meditare.

Sono d'accordo che fare molta meditazione formale per sviluppare il
\emph{samādhi} sia una buona cosa. Vi sono familiari termini come
\emph{ceto-vimutti}\footnote{\emph{ceto-vimutti}. Liberazione della
  mente-cuore.} e \emph{paññā-vimutti}?\footnote{\emph{paññā-vimutti}.
  Liberazione per mezzo del discernimento o saggezza.} Ne comprendete il
significato? \emph{Vimutti} significa liberazione dalle contaminazioni
mentali (\emph{āsavā}).\footnote{\emph{āsava}. Influsso impuro, macchia,
  fermentazione o effluenza.} Ci sono due modi per mezzo dei quali la
mente può ottenere la Liberazione: \emph{ceto-vimutti} si riferisce alla
liberazione che giunge dopo che il \emph{samādhi} è stato sviluppato e
perfezionato al suo livello più potente e raffinato. Il praticante
inizialmente sviluppa l'abilità di sopprimere del tutto le
contaminazioni per mezzo del potere del \emph{samādhi} e poi si volge
allo sviluppo della visione profonda per ottenere finalmente la
Liberazione. \emph{Paññā-vimutti} significa Liberazione mediante la
saggezza, ossia il praticante sviluppa il \emph{samādhi} fino al livello
in cui la mente è completamente unificata e sufficientemente stabile per
supportare e sostenere la visione profonda, che poi conduce
all'eliminazione delle contaminazioni.

Questi due generi di Liberazione possono essere paragonati a differenti
tipi di alberi. Alcune specie crescono e sono fiorenti innaffiandole di
frequente, ma altre possono morire se si dà loro troppa acqua. A questo
tipo di alberi bisogna darne solo poca, solo quella sufficiente per
farli continuare a vivere. Così sono alcune specie di pini. Se a essi si
dà troppa acqua, muoiono. Ce n'è bisogno solo di un po' ogni tanto.
Strano, vero? Guardate questo pino. Sembra così secco e bruciato dal
caldo che ci si chiede come riesca a crescere. Pensateci. Da dove prende
l'acqua di cui ha bisogno per sopravvivere e per produrre questi rami
lussureggianti? Altri tipi di alberi necessitano di molta più acqua per
crescere altrettanto. Poi ci sono quelle piante che si mettono nei vasi
e si appendono qui e là con le radici che penzolano per aria. Si
potrebbe pensare che muoiano e basta, ma le foglie crescono e si
allungano molto velocemente quasi senz'acqua. Se si trattasse delle
normali piante che crescono nella terra, probabilmente avvizzirebbero.
Con quei due diversi tipi di Liberazione è la stessa cosa. Capite? È
semplicemente che differiscono in questo modo naturale l'una dall'altra.

\emph{Vimutti} significa Liberazione. \emph{Ceto-vimutti} è la
Liberazione che proviene dalla forza della mente che è stata addestrata
al \emph{samādhi} al massimo grado. È come quelle piante che necessitano
di molta acqua per prosperare. Altri alberi ne hanno bisogno solo di
poca. Con troppa acqua muoiono. È nella loro natura di crescere
rigogliosamente solo con una piccola quantità d'acqua. Perciò il Buddha
insegnò che ci sono due tipi di Liberazione dalle contaminazioni,
\emph{ceto-vimutti} e \emph{paññā-vimutti}. Per ottenere la Liberazione
è necessaria sia la saggezza sia l'energia del \emph{samādhi}. C'è
differenza tra \emph{samādhi} e saggezza?

Un \emph{bhikkhu}: No.

Ajahn Chah. Allora perché vengono chiamati in modo diverso? Perché c'è
una differenziazione tra \emph{ceto-vimutti} e \emph{paññā-vimutti}?

Un \emph{bhikkhu}: Si tratta solo di una distinzione verbale.

Ajahn Chah: È giusto. Lo capite? Se non lo capite, è molto facile che ve
ne andiate in giro correndo qua e là a etichettare le cose e a fare
distinzioni, e che ne siate trasportati via fino al punto di perdere i
contatti con la realtà. In verità, però, ognuno di queste due tipi di
Liberazione ha una connotazione leggermente diversa. Non sarebbe esatto
dire che sono esattamente la stessa cosa, ma non sono neanche due cose
diverse. Dico bene se rispondo in questo modo? Dirò che queste due cose
non sono né esattamente la stessa cosa né sono diverse. È così che
rispondo a questa domanda. Dovete prendere quel che ho detto, portarlo
con voi e rifletterci su.

Parlare della velocità e della fluidità della consapevolezza mi fa
pensare a quando, durante i miei viaggi, ero da solo in cammino e mi
imbattei in un vecchio monastero abbandonato. Sistemai il mio ombrello
con la zanzariera per accamparmi lì per qualche giorno e praticare la
meditazione. Sul terreno del monastero c'erano molti alberi da frutta, i
cui rami erano carichi di frutti maturi. Volevo davvero mangiarne
qualcuno, ma non osai farlo perché temevo che quegli alberi fossero di
proprietà del monastero e io non avevo alcun permesso di prenderli. Più
tardi arrivò un abitante del villaggio con una cesta e, vedendo che
stavo lì, mi chiese il permesso di raccogliere la frutta. Forse me lo
chiese perché pensò che fossi il proprietario degli alberi. Pensandoci,
capii che non avevo alcuna autorità per consentirgli di raccogliere la
frutta, ma se glielo avessi proibito mi avrebbe criticato di essere
possessivo e avaro: in entrambi i casi ci sarebbero state alcune
conseguenze nocive. Risposi al laico in questo modo: «~Anche se mi trovo
in questo monastero, non sono il proprietario degli alberi. Capisco che
vuoi un po' di frutta. Non ti proibirò di prenderla, ma non ti darò
neanche il permesso di farlo. Dipende da te.~» Era quello di cui aveva
bisogno. Non prese nulla! Parlare in questo modo fu molto utile. Non
avevo proibito nulla né avevo dato il permesso, perciò non aveva alcun
senso farsi carico della questione. Questo era un modo saggio di
affrontare una situazione del genere: fui in grado di evitare ogni
problema. Parlare in quel modo portò buoni risultati ed è tutt'oggi un
modo utile di parlare. Se si parla alle persone in questo modo insolito,
ciò è sufficiente per far loro temere di fare qualcosa di sbagliato.

Che cosa si intende con la parola temperamento (\emph{carita})?

Un \emph{bhikkhu}: Temperamento? Non so come rispondere.

Ajahn Chah: La mente è una cosa, il temperamento è un'altra, e la
saggezza un'altra ancora. Come vi addestrate con queste cose?
Contemplatele. In che modo se ne parla? Ci sono persone con un
temperamento lascivo, altre con un temperamento pieno di odio, altre
ancora con un temperamento colmo di illusioni, con un temperamento
intelligente e così via. Il temperamento è determinato da quegli stati
mentali ai quali la mente più spesso si attacca e concepisce se stessa.
Per alcuni è la brama, per altri è l'avversione. In realtà si tratta
solo di descrizioni verbali delle caratteristiche della mente, ma
possono essere con chiarezza distinte le une dalle altre.

Siete monaci già da sei anni. Forse avete corso dietro ai vostri
pensieri e ai vostri stati mentali abbastanza a lungo, avete dato la
caccia a essi già per molti anni. Non pochi sono i monaci che vogliono
andare a vivere da soli, e io non ho nulla in contrario. Se volete
vivere da soli, provateci. Se vivete in una comunità, continuate a
farlo. Se non pensate in modo sbagliato, nessuna delle due è una cosa
sbagliata. Se vivete da soli e siete catturati da pensieri sbagliati,
questo vi impedirà di trarre profitto dalla vostra esperienza. Un posto
calmo e sereno è più appropriato per praticare la meditazione. Quando
però un posto adatto non è disponibile, se non fate attenzione la vostra
pratica meditativa perirà. Avrete dei problemi. Fate perciò attenzione a
non disperdere le vostre energie e la vostra consapevolezza nella
ricerca di troppi e vari insegnanti, di diverse tecniche o luoghi per
meditare. Riunite i vostri pensieri e focalizzate le vostre energie.
Rivolgete l'attenzione verso l'interno e sostenete la consapevolezza
sulla mente stessa. Utilizzate questi insegnamenti per osservare e
investigare la mente per un lungo periodo di tempo, non gettateli via.
Teneteli con voi come argomento di riflessione. Osservate quel che vi ho
detto sui fenomeni condizionati soggetti al cambiamento. L'impermanenza
è una cosa da investigare nel corso del tempo. Non ci vorrà molto prima
che otteniate una chiara visione profonda nell'impermanenza. Un
insegnamento datomi da un monaco anziano quando avevo da poco cominciato
a meditare e che è rimasto dentro di me è semplicemente questo: andare
avanti ad addestrare la mente. La cosa importante è non restare vittime
dei dubbi. Per ora è abbastanza.

