\chapter{La natura del Dhamma}

A volte, quando un albero da frutto è in fiore, la brezza scuote i fiori
e li sparpaglia a terra. Alcuni boccioli restano sull'albero, crescono e
divengono piccoli frutti verdi. Il vento soffia, e anche alcuni di essi
cadono! Altri ancora possono diventare frutti quasi maturi e alcuni
altri, prima di cadere, frutti del tutto maturi. Altrettanto avviene con
gli esseri umani. Anche loro cadono come fiori e frutti al vento, in
stadi diversi della loro vita. Alcuni muoiono quando sono ancora
nell'utero o solo pochi giorni dopo la nascita. Altri vivono per pochi
anni e poi muoiono, senza aver raggiunto la maturità. Uomini e donne
muoiono durante la giovinezza. Altri ancora raggiungono una matura o
tarda età. Quando riflettiamo sulla gente, consideriamo la natura dei
frutti al vento. Entrambi sono cose molto incerte.

Anche nella vita monastica si può scorgere questa natura incerta delle
cose. Alcuni vengono in monastero con l'intenzione di ricevere
l'ordinazione, ma cambiano idea e se ne vanno, e qualcuno la cambia
quando ha già la testa rasata. Altri sono novizi quando decidono di
andarsene. Altri ancora restano monaci per un solo Ritiro delle
Piogge\footnote{L'annuale periodo di tempo di tre mesi, che in India
  corrisponde a quello dei primi tre mesi monsonici, durante i quali i
  monaci hanno la regola dell'obbligo di residenza in monastero, un
  periodo che tradizionalmente è dedicato a una formazione più
  intensiva.} e poi lasciano l'abito. Proprio come frutti al vento:
tutto è veramente incerto! La nostra mente è simile. Un'impressione
mentale sorge, attrae la mente e l'afferra, e la mente cade, proprio
come un frutto.

Il Buddha comprese questa natura incerta delle cose. Osservò il fenomeno
dei frutti al vento e rifletté sui monaci e sui novizi suoi discepoli.
Trovò che pure loro avevano essenzialmente la stessa natura. Incerta!
Come poteva essere altrimenti? Questo è il modo in cui tutte le cose
sono. Così, per chi pratica con consapevolezza non è necessario che ci
sia qualcuno a dargli consigli e insegnargli, tutto quel che serve è
essere in grado di vedere e di capire. Un esempio è il caso del Buddha
che, in una vita precedente, era il re Mahājanaka. Non aveva bisogno di
studiare molto. Tutto quel che doveva fare era osservare un albero di
manghi.

Un giorno, mentre visitava un parco con il suo seguito di ministri,
dall'alto del suo elefante notò alcuni alberi di manghi carichi di
frutti maturi. Non potendo fermarsi in quel momento, decise di tornarvi
per mangiarne alcuni. Ovviamente, non sapeva che i ministri al suo
seguito li avrebbero raccolti tutti con avidità e che avrebbero usato
delle pertiche per buttarli giù, scuotendo e rompendo i rami degli
alberi, e strappando e sparpagliandone le foglie. Quando nel pomeriggio
fece ritorno al boschetto di manghi, il re, mentre già immaginava il
delizioso sapore dei manghi, improvvisamente scoprì che erano tutti
andati, completamente finiti! E come se non bastasse, i rami e le foglie
erano stati percossi e sparpagliati.

Il re, allora, completamente deluso e dispiaciuto, notò lì vicino un
altro albero di manghi con le foglie e i rami assolutamente intatti. Si
domandò come mai. Poi comprese che era perché quell'albero non aveva
frutti. Se un albero non ha frutti, nessuno lo disturba e così le sue
foglie e i suoi rami non vengono danneggiati. Rimase assorto su questa
lezione lungo tutto il tragitto di ritorno al palazzo. «~Essere re è
spiacevole, fastidioso e difficile. Richiede una continua sollecitudine
nei riguardi di tutti i sudditi. E se ci sono tentativi di aggredire,
saccheggiare e impossessarsi di parti del regno?~» Non poteva riposare
serenamente. Perfino quando dormiva era disturbato dai sogni. Ancora una
volta vide nella sua mente l'albero di manghi privo di frutti, e con le
foglie e i rami intatti. «~Se si diventa come quell'albero di manghi --
pensò -- anche le nostre ``foglie'' e i nostri ``rami'' non saranno
danneggiati.~»

Nella sua camera si mise a sedere e meditò. Alla fine decise di farsi
ordinare monaco, ispirato da questa lezione dell'albero di manghi.
Paragonò se stesso a quell'albero di manghi e giunse alla conclusione
che se non si era coinvolti nelle vie del mondo era possibile essere
davvero indipendenti e liberi da preoccupazioni e difficoltà. La mente
sarebbe stata imperturbata. Fu riflettendo in questo modo che decise di
farsi ordinare monaco. Da allora in poi, ovunque andasse, quando gli
veniva chiesto chi fosse il suo maestro, rispondeva: «~Un albero di
manghi.~» Non ebbe bisogno di ricevere molti insegnamenti. Un albero di
manghi fu la causa del suo Risveglio all'\emph{opanayiko
dhamma},\footnote{\emph{Opanayiko}: ``Che conduce all'interno'', degno
  di essere realizzato e condotto all'interno della mente; un attributo
  del Dhamma.} all'insegnamento che porta verso l'interiorità. E con
questo Risveglio divenne un monaco, uno che aveva poche preoccupazioni,
che si accontentava di poco e che provava diletto nella solitudine.
Avendo rinunciato alla condizione di sovrano, la sua mente trovò
finalmente la pace.

In questa storia il Buddha era un \emph{bodhisatta}\footnote{\emph{Bodhisatta}:
  Un essere che si impegna per raggiungere il Risveglio.} che sviluppò
continuamente la pratica in questo modo. Come il Buddha quando era il re
Mahājanaka, pure noi dovremmo guardarci intorno e osservare, perché
tutto nel mondo è pronto a insegnarci. Anche solo con un po' di saggezza
intuitiva saremo in grado di vedere con chiarezza attraverso le vie del
mondo. Perverremo a comprendere che nel mondo tutto insegna. Alberi e
piante, ad esempio, possono rivelarci la vera natura della realtà. Se si
ha saggezza non c'è bisogno di chiedere a nessuno, non c'è bisogno di
studiare. Dalla natura possiamo imparare abbastanza per diventare
illuminati, come nella storia del re Mahājanaka, perché tutto segue la
via della Verità. Nulla si discosta dalla Verità.

La compostezza e la moderazione sono associate alla saggezza, le quali,
a loro volta, possono condurre a una visione più profonda delle vie
della natura. In questo modo giungeremo a conoscere che la Verità Ultima
di tutto è \emph{aniccā}-\emph{dukkha}-\emph{anattā}. Prendiamo gli
alberi, ad esempio. Tutti gli alberi del mondo sono uguali, sono Uno, se
vengono visti attraverso la realtà di
\emph{aniccā}-\emph{dukkha}-\emph{anattā}. Inizialmente, nascono, poi
crescono e maturano cambiando in continuazione, finché alla fine muoiono
come a ogni albero deve succedere. Allo stesso modo, le persone e gli
animali nascono, crescono e cambiano durante la loro vita, finché alla
fine muoiono. Gli innumerevoli cambiamenti che avvengono durante questa
transizione dalla nascita alla morte mostrano la Via del Dhamma. Questo
per dire che tutte le cose sono impermanenti, poiché la decadenza e la
dissoluzione rappresentano le loro condizioni naturali.

Se abbiamo presenza mentale e comprensione, se studiamo con saggezza e
consapevolezza, vedremo il Dhamma come una realtà. Così, vedremo come la
gente nasce in continuazione, cambia e infine muore. Tutti sono soggetti
al ciclo della nascita e della morte e, in ragione di questo, tutti
nell'universo sono un unico essere. Perciò, vedere una sola persona in
modo chiaro e distinto è come vedere tutte le persone del mondo. Alla
stessa maniera, tutto è Dhamma. Non solo le cose che vediamo con i
nostri occhi fisici, ma pure quelle che vediamo nella nostra mente. Un
pensiero sorge, poi cambia e svanisce. È \emph{nāma dhamma}, solo
un'impressione mentale che sorge e svanisce. Questa è la natura reale
della mente. Complessivamente, questa è la nobile verità del Dhamma. Se
non si guarda e non si osserva in questo modo, non si vede realmente! Se
non si vede realmente, non si ha la saggezza di ascoltare il Dhamma così
come venne proclamato dal Buddha.

Dov'è il Buddha? Il Buddha è nel Dhamma.

Dov'è il Dhamma? Il Dhamma è nel Buddha. Proprio qui, ora!

Dov'è il Saṅgha? Il Saṅgha è nel Dhamma.

Il Buddha, il Dhamma e il Saṅgha\footnote{Si tratta della ``Triplice
  Gemma'', composta dal Buddha, dal Dhamma e dal Saṅgha
  (\emph{tiratana}), ai quali tutti i buddhisti si rivolgono come a dei
  rifugi (\emph{tisaraṇa}).} esistono nella nostra mente, ma dobbiamo
capirlo con chiarezza. Alcuni lo dicono solo con disinvoltura: «~Oh! Il
Buddha, il Dhamma e il Saṅgha esistono nella mia mente.~» Già questo
rende la loro pratica non idonea, non appropriata. Non è perciò corretto
dire che il Buddha, il Dhamma e il Saṅgha si trovano nella mente prima
che essa conosca il Dhamma.

Riconducendo tutto a questo livello di Dhamma, giungeremo a conoscere
che la Verità esiste nel mondo, e che per noi è perciò possibile
praticare per realizzarla. Ad esempio, i \emph{nāma dhamma} --
sensazioni, pensieri, immaginazione e così via -- sono tutti incerti.
Dopo essere sorta, la collera cresce, cambia e alla fine scompare. Anche
la felicità sorge, cresce, cambia e alla fine scompare. Sono vuote. Non
sono alcuna ``cosa''. Questo è sempre il percorso di tutti i fenomeni,
sia mentali che materiali. Internamente ci sono questo corpo e questa
mente. Esternamente ci sono alberi, piante e ogni genere di cose che
mostrano questa legge universale dell'incertezza. Che si tratti di un
albero, di una montagna o di un animale, è tutto Dhamma, qualsiasi cosa
è Dhamma. Dov'è questo Dhamma? Parlando in modo semplice, quel che non è
Dhamma non esiste. Dhamma è la natura. Questo è chiamato
\emph{saccadhamma},\footnote{\emph{Saccadhamma}: Verità Ultima.}
il Vero Dhamma. Chi vede la natura vede il Dhamma, chi
vede il Dhamma vede la natura. Vedendo la natura si conosce il Dhamma. E
allora a che serve tanto studio, se la realtà ultima della vita, in ogni
suo momento, in ogni suo atto, è solo un ciclo senza fine di nascite e
di morti? Se abbiamo presenza mentale e siamo chiaramente consapevoli in
ogni postura -- seduti, in piedi, camminando, distesi -- la conoscenza
di se stessi è allora pronta a nascere: conoscere la verità del Dhamma
esistente proprio qui e ora.

Attualmente il Buddha, il vero Buddha, è ancora in vita, perché Egli è
il Dhamma stesso, il \emph{saccadhamma}. E il \emph{saccadhamma}, ciò
che consente di diventare Buddha, esiste ancora. Non è scappato da
nessuna parte! Fa sorgere due Buddha, uno nel corpo e l'altro nella
mente. «~Il vero Dhamma -- disse il Buddha ad ānanda -- può essere
realizzato solo mediante la pratica.~» Chiunque veda il Buddha, vede il
Dhamma. Perché? Prima il Buddha non esisteva. Fu solo quando Siddhattha
Gotama\footnote{Siddhattha Gotama: Il nome proprio del Buddha storico;
  nei testi canonici più antichi si menziona il Buddha soltanto con il
  nome di Gotama.} realizzò il Dhamma che egli divenne il Buddha. Se lo
spieghiamo in questo modo, allora Egli è uguale a noi. Se noi
realizzeremo il Dhamma, saremo allora come il Buddha. Questo si chiama
il Buddha nella mente o \emph{nāma dhamma}.\footnote{\emph{Nāma}:
  fenomeno mentale.}

Dobbiamo essere consapevoli di qualsiasi cosa facciamo, perché siamo gli
eredi delle nostre buone e cattive azioni. Nel fare il bene maturiamo
nel bene. Nel fare il male maturiamo nel male. Per sapere che è così,
tutto quello che dovete fare è guardare nella vostra vita quotidiana.
Siddhattha Gotama divenne illuminato per la realizzazione di questa
Verità, e questo fece apparire nel mondo un Buddha. Allo stesso modo,
chiunque pratichi, tutti quelli che praticano per raggiungere questa
Verità, anche loro diventeranno Buddha. Il Buddha perciò esiste ancora.
Alcuni sono molto felici quando dicono: «~Se il Buddha esiste ancora,
allora posso praticare il Dhamma!~» È così che dovreste vedere le cose.

Il Dhamma realizzato dal Buddha è il Dhamma che esiste nel mondo in modo
permanente. Può essere paragonato a una falda d'acqua nel terreno.
Quando una persona desidera scavare un pozzo, deve scavare abbastanza a
fondo da raggiungere la falda. La falda d'acqua è già lì. Non crea
l'acqua, la scopre solamente. Allo stesso modo il Buddha non inventò il
Dhamma, non creò il Dhamma. Rivelò semplicemente quel che già c'era. Per
mezzo della contemplazione il Buddha vide il Dhamma. Per questa ragione
si dice che il Buddha fu illuminato, perché l'Illuminazione è conoscere
il Dhamma. Il Dhamma è la Verità di questo mondo. Avendola vista,
Siddhattha Gotama è detto ``Il Buddha''. Il Dhamma è ciò che consente ad
altre persone di diventare un Buddha, ``Colui che Conosce'', che conosce
il Dhamma.

Quegli esseri che hanno un buon comportamento e sono leali nei riguardi
del Buddha-Dhamma non saranno mai a corto di virtù e di bontà. Con la
comprensione, vedremo che veramente non siamo lontani dal Buddha, ma che
sediamo faccia a faccia con Lui. Quando comprenderemo il Dhamma, in quel
momento vedremo il Buddha. Chi pratica davvero, ascolterà il
Buddha-Dhamma quale che sia la sua postura, se siede ai piedi di un
albero o se è disteso. Non è una cosa alla quale pensare soltanto. Sorge
dalla mente pura. Solo ricordare queste parole non è sufficiente, perché
tutto dipende dal vedere il Dhamma stesso, solo da questo. Perciò
dobbiamo essere determinati a praticare per essere in grado di vederlo,
e così la nostra pratica sarà davvero completa. Seduti o in piedi,
camminando o distesi, sentiremo il Dhamma del Buddha.

Per praticare il suo insegnamento, il Buddha ci insegnò a vivere in un
luogo tranquillo affinché imparassimo il raccoglimento e a contenere i
sensi dell'occhio, dell'orecchio, del naso, della lingua, del corpo e
della mente. Questo è il fondamento della nostra pratica, dal momento
che sono questi gli unici luoghi in cui tutte le cose sorgono. Così, ci
raccogliamo e conteniamo questi sei sensi al fine di conoscere i
fenomeni condizionati che lì sorgono. Tutto il bene e tutto il male
sorgono attraverso questi sei sensi. Sono le facoltà predominanti del
corpo. L'occhio predomina nel vedere, l'orecchio nel sentire, il naso
nell'odorare, la lingua nell'assaporare, il corpo nel contatto con il
caldo, il freddo, il duro e il morbido, e la mente nel sorgere delle
impressioni mentali. Tutto quel che resta da fare è costruire la nostra
pratica attorno a questi punti.

Praticare è facile, perché tutto quello che serve è già stato indicato
dal Buddha. Lo si può paragonare al Buddha che pianta un frutteto e che
ci invita a mangiarne i frutti. Non abbiamo bisogno di piantarne uno
noi. Che si tratti di moralità, di meditazione o di saggezza non c'è
bisogno di creare, sentenziare o speculare, perché tutto quello che
abbiamo bisogno di fare è seguire quello che già esiste
nell'insegnamento del Buddha. Siamo perciò degli esseri con molto
merito, fortunati di aver ascoltato l'insegnamento del Buddha. Il
frutteto esiste già, il frutto è già maturo. Tutto è già completo e
perfetto. Manca solo che qualcuno mangi il frutto, qualcuno che abbia
abbastanza fiducia per praticare! Dovremmo ritenere molto preziosi il
nostro merito e la nostra fortuna. Dobbiamo solo guardarci attorno per
vedere quanto sono numerosi gli esseri sfortunati. Ad esempio i cani, i
maiali, i serpenti e altre creature ancora. Per loro non è possibile
studiare il Dhamma, non è possibile conoscere il Dhamma, non è possibile
praticare il Dhamma. Questi esseri posseduti dalla cattiva sorte stanno
ricevendo la retribuzione del loro \emph{kamma}. Quando non si ha la
possibilità di studiare, di conoscere, di praticare il Dhamma, non si ha
la possibilità di liberarsi dalla sofferenza.

In quanto esseri umani, non dovremmo essere privi di buone maniere e di
disciplina, né permetterci di diventare vittime della cattiva sorte. Non
siate vittime della cattiva sorte! Non privatevi della speranza di
sviluppare la virtù e di raggiungere il Sentiero della Libertà, verso il
\emph{Nibbāna}.\footnote{\emph{Nibbāna}: La liberazione finale da ogni
  sofferenza, lo scopo della pratica buddhista.} Non pensate di non
avere alcuna speranza! Pensando in questo modo la cattiva sorte si
impossesserà di noi, come avviene per gli altri esseri. Siamo creature
giunte all'interno della sfera d'influsso del Buddha. Noi esseri umani
abbiamo già sufficienti meriti e risorse. Se nel presente correggiamo e
sviluppiamo la nostra comprensione, le nostre opinioni e la nostra
conoscenza, tutto questo ci condurrà a comportarci e a praticare per
vedere e conoscere il Dhamma in questa nostra vita di esseri umani.
Siamo creature che dovremmo illuminarci al Dhamma, è per questo che
siamo diverse dalle altre. Il Buddha insegnò che proprio in questo
momento il Dhamma esiste di fronte a noi. Il Buddha siede di fronte a
noi proprio qui e ora! Quale altro momento, quale altro luogo pensate di
poter guardare?

Se non pensiamo rettamente, se non pratichiamo rettamente, torneremo a
essere animali o creature infernali, oppure spiriti famelici o
demoni.\footnote{Secondo il pensiero buddhista, gli esseri nascono in
  una delle otto condizioni dell'esistenza a seconda del loro
  \emph{kamma}. Esse includono tre condizioni paradisiache (nelle quali
  predomina la felicità), la condizione umana, e le suddette quattro
  condizioni dolorose o infernali (ove predomina la sofferenza). Il
  venerabile Ajahn Chah sottolinea continuamente che dovremmo vedere
  queste condizioni nella nostra stessa mente nel momento presente.
  Così, a seconda della condizione della mente, possiamo dire che in
  continuazione nasciamo in queste varie condizioni. Ad esempio, quando
  la mente è infiammata dall'ira, allora siamo caduti dalla condizione
  umana all'inferno proprio qui e ora.} Come sono queste condizioni?
Basta che guardiate nella vostra mente. Quando sorge la collera, com'è?
Eccola, guardate! Quando sorge l'illusione, com'è? Eccola, è proprio lì.
Quando sorge l'avidità, com'è? Guardatela, è proprio lì! Se non
riconosciamo e non comprendiamo con chiarezza questi stati mentali, la
mente non è più quella di un essere umano. Tutti i fenomeni condizionati
sono in divenire. Il divenire dà luogo alla nascita o all'esistenza,
così come viene determinato dalle condizioni presenti. Così, noi
diveniamo ed esistiamo in base al modo in cui ci condiziona la nostra
mente.

