\chapter{Nota ai testi}

\section{Nota al testo inglese}

Questa raccolta di insegnamenti di Ajahn Chah è il risultato della
trascrizione, traduzione e cura redazionale di discorsi originariamente
pronunciati da Ajahn Chah in thailandese o in laotiano. Alcuni di essi
furono impartiti a seguaci laici e molti, forse la maggior parte,
vennero offerti a gruppi monastici, per lo più maschili, che vivevano in
Thailandia con lui.

Alcuni fattori influiscono inevitabilmente non solo sul contenuto, ma
pure sul tono e sull'intensità espressiva degli insegnamenti originari.
Vorrei incoraggiare i lettori a tenere a mente queste circostanze al
fine di apprezzare appieno la portata, l'applicabilità e il senso
complessivo di questi insegnamenti di Dhamma. In un certo senso, man
mano che procedono nella lettura, i laici occidentali dovranno
realizzare una loro propria traduzione interiore, individuando gli
equivalenti per tutte quelle analogie che coinvolgono gli asiatici
bufali d'acqua e il contesto della vita monastico-ascetica nella
foresta; d'altra parte, questo genere di riflessione partecipata --
contemplare come queste parole possano applicarsi alle nostre stesse
vite -- corrisponde esattamente al modo di relazionarsi agli
insegnamenti incoraggiato da Ajahn Chah.

Tra i suddetti fattori, vi sono in primo luogo le difficoltà connesse
alla traduzione dal thailandese, una lingua tonale asiatica
profondamente influenzata dal buddhismo, a una lingua europea con le sue
eredità culturali. Per di più, diversi traduttori hanno lavorato agli
insegnamenti raccolti in questo volume. Le differenti nazionalità e
formazioni di questi traduttori comportano inevitabilmente delle
variazioni nel tono, nello stile e nel vocabolario dei vari capitoli.

In secondo luogo, in Occidente la cultura buddhista è molto cambiata nel
trentennio durante il quale sono state effettuate le traduzioni. Mentre
in precedenza i traduttori hanno forse pensato che molti concetti
buddhisti necessitassero di essere trasposti in termini più familiari
agli occidentali, oggi vi è una maggiore consapevolezza della visione
buddhista del mondo; ad esempio, termini quali ``kamma'' e ``Nibbāna''
fanno ora parte del vocabolario inglese. I discorsi raccolti in questo
volume manifestano perciò un'ampia gamma di modi di tradurre termini e
concetti buddhisti.

In terzo luogo, il contesto monastico buddhista implicava che termini
thailandesi e in lingua pāli con significato tecnico costituissero una
parte consueta e accettata dello stile d'insegnamento vernacolare. I
diversi traduttori hanno preso varie decisioni a proposito di come
rendere tali termini tecnici. Ad esempio, nella lingua thailandese la
stessa parola può significare sia ``cuore'' che ``mente'', e i
traduttori sono stati costretti a scegliere. I lettori dovrebbero
rammentarlo qualora incontrino termini utilizzati in modi che a loro
sembrano non del tutto consueti o non coerenti in un discorso rispetto a
un altro. Spesso alcuni termini sono spiegati nel contesto o con una
nota al testo e, in aggiunta, in un \emph{Glossario} che può essere
rintracciato alla fine del libro.

Confidiamo di essere riusciti, con i nostri sforzi, a restituire in
forma scritta istruzioni orali senza oscurare le intenzioni del maestro.
È stato inevitabile ricorrere ad alcuni compromessi, perché diversi sono
stati i traduttori che hanno cercato di trovare un equilibrio fra
traduzione letterale e traduzione libera. Per questa compilazione
abbiamo revisionato alcune delle traduzioni al fine di standardizzare
termini e stile. Ci siamo ovviamente attenuti al minimo indispensabile.
Ulteriori edizioni di questi scritti potrebbero mirare a un più alto
grado di standardizzazione.

Infine, soprattutto nella terza parte, \emph{Pratica della rinuncia}, i
discorsi di Ajahn Chah furono pronunciati in un contesto nel quale
l'uditorio era per lo più impegnato nella vita celibataria di rinuncia.
Questa circostanza inevitabilmente dà un colore netto al modo in cui il
Dhamma viene presentato. Inoltre, Ajahn Chah parlò molto spesso solo a
uomini. Tale fatto spiega il costante uso di pronomi esclusivamente
maschili in molti di questi discorsi; anche se ad alcuni il fatto che un
tal genere di linguaggio sia stato lasciato intatto può apparire un
ostacolo, a noi è parso inopportuno prenderci la libertà di eliminarlo.
Così, ai lettori potrà talvolta essere di nuovo necessario ricorrere a
traduzioni interiori oppure o all'immaginazione, affinché si palesi
l'importanza di quegli insegnamenti per la loro stessa vita.

Ajahn Chah insegnò ai monaci riuniti in piccole \emph{sālā} in legno
debolmente illuminate da lampade a cherosene. Gli insegnamenti spesso
avevano la forma di esortazioni offerte al termine della recitazione del
\emph{Pāṭimokkha}, il codice di disciplina monastica, che ha luogo ogni
due settimane. Questi insegnamenti erano espressamente rivolti ai monaci
residenti, e per questo i lettori laici dovrebbero ricordare di trovarsi
al cospetto sia di una pratica di rinuncia buddhista sia di un insieme
di insegnamenti di Dhamma. I tre titoli \emph{Pratica quotidiana},
\emph{Pratica formale} e \emph{Pratica della rinuncia} utilizzati per
organizzare questi discorsi non devono essere presi troppo alla lettera.
In ciascun discorso sono presenti sovrapposizioni e, di conseguenza, non
è necessario che ognuno di essi sia letto nell'ordine in cui è stato
presentato.

La preparazione e la presentazione di questa compilazione è il frutto
del lavoro di un gruppo che si è avvalso del tempo e dell'abilità di
numerosi lettori di bozze, tecnici e grafici. Una menzione particolare
meritano i contributi di due dei traduttori originari, Paul Breiter e
Bruce Evans. Siamo debitori verso tutti coloro che hanno contribuito con
il loro tempo e il loro impegno a condurre a completamento questo
progetto.

Speriamo sinceramente che, con queste prospettive nel cuore, le parole
contenute in questo volume siano utili a ogni lettore e possano essere
una condizione per la realizzazione del \emph{Nibbāna}. Fu con la stessa
intenzione che Ajahn Chah parlò per tanti anni. Che queste intenzioni
possano maturare nella vita del lettore e condurlo alla pace e alla
libertà complete.

I compilatori

\section{Nota al testo italiano}

È opportuno illustrare sia i criteri soggiacenti alla traduzione
dall'inglese all'italiano sia qualche scelta redazionale. Nel testo
degli \emph{Insegnamenti} di Ajahn Chah gli interventi sono rarissimi:
sono stati ritenuti necessari solo quando inevitabili per aiutare il
lettore nella comprensione. D'altra parte, nella traduzione sono state
di proposito utilizzate le parole più semplici, per evitare che i testi
avessero un tono ricercato o intellettualistico. Sono state lasciate
spesso inalterate le ripetizioni di concetti e di parole, connesse al
tenore orale dell'esposizione. Di tanto in tanto la punteggiatura è
stata ritoccata, e talora è stato aggiunto o eliminato qualche
capoverso.

Per molti dei discorsi presenti negli \emph{Insegnamenti}, ai differenti
traduttori dal thailandese all'inglese si sono in passato sovrapposti
vari traduttori italiani che hanno lavorato sui testi in inglese. Questa
traduzione, effettuata da una sola persona ma con la collaborazione di
altre, è stata realizzata del tutto indipendentemente dalle precedenti
traduzioni italiane al fine di uniformare -- come auspicato pure nella
\emph{Nota al testo} dell'edizione inglese -- le variazioni di stile e
di vocabolario.

In apertura sono state tradotte entrambe le \emph{Prefazioni}, sia
quella di Ajahn Sumedho, presente nell'edizione in tre volumi, sia
l'altra di Ajahn Munindo, disponibile invece nell'edizione in volume
unico\emph{.} L'\emph{Introduzione} di Ajahn Amaro è tradotta dal testo
ampliato, rivisto dallo stesso autore e pubblicato successivamente a
parte con il titolo \emph{An Introduction to the Life and Teachings of
Ajahn Chah} (Amaravati Publications 2012).

Negli \emph{Insegnamenti} i termini in lingua pāli sono stati resi in
corsivo -- ad eccezione di ``Buddha'', ``Dhamma'', ``Saṇgha'',
``Vinaya'', che invece sono in tondo -- e compaiono nella forma tematica
oppure al nominativo singolare o plurale, in base al contesto. Questi
termini, insieme ad altri concetti di rilievo, sono brevemente spiegati
alla fine del volume, nel \emph{Glossario}, e nelle note a pié di
pagina, che da quest'ultimo sono per lo più riprese. Sia nel
\emph{Glossario} sia nelle note predomina invece la forma tematica, un
fatto che spiega alcune difformità tra il testo e le note. Per quanto
concerne il \emph{Glossario}, in alcuni casi si è ritenuto necessario
ampliare qualche voce e aggiungerne altre.

Fra i tanti che hanno contribuito a questa edizione italiana, i lettori
siano grati anche a bhikkhu Mahāpañño, che ha rivisto il testo con
grande accuratezza, a Mario Bracchetti e a Sara Bellettato, che hanno
contribuito alla revisione e, infine, ad Antonella Serena Comba, che ha
collaborato alla stesura del \emph{Glossario}.

