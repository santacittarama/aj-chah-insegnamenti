\chapter{Introduzione}

\section{Una sera nel nord-est della Thailandia}

La notte sta scendendo rapidamente. La foresta risuona dell'ondoso
brusio di innumerevoli grilli e dell'inquietante e crescente gemito
delle cicale tropicali. Poche stelle si intrufolano fiocamente tra le
cime degli alberi. Nella crescente oscurità, un paio di lanterne a
cherosene producono una pozza di caldo chiarore, illuminando l'area
all'aperto sottostante una capanna issata su pali di legno. Sotto il
bagliore, due dozzine di persone sono raccolte attorno a un monaco,
piccolo ma di solida costituzione, che siede a gambe incrociate su una
grande sedia di vimini. Una pace vibrante è nell'aria. Il venerabile
Ajahn Chah sta insegnando.

Il gruppo riunito è per alcuni aspetti eterogeneo. Accanto ad Ajahn Chah
-- o Luang Por,\footnote{Luang Por (in thailandese \thai{หลวงพ่อ}).
  ``Venerabile padre''; è un'espressione che viene utilizzata in
  Thailandia per rivolgersi ai monaci anziani.} venerabile padre, come è
affettuosamente chiamato dai suoi allievi -- vi sono i \emph{bhikkhu},
ossia i monaci, e i novizi. La maggior parte di loro è thailandese o
laotiana, ma ve ne sono alcuni dalla pelle chiara: un canadese, due
statunitensi, un giovane australiano e un inglese. Di fronte ad Ajahn
Chah siede una ben curata coppia di mezz'età, lui in giacca e pantaloni
e lei ingioiellata e acconciata alla moda; stanno cogliendo
l'opportunità -- lui è un membro del Parlamento thailandese e proviene
da una lontana provincia, e ora si trova in zona per questioni ufficiali
-- per venire a porgere i loro omaggi a Luang Por e per fare offerte al
monastero.

Un po' più indietro, da entrambi i lati, è sparso un consistente gruppo
di persone che abitano nei villaggi dei dintorni. Le magliette e le
bluse che indossano sono usurate e la pelle delle loro magre membra è
scura e bruciata dal sole, aggrinzita, cotta come la povera terra di
questa regione. Luang Por, da bambino, con alcuni di loro aveva giocato,
aveva catturato rane e si era arrampicato sugli alberi; altri li aveva
aiutati -- ed era stato da loro aiutato -- prima di diventare
\emph{bhikkhu} quando arrivava l'annuale turno di piantare il riso e poi
di mieterlo nei campi alla fine del monsone.

Da un lato, nei pressi del retro, c'è una professoressa di Friburgo,
giunta in Thailandia per studiare il buddhismo con un'amica del suo
gruppo di Dhamma; una monaca statunitense della sezione femminile del
monastero è venuta con lei per guidarla tra i sentieri della foresta e
per farle da traduttrice. Accanto a loro siedono tre o quattro altre
monache più anziane del monastero, che hanno deciso di cogliere
l'opportunità per venire a chiedere consiglio a Luang Por su un problema
della comunità femminile e per domandargli di visitare -- sono già
passati molti giorni dall'ultima volta che lo ha fatto -- il lato della
foresta nel quale dimorano e di offrire un discorso di Dhamma al loro
gruppo. Sono rimaste già per un paio d'ore e perciò, dopo aver prestato
omaggio, hanno preso congedo insieme alle altre visitatrici provenienti
dalla sezione monastica femminile: devono tornare prima che sia buio e
sono già un po' in ritardo.

Sempre nei pressi del retro, quasi sul limitare della pozza di luce,
siede con il volto severo un uomo sulla trentina. Per metà è girato di
lato, come se si senta a disagio e quasi che la sua presenza sia
provvisoria. È un ``duro'' del luogo, un \emph{nak leng}.\footnote{In
  thailandese \thai{นักเลง}.} Profondamente sprezzante nei riguardi di tutto
quanto possa essere religioso, seppur a denti stretti nutre tuttavia per
Luang Por un rispetto che nasce sia dalla reputazione di
imperturbabilità, forza e resistenza dell'\emph{ajahn} sia dal fatto che
se i fedeli si recano da lui è perché si tratta di una persona genuina.
«~Nell'intera provincia, è probabilmente l'unico cui valga la pena di
prostrarsi.~»

È arrabbiato e sconvolto. È infelice. Una settimana prima, il suo amato
fratello minore, che faceva parte della sua banda e insieme al quale
aveva superato mille difficoltà, s'è ammalato di malaria cerebrale e, in
pochi giorni, è morto. Da allora è come se una lancia gli avesse
trafitto il cuore e nulla al mondo ha più senso, o sapore. «~Se fosse
stato accoltellato, almeno lo avrei potuto vendicare. Che posso fare?
Rintracciare la zanzara che lo ha punto e ucciderla?~» Un amico gli ha
detto: «~Perché non vai a trovare Luang Por Chah?~» Così, eccolo qui.

Quando Luang Por giunge a un punto importante del suo discorso fa un
ampio sorriso e alza un bicchiere per illustrare la sua analogia. Ha
notato la rigida e desolata figura del giovane nell'ombra. Come se
stesse riavvolgendo il filo di una canna da pesca per catturare un pesce
forte e scaltro, presto riesce in un qualche modo a convincerlo a venire
in prima fila. Subito dopo, il \emph{nak leng} piange come un bambino,
mentre Luang Por gli tiene la testa fra le mani. L'uomo ride della sua
stessa arroganza e auto-ossessione, e capisce di non essere stato il
primo o il solo ad avere perso una persona cara: le lacrime di rabbia e
dolore si sono trasformate in lacrime di sollievo.

Tutto ciò avviene alla presenza di venti estranei, e ora l'atmosfera è
di sicurezza e fiducia. Perché sebbene le persone qui riunite provengano
da differenti ceti sociali e da diverse parti del mondo, sono tutte
accomunate in questo momento e in questo luogo dall'essere
\emph{saha-dhammika}, ``compagni di viaggio nel Dhamma'' o, per usare
un'altra espressione vernacolare buddhista, ``tutti fratelli e sorelle
nella vecchiaia, malattia e morte'', e appartengono perciò alla stessa
famiglia.

Questo genere di situazioni si verificò innumerevoli volte durante i
trent'anni d'insegnamento di Ajahn Chah. È significativo che sia nelle
più lunghe esposizioni legate a occasioni formali sia in dialoghi di tal
genere, all'impronta, il fluire dell'insegnamento e la scelta di coloro
ai quali esso doveva essere specificamente indirizzato fossero del tutto
spontanei e imprevedibili. Per molti aspetti, quando Ajahn Chah
insegnava era come un maestro musicista che guida il flusso dell'armonia
e la produce in assoluta aderenza alle caratteristiche e agli stati
d'animo delle persone con le quali si trova. Integrava le loro parole,
sentimenti e interrogativi nel crogiolo del suo cuore e lasciava che le
risposte sgorgassero liberamente.

Quale che fosse il tipo di persone che si raccoglieva attorno a lui, con
identica impassibilità poteva usare come esempio i modi giusto e
sbagliato di sbucciare un mango e, subito dopo, descrivere la natura
della Realtà Ultima. Poteva essere burbero e freddo con le persone
tronfie e, il momento successivo, incantevole e gentile con quelle
timide; oppure, raccontare una barzelletta con un vecchio amico del
villaggio e, poi, guardare negli occhi un corrotto colonnello di polizia
e parlargli con sincerità della centrale importanza dell'onestà nel
Sentiero del Buddha. Poteva rimproverare un \emph{bhikkhu} perché
indossava l'abito in modo sciatto e poi, nel volgere di pochi minuti,
lasciare che la sua stessa veste, scivolatagli dalla spalla, scoprisse
la sua pancia tonda.

Una domanda intelligente posta da un accademico alla ricerca di
discussioni filosofiche di alto livello per mostrare il suo acume,
facilmente induceva Luang Por a rimuoversi la dentiera e a passarla
all'assistente \emph{bhikkhu} affinché gli desse una pulita.
L'interlocutore avrebbe così dovuto superare la prova: il grande maestro
rispondeva al suo profondo quesito con le ampie labbra ripiegate
all'indietro, sulle gengive, prima che la dentiera, ripulita, fosse
rimessa al suo posto~\ldots{}

La maggior parte delle volte Ajahn Chah impartiva i suoi insegnamenti in
riunioni spontanee, ma offriva molto generosamente la sua saggezza anche
in occasioni più formali, come dopo la recitazione delle regole per i
\emph{bhikkhu}, oppure all'intera assemblea di laici e monaci nella
notte di settimanale osservanza lunare. Ovviamente, sia che si trattasse
di insegnamenti per il primo o per il secondo tipo di riunioni, Ajahn
Chah non pianificava mai nulla. Non una sola sillaba di ciò che
insegnava era annotata prima di iniziare a parlare. Pensava che questo
fosse un principio estremamente importante, perché il compito
dell'insegnante consisteva nel togliersi di mezzo, e lasciare che il
Dhamma sorgesse in accordo con le necessità del momento. Diceva: «~Se
non è vivo nel presente, non è Dhamma.~»

Una volta invitò il suo primo discepolo occidentale, Ajahn Sumedho, a
tenere un discorso all'assemblea di monaci e di laici del monastero
principale, il Wat Pah Pong. Fu una prova traumatizzante: si trattava
non solo di parlare a circa duecento persone abituate all'alto standard
di arguzia e di saggezza di Ajahn Chah, ma per di più in thailandese,
una lingua che Ajahn Sumedho aveva iniziato a imparare solo tre o
quattro anni prima. Nella sua mente si affollarono idee e paure. In quei
giorni stava leggendo testi riguardanti i Sei Regni della cosmologia
buddhista e i correlati stati psicologici: l'ira e i regni infernali, la
felicità sensoriale e i regni paradisiaci, e così via. Decise che
sarebbe stato un buon argomento e pensò al modo opportuno di esprimere
tutte le sue idee.

Quando giunse la notte fatidica, Ajahn Sumedho tenne il suo discorso e
ritenne che fosse andata piuttosto bene. Il giorno seguente molti membri
del Saṅgha si recarono da lui per dirgli quanto avessero apprezzato le
sue parole. Si sentì sollevato e abbastanza soddisfatto di se stesso. Un
po' di tempo dopo, in un momento di tranquillità, Ajahn Chah catturò la
sua attenzione, lo fissò negli occhi e gli disse con gentilezza: «~Non
farlo mai più.~» Questo modo di insegnare non era tipico solo di Ajahn
Chah, ma era adottato da tutta la cosiddetta Tradizione Thailandese
della Foresta. Perciò, può essere utile descrivere le caratteristiche e
le origini di questo lignaggio, per mostrare un po' il significato del
contesto dal quale scaturì la saggezza di Ajahn Chah.

\section{La Tradizione della Foresta}

In relazione alla meditazione, la Tradizione della Foresta è in un certo
qual senso addirittura precedente al Buddha. Prima dei suoi tempi, in
India e nella regione dell'Himalaya, non era inconsueto per coloro che
cercavano la liberazione spirituale abbandonare la vita delle città e
dei villaggi per recarsi nella natura incontaminata di montagne e
foreste. In quanto gesto per lasciarsi alle spalle i valori del mondo,
ciò recava in sé un senso perfettamente compiuto: la foresta era un
posto selvaggio e naturale, e lì si potevano incontrare solo criminali,
folli, emarginati e rinuncianti sulla via della spiritualità. Si
trattava di un ambiente estraneo all'influsso delle regole culturali
materialistiche, quindi ideale alla coltivazione degli aspetti
spirituali che le trascendevano.

Quando all'età di 29 anni il \emph{bodhisatta}\footnote{\emph{bodhisatta}.
  Un essere che si impegna per raggiungere il Risveglio.} abbandonò la
vita del palazzo, lo fece per trasferirsi nella foresta e addestrarsi
nelle discipline yoga praticate in quel tempo. La storia di come Egli,
insoddisfatto dagli insegnamenti dei suoi primi istruttori, li lasciò
per cercare la sua propria strada per la Liberazione è nota. Vi riuscì,
scoprendo all'ombra dell'albero della bodhi, sulla riva del fiume
Nerañjara, nel luogo ora chiamato Bodh-Gaya nello stato del Bihar in
India, quel primario fondamento della Verità che chiamò ``la Via di
Mezzo''.

Spesso si afferma che il Buddha nacque in una foresta, ottenne
l'Illuminazione in una foresta, visse e insegnò per tutta la vita in una
foresta e, infine, morì in una foresta. Quando gli era possibile
scegliere, era l'ambiente che preferiva perché, come Egli diceva: «~I
\emph{Tathāgata}\footnote{\emph{Tathāgata}. Letteralmente, ``così
  andato'', ``così venuto''.} provano diletto nei luoghi isolati.~» Il
lignaggio ora conosciuto come Tradizione Thailandese della Foresta cerca
di vivere nello spirito della via abbracciata dal Buddha stesso, e di
praticare in accordo con gli stessi criteri da Lui incoraggiati durante
la sua vita. Si tratta di un ramo della Scuola Meridionale del
Buddhismo, più comunemente denominato ``Theravāda''.

Secondo quanto ci dicono le pur approssimative narrazioni storiche,
pochi mesi dopo la morte del Buddha fu tenuto un grande concilio di
anziani per stabilire e formalizzare gli Insegnamenti -- i discorsi e le
regole monastiche -- in una vernacolare forma standardizzata detta
\emph{pālibhasa}, ``il linguaggio dei testi''. Gli insegnamenti di
Dhamma formulati in tal modo durante il secolo successivo formano il
nucleo del Canone in lingua pāli, la base comune per varie scuole
buddhiste successive. Cento anni dopo fu tenuto un secondo concilio che,
per tentare di mettere tutti d'accordo, tornò nuovamente su tutti gli
Insegnamenti.

Ovviamente fu proprio allora che, come si è scoperto, avvenne il più
rilevante scisma nel Saṅgha. La maggioranza volle modificare alcune
delle regole, anche permettendo ai monaci di usare il denaro. In
relazione ai cambiamenti proposti, un piccolo gruppo fu cauto e pensò:
«~Bene, che abbia senso o meno, vogliamo fare le cose nel modo in cui le
fecero il Buddha e i suoi primi discepoli.~» I suoi membri sono
conosciuti in sanscrito come \emph{sthavira} e in pāli come
\emph{thera}, ``anziani''. Dopo centotrenta anni circa, diedero origine
alla Scuola del Theravāda. ``Theravāda'', che significa letteralmente
``la Via degli Anziani'', da allora ha proprio questa caratteristica
costante. L'etica della tradizione può essere racchiusa nella frase
``nel bene e nel male, questa è la via fissata dal Buddha, e così noi
faremo''. Perciò, può essere riscontrato da sempre in essa un tratto
particolarmente ``conservatore''.

Com'è avvenuto per tutte le tradizioni religiose e le istituzioni
dell'uomo, col trascorrere del tempo numerosi furono i rami che
germogliarono dalla radice del Buddha. È stato detto che circa 250 anni
dopo di Lui, durante l'impero di Asoka, vi furono fino a diciotto scuole
e lignaggi in India, e forse più, con divergenti modi di vedere il
\emph{Buddha-sāsana}, la dottrina del Buddha. Un lignaggio si stabilì
nello Sri Lanka, a una certa distanza dal fermento culturale dell'India,
dove giungevano influssi religiosi dall'Occidente e dall'Oriente e
andava contestualmente verificandosi un risveglio del brahmanesimo che
si aggiungeva all'agitarsi di nuove forme del pensiero buddhista. Tale
lignaggio si sviluppò per conto proprio, e fu meno soggetto a influssi e
stimoli. Formulò i propri commenti e interpretazioni delle Scritture in
lingua pāli senza l'intenzione di sviluppare nuove forme che
rispecchiassero sollecitazioni provenienti da altre fedi, ma con lo
scopo di aggiungere alcuni dettagli. Alcuni di essi avevano
caratteristiche fiabesche, miranti a catturare il cuore della gente
comune, altri erano più filosofici e metafisici, di genere erudito.

Anche il buddhismo \emph{theravādin} tuttavia si cristallizzò.
Nonostante guerre, carestie e altri rivolgimenti culturali del
subcontinente indiano, il Theravāda è sopravvissuto fino ai nostri
giorni, soprattutto perché si consolidò originariamente nell'isola di
Sri Lanka, un rifugio più sicuro di molti altri. Anche altre furono le
scuole buddhiste che qui operarono, ma il buddhismo \emph{theravādin} fu
continuamente rigenerato e conservato quale principale religione
dell'isola.

Questo lignaggio si diffuse infine, in tempi differenti, in tutto il
sud-est asiatico, allorché furono invitati missionari dallo Sri Lanka e
dall'India. Raggiunse la Birmania e in seguito la Thailandia, la
Cambogia e il Laos e, da ultimo, arrivò da tali territori pure in
Occidente. Durante questa diffusione geografica della tradizione
\emph{theravādin}, si continuò a guardare al Canone in pāli come
normativo. Quando il lignaggio si stabilì in nuovi paesi, vi fu sempre
un forte senso di deferenza e venerazione per gli insegnamenti
originali, come pure rispetto per lo stile di vita incarnato dal Buddha
e dal Saṅgha originario, i monaci dei primi tempi che dimoravano nella
foresta.

Tale modello, utilizzato in quegli anni e da allora in poi impiegato per
così tanti secoli, ebbe ovviamente un gran numero di momenti sia propizi
sia sfavorevoli. Talora la religione si affievolì nello Sri Lanka e
allora, per rivitalizzarla, vi arrivarono monaci dalla Thailandia.
Quando poi ebbe la tendenza a smorzarsi in quest'ultimo territorio,
furono monaci provenienti dalla Birmania a rafforzarla. I seguaci del
Theravāda si supportarono vicendevolmente nei secoli e, così, questa
tradizione riuscì a mantenersi a galla per larga misura nella sua forma
originale.

Assieme alla degenerazione, un altro aspetto problematico di questi
cicli fu il successo. Quando la religione si sviluppò bene, spesso i
monasteri si arricchirono e l'intero sistema si corruppe, divenne
``obeso'' e iniziò a collassare sotto il suo stesso peso. Si verificava
allora la scissione di un gruppo che -- affermando: «~Torniamo alle cose
essenziali~» -- si allontanava nella foresta e tornava di nuovo ai
modelli originari, mantenendo i precetti monastici, praticando la
meditazione e studiando gli insegnamenti originari.

È significativo notare che questo ciclo di progresso, enfiagione,
corruzione e riforma si sia verificato numerose volte nel corso dei
secoli in molte altre nazioni buddhiste. È impressionante quanto le vite
e la pratica di luminari quali il venerabile Patrul Rimpoche nel Tibet e
il venerabile maestro Xu Yun in Cina -- entrambi della fine del XIX e
degli inizi del XX secolo -- siano in completo accordo con lo spirito
della Tradizione della Foresta. Entrambi scelsero di vivere in grande
semplicità, osservarono la disciplina monastica molto rigorosamente,
furono esperti meditanti e maestri molto dotati. Evitarono in larga
misura gli oneri gerarchici e responsabilità ufficiali, ma
inevitabilmente ascesero a posizioni di grande influenza mediante il
potere puro della saggezza e della virtù. Questo è esattamente il
modello di vita incarnato anche dai grandi maestri della Tradizione
Thailandese della Foresta.

Intorno alla metà del XIX secolo, il buddhismo in Thailandia era
caratterizzato da una ricca varietà di pratiche e tradizioni regionali,
ma lo standard generale della vita spirituale si era in certo qual modo
corrotto, la disciplina monastica si era rilassata e gli insegnamenti di
Dhamma erano fusi con caotici elementi tantrici e animistici, senza
contare il fatto che quasi nessuno praticava più la meditazione. Inoltre
-- e forse proprio questa è la cosa più importante -- la posizione
ortodossa, rappresentata da studiosi e non solo da monaci negligenti,
poco colti o confusi, affermava che non era possibile ai nostri giorni
realizzare il \emph{Nibbāna} e nemmeno conseguire i \emph{jhāna}, i vari
livelli di assorbimento meditativo. I rianimatori della Tradizione della
Foresta rifiutavano di accettarlo. Era anche una delle ragioni per cui
erano considerati dalla gerarchia ecclesiastica di quegli anni alla
stregua di dissidenti e piantagrane, mentre molti di loro, compreso
Ajahn Chah, erano conseguentemente disprezzati -- lo era anche il loro
ritornello: «~Non otterrai la saggezza dai libri~» -- dalla maggior
parte dei monaci eruditi del loro stesso lignaggio \emph{theravādin}.

È necessario precisare questo punto, per evitare che induca perplessità
il fatto che Ajahn Chah possa aver avuto un'attitudine tanto negativa
nei riguardi dello studio, soprattutto perché, in quanto appartenente
alla tradizione \emph{theravādin}, si suppone che egli dovesse invece
nutrire grande venerazione per la parola del Buddha. È in questione un
motivo cruciale, che caratterizza i monaci appartenenti alla Tradizione
della Foresta: la determinazione a focalizzare l'attenzione sullo stile
di vita e sull'esperienza personale, piuttosto che sui libri e
soprattutto sui Commenti al Canone in pāli. Si potrebbe pensare che una
tale attitudine possa essere presuntuosa e arrogante, o un'espressione
di gelosia di una mente poco colta per altre migliori: non è così, se si
riesce a comprendere che proprio le interpretazioni degli studiosi
stavano trascinando il buddhismo in un abisso. In poche parole, era
proprio il tipo di situazione a rendere il panorama spirituale maturo
per il rinnovamento: fu da questo fertile terreno che emerse la
rinascita della Tradizione della Foresta.

La Tradizione Thailandese della Foresta non sarebbe così com'è oggi se
non vi fosse stato l'influsso di un grande maestro in particolare. Si
tratta del Venerabile Ajahn Mun\footnote{Nella traduzione si è scelto di
  lasciare ``Mun'', come di solito si rinviene nei testi inglesi. Si
  avverte il lettore italiano che, però, l'esatta pronuncia thailandese
  è ``Màn''.} Bhuridatta. Nacque nel 1870 nella provincia di Ubon, dove
la Thailandia s'incontra con il Laos e la Cambogia. Allora era, e lo è
ancora, una delle zone più povere del paese, ma pure quella in cui la
durezza della terra e il carattere affabile delle persone hanno indotto
nel mondo una spiritualità di rara profondità.

Ajahn Mun era un giovane di mente vivace. Eccelleva nel \emph{mor
lam},\footnote{In thailandese \thai{หมอลำ}.} l'arte locale di comporre canzoni
popolari in versi, ed era anche fortemente attratto dalla pratica
spirituale. Subito dopo l'ordinazione a \emph{bhikkhu}, cercò il
Venerabile Ajahn Sao, uno dei rari monaci della foresta del luogo, e gli
chiese di insegnargli la meditazione. Si era pure reso conto del fatto
che una rigorosa adesione alla disciplina monastica sarebbe stata
fondamentale per i suoi progressi spirituali. Divenne discepolo di Ajahn
Sao e si dedicò alla pratica con grande vigore.

Se oggi, dal nostro punto di vista, queste due cose -- disciplina
rigorosa e meditazione -- potrebbero sembrarci scontate, allora la
disciplina era diventata piuttosto trasandata in tutta la regione e la
meditazione era considerata con grande sospetto. Probabilmente solo chi
era interessato alla magia nera era abbastanza folle per avvicinarsi
alla meditazione, e si riteneva probabile che essa conducesse alla
pazzia o causasse possessioni spiritiche.

Col tempo Ajahn Mun riuscì a spiegare con successo e a dimostrare
l'utilità della meditazione a molte persone, e divenne anche un esempio
di un più alto standard di vita per la comunità monastica. Inoltre egli
divenne il più considerato maestro spirituale della Thailandia,
nonostante il fatto che vivesse in luoghi remoti. Quasi tutti i più
esperti e venerati maestri di meditazione thailandesi del XX secolo
furono o suoi diretti discepoli o ne subirono profondamente l'influsso.
Tra essi, Ajahn Chah.

\section{Ajahn Chah}

Ajahn Chah nacque in una famiglia grande e agiata, in un villaggio della
Thailandia nord-orientale. Dietro sua stessa iniziativa, alla tenera età
di nove anni scelse di lasciare la casa paterna e andò a vivere nel
monastero del luogo. Fu ordinato novizio e, sentendo il richiamo della
vita religiosa, a vent'anni ricevette l'ordinazione completa. In quanto
giovane \emph{bhikkhu}, studiò i fondamenti del Dhamma, la disciplina e
altre scritture.

In seguito, insoddisfatto del blando standard di vita nel tempio del suo
villaggio, e desiderando una guida nella meditazione, abbandonò questi
luoghi piuttosto sicuri e intraprese la vita del \emph{bhikkhu} errante,
in continuo \emph{tudong}.\footnote{\emph{tudong} (in thailandese
  \thai{ธุดงค์}). La pratica ascetica di errare a piedi, nelle campagne, in
  pellegrinaggio o alla ricerca di posti tranquilli per ritiri solitari,
  vivendo di cibo offerto in elemosina.} Cercò vari maestri locali di
meditazione e praticò sotto la loro guida. Peregrinò per un certo numero
di anni come fanno i \emph{bhikkhu} che seguono le pratiche ascetiche,
dormendo in foreste, caverne e luoghi di cremazione, e trascorse un
breve ma illuminante periodo con lo stesso Ajahn Mun. Questa è la
descrizione di quell'incontro altamente significativo, tratta
dall'ancora inedita biografia di Luang Por Chah, \emph{Uppalamani} -- un
gioco di parole che significa sia ``Il gioiello della provincia di
Ubon'' sia ``Il gioiello nel loto'' -- scritta da Phra Ong
Neung.\footnote{La biografia di Ajahn Chah è nel frattempo stata
  pubblicata in thailandese ed è anche stata tradotta in inglese con il
  titolo \emph{Stillness Flowing. The Life and Teachings of Ajahn Chah}
  (Panyaprateep Foundation 2017); la traduzione italiana è in corso.}

\begin{quote}
Alla fine del Ritiro delle Piogge,\footnote{L'annuale periodo di tempo
  di tre mesi, che in India corrisponde a quello dei primi tre mesi
  monsonici, durante i quali i monaci hanno la regola dell'obbligo di
  residenza in monastero, un periodo che tradizionalmente è dedicato a
  una formazione più intensiva.} insieme ad altri tre monaci, un novizio
e due laici, Ajahn Chah si incamminò per tornare nell'Isan, il nord-est
della Thailandia. Interruppero il viaggio a Bahn Gor e, dopo pochi
giorni di riposo, iniziò la lunga escursione di 250 chilometri verso
nord. Il decimo giorno raggiunsero l'elegante \emph{stūpa} bianco di
That Phanom, un antico luogo di pellegrinaggio sulle rive del Mekong, e
prestarono omaggio alle reliquie del Buddha lì custodite. Continuarono
il loro itinerario per tappe, cercando monasteri della foresta ubicati
lungo il cammino, nei quali trascorrere la notte. Anche così era un
viaggio arduo, e il novizio e un laico chiesero di tornare indietro.
Quando finalmente arrivarono al Wat Peu Nong Nahny, ove dimorava il
Venerabile Ajahn Mun, il gruppo si era ridotto a soli tre monaci e un
laico.

Allorché entrarono nel monastero, Ajahn Chah fu immediatamente colpito
dall'atmosfera serena e appartata. L'area centrale, nella quale si
trovava una piccola \emph{sālā}, un luogo di ritrovo in legno, era
perfettamente ramazzata, immacolata, e i pochi monaci che furono in
grado di vedere erano silenziosamente intenti a svolgere, con grazia
misurata e composta, le loro faccende quotidiane. C'era un qualcosa nel
monastero che lo rendeva diverso da tutti gli altri nei quali era stato
prima di allora: il silenzio era stranamente denso e vibrante. Ajahn
Chah e i suoi compagni furono accolti educatamente e, dopo essere stati
informati su dove avrebbero dovuto lasciare i loro
\emph{glot},\footnote{\emph{glot} (in thailandese \thai{กลค}). Ombrello con
  una zanzariera tutt'intorno all'estremità, utilizzato sia per la
  meditazione sia come riparo dai monaci che intraprendono il
  \emph{tudong}; viene appeso ai rami degli alberi così da potercisi
  sedere sotto, al riparo dagli insetti.} fecero un bagno di benvenuto
per ripulirsi dalla sporcizia del viaggio.

Verso sera, i tre giovani monaci, con il \emph{saṅghāti}\footnote{\emph{saṅghāti},
  in thailandese \emph{sanghati} (\thai{สังฆาฏ}). La veste esterna a doppio
  strato che costituisce, assieme alla veste superiore e inferiore,
  l'abito completo da monaco; generalmente viene portata ripiegata lungo
  la spalla sinistra in situazioni cerimoniali.} ordinatamente ripiegato
sulla loro spalla sinistra e con il cuore che oscillava tra appassionata
attesa e freddo timore, si incamminarono verso la \emph{sālā} per
rendere omaggio ad Ajahn Mun. Avanzando lentamente sulle ginocchia verso
il grande maestro, affiancato da entrambi i lati dai \emph{bhikkhu} del
monastero, Ajahn Chah avvicinò una figura anziana ed esile, di presenza
invincibilmente adamantina. È facile immaginare gli occhi senza fondo di
Ajahn Mun, mentre con il suo sguardo penetrante forava Ajahn Chah,
prostratosi per tre volte e poi sedutosi più in basso a conveniente
distanza. La maggior parte dei monaci era seduta in meditazione, a occhi
chiusi; uno sedeva dietro Ajahn Mun, a poca distanza, e con un ventaglio
allontanava dolcemente da lui le zanzare della sera.

Alzando lo sguardo, Ajahn Chah notò sia la clavicola di Ajahn Mun, che
prominente al di sopra dell'abito sporgeva attraverso il pallido
incarnato, sia le sue labbra sottili che, tinte di rosso dal succo di
betel, erano in forte contrasto con la strana luminosità della sua
presenza. Seguendo un'usanza da tempo onorata tra i monaci buddhisti,
inizialmente Ajahn Mun chiese ai visitatori da quanto tempo indossavano
l'abito monastico, in quale tempio praticavano e i particolari del loro
viaggio. Avevano dubbi sulla pratica? Ajahn Chah deglutì. Sì, lui ne
aveva. Aveva cominciato a studiare i testi del Vinaya con grande
entusiasmo, ma poi si era scoraggiato. La disciplina sembrava troppo
minuziosa per essere concreta; non pareva possibile osservare ogni
singola regola. Quale criterio seguire?

Quale principio basilare, Ajahn Mun consigliò Ajahn Chah di avvalersi
dei ``Due Guardiani del Mondo'', \emph{hiri} e \emph{ottappa}, il senso
di vergogna e l'intelligente timore delle conseguenze. In presenza di
queste due virtù, tutto il resto sarebbe venuto da sé. Poi, con gli
occhi socchiusi, cominciò a parlare del triplice addestramento di
\emph{sīla}, \emph{samādhi e paññā,} delle Quattro Basi del Potere
Psichico e dei Cinque Poteri Spirituali\footnote{In pāli rispettivamente
  \emph{iddhipāda} e \emph{bala}; per questi due termini e per
  \emph{sīla}, \emph{samādhi paññā}, il cui significato sarà spiegato
  poco più avanti, si veda il \emph{Glossario}.}, mentre, man mano che
procedeva, la sua voce diveniva più potente e veloce, come se stesse
ingranando marce sempre più alte. Con autorità assoluta, descrisse ``il
modo in cui le cose sono secondo Verità'' e il Sentiero verso la
Liberazione. Ajahn Chah e i suoi compagni sedevano, completamente
rapiti. In seguito, Ajahn Chah disse che, sebbene fosse esausto per la
giornata trascorsa in viaggio, ascoltando il discorso di Dhamma di Ajahn
Mun la stanchezza scomparve e la mente gli divenne serena e chiara, e
sentì come se stesse fluttuando in aria, al di sopra del luogo in cui
sedeva. Era notte tarda quando Ajahn Mun disse che l'incontro era finito
e Ajahn Chah tornò, ardente, sotto il suo \emph{glot}.

La seconda notte Ajahn Mun diede altri insegnamenti, e Ajahn Chah
percepì che i suoi dubbi sulla pratica erano spariti. Provava una gioia
e un rapimento nel Dhamma mai sentiti in precedenza. Ora, doveva solo
mettere in pratica quanto sapeva. Uno degli insegnamenti di quelle due
sere che più lo aveva ispirato era stata l'esortazione a rendersi
\emph{sītibhūto}, ossia testimone della Verità. Però, la spiegazione più
chiarificatrice, quella che gli fornì il necessario contesto, e un
fondamento per la pratica che gli era fino a quel momento mancato, fu la
distinzione tra la mente stessa e gli stati transitori della mente che,
all'interno di essa, sorgono e scompaiono.

«~Tan Ajahn Mun disse che sono semplici stati. Se non si capisce questo,
li prendiamo per reali, per la mente stessa. Non appena egli lo disse,
le cose divennero improvvisamente chiare. Supponiamo che nella mente sia
presente la felicità; è una cosa diversa dalla mente stessa, è a un
livello differente. Se lo vedi, allora puoi fermare le cose, puoi
posarle. Quando le realtà convenzionali sono viste per quello che sono,
questa è la Verità ultima. La maggior parte delle persone mette tutto
insieme come se si trattasse della mente stessa, ma in realtà sono stati
della mente mescolati con la conoscenza di essi. Se si comprende questo,
allora non rimane molto da fare.~»

Il terzo giorno Ajahn Chah rese omaggio a Luang Por Mun e condusse di
nuovo il suo piccolo gruppo nella solitaria foresta di Poopahn. Si
lasciò alle spalle Nong Peu e non vi sarebbe più tornato, ma il suo
cuore era pieno di un'ispirazione che sarebbe rimasta con lui per il
resto dei suoi giorni.
\end{quote}

Nel 1954, dopo numerosi anni di spostamenti e di pratica, fu invitato a
stabilirsi nella fitta foresta nei pressi del suo villaggio natale, Bahn
Gor. Era un bosco disabitato, noto come luogo di dimora di cobra, tigri
e fantasmi, e per questo -- diceva -- era il posto perfetto per un
\emph{bhikkhu} della foresta. Un numero sempre maggiore di
\emph{bhikkhu}, monache e laici giunse ad ascoltare i suoi insegnamenti
e si fermò per praticare con lui: attorno ad Ajahn Chah si costituì un
grande monastero. Ora i suoi discepoli, che praticano e insegnano
meditazione, si trovano in più di 300 monasteri affiliati, presenti
nelle montagne e nelle foreste di tutta la Thailandia e d'Occidente.

Benché Ajahn Chah sia morto nel 1992, l'addestramento da lui ideato è
ancora praticato al Wat Pah Pong e nei monasteri affiliati. Di norma, la
meditazione di gruppo è praticata due volte al giorno e talvolta
l'insegnante più anziano tiene un discorso. Il cuore della meditazione,
però, sta nel modo di vita. I monaci svolgono lavoro manuale, tingono e
cuciono i loro abiti, si occupano personalmente di quanto è
indispensabile, e mantengono immacolati gli edifici e il suolo del
monastero. Vivono in modo estremamente semplice, osservano i precetti
ascetici, mangiando una volta al giorno dalla ciotola per la questua e
limitando i loro beni e abiti. Ogni \emph{bhikkhu} e ogni monaca vive e
medita in solitudine in capanne singole, sparse per tutta la foresta,
nei pressi delle quali pratica la meditazione camminata su sentieri
mantenuti ben puliti sotto gli alberi.

In alcuni monasteri occidentali, e in pochi di quelli thailandesi,
l'ubicazione del centro monastico comporta talune piccole variazioni: ad
esempio, il monastero in Svizzera si trova in un ex-albergo di legno, ai
margini di un villaggio di montagna. A parte queste differenze, il tono
certo dominante è dato proprio dallo stesso spirito di semplicità, calma
e scrupolosità. La disciplina è osservata rigorosamente, consentendo a
ognuno di condurre una vita semplice e pura in una comunità regolata
armoniosamente, nella quale virtù, meditazione e comprensione possano
essere abilmente e continuamente coltivate.

Assieme all'esperienza monastica vissuta entro i limiti di località
prestabilite, la pratica del \emph{tudong} -- errare a piedi, nelle
campagne, in pellegrinaggio o alla ricerca di posti tranquilli per
ritiri solitari -- è ancora considerata un esercizio spirituale di
centrale importanza. Sebbene le foreste stiano rapidamente scomparendo
in tutta la Thailandia, e le tigri con le altre creature selvagge che
spesso si incontravano nei \emph{tudong} del passato siano diminuite al
punto da essere quasi estinte, è ancora possibile continuare questo modo
di vita e di pratica.

Questa pratica, infatti, è stata conservata non solo da Ajahn Chah, dai
suoi discepoli e da molti altri monaci della Tradizione della Foresta in
Thailandia. È sostenuta anche dai suoi monaci e dalle sue monache in
molti paesi d'Occidente e in India. In tutte queste situazioni è ancora
osservato un rigoroso standard di vita: sostentarsi unicamente mediante
il cibo liberamente offerto dalla gente del posto durante la questua,
mangiare solo tra l'alba e mezzogiorno, non portare con sé del denaro e
non farne uso, dormire ovunque vi sia un ricovero. La saggezza è un modo
di vivere e di essere, e Ajahn Chah si applicò a preservare uno stile di
vita monastica semplice in tutte le sue dimensioni, in modo che anche
oggi le persone possano studiare e praticare il Dhamma.

\section{L'insegnamento di Ajahn Chah agli occidentali}

Secondo un racconto ben attestato e ampiamente diffuso, poco prima che
Ajahn Sumedho, da poco ordinato \emph{bhikkhu}, giungesse nel 1967 al
Wat Pah Pong per chiedere di essere addestrato sotto la guida del
maestro thailandese, Ajahn Chah iniziò la costruzione di una nuova
\emph{kuṭī} -- una capanna per la meditazione -- nella foresta. Allorché
le travi che componevano i montanti angolari furono collocate al loro
posto, uno degli abitanti del villaggio che stava collaborando alla
costruzione chiese: «~Eh, Luang Por, come mai la stiamo costruendo così
alta? Il tetto è molto più su di quanto dovrebbe.~» Era perplesso,
perché tali strutture erano di norma destinate a offrire abbastanza
spazio a una persona per viverci comodamente: le regole prevedevano
circa due metri e mezzo per tre metri, con la sommità del tetto a poco
più di due metri. «~Non ti preoccupare, non andrà sprecato~», rispose
Ajahn Chah. «~Un giorno qui verranno alcuni monaci \emph{farang} --
ossia occidentali -- e loro sono molto più alti di noi.~»

Negli anni seguenti all'arrivo di questo primo discepolo,
dall'Occidente vi fu un lento ma costante flusso di persone che
continuò a varcare i cancelli dei monasteri di Ajahn Chah. Fin
dall'inizio, egli non volle che gli stranieri fossero oggetto di un
trattamento particolare, lasciò che si adattassero al clima, al cibo e
alla cultura come meglio potevano, e decise di utilizzare tutti i loro
disagi come nutrimento per lo sviluppo della saggezza e della paziente
sopportazione, due delle qualità da lui ritenute centrali per qualsiasi
progresso spirituale.

Nonostante il primario valore attribuito a un comune e armonioso
standard di vita -- per tutta la comunità monastica, senza che gli
occidentali fossero in alcun modo ritenuti speciali -- nel 1975 le
circostanze fecero sì che fosse fondato, non lontano dal Wat Pah Pong,
il Wat Pah Nanachat: il Monastero Internazionale della Foresta, il luogo
per la pratica degli occidentali. Ajahn Sumedho e un piccolo gruppo di
altri \emph{bhikkhu} occidentali erano alla ricerca di un posto per
temprare nel fuoco le loro ciotole per la questua, e a tal fine fu loro
suggerita la foresta nei pressi del villaggio di Bung Wai. La si poteva
raggiungere a piedi dal Wat Pah Pong e vi era una gran quantità di bambù
da ardere, e inoltre vi erano dei fedeli del villaggio che erano da
lungo tempo discepoli di Ajahn Chah e che sarebbero stati ben contenti
di dare una mano. Ajahn Chah li fece andare con un sorriso e disse loro
che non vi era alcuna fretta di tornare.

Nel giro di pochi giorni gli abitanti del villaggio costruirono un
ricovero con un tetto di paglia, ove il gruppo di monaci occidentali
poteva riunirsi per i pasti e per la meditazione. Circa un mese dopo,
erano pronti a iniziare la costruzione degli edifici che avrebbero
ospitato i monaci e consentito loro di stanziarsi in quel luogo. Il
progetto fu approvato da Ajahn Chah, e questi furono gli inizi di un
monastero appositamente dedicato all'addestramento del crescente numero
di occidentali interessati a intraprendere la pratica monastica. Non
molto tempo dopo, nel 1976, Ajahn Sumedho fu invitato a recarsi a Londra
per fondare un monastero \emph{theravādin} in Inghilterra. Ajahn Chah lo
raggiunse l'anno seguente e gli diede il permesso di risiedere, insieme
a un piccolo gruppo di monaci, nell'Hampstead Buddhist Vihāra di Londra,
una casa che dava su una trafficata strada a nord della città. Pochi
anni dopo si trasferirono in campagna e vennero fondati numerosi altri
monasteri affiliati.

Da allora, molti dei primi discepoli occidentali di Ajahn Chah furono
impegnati a fondare monasteri e a diffondere il Dhamma in vari
continenti. Sorsero monasteri in Australia, Nuova Zelanda, Svizzera,
Italia, Canada e Stati Uniti. Lo stesso Ajahn Chah si recò due volte in
Europa e in America settentrionale, nel 1977 e nel 1979, e supportò con
tutto il cuore queste nuove fondazioni. Una volta disse che il buddhismo
in Thailandia era come un vecchio albero, un tempo pieno di vigore e
ricco di frutti, ma che adesso era invecchiato al punto da riuscire a
produrne solo pochi, piccoli e amari. Al contrario, paragonò il
buddhismo in Occidente a un giovane alberello, pieno di energia
giovanile e potenzialmente in crescita, ma con la necessità di essere
accudito nel modo giusto e aiutato a svilupparsi.

Allo stesso modo, durante la sua visita negli Stati Uniti nel 1979,
disse: «~L'Inghilterra è un buon posto per fondare il buddhismo in
Occidente, ma è anch'essa un luogo di antica cultura. Invece gli Stati
Uniti hanno l'energia e la flessibilità di un giovane paese -- tutto è
nuovo qui -- ed è qui che il Dhamma può veramente prosperare.~» Parlando
a un gruppo di giovani statunitensi che avevano appena aperto un centro
di meditazione buddhista, aggiunse questo ammonimento: «~Solo se non
avrete timore di sfidare i desideri e le opinioni dei vostri
discepoli\footnote{Letteralmente, ``di trafiggere i loro cuori''.}
riuscirete davvero a diffondere il Buddha-Dhamma. Se lo farete, avrete
successo; se non lo farete, se modificherete gli Insegnamenti e la
pratica per adeguarla alle abitudini correnti e alle opinioni delle
persone per un'errata volontà di compiacerli, fallirete nel vostro
dovere di essere utili nel migliore dei modi possibili.~»

Prima di descrivere i punti nodali degli insegnamenti di Ajahn Chah può
essere di giovamento, soprattutto per chi non ha familiarità con il
buddhismo \emph{theravādin} in generale, o con la Tradizione Thailandese
della Foresta in particolare, cominciare presentando qualche termine
chiave e alcuni punti di vista e concetti presenti in entrambi. Gli
insegnamenti di Ajahn Chah e il suo modo di insegnare sono da collocare
nel contesto di questa tradizione, ed è utile avere un'idea di massima
di queste radici fondamentali per capire meglio come Ajahn Chah sia
stato in grado di applicarle e illustrarle.

\section{Le Quattro Nobili Verità}

Sebbene nelle varie tradizioni siano numerosi i volumi contenenti i
discorsi del Buddha, si dice pure che il suo Insegnamento è tutto
contenuto nel suo primo discorso, quello della ``Messa in Moto della
Ruota del Dhamma'',\footnote{\emph{Dhammacakkappavattana sutta}, in
  \emph{Saṃyutta Nikaya} 56.11.} tenuto poco dopo la sua Illuminazione
nel Parco delle Gazzelle nei pressi di Varanasi per i suoi cinque
compagni asceti. In questo breve discorso -- sono necessari solo venti
minuti per recitarlo -- il Buddha espose le caratteristiche della Via di
Mezzo e le Quattro Nobili Verità. Questo insegnamento è presente in
tutte le tradizioni buddhiste, e proprio come una ghianda contiene in sé
il codice genetico di ciò che assumerà la forma di una grande quercia,
allo stesso modo si potrebbe dire che pure la miriade di insegnamenti
buddhisti derivi da questa essenziale matrice di saggezza.

Le Quattro Nobili Verità sono formulate come una diagnosi medica della
tradizione ayurvedica: i sintomi della malattia, la causa, la prognosi e
la cura. Il Buddha si avvalse sempre di strutture e forme familiari alla
gente dei suoi tempi, e in questo caso, impostò la descrizione in questo
modo.

La Prima Nobile Verità è che c'è il ``sintomo'', \emph{dukkha}:
percepiamo un senso di incompletezza, d'insoddisfazione, di sofferenza.
Ci possono essere momenti e anche lunghi periodi durante i quali
proviamo una felicità di natura ordinaria o perfino trascendente, ma
altre volte il cuore è scontento. Ciò può variare tra i due estremi di
un'ampia scala, una profonda angoscia da un lato e, dall'altro, la più
tenue sensazione che la felicità che stiamo vivendo non durerà a lungo:
tutto ciò può essere definito \emph{dukkha}.

A volte alcuni leggono questa Prima Verità travisandola, come se si
trattasse di un'affermazione assoluta, come a dire che la realtà è
\emph{dukkha} in ogni sua dimensione. L'affermazione è intesa come un
giudizio di valore per qualsiasi cosa, ma non è questo il suo
significato. Se così fosse, ciò indicherebbe che non vi è speranza di
Liberazione per nessuno, e che comprendere la Verità di come sono le
cose, il Dhamma, non potrebbe condurre alla pace e alla felicità
permanenti che, secondo l'intuizione del Buddha, tale comprensione
produce. Ciò che più conta, perciò, è che queste sono verità
\emph{nobili}, non \emph{assolute}. Sono nobili nel senso che, sebbene
siano relative, quando sono comprese ci conducono alla realizzazione
dell'Assoluto o della Realtà Ultima.

La Seconda Nobile Verità è che la causa di \emph{dukkha} è la brama
centrata sull'io, \emph{taṇhā} in pāli -- in sanscrito, \emph{trshna} --
che letteralmente significa ``sete''. È questa brama, questa avidità, la
causa di \emph{dukkha}. Può trattarsi di brama per i piaceri dei sensi,
brama di diventare qualcosa e di identificarsi con qualcosa, oppure di
non essere, desiderare di scomparire, di annullarsi o di sbarazzarsi di
qualcosa. Le dimensioni della brama sono numerose e sottili.

La Terza Nobile Verità è la prognosi, \emph{dukkha-nirodha}.
\emph{Nirodha} significa ``cessazione'' e indica che questa esperienza
di \emph{dukkha}, di incompletezza, può venir meno, può essere trascesa.
Può terminare. In altre parole, \emph{dukkha} non è una realtà assoluta,
è solamente un'esperienza temporanea, dalla quale il cuore può essere
liberato.

La Quarta Nobile Verità è quella del Sentiero, il modo in cui ci
muoviamo dalla Seconda alla Terza Verità, dalla causa di \emph{dukkha}
alla sua cessazione. La cura è il Nobile Ottuplice Sentiero, che può
essere riassunto come virtù, concentrazione e saggezza.

\section{La Legge del kamma}

Uno dei fondamenti della visione buddhista del mondo consiste
nell'inviolabilità della legge di causa ed effetto: a ogni azione
corrisponde un risultato. Questo si applica non solo al regno della
realtà fisica, ma anche -- ed è ciò che più conta -- ai regni
psicologici e sociali.

Il Buddha comprese la natura della realtà e ciò lo condusse a vedere la
connotazione morale dell'universo. Le buone azioni fanno maturare
risultati piacevoli, atti dannosi fanno maturare risultati dolorosi: la
natura funziona così. Gli effetti possono giungere subito dopo l'atto
oppure in un futuro davvero lontano, ma un effetto che riecheggerà la
causa, debole o forte che sia, seguirà necessariamente. In lingua pāli
questa diade ``azione-risultati'' è chiamata \emph{kamma-vipāka} e ha un
significato prossimo al più familiare termine sanscrito \emph{karma}.

Il Buddha chiarì che l'elemento chiave del \emph{kamma} è l'intenzione,
come affermano le parole iniziali del \emph{Dhammapada}, il testo più
famoso e amato di tutte le scritture \emph{theravādin}:

Tutto ciò che siamo è generato dalla mente.\\
È la mente che traccia la strada.\\
Come la ruota del carro segue\\
l'impronta del bue che lo traina\\
così la sofferenza ci accompagna\\
quando sventatamente parliamo o agiamo\\
con mente impura.

Tutto ciò che siamo è generato dalla mente.\\
È la mente che traccia la strada.\\
Come la nostra ombra incessante ci segue\\
così ci segue il benessere\\
quando parliamo o agiamo\\
con purezza di mente.\footnote{\emph{Dhammapada}, vv. 1-2, in
  \emph{Khuddaka Nikaya} 2.}

Questa comprensione, imparata in tenera età e data per scontata nella
maggior parte dell'Asia, risuona in varie forme nella maggior parte
degli insegnamenti di Dhamma. Sebbene sia quasi un articolo di fede nel
mondo buddhista, è certamente anche una legge che ognuno, invece di
accettarla ciecamente per fiducia nei riguardi del maestro o in quanto
imperativo culturale cui adeguarsi, può conoscere per esperienza
personale.

Allorché Ajahn Chah incontrò gli occidentali che affermavano di non
credere nel \emph{kamma} così come lui ne parlava, invece di criticarli
o di respingerli come detentori di ``errata visione'' e costringerli a
pensare come lui, si interessò al fatto che qualcuno potesse vedere le
cose in modo differente. Chiese di descrivergli come pensavano che
stessero le cose e assunse proprio quel punto di partenza per i suoi
insegnamenti.

\section{Tutto è incerto}

Un altro degli insegnamenti centrali e spesso ripetuti è quello delle
Tre Caratteristiche dell'esistenza. Dal secondo discorso tenuto dal
Buddha -- l'\emph{Anattālakkaṇa sutta}\footnote{\emph{Saṃyutta Nikaya}
  22.59.} -- e nel prosieguo per tutto il suo Insegnamento, Egli
illustrò il fatto che tutti i fenomeni, sia interni sia esterni, sia
mentali sia fisici, hanno tre qualità invariabili:
\emph{aniccā-dukkha-anattā}, impermanenza, incompletezza, non-sé. Tutto
è in costante cambiamento, nulla può essere soddisfacente e affidabile
in modo durevole, niente si può dire che sia davvero nostro e nemmeno si
può affermare chi e cosa siamo in senso assoluto. E allorché queste tre
qualità sono state viste e conosciute per esperienza diretta, si può
davvero dire che siamo all'alba della conoscenza.

\emph{Aniccā} è il primo membro della triade che forma la conoscenza, e
Ajahn Chah costantemente sottolineò per anni che la contemplazione di
tale triade è il primario varco d'accesso alla saggezza. Così afferma in
uno dei suoi discorsi, \emph{Acqua ferma che scorre}:

\begin{quote}
Quando parliamo di ``incertezza'', stiamo parlando del Buddha. Il Buddha
è il Dhamma. Il Dhamma è la caratteristica dell'incertezza. Chi vede
l'incertezza delle cose, vede quella che è la loro realtà immutabile. È
così che è il Dhamma. E questo è il Buddha. Se vedete il Dhamma vedete
il Buddha, vedendo il Buddha vedete il Dhamma. Se conoscete
\emph{aniccā}, l'incertezza, lascerete andare le cose e non vi
aggrapperete a nulla.
\end{quote}

Una caratteristica dell'insegnamento di Ajahn Chah è che, al posto di
\emph{aniccā}, egli utilizzò abitualmente la meno consueta
interpretazione di ``incertezza'', in thailandese \emph{mai
neh}.\footnote{In thailandese \thai{ไม่แน่}.} Mentre ``impermanenza'' può avere
una sfumatura più astratta o tecnica, ``incertezza'' descrive meglio ciò
che il cuore prova quando incontra la qualità del cambiamento.

\section{Scelta espressiva: ``sì'' o ``no''}

Una delle caratteristiche più suggestive degli insegnamenti del
Theravāda è che sia la Verità sia la strada che a questa conduce sono
entrambe spesso indicate parlando di ciò che esse non sono, piuttosto
che di ciò che sono. Nel linguaggio teologico cristiano si parla di
``metodo apofatico'' -- dire ciò che Dio non è -- in contrasto con il
``metodo catafatico'' -- dire ciò che Dio è.

Quest'approccio apofatico, conosciuto anche come ``via negativa'', fu
utilizzato nel corso dei secoli da un certo numero di illustri
cristiani; viene subito in mente il famoso mistico e teologo san
Giovanni della Croce. Quale esempio di tale approccio, così si procede
nella sua \emph{Salita al Monte Carmelo} per descrivere il metodo
spirituale più diretto, ossia su per la montagna: «~Nulla, nulla, nulla,
nulla e, perfino sulla Montagna, nulla.~»

Il Canone in pāli ha per molti aspetti lo stesso sapore della ``via
negativa'' e, per questo, taluni lettori hanno spesso frainteso la
visione della vita in esso contenuta come nichilistica. Niente potrebbe
essere più lontano dal vero, ma è facile comprendere come un tale errore
sia possibile, soprattutto se si proviene da una cultura impegnata ad
affermare la vita.

La storia vuole che, poco dopo l'Illuminazione, il Buddha fosse in
cammino su una strada che attraversava la campagna del Magadha per
ritrovare i cinque compagni con i quali aveva praticato in austerità
prima di andare alla ricerca della Verità da solo, per conto suo. Per
strada un altro asceta itinerante, di nome Upaka, vide avvicinarsi il
Buddha e ne fu grandemente colpito. Il Buddha aveva non solo
l'apparenza di un nobile principe guerriero per il portamento regale
che gli proveniva dalla sua educazione. Oltre a essere alto più di un
metro e ottanta era straordinariamente gentile e, benché fosse vestito
di stracci come gli asceti itineranti, risplendeva radioso. Upaka ne fu
impressionato:

\begin{quote}
«~Chi sei, amico? Il tuo volto è così chiaro e luminoso, il tuo
portamento è gentile e sereno. Certamente devi aver scoperto una qualche
grande verità. Chi è il tuo maestro, amico, e cosa hai scoperto?~»

Il Buddha, che da poco aveva conseguito il Risveglio, rispose: «~Io sono
Colui che tutto ha trasceso, il Conoscitore di tutto. Non ho maestro. In
tutto il mondo io solo sono perfettamente illuminato. Non c'è nessuno
che me l'abbia insegnato. Vi sono giunto per mezzo dei miei sforzi.~»

«~Vuoi intendere che pretendi di avere ottenuto la vittoria sulla
nascita e sulla morte?~»

«~Infatti, amico, io sono il Vittorioso; e ora, in questo mondo di
cecità spirituale, vado a Varanasi a suonare il tamburo di Ciò che Non
Muore.~»

«~Bene, buon per te amico», disse Upaka e, scuotendo il capo, andò via e
prese un'altra direzione.\footnote{\emph{Vinaya}, \emph{Mahāvagga} 1.6.}
\end{quote}

Questo incontro fece comprendere al Buddha che semplici dichiarazioni
sulla Verità non necessariamente fanno sorgere la fede e, quando si
cerca di comunicarla agli altri, possono anche non essere efficaci.
Così, quando raggiunse il Parco delle Gazzelle nei pressi di Varanasi e
incontrò i suoi precedenti compagni, Egli adottò un metodo molto più
analitico -- \emph{vibhajjāvada}, in pāli -- e così formulò le Quattro
Nobili Verità. Ciò rifletteva lo spostamento di piano dall'espressione
``io ho realizzato la completezza'' a ``investighiamo affinché tutti
conoscano l'incompletezza''.

Nel secondo discorso del Buddha -- l'\emph{Anattālakkaṇa sutta} -- che
fu pure pronunciato nel Parco delle Gazzelle nei pressi di Varanasi e
che indusse tutti e cinque i suoi compagni a realizzare l'Illuminazione,
tale metodo della ``via negativa'' si mostra con grandissima chiarezza.
Non è questo il luogo per analizzare dettagliatamente questo
\emph{sutta},\footnote{\emph{sutta}. Letteralmente, ``filo''. Un
  discorso o sermone del Buddha o dei discepoli suoi contemporanei.}
però, in breve, potremmo dire che il Buddha utilizza come tema la
ricerca del sé -- \emph{attā} in pāli, \emph{ātman} in sanscrito -- e,
avvalendosi di un metodo analitico, dimostra che un ``sé'' non può
essere rintracciato in alcun elemento del corpo o della mente.

Dopo averlo dimostrato, il Buddha afferma: «~Il saggio e nobile
discepolo diventa distaccato nei riguardi del corpo, delle sensazioni,
delle percezioni, delle formazioni mentali e della coscienza.~» Così, il
cuore si libera. Una volta che lasciamo andare ciò che non siamo, appare
la natura di ciò che è reale. E siccome quella realtà è al di là di ogni
descrizione, è più opportuno e meno fuorviante non descriverla: questa è
l'essenza della ``via della negazione''.

Soprattutto nella tradizione \emph{theravādin}, la parte del leone
nell'insegnamento del Buddha la fanno l'indicazione della ``natura'' del
Sentiero e il miglior modo di percorrerlo, non una magnificazione
poetica della meta finale. Per gran parte, questo è vero anche per lo
stile di Ajahn Chah. Egli evitò quanto più possibile di parlare dei
livelli di conseguimento e di assorbimento meditativo, sia per
contrastare il materialismo spirituale -- progresso mentale,
competitività e gelosia -- sia per far sì che gli occhi della gente
guardassero verso ciò di cui più aveva bisogno: il Sentiero.

Ajahn Chah, quando l'occasione lo richiedeva, parlava con notevole
prontezza e immediatezza della Realtà Ultima, indipendentemente dal
fatto che quanti erano riuniti per ascoltarlo fossero giovani o anziani,
laici o monaci. Ovviamente, se riteneva che in una persona non ci fosse
sufficiente maturità per comprendere -- anche in questo caso non
importava se avesse ricevuto o meno l'ordinazione monastica -- e
insisteva nel porre domande su questioni riguardanti la Trascendenza,
egli poteva rispondere come fece una volta, quando gli venne chiesto se
ci fosse qualcosa oltre ai cinque \emph{khandhā}, ossia oltre alla
convenzione mente-corpo. «~Non è nulla e non lo chiamiamo nulla, questo
è tutto quello che ci deve essere. Piantatela con tutto.~»
Letteralmente: «~Se lì non c'è niente, allora datelo semplicemente in
pasto ai cani!~»

\section{L'enfasi sulla Retta Visione e sulla Virtù}

Se gli si chiedeva quali fossero per lui gli elementi essenziali
dell'insegnamento, spesso Ajahn Chah rispondeva che la sua esperienza
gli aveva mostrato che ogni progresso spirituale dipendeva dalla Retta
Visione e dalla purezza della condotta. Della Retta Visione, una volta
il Buddha disse: «~Non vi è fattore più utile della Retta Visione per
far sorgere stati mentali benefici.~»\footnote{\emph{Anguttura Nikaya}
  1.16.2.}

Instaurare la Retta Visione significa in primo luogo avere un'affidabile
mappa del terreno della mente e del mondo -- soprattutto per valutare
tenendo conto della legge del \emph{kamma} -- e, in secondo luogo,
osservare l'esperienza alla luce delle Quattro Nobili Verità, per poi
trasformare quel fluire di percezioni, pensieri e umori in combustibile
per la visione profonda.\footnote{Per uniformarsi ad una interpretazione
  diffusa, con ``visione profonda'' viene tradotto qui come altrove nel
  testo italiano il termine inglese ``insight''.} Questi quattro cardini
diventano le direzioni della bussola che utilizziamo per orientare la
nostra comprensione e, perciò, per guidare le nostre azioni e
intenzioni.

Ajahn Chah pensava che \emph{sīla}, la virtù, fosse il gran protettore
del cuore e incoraggiava un sincero impegno nei Precetti da parte di
tutti coloro che prendevano seriamente la ricerca della felicità e
miravano a una vita sapientemente vissuta, sia che fossero in questione
i Cinque Precetti dei laici o gli Otto, Dieci o 227 Precetti dei vari
livelli della comunità monastica.\footnote{Si veda la voce
  \emph{Precetti} nel \emph{Glossario.}} Azioni e linguaggio virtuosi --
\emph{sīla} -- mettono direttamente il cuore in sintonia con il Dhamma e
divengono così il fondamento per la concentrazione, per la visione
profonda e, infine, per la Liberazione.

Per molti aspetti \emph{sīla} è il corollario esteriore delle qualità
interiori della Retta Visione, e tra loro vi è una relazione di
reciprocità. Se comprendiamo la causalità e vediamo le relazioni tra
brama e \emph{dukkha}, le nostre azioni avranno allora certo una
maggiore possibilità di essere armoniose e contenute; similmente, se le
nostre azioni e il nostro linguaggio sono rispettosi, onesti e non
violenti, creiamo dentro di noi i presupposti della pace e ci risulterà
molto più agevole vedere le leggi che governano la mente e come queste
funzionino, così che la Retta Visione si svilupperà con maggiore
facilità.

Uno dei risultati specifici di questa relazione di reciprocità -- Ajahn
Chah ne parlò costantemente -- risiede nel fatto che, nonostante
l'intrinseca vacuità di tutte le convenzioni, quali ad esempio il
denaro, il monachesimo, i costumi sociali, esse necessitano comunque di
essere del tutto rispettate. Ciò può suonare in un certo qual modo
paradossale, ma egli considerò la Via di Mezzo come sinonimo per
risolvere questo enigma. Se ci attacchiamo alle convenzioni, esse
saranno gravose e ci limiteranno, ma se cerchiamo di sfidarle o di
negarle ci sentiremo perduti, in conflitto e confusi. Egli vide che con
il giusto atteggiamento entrambi tali aspetti potevano essere evitati in
un modo naturale e liberatorio, né forzato né compromissorio.

Fu probabilmente a causa della sua profonda comprensione di tutto questo
che Ajahn Chah fu in grado di essere come monaco buddhista sia
straordinariamente ortodosso e austero sia completamente rilassato e
libero dalle stesse regole che osservava. Molti di coloro che lo
incontrarono ebbero l'impressione che egli fosse la persona più felice
del mondo, forse un'ironia per un uomo che mai nella sua vita aveva
provato il sesso, non aveva denaro, non aveva mai ascoltato musica, era
sempre a disposizione della gente da diciotto a venti ore al giorno,
dormiva su una sottile stuoia, era diabetico e affetto da varie forme di
malaria, e si deliziava del fatto che il Wat Pah Pong fosse considerato
il posto dove il cibo era il peggiore del mondo.

\section{Insegnare ai laici, insegnare ai monaci}

Le occasioni in cui gli insegnamenti di Ajahn Chah potevano essere
applicati sia ai laici sia ai monaci erano certo numerose, ma vi erano
anche molti altri casi nei quali non era così. Una tale distinzione non
era dovuta al fatto che alcuni insegnamenti fossero ``segreti'' o per
certi versi più ``alti'', ma piuttosto alla necessità di parlare in modi
che fossero appropriati e utili per chi di volta in volta si trovava ad
ascoltare.

Rispetto ai monaci, i praticanti laici avrebbero ovviamente avuto una
diversa gamma di preoccupazioni e condizionamenti durante la loro vita
quotidiana: per esempio, cercare di trovare il tempo per praticare la
meditazione formale, conservare una fonte di reddito, vivere in coppia.
Inoltre, più in particolare, la comunità laica non si era impegnata nei
voti per una vita di rinuncia. Un discepolo laico di Ajahn Chah si
sarebbe mediamente impegnato nello standard di rispettare i Cinque
Precetti mentre, in contesto monastico, i Precetti erano Otto, Dieci o
227 a seconda dei vari livelli della comunità religiosa.

Insegnando solo ai monaci, era molto più rilevante lo specifico utilizzo
della vita di rinuncia quale metodo chiave di addestramento; la
formazione avrebbe perciò coinvolto gli ostacoli, le insidie e le glorie
connesse a quel genere di vita. Dal momento che l'età media dei
componenti di una comunità monastica in Thailandia si aggira di norma
tra i 25 e i 30 anni, e che i precetti concernenti la castità sono
osservati in modo estremamente severo, vi era una naturale necessità per
Ajahn Chah di orientare l'irrequietezza e l'energia sessuale di sovente
sperimentate dai suoi monaci. Se ben indirizzati, i singoli sarebbero
stati in grado di contenere e di impiegare quell'energia, e di
trasformarla per contribuire a sviluppare concentrazione e saggezza.

I toni di alcuni dei suoi discorsi ai monaci potrebbero in qualche caso
essere considerati ben più aspri di quelli rivolti alla comunità laica.
Questo modo di esprimersi rappresenta un aspetto del caratteristico
stile ``senza compromessi'', tipico di molti maestri della Tradizione
Thailandese della Foresta. È un modo di parlare che mira a risvegliare
il ``cuore guerriero'', quell'atteggiamento nei riguardi della pratica
spirituale che rende pronti a sopportare ogni difficoltà, saggi,
pazienti e fedeli, indipendentemente da quanto le cose si facciano
difficili.

Talora i toni di un tal modo di esprimersi possono risultare troppo duri
o combattivi; chi ascoltava questi insegnamenti teneva ovviamente fermo
nella mente che lo spirito soggiacente a un tale linguaggio mirava
sempre a incoraggiare, ad allietare il cuore e a fornire energia di
supporto per affrontare le multiformi sfide necessarie per liberare il
cuore da ogni avidità, odio e illusione. Come Ajahn Chah disse una
volta: «~Tutti coloro che si impegnano seriamente nella pratica
spirituale devono attendersi di sperimentare una gran quantità di
attriti e difficoltà.~» Il cuore viene addestrato per andare contro
l'intensa corrente delle abitudini incentrate sul sé, ed è perciò
naturale che sia sballottato.

Per concludere su quest'aspetto degli insegnamenti di Ajahn Chah, in
particolare quelli che si possono definire ``più alti'' o
``trascendenti'', significativamente egli non ritenne che ai monaci
fosse riservato un qualcosa di specifico. Se sentiva che un gruppo di
persone era pronto per il più alto livello d'insegnamento, lo impartiva
in modo libero e aperto. Ad esempio, in uno dei suoi discorsi per un
gruppo di laici osservò: «~Di questi tempi la gente va lontano per
studiare, in cerca del bene e del male. Ma nulla sanno di ciò che è al
di là del bene e del male~» e proseguì offrendo esaustive istruzioni per
trascendere tale dualismo. Come il Buddha, Ajahn Chah non era un
``maestro dal pugno chiuso'',\footnote{\emph{Maha Parinibbana sutta}, in
  \emph{Digha Nikaya} 16.} un maestro che trattiene qualcosa per sé, e
faceva le sue scelte su cosa insegnare sulla sola base di ciò che
sarebbe stato utile ai suoi ascoltatori, indipendentemente dal numero
dei precetti che osservavano e della loro affiliazione religiosa,
ammesso che ne avevessero una.

\section{Contrastare la superstizione}

Una delle caratteristiche che più rese noto Ajahn Chah era la sua
arguzia nel dissolvere la superstizione che in Thailandia è connessa
alla pratica buddhista. Egli criticò fortemente i ciondoli magici, gli
amuleti e la divinazione che tanto pervadono quella società. Raramente
parlò di vite passate o future, di altri regni dell'esistenza e di
esperienze psichiche. Chiunque si recasse da lui per chiedergli un
suggerimento sul prossimo numero vincente della lotteria -- una
richiesta molto comune dei thailandesi che vanno a trovare famosi
maestri -- otteneva in genere scarsissima attenzione.

Egli pensava che il Dhamma stesso fosse il gioiello più inestimabile per
fornire autentica protezione e sicurezza nella vita, un gioiello che era
però continuamente trascurato per l'ottenimento di lievi miglioramenti
nel \emph{saṃsāra}. Mosso da un genuino sentimento di gentilezza per gli
altri, sottolineò ripetutamente l'utilità e la fattibilità della pratica
buddhista, contrastando la comune credenza che il Dhamma fosse troppo
elevato o astruso per una persona comune. Le sue critiche miravano non
ad abbattere infantili dipendenze da buona sorte e magici amuleti. Egli
piuttosto voleva che le persone investissero in qualcosa che si sarebbe
rivelato di vera utilità.

Alla luce di questo impegno durato tutta una vita, nel 1993 circostanze
dai risvolti ironici accompagnarono il suo funerale. Egli morì il 16
gennaio del 1992 e il suo funerale si svolse esattamente un anno dopo.
Lo \emph{stūpa} commemorativo ebbe 16 colonne, fu alto 32 metri e venne
dotato di fondamenta profonde 16 metri. Di conseguenza, un gran numero
di persone della provincia di Ubon acquistò biglietti della lotteria che
recassero contemporaneamente i numeri uno e sei. Il giorno dopo i titoli
dei quotidiani locali proclamarono: «~L'ultimo regalo di Luang Por Chah
ai suoi discepoli. I 16 hanno fatto piazza pulita e qualche
scommettitore è perfino andato in bancarotta.~»

\section{Umorismo}

Questo aneddoto ci conduce infine a un'altra caratteristica dello stile
d'insegnamento di Ajahn Chah. Egli era un uomo sorprendentemente arguto,
un attore per natura. Benché potesse essere sia davvero freddo e
minaccioso sia sensibile e gentile nei suoi modi di esprimersi,
nell'insegnamento egli utilizzò anche un alto grado di umorismo. Aveva
un modo tutto suo di far lavorare l'arguzia nei cuori dei suoi
ascoltatori, non tanto per divertire, ma per facilitare la trasmissione
di verità che altrimenti non sarebbero state accolte così facilmente. Il
suo spirito e il suo occhio, esperti nelle tragicomiche assurdità della
vita, consentivano alle persone di vedere le situazioni in modo da poter
ridere di se stesse, guidate da una più saggia prospettiva.

Ciò poteva avvenire a riguardo del comportamento, come in una sua famosa
esibizione sui numerosi modi sbagliati in cui i monaci portano la
\emph{yarm}\footnote{\emph{yarm} (in thailandese \thai{ย่าม}). Borsa tipica
  utilizzata dai monaci.} -- a tracolla sulla schiena, avvolta attorno
al collo, stretta nel pugno, trascinata sul terreno -- oppure \ldots{} in
relazione a qualche dolorosa lotta personale. Una volta un giovane
\emph{bhikkhu} andò da lui davvero abbattuto. Aveva visto le pene del
mondo e l'orrore degli esseri intrappolati nella nascita e nella morte,
e aveva deciso: «~Non sarò mai più in grado di ridere, tutto è così
triste e doloroso.~» Dopo tre quarti d'ora, grazie a una vignetta su un
giovane scoiattolo che cadeva in continuazione durante i suoi sforzi per
imparare ad arrampicarsi sugli alberi, il monaco, scosso da una risata
che sembrava non dover più cessare, si rotolava sul pavimento
stringendosi i fianchi, mentre le lacrime gli scendevano in volto.

\section{Gli ultimi anni}

Durante il Ritiro delle Piogge del 1981 Ajahn Chah si ammalò gravemente,
sembrerebbe per una qualche forma di colpo apoplettico. Negli ultimi
anni la sua salute era stata traballante, aveva avuto vertigini e
problemi di diabete, ed era giunto il crollo. Nei mesi immediatamente
successivi ricevette vari tipi di cure, incluse alcune operazioni, ma
tutto questo non servì a nulla. Peggiorò continuamente, fino a che,
intorno alla metà dell'anno seguente, a parte qualche piccola
possibilità di movimento per una mano, si paralizzò e perse la facoltà
della parola. Poteva ancora battere le palpebre.

Rimase in queste condizioni per dieci anni, ma diminuirono lentamente le
poche aree del corpo che poteva controllare, fino a che andò perduta
ogni possibilità di movimento volontario. Durante questo periodo si
disse spesso che egli stava ancora insegnando ai suoi discepoli: non
aveva incessantemente ripetuto che ammalarsi e decadere è nella natura
del corpo, e che nessuno può esercitare su di esso alcun controllo?

Ebbene, è esattamente in questione proprio una lezione fondamentale: né
un grande maestro e nemmeno il Buddha stesso possono sfuggire alle
inesorabili leggi della natura. Come sempre, il compito è quello di
trovare pace e libertà mediante la non identificazione con le forme in
mutamento.

Durante questo periodo, nonostante le sue gravi limitazioni, Ajahn Chah
riuscì occasionalmente a insegnare non solo in quanto esempio
dell'incertezza dei processi della vita e offrendo ai suoi monaci e
novizi l'opportunità di fornirgli assistenza infermieristica. I
\emph{bhikkhu} erano soliti lavorare a turno, tre o quattro per volta,
per provvedere alle esigenze fisiologiche di Ajahn Chah, che necessitava
di assistenza giornaliera ventiquattr'ore su ventiquattro.

Durante un turno di assistenza due monaci si misero a discutere,
dimenticandosi completamente -- come spesso avviene attorno a persone
paralizzate o in stato comatoso -- che l'altro occupante della stanza
potesse essere del tutto conscio di quel che stava accadendo. Se Ajahn
Chah fosse stato completamente attivo, sarebbe stato impensabile che si
fossero messi a battibeccare di fronte a lui. Man mano che le parole si
facevano più roventi, un moto di agitazione iniziò a palesarsi nel letto
e attraversò la stanza. Improvvisamente Ajahn Chah tossì in modo
violento e -- secondo i racconti -- un consistente grumo di muco,
attraversando la stanza, passò attraverso i due monaci e andò a
schioccare sul muro proprio accanto a loro. L'insegnamento era stato
debitamente impartito e la discussione si concluse in modo brusco e
imbarazzato.

Durante il decorso della malattia, la vita dei monasteri continuò come
prima. Il fatto che il Maestro ci fosse e al tempo stesso non ci fosse
contribuì in uno strano modo ad aiutare la comunità ad adattarsi a
prendere decisioni collegiali e a concepire la vita monastica senza che
l'amato insegnante fosse al centro di tutto. Dopo la morte di un così
grande anziano, non è inusuale che le cose si dissolvano rapidamente e
che i discepoli vadano ognuno per la propria strada, così che la sua
eredità svanisce nel corso di una o due generazioni. È forse una
testimonianza di quanto Ajahn Chah abbia ben addestrato le persone a
essere autosufficienti il fatto che, quando si ammalò, i monasteri
affiliati erano circa 75 e che, dopo il suo decesso, crebbero a più di
100, mentre ora sono aumentati a più di 300 in Thailandia e in tutto il
mondo.

Dopo la sua scomparsa nel 1992, la sua comunità monastica organizzò il
funerale. Conservando lo spirito della sua vita e del suo insegnamento,
questo evento non fu solo una cerimonia, ma anche un'occasione per
ascoltare e per praticare il Dhamma. Durò dieci giorni e più, con
numerosi periodi di meditazione di gruppo e discorsi quotidiani
d'istruzione, tenuti da molti dei più esperti e realizzati insegnanti di
Dhamma thailandesi.

Circa 6.000 monaci, 1.000 monache e più di 10.000 laici si accamparono
nella foresta nel periodo durante il quale si svolse la pratica. Oltre a
costoro, circa un milione di persone giunse al monastero. In 400.000,
compresi il re e la regina e il primo ministro della Thailandia, vennero
nel giorno della sua cremazione.

Nello spirito dei principi esposti da Ajahn Chah nel corso di tutta la
sua attività d'insegnamento, per tutto questo tempo non venne richiesto
un solo centesimo: il cibo fu gratuitamente offerto a tutti grazie a 42
cucine, gestite e approvvigionate da molti dei monasteri affiliati;
furono regalati libri di Dhamma per un valore superiore a 200.000 euro;
una ditta del posto distribuì tonnellate di acqua imbottigliata e i
proprietari delle compagnie locali di autobus e di autotrasporti si
incaricarono di portare ogni mattina i monaci per la questua nei
villaggi e nelle città delle vicinanze. Fu una grande festa della
generosità e un modo appropriato per dire addio a un grand'uomo.

Ajahn Amaro

