\chapter{Trascendenza}

\begin{openingQuote}
  \centering

  Discorso offerto in una notte di osservanza lunare (uposatha) al\\
  Wat Pah Pong, nel 1975.
\end{openingQuote}

Quando il gruppo dei cinque asceti\footnote{Ajahn Chah si riferisce a un
  episodio della vita del Buddha, quando Egli, ancora alla ricerca
  dell'Illuminazione, fu abbandonato dai cinque asceti perché -- come
  ben spiega l'\emph{ajahn} nel testo -- nel corso della sua pratica
  spirituale iniziò a percorrere una via nuova, la ``Via di Mezzo'', del
  tutto differente rispetto ai tradizionali approcci.} abbandonò il
Buddha, Egli pensò che fosse un colpo di fortuna, perché così avrebbe
potuto continuare la sua pratica senza impedimenti. La situazione non
era molto tranquilla quando i cinque asceti vivevano con il Buddha, Lui
aveva delle responsabilità. Lo avevano lasciato perché pensavano che
avesse allentato la sua pratica e fosse tornato a indulgere ai piaceri
sensoriali. Precedentemente si era dedicato alle pratiche ascetiche e
all'auto-mortificazione. Nei riguardi del mangiare, del dormire e così
via, si era tormentato severamente, ma giunse il momento in cui,
guardando con onestà dentro di sé, Egli vide che queste pratiche non
funzionavano. Si trattava solo di opinioni, era praticare partendo da
orgoglio e attaccamento. Aveva frainteso dei valori mondani e fatto
confusione tra se stesso e la Verità.

Se ad esempio decidiamo d'impegnarci nelle pratiche ascetiche con
l'intenzione d'essere elogiati, questo genere di pratica è del tutto
``ispirata dal mondo'', significa praticare per adulazione e fama.
Praticare con questo tipo d'intenzione è detto ``scambiare le vie del
mondo con la Verità''. Un altro modo di praticare è ``scambiare le
proprie opinioni con la Verità''. Si crede solo in se stessi, nella
propria pratica. Non importa quel che dicono gli altri, si resta
attaccati alle proprie preferenze. Non si esamina con attenzione la
pratica. Questo è chiamato ``scambiare se stessi con la Verità''. Se
considerate il mondo o voi stessi come Verità, si tratta solo di cieco
attaccamento. Il Buddha lo capì, e capì che non c'era ``adesione al
Dhamma'', al praticare per la Verità. Per questo la sua pratica era
stata infruttuosa, perché non aveva ancora abbandonato le
contaminazioni.

Allora si voltò indietro e ripensò in termini di risultati a tutto
l'impegno applicato alla pratica fin dagli inizi. Quali erano i
risultati di tutta quella pratica? Guardando a fondo in essa, vide che
non andava bene. Era colma di presunzione, colma di mondo. Non vi era
alcun Dhamma, alcuna visione profonda nel non-sé, \emph{anattā}, alcuna
vacuità o lasciar andare. Ci poteva essere stato un lasciar andare di un
certo tipo, ma era proprio quel certo tipo di attaccamento al lasciar
andare che non era ancora stato lasciato andare.

Considerando attentamente la situazione, il Buddha comprese che pure se
avesse spiegato tutto questo ai cinque asceti, loro non sarebbero stati
in grado di capire. Non era una cosa che poteva essere trasmessa con
facilità, perché quegli asceti erano radicati nel vecchio modo di
praticare e di vedere le cose. Il Buddha capì che si può praticare in
quel modo fino al giorno della morte, forse anche morire di fame, senza
ottenere nulla, perché una tale pratica è ispirata da valori mondani e
dall'orgoglio.

Valutando a fondo le cose, vide la retta pratica, \emph{sammā-paṭipadā:}
la mente è la mente, il corpo è il corpo. Il corpo non è desiderio o
contaminazione. Anche se distruggete il corpo, non distruggete le
contaminazioni. Non è la fonte di esse. Le contaminazioni non si erano
esaurite nemmeno digiunando e restando senza dormire finché il corpo era
divenuto uno spettro avvizzito. L'opinione che le contaminazioni
potessero essere eliminate in quel modo, l'insegnamento
dell'auto-mortificazione, era invece profondamente radicata nei cinque
asceti.

Il Buddha iniziò allora ad assumere più cibo, a mangiare normalmente, a
praticare in modo più naturale. Quando i cinque asceti videro questo
cambiamento nella pratica del Buddha, pensarono che Egli avesse
rinunciato e fosse tornato a indulgere ai piaceri dei sensi. La
comprensione del Buddha si era librata su un piano più alto, ma gli
altri pensavano che la sua prospettiva fosse scivolata verso il basso,
che fosse tornata alle comodità. L'auto-mortificazione era profondamente
radicata nelle menti dei cinque asceti, perché il Buddha in precedenza
aveva insegnato e praticato in quel modo. Capì l'errore. Vedendo con
chiarezza l'errore nella sua pratica, fu in grado di lasciarlo andare.

Quando i cinque asceti notarono che il Buddha si comportava in questo
modo, lo abbandonarono pensando che non lo avrebbero più seguito perché
stava praticando in modo errato. Come gli uccelli abbandonano un albero
che non offre più abbastanza ombra o come un pesce lascia una pozza
d'acqua troppo piccola, sporca o non fresca, allo stesso modo i cinque
asceti abbandonarono il Buddha.

Il Buddha si concentrò allora nella contemplazione del Dhamma. Mangiò
meglio e visse più in modo più naturale. Lasciò che la mente fosse solo
la mente e che il corpo fosse solo il corpo. Non forzò eccessivamente la
pratica, lo fece solo abbastanza per allentare la presa dell'avidità,
dell'avversione e dell'illusione. Prima aveva percorso i due estremi. Se
fossero sorti felicità e amore Egli sarebbe stato soggetto
all'eccitazione e si sarebbe attaccato a essi, si sarebbe identificato
con essi e non avrebbe lasciato andare. Se avesse incontrato piacere, vi
si sarebbe attaccato, se avesse incontrato sofferenza, vi si sarebbe
attaccato. Questi due estremi li chiamò \emph{kāmasukhallikānuyogo} e
\emph{attakilamathānuyogo}.

Il Buddha era rimasto bloccato sui fenomeni condizionati. Vide con
chiarezza che queste due vie non sono quelle di un \emph{samaṇa}.
Attaccarsi alla felicità, attaccarsi alla sofferenza: un \emph{samaṇa}
non fa così. Attaccarsi a queste cose non è la Via. Attaccandosi a
queste cose era rimasto bloccato nelle prospettive del sé e del mondo.
Se fosse rimasto a brancolare lungo queste due vie, non sarebbe mai
diventato uno che conosce il mondo con chiarezza. Sarebbe sempre rimasto
a correre da un estremo all'altro. Il Buddha fissò allora l'attenzione
sulla mente stessa e si dedicò ad addestrarla.

Tutte le sfaccettature della natura si regolano in base alle condizioni
che le supportano e che, di per sé, non costituiscono alcun problema. Le
\mbox{malattie} del corpo, ad esempio. Il corpo sperimenta dolore, malattia,
febbre e raffreddori, e così via. Tutto ciò avviene naturalmente. In
effetti, la gente si preoccupa troppo del proprio corpo. Le persone si
preoccupano e si attaccano così tanto al loro corpo a causa dell'errata
visione, non riescono a lasciar andare.

Guardate questa sala. La costruiamo e diciamo che è nostra, ma le
lucertole arrivano qui e ci vivono, topi e gechi arrivano qui e ci
vivono, e continuiamo a scacciarli perché pensiamo che appartenga a noi,
non ai topi e alle lucertole. Lo stesso avviene con le malattie del
corpo. Consideriamo questo corpo come la nostra casa, un qualcosa che
davvero ci appartiene. Se capita che abbiamo un mal di testa o un mal di
stomaco ci agitiamo, non vogliamo il dolore e la sofferenza. Queste
gambe sono le ``nostre gambe'', non vogliamo che ci facciano male.
Queste braccia sono le ``nostre braccia'', non vogliamo che ci sia
qualcosa che in esse non funziona. Dobbiamo curare tutti i dolori e
tutte le malattie, a ogni costo.

È qui che ci inganniamo e che ci allontaniamo dalla Verità. Noi siamo
solo ospiti di questo corpo. Proprio come questa sala non è davvero
nostra. Siamo solo inquilini temporanei, come i topi, le lucertole e i
gechi, ma non lo sappiamo. Per questo corpo è lo stesso. Il Buddha ha
insegnato che in realtà non c'è alcun sé che dimora dentro questo corpo,
ma noi ci aggrappiamo a esso come se fosse il nostro sé, come se fosse
davvero ``noi'' e ``loro''. Quando il corpo cambia, non vogliamo che lo
faccia. Non importa quanto ci è stato detto, non comprendiamo. Se ve lo
dico in modo diretto, v'ingannate ancora di più. Io dico «~Questo non è
il tuo sé~», e voi vi sentite più smarriti e confusi, e la vostra
pratica non fa altro che rinforzare l'io.

La maggior parte della gente non vede il sé. Uno che vede il sé è uno
che vede che ``questo non è né il sé né appartiene al sé''. Vede il sé
quale è nella natura. Vedere il sé attraverso la forza
dell'attaccamento non è vero vedere. L'attaccamento interferisce con
tutta la questione. Non è facile comprendere com'è in realtà questo
corpo, perché \emph{upādāna} si aggrappa velocemente a tutto.

Per questa ragione ci viene detto che dobbiamo investigare per conoscere
chiaramente, con saggezza. Ciò significa che per investigare i
\emph{saṅkhāra} in coerenza con la loro vera natura bisogna usare la
saggezza. Conoscere la vera natura dei \emph{saṅkhāra} è saggezza. Se
non conoscete la vera natura dei \emph{saṅkhāra} siete in contrasto con
essi, resistete in continuazione. È meglio lasciar andare i
\emph{saṅkhāra}, piuttosto che cercare di opporsi o di resistere. Invece
li imploriamo perché assecondino i nostri desideri. Cerchiamo ogni
genere di mezzi per sistemarli come pare a noi, per ``fare un patto''
con loro.

Se il corpo si ammala e prova dolore, non vogliamo che sia così, e così
cerchiamo vari \emph{sutta}\footnote{\emph{Sutta.} Letteralmente,
  ``filo''. Un discorso o sermone del Buddha o dei discepoli suoi
  contemporanei.} da cantare, come \emph{Bojjhango}, il
\emph{Dhammacakkappavattana Sutta}, l'\emph{Anattalakkhaṇa Sutta} e così
via. Non vogliamo che il corpo sia dolorante, vogliamo proteggerlo,
controllarlo. Questi \emph{sutta} diventano una specie di cerimonia
mistica che ci intrappola ancor di più nell'attaccamento. È perché li
cantano al fine di scacciare la malattia, di prolungare la vita e così
via. In verità il Buddha ci diede questi insegnamenti per farci vedere
con chiarezza, ma finiamo per cantarli con lo scopo di accrescere la
nostra illusione. \emph{Rūpaṃ aniccaṃ, vedanā aniccā, saññā aniccā,
saṅkhāra} \emph{aniccā, viññāṇaṃ aniccaṃ}.\footnote{La forma è
  impermanente, la sensazione è impermanente, la percezione è
  impermanente, i fenomeni condizionati sono impermanenti, la coscienza
  è impermanente.} Non cantiamo queste parole per far crescere la nostra
illusione. Sono rammemorazioni che ci aiutano a conoscere la verità del
corpo, così da consentirci di lasciarlo andare e di abbandonare la
nostra brama.

Questo è cantare per ridurre le cose, ma noi tendiamo a cantare per
aumentarle tutte o, se pensiamo che siano troppo lunghe, cantiamo per
accorciarle, per forzare la natura a conformarsi ai nostri desideri. È
solo un'illusione. Tutte le persone sedute qui, in questa sala, sono
illuse, senza eccezioni. Quelli che cantano sono illusi, quelli che
ascoltano sono illusi, sono tutti illusi! Tutto ciò che riescono a
pensare è: «~Come possiamo evitare la sofferenza?~» Quando mai
praticheranno?

In qualsiasi momento le malattie dovessero sorgere, coloro che conoscono
non ci vedono nulla di strano. Nascere in questo mondo implica
sperimentare la malattia. Anche il Buddha e gli Esseri Nobili curavano
le malattie con medicinali quando a loro capitava di contrarle. Si
trattava semplicemente di correggere gli elementi. Non si attaccavano
ciecamente al corpo né si aggrappavano a cerimonie mistiche o a cose di
questo genere. Curavano le malattie con Retta Visione, non le curavano
con illusione. Se guarisce guarisce, se non guarisce non guarisce: ecco
come vedevano le cose.

Dicono che oggigiorno il buddhismo in Thailandia è fiorente. A me invece
sembra talmente sfiorito che di più quasi non si può. Le sale di Dhamma
sono piene di orecchi attenti, ma che prestano attenzione in modo
errato. Persino i membri più anziani della comunità sono così. Perciò si
trascinano l'un l'altro sempre più nell'illusione. Chi lo capisce saprà
che la vera pratica va in direzione per lo più opposta rispetto a dove
va la maggior parte della gente. Gli uni e gli altri possono a mala pena
comprendersi a vicenda. Questa gente come farà a trascendere la
sofferenza? Hanno dei canti per comprendere la Verità, ma li
fraintendono e li usano per aumentare la loro illusione. Voltano le
spalle al retto Sentiero. Questo va a est, loro a ovest. Come potranno
mai incontrarsi? Non sono nemmeno vicini.

Se esaminerete la questione per bene, vedrete che è proprio così. La
maggior parte della gente si è perduta. In quale modo è possibile
dirglielo? Tutto è diventato riti, rituali e cerimonie mistiche.
Cantano, ma cantano con stoltezza, non cantano con saggezza. Studiano,
ma studiano con stoltezza, non con saggezza. Sanno, ma sanno con
stoltezza, non con saggezza. Perciò, finiscono per procedere
stoltamente, vivere stoltamente e conoscere stoltamente. È così. Poi,
per quanto riguarda l'insegnamento, tutto quello che fanno è insegnare
alla gente a essere sciocca. Dicono che stanno insegnando alle persone a
essere intelligenti, per dare loro la conoscenza, ma quando osservate
questa conoscenza in termini di Verità, vedete che stanno davvero
insegnando alla gente ad andare fuori strada e ad aggrapparsi agli
inganni.

Il reale fondamento dell'Insegnamento ha come finalità vedere
\emph{attā}, il senso del sé, come vuoto, privo di stabile identità. È
privo di un'essenza intrinseca. La gente però studia il Dhamma per
accrescere la prospettiva del sé. Non vogliono provare sofferenza o
difficoltà. Vogliono che tutto sia facile. Possono anche voler
trascendere la sofferenza, ma come potranno mai riuscirci se c'è ancora
un sé?

Supponiamo di entrare in possesso di un oggetto molto costoso. Nel
momento in cui ci impossessiamo di quest'oggetto, la nostra mente
cambia: «~Dove posso metterlo? Se lo lascio qui qualcuno potrebbe
rubarlo.~» Mentre cerchiamo il posto in cui riporlo, siamo preoccupati.
Quando è cambiata la mente? È cambiata quando è arrivato il possesso: è
proprio allora che nasce la sofferenza. Non importa dove lo mettiamo,
non riusciamo a rilassarci, e così abbiamo un problema. Sia che stiamo
seduti o in piedi, che camminiamo o che ci sdraiamo, ci perdiamo nelle
preoccupazioni.

Questa è sofferenza. E quando è sorta? È sorta appena abbiamo compreso
di aver ottenuto qualcosa, è lì che sta la sofferenza. Prima di avere
quell'oggetto, la sofferenza non c'era. Non era ancora sorta perché non
c'era ancora un oggetto al quale potessimo aggrapparci. \emph{Attā}, il
sé, è la stessa cosa. Se pensiamo nei termini del ``mio sé'', allora
attorno a noi tutto diventa ``mio''. Segue la confusione. Perché è così?
La causa di tutto questo sta nel fatto che sia un sé; non rimuoviamo
l'apparenza per vedere la trascendenza. Capite? Il sé è solo
un'apparenza. Dovete rimuovere le apparenze per vedere il cuore della
questione, la trascendenza. Capovolgete l'apparenza per trovare la
trascendenza.

Lo si potrebbe paragonare al riso non trebbiato. Si può mangiare il riso
non trebbiato? Certamente sì, ma dovete prima trebbiarlo. Eliminate la
lolla e dentro troverete il chicco. Se non trebbiamo la lolla, non
troveremo il chicco. Come il cane che dorme su un mucchio di chicchi non
trebbiati. Il suo stomaco brontola -- gurgle, gurgle, gurgle -- ma tutto
quel che può fare è starsene lì sdraiato a pensare: «~Dove posso trovare
qualcosa da mangiare?~» Quando è affamato abbandona il mucchio di riso e
corre a cercare degli avanzi di cibo. Benché stia dormendo proprio su un
mucchio di cibo, non lo sa. Perché? Perché non può vedere il riso. I
cani non mangiano riso non trebbiato. Il cibo è lì, ma il cane non può
mangiarlo.

Possiamo anche aver imparato, ma se non pratichiamo di conseguenza non
conosceremo veramente; siamo ignari proprio come il cane che dorme sul
mucchio di chicchi di riso. Sta dormendo su un mucchio di cibo, ma non
lo sa. Quando gli viene fame salta giù e se ne va trotterellando altrove
alla ricerca di cibo. È un peccato, vero? Ci sono dei chicchi di riso,
ma che cosa li nasconde? La lolla nasconde i chicchi, e perciò il cane
non può mangiare il riso. C'è il trascendente. Che cosa lo nasconde? Ciò
che appare nasconde il trascendente, facendo in modo che la gente
``sieda sulla sommità del mucchio di riso, incapace di mangiarlo'',
incapace di praticare, incapace di vedere il trascendente. E così
restano solo ripetutamente bloccati nelle apparenze. Se siete bloccati
nelle apparenze, in serbo vi è la sofferenza. Sarete assediati dal
divenire, dalla nascita, dalla vecchiaia, dalla malattia e dalla morte.

Non c'è nient'altro che blocchi la gente, resta bloccata proprio qui.
Chi studia il Dhamma senza penetrare nel suo vero significato è proprio
come il cane sul mucchio di riso non trebbiato che non sa nulla del
riso. Se non trova niente da mangiare potrebbe anche morire di fame. Il
cane non può mangiare riso non trebbiato, nemmeno sa che lì c'è del
cibo. Dopo molto tempo senza mangiare potrebbe anche morire, là, sulla
sommità di quel mucchio di riso! La gente è così. Non conta quanto
studiamo il Dhamma del Buddha, se non pratichiamo non lo vedremo. Se non
lo vedremo, non lo conosceremo.

Non pensiate che imparando molto e sapendo molto conoscerete il
Buddha-Dhamma. È come dire che avete visto tutto quello che c'è da
vedere solo perché avete gli occhi. Siete in grado di vedere, ma non
vedete pienamente. Vedete solo con l'``occhio esteriore'', non con
l'``occhio interiore''. E se sentite, sentite solo con l'``orecchio
esteriore'', non con l'``orecchio interiore''. Se capovolgete
l'apparenza e svelate la trascendenza, raggiungerete la Verità e vedrete
con chiarezza. Sradicherete l'apparenza e sradicherete l'attaccamento.

È come se si trattasse di un frutto dolce. Sebbene il frutto sia dolce,
dobbiamo fare affidamento sul contatto con l'esperienza di quel frutto,
prima di sapere com'è il suo sapore. Sebbene nessuno lo abbia
assaggiato, quel frutto è ugualmente dolce. Però nessuno lo sa. Il
Dhamma del Buddha è così. Sebbene si tratti della Verità, non è cosa
vera per coloro che non la conoscono davvero. Non conta quanto il
``frutto'' possa essere eccellente e pregiato, per loro non ha alcun
valore.

Perché allora la gente si aggrappa a qualcosa dopo aver sofferto? Chi
mai in questo mondo desidera infliggere sofferenza a se stesso? Nessuno,
naturalmente. Nessuno vuole soffrire e tuttavia la gente continua a
creare le cause della sofferenza, proprio come se stesse girovagando
alla ricerca della sofferenza. Nel proprio cuore la gente è alla ricerca
della felicità, non vuole la sofferenza. Allora perché succede che
questa nostra mente crei così tanta sofferenza? Anche comprendere solo
questo è già abbastanza. Se non ci piace la sofferenza, perché allora
creiamo sofferenza per noi stessi? Capirlo è facile, succede solo perché
non conosciamo la sofferenza, non conosciamo la fine della sofferenza.
Ecco perché la gente si comporta come si comporta. Come potrebbero non
soffrire se continuano a comportarsi in questo modo?

Questa gente ha \emph{micchā-diṭṭhi},\footnote{\emph{Micchā-diṭṭhi.} La
  visione errata; i principali tipi di \emph{diṭṭhi} sono due:
  \emph{sammā-diṭṭhi}, la Retta Visione, il primo fattore del Nobile
  Ottuplice Sentiero, e \emph{micchā-diṭṭhi}, la visione errata, che si
  contrappone alla retta visione.} ma non capisce che è
\emph{micchā-diṭṭhi}. Qualsiasi cosa in cui crediamo, diciamo o facciamo
che abbia quale risultato la sofferenza è errata visione. Se non fosse
errata visione non avrebbe la sofferenza come risultato. Non possiamo
aggrapparci alla sofferenza né alla felicità né a qualsiasi altra
condizione. Dovremmo lasciare che le cose siano nel loro modo naturale,
come un ruscello d'acqua che scorre. Non dobbiamo bloccarlo, dovremmo
semplicemente lasciarlo scorrere secondo il suo naturale fluire. Il
fluire del Dhamma è così, ma il fluire della mente ignorante cerca di
opporsi al Dhamma nella forma dell'errata visione. La sofferenza c'è a
causa dell'errata visione, ma la gente non lo capisce. È questo che
merita di essere esaminato. Tutte le volte che c'è errata visione
sperimenteremo la sofferenza. Se non la sperimentiamo ora, si
manifesterà in seguito.

È proprio qui che la gente va fuori strada. Che cos'è che la blocca?
L'apparenza ostruisce la trascendenza, impedendo alla gente di vedere le
cose con chiarezza. La gente studia, impara, pratica, ma pratica con
ignoranza, proprio come chi ha perso l'orientamento. Cammina verso
ovest, ma pensa di camminare verso est, oppure cammina verso nord,
pensando di camminare verso sud. La gente si è smarrita fino a questo
punto. Questo genere di pratica è davvero solo la feccia della pratica.
È un disastro. È un disastro perché la gente si gira e va nella
direzione opposta, fallisce l'obiettivo della vera pratica del Dhamma.

Questo stato di cose induce sofferenza, ma la gente pensa che fare
questo, memorizzare quello, studiare quell'altro sia una causa per la
cessazione della sofferenza. Proprio come chi vuole un sacco di cose.
Cerca di accumulare quanto più può, pensando che se avrà cose a
sufficienza la sua sofferenza cesserà. Così pensa la gente, ma i loro
pensieri non sono sul retto sentiero, proprio come uno che va a nord e
un altro che va a sud, anche se entrambi credono di percorrere la stessa
strada.

La maggior parte delle persone è bloccata nell'ammasso della sofferenza,
sta ancora girovagando nel \emph{saṃsāra} proprio perché pensa in questo
modo. Se sorge una malattia o il dolore, tutto quello che riescono a
fare è chiedersi come sbarazzarsene. Vogliono che smetta il prima
possibile, devono curarsi a tutti i costi. Non considerano che tutto
questo è normale per i \emph{saṅkhāra}. Nessuno la pensa in questo modo.
Il corpo cambia e la gente non riesce a sopportarlo, non può accettarlo,
vuole evitarlo a tutti i costi. Ovviamente, alla fine non possono
vincere, non possono sconfiggere la Verità. Tutto cade a pezzi. Si
tratta di una cosa che la gente non vuole vedere e ciò rinforza
continuamente l'errata visione.

Praticare per comprendere il Dhamma è la cosa più eccelsa. Perché il
Buddha sviluppò tutte le perfezioni? Per essere in grado di comprendere
il Dhamma e per consentire agli altri di vedere il Dhamma, conoscere il
Dhamma, praticare il Dhamma ed essere il Dhamma: così non sarebbero
stati più gravati, sarebbero stati in grado di lasciar andare. «~Non
attaccatevi alle cose.~» Oppure, per dirla in altro modo: «~Tenete, ma
non trattenete.~» Anche questo è giusto. Se vediamo qualcosa, lo
prendiamo: «~Ah! Ecco cos'è.~» Poi lo posiamo. Vediamo un'altra cosa, la
prendiamo e la teniamo in mano, senza stringerla. La teniamo abbastanza
a lungo per esaminarla, per conoscerla, poi la lasciamo andare. Se la
tenete senza lasciar andare, se la portate senza lasciar andare il
fardello, diventerete pesanti. Se prendete una cosa e la trasportate per
un po', quando diventa pesante dovreste posarla, sbarazzarvene. Non
createvi sofferenza da soli.

Dovremmo sapere che questa è la causa della sofferenza. Se conosciamo la
causa della sofferenza, la sofferenza non può sorgere. Affinché sorga la
felicità o la sofferenza ci deve essere \emph{attā}, il sé. Ci deve
essere l'``io'' e il ``mio'', è necessaria quest'apparenza. La mente
rimuove le apparenze se, quando sorgono tutte queste cose, va dritta al
trascendente. Rimuove il piacere, l'avversione e l'attaccamento dalle
cose che li fanno sorgere. Proprio come svaniscono le nostre
preoccupazioni quando ritroviamo una cosa di valore che pensavamo fosse
andata perduta.

Anche prima di ritrovare quell'oggetto smarrito le nostre preoccupazioni
possono essere mitigate. All'inizio pensiamo che sia andato perduto e
soffriamo, ma viene improvvisamente il giorno in cui ricordiamo: «~Ah!
Giusto! L'ho messo lì, ora ricordo!~» Appena lo ricordiamo, appena
ricordiamo la verità, anche se non abbiamo ancora posato lo sguardo su
quell'oggetto, siamo felici. È quel che si dice ``vedere all'interno'',
vedere con l'occhio della mente, non vedere con l'occhio esteriore. Se
vediamo con l'occhio della mente, ci sentiamo già sollevati anche se non
abbiamo posato lo sguardo su quell'oggetto. Allo stesso modo, quando
coltiviamo la pratica del Dhamma e otteniamo il Dhamma, vediamo il
Dhamma e tutte le volte che incontriamo un problema lo risolviamo
istantaneamente, lì per lì. Scompare completamente, viene posato,
rilasciato.

Il Buddha voleva che entrassimo in contatto con il Dhamma, ma la gente
entra in contatto solo con le parole, con i libri e le Scritture. Questo
significa entrare in contatto con ciò che riguarda il Dhamma, non è
entrare in contatto con il vero Dhamma, come ci è stato insegnato dal
nostro grande Maestro. Come può la gente dire che sta praticando bene e
propriamente? È del tutto fuori strada. Il Buddha era conosciuto come
\emph{lokavidū} poiché aveva compreso con chiarezza il mondo. Proprio
ora il mondo lo vediamo bene, ma non con chiarezza. Più sappiamo, più il
mondo diventa oscuro, perché il nostro sapere è torbido, non è chiara
conoscenza. È un sapere difettoso. È quel che si dice ``conoscere
mediante l'oscurità'', in assenza di luce e radiosità.

Proprio qui la gente resta bloccata, perciò non si tratta di una
questione insignificante. È importante. La maggior parte delle persone
vuole solo bontà e felicità, ma non sa quali siano le cause della bontà
e della felicità. Come che sia, se non abbiamo ancora compreso il danno
che una cosa ci arreca, non possiamo rinunciarvi. Non conta quanto possa
essere dannosa, non riusciamo ancora a rinunciarvi perché non abbiamo
ancora davvero compreso il danno che ci arreca. Ovviamente, se possiamo
davvero vedere al di là di ogni dubbio quanto una cosa sia dannosa,
allora possiamo lasciar andare. Appena vediamo i pericoli di un qualcosa
e i benefici del rinunciarvi, c'è un cambiamento immediato.

Perché non siamo ancora Realizzati? Perché non riusciamo ancora a
lasciar andare? È perché non vediamo ancora con chiarezza il pericolo,
la nostra conoscenza è difettosa, oscura. Ecco perché non riusciamo a
lasciar andare. Se conoscessimo con chiarezza come il Buddha o come i
suoi discepoli \emph{arahant} certamente lasceremmo andare, i nostri
problemi si dissolverebbero del tutto, senza alcuna difficoltà.

Quando i nostri orecchi sentono un suono, lasciamoli fare il loro
lavoro. Quando i vostri occhi svolgono la loro funzione con le forme,
lasciateli fare. Quando il vostro naso lavora con gli odori, lasciatelo
svolgere il suo compito. Quando il vostro corpo sperimenta delle
sensazioni, svolge le sue funzioni naturali. Dove sono i problemi? Non
ci sono problemi. Nello stesso modo, tutte quelle cose che fanno parte
dell'apparenza lasciatele all'apparenza e riconoscete quel che è
trascendenza. Siate semplicemente ``Colui che Conosce'', che conosce
senza fissazione, conosce e lascia che le cose seguano il loro corso
naturale. Tutte le cose sono solo quello che sono.

Tutti i nostri possessi: c'è davvero qualcuno che li possiede? Li
possiede nostro padre o nostra madre, oppure a possederli sono i nostri
parenti? Nessuno ottiene proprio nulla. Questa è la ragione per cui il
Buddha disse di lasciare che tutte queste cose siano, di lasciarle
andare. Conoscerle chiaramente. Conoscetele tenendole, non
trattenendole. Usate le cose in modo benefico, non in modo dannoso,
trattenendole finché non sorge la sofferenza. Per conoscere il Dhamma
dovete conoscere in questo modo. Ossia conoscere in modo tale da
trascendere la sofferenza. Questo genere di conoscenza è importante.
Sapere come fare le cose, usare degli strumenti, sapere tutte le varie
scienze del mondo e così via, c'è spazio per tutto, ma non è conoscenza
suprema. Il Dhamma deve essere conosciuto come vi ho appena spiegato.
Non dovete sapere un sacco di cose, solo questo è già abbastanza per i
praticanti di Dhamma, conoscere e poi lasciar andare.

Sapete, non è che si debba morire prima di riuscire a trascendere la
sofferenza. La sofferenza la trascendete proprio in questa vita perché
sapete come risolvere i problemi. Conoscete l'apparenza e conoscete la
trascendenza. Fatelo in questa vita, mentre state praticando qui. Non lo
farete in nessun altro luogo. Non attaccatevi alle cose. Tenete, ma non
vi attaccate.

Potreste chiedervi: «~Perché l'\emph{ajahn} continua a dirlo?~» Come
potrei insegnare altrimenti, come potrei dire altrimenti, se la Verità è
solo ciò che vi ho detto? Anche se è la Verità, non trattenete nemmeno
quella! Se vi attaccate ciecamente a essa diverrà falsa. È come
afferrare la zampa di un cane. Se non lasciate andare, il cane si
volterà e vi morderà. Provateci. Tutti gli animali si comportano in
questo modo. Se non lasciate andare, non hanno altra scelta che mordere.
L'apparenza è la stessa cosa. Viviamo in conformità con le convenzioni.
Esse hanno una loro utilità in questa vita, ma non sono cose da
afferrare tanto strette da far sorgere la sofferenza. Basta lasciare che
le cose passino. Tutte le volte che pensiamo di avere completamente
ragione, fino al punto di rifiutare di aprirci a qualsiasi altra cosa o
persona, è proprio lì che abbiamo torto. Diventa errata visione. Quando
la sofferenza sorge, da dove sorge? La causa è l'errata visione, il
frutto è la sofferenza. Se fosse stata Retta Visione, non avrebbe
causato sofferenza.

Perciò vi dico: «~Lasciate spazio, non attaccatevi alle cose.~»
``Giusto'' è solo un'altra supposizione, lasciatela andare.
``Sbagliato'' è un'altra condizione apparente, lasciatela essere quello
che è. Se pensate di avere ragione e l'altro controbatte, non discutete,
lasciate andare. Appena lo capite, lasciate andare. Questa è la retta
via. In genere non succede così. La gente non cede spesso. Ecco perché
alcune persone, perfino i praticanti di Dhamma che non conoscono ancora
se stessi, possono affermare cose che sono assolutamente sciocche pur
pensando di essere stati saggi. Possono dire qualcosa di così stupido
che è impossibile perfino riuscire ad ascoltare, e tuttavia pensano di
essere più intelligenti degli altri. Stanno solo manifestando la loro
stupidità.

Ecco perché il saggio dice: «~Ogni discorso che trascuri \emph{aniccā}
non è il discorso di un saggio, è il discorso di un folle, è un discorso
illusorio, è il discorso di uno che non sa che la sofferenza sta per
sorgere proprio lì.~» Supponiamo ad esempio che abbiate deciso di andare
a Bangkok domani, e che qualcuno vi chieda: «~Domani vai a Bangkok?~»
«~Spero di andare. Se non ci sono ostacoli probabilmente andrò~»: questo
è parlare con il Dhamma nella mente, parlare con \emph{aniccā} nella
mente, tenere conto della Verità, della transitorietà, della natura
incerta del mondo. Non dite: «~Sì, certamente domani andrò.~» Se poi
succede che non andate, che fate? Avvertite tutti quelli ai quali avete
detto che sareste andati? Avreste detto solo sciocchezze.

C'è molto altro ancora nella pratica del Dhamma, essa diventa sempre più
raffinata man mano che si procede. Se non lo capite, potreste pensare di
parlare in modo giusto anche se lo state facendo in modo sbagliato, e
con ogni vostra parola vi state allontanando dalla vera natura delle
cose. Potreste tuttavia pensare che stiate dicendo la Verità. Per dirla
semplicemente: qualsiasi cosa diciamo o facciamo che causi il sorgere
della sofferenza dovrebbe essere conosciuto come \emph{micchā-diṭṭhi}. È
illusione e follia. La maggior parte dei praticanti non riflette in
questo modo. Pensa che sia giusto tutto quello che a loro piace, e così
le persone vanno avanti credendo solo in se stesse. Ad esempio, ricevono
un dono o un titolo, un oggetto, una carica o anche delle parole di
lode, e pensano che sia un bene. Pensano che si tratti d'una sorta di
condizione permanente. Così, si gonfiano di orgoglio e di presunzione,
senza pensare: «~Chi sono io? Dov'è questo cosiddetto ``bene''? Da dove
viene? Ci sono altre persone che hanno queste stesse cose?~»

Il Buddha insegnò che dovremmo comportarci normalmente. Se non scaviamo
dentro tale questione, se non la consideriamo attentamente e non la
osserviamo, questo significa che è ancora sepolta dentro di noi.
Significa che queste condizioni sono ancora sepolte nei nostri cuori,
che sprofondiamo ancora nel benessere, nel rango e nella lode. Perciò, a
causa di essi diventiamo qualcos'altro. Pensiamo di essere migliori di
prima, di essere speciali, e così sorge ogni genere di confusione.

In effetti, la Verità è che gli esseri umani sono niente. Qualsiasi cosa
si possa essere, lo siamo solo nel regno delle apparenze. Se eliminiamo
l'apparenza e vediamo la trascendenza, comprendiamo che lì non c'è
niente. Ci sono semplicemente le caratteristiche universali: nascita
all'inizio, cambiamento nel mezzo e cessazione alla fine. Questo è
tutto, c'è solo questo. Se vediamo che tutte le cose sono così, non
sorgeranno problemi. Se lo comprendiamo, saremo appagati e sereni.

Le difficoltà nascono quando pensiamo nella stessa maniera dei cinque
asceti discepoli del Buddha. Seguirono le istruzioni del loro Maestro
ma, quando Egli modificò la sua pratica, non furono in grado di capire
cosa pensasse o sapesse. Decisero che il Buddha aveva abbandonato la
pratica e fosse tornato a indulgere ai piaceri dei sensi. Se fossimo
stati al loro posto, avremmo pensato la stessa cosa e, così, sarebbe
stato impossibile correggerci. Pensando in modo pessimo, ma ritenendo di
pensare in modo elevato, ci saremmo attaccati ai vecchi metodi. Avremmo
guardato il Buddha pensando che avesse abbandonato la pratica e fosse
tornato a indulgere ai piaceri dei sensi, proprio come quei cinque
asceti: considerate da quanti anni stessero praticando e, nonostante
questo, andarono fuori strada. Non erano ancora abili.

Per questo vi dico di praticare, e pure di osservare i risultati della
vostra pratica. Osservate soprattutto dove vi rifiutate di seguire gli
insegnamenti, dove c'è attrito. Dove non c'è attrito, non c'è problema,
le cose fluiscono. Se c'è attrito, non fluiscono; create un sé e le cose
divengono solide, divengono una massa di attaccamento. Non c'è dare né
avere. La maggior parte dei monaci e dei praticanti tende a essere così.
Continuano a pensare nello stesso modo di prima. Rifiutano di cambiare,
non riflettono. Pensano di essere nel giusto e che perciò non possono
avere torto, ma in realtà l'``errore'' sta sepolto nel ``giusto'', anche
se la gente per lo più non lo sa. Com'è che è così? «~Questo è giusto.~»
\ldots{} Se però qualcun altro dice che non lo è, non cedete, dovete
discutere. Che cos'è questo? \emph{Diṭṭhi-māna}. \emph{Diṭṭhi} significa
opinione, \emph{māna} è l'attaccamento a quell'opinione. Anche se ci
attacchiamo a quello che è giusto, rifiutando ogni genere di concessione
a chicchessia quello che è giusto diventa sbagliato. Aggrapparci
saldamente a ciò che è giusto è solo nascita di un sé, non c'è lasciar
andare.

Si tratta di un aspetto che causa molte difficoltà alla gente, ma non a
quei praticanti di Dhamma che conoscono questo problema, che è davvero
importante. Ne prenderanno atto. Se sorge mentre parlano,
l'attaccamento arriva in scena di corsa. Forse durerà per un po', forse
uno o due giorni, tre o quattro mesi, un anno o due. Questo vale per le
persone lente. Per quelle veloci, la risposta è istantanea: loro
lasciano subito andare. L'attaccamento sorge, e immediatamente c'è il
lasciar andare, costringono la mente a lasciar andare lì per lì.

Dovete capire come operano queste due funzioni. Qui c'è l'attaccamento.
Ora, chi è che resiste a quell'attaccamento? Tutte le volte che
sperimentate un'impressione mentale dovreste osservare queste due
funzioni in azione. C'è l'attaccamento e c'è chi proibisce
l'attaccamento. Osservatele, queste due cose. Forse sperimenterete a
lungo l'attaccamento prima di lasciar andare. Riflettete e praticate
continuamente in questo modo, e l'attaccamento diverrà meno tenace,
diminuirà sempre più. La Retta Visione cresce man mano che l'errata
visione decresce. L'attaccamento decresce, il non attaccamento sorge.
Questo vale per tutti. Ecco perché dico di prendere in considerazione
questo punto. Imparate a risolvere i problemi nel momento presente.

