\chapter{Elenco degli Insegnamenti}

(\textsuperscript{*} Negli elenchi relativi alla II e alla III parte i numeri
tra parentesi tonde rinviano alla posizione dei \emph{Discorsi} nell'edizione
dei \emph{Collected Teachings} in tre volumi. Si è rinunciato, come
nell'edizione inglese, a fornire descrizioni bibliografiche complete; le
precedenti traduzioni italiane sono state per lo più pubblicate dall'editore
Ubaldini di Roma.)

PARTE I

I, 1. \emph{La Via di Mezzo dentro di noi}; tit. orig. ingl. \emph{The
Middle Way Within}, vol. I, pp. 1-7; pubblicato per la prima volta in
\emph{A Taste of Freedom}; trad. ital. \emph{La Via di Mezzo interiore},
in \emph{Il sapore della libertà}, pp. 124-131. Il discorso fu
pronunciato nel 1970 per una riunione di monaci e laici, nel dialetto
del nord-est della Thailandia.

I, 2. \emph{Al di là}; tit. orig. ingl. \emph{The Peace beyond}, vol. I,
pp. 9-19; pubblicato per la prima volta in \emph{A Taste of Freedom};
trad. ital. \emph{Al di là degli opposti}, in \emph{Il sapore della
libertà}, pp. 132-142. Versione abbreviata di un discorso tenuto al Wat
Nong Pah Pong nel 1978 per il Chief Privy Councillor della Thailandia
(1975-1988), Sanya Dharmasakti (Thammasak), già Primo Ministro del paese
dal 1973 al 1975.

I, 3. \emph{Convenzione e Liberazione}; tit. orig. ingl.
\emph{Convention and Liberation}, vol. I, pp. 21-27; pubblicato per la
prima volta in \emph{A Taste of Freedom} (una differente traduzione è
stata pubblicata con il titolo \emph{Suppositions and Release} da Ajahn
Thanissaro); trad. ital. \emph{Convenzione e liberazione}, in \emph{Il
sapore della libertà}, pp. 159-165. Si tratta di un discorso informale
pronunciato nel dialetto del nord-est della Thailandia, da una
registrazione non si sa da chi realizzata.

I, 4. \emph{Senza dimora}; tit. orig. ingl. \emph{No abiding}, vol. I,
pp. 29-35; pubblicato per la prima volta in \emph{A Taste of Freedom};
trad. ital. \emph{Lo spazio vuoto}, in \emph{Il sapore della libertà},
pp. 166-172. Un discorso per monaci, novizi e laici del Wat Pah Nanachat
in visita al Wat Nong Pah Pong tenuto durante la Stagione delle Piogge
del 1980.

I, 5. \emph{Seduta serale}; tit. orig. ingl. \emph{Evening sitting},
vol. I, pp. 37-47; pubblicato per la prima volta in \emph{The Path to
Peace}. Con alcuni tagli e differente titolo anche in \emph{Being
Dharma}; la già realizzata trad. ital. è lacunosa, perché tratta da
quest'ultimo volume, con il titolo \emph{La pratica della meditazione},
in \emph{Essere Dhamma}, pp. 103-112; ibid., p. 103, si legge quanto
segue: «La presenza mentale del respiro: discorso tenuto a un ritiro
della Insight Meditation Society tenuto a Barre, Massachusetts, nel
1979».

I, 6. \emph{Essere attenti}; tit. orig. ingl. \emph{About being
careful}, vol. I, pp. 49-66; pubblicato per la prima volta in
\emph{Everything is teaching us}; trad. ital. \emph{Essere attenti}, in
\emph{Tutto insegna}, pp. 5-25.

I, 7. \emph{Si può fare}; tit. orig. ingl. \emph{It can be done}, vol.
I, pp. 67-81; pubblicato per la prima volta in \emph{Everything is
teaching us}; trad. ital. \emph{Si può fare}, in \emph{Tutto insegna},
pp. 26-44.

I, 8. \emph{Comprendere la sofferenza}; tit. orig, ingl.
\emph{Undestanding dukkha}, vol. I, pp. 83-91; altrove anche con il
titolo \emph{Giving up Good and Evil}; pubblicato per la prima volta in
\emph{Everything is teaching us}; trad. ital. \emph{Rinunciare al bene e
al male}, in \emph{Tutto insegna}, pp. 62-72.

I, 9. \emph{Il Dhamma va in Occidente}; tit. orig. ingl. \emph{The
Dhamma goes westward}, vol. I, pp. 93-107; pubblicato per la prima volta
in \emph{Everything is teaching us}; trad. ital. \emph{Il Dhamma va in
Occidente}, in \emph{Tutto insegna}, pp. 118-135. Al \emph{Saṅgha}
occidentale da poco giunto in Inghilterra, 1979.

I,~10.~\emph{Una parola è già abbastanza}; tit. orig. ingl. \emph{Even
One Word Is Enough}, vol. I, pp. 109-120; pubblicato per la prima volta
in \emph{Everything is teaching us}; nessuna trad. ital. individuata.

I, 11.~\emph{Rendere buono il cuore}; tit. orig. ingl. \emph{Making the
Heart Good}, vol. I, pp. 121-128; pubblicato per la prima volta in
\emph{Living Dhamma}; trad. ital. \emph{Rendere buono il cuore}, in
\emph{Il Dhamma vivo}, pp. 7-15. Discorso tenuto a un vasto gruppo di
laici giunto al Wat Pah Pong per fare offerte in supporto del monastero.

I,~12.~\emph{Perché siamo qui?}; tit. orig. ingl. \emph{Why are we
here?}, vol . I, pp. 129-139; pubblicato per la prima volta in
\emph{Living Dhamma}; trad. ital. \emph{Perché siamo qui?}, in \emph{Il
Dhamma vivo}, pp. 16-27. Discorso pronunciato per un gruppo di laici al
Wat Tham Saeng Phet (il monastero della Caverna della Luce di Diamante)
durante il Ritiro delle Piogge del 1981, poco prima che la sua salute
peggiorasse.

I,~13. \emph{La nostra vera casa}; tit. orig. ingl. \emph{Our Real
Home}, vol. I, pp. 141-151; pubblicato per la prima volta in
\emph{Living Dhamma}; trad. ital. \emph{La nostra vera casa (Consigli a
una moribonda)}, in \emph{Il Dhamma vivo}, pp. 28-40. Il discorso è
rivolto a un'anziana discepola laica prossima alla morte.

I,~14. \emph{Le Quattro Nobili Verità}; tit. orig. ingl. \emph{The Four
Noble Truths}, vol. I, pp. 153-163; pubblicato per la prima volta in
\emph{Living Dhamma}; trad. ital. \emph{Le quattro nobili verità}, in
\emph{Il Dhamma vivo}, pp. 41-52. Questo insegnamento è stato offerto al
Manjushri Institute nella contea di Cumbria, in Inghilterra, nel 1977.

I,~15.~\emph{Vivere nel mondo}; tit. orig. ingl. \emph{Living In The
World}, vol. I, pp. 165-172; altrove anche con il titolo \emph{Living In
The World with Dhamma}; pubblicato per la prima volta in \emph{Living
Dhamma}; trad. ital. \emph{Vivere nel mondo con il Dhamma}, in \emph{Il
Dhamma vivo}, pp. 65-74. Discorso informale offerto dopo un invito a
ricevere cibo in elemosina presso l'abitazione di un laico a Ubon,
capoluogo del distretto, vicino al Wat Pah Pong.

I,~16.~\emph{Dottrina vuota}; tit. orig. ingl. \emph{Tuccho Pothila},
vol. I, pp. 173-184; altrove anche con il titolo \emph{Tuccho Pothila --
Venerable Empty-Scripture}; pubblicato per la prima volta in
\emph{Living Dhamma}; trad. ital. \emph{Tuccho Pothila: il venerabile
``Dottrina Vuota''}, in \emph{Il Dhamma vivo}, pp. 75-88. Discorso
informale tenuto da Ajahn Chah presso la sua \emph{kuṭī} a un gruppo di
laici, una sera del 1978.

I,~17.~\emph{Trascendenza}; tit. orig. ingl. \emph{Transcendence}, vol.
I, pp. 185-198; pubblicato per la prima volta in \emph{Food for the
Heart}; trad. ital. \emph{Trascendenza}, in \emph{Cibo per il cuore},
pp. 132-147. Discorso offerto in una notte di osservanza lunare
(\emph{uposatha}) al Wat Pah Pong, nel 1975.

PARTE II

II,~18~(1). \emph{Insegnamenti} \emph{senza tempo}; tit. orig. ingl.
\emph{Timeless Teachings}, vol. II, pp. 1-5; pubblicato per la prima
volta in ``Forest Sangha Newsletter'', n. 39, genn. 1997; trad. ital.
http://santacittarama.altervista.\\
org/insegnamenti\_senza\_tempo.pdf

II,~19~(2). \emph{Frammenti di un insegnamento}; tit. orig. ingl.
\emph{Fragments of a Teaching}, vol. II, pp. 7-15; pubblicato per la
prima volta in \emph{Bodhinyana}; trad. ital. \emph{Frammenti di un
insegnamento}, in \emph{Il sapore della libertà}, pp. 9-17. Discorso
tenuto per la comunità laica del Wat Pah Pong nel 1972.

II,~20~(3). \emph{Un dono di Dhamma}; tit. orig. ingl. \emph{A Gift of
Dhamma}, vol II, pp. 17-23; pubblicato per la prima volta in
\emph{Bodhinyana}; trad. ital. \emph{Il dono del Dhamma}, in \emph{Il
sapore della libertà}, pp. 18-24. Discorso pronunciato per i monaci
occidentali, i novizi e i discepoli laici riuniti nel monastero della
foresta Bung Wai a Ubon, il 10 ottobre 1977; il discorso fu offerto ai
genitori di un monaco, che lo erano venuti a trovare dalla Francia.

II,~21~(4).~\emph{Vivere con un cobra}; tit. orig. ingl. \emph{Living
with the Cobra}, vol. II, pp. 25-28; pubblicato per la prima volta in
\emph{Bodhinyana}; trad. ital. \emph{Vivere con il cobra}, in \emph{Il
sapore della libertà}, pp. 61-64. Un breve discorso tenuto come
istruzione finale per un'anziana signora inglese che trascorse due mesi
sotto la guida di Ajahn Chah tra la fine del 1978 e l'inizio del 1979.

II,~22~(5).~\emph{La Mente Naturale}; tit. orig. ingl. \emph{Reading the
Natural Mind}, vol. II, pp. 29-46; pubblicato per la prima volta in
\emph{Bodhinyana}; trad. ital. \emph{Leggere la mente naturale}, in
\emph{Il sapore della libertà}, pp. 65-83. Discorso informale offerto,
dopo i Canti della sera, a metà del Ritiro delle Piogge del 1978, a un
gruppo di monaci che avevano da poco ricevuto l'ordinazione.

II,~23~(6).~\emph{Fatelo!}; tit. orig. ingl. \emph{Just do it!}, vol.
II, pp. 47-54; pubblicato per la prima volta in \emph{Bodhinyana}; trad.
ital. (il testo differisce però molto) \emph{Cominciate a praticare!} in
\emph{Il sapore della libertà}, pp. 84-91. Una diversa traduzione
inglese di questo discorso è stata pubblicata con il titolo \emph{Start
doing it!} È un vivace discorso in laotiano offerto nel Wat Pah Pong a
un'assemblea di monaci da poco ordinati, il primo giorno del Ritiro
delle Piogge, nel luglio del 1978.

II,~24~(7).~\emph{Domande e risposte}; tit. orig. ingl. \emph{Questions
and answers}, vol. II, pp. 55-68. pubblicato per la prima volta in
\emph{Bodhinyana}; trad. ital. \emph{Domande e risposte}, in \emph{Il
sapore della libertà}, pp. 92-107. Si tratta di appunti presi nel 1972,
nel corso di pochi giorni durante una seduta di domande e risposte con
un gruppo di monaci occidentali.

II,~25~(8).~\emph{Pratica costante}; tit. orig. ingl. \emph{Steady
Practice}, vol. II, pp. 69-82; questo discorso è stato pubblicato anche
con il titolo \emph{Right practice - Steady practice}, in \emph{Food}
\emph{for the Heart}; tr. ital. \emph{Retta pratica, pratica costante},
in \emph{Cibo per il cuore}, pp. 41-55. Discorso tenuto al Wat Keuan per
un gruppo di studenti universitari che avevano ricevuto l'ordinazione
monastica temporanea, durante l'estate del 1978.

II,~26~(9).~\emph{Attività distaccata}; tit. orig. ingl.
\emph{Detachment within Activity}, vol. II, pp. 83-95; questo discorso è
stato pubblicato anche con il titolo \emph{Sammā-samādhi - Detachment
with Activity}, in \emph{Food} \emph{for the Heart}; trad. ital.
\emph{Sammā-samādhi: attività distaccata}, in \emph{Cibo per il cuore},
pp. 56-68. Discorso tenuto al Wat Pah Pong durante il Ritiro delle
Piogge del 1977.

II,~27~(10).~\emph{Addestrare la mente}; tit. orig. ingl. \emph{Training
this mind}, vol. II, pp. 97-98; pubblicato per la prima volta in \emph{A
Taste of Freedom}; una differente traduzione inglese è stata altrove
pubblicata con il titolo \emph{About this mind}; da quest'ultima deriva
la trad. ital. con il titolo \emph{La mente}, in \emph{Il} \emph{sapore
della libertà}, p. 111.

II, 28 (11).~\emph{Tranquillità e visione profonda}; tit. orig. ingl.
\emph{Tranquillity and Insight}, vol. II, pp. 99-104; pubblicato per la
prima volta in \emph{A Taste of Freedom}; una differente traduzione
inglese è stata altrove pubblicata con il titolo \emph{On meditation};
da quest'ultima deriva la trad. ital. con il titolo \emph{La
meditazione}, in \emph{Il} \emph{sapore della libertà}, pp. 112-117. Si
tratta di un discorso informale, offerto nel dialetto del nord-est della
Thailandia, tratto da una registrazione non identificata (``unidentified
tape'').

II, 29 (12).~\emph{Il Sentiero in armonia}; tit. orig. ingl. \emph{The
Path in Harmony}, vol. II, pp. 105-110; pubblicato per la prima volta in
\emph{A Taste of Freedom}; trad. ital. \emph{Il sentiero in armonia}, in
\emph{Il} \emph{sapore della libertà}, pp. 118-123. Fusione di due
discorsi offerti in Inghilterra rispettivamente nel 1979 e nel 1977.

II, 30 (13).~\emph{Dove c'è frescura}; tit. orig. ingl. \emph{The Place
of Coolness}, vol. II, pp. 111-115; pubblicato per la prima volta in
\emph{A Taste of Freedom}; una differente traduzione inglese è stata
altrove pubblicata con il titolo \emph{Right View -- The Path in
Harmony}; da quest'ultima deriva la trad. ital. con il titolo
\emph{Retta concezione: il nostro rifugio spirituale}, in \emph{Il}
\emph{sapore della libertà}, pp. 173-177. Discorso tenuto per
l'assemblea dei monaci e dei novizi al Wat Pah Nanachat durante il
Ritiro delle Piogge del 1978.

II, 31 (14). \emph{Il monastero della confusione}; tit. orig. ingl.
\emph{Monastery of Confusion}, vol. II, pp. 117-131; pubblicato per la
prima volta in \emph{Everything is teaching us}; pubblicato anche
altrove con il titolo \emph{Free from Doubt}; trad. ital. \emph{Liberi
dal dubbio}, in \emph{Tutto insegna}, pp. 45-61.

II, 32 (15). \emph{Conoscere il mondo}; tit. orig. ingl. \emph{Knowing
the World}, vol. II, pp. 133-147; pubblicato per la prima volta in
\emph{Everything is teaching us}; pubblicato anche altrove con il titolo
\emph{Seeking the Source}; trad. ital. \emph{Cercare la fonte}, in
\emph{Tutto insegna}, pp. 104-117\emph{.}

II, 33 (16). \emph{Consigli per la meditazione}; tit. orig. ingl.
\emph{Supports for Meditation}, vol. II, pp. 149-159; pubblicato per la
prima volta in \emph{Living Dhamma}; pubblicato anche altrove con il
titolo \emph{Meditation}; trad. ital. \emph{La meditazione}, in \emph{Il
Dhamma vivo}, pp. 53-64. Il discorso venne tenuto allo Hampstead Vihara
a Londra, nel 1977.

II, 34 (17). \emph{Acqua ferma che scorre}; tit. orig. ingl.
\emph{Still, flowing Water}, vol. II, pp. 161-172; pubblicato per la
prima volta in \emph{Living Dhamma}; trad. ital. \emph{Una corrente
d'acqua ferma}, in \emph{Il Dhamma vivo}, pp. 89-103. Discorso offerto
al Wat Tham Saeng Phet, durante il Ritiro delle Piogge del 1981.

II, 35 (18). \emph{Verso l'incondizionato}; tit. orig. ingl.
\emph{Toward the Unconditiones}, vol. II, pp. 173-189; pubblicato per la
prima volta in \emph{Living Dhamma}; trad. ital. \emph{Tendere
all'incondizionato}, in \emph{Il Dhamma vivo}, pp. 104-124. Discorso
offerto durante una notte d'osservanza lunare (\emph{Uposatha}) al Wat
Pah Pong nel 1976.

II, 36 (19). \emph{Chiara visione profonda}; tit. orig. ingl.
\emph{Clarity of Insight}, vol. II, pp. 191-213; pubblicato per la prima
volta in \emph{Clarity of Insight}; nessuna trad. ital. individuata.
Discorso tenuto nell'aprile del 1979 a Bangkok per un gruppo di
meditanti laici.

II, 37 (20). \emph{Imparare ad ascoltare}; tit. orig. ingl.
\emph{Learning to Listen}, vol. II, pp. 215-216; nessuna trad. ital.
individuata. Discorso offerto nel settembre del 1978 al Wat Pah Pong.

II, 38 (21). \emph{Una pace incrollabile}; tit. orig. ingl.
\emph{Unshakeable Peace}, vol. II, pp. 217-260. Una differente trad.
inglese di questo saggio è stata altrove pubblicata con il titolo
\emph{The Key to Liberation}; trad. ital. \emph{Una pace incrollabile}.
Discorso informale offerto a un monaco studioso giunto a porgere omaggio
al venerabile Ajan Chah.

II, 39 (22). \emph{Solo questo}; tit. orig. ingl. \emph{Just this much},
vol. II, pp. 261-262; pubblicato per la prima volta in \emph{A Taste of
Freedom}, come epilogo; trad. ital. \emph{Epilogo}, in \emph{Il}
\emph{sapore della libertà}, pp. 178-179. Tratto da un discorso offerto
in Inghilterra nel 1977 a uno studente di Dhamma.

PARTE III

III, 40 (1). \emph{Che cos'è la contemplazione?}; tit. orig. ingl.
\emph{What is contemplation?}, vol. III, pp. 1-5; pubblicato per la
prima volta in \emph{Seeing the Way}, vol. I; trad. ital.
\href{http://santacittarama.altervista.org/contemplazione.htm}{http://santacittarama.altervista.org/\\
contemplazione.htm}. Questo insegnamento è tratto da una sessione di
domande e risposte intercorse tra un gruppo di discepoli di lingua
inglese e il venerabile Ajahn Chah che ebbe luogo al monastero Wat Gor
Nork durante il Vassa del 1979. Sono stati necessari alcuni ritocchi
nella sequenza della conversazione per facilitare la comprensione.

III,~41 (2). \emph{La natura del Dhamma}; tit. orig. ingl. \emph{Dhamma
Nature}, vol. III, pp. 7-14; pubblicato per la prima volta in
\emph{Bodhinyana}; trad. ital. \emph{La natura del Dhamma}, in \emph{Il
sapore della libertà}, pp. 25-32. Discorso tenuto per i discepoli
occidentali nel monastero della foresta Bung Wai durante il Ritiro delle
Piogge del 1977, dopo che uno dei monaci anziani svestì l'abito
monastico e lasciò il monastero.

III,~42 (3). \emph{I due volti della realtà}; tit. orig. ingl. \emph{Two
Faces of Reality}, vol. III, pp. 15-28; pubblicato per la prima volta in
\emph{Bodhinyana}; trad. ital. \emph{Le due facce della realtà}, in
\emph{Il sapore della libertà}, pp. 33-48. Discorso tenuto al Wat Pah
Pong durante il Ritiro delle Piogge del 1976 per un'assemblea di monaci
dopo la recitazione del \emph{Pāṭimokkha}, il codice di disciplina
monastica.

III,~43 (4). \emph{L'addestramento del cuore}; tit. orig. ingl.
\emph{The Training of the Heart}, vol. III, pp. 29-40; pubblicato per la
prima volta in \emph{Bodhinyana}; trad. ital. \emph{L'educazione del
cuore}, in \emph{Il sapore della libertà}, pp. 49-60. Discorso offerto
nel marzo del 1977 a un gruppo di monaci occidentali provenienti dal Wat
Bovornives di Bangkok. In questo testo inglese viene utilizzato
``heart'' (cuore) e in altre traduzioni si usa ``mind'' (mente).

III,~44 (5). \emph{Dove l'onda finisce}; tit. orig. ingl. \emph{The Wave
ends}, vol. III, pp. 41-48, pubblicato per la prima volta come
\emph{Questions and answers}, in \emph{The Path to Peace}; nessuna trad.
ital. individuata. Estratti da una conversazione tra Ajahn Chah e un
laico buddhista.

III,~45 (6). \emph{La battaglia del Dhamma}; tit. orig. ingl.
\emph{Dhamma Fighting}, vol. III, pp. 49-54; pubblicato per la prima
volta in \emph{Food for the Heart}; trad. ital. \emph{La battaglia del
Dhamma}, in \emph{Cibo per il cuore}, pp. 7-16. Estratti da un discorso
offerto a monaci e novizi al Wat Pah Pong.

III,~46 (7). \emph{Comprendere il Vinaya}; tit. orig. ingl.
\emph{Understanding Vinaya}, vol. III, pp. 55-66; pubblicato per la
prima volta in \emph{Food for the Heart}; trad. ital. \emph{Comprendere
il Vinaya}, in \emph{Cibo per il cuore}, pp. 17-30. Discorso offerto ai
monaci riuniti dopo la recitazione del \emph{pātimokkha} al Wat Pah Pong
durante il Ritiro delle Piogge del 1980.

III,~47 (8). \emph{Un buon livello di pratica}; tit. orig. ingl.
\emph{Maintaining the Standard}, vol. III, pp. 67-76; pubblicato per la
prima volta in \emph{Food for the Heart}; trad. ital. \emph{Curare la
qualità della pratica}, in \emph{Cibo per il cuore}, pp. 31-40. Discorso
tenuto nel 1978 al Wat Pah Pong dopo gli esami di Dhamma in lingua pāli.

III,~48 (9). \emph{Sommersi dai sensi}; tit. orig. ingl. \emph{The Flood
of Sensuality}, vol. III, pp. 77-86; pubblicato per la prima volta in
\emph{Food for the Heart}; trad. ital. \emph{La marea dei sensi}, in
\emph{Cibo per il cuore}, pp. 69-79. Discorso tenuto ai monaci riuniti
dopo la recitazione del \emph{Pāṭimokkha} al Wat Pah Pong durante il
Ritiro delle Piogge del 1978.

III,~49 (10). \emph{Nel cuore della notte}; tit. orig. ingl. \emph{In
the Dead of the Night\ldots{}}, vol. III, pp. 87-104; pubblicato per la prima
volta in \emph{Food for the Heart}; trad. ital. \emph{Nel cuore della
notte}, in \emph{Cibo per il cuore}, pp. 80-99. Discorso offerto per
un'osservanza lunare (\emph{Uposatha}) al Wat Pah Pong verso la fine del
1960.

III,~50 (11). \emph{Una sorgente di saggezza}; tit. orig. ingl.
\emph{The Fountain of Wisdom}, vol. III, pp. 105-120; pubblicato per la
prima volta in \emph{Food for the Heart}; trad. ital. \emph{Il contatto
sensoriale come fonte di saggezza}, in \emph{Cibo per il cuore}, pp.
100-116. Discorso offerto ai monaci riuniti dopo la recitazione del
\emph{Pāṭimokkha} al Wat Pah Pong durante il Ritiro delle Piogge del
1978.

III,~51 (12). \emph{Non è sicuro}; tit. orig. ingl. \emph{Not sure},
vol. III, pp. 121-134; pubblicato per la prima volta in \emph{Food for
the Heart}; pubblicato anche altrove con con il titolo \emph{Not Sure!
The Standard of the Noble Ones}; trad. ital. \emph{``Incerto!'': il
parametro dei nobili}, in \emph{Cibo per il cuore}, pp. 117-131.
Discorso informale offerto presso la \emph{kuṭī} di Ajahn Chah ad alcuni
monaci e novizi una sera dell'anno 1980.

III,~52 (13). \emph{Con tutto il cuore}; tit. orig. ingl.
\emph{Wholehearted Training}, vol. III, pp. 135-159; pubblicato per la
prima volta in \emph{Everything is Teaching Us}; trad. ital.
\emph{Addestrarsi con tutto il cuore}, in \emph{Tutto insegna}, pp.
73-103.

III,~53 (14). \emph{Retto contenimento}; tit. orig. ingl. \emph{Right
Restraint}, vol. III, pp. 161-173; pubblicato per la prima volta in
\emph{Everything is Teaching Us}; trad. ital. parziale \emph{Ascoltare
al di là delle parole}, in \emph{Tutto insegna}, pp. 136-140 (pp.
170-173 di \emph{Right} \emph{Restraint}, cfr. sopra) sulla base della
parziale ed. ingl. recante il titolo \emph{Listening Beyond Words}.

III,~54 (15). \emph{Soffrire in cammino}; tit. orig. ingl.
\emph{Suffering on the Road}, vol. III, pp. 175-203; pubblicato per la
prima volta in \emph{Living Dhamma}; nessuna trad. ital. individuata.
Discorso offerto a un gruppo di monaci in procinto di andar via dal
monastero dopo il loro quinto anno trascorso sotto la guida di Ajahn
Chah.

III,~55 (16). \emph{L'Occhio del Dhamma}; tit. orig. ingl. \emph{Opening
the Dhamma Eye}, vol. III, pp. 205-218; pubblicato per la prima volta in
\emph{A Taste of Freedom}; trad. ital. \emph{L'occhio del Dhamma}, in
\emph{Il sapore della libertà}, pp. 143-158. Discorso offerto al Wat Pah
Pong ai monaci e ai novizi nell'ottobre del 1968.

III,~56 (17). \emph{Il Sentiero verso la pace}; tit. orig. ingl.
\emph{The Path to Peace}, vol. III, pp. 219-241; pubblicato per la prima
volta in \emph{The Path to Peace}; trad. ital.
\url{http://santacittarama.altervista.org/sentiero.htm}

III,~57 (18). \emph{I gabinetti e il Sentiero}; tit. orig. ingl.
\emph{Toilets on the Path}, vol. III, pp. 243-264 (pp. 243-245,
introduzione di Ajahn Jayasaro). Questo discorso, originariamente
offerto in laotiano, è stato tradotto in thailandese per la biografia di
Ajahn Chah, \emph{Upalamani}; nessuna trad. ital. individuata.

III,~58 (19). \emph{Un messaggio dalla Thailandia}; tit. orig. ingl.
\emph{A Message from Thailand}, vol. III, pp. 265-267. Questo messaggio
di Ajahn Chah fu inviato ai suoi discepoli in Inghilterra mentre egli
risiedeva in una filiazione monastica, chiamata ``La Grotta della Luce
del Diamante'', immediatamente prima che la sua salute peggiorasse
gravemente durante il Ritiro delle Piogge del 1981; nessuna trad. ital.
individuata.
