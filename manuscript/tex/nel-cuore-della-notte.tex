\chapter{Nel cuore della notte}

\begin{openingQuote}
  \centering

  Discorso offerto per un'osservanza lunare (uposatha) al Wat Pah Pong
  verso la fine del 1960.
\end{openingQuote}

Date un'occhiata alle vostre paure. Un giorno, mentre si avvicinava il
calare della notte, non avevo alternative. Se avessi cercato di
spingermi ad andare ragionandoci su, non l'avrei mai fatto, e così
agguantai un \emph{pah-kao}\footnote{\emph{Pah-kao}: Termine thailandese
  (\thai{ผ้าขาว; ปะขาว}) per \emph{anāgārika}; letteralmente, ``non
  cittadino'', ossia ``senza casa'' un postulante che ha assunto gli
  Otto Precetti.} e mi misi in cammino. «~Se è giunta la mia ora, che
allora sia. Se la mia mente è così ostinata e stupida, che muoia pure.~»
Così pensavo tra me e me. In realtà nel mio cuore non è che volessi
davvero andare, ma mi forzai a farlo. Quando si tratta di cose come
questa, se si aspetta fino a quando è il momento giusto, si finisce per
non andare. Addestreremo mai noi stessi? Così andai e basta. Prima di
allora non mi ero mai trattenuto in un campo di cremazione.\footnote{I
  campi di cremazione isolati nelle foreste ben si prestano a essere
  teatro di inquietanti storie di fantasmi, del resto molto presenti
  nella cultura popolare thailandese.} Non riesco a descrivere con le
parole come mi sentivo quando arrivai. Il \emph{pah-kao} voleva
accamparsi proprio accanto a me, ma io non volli. Gli dissi di
sistemarsi lontano da me. In verità desideravo che mi stesse vicino a
farmi compagnia, ma nello stesso tempo non volevo. Lo feci allontanare,
perché altrimenti avrei fatto affidamento sul suo sostegno. «~Se la mia
mente ha tutta questa paura, allora che io muoia questa notte.~» Paura
sì che ne avevo, ma la affrontai. Non è che non avessi paura, è che ero
coraggioso. Tanto alla fine bisogna comunque morire.

Bene, proprio quando stava facendo buio ebbi la mia occasione: portarono
un cadavere. Fortunato come al solito! I miei piedi non li sentivo
toccare neanche il terreno, tanto volevo andarmene al più presto da lì.
Vollero che fossi io a recitare i canti funebri, ma non intendevo essere
coinvolto e me ne andai. Nel giro di pochi minuti, dopo che erano
usciti, tornai indietro e vidi che avevano depositato il cadavere
proprio vicino al posto in cui io mi ero accampato, e avevano
trasformato in un letto per me la lettiga di bambù usata per trasportare
il cadavere. Cos'altro mi restava da fare? Il villaggio non era nelle
vicinanze, era a due o tre chilometri di distanza. «~Bene, se devo
morire, morirò.~» Se non si ha mai il coraggio di farlo, non si saprà
mai com'è. È veramente un'esperienza. Mentre diventava sempre più buio,
mi chiedevo dove sarei mai potuto scappare in mezzo a quel campo di
cremazione. «~E che io muoia, allora. Comunque è solo per morire che si
nasce.~»

Appena il sole calò, la notte mi disse di entrare nel mio
\emph{glot}.\footnote{\emph{Glot} (in thailandese \thai{กลค}): Ombrello con una
  zanzariera tutt'intorno all'estremità, utilizzato sia per la
  meditazione sia come riparo dai monaci che intraprendono i
  \emph{dhutaṅga}; viene appeso ai rami degli alberi così da potercisi
  sedere sotto, al riparo dagli insetti; questo è un termine diverso
  rispetto a quello utilizzato per l'ombrello dei laici, \emph{rom} (in
  thailandese \thai{ร่ม}).} Non avevo proprio alcuna intenzione di fare la
meditazione camminata, volevo solo entrare nel mio \emph{glot}. Tutte le
volte che camminavo verso il cadavere, era come se qualcosa mi tirasse
per farmi tornare indietro, per impedirmi di camminare. Era come se la
mia paura e il mio coraggio facessero a braccio di ferro. Però ci andai.
È così che dovete addestrare voi stessi. Quando si fece buio entrai
nella mia zanzariera. Mi sembrava un muro inespugnabile. Vedere vicino a
me la mia fedele ciotola per la questua era come vedere un vecchio
amico. A volte perfino una ciotola può essere un amico. La sua presenza
al mio fianco mi confortava. Almeno una ciotola mi era amica. Mi misi a
sedere nella mia zanzariera e osservai attentamente per tutta la notte
il cadavere. Non mi coricai e nemmeno mi assopii, rimasi seduto in
silenzio, tutto qui. Anche se avessi voluto dormire, non ci sarei
riuscito, tanta era la paura. Sì, avevo paura, tuttavia lo feci. Restai
seduto per tutta la notte.

Chi ha abbastanza fegato da praticare in questo modo? Provateci e
vedrete. Quando si tratta di esperienze di questo genere, chi avrebbe il
coraggio di andare a stare in un campo di cremazione? Se non lo fate
realmente, non ne otterrete i benefici, non praticate davvero. Quella
volta praticai veramente. Quando giunse l'alba pensai: «~Sono
sopravvissuto!~» Ero proprio contento. Volevo solo che arrivasse il
giorno, la notte mi era bastata. Volevo fare fuori tutto il buio della
notte e lasciare che restasse solo la luce del giorno. Mi sentivo
proprio bene, ero riuscito a sopravvivere. Pensai: «~Non è nulla, è solo
la mia paura, questo è tutto.~» Dopo il giro per la questua e dopo aver
mangiato mi sentivo bene. Giunse la luce del sole che, con il suo
calore, mi fece sentire a mio agio. Mi riposai e feci per un po' la
meditazione camminata. Pensai: «~Siccome la notte scorsa ce l'ho fatta,
questa sera farò meditazione, ed essa andrà bene e sarà serena. Non c'è
nulla di più da fare.~»

Poi, nel pomeriggio -- chi lo avrebbe mai detto? -- arrivò un altro
cadavere, questa volta uno bello grosso.\footnote{Il cadavere della
  prima notte era quello di un bambino.} Lo portarono dentro e lo
cremarono proprio vicino al posto in cui stavo, proprio di fronte al mio
\emph{glot}. Certamente peggio della notte già trascorsa! «~Bene --
pensai -- molto bene.~» «~Il fatto che abbiano portato proprio qui
questo cadavere per la cremazione mi aiuterà nella pratica.~» Comunque
non andai a svolgere alcun rito e prima di andare a guardare aspettai
che tutti se ne fossero andati. Non riesco a descrivervi come mi sentivo
mentre stavo lì seduto a osservare per tutta la notte quel corpo che
bruciava. Non so come esprimere la paura che provavo. Badate bene, nel
cuore della notte. Mentre le fiamme crepitavano lievemente, dal cadavere
ardente il fuoco sfavillava rosso e verde. Volevo fare la meditazione
camminata di fronte al corpo, ma non riuscivo a convincermi a farlo.
Alla fine entrai nella mia zanzariera. Il fetore che si levava dalla
carne bruciata ristagnò per tutta la notte.

Tutto questo si verificò prima che iniziasse a succedere davvero
qualcosa. Mentre le fiamme continuavano a crepitare in modo sommesso,
volsi le spalle al fuoco. A dormire non ci pensavo nemmeno, la paura mi
faceva tenere gli occhi sbarrati. E non c'era nessuno al quale
rivolgermi, c'ero solo io. Potevo far affidamento solo su me stesso. Non
c'era posto nel quale potessi pensare di andare, un posto nel quale
scappare in quella notte nera come la pece. «~Resterò seduto e morirò
qui. Da qui non mi muovo.~» La mente ordinaria potrebbe voler fare
queste cose? Vi condurrebbe in una situazione come questa? Se provaste a
ragionarci su, non ci andreste mai. Chi potrebbe desiderare di fare una
cosa del genere? Se non si ha una salda fiducia nell'insegnamento del
Buddha, è impossibile farlo.

Verso le dieci ero ancora seduto con le spalle rivolte al fuoco. Non so
cosa fosse, ma da dietro giunse un rumore, come uno strascichio. La pira
aveva ceduto? O forse un cane stava cercando di impadronirsi del
cadavere. Ma no, sembrava piuttosto un bufalo che si aggirava lì
attorno. «~Non preoccuparti.~» Però, fu allora che iniziò a camminare
verso di me, proprio come una persona! Camminava dietro di me, con il
passo pesante di un bufalo, o forse no. Le foglie crocchiavano sotto
quei passi come se ora stesse di fronte a me. Bene, potevo solo
prepararmi al peggio, dove sarei mai potuto andare? Però non veniva
proprio verso di me, era come se mi girasse davanti in tondo e poi
andasse in direzione del \emph{pah-kao}. Poi ci fu il silenzio. Non
sapevo di cosa si trattasse, ma la paura mi fece pensare a varie
possibilità.

Credo che passò circa mezz'ora prima che dei passi iniziassero a
provenire dalla direzione del \emph{pah-kao}. Proprio come quelli di una
persona! Questa volta veniva giusto verso di me, come se volesse
calpestarmi! Chiusi gli occhi, mi rifiutai di aprirli. «~Morirò a occhi
chiusi.~» Si avvicinò sempre di più, finché si fermò esattamente di
fronte a me e rimase immobile in piedi. Ebbi l'impressione che due mani
ardenti si muovessero avanti e indietro davanti ai miei occhi chiusi.
Oh! Era proprio così! Non mi importò più di nulla, dimenticai tutto del
Buddha, del Dhamma e del Saṅgha. Dimenticai qualsiasi altra cosa, in me
c'era solo una paura che mi riempiva tutto. I miei pensieri non potevano
andare da nessun'altra parte, c'era soltanto la paura. Da quando sono
nato non ho mai provato una paura come quella. Il Buddha e il Dhamma
erano scomparsi, non so dove fossero andati a finire. C'era solo tanta
di quella paura che mi si riversava nel petto al punto da farmelo
sentire teso come la pelle di un tamburo.

«~Bene, lascerò le cose così come stanno, non c'è nient'altro da fare.~»
Stavo seduto, era come se nemmeno toccassi il terreno, e mi limitai
semplicemente a osservare quello che succedeva. La paura era così tanta
da riempirmi completamente, come una giara colma d'acqua. Se versate
acqua fino a quando è del tutto piena e poi ne versate ancora un po',
traboccherà. Allo stesso modo, la paura crebbe in me fino a raggiungere
il punto massimo, e poi iniziò a tracimare. Dentro di me una voce
chiese: «~Di che cos'è che ho così paura?~» Un'altra voce rispose: «~Ho
paura di morire.~» «~Bene allora. Dov'è questa cosa, la ``morte''?
Perché tutto questo panico? Guarda dove la morte dimora. Dov'è la
morte?~» «~La morte è dentro di me!~» «~Se la morte è dentro di te, dove
pensi di poter scappare per sfuggirle! Se scappi via, muori. Se resti
qui, muori. Ovunque tu vada essa viene con te, perché la morte è dentro
di te, non puoi scappare da nessuna parte. Che tu abbia paura o no,
morirai ugualmente, non si può fuggire dalla morte.~» Quando pensai
queste cose, la mia percezione mutò completamente. Tutta quella paura
scomparve per intero, con la stessa facilità con cui si volge verso
l'alto la palma della mano. Fu davvero stupefacente. Una paura così
grande che sparisce in questo modo! Al suo posto sorse
l'assenza-di-paura. Percepii che la mia mente saliva sempre più su,
finché mi sentii come se mi trovassi tra le nuvole.

Appena vinsi la paura, iniziò a piovere. Non so di che pioggia si
trattasse, visto che il vento era così forte. Però non avevo più paura
di morire. Nemmeno temevo che i rami degli alberi, spezzandosi,
potessero abbattersi su di me. Non ci prestai attenzione. La pioggia
veniva giù scrosciando in modo davvero pesante, come un torrente nella
stagione calda. Quando smise di piovere, tutto era bagnato fradicio.
Restai seduto, immobile. Cosa feci allora, zuppo com'ero? Piansi! Le
lacrime mi scorrevano giù per le guance. Piansi quando pensai: «~Perché
sto qui seduto come un orfano o un bambino abbandonato, seduto, a
bagnarmi sotto la pioggia come chi non ha nulla, come un esule?~» Poi
ancora: «~Tutta quella gente che in questo momento sta seduta
comodamente nella propria casa forse non sospetta nemmeno che un monaco
stia qui per tutta la notte, seduto e fradicio per la pioggia. A che
serve tutto questo?~» Pensare queste cose mi fece sentire così
profondamente addolorato che le lacrime sgorgarono a profusione.
Praticavo così.

Ora non so come descrivere quello che avvenne dopo. Stavo seduto e
ascoltavo. Dopo aver vinto le mie sensazioni, rimasi seduto a osservare
tutte le cose che mi sorgevano dentro. Erano così tante che era
possibile conoscerle, ma impossibile descriverle. Allora pensai alle
parole del Buddha, \emph{paccattaṃ veditabbo viññūhi}: «~Il saggio
conoscerà da sé.~» Avevo sopportato tutta quella sofferenza ed ero
rimasto seduto sotto la pioggia. Chi era lì a fare questa esperienza con
me? Solo io posso sapere come fu. C'era stata così tanta paura, e
tuttavia la paura era scomparsa. Chi altro potrebbe testimoniarlo? La
gente nella propria casa di città non ne sapeva nulla, solo io potevo
capirlo. Era un'esperienza personale. Qualora lo avessi detto ad altre
persone, non avrebbero potuto capire veramente, era una cosa di cui
ognuno avrebbe dovuto fare esperienza da sé. Più lo contemplavo, più
diveniva chiaro. Prima che spuntasse il sole divenni sempre più forte,
il mio convincimento divenne sempre più stabile.

Quando all'alba aprii gli occhi, tutto era giallo. Durante la notte
avevo avvertito il bisogno di urinare, ma alla fine quella sensazione se
n'era andata. Quando al mattino mi rialzai dalla posizione seduta, tutto
quel che guardavo era giallo, proprio come avviene alcuni giorni con la
prima luce del mattino. Quando andai a urinare, nelle urine c'era del
sangue! «~Eh? Nel mio ventre c'è una lacerazione o qualcosa del
genere?~» Ebbi un po' di paura. «~Forse qui dentro si è davvero lacerato
qualcosa.~» Subito dentro di me una voce mi disse: «~Bene, e allora? Se
si è lacerato qualcosa, s'è lacerato, con chi me la posso prendere?~»
Quella voce poi aggiunse: «~Se s'è lacerato, s'è lacerato. Se muoio,
muoio. Io stavo solo seduto qui, non stavo facendo nulla di male. Se
qualcosa sta per spaccarsi, che si spacchi pure.~» Era come se la mia
mente stesse discutendo o combattendo con se stessa. Da una parte
giungeva una voce: «~Ehi, si tratta di una cosa pericolosa!~» Un'altra
voce la contrastava e la sfidava, sovrastandola. La mia urina era
macchiata di sangue. «~Hmm. Dove posso trovare una medicina?~» «~Non
intendo preoccuparmi. E comunque un monaco non può tagliare piante per
curarsi. Se muoio, muoio, e allora? Che altro c'è da fare? Se muoio
mentre pratico, sono pronto allora. Se io dovessi morire mentre faccio
qualcosa di male questo non andrebbe bene, ma se devo morire praticando
sono pronto.~»

Non seguite i vostri stati mentali. Addestrate voi stessi. La pratica
consiste anche nel mettere in gioco la vostra stessa vita. Dovreste aver
pianto almeno due o tre volte. Va bene, così è la pratica. Se vi sentite
assonnati e volete coricarvi, non consentitevelo. Prima di coricarvi
fate in modo che la sonnolenza vada via. Guardatevi invece, tutti voi
che non sapete come praticare. A volte, quando tornate dalla questua e
contemplate il cibo prima di mangiare, non ce la fate a calmarvi, la
vostra mente somiglia a un cane impazzito. Siete così affamati che la
saliva vi riempie la bocca. Talvolta non vi preoccupate neanche di
contemplare, vi ci buttate sopra e basta. È un disastro. Se la vostra
mente non si calma e non è in grado di pazientare, spingete via la
ciotola e non mangiate. Addestratevi, esercitatevi, questa è la pratica.
Non continuate solo a seguire la vostra mente. Spingete via la ciotola,
alzatevi e andatevene, non consentitevi di mangiare. Se la mente vuole
davvero così tanto mangiare e agisce così caparbiamente, allora non
permettetele di mangiare. La saliva andrà via. Se sanno che non avranno
nulla da mangiare, le contaminazioni si spaventeranno. Il giorno dopo
non oseranno disturbarvi, avranno timore di non aver nulla da mangiare.
Provateci se non mi credete.

La gente non ripone la propria fiducia nella pratica, non osa praticare
veramente. Ha paura di aver fame, ha paura di morire. Se non ci provate,
non saprete di cosa si tratta. La maggior parte di noi non osa
praticare, non osiamo provarci, abbiamo paura. Ho sofferto a lungo per
il cibo e per cose di questo genere, e perciò conosco il problema, e
questo è solo un problema di scarso rilievo. È per questo motivo che la
nostra pratica non è facile. Qual è la cosa più importante di tutte?
Rifletteteci su. La morte, solo questo. La morte è la cosa più
importante del mondo. Riflettete, praticate, indagate. Se non avete di
che vestirvi, non morirete. Se non avete betel da masticare o sigarette
da fumare, nemmeno in questo caso morirete. Se però non avete riso o
acqua, allora sì che morirete. In questo mondo solo queste due cose
considero essenziali. Avete bisogno di riso e di acqua per nutrire il
corpo. È per questa ragione che non mi interessava nient'altro, mi
accontentavo di tutto quello che mi veniva offerto. Finché avevo riso e
acqua, avevo a sufficienza per praticare, ero soddisfatto. E per voi è
sufficiente? Tutte le altre cose non sono essenziali. Che le otteniate o
meno non importa, le cose veramente importanti sono il riso e l'acqua.
Mi chiedevo: «~Se vivo in questo modo, posso sopravvivere?~» «~C'è
abbastanza per andare avanti bene. Probabilmente riesco a ottenere
almeno del riso durante la questua in quasi tutti i villaggi, un boccone
da ogni casa. L'acqua è di norma disponibile. Queste due cose sono
sufficienti.~» Non miravo ad avere di più.

Per quanto concerne la pratica, di solito giusto e sbagliato coesistono.
Dovete osare praticare, dovete osare farlo. Se non siete mai stati in un
campo di cremazione, dovreste esercitarvi per andarci. Se non riuscite
ad andarci di notte, andateci di giorno. Esercitatevi ad andarci sempre
un po' più tardi, finché riuscite ad andarci quando è quasi buio, e
rimaneteci. Allora vedrete gli effetti della pratica, allora capirete.

Questa mente è preda dell'illusione da chissà quante vite. Vogliamo
evitare tutto quello che non ci piace e che non amiamo. Siamo indulgenti
con le nostre paure. E poi diciamo che stiamo praticando. Questa non può
essere chiamata ``pratica''. Se si trattasse davvero di pratica,
rischiereste anche la vita. Se foste veramente determinati a praticare
perché dovreste mai interessarvi di questioni del tutto secondarie? «~Ne
ho avuto solo un po', tu invece molto.~» «~Tu non sei stato d'accordo
con me e io non lo sono con te.~» Non ho mai pensato cose di questo
genere, perché non mi interessavano. Quello che gli altri facevano era
affar loro. Quando andavo in altri monasteri non mi facevo coinvolgere
da queste cose. Non mi curavo del fatto che la pratica altrui fosse di
alto o di basso livello, badavo solo ai fatti miei. E così osai
praticare, e la pratica fece sorgere la saggezza e la visione profonda.

Se la vostra pratica è davvero corretta, allora state praticando
veramente. Praticate giorno e notte. Di notte, quando c'era silenzio,
sedevo in meditazione e poi facevo la meditazione camminata, le
alternavo almeno due o tre volte ogni notte. Meditazione camminata, poi
meditazione seduta, poi ancora un po' di meditazione camminata. Non mi
annoiavo, mi piaceva. A volte pioveva appena e pensavo a quando lavoravo
nelle risaie. I pantaloni già indossati il giorno prima erano ancora
bagnati, ma ero costretto ad alzarmi prima dell'alba e a metterli di
nuovo. Poi dovevo scendere sotto casa a tirare fuori il bufalo dal suo
recinto. Tutto quel che potevo vedere del bufalo era coperto dai suoi
stessi escrementi. Il bufalo faceva poi piroettare la coda e me li
spargeva addosso. Mentre continuavo a camminare con i piedi doloranti --
avevo il piede d'atleta -- pensavo: «~Perché la mia vita è così
miserevole?~» Allora stavo invece facendo la meditazione camminata: per
me che poteva mai essere solo un po' di pioggia? Pensavo a queste cose
per incoraggiarmi a praticare.

Se la pratica è ``entrata nella Corrente'', allora non può essere
paragonata a nulla. Non c'è sofferenza simile a quella del praticante di
Dhamma e nemmeno c'è felicità simile alla sua. Non c'è zelo che possa
essere paragonato allo zelo del praticante di Dhamma e nemmeno c'è
pigrizia simile alla sua. I praticanti del Dhamma sono i migliori in
tutto. Ecco perché dico che, se praticate davvero, ne vale proprio la
pena. Però, la maggior parte di noi si limita a parlare di pratica senza
aver praticato, senza essere mai giunta a praticare. La nostra pratica
somiglia a un uomo che ha una casa nella quale da una parte ci piove
perché il tetto ha una falla. Lui si limita però a dormire dall'altra
parte. Quando il sole entra dalla falla e lo raggiunge, si sposta di
nuovo e pensa: «~Quand'è che avrò una casa decente, come quella di tutti
gli altri?~» Se poi è tutto il tetto a perdere, si alza e se ne va. Non
è questo il modo di fare le cose, ma la maggior parte della gente è
così.

Questa nostra mente, queste contaminazioni: se le seguite vi daranno
problemi. Più le seguite, più la pratica degenera. Quando praticate
veramente, a volte vi stupite da soli per il vostro stesso zelo. Che gli
altri pratichino o meno, la cosa non vi riguarda, pensate solo a
praticare voi con costanza. Chi va e chi viene non importa, praticate e
basta. Dovete guardare voi stessi, prima che possa essere chiamata
``pratica''. Quando praticate davvero non ci sono conflitti nella vostra
mente, c'è solo il Dhamma. Ovunque incontriate delle difficoltà, ovunque
abbiate delle mancanze, è proprio lì che dovete impegnarvi. Finché non
ce la fate, non mollate. Dopo aver risolto una cosa, vi bloccherete in
un'altra, e perciò continuate finché ci riuscite, continuate a
sforzarvi. Non sentitevi soddisfatti prima di aver portato a termine il
lavoro. Concentrate la vostra attenzione su quel punto. Mentre sedete,
mentre camminate, mentre siete distesi, guardate proprio quel punto.

È proprio come un contadino che non ha ancora finito il suo lavoro. Ogni
anno pianta il riso, ma quest'anno non ce l'ha fatta a finire del tutto,
e per questa ragione la sua mente si è bloccata lì, non riesce a
riposare tranquillo. Non riesce a rilassarsi nemmeno quando sta con gli
amici, è sempre infastidito dal pensiero del lavoro non portato a
termine. Oppure è come quando una madre lascia il figlio al piano di
sopra, mentre lei va di sotto a dar da mangiare agli animali. Ha il
figlio in mente in continuazione, teme che possa cadere. Anche se fa
altro, il figlio è sempre presente nei suoi pensieri. Per la nostra
pratica è la stessa cosa: non la dimentichiamo mai. Possiamo anche fare
altre cose, ma la nostra pratica è presente nei nostri pensieri, è
costantemente con noi, giorno e notte. Deve essere così, se davvero
volete fare progressi.

All'inizio dovete far affidamento su un insegnante che vi istruisca e vi
consigli. Comprendete e poi praticate. Quando l'insegnante vi dà delle
indicazioni, seguitele. Se capite la pratica non è più necessario che
l'insegnante vi istruisca, potete fare il lavoro da soli. Tutte le volte
che sorgono distrazione o stati mentali non salutari, riconoscetelo da
voi stessi, insegnate a voi stessi. Praticate da voi stessi. La mente è
Colui che Conosce, il testimone. La mente conosce da sé se avete molte
illusioni o se ne avete solo poche. Ovunque siate ancora manchevoli,
cercate di praticare proprio in quel punto, dedicateci tutta la vostra
attenzione. Così è la pratica. È quasi come essere pazzi, o potete anche
dire che si è pazzi. Quando praticate davvero siete folli, siete
``sottosopra''. Prima la vostra percezione è distorta, e poi la
aggiustate. Se non l'aggiustate, avrete gli stessi problemi di prima,
starete altrettanto male di prima.

C'è molta sofferenza nella pratica, ma se non conoscete la vostra stessa
sofferenza non comprenderete la Nobile Verità della sofferenza. Per
comprendere la sofferenza, per eliminarla, dovete prima incontrarla. Se
volete cacciare un uccello, ma non uscite a cercarlo e non lo trovate,
come potrete mai sparargli? Sofferenza, sofferenza \ldots{} il Buddha insegnò
in relazione alla sofferenza. La sofferenza della nascita, la sofferenza
della vecchiaia. Se non volete sperimentare la sofferenza, non vedrete
la sofferenza. Se non vedete la sofferenza, non comprenderete la
sofferenza. Se non comprendete la sofferenza, non sarete in grado di
vincere la sofferenza. Ora la gente non vuole vedere la sofferenza, non
vuole sperimentarla. Se soffrono qui, se ne scappano là. Capite? Se la
stanno solo trascinando dietro, non la uccideranno mai. Non la
contemplano, non la investigano. Se provano sofferenza qui, se ne
scappano lì. Se è lì che sorge, scappano di nuovo qui. Cercano di
fuggire fisicamente dalla sofferenza. Fin quando sarete ignoranti,
incontrerete la sofferenza ovunque andiate. Anche se salite su un
aeroplano per andare via dalla sofferenza, essa salirà sull'aeroplano
con voi. Se vi tuffate in acqua, si tufferà con voi, perché la
sofferenza è dentro di noi. Però non lo sappiamo. Se sta dentro di noi,
dove possiamo scappare per sfuggirle?

La gente soffre in un posto e così se ne va da qualche altra parte.
Quando la sofferenza sorge, se ne va di nuovo. Le persone pensano di
scappare dalla sofferenza ma non è così, perché la sofferenza li segue.
Se la portano dietro senza saperlo. Se non conosciamo la causa della
sofferenza, non possiamo conoscere la cessazione della sofferenza, non
c'è modo di sfuggirle. Dovete guardare dentro tutto questo con
determinazione, fino a che non andate al di là del dubbio. Dovete osare
praticare. Non eludete la pratica, sia in gruppo che da soli. Se gli
altri sono pigri non importa. A chiunque faccia molta meditazione
camminata, a chiunque pratichi molto garantisco che otterrà dei
risultati. Se davvero praticate costantemente, anche se gli altri vanno
e vengono o quale che sia la situazione, un Ritiro delle Piogge è
sufficiente. Fate come vi ho appena detto. Ascoltate quel che vi dice
l'insegnante, non siate polemici, non siate testardi. Qualsiasi cosa vi
venga detto di fare, andate avanti e fatela. Non c'è bisogno di avere
timore della pratica, da essa certamente sorgerà la Conoscenza.

Pratica è anche \emph{paṭipadā}.\footnote{\emph{Paṭipadā}: Strada, via,
  sentiero; i mezzi per raggiungere lo scopo o la destinazione finale,
  il Nibbāna.} Che cos'è \emph{paṭipadā}? Pratica uniforme,
costante. Non praticate come il vecchio Peh. Durante un Ritiro delle
Piogge decise di smettere di parlare. Smise di parlare, e va bene, però
iniziò a scrivere dei biglietti. «~Per favore, domani fatemi del riso
tostato.~» Voleva mangiare riso tostato! Smise di parlare ma scrisse
così tanti bigliettini da essere ancor più distratto di prima. Ora
scriveva una cosa, ora un'altra, che farsa! Non so perché prese la
decisione di non parlare. Non sapeva che cosa fosse la pratica.

In realtà la nostra pratica consiste nell'accontentarsi di poco, solo
nell'essere naturali. Non preoccupatevi sia che vi sentiate pigri sia
che vi sentiate diligenti. Non dite neanche né «~sono diligente~» né
«~sono pigro.~» La maggior parte delle persone pratica solo quando si
sente diligente e, se si sente pigra, non si preoccupa di farlo. Di
solito la gente è così. I monaci non dovrebbero pensare in questo modo.
Se siete diligenti praticate, quando siete pigri praticate ugualmente.
Non vi preoccupate di altre cose, tagliatele via, gettatele, addestrate
voi stessi. Praticate con costanza, indipendentemente dal fatto che sia
giorno o notte, quest'anno, l'anno dopo, come che sia e quando che sia,
non prestate attenzione a pensieri di diligenza o di pigrizia, non
preoccupatevi del freddo o del caldo, praticate e basta. Questa è
chiamata \emph{sammā-paṭipadā}, retta pratica.

Alcuni s'impegnano veramente nella pratica per sei o sette giorni. Poi,
quando non ottengono i risultati che desiderano, ci rinunciano e
cambiano del tutto direzione, indulgono alle chiacchiere, socializzano e
così via. Poi si rammentano della pratica e vi si dedicano per altri sei
o sette giorni, e poi l'abbandonano di nuovo. Somiglia al modo di
lavorare di certe persone. Inizialmente ci si buttano a capofitto, poi,
quando si fermano, non si preoccupano neanche di raccattare i loro
strumenti di lavoro, se ne vanno e li lasciano lì. Più tardi, quando la
terra si è completamente indurita, si ricordano del loro compito e
lavorano ancora un po', ma solo per poi andarsene di nuovo. Facendo le
cose in questo modo non avrete mai un orto o una risaia decente. Per la
nostra pratica è la stessa cosa. Se pensate che \emph{paṭipadā} non sia
importante, non arriverete da nessuna parte con la pratica. \emph{Sammā
paṭipadā} è di assoluta importanza. Praticate costantemente. Non fate
caso al vostro umore. Che cambia se siete di buon umore o no? Il Buddha
non si preoccupava di queste cose. Sperimentò tutte le cose buone e
tutte quelle cattive, quelle giuste e quelle sbagliate. Questa era la
sua pratica. Prendere solo quel che vi piace e scartare tutto quello che
non vi piace non è praticare, è un disastro. Ovunque andrete non sarete
mai soddisfatti, ovunque starete lì ci sarà sofferenza.

Chi pratica in questo modo fa come i brāhmaṇi,\footnote{\emph{Brāhmaṇo}: Membro
  della casta dei brāhmaṇi, ``sacerdote''; la casta dei brāhmaṇi in
  India ha per molto tempo ritenuto che, per nascita, i suoi componenti
  fossero degni del più alto rispetto; si veda \emph{brāhmaṇa}, nel
  \emph{Glossario}, p. \pageref{glossary-brahmana}.} con le loro cerimonie sacrificali. Perché le fanno?
Perché vogliono qualcosa in cambio. Alcuni di noi praticano così. Perché
pratichiamo? Perché miriamo alla rinascita, a un'altra condizione
dell'esistenza, vogliamo ottenere qualcosa. Se non otteniamo quel che
desideriamo, ecco che non vogliamo praticare, proprio come fanno i
brāhmaṇi con le loro cerimonie sacrificali. Si comportano così a causa
del desiderio. Il Buddha non insegnò queste cose. La coltivazione della
pratica serve alla rinuncia, a lasciar andare, a fermarsi, a sradicare,
non a praticare per rinascere in una qualche particolare condizione.

Una volta c'era un \emph{Thera} che inizialmente aveva lasciato casa per
entrare nella setta dei \emph{Mahānikāya}. Aveva però ritenuto che non
fosse abbastanza rigorosa, e perciò assunse l'ordinazione
\emph{Dhammayuttika}.\footnote{\emph{Mahānikāya} e
  \emph{Dhammayuttika} sono le due principali sette del Saṅgha del
  Theravāda in Thailandia.} Poi iniziò a praticare. A volte digiunava
per quindici giorni, poi, quando mangiava, si cibava solo di foglie ed
erba. Pensava che cibarsi di animali significasse accumulare cattivo
kamma, che fosse meglio mangiare foglie ed erba. Dopo un po' di
tempo pensò: «~Hmm. Essere monaco non va poi così bene, non è opportuno.
È difficile mantenere la mia pratica vegetariana come monaco. Meglio
lasciare l'abito e diventare un \emph{pah-kao}.~» Così lasciò l'abito
monastico e divenne un \emph{pah-kao} per poter raccogliere da sé foglie
ed erba, e scavare nella terra per procurarsi radici e taro. Andò avanti
in questo modo per un po', fino a quando alla fine non sapeva più cosa
fare. Lasciò perdere tutto. Aveva lasciato l'abito monastico, poi smise
di essere un \emph{pah-kao}, infine lasciò perdere tutto. Ora non so che
cosa stia facendo. Forse è morto, non so. Tutto questo avvenne perché
non era riuscito a trovare nulla che fosse adatto alla sua mente. Non
comprese che stava solo seguendo le contaminazioni. Erano le
contaminazioni a condurlo, ma lui non lo sapeva.

Il Buddha lasciò l'abito monastico e divenne un \emph{pah-kao}? Come
praticava il Buddha? Che cosa faceva? Queste cose non le prese in
considerazione. Il Buddha si mise a mangiare foglie ed erba come una
mucca? Certo, se volete mangiare in quel modo fatelo pure, se è tutto
quello che riuscite a fare, ma non andate in giro a criticare gli altri.
Quale che sia il tipo di pratica che ritenete adatta a voi, perseverate
con quella. «~Non usare troppo la sgorbia, non intagliare troppo il
manico, se vuoi che conservi la sua funzione.~»\footnote{È la traduzione
  di un proverbio thailandese che significa ``non strafare''.} Non ti
resterà nulla e alla fine lascerai perdere tutto. Alcuni sono così.
Quando si tratta di meditazione camminata la fanno seriamente per
quindici giorni o giù di lì. Non si preoccupano neanche di mangiare,
camminano e basta. Poi, quando hanno finito, si sdraiano e si mettono a
dormire. Non si preoccupano di riflettere con attenzione prima di
iniziare a praticare. Alla fine non c'è nulla che sia adatto a loro.
Essere monaco non va bene, essere un \emph{pah-kao} non va bene, e alla
fine non resta nulla.

Le persone fatte in questo modo non conoscono la pratica, non
considerano le ragioni che motivano la pratica. Pensate al motivo per
cui praticate. Questo Insegnamento serve a lasciar andare, a rinunciare.
La mente vuole amare questa persona e odiare quella. Si tratta di cose
che possono sorgere, ma non ritenetele reali. Allora, per quale ragione
stiamo praticando? Proprio per rinunciare a queste cose. Anche se volete
la pace, gettatela via. Se la conoscenza sorge, gettate via la
conoscenza. Se avete la conoscenza sapete, ma se considerate quella
conoscenza come vostra, allora pensate di sapere qualcosa. Poi pensate
di essere migliori degli altri. Dopo un po' non riuscite a vivere da
nessuna parte, ovunque viviate nascono problemi. Se praticate in modo
errato è come se non praticaste affatto.

Praticate a seconda delle vostre capacità. Dormite molto? Allora provate
ad andare controcorrente. Mangiate molto? Cercate di mangiare meno.
Prendete tutta la pratica di cui avete bisogno, utilizzando quale
fondamento \emph{sīla}, \emph{samādhi} e \emph{paññā}. E poi impegnatevi
pure nelle pratiche \emph{dhutaṅga}. Queste pratiche \emph{dhutaṅga}
servono a scavare nelle contaminazioni. Può succedere che per voi le
pratiche basilari non siano sufficienti per sradicare davvero le
contaminazioni, e allora dovete avvalervi anche delle pratiche
\emph{dhutaṅga}. Le pratiche \emph{dhutaṅga} sono proprio utili. Alcuni
non riescono a eliminare le contaminazioni con i basilari \emph{sīla} e
\emph{samādhi}, e per aiutarsi devono integrare nel loro addestramento
le pratiche \emph{dhutaṅga}. Eliminano molte cose. Vivere ai piedi di un
albero non va contro i precetti. Se però vi decidete per la pratica
\emph{dhutaṅga} di vivere in un campo di cremazione e poi non lo fate,
allora questo è un errore. Provateci. Com'è vivere in un campo di
cremazione? È come quando si vive in gruppo?

\emph{Dhutaṅga} si traduce con ``le pratiche difficili da fare''. Sono
le pratiche degli Esseri Nobili. Chiunque voglia diventare un Essere
Nobile deve avvalersi di queste pratiche \emph{dhutaṅga} per eliminare
le contaminazioni. È difficile mantenerle ed è difficile trovare persone
che si impegnino a praticarle, perché sono pratiche che vanno
controcorrente. Ad esempio dicono di disporre di un'unica veste
monastica, limitandosi ad avere solo i tre pezzi che la compongono, di
sostenersi con la questua, di mangiare solo dalla ciotola e di mangiare
solo quel che si ottiene dalla questua. Se dopo qualcuno porta del cibo,
non lo si accetta. Osservare quest'ultima pratica nel centro della
Thailandia è facile. Il cibo è più che adeguato, perché là mettono molto
cibo nella vostra ciotola. Quando però arrivate qui nel nord-est, questa
pratica \emph{dhutaṅga} assume sfumature sottili, perché qui si riceve
solo riso bianco. Questa pratica \emph{dhutaṅga} diventa allora davvero
ascetica. Si mangia solo riso bianco e tutto quel che viene offerto dopo
non viene accettato. Poi si mangia una sola volta al giorno, in una sola
ininterrotta seduta, dalla sola nostra ciotola. Quando si è terminato di
mangiare ci si alza dal posto in cui si è seduti e per tutto il giorno
non si mangia più. Queste sono chiamate pratiche \emph{dhutaṅga}. Chi le
pratica ora? Di questi tempi è difficile trovare qualcuno che si dedichi
a sufficienza per praticarle, perché sono impegnative. Proprio questa è
la ragione per cui sono così benefiche.

Quel che la gente al giorno d'oggi chiama pratica non è vera pratica.
Praticare davvero non è cosa facile. La maggior parte della gente non
osa praticare veramente, non osa andare davvero controcorrente. Non
vuole nulla che vada in senso contrario rispetto alle loro sensazioni.
La gente non vuole resistere alle contaminazioni, non vuole scavarci
dentro, non vuole vincerle. Nella nostra pratica ci viene detto di non
seguire gli stati mentali. Pensateci. Già per innumerevoli vite siamo
stati ingannati fino al punto di credere che la mente ci appartenga. In
verità non è così, la nostra mente è solo un impostore. Ci trascina
nell'avidità, ci trascina nell'avversione, ci trascina nell'illusione,
ci trascina nei furti, nei saccheggi, nel desiderio e nell'odio. Queste
cose non sono nostre. Proprio ora chiedete a voi stessi: vuoi essere
buono? Tutti vogliono essere buoni. E allora, fare tutte queste cose
significa essere buoni? Ecco! La gente commette atti malvagi, ma vuole
essere buona. Questa è la ragione per cui dico che queste cose sono
degli imbroglioni, ecco cosa sono.

Il Buddha non voleva che seguissimo questa mente, voleva che la
addestrassimo. Se va da una parte, rifugiatevi dall'altra. Quando va di
là, tornate a rifugiarvi di qua. Per dirla in modo semplice, qualsiasi
cosa la mente voglia, non fategliela avere. È come se fossimo stati
amici per anni, ma alla fine arriviamo al punto che non abbiamo più le
stesse idee. Ci separiamo e proseguiamo per vie diverse. Non ci
comprendiamo più a vicenda. Infatti litighiamo perfino, e per questo ci
separiamo. È giusto, non seguite la vostra mente. Chiunque segua la
propria mente, segue ciò che a essa piace e desidera e così via. Quella
persona non ha ancora praticato affatto. Per questo motivo dico che
quello che la gente chiama pratica, in realtà non è pratica, è un
disastro. Se non vi fermate a osservare, se non provate a praticare, non
vedrete, non realizzerete il Dhamma. Per dirla in modo diretto, nella
nostra pratica si deve mettere in ballo la vita stessa. Non è che sia
poi così difficile, è che questa pratica deve comprendere un po' di
sofferenza. Soprattutto il primo anno, o i primi due, c'è molta
sofferenza. I giovani monaci e i novizi vivono davvero tempi duri.

Ho avuto un sacco di difficoltà in passato, soprattutto con il cibo.
Cosa vi aspettavate? Diventare monaci a vent'anni, quando si pensa solo
a mangiare e dormire \ldots{} alcuni giorni sedevo da solo e sognavo di avere
del cibo. Avrei voluto mangiare banane sciroppate, o un'insalata di
papaia, e la saliva cominciava a scorrere. Fa parte dell'addestramento.
Tutte queste cose non sono facili. Questa faccenda del cibo e del
mangiare può condurre verso un kamma davvero molto cattivo.
Prendete uno che sta crescendo, che pensa solo a mangiare e dormire, e
costringetelo in questi abiti monastici: non sarà più in grado di
controllare le sue sensazioni. È come cercare di arginare dell'acqua che
scorre in modo torrenziale, a volte la diga si rompe. Se resta in piedi
è una cosa buona, ma se ciò non avviene crolla e basta. Durante il mio
primo anno, quando facevo meditazione non pensavo ad altro, solo al
cibo. Ero così irrequieto. A volte stavo seduto, ed era come se
assaporassi una banana. Mi sembrava quasi di prendere dei pezzi di
banana e di mettermeli in bocca. E tutto questo fa parte della pratica.
Non abbiate paura di queste cose. Tutti noi siamo stati ingannati per
innumerevoli vite, e così, ora che siamo giunti ad addestrare noi
stessi, non è facile. Però, anche se è difficile ne vale la pena. Perché
dovremmo occuparci di cose facili? Tutti possono fare cose facili.
Dovremmo addestrare noi stessi a fare quel che è difficile.

Per il Buddha fu la stessa cosa. Se si fosse unicamente preoccupato
della sua famiglia e dei suoi parenti, del suo patrimonio e dei piaceri
dei sensi provati in passato, non sarebbe mai diventato il Buddha. Non
si tratta di cose irrilevanti, anzi, è proprio quello che la maggior
parte della gente cerca. Lasciare il mondo in giovane età e rinunciare a
queste cose è proprio come morire. Però alcune persone vengono da me e
dicono: «~Oh, per te è facile Luang Por. Non hai mai avuto una moglie e
dei figli di cui preoccuparti, è per questo che per te è più facile!~»
Io rispondo così: «~Non avvicinarti troppo quando dici queste cose, se
non vuoi una botta in testa!~» \ldots{} Come se io fossi senza cuore! Avere a
che fare con la gente non è facile. La vita è fatta di queste cose.
Perciò noi praticanti del Dhamma dovremmo impegnarci onestamente nella
pratica, osare davvero praticare. Non credete agli altri, ascoltate solo
gli insegnamenti del Buddha. Rendete la pace il fondamento del vostro
cuore. Col tempo capirete. Praticate, riflettete, contemplate, e i
frutti della pratica arriveranno. Causa ed effetto sono in proporzione.

Non cedete ai vostri stati mentali. All'inizio è difficile anche capire
quanto a lungo sia giusto dormire. Potete decidere di dormire un certo
lasso di tempo, ma poi non ci riuscite. Quale che sia l'ora in cui
decidete di alzarvi, alzatevi appena siete coscienti. A volte riuscite a
farlo, ma altre volte appena vi svegliate il corpo non si muove neanche
se dite a voi stessi: «~Alzati!~». Potreste aver bisogno di aggiungere:
«~Uno, due, e se al tre non mi sono ancora alzato, che possa finire
all'inferno!~» Dovete insegnare a voi stessi in questo modo. Quando
arriverete al tre vi alzerete immediatamente per paura di andare
all'inferno. Dovete addestrare voi stessi, non potete rinunciare ad
addestrarvi. Dovete addestrarvi da ogni punto di vista. Non fate
affidamento sempre e solo sul vostro insegnante, sui vostri amici o sul
gruppo, altrimenti non diventerete mai saggi. Non è necessario ascoltare
molte istruzioni, ascoltate solo una o due volte l'insegnamento e poi
applicatelo. La mente ben addestrata non oserà darvi problemi, neanche
in privato. Nella mente di chi è esperto nell'addestramento non ci sono
cose come ``pubblico'' e ``privato''. Tutti gli Esseri Nobili hanno
fiducia nel loro cuore. È così che dovremmo essere.

Alcuni diventano monaci solo per aver vita facile. Da dove proviene
questo benessere? Quale ne è la causa? È stato necessario che tutto
questo benessere fosse preceduto dalla sofferenza. Così è per tutto.
Prima di ottenere il riso dovete lavorare. Per ogni cosa dovete prima
sperimentare le difficoltà. Alcuni diventano monaci solo per riposare e
prenderla alla leggera, dicono che vogliono solo trastullarsi e riposare
un po'. Se non studiate i libri, pensate di essere in grado di leggere e
scrivere? Non si può. Questa è la ragione per cui molte delle persone
che hanno studiato tanto e che diventano monaci non vanno da nessuna
parte. La loro conoscenza è di genere differente, va per un'altra
strada. Non addestrano se stessi, non osservano la loro mente. La
eccitano e la confondono, alla ricerca di cose che non conducono alla
calma e al contenimento. La conoscenza del Buddha non è una conoscenza
mondana, è una conoscenza sovramondana, una conoscenza del tutto
diversa. Per questo motivo chiunque abbandoni il mondo per il
monachesimo buddhista deve rinunciare a qualsiasi ruolo, status o
posizione raggiunti in precedenza. Perfino un sovrano quando diventa
monaco deve abbandonare il suo status, egli non porta con sé quella roba
mondana nella condizione monastica per ostentare la sua importanza. Non
porta il suo patrimonio, il suo status, la sua conoscenza, il suo potere
nella condizione monastica. La pratica implica rinuncia, lasciar andare,
sradicare, fermarsi. Dovete comprenderlo affinché la pratica funzioni.

Quando si è malati e la malattia non viene curata con i farmaci, pensate
che si curerà da sé? Dovreste andarci tutte le volte che avete paura.
Ovunque ci sia un cimitero o un campo di cremazione particolarmente
spaventoso, andateci. Mettete l'abito, andateci e contemplate,
\emph{Aniccā vata saṅkhāra},\footnote{«~In verità i fenomeni
  condizionati non possono durare.~»} fate lì la meditazione in piedi e
seduta, osservatevi dentro e guardate dov'è la vostra paura. Sarà fin
troppo ovvio. Comprendete la Verità di tutti i fenomeni condizionati.
Restateci e osservate fino a quando arriva il crepuscolo e sempre più la
notte si fa fonda, finché sarete in grado di restarci per tutta la
notte. Il Buddha disse: «~Chiunque veda il Dhamma vede il
\emph{Tathāgata}.\footnote{\emph{Tathāgata}. Letteralmente, ``così
  andato'', ``così venuto''.} Chiunque veda il \emph{Tathāgata} vede il
Nibbāna.~» Come faremo a vedere il Dhamma, se non seguiamo il suo
esempio? Se non vediamo il Dhamma, come faremo a conoscere il Buddha? Se
non vediamo il Buddha, come faremo a conoscere le qualità del Buddha?
Solo se pratichiamo seguendo le orme del Buddha sapremo che quel che il
Buddha insegnò è assolutamente certo, che l'Insegnamento del Buddha è la
Verità Suprema.

