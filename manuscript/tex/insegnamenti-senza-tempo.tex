\chapter{Insegnamenti senza tempo}

Tutti conosciamo la sofferenza, ma non la comprendiamo veramente. Se la
comprendessimo veramente, questo significherebbe la fine di essa. Gli
occidentali vanno sempre di fretta e così la loro felicità e la loro
sofferenza sono più estreme. Il fatto che abbiano molti
\emph{kilesa},\footnote{\emph{Kilesā:} Contaminazione; inquinante
  mentale; fattore mentale che oscura e contamina la mente.} può essere
fonte di saggezza. Per vivere da laici e praticare il Dhamma, si deve
essere nel mondo restando al di sopra di esso. Cominciamo con i Cinque
Precetti\footnote{Cinque Precetti: Le linee guida morali per le azioni e
  i pensieri salutari; per un elenco si veda il \emph{Glossario}, p. \pageref{glossary-precetti}, alla
  voce Precetti.} basilari per far notare che \emph{sīla} (moralità)
è il genitore più importante di tutte le
cose buone. Serve a rimuovere tutto quello che vi è di errato nella
mente, a rimuovere quello che è causa di afflizione e di agitazione.
Quando queste cose non ci saranno più, la mente sarà sempre in uno stato
di \emph{samādhi} (concentrazione).

All'inizio la cosa più importante è rendere \emph{sīla} davvero stabile.
Quando ne avete la possibilità, praticate la meditazione formale. A
volte andrà bene, altre volte no. Non preoccupatevi, continuate e basta.
Se sorgono dubbi, comprendete solo che essi sono impermanenti, come
qualsiasi altra cosa nella mente. Da questa base giungerà il
\emph{samādhi}, ma non ancora la saggezza. Bisogna osservare la mente al
lavoro, vedere piacere e dispiacere quando sorgono dal contatto
sensoriale, e non attaccarsi a essi. Non siate ansiosi di ottenere
risultati o progressi veloci. All'inizio un bimbo va a carponi, poi
impara a camminare, dopo a correre e, quando è del tutto cresciuto, può
fare mezzo giro del mondo per venire fino in Thailandia.
\emph{Dāna}\footnote{\emph{Dāna:} L'atto di donare, liberalità,
  generosità; fare offerte, elemosine.} può recare felicità a se stessi
e agli altri, se si dona con buone intenzioni. Però, finché \emph{sīla}
non è giunta a compimento, l'atto del dare non è puro, poiché potremmo
rubare qualcosa a una persona per darlo a un'altra.

La ricerca di piacere e di divertimenti non ha mai fine, non si è mai
soddisfatti. È come una giara con un foro. Cerchiamo di riempirla, ma
l'acqua fuoriesce continuamente. La pace della vita religiosa ha una
fine certa, pone termine al ciclo della ricerca interminabile. È come
tappare il buco nella giara!

Vivendo nel mondo e praticando la meditazione gli altri vi
considereranno come un gong che, non colpito, non produce alcun suono.
Vi considereranno incapaci, folli, sconfitti, anche se in verità è
l'opposto. Per quanto mi riguarda, non ho mai fatto molte domande agli
insegnanti, sono sempre stato ad ascoltare. Ascoltavo cosa avevano da
dire, non importava se fosse giusto o sbagliato; poi praticavo
solamente. Come voi che praticate qui. Non si dovrebbero avere molte
domande da fare. Se la consapevolezza è costante, allora si possono
esaminare i propri stati mentali, non c'è bisogno che nessun altro
esamini il nostro umore.

Una volta, mentre stavo con un \emph{ajahn},\footnote{\emph{Ajahn} (in
  thailandese,
  \href{http://www.thai2english.com/dictionary/1453955.html}{\thai{อาจารย์}}).
  Il termine deriva da \emph{ācariya}, in pāli, letteralmente
  ``insegnante''; spesso viene utilizzato per un monaco o per una monaca
  con più di dieci anni di vita monastica.} dovetti cucirmi un abito.
Allora non c'erano macchine da cucire, si doveva cucire a mano ed era
un'esperienza molto difficile. La stoffa era veramente spessa e l'ago
non era acuminato. Mi ferivo in continuazione con l'ago, mi facevano
molto male le mani e il sangue gocciolava sempre sulla stoffa. Poiché il
compito era difficile, ero ansioso di portarlo a termine. Ero così
assorto nel lavoro che non mi accorsi nemmeno che sedevo sotto il sole
cocente e grondavo di sudore. L'\emph{ajahn} venne da me e mi chiese
perché sedessi al sole e non al fresco, all'ombra. Gli dissi che ero
molto ansioso di finire il lavoro. Mi chiese: «~Dove hai fretta di
andare?~» «~Voglio finire questo lavoro, così posso fare la meditazione
seduta e quella camminata~», gli risposi. Egli domandò: «~Quando finirà
mai il nostro lavoro?~» «~Oh!~» Tornai a essere consapevole. «~Il nostro
lavoro nel mondo non finisce mai~», mi spiegò. «~Dovresti usare
occasioni di questo genere come esercizi di consapevolezza e, dopo che
hai lavorato abbastanza, fermati e basta. Metti il lavoro da parte e
continua con la tua pratica di meditazione seduta e camminata.~»

Iniziai a comprendere il suo insegnamento. Prima, quando cucivo, anche
la mia mente cuciva e perfino quando mettevo da parte il cucito la mia
mente continuava a cucire. Quando compresi l'insegnamento
dell'\emph{ajahn} fui davvero in grado di accantonare quel che stavo
cucendo. Quando cucivo, la mia mente cuciva, ma quando smettevo di
cucire, anche la mia mente smetteva di cucire. Quando mi fermavo, anche
la mia mente si fermava. Conoscete il bene e il male sia quando
viaggiate sia quando vivete in un posto. Non troverete la pace su una
collina o in una caverna; potete viaggiare fino al luogo in cui il
Buddha raggiunse l'Illuminazione senza essere affatto più vicini
all'Illuminazione. Quel che importa è essere consapevoli di voi stessi,
ovunque siate e qualsiasi cosa stiate facendo. \emph{Viriya}, lo sforzo,
non riguarda quel che fate esteriormente, ma solo costante
consapevolezza e contenimento interiore.

È importante non guardare gli altri per cercare i loro errori. Se si
comportano in modo errato, non è necessario che soffriate. Se voi
indicate ciò che è giusto e loro non praticano di conseguenza, lasciate
le cose come stanno. Quando il Buddha studiò con differenti insegnanti,
comprese che le loro vie erano manchevoli, ma non li denigrò. Studiò con
umiltà e rispetto, praticò onestamente e capì che i loro insegnamenti
non erano perfetti, ma fino a quando non raggiunse l'Illuminazione, non
li criticò né tentò di insegnare loro. Dopo aver raggiunto
l'Illuminazione, incontrò di nuovo coloro con cui aveva studiato e
praticato, e volle condividere la conoscenza da poco scoperta.

Pratichiamo per essere liberi dalla sofferenza, ma essere liberi dalla
sofferenza non significa ottenere tutto quello che ci piacerebbe avere
né che tutti si comportino come vorremmo noi, dicendo solo ciò che ci fa
piacere. Non credete a quello che vi dicono i vostri pensieri. In
genere, la Verità è una cosa, i nostri pensieri un'altra. La nostra
saggezza dovrebbe eccedere il pensiero, allora non c'è problema. Se è il
pensiero a eccedere la saggezza, siamo nei~guai.

Nella pratica, \emph{taṇhā}\footnote{\emph{Taṇhā:} letteralmente
  ``sete''. Bramosia per gli oggetti dei sensi, per l'esistenza o per la
  non esistenza.} può essere un amico o un avversario. All'inizio ci
sprona ad arrivare in monastero e a praticare: vogliamo cambiare le
cose, porre fine alla sofferenza. Se però desideriamo sempre qualcosa
che non abbiamo, se vogliamo che le cose siano diverse da quel che sono,
ciò è solo causa di più sofferenza. Talvolta vogliamo forzare la mente a
essere quieta, e questo sforzo la rende solo ancor più inquieta. Allora
smettiamo di esercitare pressioni, e sorge il \emph{samādhi}. Poi, in
quello stato di calma e serenità, iniziamo a farci domande: «~Che
succede? Che senso ha?~» e così torniamo a essere agitati!

Il giorno precedente il primo \emph{Saṅghayana},\footnote{\emph{Saṅghayana:} Il
  primo concilio del Saṅgha si tenne l'anno successivo alla morte del
  Buddha.} uno dei discepoli del Buddha andò a dire ad ānanda: «~Domani
c'è il primo concilio del Saṅgha; possono parteciparvi solo gli
\emph{arahant}.~»\footnote{\emph{Arahant:} Letteralmente, un
  ``Meritevole''; una persona la cui mente è libera dalle contaminazioni
  (\emph{kilesa}). È anche un titolo del Buddha e il livello più alto
  dei suoi Nobili Discepoli.} Perciò egli si decise: «~Questa notte
diverrò un \emph{arahant}.~» Praticò strenuamente per tutta la notte
cercando di conseguire l'Illuminazione, ma riuscì solo a stancarsi.
Perciò decise di lasciar andare, di riposarsi un po' come se con tutti i
suoi sforzi non volesse ottenere nulla. Dopo aver lasciato andare,
divenne un Illuminato appena si mise a giacere e la sua testa toccò il
cuscino.

Non sono le condizioni esterne a farvi soffrire, la sofferenza sorge
dall'errata comprensione. Le sensazioni piacevoli e dolorose, gradevoli
e sgradevoli, sorgono dal contatto sensoriale che a sua volta causa
nascita mentale e divenire. Per non far sorgere brama e attaccamento
dovete intercettarle appena sorgono, non dovete seguirle. Se sentite la
gente parlare potete agitarvi, pensare che distrugga la vostra calma e
la vostra meditazione, ma se ascoltate il cinguettio di un uccello non
pensate nulla di tutto questo, lo lasciate semplicemente andare come
suono, senza attribuirgli alcun significato o valore.

Non dovreste affrettare o accelerare la vostra pratica, ma pensare in
termini di lungo periodo. Proprio ora si parla di ``nuove'' meditazioni;
però, se ci si accontenta di quella ``vecchia'' si può praticare in ogni
situazione, sia che si recitino i canti monastici, che si lavori o che
si stia seduti nella \emph{kuṭī}.\footnote{\emph{Kuṭī:} La piccola
  dimora del monaco buddhista; una capanna.} Non dobbiamo andare a
cercare posti speciali per praticare. Voler praticare da soli è per metà
giusto, ma per metà è errato. Non è che non veda di buon occhio la
pratica di molta meditazione formale (\emph{samādhi}), ma si dovrebbe
sapere quando uscirne. Sette giorni, due settimane, un mese, due mesi,
per poi tornare di nuovo a relazionarsi con la gente e con le
situazioni. È lì che si guadagna in saggezza. Praticare troppo
\emph{samādhi} ha il solo vantaggio di far diventare matti. Molti
monaci, dopo essersi allontanati per il desiderio di restare isolati
sono solamente morti in solitudine.

Trascurare la normale vita quotidiana e pensare che la pratica formale
sia l'unico modo di praticare con completezza significa essere
intossicati dalla meditazione. Meditazione significa far sorgere la
saggezza nella mente. Possiamo farlo ovunque, in qualsiasi momento e in
ogni postura.

