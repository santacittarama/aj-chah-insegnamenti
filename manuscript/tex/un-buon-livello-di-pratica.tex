\chapter{Un buon livello di pratica}

\begin{openingQuote}
  \centering

  Discorso tenuto nel 1978 al Wat Pah Pong dopo gli esami di Dhamma.
\end{openingQuote}

Oggi ci incontriamo come facciamo sempre dopo gli esami annuali di
Dhamma.\footnote{Molti monaci sostengono esami scritti sulla loro
  conoscenza delle Scritture, un fatto che -- come Ajahn Chah sottolinea
  -- a volte va a detrimento dell'applicazione da parte loro degli
  insegnamenti nella vita quotidiana.} Ora tutti voi dovreste riflettere
su quanto sia importante fare il proprio dovere, assolvendo sia alle
varie incombenze monastiche sia agli impegni nei riguardi del precettore
e degli insegnanti. Sono queste le cose che ci tengono insieme come
gruppo, che ci rendono in grado di vivere in armonia e in concordia.
Inducono anche al rispetto reciproco, ciò che a sua volta reca beneficio
alla comunità. Dai tempi del Buddha fino a oggi, quale che sia la forma
assunta dalle varie comunità, la vita in comune non può riuscire senza
il reciproco rispetto. Che si tratti di comunità secolari o monastiche,
se manca il rispetto reciproco non c'è solidarietà. Se non c'è rispetto
reciproco, subentra la negligenza e alla fine la pratica degenera.
Finora la nostra comunità di praticanti di Dhamma è vissuta qui per
circa venticinque anni in crescita costante, ma potrebbe deteriorarsi. È
una cosa che dobbiamo capire. Però, se tutti siamo attenti, se abbiamo
rispetto reciproco e continuiamo a mantenere i giusti livelli di
pratica, sento che la nostra armonia sarà stabile. La pratica della
nostra comunità sarà fonte di crescita per il buddhismo ancora per molto
tempo.

Per quanto concerne lo studio e la pratica, queste due cose stanno in
coppia. Il buddhismo è cresciuto ed è fiorito fino a oggi perché lo
studio e la pratica si sono tenuti per mano. Se ci limitiamo a imparare
le Scritture in modo distratto, subentra la negligenza. Ad esempio, il
primo anno durante il Ritiro delle Piogge qui avevamo sette monaci.
Allora, tra me e me pensai: «~Tutte le volte che i monaci iniziano a
studiare per gli esami di Dhamma la pratica sembra degenerare.~»
Prendendo in considerazione questo dato di fatto, cercai di capirne la
causa, e così iniziai a insegnare ai monaci che si trovavano con me per
il Ritiro delle Piogge, a tutti e sette. Insegnai per circa quaranta
giorni, da dopo pranzo fino alle sei di pomeriggio, tutti i giorni. I
monaci andarono a sostenere gli esami e il risultato fu buono, tutti e
sette li superarono.

Questo andò bene, ma ci furono complicazioni per chi mancava di
circospezione. Per studiare è necessario recitare e ripetere molto ad
alta voce. Chi era privo di contenimento e di freni tendeva ad allentare
la pratica di meditazione e trascorreva tutto il tempo a studiare,
ripetere e memorizzare. Ciò indusse questi monaci a eliminare le loro
vecchie abitudini, i livelli soliti della loro pratica. Questo avviene
molto spesso. Succedeva che, quando i monaci avevano sostenuto gli
esami, potevo notare cambiamenti nel loro comportamento. Non c'era
meditazione camminata, solamente un po' di meditazione seduta, e una
crescita della socializzazione. C'era meno contenimento e compostezza.

Nella nostra pratica, in realtà, per fare la meditazione camminata si
dovrebbe davvero decidere di farla. Quando ci si siede in meditazione,
ci si dovrebbe concentrare a fare solo quello. Che si stia in piedi, che
si cammini, che si stia seduti o distesi, ci si dovrebbe sforzare a
essere composti. Però, quando la gente studia molto la mente è piena di
parole e se ne va per aria con i libri, e si dimentica di se stessa. Si
perde nelle cose esteriori. Questo avviene solo a chi non ha saggezza, a
chi non ha contenimento e non ha costante \emph{sati}. Per queste
persone lo studio può essere causa di regresso. Quando sono impegnate
nello studio non praticano affatto la meditazione seduta o quella
camminata, e diventano sempre meno contenute. La loro mente si distrae
sempre di più. Chiacchierano senza ragione, mancano di contenimento e la
socializzazione è all'ordine del giorno. Questo induce un declino nella
pratica. Non è a causa dello studio in sé, ma perché alcuni non si
sforzano affatto, si dimenticano di se stessi.

In realtà lungo il Sentiero della pratica le Scritture sono degli
indicatori. Se davvero comprendiamo la pratica, allora leggere e
studiare sono ulteriori aspetti della meditazione. Se però studiamo e ci
dimentichiamo di noi stessi, questo fa sorgere un sacco di parole e di
attività infruttuose. Così, la gente butta via la pratica di meditazione
e vuole subito lasciare l'abito monastico. La maggior parte di coloro
che studiano e vengono respinti agli esami lascia subito l'abito
monastico. Non è che lo studio non vada bene o che la pratica non sia
giusta. Sono le persone che non riescono a esaminare se stesse. Vedendo
tutto questo, durante il secondo Ritiro delle Piogge smisi di insegnare
le Scritture. Molti anni dopo arrivò un numero sempre maggiore di
giovani che volevano diventare monaci. Alcuni non sapevano nulla del
Dhamma-Vinaya\footnote{Dhamma-Vinaya: ``Dottrina e Disciplina'', il nome
  attribuito dal Buddha a ciò che insegnava.} e ignoravano i testi,
perciò decisi di correggere la situazione, e chiesi a chi era monaco da
tempo di insegnare, e sono loro che fino a oggi hanno insegnato. È così
che qui siamo giunti ad avere un'attività di studio.

Ovviamente, ogni anno, quando gli esami sono finiti, chiedo ai monaci di
riattivare la loro pratica, e quindi di riporre negli armadi tutte
quelle Scritture che non riguardano direttamente la pratica. Riattivate
voi stessi, tornate ai regolari livelli di pratica. Riattivate le
pratiche comunitarie, come il ritrovarsi insieme per i canti quotidiani.
Questo è il nostro standard. Fatelo anche solo per resistere alla vostra
pigrizia e alla vostra avversione. Ciò incoraggia la diligenza. Non
mettete da parte le vostre pratiche basilari: mangiare poco, parlare
poco, dormire poco; contenimento e compostezza; distacco; praticare con
regolarità la meditazione camminata e quella seduta; incontrarsi insieme
regolarmente nei tempi stabiliti. Per favore, fate uno sforzo, ognuno di
voi. Non lasciate che vada perduta questa eccellente opportunità.
Svolgete la pratica. Avete questa occasione di praticare qui perché
vivete sotto la guida di un insegnante. Da un certo punto di vista egli
vi protegge, e perciò dovreste tutti dedicarvi alla pratica. Prima
praticavate la meditazione camminata, e anche ora dovreste farlo. Prima
praticavate la meditazione seduta, e anche ora dovreste farlo. In
passato avete recitato insieme i canti del mattino e della sera, e anche
ora dovreste fare uno sforzo. Questi sono i vostri doveri specifici,
impegnatevi per favore.

Sapete, chi prende l'abito monastico e si limita ad ``ammazzare il
tempo'' è privo di forze. C'è chi si agita, chi sente nostalgia di casa,
chi è confuso. Li vedete? Sono così quelli che non applicano la loro
mente nella pratica. Non hanno alcun lavoro da svolgere. Non possiamo
limitarci a oziare. Un monaco buddhista o un novizio vivono e mangiano
bene. Non dovreste darlo per scontato.
\emph{Kāmasukhallikānuyogo}\footnote{Indulgere ai piaceri sensoriali,
  indulgere alla comodità.} è un pericolo. Fate uno sforzo per trovare
la pratica che fa per voi. Lavorate a quel che è errato per correggerlo,
non perdetevi nelle cose esteriori. Chi è zelante non manca mai di fare
la meditazione camminata e quella seduta, non smette mai di sostenere il
contenimento e la compostezza. Osservate i monaci che stanno qui.
Chiunque dopo aver terminato il pasto e le sue incombenze appenda
l'abito e faccia la meditazione camminata è un monaco che non è annoiato
dalla pratica: nei pressi della sua
\emph{kuṭī}\footnote{\emph{Kuṭī}: La piccola dimora
  del monaco buddhista; una capanna.} il sentiero per la meditazione
camminata è usato, battuto, e lo vediamo spesso camminarci sopra. È uno
che si sforza, che è zelante.

Non ci saranno molti problemi se tutti voi vi dedicate alla pratica in
questo modo. Se non dimorate nella pratica, nella meditazione camminata
e in quella seduta, non state facendo altro che andarvene in giro per il
mondo. Se non vi piace qui ve ne andate là, se non vi piace là ve ne
tornate a fare i turisti qua. Questo è tutto quello che state facendo,
andarvene in giro ovunque seguendo il vostro naso. Questa è gente che
non persevera, che non è buona a sufficienza. Non è necessario viaggiare
molto, restate qui a sviluppare la pratica, imparatela in modo
dettagliato. I viaggi possono aspettare, non sono una cosa complicata.
Fate uno sforzo, tutti voi. Prosperità e declino dipendono da questo. Se
davvero volete fare le cose per bene, allora studiate e praticate nella
giusta proporzione, usatele entrambe queste cose, insieme. Sono come il
corpo e la mente. Se la mente è a proprio agio e il corpo è sano e
libero da malattie, la mente diviene composta. Se la mente è confusa,
anche se il corpo è forte, ci saranno difficoltà. Figuriamoci quando è
il corpo a sperimentare disagio.

Lo studio della meditazione è lo studio della coltivazione della mente e
della rinuncia. Ciò che ora intendo con la parola ``studio'' è questo:
tutte le volte che la mente sperimenta una sensazione, ci attacchiamo
ancora a essa? Ci costruiamo problemi attorno? Sperimentiamo felicità o
avversione? Per dirla in modo semplice: ci perdiamo ancora nei nostri
pensieri? Sì, lo facciamo. Se non ci piace qualcosa, reagiamo con
avversione. Se qualcosa ci piace, reagiamo con piacere, e la mente si
contamina e si macchia. Se è questo che succede, allora dobbiamo capire
che facciamo ancora errori, che siamo ancora imperfetti, che abbiamo
ancora del lavoro da svolgere. Ci deve essere più rinuncia e occorre
addestrarsi in modo più tenace. Questo è quel che io intendo per
``studio''. Se restiamo bloccati in qualcosa, riconosciamo che siamo
bloccati. Sappiamo qual è il nostro stato mentale, e lavoriamo per
correggere noi stessi.

Vivere con l'insegnante o vivere lontani da lui dovrebbe essere la
stessa cosa. Alcuni hanno paura. Hanno paura che se non fanno la
meditazione camminata l'insegnante li rimproveri, li sgridi. Da un certo
punto di vista questo va bene, ma nella vera pratica non c'è bisogno di
avere timore degli altri, basta fare attenzione agli errori che sorgono
all'interno delle nostre azioni, delle nostre parole e dei nostri
pensieri. Quando vedete degli errori nelle vostre azioni, nelle vostre
parole e nei vostri pensieri, dovete sorvegliare voi stessi.
\emph{Attano codayattānam}: «~Devi esortare te stesso.~» Non lasciare
che siano gli altri a farlo. Dobbiamo migliorare noi stessi celermente,
conoscere noi stessi. Questo si chiama ``studiare'', coltivare la mente
e rinunciare. Guardateci dentro finché lo vedete con chiarezza. Vivendo
in questo modo facciamo affidamento sulla pazienza, perseveriamo con
fermezza affrontando tutte le contaminazioni. Sebbene tutto ciò vada
bene, siamo ancora al livello di ``praticare il Dhamma senza averlo
visto''. Se abbiamo praticato il Dhamma e lo vediamo, allora abbiamo già
rinunciato a tutto quello che è errato, abbiamo già coltivato tutto
quello che è utile. Vedendo queste cose dentro noi stessi, sperimentiamo
una sensazione di benessere. Quello che dicono gli altri non importa,
conosciamo la nostra mente, restiamo impassibili. Possiamo essere in
pace ovunque.

I monaci più giovani e i novizi che hanno appena iniziato a praticare
potrebbero pensare che non sembra che l'anziano \emph{ajahn} faccia poi
molta meditazione seduta o molta meditazione camminata. In questo non
imitatelo. Dovreste emularlo, non imitarlo. L'emulazione è una cosa,
l'imitazione è un'altra. L'anziano \emph{ajahn} dimora soddisfatto in
una sua particolare condizione. Sebbene esteriormente sembra che non
pratichi, egli pratica interiormente. Tutto quello che è nella sua mente
non può essere visto con gli occhi. La pratica del buddhismo è la
pratica della mente. Benché la pratica non si manifesti nelle sue azioni
o nelle sue parole, la mente è un'altra cosa. Per questa ragione un
insegnante che ha praticato a lungo, che è esperto nella pratica, può
sembrare che lasci andare le sue azioni e le sue parole, ma egli
custodisce la sua mente. È composto. Guardando solo le sue azioni
esteriori potreste cercare di imitarlo, lasciando andare e dicendo tutto
quello che volete dire, ma non è la stessa cosa. Non siete allo stesso
livello. Rifletteteci.

Si tratta di una differenza importante, voi partite da una posizione
diversa. Sebbene possa sembrare che l'\emph{ajahn} se ne stia solo
seduto senza far nulla, egli non è distratto. Vive con le cose, ma le
cose non lo confondono. Tutto questo non possiamo vederlo, perché quel
che è nella sua mente è invisibile. Non limitatevi a giudicare sulla
base delle apparenze, quel che conta è la mente. Quando parliamo, la
mente segue quelle parole. Quali che siano le nostre azioni, la nostra
mente le segue, ma chi ha già praticato può fare o dire cose che la sua
mente non segue, perché essa aderisce al Dhamma e al Vinaya. Ad esempio,
a volte l'\emph{ajahn} può essere severo con i suoi discepoli, le sue
parole possono sembrare dure e incuranti, le sue azioni possono parere
grossolane. Osservando tutto questo, è possibile vedere le sue azioni
corporee e verbali, ma la mente che aderisce al Dhamma e al Vinaya non
la si può vedere. Aderisce alle istruzioni del Buddha: «~Non essere
distratto.~» «~La consapevolezza è la Via che conduce a Ciò che Non
Muore. La distrazione è morte.~» È questo che dovete prendere in
considerazione. Quello che gli altri fanno non è importante, siete voi
che non dovete essere distratti, questa è la cosa importante.

Con queste parole intendo solo dirvi che avete sostenuto gli esami e che
avete l'opportunità di viaggiare e di fare molte cose, ma che dovete
fare attenzione. Che possiate costantemente ricordarvi che siete
praticanti del Dhamma. Un praticante del Dhamma deve essere raccolto,
contenuto e circospetto. Prendiamo in considerazione quell'insegnamento
che dice: «~\emph{Bhikkhu} è colui che chiede l'elemosina.~» Se
accettiamo questa definizione, la nostra pratica assume una forma
davvero rozza. Se intendiamo questa parola nel modo in cui la intese il
Buddha, un \emph{bhikkhu} è uno che vede il pericolo del
\emph{saṃsāra},\footnote{\emph{Saṃsāra}: Flusso del Divenire o
  dell'Esistenza; un vagare perpetuo, il continuo processo del nascere,
  invecchiare e morire.} e allora la definizione è più profonda. Uno che
vede il pericolo del \emph{samsāra}, è uno che vede i difetti, gli
svantaggi di questo mondo. In questo mondo ci sono così tante insidie,
ma la maggior parte della gente non le vede, vede solo il piacere e la
felicità del mondo. Ora, il Buddha afferma che un \emph{bhikkhu} è uno
che vede il pericolo del \emph{saṃsāra}. Che cos'è il \emph{saṃsāra}? La
sofferenza del \emph{saṃsāra} è travolgente, intollerabile. Anche la
felicità è \emph{saṃsāra}. Il Buddha ci insegnò a non attaccarci a essa.
Se non vediamo il pericolo del \emph{saṃsāra}, quando c'è la felicità ci
attacchiamo alla felicità e dimentichiamo la sofferenza. La ignoriamo,
come un bambino che non conosce il fuoco.

Un \emph{bhikkhu} è uno che vede il pericolo del \emph{saṃsāra.} Se
intendiamo la pratica del Dhamma in questo modo, se abbiamo questo
genere di comprensione mentre camminiamo, stiamo seduti o distesi,
ovunque ci troviamo proveremo distacco. Riflettiamo su noi stessi, la
presenza mentale è lì. Anche quando stiamo comodamente seduti, ci
sentiamo in questo modo. Qualsiasi cosa facciamo vediamo questo
pericolo, e per questo ci troviamo in una situazione molto diversa.
Questa pratica è chiamata essere ``uno che vede il pericolo del
\emph{saṃsāra}''. Uno che vede i pericoli del \emph{saṃsāra} vive
all'interno del \emph{saṃsāra} e nello stesso tempo non ci vive.
Comprende i concetti e comprende ciò che li trascende. Qualsiasi cosa
una persona di questo genere dica, non è la stessa cosa di quello che
dice la gente ordinaria. Qualsiasi cosa faccia, non è la stessa cosa.
Qualsiasi cosa pensi, non è la stessa cosa. Il suo comportamento è molto
più saggio. Per questo ho detto: «~Emulate, non imitate.~» Ci sono due
vie: l'emulazione e l'imitazione. Un folle si attaccherà a tutto. Voi
non dovete farlo! Non dimenticatevi di voi stessi.

Per quanto mi riguarda, quest'anno il mio corpo non sta molto bene.
Lascerò alcune cose alle cure di altri monaci e novizi. Forse mi
prenderò un periodo di riposo. Da tempo immemorabile è così, e nel mondo
dei laici è la stessa cosa: fino a quando il padre e la madre sono
ancora in vita, i figli stanno bene e prosperano. Quando i genitori
muoiono, i figli si separano. Dopo essere stati ricchi, diventano
poveri. Di solito va così, pure nella vita dei laici, e lo si può vedere
anche qui. Quando ad esempio l'\emph{ajahn} è ancora in vita, tutti
stanno bene e prosperano. Appena muore, immediatamente si innesca il
declino. Perché è così? Perché mentre l'insegnante è ancora in vita le
persone sono condiscendenti e dimenticano se stesse. Non si sforzano
davvero nello studio e nella pratica. Come nella vita dei laici, quando
la madre e il padre sono ancora vivi, i figli lasciano tutto alle loro
responsabilità. Fanno affidamento sui loro genitori e non sanno come
badare a se stessi. Quando i genitori muoiono diventano poveri. Se
l'\emph{ajahn} se ne va o muore, i monaci tendono a socializzare, si
dividono in gruppi e, quasi sempre, si scivola verso il declino. Perché
è così? Perché dimenticano se stessi. Vivendo dei meriti dell'insegnante
tutto fila liscio. Quando l'insegnante muore, i discepoli tendono a
dividersi. I loro modi di vedere entrano in collisione. Quelli che
pensano in modo errato vivono in un luogo, gli altri che pensano in modo
giusto vivono in un altro. Quelli che non si sentono a proprio agio
lasciano i loro vecchi compagni, fondano nuovi luoghi e iniziano nuovi
lignaggi con i loro discepoli. È così che vanno le cose. Anche
oggigiorno è così. Questo avviene perché siamo manchevoli. Mentre
l'insegnante è ancora in vita siamo manchevoli, viviamo distrattamente.
Non manteniamo i livelli di pratica insegnati dall'\emph{ajahn}, non
facciamo in modo che mettano radici nel nostro cuore. Non seguiamo
davvero le sue orme.

Anche ai tempi del Buddha era così. Ricordate le Scritture? Quel monaco
anziano, com'è che si chiamava \ldots{} ? Subhadda Bhikkhu! Quando il
venerabile Mahā Kassapa era di ritorno da Pāvā, chiese a un asceta che
incontrò lungo il cammino: «~Sta bene il Buddha?~» L'asceta rispose:
«~Il Buddha è entrato sette giorni fa nel
\emph{Parinibbāna}.~»\footnote{\emph{Parinibbāna}: Nibbāna
  completo o definitivo, un termine associato alla morte fisica del
  Buddha.} I monaci che non avevano ancora conseguito l'Illuminazione
erano addolorati, gemevano e piangevano. Quelli che avevano raggiunto il
Dhamma così riflettevano: «~Ah, il Buddha è morto. Ha continuato il suo
viaggio.~» Però, coloro che erano ancora pieni di contaminazioni, come
il venerabile Subhadda, dissero: «~Perché piangete? Il Buddha è morto.
Bene! Ora possiamo vivere con agio. Quando il Buddha era ancora vivo ci
infastidiva in continuazione con qualche regola o con altre cose, non si
poteva fare questo, non si poteva dire quello. Ora che il Buddha è
morto, tutto è perfetto! Possiamo fare quello che vogliamo, possiamo
dire ciò che vogliamo. Perché piangere?~» È stato così da allora fino a
oggi.

Come che sia, è impossibile preservare le cose completamente. Supponiamo
di avere un bicchiere e di fare attenzione che si conservi intatto.
Tutte le volte che lo usiamo lo puliamo e lo riponiamo in un luogo
sicuro. Se facciamo davvero attenzione a quel bicchiere possiamo usarlo
a lungo e, quando non lo usiamo noi, possono essere gli altri a farlo.
Usare dei bicchieri senza averne cura e romperli ogni giorno, oppure
usare un bicchiere per dieci anni prima che si rompa. Allora, cos'è
meglio? Altrettanto vale per la nostra pratica. Ad esempio, se fra tutti
noi che viviamo qui praticando costantemente solo dieci monaci praticano
bene, il Wat Pah Pong prospererà. Proprio come nei villaggi: se in un
villaggio composto di cento case ci sono anche solo cinquanta brave
persone, quel villaggio prospererà. In realtà, sarebbe difficoltoso
anche trovarne dieci di brave persone. Prendiamo un monastero come
questo: è difficile trovare anche solo cinque o sei monaci che si
impegnino realmente, che svolgano davvero la pratica.

Ad ogni modo, qui non abbiamo altra responsabilità che non sia quella di
praticare bene. Pensateci, che cosa abbiamo qui? Non abbiamo più beni,
possessi o famiglie. Per quanto concerne il cibo, mangiamo solo una
volta al giorno. Abbiamo già rinunciato a molte cose, anche a cose
migliori di queste. Come monaci e novizi abbiamo rinunciato a tutto. Non
possediamo nulla. Abbiamo abbandonato tutte quelle cose dalle quali la
gente trae piacere. Si abbandona il mondo come monaci buddhisti per la
pratica. Perché dovremmo allora bramare altre cose, indulgere
all'avidità, all'avversione e all'illusione? Non è più opportuno che i
nostri cuori si occupino di altre cose. Pensateci: perché abbiamo
lasciato il mondo?\footnote{%
  Nel testo inglese si ha ``going forth'', per indicare l'ordinazione
  monastica, con il senso di ``lasciare la propria dimora per diventare
  senza dimora''. Nei testi buddhisti in pāli è con il termine
  \emph{pabbaja} che si indica il passaggio dalla vita laica a quella di
  monaco privo di dimora; è un termine utilizzato nella prima
  ordinazione d'ingresso nel Saṅgha, tramite la quale si diventa novizi
  o \emph{samanera}.}
Perché stiamo praticando? Abbiamo lasciato il mondo per
praticare. Se non pratichiamo, oziamo solo. Se non pratichiamo siamo
peggio dei laici, non svolgiamo alcuna funzione. Non svolgere alcuna
funzione né accettare le nostre responsabilità significa sprecare la
vita da \emph{samaṇa}.\footnote{\emph{Samaṇa}: Un contemplativo.
  Letteralmente, chi abbandona gli obblighi convenzionali della vita
  sociale per un modo di vivere più in sintonia con la natura.} Questo è
in contraddizione con gli scopi di un \emph{samaṇa}.

Se è così siamo distratti. Essere distratti è come essere morti.
Chiedetevelo. Avrete tempo per praticare quando sarete morti? Chiedetevi
in continuazione: «~Quando morirò?~» Se contempliamo in questo modo la
nostra mente starà allerta in ogni momento, saremo sempre consapevoli.
Quando non c'è distrazione, segue immediatamente \emph{sati}, la
rammemorazione di ciò che è. La saggezza sarà limpida, vedrà chiaramente
tutte le cose per quello che sono. La rammemorazione custodisce la
mente, conosce sempre il sorgere delle sensazioni, giorno e notte.
Questo è avere \emph{sati}. Avere \emph{sati} significa essere composti.
Essere composti significa essere attenti. Se si è attenti allora si
pratica rettamente. Si tratta della nostra specifica responsabilità.

Questo è ciò che oggi ho voluto offrire a tutti voi. Se in futuro
lascerete questo monastero per uno dei nostri monasteri affiliati o per
un qualsiasi altro posto, non dimenticatevi di voi stessi. Il dato di
fatto è che non siete ancora perfetti, non siete ancora completi. Ancora
molto è il lavoro che dovete fare, molte sono le responsabilità che
dovete addossarvi: la pratica della coltivazione mentale e della
rinuncia, per la precisione. Preoccupatevi di questo, ognuno di voi. Che
viviate in questo monastero o in uno dei nostri monasteri affiliati,
preservate il buon livello della pratica. Attualmente siamo molti e
tanti sono i monasteri affiliati. Per quanto concerne le loro origini,
tutti i monasteri affiliati sono debitori nei riguardi del Wat Pah Pong.
Potremmo dire che il Wat Pah Pong è il ``genitore'', l'insegnante,
l'esempio per tutti i monasteri affiliati. Per questo soprattutto gli
insegnanti, i monaci e i novizi del Wat Pah Pong dovrebbero essere
d'esempio, la guida di tutti gli altri monasteri affiliati, continuando
a essere diligenti nella pratica e nelle responsabilità di un
\emph{samaṇa}.

