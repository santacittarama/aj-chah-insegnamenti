\chapter{Domande e risposte}

\begin{openingQuote}
  \centering

  Si tratta di appunti presi nel 1972, nel corso di pochi giorni\\
  durante una seduta di domande e risposte con un gruppo di\\
  monaci occidentali.
\end{openingQuote}

\emph{Domanda:} Mi sto impegnando molto nella pratica, ma sembra che non mi
porti da nessuna parte.

\emph{Risposta:} È una cosa davvero importante: nella pratica non cercare di
andare da qualche parte. Proprio il desiderio di essere libero o
illuminato sarà il desiderio che ostacolerà la tua libertà. Puoi
praticare con tutto l'impegno che vuoi, di giorno e di notte con ardore,
ma se nella mente c'è ancora il desiderio di ottenere qualcosa, non
troverai mai la pace. L'energia che proviene da questo desiderio sarà
causa di dubbi e inquietudini. Non importa quanto a lungo e con quanto
impegno pratichi, la saggezza non sorgerà dal desiderio. Perciò, lascia
solo andare. Osserva la mente e il corpo con consapevolezza, ma non
cercare di ottenere qualcosa. Non aggrapparti neanche alla pratica
dell'Illuminazione.

\emph{D.:} E per quanto concerne il dormire? Quanto dovrei dormire?

\emph{R.:} Non chiedermelo, non posso saperlo. Per alcuni va bene una media di
quattro ore per notte. Quello che è importante, è che tu osservi e
conosca te stesso. Se dormi troppo poco il corpo si sentirà a disagio e
sarà difficile mantenere la consapevolezza. Dormire troppo rende la
mente opaca o inquieta. Trova da te il tuo naturale equilibrio. Osserva
accuratamente la mente e il corpo, e sperimenta quante ore hai bisogno
di dormire, fino a quando trovi la quantità ottimale. Se ti svegli e ti
volti per farti un sonnellino, è una contaminazione.

\emph{D.:} E il cibo? Quanto dovrei mangiare?

\emph{R.:} Per il mangiare è come per il dormire. Devi conoscere te stesso. Il
cibo deve soddisfare le necessità del corpo. Considera il cibo come una
medicina. Mangi così tanto da sentirti assonnato dopo i pasti e diventi
ogni giorno più grasso? Fermati! Esamina il tuo corpo e la tua mente.
Non c'è bisogno di digiunare. Fai esperienza con la quantità di cibo che
assumi. Trova il naturale equilibrio per il tuo corpo. Seguendo la
pratica ascetica, metti tutto insieme il cibo nella ciotola per la
questua. Così puoi facilmente valutare quanto ne hai preso. Osserva te
stesso con cura mentre mangi. Conosci te stesso. L'essenza della nostra
pratica sta tutta qui. Non devi fare niente di speciale. Osserva solo.
Esamina te stesso. Osserva la mente. Allora saprai quale è il naturale
equilibrio per la tua pratica.

\emph{D.:} La mente degli asiatici e quella degli occidentali sono diverse?

\emph{R.:} Essenzialmente non ci sono differenze. Consuetudini esteriori e
linguaggio possono apparire diversi, ma la mente degli esseri umani ha
caratteristiche naturali che sono le stesse per tutti. L'avidità e
l'odio sono uguali in una mente orientale e in una occidentale. La
sofferenza e la cessazione della sofferenza sono le stesse per tutti.

\emph{D.:} Leggere molto o studiare le scritture è consigliabile per la
pratica?

\emph{R.:} Il Dhamma del Buddha non si trova nei libri. Se vuoi davvero capire
da te stesso di cosa parlò il Buddha, non c'è bisogno di preoccuparti
dei libri. Osserva la tua mente. Esaminala per vedere come le sensazioni
vanno e vengono, come i pensieri vanno e vengono. Non ti attaccare a
nulla. Sii consapevole di tutto quel che vi è da vedere. Questa è la Via
verso le Verità del Buddha. Sii naturale. Nella tua vita tutto quel che
fai qui è un'occasione per praticare. Tutto è Dhamma. Quando sbrighi le
faccende quotidiane cerca di essere consapevole. Se stai svuotando una
sputacchiera o pulendo un bagno, non pensare che tu stia facendo un
favore a qualcun altro. Nello svuotare le sputacchiere c'è il Dhamma.
Non pensare che tu stia praticando solo quando stai seduto tranquillo a
gambe incrociate. Alcuni di voi si sono lamentati che non c'è abbastanza
tempo per meditare. Di tempo per respirare ce n'è abbastanza? Questa è
la tua meditazione: consapevolezza, naturalezza in qualsiasi cosa tu
faccia.

\emph{D.:} Perché non abbiamo colloqui quotidiani con il maestro?

\emph{R.:} Qualsiasi domanda tu abbia, sei benvenuto. Puoi chiedere tutte le
volte che vuoi. Qui però non abbiamo bisogno di colloqui quotidiani. Se
rispondo a ogni minima domanda, non capirai mai come si sviluppa il
dubbio nella tua mente. È essenziale che tu impari a esaminare te
stesso, a colloquiare con te stesso. Ascolta attentamente per pochi
giorni la lezione, poi paragona questo insegnamento con la tua pratica.
È ancora la stessa? È diversa? Perché hai dubbi? Chi è che dubita? Solo
se esamini te stesso puoi capire.

\emph{D.:} A volte mi preoccupo della disciplina monastica. È male se
accidentalmente uccido un insetto?

\emph{R.:} \emph{Sīla}, o disciplina e moralità, è essenziale per la nostra
pratica, ma non ti devi attaccare ciecamente alle regole. A riguardo
dell'uccidere gli animali o dell'infrangere le altre regole, è
importante l'intenzione. Conosci la tua mente. Non dovresti preoccuparti
eccessivamente della disciplina monastica. Se è usata correttamente,
essa è di supporto alla pratica, ma alcuni monaci sono a tal punto
preoccupati delle piccole regole da non riuscire a dormire bene. La
disciplina non è un fardello. Il fondamento della nostra pratica qui è
la disciplina, la buona disciplina, più le regole e le pratiche
ascetiche. È di grande beneficio essere consapevoli e attenti sia in
relazione alle numerose regole supplementari come pure ai 227 basilari
precetti. Rende la vita molto semplice. Non c'è bisogno di chiedersi
come comportarsi, e così si può evitare di pensare ed essere, invece,
semplicemente consapevoli. La disciplina ci consente di vivere insieme
in armonia; la vita comunitaria scorre dolcemente. Esteriormente tutti
sembrano essere uguali e agire allo stesso modo. La disciplina e la
moralità sono il primo passo per una maggiore concentrazione e una
maggiore saggezza. Usando bene la disciplina monastica e i precetti
ascetici siamo obbligati a vivere con semplicità, a limitare i nostri
possessi. Così abbiamo la pratica completa del Buddha: astenersi dal
male e fare il bene, vivere solo attenendosi alle necessità basilari,
purificare la mente. Ossia, vigilare la mente e il corpo in tutte le
posture: conosci te stesso quando sei seduto, in piedi, camminando o
giacendo.

\emph{D.:} Che posso fare con i dubbi? Certi giorni sono afflitto da dubbi
sulla pratica, sui progressi o sul maestro.

\emph{R.:} Dubitare è naturale. Tutti cominciamo con i dubbi. Puoi imparare
molto da essi. Quel che è importante è non identificarsi con i dubbi.
Non farti coinvolgere. Altrimenti questo farà girare la tua mente in
tondo, senza fine. Osserva invece l'intero processo del dubitare, del
domandarsi. Vedi chi è che dubita. Vedi come i dubbi vanno e vengono.
Allora non sarai più vittima dei tuoi dubbi. Ne uscirai, e la tua mente
sarà tranquilla. Puoi vedere come tutte le cose vanno e vengono. Lascia
solo andare ciò a cui sei attaccato. Lascia andare i tuoi dubbi e,
semplicemente, osserva. Così si pone termine ai dubbi.

\emph{D.:} E a proposito degli altri metodi di praticare? Oggigiorno sembra che
gli insegnanti siano così tanti e i sistemi di meditazione così diversi
da sentirsi confusi.

\emph{R.:} È come andare in città. Puoi raggiungerla da nord, da sud-est, da
molte strade. Spesso i sistemi differiscono solo esteriormente. Se tu
cammini su una strada o su un'altra, veloce o lento che tu vada, se sei
consapevole tutto è uguale. C'è un solo punto essenziale al quale deve
alla fine pervenire ogni buona pratica, il non attaccamento. Alla fine,
tutti i sistemi di meditazione devono essere lasciati andare. Non ci si
può attaccare nemmeno al maestro. Se un sistema conduce alla rinuncia,
al non attaccamento, allora si tratta di pratica corretta.

Puoi desiderare di viaggiare per incontrare altri insegnanti e per
provare altri sistemi. Alcuni di voi l'hanno già fatto. È un desiderio
naturale. Scoprirete che fare migliaia di domande e conoscere diversi
sistemi non vi condurrà alla Verità. Alla fine vi sentirete annoiati.
Capirete che solo fermandovi ed esaminando la vostra stessa mente
scoprirete di cosa parlò il Buddha. Non c'è bisogno di andare a cercare
al di fuori di voi stessi. Alla fine dovrete tornare ad affrontare la
vostra vera natura. È qui che potete capire il Dhamma.

\emph{D.:} Spesso sembra che molti monaci non stiano praticando. Sembrano
sciatti e poco consapevoli. Ciò mi infastidisce.

\emph{R.:} Non è opportuno guardare gli altri. Questo non aiuterà la tua
pratica. Se ti senti infastidito, osserva il fastidio nella tua mente.
Se la disciplina altrui è scadente o loro non sono buoni monaci, non
spetta a te giudicarlo. Non troverai la saggezza guardando gli altri. La
disciplina monastica è uno strumento che devi usare per la tua
meditazione. Non è un'arma da usare per criticare gli altri o per
individuare difetti. Nessuno può svolgere la pratica per te, né tu puoi
praticare per qualcun altro. Sii consapevole di quel che fai tu. Questo
è il modo corretto di praticare.

\emph{D.:} Applico molta attenzione nel praticare il contenimento dei sensi.
Tengo sempre lo sguardo basso e sono consapevole di ogni minima azione.
Quando mangio, ad esempio, ci metto molto tempo e cerco di osservare
ogni contatto: masticare, assaporare, deglutire, e così via. Ogni passo
lo compio intenzionalmente e con attenzione. Sto praticando in modo
giusto?

\emph{R.:} Avere un senso di contenimento è una giusta pratica. Dovremmo essere
consapevoli del contenimento per tutto il giorno. Ma non esagerare!
Cammina, mangia e agisci con naturalezza. Sviluppa poi la naturale
consapevolezza in relazione a che cosa sta succedendo dentro di te. Non
forzare la meditazione e non costringerti all'interno di scomodi schemi.
È un'altra forma di brama. Sii paziente. Pazienza e sopportazione sono
necessarie. Se agisci con naturalezza e sei consapevole, anche la
saggezza arriverà con naturalezza.

\emph{D.:} È necessario stare seduti in meditazione per intervalli di tempo
molto lunghi?

\emph{R.:} No, alla fine stare seduti per ore non è necessario. Alcuni pensano
che più si sta seduti più si diventa saggi. Ho visto galline sedute nei
loro nidi per giorni! La saggezza giunge quando si è consapevoli in
tutte le posture. La tua pratica dovrebbe iniziare fin dal mattino,
quando ti svegli. E dovrebbe continuare fino a quando ti addormenti. Non
preoccuparti di quanto tempo riesci a stare seduto. Quel che importa è
che tu sia vigile quando lavori o siedi o vai in bagno.

Ognuno ha i suoi ritmi naturali. Alcuni muoiono a cinquant'anni, altri a
sessanta e altri ancora a novanta. Perciò, anche la pratica non è
affatto uguale per tutti. Non pensarci e non preoccupartene. Cerca di
essere consapevole e lascia che le cose seguano il loro corso naturale.
La tua mente diverrà sempre più quieta, in tutte le circostanze.
Diventerà immobile come una pozza di limpida acqua nella foresta.
Verranno ad abbeverarsi animali di ogni genere, meravigliosi e rari.
Vedrai la natura di tutte le cose (\emph{saṅkhāra})\footnote{\emph{Saṅkhāra.}
  Formazione, fenomeno condizionato.} del mondo con chiarezza. Vedrai
sorgere e svanire molte cose meravigliose e strane. Ma tu sarai
tranquillo. Quando nasceranno dei problemi, saprai immediatamente come
superarli. Questa è la felicità del Buddha.

\emph{D.:} Ho ancora molti pensieri. La mia mente vaga molto benché stia
cercando di essere consapevole.

\emph{R.:} Non preoccupartene. Cerca di mantenere la mente nel momento
presente. Osserva solo qualsiasi cosa sorga nella mente. E lasciala
andare. Non desiderare nemmeno di sbarazzarti dei pensieri. Allora la
mente raggiungerà il suo stato naturale. Non discriminare bene e male,
caldo e freddo, veloce e lento. Né io né tu, nessun sé. Solo quello che
c'è. Quando cammini per la questua non c'è bisogno di fare nulla di
particolare. Cammina e guarda quel che c'è, semplicemente. Non c'è
bisogno di attaccarsi all'isolamento e alla solitudine. Dovunque tu sia,
conosci te stesso mediante l'osservazione con naturalezza. Se sorgono
dei dubbi, osservali andare e venire. È molto semplice. Non aggrapparti
a nulla.

È come se tu stessi camminando lungo una strada. Periodicamente ti
imbatti in ostacoli. Quando incontri delle contaminazioni, osservale
solo e superale unicamente lasciandole andare. Non pensare agli ostacoli
che hai già superato. Non ti preoccupare di quelli che non hai ancora
visto. Resta incollato al presente. Non interessarti alla lunghezza del
cammino o alla destinazione. Tutto cambia. Qualsiasi cosa attraversi,
non attaccarti a essa. Alla fine la mente raggiungerà automaticamente il
suo naturale equilibrio nella pratica. Ogni cosa andrà e verrà da sé.

\emph{D.:} Hai mai dato un'occhiata al \emph{Sutra dell'altare} di Hui Neng, il
Sesto Patriarca?

\emph{R.:} La saggezza di Hui Neng è molto penetrante. È un insegnamento molto
profondo, per i principianti non è facile da capire. Però, se pratichi
pazientemente con la nostra disciplina, se pratichi il non attaccamento,
alla fine capirai. Una volta un mio discepolo stava in una capanna con
il tetto impagliato. Durante quella Stagione delle Piogge\footnote{La
  Stagione delle Piogge coincide con l'annuale periodo di tempo di tre
  mesi, che in India corrisponde a quello dei primi tre mesi monsonici,
  durante i quali i monaci hanno la regola dell'obbligo di residenza in
  monastero, un periodo che tradizionalmente è dedicato a una formazione
  più intensiva.} spesso pioveva e, un giorno, un forte vento fece volar
via metà del tetto. Non si preoccupò di aggiustarlo, semplicemente
lasciò che piovesse dentro la capanna. Dopo molti giorni gli chiesi
della sua capanna. Disse che stava praticando il non attaccamento.
Questo è non attaccamento privo di saggezza. È più o meno come
l'equanimità di un bufalo d'acqua. Se vivi un'esistenza semplice e
buona, se sei paziente e altruista, capirai la saggezza di Hui Neng.

\emph{D.:} Hai detto che \emph{samatha} e \emph{vipassanā}, o concentrazione e
visione profonda, sono la stessa cosa. Potresti spiegarmelo meglio?

\emph{R.:} È piuttosto semplice. La concentrazione (\emph{samatha}) e la
saggezza (\emph{vipassanā}) lavorano insieme. Prima la mente diviene
tranquilla attenendosi a un oggetto di meditazione. È quieta solo mentre
si sta seduti a occhi chiusi. Questo è \emph{samatha}, e alla fine tale
fondamento del \emph{samādhi} è la causa del sorgere della saggezza o
della \emph{vipassanā}. Allora la mente è quieta sia che si stia seduti
a occhi chiusi sia che si cammini nel caos cittadino. È così. Prima eri
un bambino. Adesso sei un adulto. Il bambino e l'adulto sono la stessa
persona? Puoi rispondere di sì, oppure, da un altro punto di vista, puoi
dire che sono persone diverse. Così, anche \emph{samatha} e
\emph{vipassanā} potrebbero essere considerate separate. Oppure, come il
cibo e le feci. Si potrebbe dire che il cibo e le feci sono la stessa
cosa o che sono cose diverse. Non credere a quel che ti dico, pratica e
vedi da te. Non c'è bisogno di niente di speciale. Se esaminerai come
sorgono la concentrazione e la saggezza, conoscerai la Verità da te.
Oggigiorno la gente si attacca alle parole. Chiamano la loro pratica
\emph{vipassanā}. \emph{Samatha} è guardata dall'alto in basso. Oppure
la loro pratica la chiamano \emph{samatha}. Dicono che è essenziale fare
\emph{samatha} prima di \emph{vipassanā}. Tutto questo è sciocco. Non
cercare di pensare in questo modo. Pratica, e capirai da te.

\emph{D.:} Nella nostra pratica è necessario essere in grado di entrare in uno
stato di assorbimento meditativo?

\emph{R.:} No, l'assorbimento meditativo non è necessario. Devi instaurare un
minimo di tranquillità e unificare la mente. Poi, usa tutto questo per
esaminare te stesso. Non c'è bisogno di niente di speciale. Se nella tua
pratica giunge l'assorbimento meditativo, anche questo va bene. Non ti
ci attaccare, però. Alcuni restano aggrappati all'assorbimento. Può
essere molto divertente giocarci. Devi conoscere il giusto limite. Se
sei saggio, conoscerai gli impieghi e i limiti dell'assorbimento,
proprio come conosci i limiti dei bambini rispetto agli adulti.

\emph{D.:} Perché seguiamo pratiche ascetiche, come mangiare solo dalle nostre
ciotole?

\emph{R.:} I precetti ascetici servono ad aiutarci a eliminare le
contaminazioni. Seguendo uno di essi, come mangiare dalla ciotola per la
questua, possiamo essere più consapevoli del fatto che il nostro cibo è
una medicina. Se non abbiamo contaminazioni, non importa come mangiamo.
Però, qui usiamo una forma per semplificare la nostra pratica. Il Buddha
non ritenne che i precetti ascetici fossero necessari per tutti i
monaci, ma consentì che li seguissero coloro che desideravano praticare
rigorosamente. Si aggiungono alla nostra disciplina esteriore e aiutano
ad accrescere la forza e la saldezza mentale. Queste regole devono
essere osservate per te stesso. Non guardare come praticano gli altri.
Osserva la tua mente e vedi ciò che per te è salutare. La regola di
dover accettare qualsiasi capanna di meditazione ci venga assegnata è
una norma disciplinare altrettanto utile. Evita che i monaci si
attacchino al luogo in cui dimorano. Se vanno via e poi tornano, devono
prenderne una diversa. Questa è la nostra pratica, non attaccarsi a
nulla.

\emph{D.:} Se è importante mettere il cibo tutto insieme nella nostra ciotola,
perché tu, che sei l'insegnante, non lo fai?

\emph{R.:} Si, è vero, un insegnante dovrebbe essere d'esempio per i suoi
discepoli. Non mi dispiace che tu mi stia criticando. Chiedi tutto quel
che vuoi. Però, è importante che non ci si attacchi all'insegnante. Se
io fossi assolutamente perfetto nel mio comportamento esteriore, sarebbe
terribile. Tutti voi mi sareste troppo attaccati. Perfino il Buddha
talvolta disse ai suoi discepoli di fare una cosa e poi lui stesso ne
fece un'altra. I dubbi al riguardo del vostro insegnante possono
aiutarvi. Dovreste osservare le vostre reazioni. Ritieni che io possa
conservare un po' di cibo nei piatti, fuori dalla mia ciotola, per
nutrire i laici che lavorano attorno al monastero?

Tocca a te osservare e sviluppare la saggezza. Prendi quel che vi è di
buono nell'insegnante. Sii consapevole della tua stessa pratica. Provi
rabbia se io riposo mentre voi tutti dovete restare seduti? Se io dico
che il blu è rosso o che maschio è femmina, non seguirmi ciecamente.

Uno dei miei insegnanti mangiava molto velocemente. Quando mangiava
faceva dei rumori. Però, ci aveva detto di mangiare lentamente e con
consapevolezza. Ero solito osservarlo e arrabbiarmi davvero. Io
soffrivo, ma lui no! Osservavo l'esteriorità. Dopo ho imparato. Alcuni
guidano velocemente, ma con molta attenzione. Altri guidano lentamente e
hanno molti incidenti. Non attaccarti alle regole, alla forma esteriore.
Osserva gli altri tutt'al più per un dieci per cento, e te stesso per il
novanta per cento: questa è retta pratica. All'inizio ero solito
osservare il mio insegnante, Ajahn Tongrat, e avevo molti dubbi. La
gente pensava perfino che fosse matto. Faceva cose strane ed era molto
aspro con i suoi discepoli. Esteriormente era arrabbiato, ma dentro non
c'era nulla. Non c'era nessuno lì dentro. Era eccezionale. Restò limpido
e consapevole fino al momento della morte.

Guardare il sé esteriore significa paragonare, discriminare. In questo
modo non troverai la felicità. Non troverai la pace neanche se passi il
tuo tempo cercando un uomo o un maestro perfetto. Il Buddha ci insegnò a
guardare il Dhamma, la Verità, non a guardare gli altri.

\emph{D.:} Come possiamo vincere la lussuria? A volte penso di essere schiavo
del mio desiderio sessuale.

\emph{R.:} La lussuria deve essere bilanciata mediante la repulsione.
L'attaccamento alla forma del corpo è un estremo e si dovrebbe tenere a
mente l'opposto. Esamina il corpo come un cadavere e vedine il processo
di decomposizione, oppure pensa alle parti del corpo, come i polmoni, la
milza, il grasso, le feci e così via. Ricorda queste cose e, quando
sorge la lussuria, visualizza questi aspetti repellenti del corpo. Te ne
libererà.

\emph{D.:} E la collera? Cosa dovrei fare quando sento che sta sorgendo la
collera?

\emph{R.:} Devi usare la gentilezza amorevole. Quando stati mentali collerici
sorgono durante la meditazione, equilibrali sviluppando sentimenti di
gentilezza amorevole. Se qualcuno fa qualcosa di male o si arrabbia, non
arrabbiarti anche tu. Se lo fai, sei più ignorante di lui. Sii saggio.
Tieni a mente la compassione, perché quella persona sta soffrendo. Colma
la tua mente di gentilezza amorevole come se si trattasse di un caro
fratello. Concentrati sul sentimento della gentilezza amorevole come
oggetto di meditazione. Effondilo su tutti gli esseri del mondo. È
possibile sconfiggere l'odio solo mediante la gentilezza amorevole.

A volte potresti vedere dei monaci che si comportano male. Potresti
irritarti. Questa sofferenza non è necessaria. Non è il nostro Dhamma.
Potresti pensare: «~Non sono rigorosi come me. Non sono meditanti seri
come noi. Quei monaci non sono buoni monaci.~» Sarebbe una tua grande
contaminazione. Non fare paragoni. Non discriminare. Lascia andare le
tue opinioni, osservale e osserva te stesso. Questo è il nostro Dhamma.
Non puoi fare in modo che tutti agiscano come tu desideri o che siano
come te. Questo desiderio ti farà solo soffrire. Per i meditanti si
tratta di un errore comune, ma osservare gli altri non sviluppa la
saggezza. Esamina solo te stesso, le tue sensazioni. È così che
comprenderai.

\emph{D.:} Mi sento molto assonnato. Ciò rende difficile la meditazione.

\emph{R.:} Ci sono molti modi per vincere il sonno. Se stai sedendo al buio,
spostati in un posto illuminato. Apri gli occhi. Alzati e sciacquati il
viso o fatti una doccia. Se sei assonnato, cambia postura. Cammina
molto. Cammina all'indietro. Il timore di andare a sbattere contro
qualcosa ti terrà sveglio. Siediti in prossimità di un dirupo o
sull'orlo di un pozzo. Non oserai dormire! Se non c'è niente che
funzioni, allora vai a dormire. Mettiti disteso con accuratezza e cerca
di essere consapevole fino a quando ti addormenti. Appena ti svegli,
alzati subito. Non guardare che ora è, non girarti. Inizia con la
consapevolezza dal momento in cui ti svegli. Se ti senti assonnato tutti
i giorni, cerca di mangiare meno. Esamina te stesso. Appena ti accorgi
che dopo altri cinque bocconi sarai sazio, fermati. E bevi fino a
sentirti lievemente sazio, in modo corretto. Va a sederti. Osserva la
tua sonnolenza e la tua fame. Devi imparare a calibrare il cibo. Man
mano che la pratica andrà avanti, avrai naturalmente più energia e
mangerai meno. Però devi regolare te stesso.

\emph{D.:} Perché qui ci si prostra così tanto?

\emph{R.:} Prostrarsi è molto importante. È una forma esteriore che fa parte
della pratica. Questa forma dovrebbe essere svolta correttamente. Porta
la fronte del tutto a contatto con il pavimento. I gomiti devono toccare
le ginocchia e le palme delle mani devono stare sul pavimento a meno di
dieci centimetri di distanza. Prostrati lentamente, sii consapevole del
tuo corpo. È un buon rimedio per la nostra presunzione. Dovremmo
prostrarci spesso. Quando ti prostri tre volte puoi pensare alle qualità
del Buddha, del Dhamma e del Saṅgha, ossia alle qualità della mente
pura, radiosa e serena. In questo modo utilizziamo la forma esteriore
per addestrare noi stessi. Corpo e mente diventano armoniosi. Non fare
l'errore di guardare come si prostrano gli altri. Se i giovani novizi
sono sciatti o i monaci anziani non sembrano consapevoli, non sta a te
giudicarlo. La gente può essere difficile da addestrare. Alcuni imparano
in fretta e altri lentamente. Giudicare gli altri farà solo aumentare il
tuo orgoglio. Osserva te stesso, invece. Prostrati spesso, vinci il tuo
orgoglio.

Chi è davvero entrato in armonia con il Dhamma va ben al di là della
forma esteriore. Tutto quello che queste persone fanno è un modo di
prostrarsi. Camminando si prostrano; mangiando si prostrano; defecando
si prostrano. È perché sono andati al di là dell'egoismo.

\emph{D.:} Qual è il problema più grande dei tuoi nuovi discepoli?

\emph{R.:} Le opinioni. Punti di vista e idee a proposito di ogni cosa: di se
stessi, della pratica, degli insegnamenti del Buddha. Molti di quelli
che vengono qui hanno un'alta posizione sociale. Sono ricchi
commercianti o laureati, insegnanti o funzionari governativi. La loro
mente è piena di opinioni sulle cose. Sono troppo intelligenti per
ascoltare gli altri. È come l'acqua in una tazza. Una tazza è inutile se
è colma di acqua sporca e stagnante. Diventa utile solo dopo che l'acqua
vecchia viene gettata via. Devi svuotare la tua mente dalle opinioni,
allora capirai. La nostra pratica va al di là dell'intelligenza e al di
là della stupidità. Se pensi «~io sono intelligente, io sono ricco, io
sono importante, io capisco tutto del buddhismo~» nascondi la verità
dell'\emph{anattā} o del non-sé. Tutto ciò che vedrai è il sé, l'io, il
mio. Invece il buddhismo è lasciar andare il sé. La vacuità, il vuoto,
il Nibbāna.

\emph{D.:} Contaminazioni come l'avidità e la collera sono reali o meramente
illusorie?

\emph{R.:} Entrambe le cose. Le contaminazioni le chiamiamo lussuria e avidità
oppure collera e illusione, ma queste sono solo denominazioni esteriori,
apparenze. Proprio come quando parliamo di una ciotola grande, piccola,
bella o come che sia. Non è la realtà. È un concetto che nasce dalla
brama. Se vogliamo una ciotola grande, diciamo che questa è piccola. È
la brama che ci induce a discriminare. La verità, invece, è solo quel
che è. Vedila in questo modo. Sei un uomo? Puoi rispondere di sì. È solo
un'apparenza. In realtà tu sei solo una combinazione di elementi o un
gruppo di mutevoli aggregati. Se la mente è libera, non discrimina. Né
grande né piccolo, né tu né io. Non c'è niente. \emph{Anattā} diciamo
noi, o non-sé. Davvero, alla fine non c'è né \emph{attā} né
\emph{anattā}.

\emph{D.:} Potresti dirmi qualcosa in più sul kamma?

\emph{R.:} Kamma è azione. Kamma è attaccamento. Corpo, parola e
mente producono kamma quando ci attacchiamo. Creiamo abitudini
che, in futuro, possono farci soffrire. Questo è il frutto del nostro
attaccamento, delle nostre passate contaminazioni. Ogni attaccamento
comporta la produzione di kamma. Supponi che, prima di diventare
monaco, tu fossi un ladro. Hai rubato, hai reso altre persone infelici,
hai reso infelici i tuoi genitori. Ora sei un monaco, ma quando ricordi
di aver reso infelici gli altri, ti senti male e soffri anche oggi.
Ricorda, non solo azioni del corpo, ma anche quelle della parola e della
mente possono produrre condizioni per effetti futuri. Se hai compiuto
qualche azione gentile in passato e oggi la rammenti, sei felice. Questo
stato mentale di felicità è il risultato del kamma trascorso.
Tutte le cose sono prodotte da cause, sia a lungo termine sia, se le
esaminiamo, istante dopo istante. Non devi però preoccuparti di pensare
al passato, al presente o al futuro. Osserva solo il corpo e la mente.
Devi capire il kamma da te. Osserva la tua mente. Pratica e
vedrai con chiarezza. Accertati, ovviamente, di lasciare agli altri il
loro kamma. Non attaccarti agli altri e non starli a guardare. Se
prendo del veleno, soffro. Non c'è bisogno che tu lo condivida! Prendi
quel che di buono il tuo insegnante ti offre. Allora potrai essere
sereno, la tua mente diverrà come quella del tuo insegnante. Se esamini
questa cosa, la capirai. Anche se ora non capisci, con la pratica ti
diverrà chiara. Conoscerai da te. Questo è praticare il Dhamma.

Quando eravamo giovani, i nostri genitori ci punivano e si arrabbiavano.
Volevano davvero aiutarci. Devi considerare la cosa a lungo termine.
Genitori e insegnanti ci criticano e noi ci arrabbiamo. In seguito
capiremo perché. Dopo molta pratica capirai. Devi sbarazzarti della tua
intelligenza. Se pensi di essere meglio degli altri, soffrirai e basta.
Che peccato! Non c'è bisogno di arrabbiarsi. Limitati a osservare.

\emph{D.:} A volte sembra che da quando sono diventato monaco il mio disagio e
la mia sofferenza siano aumentate.

\emph{R.:} So che alcuni di voi hanno alle loro spalle benessere materiale e
libertà esteriore. Al confronto ora la tua vita è austera. Per la
pratica, spesso vi faccio stare seduti ad aspettare per lunghe ore. Il
cibo e il clima sono diversi da quelli di casa vostra. Tutti devono
passare attraverso cose di questo genere. Questa è la sofferenza che
conduce alla fine della sofferenza. È così che impari. Quando ti arrabbi
e ti dispiace per te stesso, questa è una grande occasione per capire la
mente. Il Buddha disse che le contaminazioni sono i nostri insegnanti.

Tutti i miei discepoli sono come figli. Nella mia mente ho solo il loro
benessere e gentilezza amorevole. Se sembra che vi faccia soffrire, è
per il vostro bene. So che alcuni di voi sono colti e competenti.
Persone con poca istruzione e poca cultura mondana possono praticare con
facilità. Però, è come se voi occidentali aveste una casa molto grande
da pulire. Quando la casa sarà pulita, potrete vivere in uno spazio
molto grande. Potrete usare la cucina, la biblioteca, il soggiorno. Devi
avere pazienza. Pazienza e sopportazione sono essenziali per la nostra
pratica. Per me, quando ero un giovane monaco, non è stata dura come per
te. Parlavo la lingua del mio paese e mangiavo il cibo del mio paese.
Alcuni giorni ero egualmente disperato. Volevo lasciare l'abito
monastico o perfino suicidarmi. Questo genere di sofferenza proviene da
punti di vista errati. Quando hai visto la Verità, però, sei libero da
punti di vista e opinioni. Tutto diventa tranquillo.

\emph{D.:} Ho sviluppato stati mentali davvero sereni grazie alla meditazione.
Ora che cosa dovrei fare?

\emph{R.:} È una buona cosa rendere serena la mente, concentrata. Utilizza
questa concentrazione per esaminare la mente e il corpo. Dovresti
osservare anche quando la mente non è serena. Allora conoscerai la vera
pace. Perché? Perché vedrai l'impermanenza. Perfino la pace deve essere
vista come impermanente. Se ti attacchi a stati mentali sereni, quando
non li avrai soffrirai. Rinuncia a tutto, anche alla pace.

\emph{D.:} Hai detto che hai paura per i discepoli molto diligenti: ho sentito
bene?

\emph{R.:} Si, è vero, ho paura. Temo che siano troppo seri. Si sforzano
troppo, ma senza saggezza. Si impongono sofferenze non necessarie.
Alcuni di voi sono determinati a diventare Illuminati. Serrate i denti e
lottate sempre. Questo è sforzarsi troppo. La gente è tutta uguale. Non
conosce la natura delle cose (\emph{saṅkhāra}). Tutte le formazioni,
mente e corpo, sono impermanenti. Osservare semplicemente, senza
attaccarsi.

Gli altri pensano di sapere. Criticano, guardano, giudicano. Va bene.
Lasciali alle loro opinioni. Discriminare è pericoloso. È come una
strada con una curva molto stretta. Se pensiamo che gli altri siano
peggiori, migliori o uguali a noi, andiamo fuori strada. Se
discriminiamo, soffriremo solo.

\emph{D.:} È già da molti anni che faccio meditazione. La mia mente è aperta e
serena in quasi tutte le circostanze. Adesso mi piacerebbe far marcia
indietro e praticare alti stati di concentrazione o assorbimento
mentale.

\emph{R.:} Ottimo. È un esercizio mentale benefico. Se hai saggezza, non ti
fisserai sugli stati di concentrazione mentale. È come voler sedere in
meditazione per lungo tempo. È un ottimo esercizio, ma in realtà la
pratica è separata da qualsiasi postura. Si tratta di osservare la mente
in modo diretto. Questa è saggezza. Se hai esaminato e compreso la
mente, allora hai la saggezza per conoscere i limiti della
concentrazione, o dei libri. Se hai praticato e compreso il
non-attaccamento, allora puoi tornare ai libri. Saranno un dolce
squisito. Possono aiutarti a insegnare agli altri. Oppure puoi tornare a
praticare l'assorbimento mentale. Hai la saggezza per sapere che non ti
devi attaccare a nulla.

\emph{D.:} Potresti passare in rassegna i punti più importanti della nostra
discussione?

\emph{R.:} Devi esaminare te stesso. Sapere chi sei. Conoscere il tuo corpo e
la tua mente osservando e basta. Quando siedi, dormi, mangi, conosci i
tuoi limiti. Usa la saggezza. La pratica non è cercare di ottenere
qualcosa. Sii solo consapevole di quel che è. La nostra meditazione
completa è guardare la mente in modo diretto. Vedrai la sofferenza, la
causa di essa e la sua fine. Devi avere pazienza, però; molta pazienza e
sopportazione. Imparerai gradualmente. Il Buddha insegnò ai suoi
discepoli di stare con i loro insegnanti almeno cinque anni. Devi
imparare i valori del donare, della pazienza e della devozione.

Non praticare in modo troppo severo. Non farti catturare dalla forma
esteriore. Osservare gli altri è cattiva pratica. Semplicemente, sii
naturale e osserva. La nostra disciplina monastica e le regole
monastiche sono molto importanti. Creano un ambiente semplice e
armonioso. Usale bene. Ricorda, però. L'essenza della disciplina
monastica è osservare l'intenzione, esaminare la mente. Devi avere
saggezza. Non discriminare. Ti arrabbieresti con un alberello della
foresta perché non è grande e dritto come alcuni altri alberi? È
sciocco. Non giudicare gli altri. Ce n'è di tutti i tipi. Non c'è
bisogno di accollarsi il desiderio di cambiarli tutti.

Sii paziente. Pratica la moralità. Vivi con semplicità e naturalezza.
Osserva la mente. Questa è la nostra pratica. Ti porterà alla
generosità, alla pace.

