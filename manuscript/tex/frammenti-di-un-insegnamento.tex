\chapter{Frammenti di un insegnamento}

\begin{openingQuote}
  \centering

  Discorso tenuto per la comunità laica del Wat Pah Pong nel 1972.
\end{openingQuote}

Siccome avete ascoltato insegnamenti buddhisti da molte fonti,
soprattutto da numerosi monaci e insegnanti, tutti voi avete da molti
anni fiducia nel buddhismo. In alcuni casi, però, il Dhamma è insegnato
in termini così generali e vaghi che è difficile sapere come metterlo in
pratica nella vita quotidiana. Altre volte è insegnato con un linguaggio
talmente erudito o specialistico che la maggior parte della gente lo
trova di difficile comprensione, soprattutto se l'insegnamento è tratto
dalle Scritture in modo troppo letterale. Può infine essere insegnato
con equilibrio, in modo né troppo vago né troppo profondo, né troppo
generico né troppo esoterico, così da consentire a chi ascolta di capire
e praticare gli insegnamenti a proprio beneficio. Oggi vorrei
condividere con voi degli insegnamenti che ho spesso utilizzato in
passato per addestrare i miei discepoli e che spero possano essere di
beneficio per chi mi ascolta qui, oggi.

\section{Chi desidera giungere al Buddha-Dhamma}

Chi desidera giungere al Buddha-Dhamma deve essere fondato nella fede e
nella fiducia. Deve intendere il significato del Buddha-Dhamma in questo
modo:

\begin{itemize}

\item ``Buddha'' è ``Colui che Conosce'',\footnote{Colui che Conosce: La qualità
    della presenza mentale, quella facoltà della mente che, se rettamente
    coltivata, conduce alla Liberazione.} è Colui che è puro, radioso e con la
  pace nel cuore.

\item ``Dhamma'' indica le caratteristiche della purezza, della radiosità e
  della pace che sorgono dalla moralità, dalla concentrazione e dalla saggezza.

\end{itemize}

Chi desidera giungere al Buddha-Dhamma perciò coltiva e sviluppa dentro
di sé la moralità, la concentrazione e la saggezza.

\section{Camminare sul sentiero del Buddha-Dhamma}

Le persone che desiderano raggiungere la loro casa non stanno ovviamente
solo seduti a pensare di viaggiare. Per arrivare a casa devono
effettivamente intraprendere il viaggio, passo dopo passo, e anche nella
giusta direzione. Se prendono la strada sbagliata, possono incorrere in
difficoltà, ad esempio imbattersi in paludi o in altri ostacoli
difficili da superare. In questa direzione sbagliata, possono anche
imbattersi in situazioni pericolose e forse, così, non raggiungere mai
la loro abitazione. Chi vi giunge può rilassarsi e dormire bene, perché
la propria casa è un luogo di benessere per il corpo e per la mente.
Allora ci è arrivato davvero. Se però il viaggiatore si limita a passare
davanti alla propria casa o a girarci attorno, dopo aver fatto tutta
quella strada non ricaverà alcun beneficio.

Allo stesso modo, camminare sul Sentiero per giungere al Buddha-Dhamma è
una cosa che ognuno deve fare da sé, perché nessuno può farlo per noi. E
dobbiamo viaggiare sul giusto Sentiero della moralità, della
concentrazione e della saggezza finché non raggiungiamo le benedizioni
della purezza, della radiosità e della pacificazione della mente, che
rappresentano i frutti di chi percorre il Sentiero.

Se la conoscenza è tratta solo dai libri e dalle Scritture, dai discorsi
e dai \emph{sutta},\footnote{\emph{Sutta:} Letteralmente, ``filo''. Un
  discorso o sermone del Buddha o dei discepoli suoi contemporanei.} ciò
significa solo essere a conoscenza della mappa e del percorso del
viaggio, e nemmeno in centinaia di vite si conosceranno la purezza, la
radiosità e la pacificazione della mente. Si sprecherà solo tempo, senza
mai ottenere i reali benefici della pratica. Gli insegnanti possono solo
indicare la direzione del Sentiero. Dopo aver ascoltato gli insegnanti,
percorrere o meno il Sentiero con la nostra stessa pratica e farne così
maturare i frutti dipende solo da ognuno di noi.

Un altro modo di osservare la questione consiste nel paragonare la
pratica a una boccetta di medicinale che un medico lascia al paziente.
Sulla boccetta sono scritte dettagliate istruzioni su come si debba
prendere la medicina, ma non conta quante centinaia di volte il paziente
possa leggere queste istruzioni, è destinato a morire se si limiterà a
fare solo questo. Non otterrà alcun beneficio dalla medicina. Prima di
morire potrebbe perfino lamentarsi con amarezza dell'assoluta
incompetenza del medico e del fatto che il farmaco non lo abbia curato.
Il paziente penserà che il medico era fasullo o che la medicina non
valeva niente, ma è lui, invece, che ha usato il suo tempo solo per
esaminare la boccetta e leggere le istruzioni. Non ha seguito il
consiglio del medico e non ha preso la medicina.

Il paziente ovviamente guarirà, se seguirà davvero il consiglio del
medico e prenderà la medicina regolarmente come prescritto. Se è
veramente malato, sarà necessario prendere molta medicina, se invece è
solo mediamente malato ne basterà solo un po' per curarlo. Il fatto che
sia necessaria molta medicina è dovuto alla gravità della malattia. Si
tratta di una cosa naturale e, se la considerate con attenzione, potete
capirla da soli. Gli insegnamenti del Buddha servono a curare i mali
della mente, a ricondurla alla sua naturale condizione di salute. Il
Buddha può essere ritenuto un medico che prescrive le cure per le
malattie della mente. È il miglior medico del mondo.

Ognuno di noi, senza eccezioni, ha malattie mentali. Quando vedete
queste malattie mentali, è vero che ha senso guardare al Dhamma come a
un sostegno, come a una medicina in grado di curare le vostre malattie?
Non si viaggia sul sentiero del Buddha-Dhamma con il corpo. Per ottenere
benefici, dovete viaggiarci con la mente. Possiamo suddividere i
viaggiatori nei seguenti tre gruppi.

Primo livello. Questo gruppo è composto da chi comprende che si deve
praticare in prima persona, e sa come farlo. Costoro prendono il Buddha,
il Dhamma e il Saṅgha come rifugio e decidono di praticare con diligenza
secondo gli insegnamenti. Queste persone hanno rifiutato di seguire solo
consuetudini e tradizioni, e invece utilizzano la ragione per esaminare
da soli la natura del mondo. È il gruppo dei ``buddhisti credenti''.

Livello medio. Questo gruppo è composto da chi ha praticato fino al
punto di avere una fiducia incrollabile negli insegnamenti del Buddha,
del Dhamma e del Saṅgha. Costoro hanno anche raggiunto la comprensione
della vera natura di tutti i fenomeni condizionati. Riducono
gradualmente gli attaccamenti e l'aggrapparsi. Non si aggrappano alle
cose e la loro mente ha una profonda comprensione del Dhamma. A seconda
del loro grado di non-attaccamento e della loro saggezza sono conosciuti
-- in linea di progressione -- come ``Chi è entrato nella
Corrente'',\footnote{\emph{Sotāpanna:} ``Chi è entrato nella Corrente''
  e ha così conseguito il primo livello dell'Illuminazione.} ``Chi torna
una sola volta''\footnote{\emph{Sakadāgāmī:} Il secondo stadio
  dell'Illuminazione, ``Chi torna una sola volta'' a esistere in forma
  umana prima di conseguire la Liberazione, di entrare nel
  Nibbāna.} e ``Chi è senza ritorno'',\footnote{\emph{Anāgāmī:}
  ``Chi è senza ritorno'' dopo la morte apparirà in uno dei mondi di
  Brahmā, per poi entrare nel Nibbāna, senza mai tornare in
  questo mondo.} o più semplicemente come Esseri Nobili.\footnote{\emph{Ariya:}
  Nobile è chi ha ottenuto la visione trascendente in uno dei quattro
  livelli dell'Illuminazione, il più alto dei quali è quello
  dell'\emph{arahant}, di cui si parla nel successivo capoverso.}

Livello più alto. Questo è il gruppo delle persone la cui pratica li ha
condotti al corpo, alla parola e alla mente del Buddha. Sono al di sopra
del mondo, sono liberi dal mondo e da ogni attaccamento. Sono gli
\emph{arahant} o Esseri Liberi, il più alto livello degli Esseri Nobili.

\section{Come purificare la moralità}

La moralità è contenimento e disciplina del corpo e della parola. A
livello formale tutto questo si suddivide in classi di precetti per
laici e per monaci e monache. Per parlare in termini generali, la
caratteristica basilare è senza dubbio l'intenzione. Quando abbiamo
consapevolezza o auto-rammemorazione, vi è retta intenzione. Praticare
la consapevolezza (\emph{sati})\footnote{\emph{Sati:} Consapevolezza,
  presenza mentale, attenzione; il termine, molto importante nella
  pratica meditativa buddhista, può significare anche ``memoria''.} e
l'auto-rammemorazione (\emph{sampajañña})\footnote{\emph{Sampajañña:}
  ``Chiara comprensione'', consapevolezza di sé, auto-rammemorazione,
  attenzione, consapevolezza, presenza mentale, comprensione profonda.}
genera la moralità.

Se indossiamo indumenti sporchi e il nostro corpo è sporco, è normale
che pure la nostra mente si senta a disagio e depressa. Ovviamente, se
manteniamo pulito il nostro corpo e indossiamo abiti lindi e puliti, ciò
rende la nostra mente leggera e contenta. Allo stesso modo, quando non
conserviamo la moralità le nostre azioni e le nostre parole sono
sporche, e questo è per la mente causa di infelicità, di afflizione e di
pesantezza. Siamo disgiunti dalla retta pratica e ciò impedisce che
l'essenza del Dhamma penetri nella nostra mente. Le stesse azioni
salutari del corpo e della parola dipendono dalla mente addestrata in
modo giusto, perché è la mente a governare il corpo e la parola. Per
questa ragione dobbiamo praticare con costanza, addestrando la nostra
mente.

\section{La pratica della concentrazione}

Si pratica l'addestramento alla concentrazione (\emph{samādhi}) per
rendere la mente stabile e salda. Ciò determina la tranquillità della
mente. Di solito, le menti non addestrate sono mobili e inquiete,
difficili da controllare e da gestire. La mente segue le distrazioni dei
sensi in modo selvaggio, proprio come l'acqua che, scorrendo, fluisce
verso il basso. Gli agricoltori e gli ingegneri sanno tuttavia come
controllare l'acqua, affinché questa possa essere della maggior utilità
possibile per il genere umano. Gli uomini sono ingegnosi, sanno come
arginare l'acqua, sanno costruire grandi bacini idrici e canali: tutto
questo solo per incanalare l'acqua e renderla più utile. Per di più,
l'acqua accumulata diviene una fonte di energia elettrica e di luce. Un
ulteriore beneficio derivante dal controllo del flusso dell'acqua è che
essa non scende in modo impetuoso né si ferma a caso qua e là, in un
qualche punto più basso, facendo andare sprecata tutta la sua utilità.

Così, anche la mente incanalata, controllata e addestrata con costanza
sarà d'incommensurabile beneficio. Lo insegnò il Buddha stesso: «~La
mente domata~reca vera felicità: perciò addestrate bene le vostre menti
per ottenere il più alto dei benefici.~» In modo simile, gli animali che
vediamo attorno a noi -- gli elefanti, i cavalli, il bestiame, i bufali
d'acqua e via dicendo -- devono essere addestrati prima di poter essere
impiegati per lavorare. La loro forza ci è utile solo dopo che sono
stati addestrati.

Anche la mente addestrata ci recherà un numero di benedizioni molto
maggiore rispetto a una mente non addestrata. Il Buddha e i suoi Nobili
Discepoli iniziarono tutti nello stesso modo, come noi, con menti non
addestrate. Guardate però come divennero oggetto di venerazione per
tutti noi, e vedete quanto beneficio abbiamo potuto trarre dai loro
insegnamenti. Osservate quale beneficio ha ottenuto il mondo intero da
questi uomini che, passando attraverso l'addestramento della mente,
hanno raggiunto la Libertà. La mente controllata e addestrata è meglio
attrezzata per aiutarci in ogni professione, in ogni situazione. La
mente disciplinata manterrà in equilibrio la nostra vita, renderà più
facile il lavoro, e svilupperà e nutrirà la ragione nel governo delle
nostre azioni. Alla fine, la nostra felicità crescerà man mano che
seguiremo il giusto addestramento mentale.

\section{La consapevolezza e il respiro}

La mente può essere addestrata in molti modi e con vari metodi. Il
metodo più utile, che può essere praticato da tutti, è conosciuto come
``consapevolezza del respiro''. Significa sviluppare la presenza mentale
dell'inspirazione e dell'espirazione. In questo monastero concentriamo
l'attenzione sulla punta del naso e sviluppiamo la consapevolezza
dell'inspirazione e dell'espirazione con il mantra
``Bud-dho''.\footnote{Buddha (\emph{Buddho}): Letteralmente,
  ``Risvegliato'', ``Illuminato''. Questa parola viene anche usata per
  la meditazione, recitando interiormente \emph{Bud-} nel corso
  dell'inspirazione e \emph{-dho} durante l'espirazione.} Se il
meditante desidera utilizzare un'altra parola o essere semplicemente
consapevole dell'aria che entra ed esce, va bene ugualmente. Adeguate la
pratica affinché vi sia di giovamento. Nella meditazione il fattore
essenziale è che l'osservazione o la consapevolezza del respiro sia
sostenuta nel momento presente, in modo tale da essere mentalmente
presenti a ogni inspirazione ed espirazione, così come sono. Quando
facciamo la meditazione camminata, cerchiamo di essere costantemente
consapevoli della sensazione di contatto tra i piedi e il terreno.

Questa pratica di meditazione per essere fruttuosa dev'essere eseguita
nel modo più continuo possibile. Non meditate per poco tempo un giorno e
poi di nuovo una o due settimane o, addirittura, un mese dopo. Non
otterrete alcun risultato. Il Buddha insegnò a praticare spesso e con
diligenza, ossia a essere il più costanti possibile nella pratica
dell'addestramento mentale. Per praticare la meditazione dovremmo
trovare un posto convenientemente tranquillo, ove non ci siano
distrazioni. In giardino sotto l'ombra degli alberi, oppure in posti nei
quali si può essere soli: sono tutti ambienti adatti. Se siamo monaci o
monache, dovremmo trovare una capanna appropriata, una foresta o una
caverna silenziosa. Le montagne offrono posti particolarmente adatti
alla pratica.

Ad ogni modo, dobbiamo sforzarci di essere continuamente consapevoli
dell'inspirazione e dell'espirazione, ovunque ci troviamo. Se
l'attenzione vaga verso altre cose, cercate di riportarla sull'oggetto
della concentrazione. Cercate di mettere da parte tutti gli altri
pensieri e ogni altra preoccupazione. Non pensate a nulla, osservate
solo il respiro. Se siamo consapevoli dei pensieri appena essi sorgono e
continuiamo a tornare con gentilezza all'oggetto della meditazione, la
mente diverrà sempre più quieta.

Quando la mente è in pace e concentrata, staccatevi dal respiro quale
oggetto di meditazione. Iniziate allora a esaminare il corpo e la mente
nei cinque \emph{khandhā}\footnote{\emph{Khandhā:} Aggregato, insieme di
  elementi col quale ci si identifica; le componenti fisiche e mentali
  della personalità e dell'esperienza sensoriale in generale.} che li
compongono: forma materiale, sensazioni, percezioni, formazioni mentali
e coscienza. Esaminate questi cinque \emph{khandhā}, quando arrivano e
quando se ne vanno. Vedete con chiarezza la loro impermanenza e come
quest'impermanenza li renda insoddisfacenti e indesiderabili, come essi
vadano e vengano da soli, come non ci sia alcun ``sé'' a governare le
cose. Vi è solo la natura che si muove secondo causa ed effetto. Tutto
nel mondo è soggetto all'instabilità, ha un carattere insoddisfacente ed
è privo di un sé o di un'anima permanente. Vedere ogni cosa esistente in
questa prospettiva ridurrà gradualmente l'attaccamento e l'aggrapparsi
ai \emph{khandhā}. Ciò avviene perché si vedono le vere caratteristiche
del mondo. Diciamo che sta sorgendo la saggezza.

\section{Sorge la saggezza}

Saggezza (\emph{paññā}) è vedere la Verità delle varie manifestazioni
del corpo e della mente. Quando utilizzeremo la nostra mente addestrata
e concentrata per esaminare i cinque \emph{khandhā}, vedremo con
chiarezza che sia il corpo sia la mente sono impermanenti,
insoddisfacenti e privi di un sé. Se vediamo con saggezza tutte le cose
composte, non ci attacchiamo né ci aggrappiamo a esse. Qualsiasi cosa
riceviamo, la riceviamo consapevolmente. Non siamo eccessivamente
felici. Quando le cose che ci appartengono vanno in pezzi o svaniscono,
non siamo infelici e non soffriamo per le sensazioni di dolore che
sorgono, perché vediamo con chiarezza che tutto è impermanente. Quando
ci imbattiamo in malattie e in sofferenze di qualsiasi genere, siamo
equanimi perché le nostre menti sono state ben addestrate. Il vero
rifugio è una mente addestrata.

Tutto questo è noto come un tipo di saggezza che, quando sorgono le
cose, conosce le loro vere caratteristiche. Questa saggezza sorge dalla
consapevolezza e dalla concentrazione. La concentrazione sorge da una
base di moralità o virtù. Moralità, concentrazione e saggezza sono così
correlate da non poter essere separate. Concretamente, si può vedere
tutto ciò in questo modo. Inizialmente vi è la disciplina della mente,
per renderla attenta al respiro, e questo è il sorgere della moralità.
Quando la consapevolezza del respiro è praticata in continuazione fino a
che la mente è calma, questo è il sorgere della concentrazione. Poi, la
disamina che mostra il respiro come impermanente, insoddisfacente e
privo di un sé, e il conseguente non attaccamento a esso, è il sorgere
della saggezza. La pratica della consapevolezza del respiro può perciò
essere ritenuta un percorso per lo sviluppo della moralità, della
concentrazione e della saggezza. Sono tutte in contatto.

Quando la moralità, la concentrazione e la saggezza sono ben sviluppate,
diciamo che stiamo praticando il Nobile Ottuplice Sentiero\footnote{Nobile
  Ottuplice Sentiero: Gli otto fattori che conducono alla fine della
  sofferenza; tali fattori sono elencati nel \emph{Glossario}, p. \pageref{glossary-ottuplice}.}
insegnatoci dal Buddha come unica via d'uscita dalla sofferenza. Il
Nobile Ottuplice Sentiero è al di sopra di tutti gli altri perché, se
correttamente praticato, conduce direttamente al
Nibbāna,\footnote{Nibbāna (sanscrito \emph{Nirvāṇa}): La
  Liberazione finale da ogni sofferenza, lo scopo della pratica
  buddhista.} alla pace. Possiamo dire che questa pratica raggiunge in
verità, e con precisione, il Buddha-Dhamma.

\section{Benefici della pratica}

Quando abbiamo praticato la meditazione nel modo appena spiegato, i
frutti della pratica sorgeranno nei seguenti tre stadi:

\begin{itemize}

\item Per chi si trova al livello dei ``buddhisti credenti'', aumenterà
  la fiducia nel Buddha, nel Dhamma e nel Saṅgha. Questa fiducia diverrà
  il vero supporto interiore per ognuno di loro. Inoltre, queste persone
  comprenderanno che il rapporto di causa-effetto vale per ogni cosa,
  che azioni salutari comportano risultati salutari e che azioni non
  salutari comportano risultati non salutari. La loro felicità e la loro
  pace mentale aumenterà molto.

\item Coloro che conseguiranno la nobile condizione di ``Chi è entrato nella
  Corrente'', ``Chi torna una sola volta'' o ``Chi è senza ritorno'', avranno
  una fiducia incrollabile nel Buddha, nel Dhamma e nel Saṅgha. Saranno gioiosi
  e sospinti verso il Nibbāna.

\item Per gli \emph{arahant} o ``Esseri giunti alla perfezione'', la felicità
  sarà priva di ogni sofferenza. Costoro sono i Buddha, sono liberi dal mondo,
  hanno completato il percorso sulla santa via.

\end{itemize}

Tutti noi abbiamo avuto la fortuna di nascere come esseri umani e di
ascoltare gli insegnamenti del Buddha. È un'opportunità che milioni di
altri esseri non hanno. Perciò, non siate trascurati o distratti.
Affrettatevi e sviluppate meriti, fate il bene e seguite il Sentiero
della pratica al livello iniziale, medio e a quello più alto. Non
lasciate che il tempo passi inutilizzato e senza scopo. Cercate di
raggiungere la Verità indicata dagli insegnamenti del Buddha anche oggi
stesso. Consentitemi di terminare con un detto popolare laotiano:
«~Molti sono i momenti di allegria e di piacere già passati, sarà subito
sera. Ebbro di lacrime, fermati e osserva, presto sarà troppo tardi per
terminare il viaggio.~»

