\chapter{Rendere buono il cuore}

\begin{openingQuote}
  \centering

  Discorso tenuto a un vasto gruppo di laici giunto al Wat Pah Pong per fare
  offerte in supporto del monastero.
\end{openingQuote}

Di questi tempi la gente va in giro ovunque in cerca di
meriti.\footnote{``Cercare meriti'' è una frase comunemente utilizzata
  dai thailandesi. Si riferisce alla consuetudine di recarsi nei
  monasteri (``wát'', tempio: \thai{วัด}) per porgere omaggio a venerati
  maestri e fare offerte.} E pare che ci si fermi sempre al Wat Pah
Pong. Se non ci si ferma all'andata, lo si fa al ritorno del viaggio. Il
Wat Pah Pong è diventato un luogo di sosta. Alcuni vanno talmente di
fretta che non ho né l'opportunità di vederli né di parlare con loro. La
maggioranza delle persone va in cerca di meriti. Non ne vedo molti in
cerca di una via d'uscita dalle cattive azioni. Sono così intenti a
cercare meriti che non sanno dove metterli. È come cercare di tingere
una stoffa sporca, non lavata.

I monaci parlano in modo diretto, ma per la maggior parte delle persone
è difficile mettere in pratica questo genere d'insegnamento. È
difficile perché non capiscono. Se capissero sarebbe molto più facile.
Supponiamo che ci sia una buca con qualcosa sul fondo. Ora, chi mettesse
la mano nella buca senza riuscire a raggiungere il fondo, direbbe che la
buca è troppo profonda. Cento o mille persone che mettessero le mani giù
in quella buca, direbbero tutte quante che è troppo profonda. Nessuno
dirà che il braccio è troppo corto!

C'è tanta gente in cerca di meriti. Prima o poi dovranno cominciare a
cercare una via d'uscita dalle cattive azioni. Però non è poi così tanta
la gente interessata a questo. L'insegnamento del Buddha è così conciso,
ma la maggioranza della gente ci passa solo vicino, proprio come passa
al Wat Pah Pong. Per la maggioranza ecco cos'è il Dhamma, un luogo di
sosta.

Solo tre parole, quasi niente: \emph{Sabbapāpassa akaraṇaṃ}. Astenersi
da ogni cattiva azione. Questo è l'insegnamento di tutti i Buddha.
Questo è il cuore del buddhismo. La gente continua a scavalcarlo, non lo
vuole. La rinuncia a tutte le cattive azioni, grandi e piccole, del
corpo, della parola e della mente, questo è l'insegnamento dei Buddha.

% FIXME: text to be removed from below

Se vogliamo tingere un pezzo di stoffa, dobbiamo prima lavarlo. Ma la
maggioranza non lo fa. Senza guardare la stoffa, la immerge subito nella
tinta. Se la stoffa è sporca, tingerla la fa diventare peggio di prima.
Pensateci. Tingere uno straccio vecchio e sporco: avrebbe un
bell'aspetto? Capite? Questo insegna il buddhismo, ma la maggioranza
della gente semplicemente lo ignora. Vogliono fare opere buone, ma non
vogliono rinunciare alle cattive azioni. È proprio come dire: «~La buca
è troppo profonda.~» Tutti dicono che la buca è troppo profonda, nessuno
dice che il braccio è troppo corto. Dovete tornare a voi stessi. Con
questo insegnamento, dovete fare un passo indietro, e guardare voi
stessi.

A volte le persone vanno in cerca di meriti in autobus. Sull'autobus
forse discutono perfino, o si ubriacano. Se chiedete loro dove stanno
andando, dicono che vanno in cerca di meriti. Vogliono i meriti, ma non
rinunciano ai vizi. In questo modo non troveranno mai i meriti. Così è
la gente. Dovete guardare vicino, guardate voi stessi. Il Buddha disse
di avere rammemorazione e consapevolezza di sé in tutte le situazioni. I
comportamenti sbagliati sorgono nelle azioni del corpo, della parola e
della mente. La fonte di tutto il bene, il male, il benessere e il
pericolo è legato alle azioni, alle parole e ai pensieri. Oggi con voi
avete portato le vostre azioni, le vostre parole e i vostri pensieri?
Oppure li avete lasciati a casa? È qui che dovete guardare, proprio qui.
Non dovete guardare molto lontano. Guardate le vostre azioni, le vostre
parole e i vostri pensieri. Guardate, per vedere se la vostra condotta è
sbagliata o no.

In realtà la gente queste cose non le guarda. Come la casalinga, che
lava i piatti accigliata. È intenta a lavare i piatti, ma non si accorge
che la sua mente è sporca! Non lo avete mai notato? Vede solo i piatti.
Sta guardando troppo lontano, o no? Alcuni di voi l'hanno sperimentato,
direi. È lì che dovete guardare. La gente si concentra a pulire i
piatti, e lascia che la mente si sporchi. Questo non va bene, sta
dimenticando se stessa. Siccome non vede se stessa, la gente commette
ogni genere di cattive azioni. Le persone non guardano la loro mente.
Quando stanno per fare qualcosa di male, prima di tutto si guardano
attorno per controllare se c'è qualcuno che li vede. «~Mia madre mi
vedrà?~» «~Mio marito mi vedrà?~» «~I miei figli mi vedranno?~» «~Mia
moglie mi vedrà?~» E se nessuno li guarda, vanno avanti e lo fanno.
Questo è insultare se stessi. Dicono che nessuno sta osservando, così
possono finire in fretta il lavoro prima che qualcuno li veda. Ma che
dire di loro stessi? Non sono ``qualcuno''?

Capite? Siccome trascura se stessa in questo modo, la gente non trova
mai ciò che ha realmente valore, il Dhamma. Se guardate voi stessi,
vedrete voi stessi. Ogni volta che state per fare qualcosa di male, se
guardate voi stessi potete fermarvi in tempo. Se volete fare qualcosa di
utile, guardate la vostra mente. Se conoscete il modo per guardare voi
stessi, allora saprete quello che è giusto e quello che è sbagliato,
quello che è dannoso e quello che è benefico, il vizio e la virtù. Sono
cose che dovremmo conoscere. Se non vi parlo di queste cose, non le
conoscerete. Nella mente avete avidità e illusione, ma non lo sapete.
Non conoscerete mai nulla, se guardate sempre all'esterno. Questo è il
problema con le persone, che non guardano se stesse. Guardandovi dentro
vedrete bene e male. Vedendo la bontà, possiamo prenderla a cuore e
praticare di conseguenza.

Rinunciare al male, praticare il bene; questo è il cuore del buddhismo.
\emph{Sabba-pāpassa akaraṇaṃ}, non commettere nessuna cattiva azione, né
col corpo, né con la parola, né con la mente. Questa è la retta pratica,
l'insegnamento del Buddha. Ora la nostra ``stoffa'' è pulita. Allora c'è
\emph{kusalassūpasampadā}, rendiamo la mente virtuosa e abile. Se la
mente è virtuosa e abile, non dobbiamo percorrere in autobus tutte le
campagne della regione in cerca di meriti. Anche seduti a casa possiamo
ottenere meriti. La maggioranza della gente, però, va solo in cerca di
meriti per tutte le campagne della regione, senza rinunciare ai vizi.
Poi, a mani vuote, torna a casa alla solita faccia truce. E lì lava i
piatti. È tutta intenta a pulirli con la faccia truce. Questo è ciò a
cui la gente non presta attenzione. È così lontana dai meriti!

Possiamo conoscerle queste cose, ma non le conosciamo davvero se non
conosciamo le nostre stesse menti. Il buddhismo non entra nel nostro
cuore. Se è buona e virtuosa, la nostra mente è felice. C'è un sorriso
nel nostro cuore. Per la maggior parte di noi, però, è difficile trovare
il tempo per sorridere, vero? Riusciamo a sorridere solo quando le cose
vanno a modo nostro. La felicità della maggior parte delle persone
dipende dal fatto che le cose vadano come piace a loro. Hanno bisogno
che nel mondo tutti dicano solo cose piacevoli. È questo il modo di
trovare la felicità? È possibile che nel mondo tutti dicano solo cose
piacevoli? Se è così, quando troverete mai la felicità?

Per trovare la felicità dobbiamo usare il Dhamma. Di qualsiasi cosa si
tratti, giusta o sbagliata che sia, non attaccatevi ciecamente a essa.
Notatela solo e, poi, posatela. Quando la mente è a proprio agio, ecco
quando potete sorridere. Nel momento in cui provate avversione per una
cosa, la mente si guasta. Allora non c'è nulla che vada bene.
\emph{Sacittapariyodapanaṃ}: dopo aver eliminato le impurità, la mente è
libera dalle preoccupazioni. È serena, gentile e virtuosa. Quando la
mente è radiosa e ha rinunciato al male, c'è sempre benessere. Una mente
serena e tranquilla è la quintessenza, lo scopo dell'esistenza umana.

Quando gli altri dicono cose che ci piacciono, sorridiamo. Se dicono
cose che ci dispiacciono, ci accigliamo. Com'è possibile che tutti i
giorni gli altri dicano solo cose che ci piacciono? È possibile? Perfino
i vostri figli, hanno mai detto cose che non vi piacciono? Avete mai
fatto arrabbiare i vostri genitori? Non solo gli altri, ma anche le
vostre stesse menti possono farvi arrabbiare. A volte le cose che
pensiamo non sono piacevoli. Che potete fare? Tutt'a un tratto, mentre
state camminando, potreste dare un calcio al ceppo di un albero \ldots{}
Dump! \ldots{} «~Ahi!~» \ldots{} Dov'è il problema? E comunque, chi è che ha dato
il calcio? Con chi potete prendervela? L'errore è solo vostro. Anche le
nostre stesse menti possono causarci dispiacere. Se ci pensate, vedrete
che è vero. A volte facciamo cose che non ci piacciono. Tutto quello che
potete dire è: «~Accidenti!~» Non potete incolpare nessuno.

Nel buddhismo ottenere meriti o benedizioni consiste nel rinunciare a
ciò che è sbagliato. Quando abbandoniamo l'errore, non siamo più in
torto. Quando non c'è più tensione, c'è calma. La mente calma è una
mente pulita, non vi albergano pensieri rabbiosi, è chiara. Come
possiamo rendere chiara la mente? Solo conoscendola. Ad esempio,
potreste pensare: «~Oggi sono proprio di cattivo umore, tutto quello che
guardo mi urta, perfino i piatti nella credenza.~» Potrebbe venirvi il
desiderio di romperli, uno per uno. Qualsiasi cosa guardiate pare
brutta. Le galline, le anatre, i gatti e i cani \ldots{} odiate tutto. Tutto
quello che vostro marito dice vi suona offensivo. Anche guardare nella
vostra stessa mente non vi soddisfa. Che potete fare in questa
situazione? Da dove viene questa sofferenza? È quel che si dice ``non
avere meriti''. Ora in Thailandia quando qualcuno muore si dice che i
suoi meriti sono finiti. Non è così. C'è un gran numero di persone
ancora in vita i cui meriti sono già finiti. Questa è gente che non sa
cosa siano i meriti. La mente cattiva ammucchia solo sempre più
cattiveria.

Prendere parte a queste gite per accumulare meriti è come costruire una
bella casa senza prima preparare l'area sulla quale dovrà sorgere. Non
molto tempo dopo la casa crolla, vero? Il progetto non era buono. Ora
dovete provare di nuovo, provare una via diversa. Dovete guardare dentro
di voi, guardare le mancanze nelle vostre azioni, nelle vostre parole e
nei vostri pensieri. In quale altro posto volete praticare, se non nelle
vostre azioni, nelle vostre parole e nei vostri pensieri? La gente si
perde. Vuole andare a praticare il Dhamma dove c'è vera tranquillità,
nella foresta e al Wat Pah Pong. È un posto tranquillo il Wat Pah Pong?
No, non è davvero tranquillo. È a casa vostra che c'è vera tranquillità.

Se avete saggezza, ovunque andiate sarete liberi da preoccupazioni.
L'intero mondo va bene così com'è. Tutti gli alberi della foresta vanno
bene così come sono, ce n'è di alti, bassi, cavi \ldots{} di tutti i tipi.
Sono semplicemente nel modo in cui sono. Ignorando la loro vera natura,
forziamo le cose e applichiamo a essi le nostre opinioni. «~Oh,
quell'albero è troppo basso! Quell'albero è cavo!~» Quegli alberi sono
semplicemente alberi, sono meglio di noi.

È per questa ragione che ho fatto attaccare quelle frasi agli alberi.
Lasciate che gli alberi v'insegnino. Avete già imparato qualcosa da
loro? Dovreste cercare di imparare almeno una cosa. Ci sono così tanti
alberi, ognuno di essi ha qualcosa da insegnare. Il Dhamma è ovunque, è
in ogni cosa della natura. Dovreste capirlo. Non andate a prendervela
con la buca perché è troppo profonda. Voltatevi, e guardatevi il
braccio! Se potete vedere questo, sarete felici.

Se ottenete meriti o virtù, preservateli nella vostra mente. È il
miglior posto per conservarli. Fare meriti come avete fatto oggi va
bene, ma questo non è il modo migliore. Costruire edifici va bene, ma
non è la cosa migliore. È meglio costruire la vostra mente, facendola
diventare buona. Realizzate questa perfezione dentro la vostra mente. Le
strutture esteriori, come questa sala, sono la ``corteccia''
dell'``albero'', non il ``durame''.

Se avete saggezza, ovunque guardiate ci sarà il Dhamma. Se mancate di
saggezza, anche le cose buone si trasformeranno in cattive. Da dove
viene la cattiveria? Solo dalle nostre menti, ecco da dove. Guardate
come cambia, questa mente. Tutto cambia. Marito e moglie andavano
perfettamente d'accordo, potevano parlare serenamente. Un giorno, però,
il loro umore va male e tutto quello che dicono pare offensivo. La mente
s'è rovinata, è cambiata di nuovo. È così che stanno le cose.

Per rinunciare al male e coltivare il bene non dovete andare a cercare
altrove. Se la mente s'è rovinata, non state a esaminare questa o quella
persona. Guardate solo la vostra mente e scoprite da dove vengono questi
pensieri. Perché la mente pensa queste cose? Comprendete che tutto è
transitorio. L'amore è transitorio, l'odio è transitorio. Avete mai
amato i vostri figli? Ovviamente. Li avete mai odiati? Anche in questo
caso risponderò per voi. A volte li avete odiati, o no? Potete gettarli
via? No, non potete gettarli via. Perché no? I figli non sono
pallottole,\footnote{È un gioco di parole tra \emph{luuk} (\thai{ลูก}), che
  significa figli, e \emph{luuk puen} (\thai{ลูกปืน}), che letteralmente
  significa ``figli del fucile'', ossia pallottole.} vero? Le pallottole
sparano fuori, ma i figli sparano proprio sui genitori. Se sono cattivi,
questo si ripercuote sui genitori. Potreste dire che i figli sono il
vostro kamma. Ce ne sono di buoni e di cattivi. Sia il bene sia
il male sono proprio qui, nei vostri figli. Però anche un figlio che ha
problemi è prezioso. Può essere nato con la polio, storpio e deforme, ma
essere anche più prezioso degli altri. Ogni volta che vi allontanate di
casa per un po', dovete lasciare un messaggio: «~Bada al piccolo, non è
così forte.~» Lo amate anche più degli altri.

Dovreste impostare bene la vostra mente, metà amore e metà odio. Non
prendetene solo uno, abbiate sempre entrambi gli estremi nella mente. I
vostri figli sono il vostro kamma, sono adatti ai loro genitori.
Sono il vostro kamma, perciò dovete assumervi le vostre
responsabilità. Se vi fanno davvero soffrire, ricordate a voi stessi:
«~È il mio kamma~». Se vi rendono contenti, ricordate a voi
stessi: «~È il mio kamma~». A casa ci si sente talvolta così
frustrati da voler solo scappare via. Ci si sente così male che alcuni
pensano perfino di impiccarsi! È il kamma. Dovete accettare
questo dato di fatto. Evitate le cattive azioni, e poi sarete in grado
di vedere voi stessi con maggiore chiarezza.

È per questa ragione che contemplare le cose è così importante. Di
solito la gente quando pratica meditazione usa un oggetto di
meditazione, come \emph{Bud-dho}, \emph{Dham-mo} o \emph{Saṅ-gho}.
Potete rendere le cose ancora più semplici. Quando vi sentite irritati,
ogni volta che la mente va male, dite solo: «~Ecco!~» Quando vi sentite
meglio, dite solo: «~Ecco, non è una cosa certa!~» Se amate qualcuno,
dite solo: «~Ecco!~» Se sentite che state per arrabbiarvi, dite solo:
«~Ecco!~» Capite? Non dovete andare a cercare nel
\emph{Tipiṭaka}.\footnote{\emph{Tipiṭaka.} Il Canone buddhista in pāli.}
Solo ``ecco''. Significa ``è transitorio''. L'amore è transitorio,
l'odio è transitorio, il bene è transitorio, il male è transitorio. Come
potrebbero essere permanenti? Dov'è una qualche permanenza?

Potreste dire che sono permanenti perché invariabilmente impermanenti.
Da questo punto di vista si tratta di cose certe, non diventano mai
qualcos'altro. Per un minuto c'è amore, nel minuto successivo c'è odio.
Così stanno le cose. In questo senso sono permanenti. Per questa ragione
penso che quando sorge l'amore, dovete solo dire: «~Ecco!~» Si risparmia
un sacco di tempo. Non è necessario dire: \emph{aniccā},\footnote{\emph{anicca.}
  Incostante, instabile, impermanente. Si veda \emph{tilakkhaṇa} (Tre
  Caratteristiche) nel \emph{Glossario}, p. \pageref{glossary-tilakkhana}.} \emph{dukkha}, \emph{anattā}.
Se non volete lunghi temi di meditazione, usate solo questa semplice
parola. Se sorge l'amore, prima che vi perdiate completamente in esso,
dite a voi stessi: «~Ecco!~» È sufficiente.

Tutto è transitorio, ed è permanente solo in quanto è invariabilmente
così. Basta vedere solo questo per vedere il cuore del Dhamma, il vero
Dhamma. Ora, se ognuno dicesse più spesso ``ecco'', e lo applicasse a se
stesso per addestrarsi, l'attaccamento diminuirebbe sempre più. La gente
non saebbe così bloccata nell'amore e nell'odio. Non si attaccherebbe
alle cose. Riporrebbe la sua fiducia nella Verità, non in altre cose.
Sapere solo questo è sufficiente, cos'altro avete bisogno di sapere?

Dopo aver ascoltato questo insegnamento, dovreste anche cercare di
ricordarlo. Che cosa dovreste ricordare? Meditate \ldots{} Capite? Se capite,
con un ``click!'' il Dhamma entrerà in sintonia con voi e la mente si
fermerà. Se c'è rabbia nella mente, dite solo: «~Ecco!~» È abbastanza,
si fermerà immediatamente. Se non capite ancora, guardate la questione
più in profondità. Se c'è comprensione, quando la rabbia sorge nella
mente potete spegnerla solo con un: «~Ecco, è impermanente!~»

Oggi avete avuto l'opportunità di registrare il Dhamma sia interiormente
che esteriormente. Interiormente, il suono entra attraverso le orecchie
per essere registrato nella mente. Se non riuscite a farlo non va bene,
il vostro tempo al Wat Pah Pong sarà sprecato. Registratelo
esteriormente, e registratelo interiormente. Questo registratore che sta
qui non è poi così importante. La cosa davvero importante è il
``registratore'' nella mente. Il registratore è deperibile, ma se il
Dhamma raggiunge la mente è indeperibile, resta lì per sempre. E non
dovete sprecare soldi per le batterie.

