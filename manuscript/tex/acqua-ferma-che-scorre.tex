\chapter{Acqua ferma che scorre}

\begin{openingQuote}
  \centering

  Discorso offerto al Wat Tham Saeng Phet, durante\\
  il Ritiro delle Piogge del~1981.
\end{openingQuote}

Ora per favore fate attenzione, non consentite alla vostra mente di
vagare e inseguire altre cose. Proprio ora inducete in voi la sensazione
che potreste avere stando seduti su di una montagna o da qualche parte
in una foresta, completamente soli. Per quale ragione ora siete qui? C'è
il corpo e c'è la mente, questo è tutto, ci sono solamente queste due
cose. Tutto quel che è contenuto in quest'intelaiatura che sta qui
seduta è chiamata ``corpo''. La mente è ciò che è consapevole e, proprio
in questo momento, sta pensando. Queste due cose sono anche dette
\emph{nāma} e \emph{rūpa}. \emph{Nāma} si riferisce a quel che non ha
\emph{rūpa}, o forma. Ogni pensiero e sensazione, oppure i quattro
\emph{khandhā} mentali di sensazione, percezione, volizione e coscienza
sono \emph{nāma}, sono privi di forma. Tutto insieme è detto \emph{nāma}
e \emph{rūpa}, o in modo più semplice mente e corpo.

Proprio ora solo il corpo e la mente sono qui seduti, comprendetelo.
Queste due cose, però, le confondiamo. Se volete la pace dovete
conoscere la loro verità. La mente, nella sua attuale condizione, è
ancora non addestrata, è sporca, non è limpida. Non è ancora la mente
pura. Dobbiamo addestrare ulteriormente questa mente per mezzo della
pratica di meditazione. Alcuni pensano che fare meditazione significhi
stare seduti in un qualche modo particolare. In realtà, essere in piedi,
seduti, camminare e stare distesi sono tutti veicoli della pratica
meditativa. Potete praticare sempre.

\emph{Samādhi} letteralmente
significa ``mente salda e ferma''. Per sviluppare il \emph{samādhi} non
dovete imbottigliare la mente. Alcuni cercano di diventare sereni
sedendosi in tranquillità senza essere disturbati da nessuno, ma questo
è come essere morti. La pratica del \emph{samādhi} serve a sviluppare
saggezza e comprensione.

\emph{Samādhi} è mente stabile, mente unificata. Su quale punto è ferma?
È ferma sul punto di equilibrio. La sua posizione è questa. La gente,
però, pratica la meditazione cercando di far tacere la mente. Dicono:
«~Cerco di sedere in meditazione, ma la mia mente non sta ferma nemmeno
un attimo. Un momento vola in un posto, un momento dopo se ne va da
qualche altra parte. Come posso riuscire a fermarla e a rasserenarla?~»
Non dovete fermarla, non è questo il punto. Dove c'è movimento, è lì che
può sorgere la comprensione. La gente si lamenta: «~Lei scappa via e io
la riporto di nuovo indietro; poi se ne va di nuovo e io la riporto
indietro ancora una volta.~» Così, se ne stanno lì seduti ad andare
avanti e indietro in questo modo. Pensano che le loro menti stiano
scappando dappertutto, ma che la mente se ne vada in giro è solo
un'apparenza. Guardate ad esempio questa sala. Potreste dire: «~Oh, è
così grande!~» In verità, non è affatto grande. Che sembri grande o meno
dipende dalla percezione che ne avete. Nei fatti, questa sala ha solo le
dimensioni che ha, non è né grande né piccola, ma la gente gira sempre
in tondo attorno alle proprie sensazioni.

Per praticare la meditazione e trovare la pace, dovete comprendere cosa
sia la pace. Se non lo comprendete, non potrete trovarla. Supponiamo, ad
esempio, che oggi per venire in monastero abbiate portato con voi una
penna molto costosa. Supponiamo pure che, camminando, abbiate messo la
penna in un taschino sul davanti, ma l'abbiate poi spostata in una tasca
posteriore. Ora, quando cercate nel taschino, non è là! Vi spaventate.
Vi spaventate a causa del vostro stesso fraintendimento, non vedete la
verità. Il risultato è la sofferenza. Mentre state in piedi, mentre
camminate e andate su e giù, non riuscite a smettere di preoccuparvi per
la penna perduta. La vostra errata comprensione vi causa sofferenza.
Comprendere in modo errato causa sofferenza. «~Che peccato! Ho
acquistato questa penna solo pochi giorni fa, e ora l'ho perduta.~»
Però, poi ricordate: «~Ah, certo! Quando sono andato a lavarmi ho messo
la penna nella tasca posteriore.~» Appena ve lo ricordate, vi sentite
meglio, anche senza aver visto la vostra penna. Capite? Siete nuovamente
felici, potete smettere di preoccuparvi della vostra penna. Ne siete
certi. Mentre camminate, andate con la mano nella tasca posteriore, ed
eccola là. La vostra mente vi ha ingannato per tutto il tempo. La
preoccupazione proviene dalla vostra ignoranza. Ora, vedendo la penna,
avete superato i dubbi, le vostre preoccupazioni si acquietano. Questo
genere di pace deriva dall'aver compreso la causa del problema,
\emph{samudaya}, la causa della sofferenza. Appena vi ricordate che la
penna si trova nella tasca posteriore c'è \emph{nirodha}, la cessazione
della sofferenza.

Così, per trovare la pace dovete contemplare. Di solito quel che la
gente intende per pace è solo calmare la mente, non calmare le
contaminazioni. Le contaminazioni sono soggiogate solo temporaneamente,
proprio come l'erba coperta da un sasso. Dopo tre o quattro giorni
togliete il sasso e non molto tempo dopo l'erba crescerà di nuovo.
L'erba non era davvero morta, era solo occultata. Quando sedete in
meditazione è la stessa cosa. La mente si è calmata, ma le
contaminazioni non si sono calmate davvero. Perciò, il \emph{samādhi}
non è una cosa certa. Per trovare la vera pace dovete sviluppare la
saggezza. Il \emph{samādhi} è un tipo di pace, è come la pietra che
copre l'erba. Togliete la pietra e dopo pochi giorni l'erba cresce di
nuovo. È solo una pace temporanea. La pace della saggezza equivale a
posare la pietra senza toglierla più, lasciarla lì dov'è. L'erba non può
più crescere. Questa è la vera pace, l'acquietarsi delle contaminazioni,
la pace certa che proviene dalla saggezza.

Parliamo di saggezza (\emph{paññā}) e di \emph{samādhi} come di due cose
diverse, ma essenzialmente si tratta della stessa cosa, di una sola
cosa. La saggezza è la funzione dinamica del \emph{samādhi}. Il
\emph{samādhi} è l'aspetto passivo della saggezza. Sorgono dallo stesso
posto, ma prendono direzioni diverse. Hanno funzioni differenti, come
questo mango. Un piccolo mango verde diventa col passare del tempo
sempre più grande, fino a quando matura. Si tratta dello stesso mango,
quello piccolo e quello più grande e maturo sono lo stesso mango, ma le
sue condizioni sono mutate. Nella pratica del Dhamma, una condizione è
chiamata \emph{samādhi}, la condizione successiva è chiamata
\emph{paññā}, ma in realtà \emph{sīla}, \emph{samādhi} e \emph{paññā}
sono tutte la stessa cosa, proprio come il mango.

A ogni modo, indipendentemente dalla condizione alla quale vogliate far
riferimento, nella nostra pratica dovete sempre iniziare dalla mente.
Sapete che cos'è questa mente? Com'è? Cos'è? Dov'è? Nessuno lo sa.
Sappiamo solo che vogliamo andare qui o là, che vogliamo questo e che
vogliamo quello, che ci sentiamo bene o che ci sentiamo male, ma la
mente stessa sembra impossibile conoscerla. Che cos'è la mente? La mente
non ha forma. Quel che riceve le impressioni, sia belle che brutte, la
chiamiamo ``mente''. Somiglia al proprietario di una casa. Il
proprietario resta in casa quando degli ospiti vengono a trovarlo. È lui
che riceve gli ospiti. Chi riceve le impressioni sensoriali? Che cos'è
che percepisce? Chi è che lascia andare le impressioni sensoriali?
Questo è quel che chiamiamo ``mente''. Le persone però non riescono a
capirlo, pensano a se stessi girando in tondo. «~Che cos'è la mente, che
cos'è il cervello?~» Non confondete le cose in questo modo. Che cos'è
quel che riceve le impressioni? Alcune impressioni piacciono e altre no.
Chi è? C'è qualcuno a cui alcune impressioni piacciono e altre no? Certo
che c'è, ma non potete vederlo. Questo è ciò che chiamiamo mente.

Nella nostra pratica non è necessario parlare di
\emph{samatha} o di \emph{vipassanā}. Basta dire
pratica del Dhamma, è sufficiente, e poi effettuare questa pratica con
la vostra mente. Che cos'è la mente? La mente è ciò che riceve, oppure
che è consapevole delle impressioni sensoriali. Con alcune impressioni
sensoriali c'è una reazione di piacere, con altre la reazione è di
dispiacere. Il recettore delle impressioni ci conduce alla felicità e
alla sofferenza, a quel che è giusto e a quel che è sbagliato. Però, non
ha alcuna forma. Noi supponiamo che sia un sé, ma in realtà è solo
\emph{nāma-dhamma}.\footnote{\emph{Nāma-dhamma:} Fenomeno mentale.} La
``bontà'' ha una qualche forma? E il male? La felicità e la sofferenza
hanno una forma? Non potete trovarla. Sono rotonde o quadrate, corte o
lunghe? Riuscite a vederle? Queste cose sono \emph{nāma-dhamma}, non
possono essere confrontate alle cose materiali, ma sappiamo che
esistono. Per questo si dice di cominciare la pratica calmando la mente.
Mettere consapevolezza nella mente. Se la mente è consapevole, sarà in
pace. Alcuni non cercano la consapevolezza, vogliono solo la pace, una
specie di annullamento. E così non imparano mai nulla. Se non abbiamo
``Colui che Conosce'', su cosa fondiamo la nostra pratica?

Se non c'è il lungo, non c'è il corto, se non c'è il giusto, non può
esserci quello che è sbagliato. Di questi tempi la gente va a studiare
lontano, alla ricerca del bene e del male. Però non sanno nulla di ciò
che è al di là del bene e del male. Tutto quello che sanno è ciò che è
giusto e ciò che è sbagliato: «~Prenderò solo quello che è giusto. Non
voglio sapere nulla di quello che è sbagliato. Perché dovrei?~» Se
cercate di prendere solamente quello che è giusto, in poco tempo andrete
verso quello che è sbagliato. Il giusto conduce verso lo sbagliato. La
gente continua a cercare tra quello che è giusto e quello che è
sbagliato, non cerca di trovare ciò che non è né giusto né sbagliato.
Studiano il bene e il male, cercano la virtù, ma non sanno nulla di ciò
che è al di là del bene e del male. Studiano il lungo e il corto, ma non
sanno nulla di quel che non è né lungo né corto.

Il coltello ha una lama, un bordo opposto alla lama e un manico. Potete
prendere solo la lama? Potete prendere solo il bordo, oppure solo il
manico? Il manico, il bordo e la lama sono tutte parti dello stesso
coltello. Quando prendete il coltello, avete tutte e tre le parti
insieme. Allo stesso modo, se prendete quel che è bene, a esso deve
seguire il male. La gente cerca la bontà e tenta di gettare via la
malvagità, ma non studia quello che non è né bene né male. Se non
studiate questo, non c'è completezza. Se prendete la bontà, seguirà la
malvagità. Se prendete la felicità, seguirà la sofferenza. La pratica di
attaccarsi alla bontà e di rifiutare il male è il Dhamma dei bambini, è
come un giocattolo. Certo, va tutto bene, potete anche farlo, ma se vi
aggrappate alla bontà, seguirà il male. La fine di questo sentiero è
confusa, non va poi così bene.

Un semplice esempio. Supponiamo che abbiate dei figli e che vogliate
solo amarli, senza mai provare odio. Così pensa chi non conosce la
natura umana. Se vi attaccate all'amore, seguirà l'odio. Allo stesso
modo, la gente decide di studiare il Dhamma per sviluppare la saggezza,
studiano il bene e il male più da vicino che possono. Ora, dopo aver
conosciuto il bene e il male, che cosa fanno? Cercano di attaccarsi al
bene, ed ecco che a esso segue il male. Non hanno studiato ciò che è al
di là del bene e del male. Questo dovreste studiare.

«~Voglio essere in questo modo.~» «~Voglio essere in quell'altro modo.~»
Però, non dicono mai: «~Non voglio essere niente, perché in verità non
c'è alcun ``io''.~» Questo non lo studiano. Tutto quello che vogliono è
la bontà. Se ottengono la bontà, si perdono in essa. Se le cose vanno
troppo bene, loro inizieranno ad andare male, e così la gente finisce
per oscillare da una parte all'altra in questo modo. Per calmare la
mente e per diventare consapevoli del recettore delle impressioni
sensoriali, dobbiamo osservarla. Seguite ``Colui che Conosce''.
Addestrate la mente finché essa diventa pura. Quanto pura dovreste
renderla? Se è davvero pura, dovrebbe essere al di là sia del bene sia
del male, perfino al di là della purezza. È finita. Ecco quando la
pratica è finita.

Quel che la gente chiama sedersi in meditazione è solo un tipo di pace
temporaneo. Però, anche in questo genere di pace ci sono delle
esperienze. Se sorge un'esperienza, ci deve essere qualcuno che la
conosce, che ci guarda dentro, che la indaga e la esamina. Se la mente è
solo vuota, non si tratta di una cosa poi molto utile. Potreste vedere
delle persone che sembrano molto controllate e pensare che siano serene,
ma la vera pace non è solo una mente serena. La vera pace non è quella
che dice: «~Che io possa essere felice e non sperimentare mai alcuna
sofferenza.~» Con questo genere di pace, col passare del tempo perfino
la felicità diventa insoddisfacente. Ne risulta la sofferenza. Solo
quando con la vostra mente potrete andare al di là sia della felicità
sia della sofferenza troverete la vera pace. Quella è vera pace. Questo
è un argomento che la gente non studia mai, non lo comprende mai
davvero.

Il giusto modo di addestrare la mente è renderla luminosa, sviluppare la
saggezza. Non pensiate che addestrare la mente consista solo nello stare
seduti in silenzio. È solo una pietra che copre l'erba. La gente se ne
ubriaca. Pensa che il \emph{samādhi} consista nello stare seduti. Stare
seduti è una delle definizioni per \emph{samādhi}. In realtà, se la
mente ha \emph{samādhi}, allora camminare è \emph{samādhi}, stare seduti
è \emph{samādhi}, c'è \emph{samādhi} nella postura seduta, nella postura
camminata, nelle postura in piedi e in quella distesa. È tutta pratica.
Alcuni si lamentano: «~Non riesco a meditare, sono troppo irrequieto.
Tutte le volte che mi siedo, penso a questo e a quello. Non riesco a
farlo. Ho troppo cattivo kamma, dovrei prima esaurirlo e poi
cercare di tornare a meditare.~» Certo, provateci. Cercate di esaurire
il vostro cattivo kamma! Così pensa la gente.

Perché pensa in questo modo? Questi cosiddetti impedimenti sono le cose
che dobbiamo studiare. Tutte le volte che sediamo, la mente scappa
subito via. La seguiamo e cerchiamo di riportarla indietro e di
osservarla di nuovo, ma scappa via di nuovo. Questo è ciò che dovreste
studiare. La maggior parte delle persone rifiuta di imparare quanto
viene loro insegnato dalla natura, come fa un pessimo scolaro che
rifiuta di fare i compiti. Non vogliono vedere la mente che cambia. Come
potete sviluppare la saggezza? Dobbiamo viverci con questi cambiamenti.
Quando sappiamo che la mente è semplicemente così, che cambia in
continuazione, quando sappiamo che questa è la sua natura, allora la
comprenderemo. Dobbiamo sapere quando la mente sta pensando bene e
quando sta pensando male, cambia sempre. Dobbiamo conoscerle queste
cose. Se comprendiamo questo, allora possiamo essere in pace perfino
mentre stiamo pensando.

Supponiamo che a casa abbiate una scimmia come animale da compagnia. Le
scimmie non restano ferme a lungo, a loro piace saltare qui e là, e
afferrare le cose. Le scimmie sono fatte così. Ora siete in monastero e
vedete una scimmia. Neanche questa scimmia sta ferma, anche lei salta
qui e là nello stesso modo. Però la cosa non vi infastidisce, vero?
Perché non vi infastidisce? Perché in precedenza avete già allevato una
scimmia, sapete come sono. Se conoscete anche una sola scimmia, non
importa in quanti luoghi andiate, non importa quante scimmie vediate,
non sarete infastiditi dalle scimmie, o no? Così è uno che capisce le
scimmie. Se comprendiamo le scimmie, non diventeremo una scimmia. Se non
comprendete le scimmie, potreste diventare voi stessi una scimmia.
Capite? Quando la vedete che tocca questo e quello, urlate: «~Ehi!~» Vi
arrabbiate. «~Maledetta scimmia!~» Così è uno che non conosce le
scimmie. Uno che conosce le scimmie capisce che la scimmia che ha a casa
e quella che sta in monastero sono uguali. Perché dovreste irritarvi?
Quando capite come sono le scimmie, questo è sufficiente per essere in
pace.

Così è la pace. Dobbiamo conoscere le sensazioni. Alcune sensazioni sono
piacevoli, altre sono spiacevoli, ma questo non importa. Svolgono solo
il loro compito. Proprio come la scimmia, tutte le scimmie sono uguali.
Comprendiamo che le sensazioni a volte sono gradevoli, altre volte no: è
solo nella loro natura. Dovremmo capirle e sapere come lasciarle andare.
Le sensazioni sono incerte. Sono transitorie, imperfette e non
sostanziali. Tutto quello che percepiamo è così. Quando gli occhi, gli
orecchi, il naso, la lingua, il corpo e la mente ricevono le sensazioni,
noi le conosciamo, proprio come conosciamo le scimmie. Allora possiamo
essere in pace.

Quando le sensazioni sorgono, le conosciamo. Perché le rincorrete? Le
sensazioni sono incerte. Un momento sono in un modo, quello successivo
in un altro. La loro esistenza dipende dal cambiamento. Il respiro
entra, poi deve uscire. Questo cambiamento ci deve essere. Se cercate
solo di inspirare, riuscite a farlo? Oppure se cercate solo di espirare
senza poi inspirare di nuovo, potete farlo? Se non ci fosse questo
genere di cambiamento, per quanto tempo potreste vivere? Ci devono
essere sia l'inspirazione sia l'espirazione. Lo stesso vale per le
sensazioni. Sono cose che devono esserci. Se non ci fossero le
sensazioni, non potreste sviluppare la saggezza. Se non c'è quello che è
sbagliato, non ci può essere quello che è giusto. Dovete essere nel
giusto prima di poter capire quello che è sbagliato, e dovete
comprendere quello che è sbagliato prima di essere nel giusto. Così
stanno le cose.

Per chi studia il Dhamma in modo davvero coscienzioso, più sensazioni ci
sono meglio è. Però, molti meditanti evitano le sensazioni, non vogliono
affrontarle. Sono come un pessimo scolaro, che non vuole andare a
scuola, che non vuole ascoltare il maestro. Queste sensazioni ci
insegnano. Quando conosciamo le sensazioni, allora stiamo praticando il
Dhamma. La pace dentro le sensazioni è proprio come capire le scimmie
che stanno qui. Quando capite come sono le scimmie, non vi turbano più.
Così è la pratica del Dhamma. Non è che il Dhamma sia molto lontano, è
proprio qui con noi. Il Dhamma non ha a che fare con gli angeli nei
cieli, non ha a che fare con nulla di tutto questo. Riguarda
semplicemente noi, quel che stiamo facendo proprio ora. Osservate voi
stessi. A volte c'è felicità, a volte sofferenza, a volte benessere, a
volte dolore, a volte amore, a volte odio. Questo è Dhamma. Lo capite?
Dovreste conoscere questo Dhamma, dovete leggere le vostre esperienze.

Dovete conoscere le sensazioni prima di poterle lasciar andare. Quando
vedrete che le sensazioni sono impermanenti, esse non vi daranno più
problemi. Appena sorge una sensazione, dite solo a voi stessi: «~Hmm,
non è una cosa sicura.~» Quando il vostro umore cambia: «~Hmm, non è
sicuro.~» Potete essere in pace con queste cose, proprio come vedere una
scimmia senza esserne disturbati. Se conoscete la verità delle
sensazioni, questo è conoscere il Dhamma. Lasciate andare le sensazioni,
vedendo che sono tutte quante invariabilmente incerte. In ciò che
chiamiamo incertezza, proprio lì sta il Buddha. Il Buddha è il Dhamma.
Il Dhamma è la caratteristica dell'incertezza. Chiunque veda
l'incertezza delle cose, vede la loro immutabile realtà. Così è il
Dhamma. E questo è il Buddha. Se vedete il Dhamma vedete il Buddha.
Vedendo il Buddha, vedete il Dhamma. Se conoscete \emph{aniccā},
l'impermanenza, lasciate andare le cose senza aggrapparvi.

Potreste dire: «~Ehi! Non rompere il mio bicchiere!~» Riuscite a
impedire che una cosa fragile si rompa? Se non si rompe ora, si romperà
in seguito. Se non la rompete voi, lo farà qualcun altro. Se qualcun
altro non la rompe, lo farà una gallina! Il Buddha dice di accettarlo.
Egli penetrò la verità di queste cose, vide che questo bicchiere è già
rotto. Ogni volta che usate questo bicchiere dovreste pensare che è già
rotto. Lo capite? Questa fu la conoscenza del Buddha. Egli vide il
bicchiere rotto nel bicchiere intatto. Quando sarà giunto il momento, si
romperà. Sviluppate questo tipo di conoscenza. Usate il bicchiere e
prendetevene cura fino a quando, un giorno, vi scivolerà dalle mani.
«~Rotto!~» Non c'è problema. Perché non c'è problema? Perché lo avete
visto rotto ancor prima che si rompesse! Però, la gente di solito dice:
«~Questo bicchiere mi piace così tanto che spero che non si rompa mai.~»
Poco dopo è il cane a romperlo. «~Lo ucciderò quel cane maledetto!~»
Odiate il cane perché ha rotto il vostro bicchiere. Se è uno dei vostri
figli a romperlo, odierete anche lui. Perché? Perché dentro di voi avete
costruito una diga, l'acqua non può scorrere. Avete costruito una diga
senza una via di sfogo. L'unica cosa che la diga può fare è crollare,
vero? Quando costruite una diga dovete dotarla anche di una via di
sfogo. Così, quando l'acqua sale troppo, può scorrere via senza rischi.
Quando è colma fino all'orlo aprite la via di sfogo. Dovete avere una
valvola di sicurezza come questa. L'impermanenza è la valvola di
sicurezza degli Esseri Nobili. Se avete questa ``valvola di sicurezza''
sarete in pace.

Praticate costantemente, in piedi, camminando, seduti, distesi,
utilizzando \emph{sati} per sorvegliare la mente e proteggerla. Questo è
\emph{samādhi} e saggezza. Sono entrambi la stessa cosa, ma si
presentano sotto aspetti differenti. Se vediamo davvero chiaramente
l'incertezza, vedremo che essa è certa. La certezza è che le cose devono
inevitabilmente essere così, non possono essere altrimenti. Capite?
Sapendo anche solo questo, potete conoscere il Buddha, potete rendergli
omaggio in modo corretto. Se non eliminerete il Buddha che è in voi, non
soffrirete. Se lo eliminerete, la sofferenza la sperimenterete subito.
Si soffre appena si eliminano le riflessioni sulla transitorietà,
sull'imperfezione e sulla non sostanzialità. Se riuscite a praticare
solo questo, è sufficiente. La sofferenza non sorgerà o, se sorgerà, la
potrete calmare facilmente, e ciò farà sì che la sofferenza non sorga in
futuro. Questa è la fine della nostra pratica, il punto in cui la
sofferenza non sorge. Perché la sofferenza non sorge? Perché siamo
venuti a capo della causa della sofferenza, \emph{samudaya}.

Se questo bicchiere dovesse ad esempio rompersi, provereste sofferenza.
Sappiamo che questo bicchiere sarà causa di sofferenza, e così ci
liberiamo dalla causa. Tutti i dhamma sorgono in ragione di una
causa. Cessano anche in ragione di una causa. Perciò, se c'è sofferenza
a causa di questo bicchiere, dovremmo lasciar andare questa causa. Se
pensiamo in anticipo che questo bicchiere è già rotto anche quando non
lo è, la causa cessa. Quando non c'è più alcuna causa, la sofferenza non
è più in grado di esistere: cessa. Questa è la cessazione.

Non c'è bisogno di andare oltre, solo questo è abbastanza. Contemplatelo
nella vostra mente. Essenzialmente, a fondamento del vostro
comportamento dovrebbero esserci i Cinque Precetti. Non è necessario
studiare il \emph{Tipiṭaka},\footnote{\emph{Tipiṭaka:} Il Canone
  buddhista in pāli.} prima concentratevi solo sui Cinque Precetti.
All'inizio sbaglierete. Quando lo capite, fermatevi, tornate indietro e
assumete di nuovo i precetti. Forse andrete fuori strada e farete un
altro errore. Quando lo capite, riprendete il controllo di voi stessi.
Praticando in questo modo, la vostra \emph{sati} migliorerà e diverrà
più costante, proprio come l'acqua che esce da un bricco. Se incliniamo
il bricco solo un po', ne usciranno lentamente delle gocce: plic! \ldots{}
plic! \ldots{} plic! \ldots{} Se incliniamo il bricco un po' di più, le gocce
usciranno più velocemente: plic, plic, plic! \ldots{} Se incliniamo il bricco
ancora di più, i ``plic'' spariranno e l'acqua fluirà in modo costante.
I ``plic'' dove sono andati? Non sono andati da nessuna parte, sono
diventati un flusso costante d'acqua.

Dobbiamo parlare del Dhamma in questa maniera, usando delle
similitudini, perché il Dhamma è privo di forma. È quadrato o rotondo?
Non è possibile dirlo. L'unico modo per parlarne è utilizzare
similitudini come questa. Non pensiate che il Dhamma sia lontano da voi.
Sta proprio con voi, tutt'intorno a voi. Osservate. Un momento siete
felici, il momento successivo tristi, quello successivo ancora
arrabbiati. È tutto Dhamma. Guardatelo e comprendete. Qualsiasi cosa vi
causi sofferenza, dovreste porvi rimedio. Se la sofferenza c'è ancora,
se ancora non vedete con chiarezza, guardate di nuovo. Se riusciste a
vedere con chiarezza non dovreste soffrire, perché la causa della
sofferenza non dovrebbe essere più là. Se la sofferenza c'è ancora, se
dovete ancora sopportare, allora non siete ancora sulla strada giusta.
Ovunque restiate bloccati, tutte le volte che soffrite troppo, è proprio
lì che sta l'errore. Tutte le volte che siete così felici da librarvi in
volo fra le nuvole, lì siete di nuovo in errore!

Se praticate in questo modo, avrete sempre \emph{sati}, in tutte le
posture. Con \emph{sati} e \emph{sampajañña}, conoscerete quel che è
giusto e quel che è sbagliato, la felicità e la sofferenza. Conoscendo
queste cose, saprete come affrontarle. È così che insegno la
meditazione. Quando è il momento di sedere in meditazione, fatelo, non è
sbagliato. Dovreste praticare anche la meditazione seduta. Però, la
meditazione non è solo stare seduti. Dovreste consentire alla vostra
mente di avere piena esperienza delle cose, dovreste consentire alle
cose di fluire e valutare la loro natura. Come dovreste valutarle?
Vedetele come transitorie, imperfette e prive di sostanzialità. È tutto
incerto. «~È così bello, devo averlo davvero.~» Non è cosa sicura.
«~Questo non mi piace per niente.~» Proprio allora dovete dire a voi
stessi: «~Non è sicuro!~» Tutto questo è vero? Assolutamente sì, non c'è
possibilità di errore. È che vi limitate a pensare che le cose siano
reali. «~Avrò certamente questa cosa.~» Siete già fuori strada. Non
fatelo. Non importa quanto una cosa vi possa piacere, dovreste pensare
che è incerta.

Alcuni cibi paiono deliziosi, ma dovreste di nuovo pensare che non è una
cosa certa. Può sembrare così sicuro che un cibo sia così delizioso, ma
dovete dire a voi stessi: «~Non è sicuro!~» Se volete verificare se si
tratta di cosa sicura o meno, cercate di mangiare il vostro cibo
preferito tutti i giorni. Ogni giorno, sia ben chiaro. Alla fine vi
lamenterete: «~Non ha più un buon sapore.~» Alla fine penserete: «~In
realtà preferisco un altro tipo di cibo.~» Neanche questa è una cosa
sicura! Dovete consentire alle cose di fluire, proprio come le
inspirazioni e le espirazioni. Ci devono essere sia l'inspirazione sia
l'espirazione, la respirazione si basa sul cambiamento. Tutto dipende
dal cambiamento.

Queste cose stanno con noi, non in qualche altro posto. Se non dubitiamo
più, saremo in pace se stiamo seduti o in piedi, se camminiamo o se
stiamo distesi. Il \emph{samādhi} non è solo stare seduti. Alcuni stanno
seduti fino al torpore. Potrebbero anche essere morti, senza essere in
grado di distinguere il nord dal sud. Non cadete in questo estremo. Se
vi sentite assonnati, camminate, cambiate postura. Sviluppate la
saggezza. Se siete davvero stanchi, riposate. Appena vi svegliate
continuate la pratica, non permettetevi di essere preda del torpore.
Dovete praticare in questo modo. Abbiate raziocinio, saggezza e
circospezione.

Iniziate la pratica con la vostra mente e con il vostro corpo, vedeteli
come impermanenti. È lo stesso per qualsiasi altra cosa. Tenetelo a
mente quando pensate che un cibo sia delizioso. Dovete dire a voi
stessi: «~Non è cosa certa!~» Dovete colpire per primi. Di solito
succede sempre il contrario, vero? Se qualcosa non vi piace, ne
soffrite. È così che le cose ci colpiscono. «~Se a lei piaccio, lei mi
piace.~» E le cose ci colpiscono di nuovo. Con il vostro pugno non
colpite mai il bersaglio! Dovete vederla in questo modo. Tutte le volte
che vi piace qualcosa, dite solo a voi stessi: «~Non è una cosa certa!~»
Dovete andare controcorrente, se davvero volete vedere il Dhamma.
Praticate in tutte le posture, stando seduti, in piedi, camminando,
giacendo. La rabbia la potete sperimentare in tutte le posture, vero?
Potete essere arrabbiati mentre camminate, mentre state seduti, mentre
siete distesi. Potete sperimentare il desiderio in tutte le posture. È
per questo che la nostra pratica deve comprendere tutte le posture. In
piedi, camminando, seduti e distesi. Deve essere costante. Non fate
finta, fatelo davvero.

Quando sedete in meditazione, possono sorgere problemi. Prima che un
incidente si risolva, ne arriva un altro. Tutte le volte che sorgono
queste cose, ditevi solamente: «~Non è sicuro, non è sicuro.~» Colpite
voi per primi, prima che siano le cose ad avere la possibilità di
colpirvi. Questo è il punto. Se sapete che tutto è impermanente,
gradualmente i vostri pensieri si dipaneranno. Quando rifletterete
sull'incertezza di ogni cosa che passa, vedrete che tutto va nello
stesso modo. Ogni volta che sorge una cosa, avete solo bisogno di dire:
«~Eccone un'altra!~»

Avete mai visto l'acqua che scorre? Avete mai visto l'acqua ferma? Se la
vostra mente è serena, deve essere proprio come acqua ferma che scorre.
Avete mai visto acqua ferma che scorre? Ecco! Avete visto solo acqua che
scorre e acqua ferma, non è vero? Però non avete mai visto acqua ferma
che scorre. Proprio lì, dove i vostri pensieri non possono afferrarvi
nemmeno quando sono sereni, potete sviluppare la saggezza. La vostra
mente sarà come acqua che scorre, ma è ferma. È come se fosse ferma,
però sta scorrendo. Per questo posso dire «~acqua ferma che scorre.~» È
qui che può sorgere la saggezza.

