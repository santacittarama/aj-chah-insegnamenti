\chapter{Sommersi dai sensi}

\begin{openingQuote}
  \centering

  Discorso tenuto ai monaci riuniti dopo la recitazione del Pāṭimokkha al
  Wat Pah Pong durante il Ritiro delle Piogge del 1978.
\end{openingQuote}

\emph{Kāmogha}, l'inondazione dei sensi: sommersi dalle immagini, dai
suoni, dagli odori, dai sapori, dalle sensazioni tattili. Siamo sommersi
perché guardiamo solo le cose esteriori, non guardiamo interiormente. La
gente non guarda se stessa, guarda solo gli altri. Può vedere tutti gli
altri, ma non riesce a vedere se stessa. Non è una cosa così difficile a
farsi, ma è solo che la gente non ci prova davvero.

Quando guardate una bella donna, ad esempio. Che cosa vi succede? Appena
vedete il volto, vedete tutto il resto. Lo capite? Guardate solo
all'interno della vostra mente. Cosa eguaglia la vista di una donna?
Appena gli occhi vedono solo un po', la mente vede tutto il resto.
Perché è così veloce? Perché siete sommersi dall'``acqua''. Siete
sommersi, ci pensate, ci fantasticate su, siete bloccati lì. È proprio
come essere schiavi, qualcun altro vi controlla. Quando vi dicono di
sedervi dovete sedervi, quando vi dicono di camminare dovete camminare.
Non potete disobbedire perché siete schiavi. Essere ridotti in schiavitù
dai sensi è la stessa cosa. Non importa con quanto impegno ci possiate
provare, pare proprio che non si riesca a liberarsi. E se aspettate che
gli altri lo facciano per voi, siete davvero nei guai. Dovete liberarvi
da soli.

È per questa ragione che il Buddha disse che la pratica del Dhamma,
trascendere la sofferenza, dipende da noi. Prendete ad esempio il
Nibbāna. Il Buddha era un essere del tutto illuminato, e allora
perché non descrisse dettagliatamente il Nibbāna? Disse solo che
dovremmo praticare e scoprirlo da noi. Perché? Non avrebbe potuto
spiegare com'è il Nibbāna? «~Il Buddha, praticò e sviluppò le
perfezioni nel corso di innumerevoli età del mondo a vantaggio di tutti
gli esseri senzienti, e allora perché Egli non indicò il Nibbāna,
in modo tale che essi fossero in grado di vederlo e di andarci?~» Alcuni
pensano in questo modo. «~Se il Buddha lo avesse veramente conosciuto,
ce lo avrebbe detto. Perché mai avrebbe dovuto nasconderci qualcosa?~»
In realtà è un modo di pensare sbagliato. Non possiamo vedere la Verità
in questa maniera. Per vederla dobbiamo praticare, dobbiamo coltivare la
mente. Il Buddha indicò solo la Via per sviluppare la saggezza, questo è
tutto. Disse che siamo noi a dover praticare. Chiunque pratichi
raggiungerà la meta.

Il Sentiero insegnatoci dal Buddha va contro le nostre abitudini.
Davvero non ci piace essere frugali e contenuti, e così diciamo:
«~Indicaci la Via, indicaci la Via per il Nibbāna, in modo tale
che pure quelli come noi che gradiscono le cose facili possano
andarci.~» Per la saggezza avviene la stessa cosa. Il Buddha non può
mostrare la saggezza, non si tratta di una cosa che possa semplicemente
essere consegnata. Il Buddha può mostrarci la Via per sviluppare la
saggezza, ma svilupparne tanta o solo un po' è una cosa che dipende da
ognuno di noi. I meriti e le virtù accumulate dalle persone sono
ovviamente diversi. Guardate un oggetto materiale, come quei leoni di
legno che stanno qui davanti alla sala. Le persone arrivano, li guardano
e non sembrano essere d'accordo. Uno dice: «~Oh, che belli!~» Un altro
afferma: «~Quanto sono orribili!~» Si tratta di un solo leone che è sia
bello che brutto. Anche solo questo è abbastanza per capire come stanno
le cose. Perciò la realizzazione del Dhamma è a volte lenta e altre
volte veloce. Il Buddha e i suoi discepoli erano tutti uguali per il
fatto che praticarono loro stessi, ma per ricevere consigli e istruzioni
per la pratica dovettero far comunque affidamento su insegnanti.

Quando ascoltiamo il Dhamma, anche se desideriamo ascoltare fino a
quando tutti i nostri dubbi non risultino chiariti, essi non si
chiariranno mai solo ascoltando. Il dubbio non si supera con l'ascolto o
con il pensiero, dobbiamo prima ripulire la nostra mente. Ripulire la
mente significa correggere gli errori della nostra pratica. Non importa
quanto a lungo abbiamo ascoltato il maestro parlare della Verità: solo
ascoltando non potremo conoscere o vedere quella Verità. Anche se
avvenisse, si tratterebbe soltanto di supposizioni e di congetture.
Ovviamente, il semplice ascolto del Dhamma è comunque benefico anche se
non può di per sé condurre alla Realizzazione. Ai tempi del Buddha ci
furono persone che realizzarono il Dhamma -- anche la Realizzazione più
alta, la condizione di \emph{arahant} -- mentre ascoltavano un Suo
discorso. Si trattava però di persone già molto sviluppate, con la mente
che già comprendeva a un certo livello. Prendiamo come esempio un
pallone. Quando in un pallone viene pompata dell'aria, esso si gonfia.
L'aria nel pallone spinge da tutte le parti per uscire, ma non c'è un
foro a consentirlo. Appena un ago punge il pallone, l'aria fuoriesce
facendolo scoppiare. È la stessa cosa. La mente di quei discepoli che
raggiunsero l'Illuminazione ascoltando il Dhamma era così. Fino a quando
non ci fu alcun catalizzatore a causare la reazione, questa
``pressione'' era nella loro interiorità, come nel pallone. La mente non
era ancora libera a causa di una piccolissima cosa che nascondeva la
Verità. Appena ascoltarono il Dhamma venne colpito il punto giusto, e
sorse la saggezza. Compresero immediatamente, lasciarono andare
immediatamente e realizzarono il vero Dhamma. Fu così. La mente si
raddrizzò da sé. Cambiò, si capovolse, passò da un modo di vedere a un
altro. Si potrebbe dire che era lontana, oppure che era vicinissima.

Si tratta di una cosa che dobbiamo fare da noi. Al Buddha fu solo
possibile offrire delle tecniche per sviluppare la saggezza, e
altrettanto avviene con gli insegnanti al giorno d'oggi. Offrono
discorsi di Dhamma, parlano della Verità, ma non possiamo fare nostra
quella Verità. Perché no? C'è una ``patina'' che la oscura. Si potrebbe
dire che siamo sommersi, sommersi dall'acqua. \emph{Kāmogha:}
l'``inondazione'' dei sensi. \emph{Bhavogha:} l'``inondazione'' del
divenire. ``Divenire'' (\emph{bhava})\footnote{La parola thailandese per
  \emph{bhava} (\emph{pop:} \thai{ภพ; ภพชาติ}), era famigliare per chi
  ascoltava; in genere significa ``sfera della rinascita''. L'uso di
  questo termine da parte di Ajahn Chah è piuttosto non convenzionale,
  ed enfatizza un'applicazione più pratica del termine.} significa
``sfera della nascita''. Il desiderio sensoriale nasce dal contatto con
ciò che si vede, con i suoni, gli odori, i sapori, le sensazioni e i
pensieri con cui ci identifichiamo. La mente si attacca saldamente e
resta bloccata nella sensorialità.

Alcuni praticanti s'annoiano, si stancano della pratica e diventano
pigri. Non c'è bisogno di guardare molto lontano, basta osservare come
le persone sembrino incapaci di tenere a mente il Dhamma. Solo se
ricevono un rimprovero non lo dimenticano più. Se vengono rimproverati
all'inizio del Ritiro delle Piogge per una qualsiasi cosa, non lo
dimenticano nemmeno dopo che il Ritiro è terminato. Non lo
dimenticheranno per tutta la vita se il rimprovero è penetrato
abbastanza in profondità. Se però si tratta dell'insegnamento del Buddha
che ci dice di essere moderati, di essere contenuti, di praticare
coscienziosamente, perché la gente queste cose non le prende a cuore?
Perché continua a dimenticarle? Non c'è bisogno di guardare molto
lontano, è sufficiente osservare la nostra pratica qui. Ad esempio quel
che avviene nell'applicazione di regole come quella di non chiacchierare
dopo il pasto mentre si lavano le ciotole! Perfino questo sembra essere
al di là delle capacità della gente. Benché si sappia che chiacchierare
non è particolarmente utile e che lega alla sensorialità, parlare è una
cosa che continua a piacere. Piuttosto in fretta si comincia a non
essere d'accordo gli uni con gli altri e infine si inizia a discutere e
bisticciare. È tutto qui. Non è che si tratti di qualcosa di
particolarmente sottile o raffinato, è piuttosto semplice, ma la gente
non sembra volersi sforzare davvero. Le persone dicono di voler vedere
il Dhamma, ma vogliono vederlo alle loro condizioni, non vogliono
seguire il Sentiero della pratica. Non riescono ad andare più lontano di
così. Tutte queste norme relative alla pratica sono mezzi abili per
penetrare e vedere il Dhamma, ma la gente non pratica di conseguenza.

Dire ``pratica reale'' o ``pratica ardente'' non necessariamente
significa che si debbano impiegare moltissime energie. Basta effondere
un po' di impegno nella mente, fare qualche sforzo con tutte le
sensazioni che sorgono, soprattutto con quelle che sono impregnate di
sensorialità. Sono i nostri nemici. Però la gente non riesce a farlo.
Ogni anno, quando si avvicina la fine del Ritiro delle Piogge, va sempre
peggio. Alcuni monaci non ce la fanno più, hanno raggiunto ``il limite
della sopportazione''. Più ci si avvicina alla fine delle Piogge, peggio
si sentono, la loro pratica è priva di costanza. Ne parlo ogni anno, ma
la gente non riesce a ricordarlo. Fissiamo un certo livello di pratica,
ma il tutto va in rovina in meno di un anno. Si comincia quando il
ritiro è quasi finito, si chiacchiera, si socializza e tutto il resto.
La pratica cade a pezzi. Questa è la tendenza. Chi è veramente
interessato alla pratica dovrebbe pensare alle ragioni per cui tutto ciò
avviene: è perché la gente non vede gli effetti negativi di queste cose.

Quando si viene accolti nel monachesimo buddhista si vive con
semplicità. Qualcuno tuttavia lascia l'abito per andare al fronte, dove
ogni giorno fischiano proiettili: preferiscono che sia così. Vogliono
andarci veramente. Il pericolo li attornia, ma sono pronti ad andare.
Perché non vedono il pericolo? Sono pronti a morire per una fucilata, ma
nessuno vuole morire sviluppando la virtù. È sufficiente capire questo.
È perché sono schiavi, non c'è altra ragione. Basta capire solo questo e
si sa qual è il problema. La gente non vede il pericolo. È proprio
incredibile, vero? Si pensa che la gente riesca a vederlo, ma è che non
può. Se non riesce a vederlo neanche allora, non c'è modo che possa
trovare una via d'uscita. Le persone sono determinate a girare in tondo
nel \emph{saṃsāra}. Così stanno le cose. Anche solo parlando di cose
semplici come questa si può iniziare a capire.

Siccome non riescono a capire, avrebbero molti problemi qualora provaste
a chiedere: «~Perché sei nato?~» Sono sommersi dal mondo dei sensi e dal
mondo del divenire (\emph{bhava}). \emph{Bhava} è la sfera della
nascita, il nostro luogo di nascita. Per dirla semplicemente, gli esseri
nascono da \emph{bhava:} è la condizione preliminare per la nascita.
Ovunque si verifichi nascita, è \emph{bhava}. Supponiamo, ad esempio, di
avere degli alberi da frutta, dei meli ai quali siamo particolarmente
affezionati. Se non riflettiamo con saggezza, essi sono per noi
\emph{bhava}. Perché? Immaginiamo che il nostro frutteto contenga un
centinaio, un migliaio di meli: in realtà non importa che alberi siano,
conta il fatto che li consideriamo come i ``nostri'' alberi da frutta.
Allora siamo in procinto di ``nascere'' come ``verme'' in ognuno di
quegli alberi. Anche se il corpo da essere umano sta ancora lì in una
casa, i ``tentacoli'' sono protesi al di fuori di essa e finiscono
dentro ogni albero, li perforano tutti quanti.

Come facciamo a sapere che è \emph{bhava}? È \emph{bhava} (sfera
dell'esistenza) a causa del nostro attaccamento all'idea che quegli
alberi siano nostri, che quel frutteto sia nostro. Se qualcuno prendesse
un'ascia e abbattesse uno di quegli alberi, il proprietario che sta in
quella casa ``morirebbe'' insieme all'albero. Si infurierebbe, e
sentirebbe di dover andare a sistemare le cose, lottare e forse perfino
uccidere. Quel litigio è la ``nascita''. La ``sfera della nascita'' è il
frutteto al quale ci attacchiamo come nostro. ``Nasciamo'' proprio nel
momento in cui lo consideriamo come se fosse nostro, nasciamo da quel
\emph{bhava}. Anche se ci fossero un migliaio di alberi di mele, se
qualcuno ne abbattesse anche solo uno sarebbe come abbattere il
proprietario. Tutto ciò a cui ci attacchiamo, è proprio lì che nasciamo,
esistiamo proprio lì. Siamo nati appena ``conosciamo''. Questo significa
conoscere per mezzo del non-conoscere. Sappiamo che qualcuno ha
abbattuto uno dei nostri alberi, ma non sappiamo che quegli alberi non
sono realmente nostri. Ciò si chiama ``conoscere per mezzo del
non-conoscere''. Siamo vincolati a nascere dentro quel \emph{bhava}.

\emph{Vaṭṭa},\footnote{\emph{Vaṭṭa:} ``Ciò che gira'', quel che va avanti, o è consueto,
  ossia dovere, servizio, consuetudine.}
la ruota dell'esistenza dei fenomeni condizionati,
funziona in questo modo. La gente s'attacca a \emph{bhava}, dipende da
\emph{bhava}. Se prova affetto per \emph{bhava}, è nascita. E se cade
nella sofferenza a causa di quella stessa cosa, anche quello è nascita.
Tutte le volte che non riusciamo a lasciar andare restiamo bloccati nel
solco del \emph{saṃsāra}, giriamo in tondo come una ruota. Guardate
dentro questa cosa, contemplatela. Tutto ciò a cui ci attacchiamo come
``noi'' o come ``nostro'', quello è il luogo per la nascita. Ci deve
essere un \emph{bhava}, una sfera della nascita, prima che una nascita
possa verificarsi. È per questo motivo che il Buddha disse che qualsiasi
cosa tu abbia, non devi ``averla''. Lasciala lì, ma non renderla tua.
Bisogna comprendere questo ``avere'' e questo ``non avere'', conoscerne
la verità, non dibattersi nella sofferenza.

Il posto nel quale siamo nati: volete tornare lì e rinascere, vero?
Tutti voi monaci e novizi, sapete da dove provenite? Volete tornare lì,
vero? Proprio lì, riflettete su questa cosa. Vi state tutti preparando.
Più si avvicina la fine del ritiro, più iniziate a prepararvi a tornare
lì per rinascere. Si potrebbe davvero pensare che la gente sappia com'è
vivere nella pancia di una persona. Quanto scomodi si sta? Pensateci un
po', anche restare per un giorno intero nella vostra \emph{kuṭī} è
sufficiente per capirlo. Chiudete porta e finestre e già vi sentite
soffocare. Come sarebbe stare nella pancia di una persona per nove o
dieci mesi? Pensateci. La gente non vede la pesantezza delle cose.
Chiedete alle persone perché vivono, o perché sono nate, non ne hanno
alcuna idea. Volete ancora tornare lì dentro? Perché? Dovrebbe essere
ovvio, ma non lo capite. Perché non lo capite? Cos'è che vi blocca, a
cosa vi state aggrappando? Rifletteteci su per bene da soli. È perché
c'è una causa per il divenire e per la nascita. Date solo un'occhiata al
corpo del bimbo conservato nella sala principale.\footnote{Al Wat Pah
  Pong è conservato sotto formalina un feto ben sviluppato per aiutare a
  maturare vivide riflessioni, simili a quelle proposte in questo
  discorso.} C'è qualcuno che ne è turbato? No, nessuno. Un bimbo nella
pancia della madre è proprio come quel corpo. E tuttavia queste cose
volete farle di nuovo, perfino tornare a immergervi di nuovo là dentro.
Perché non riuscite a vedere il pericolo di tutto questo e i vantaggi
della pratica?

Capite? Si tratta di \emph{bhava}. La radice è proprio lì, quello è il
punto nodale. Il Buddha insegnò a contemplare questo punto. La gente ci
pensa, ma non riesce a capire. Si stanno tutti preparando a tornare di
nuovo lì. Sanno che non si sta poi molto comodi lì dentro. Mettere il
collo nel capestro è veramente scomodo, e tuttavia ci vogliono ancora
mettere la testa. Perché non lo capiscono? È a questo proposito che deve
subentrare la saggezza, è questo il punto che dovete contemplare. Quando
parlo in questo modo, la gente dice: «~Se tutti diventassero monaci, il
mondo come farebbe a funzionare?~» Non succederà mai che tutti si
facciano monaci, non preoccupatevi. Il mondo esiste a causa degli esseri
illusi, non è una questione di scarso rilievo.

Diventai novizio all'età di nove anni. Ho cominciato a praticare molto
tempo fa. Però, allora non sapevo con esattezza di cosa si trattasse.
L'ho capito quando ero monaco. Allorché fui monaco divenni estremamente
cauto. I piaceri dei sensi ai quali la gente indulgeva non mi sembravano
così divertenti. In essi vedevo sofferenza. Era come guardare una
deliziosa banana, che io sapevo essere dolcissima ma, nello stesso
tempo, velenosa. Non importava quanto fosse dolce o quanto mi tentasse,
se l'avessi mangiata sarei morto. Riflettevo sempre in questo modo.
Tutte le volte che volevo ``mangiare una banana'' vedevo il ``veleno''
di cui era impregnata e, così, alla fine ero in grado di non
interessarmi a quelle cose. Ora, alla mia età, sono cose che non
rappresentano affatto una tentazione. Alcuni non vedono il ``veleno'',
altri lo vedono, ma vogliono rischiare. «~Se la tua mano ha una ferita
non toccare, il veleno potrebbe penetrarci dentro.~»

Ero solito pensare che ci avrei provato anch'io. Dopo essere stato
monaco per cinque o sei anni, pensai al Buddha. Egli aveva praticato per
cinque o sei anni e aveva terminato il suo compito, ma io ero ancora
interessato alla vita mondana e perciò pensai di tornarvi: «~Forse
dovrei andare a dedicarmi al mondo per un po'. Farei qualche esperienza
e acquisirei alcune conoscenze. Anche il Buddha aveva un figlio, Rāhula.
Sono forse troppo rigido?~» Mi sono messo a sedere e per un po' di tempo
ci ho pensato su e alla fine ho capito: «~Si, bene, tutto questo è
ottimo, ma temo solo che questo ``Buddha'' non somigli all'altro.~» Una
voce dentro di me disse: «~Temo che questo ``Buddha'', a differenza
dell'altro, sprofonderà nel fango.~» Perciò opposi resistenza a questi
pensieri mondani. Dal mio sesto o settimo Ritiro delle Piogge fino al
ventesimo ho dovuto combattere davvero. Oggi mi sembra di aver finito le
munizioni, ho sparato a lungo. Ho solo paura che voi, giovani monaci e
novizi, abbiate ancora così tanti proiettili da poter desiderare di
andare a provare i vostri fucili. Prima di farlo, pensateci
accuratamente.

È dura rinunciare ai piaceri dei sensi. È davvero difficile vedere le
cose per quello che sono. Pensate ai piaceri dei sensi come se mangiaste
della carne e ve ne rimanesse un pezzetto tra i denti. Prima di
terminare il pasto dovete trovare uno stuzzicadenti per toglierlo.
Quando togliete il pezzetto di carne, per un po' vi sentite sollevati e,
forse, pensate che non mangerete più carne. Però, quando la vedete di
nuovo non riuscite a resistere. Ne mangiate ancora, ed ecco che un altro
pezzetto vi resta tra i denti. Così dovete toglierlo di nuovo, ciò vi
arreca ancora una volta un po' di sollievo, finché mangiate altra carne
ancora. Quando vi resta tra i denti c'è disagio. Prendete uno
stuzzicadenti, la togliete e sperimentate un po' di sollievo. Non c'è
molto di più nel desiderio dei sensi. La pressione cresce sempre di più
finché non ne lasciate uscire un po'. Oh! Questo è tutto. Non riesco a
capire la ragione di tutto questo interesse. Queste cose non le ho
imparate da nessun altro, sono capitate a me durante la pratica. Sedevo
in meditazione e pensavo al piacere dei sensi come a un nido di formiche
rosse. Qualcuno prende un bastoncino di legno e lo spinge nel formicaio
fino a che le formiche corrono giù lungo il bastoncino, raggiungono il
suo volto e gli pungono gli occhi e gli orecchi.\footnote{Le fomiche
  rosse e le loro uova sono cibo nel nord-est della Thailandia e per
  questa ragione comportamenti di tal genere nei riguardi dei loro nidi
  non sono così insoliti.} Ciò nonostante, non si capisce ancora quale
sia il problema.

Ovviamente, non si tratta di una cosa che va al di là delle nostre
capacità. Nell'insegnamento del Buddha si dice che se si vede il
pericolo insito in una cosa, non importa quanto buona essa possa
sembrare, si sa che è dannosa. Tutto ciò che non consideriamo dannoso,
pensiamo solo che sia buono. Se non abbiamo ancora visto il pericolo che
si cela in una cosa, non possiamo liberarcene. Lo avete notato? Per
quanto possa essere una cosa sporca, alla gente piace. Questo genere di
``lavoro'' non è pulito, ma la gente lo fa anche senza essere pagata, si
offrono volontari con piacere. Altri lavori sporchi la gente non vuole
farli nemmeno se ben pagata, ma a questo tipo di lavoro si sottopone
volentieri, non c'è bisogno di pagarla. Non è che sia un lavoro pulito,
è sporco. Allora perché alla gente piace? Come si può dire che la gente
è intelligente quando si comporta in questo modo? Pensateci.

Avete mai fatto caso ai cani che stanno nel territorio del monastero? Ce
ne sono branchi interi. Corrono qui e là mordendosi a vicenda, succede
che alcuni vengano pure mutilati. Entro un mese o giù di lì succederà di
nuovo. Appena uno dei più piccoli entra nel branco, i più grandi lo
aggrediscono e questo scappa via, trascinandosi dietro una zampa. Quando
però il branco corre via, lui lo segue zoppicando. È piccolo, ma pensa
che un giorno avrà un'occasione. Un morso alla zampa, e questo è tutto
quel che riesce a ricavare dai suoi sforzi. Per tutto il periodo degli
accoppiamenti non avrà una sola possibilità. Lo potete vedere voi stessi
qui in monastero. Quando questi cani corrono ululando in branco, penso
che al loro posto gli esseri umani canterebbero delle canzoni! Pensano
che sia così divertente cantare, ma non hanno la più pallida idea di
cosa li spinga a farlo, si limitano a seguire ciecamente i loro istinti.

Pensateci attentamente. Se davvero volete praticare, dovreste
comprendere le vostre sensazioni. Ad esempio, con chi dovreste
socializzare? Con i monaci, con i novizi o con i laici? Se vi
accompagnate con persone che parlano molto, esse vi indurranno a loro
volta a parlare molto. Quel che fate voi è già abbastanza, loro fanno
ancor di più: mettete le due cose insieme ed esploderanno! Alla gente
piace socializzare con quelli che chiacchierano molto e parlano di cose
frivole. Possono stare lì seduti ad ascoltare per ore. Quando si tratta
di ascoltare il Dhamma, di parlare della pratica, non è che si voglia
ascoltare così tanto. Come quando si offre un discorso di Dhamma. Appena
comincio con \emph{Namo Tassa Bhagavato}, già dormono. Il discorso non
lo sentono affatto. Quando arrivo all'\emph{Evaṃ},\footnote{Sono le
  parole iniziali dell'omaggio al Buddha in lingua pāli, recitate prima
  di iniziare un discorso formale di Dhamma. \emph{Evaṃ} è la parola
  tradizionale in pāli per la conclusione.} aprono gli occhi e si
svegliano. Tutte le volte che c'è un discorso di Dhamma, la gente si
addormenta. Come possono trarne beneficio? Dopo un discorso i veri
praticanti del Dhamma se ne vanno ispirati e sollevati, hanno imparato
qualcosa. Al fine di incoraggiare la pratica, dopo sei o sette giorni
l'insegnante tiene un altro discorso.

Questa è la vostra occasione, ora che avete ricevuto l'ordinazione
monastica. C'è solo questa opportunità, perciò osservate con attenzione.
Osservate le cose e pensate a quale via scegliere. Ora siete
indipendenti. Dove andrete? Vi trovate a un bivio: la via del mondo o la
Via del Dhamma. Quale via sceglierete? Potete scegliere una o l'altra, è
il momento di decidere. Siete voi a dover scegliere. Se il vostro
destino è quello di essere liberi, è ora che dovete cominciare.

