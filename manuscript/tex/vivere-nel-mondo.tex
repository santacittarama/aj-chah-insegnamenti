\chapter{Vivere nel mondo}

La maggior parte delle persone non conosce ancora l'essenza della
pratica di meditazione. Pensa che la pratica sia costituita dalla
meditazione camminata, dalla meditazione seduta e dall'ascolto dei
discorsi di Dhamma. Sono solo le forme esteriori della pratica. La vera
pratica ha luogo quando la mente incontra un oggetto sensoriale. Qui ha
luogo la pratica, quando avviene il contatto con i sensi. Quando la
gente dice cose che non ci piacciono c'è risentimento, se dice cose che
ci piacciono sperimentiamo piacere. Qui ha luogo la pratica. Con queste
cose come si pratica? È il punto cruciale. Se ce ne andiamo sempre in
giro a caccia di felicità e fuggendo via dalla sofferenza, potremo
praticare fino al giorno della morte senza mai vedere il Dhamma. È
inutile. Quando sorgono il piacere e il dolore, come useremo il Dhamma
per liberarci da essi? Questo è il punto nodale della pratica.

Di solito, quando la gente incontra cose sgradevoli non si apre a esse.
Quando ad esempio viene criticata dice: «~Non infastidirmi! Perché mi
biasimi?» Sono persone che si chiudono in se stesse. Proprio quello è il
punto in cui praticare. Quando la gente ci critica, dovremmo ascoltare.
Stanno dicendo la verità? Dovremmo essere aperti e prendere in
considerazione quel che stanno dicendo. Forse c'è qualcosa di vero in
ciò che dicono, forse c'è qualcosa di riprovevole in noi. Forse hanno
ragione, ma ci offendiamo subito. Se la gente evidenzia i nostri
difetti, dovremmo sforzarci di eliminarli e di migliorare noi stessi. Le
persone intelligenti praticano in questo modo.

Ove c'è confusione, proprio lì può sorgere la pace. Quando la confusione
è penetrata dalla comprensione, quel che resta è la pace. Alcune persone
non accettano di essere criticate, sono arroganti. Si voltano, e
cominciano a discutere. Succede soprattutto quando gli adulti hanno a
che fare con i bambini. In realtà i bambini possono dire cose
intelligenti, ma se capita che siate la loro madre, non potete cedere.
Se siete insegnanti, i vostri studenti possono talvolta dirvi una cosa
che non sapete, ma siccome siete gli insegnanti, non potete ascoltarli.
Questo non è pensare in modo retto.

Ai tempi del Buddha, un discepolo fu davvero astuto. Una volta il
Buddha, mentre stava esponendo il Dhamma, si rivolse a questo monaco e
chiese: «~Sāriputta, credi a ciò che ho detto?~» Il venerabile Sāriputta
rispose: «~No, non ci credo ancora.~» Il Buddha lodò la sua risposta:
«~Molto bene, Sāriputta, tu sei uno che è dotato di saggezza. Un saggio
non crede subito, prima ascolta con mente aperta e poi, prima di credere
o di non credere, valuta la verità della questione.~»

Il Buddha ha dato un buon esempio agli insegnanti. Sāriputta disse la
verità, egli aveva espresso solo le sue vere sensazioni. Alcuni possono
pensare che se si dice di non credere a quell'insegnamento, ciò equivale
a mettere in questione l'autorità dell'insegnante e, perciò, temono di
dire una cosa del genere. Vanno semplicemente avanti dicendo che sono
d'accordo. Il mondo procede in questo modo. Il Buddha però non si
offese. Disse che non bisogna vergognarsi di queste cose, che esse non
sono né sbagliate né cattive. Se non credete, non è sbagliato dire che
non credete. Questa è la ragione per cui Sāriputta disse: «~No, non ci
credo ancora.~» Il Buddha lo lodò: «~Questo monaco ha molta saggezza. Ci
pensa accuratamente, prima di credere a qualcosa.~» Il comportamento del
Buddha è un buon esempio per chi insegna. A volte si può imparare anche
dai bambini; non attaccatevi a posizioni di autorevolezza.

Sia che stiate in piedi o seduti, o che stiate camminando in posti di
vario genere, potete sempre esaminare le cose intorno a voi. Le
esaminiamo in modo naturale, siamo recettivi nei riguardi di tutto, che
si tratti di oggetti visivi, suoni, odori, sapori, sensazioni o
pensieri. La persona saggia li prende tutti in considerazione. Nella
vera pratica arriviamo al punto in cui non ci sono più problemi che
pesano sulla mente.

Se ancora non sappiamo quando piacere e dispiacere sorgono, nelle nostre
menti ci sono ancora delle perplessità. Se conosciamo la verità delle
cose, riflettiamo: «~Oh, non c'è nulla in questa sensazione di piacere.
È solo una sensazione che sorge e svanisce. Il dispiacere non è nulla di
più, solo una sensazione che sorge e svanisce. Perché trasformarli in
qualcosa?~» Se pensiamo che piacere e dolore siano dei possessi
personali, allora avremo problemi, non andremo mai oltre questa o quella
preoccupazione, al di là di questa catena senza fine. Così stanno le
cose per la maggior parte delle persone.

Gli insegnanti oggigiorno quando parlano di Dhamma non parlano spesso
della mente, non parlano della Verità. Se parlate della Verità, le
persone prendono le distanze. Dicono cose come: «~Non parla in modo
circostanziato, non sa parlare bene.~» Però, la Verità dovrebbero
ascoltarla. Un vero insegnante non usa solo la memoria per parlare, dice
la Verità. Nella società la gente si avvale in genere della memoria,
l'insegnante dice la Verità. Nella società la gente si avvale in genere
della memoria, e quello che di solito dice serve per esaltare se stessa.
Il vero monaco non parla in questo modo, dice la Verità, il modo in cui
sono le cose.

Non conta quanto l'insegnante spieghi la Verità, per la gente è
difficile capirla. È difficile comprendere il Dhamma. Se comprendete il
Dhamma, dovreste praticare di conseguenza. Non è necessario diventare
monaci, anche se la vita monastica è la forma ideale per la pratica. Per
praticare davvero, dovete abbandonare la confusione del mondo,
rinunciare alla famiglia e ai possessi, e andare nella foresta. È il
posto ideale per praticare. Però, se ancora abbiamo una famiglia e delle
responsabilità, come praticare? Alcuni dicono che è impossibile
praticare il Dhamma per i laici. Pensate: chi sono di più, i monaci o i
laici? Ci sono molti più laici. Ora, se solo i monaci praticassero e i
laici no, questo significa che ci sarebbe moltissima confusione. Questa
è errata comprensione. «~Non posso diventare monaco.~» Il punto non è
diventare monaci! Essere un monaco non significa nulla, se non si
pratica. Se davvero comprendete la pratica del Dhamma, non conta il
ruolo che rivestite o la professione che svolgete, insegnante, dottore,
funzionario statale o qualsiasi altra, potete praticare il Dhamma in
ogni momento della vostra vita.

Pensare di non poter praticare perché si è laici significa smarrire del
tutto il Sentiero. Com'è che la gente trova la spinta per fare altre
cose? Se sentono che manca loro qualcosa, fanno uno sforzo per
ottenerlo. Se vi è abbastanza desiderio, la gente può fare di tutto.
«~Non ho avuto tempo per praticare il Dhamma~», dicono alcuni. Io
rispondo: «~Allora com'è che hai avuto il tempo per respirare?~»
Respirare è vitale per l'esistenza delle persone. Se pensassero che per
la loro esistenza è vitale la pratica del Dhamma, la considererebbero
importante tanto quanto respirare.

Per praticare il Dhamma non è necessario correre di qua e di là o
esaurire le proprie energie. Basta guardare le sensazioni che sorgono
nella mente. Quando l'occhio vede delle forme, l'orecchio ode dei suoni,
il naso sente degli odori e così via, tutto raggiunge quest'unica mente,
``Colui che Conosce''. Ora, che succede quando la mente percepisce
queste cose? Se l'oggetto che sperimentiamo ci aggrada proviamo piacere,
se non ci piace proviamo dispiacere. Questo è tutto quel che accade.

Dove pensate di andare per trovare la felicità nel mondo? Vi aspettate
che tutti vi dicano per tutta la vita solo cose piacevoli? È possibile?
No, non lo è. Se non è possibile, dove andrete allora? Il mondo è
semplicemente così, dobbiamo conoscere il mondo -- \emph{lokavidū} --
conoscere la verità di questo mondo. Il mondo è una cosa che dovremmo
capire con chiarezza. Il Buddha visse in questo mondo, non visse in
qualche altro posto. Sperimentò la vita familiare, ma ne vide i limiti e
se ne distaccò. Ora, voi laici come potete praticare? Se volete
praticare, dovete fare uno sforzo per seguire il Sentiero. Se
perseverate con la pratica, vedrete i limiti di questo mondo e sarete in
grado di lasciar andare.

A volte, la gente dedita agli alcolici dice: «~Non riesco a smettere.~»
Perché non riesce a smettere? Perché non riesce ancora a vederne gli
svantaggi. Se ne vedesse con chiarezza gli svantaggi, non sarebbe
costretta ad attendere che qualcuno dica loro di smettere. Se non vedete
gli svantaggi di qualcosa, significa pure che non riuscite a vedere i
benefici di smettere. La vostra pratica diventa priva di frutti, state
solo giocando a praticare. Se vedete con chiarezza gli svantaggi e i
benefici di una cosa, non dovrete aspettare che qualcun altro ve ne
parli.

Riflettete sulla storia del pescatore che trova qualcosa nella sua
nassa. Sa che dentro c'è qualcosa, sente che qualcosa si agita in essa.
Pensando che si tratti di un pesce, mette una mano nella nassa, ma può
solo constatare che si tratta di un altro animale. Non riesce ancora a
vederlo, e così non riesce a prendere una decisione. Potrebbe essere
un'anguilla,\footnote{In alcune zone della Thailandia le anguille sono
  ritenute una prelibatezza.} ma anche un serpente. Se però continua a
stringere e quel che tiene con la mano fosse un serpente, il serpente
potrebbe voltarsi e morderlo. È in preda al dubbio. Il suo desiderio è
così forte che continua a stringere, perché potrebbe trattarsi di
un'anguilla. Però, appena lo tira fuori e vede che ha la pelle a
strisce, lo getta subito via. Non deve aspettare che qualcuno gli dica:
« È un serpente, è un serpente, lascia andare!~» Vedere il serpente gli
dice che cosa bisogna fare con molta più chiarezza di quanto potrebbero
fare le parole. Perché? Perché vede il pericolo, i serpenti possono
mordere! Nessuno deve dirglielo. Allo stesso modo, se pratichiamo fino a
che vediamo com'è la realtà, non andremo a immischiarci in cose dannose.

Di solito la gente non pratica in questo modo, in genere fa altrimenti.
Non contempla le cose, non riflette sulla vecchiaia, sulla malattia e
sulla morte. Le persone parlano solo di non invecchiare e di non morire,
e così non sviluppano mai la sensibilità giusta per la pratica del
Dhamma. Vanno a sentire i discorsi di Dhamma, ma in realtà non li
ascoltano. Ogni tanto sono invitato a tenere discorsi di Dhamma in
occasioni importanti, ma per me andarci è una seccatura. Come mai?
Perché quando guardo la gente lì riunita, noto che non sono venuti per
ascoltare il Dhamma. Alcuni odorano di alcol, altri fumano sigarette,
altri ancora chiacchierano. Non sembrano proprio persone venute per la
loro fiducia nel Dhamma. Tenere discorsi in questi posti è poco
fruttuoso. Chi sprofonda nella negligenza tende a pensare cose di questo
genere: «~Quando la smetterà di parlare? Non si può fare questo, non si
può fare quest'altro~...~» La loro mente vaga ovunque.

Talvolta m'invitano a tenere un discorso solo per ragioni formali: «~Per
cortesia, venerabile, ci offra solo un breve discorso di Dhamma.~» Non
vogliono che parli troppo, potrei annoiarli! Non appena sento queste
parole, già so che cosa vogliono intendere. Questa gente non vuole
ascoltare il Dhamma. Il Dhamma li annoia. Se offro un breve discorso,
non comprendono. Se prendete solo un po' di cibo, è sufficiente? No,
naturalmente.

A volte sto solo introducendo l'argomento ma, mentre parlo, un ubriaco
tenta di mandarmi via gridando: «~Bene! Fate largo, fate largo al
venerabile, sta per andarsene!~» Quando incontro questo genere di
persone ho molto su cui riflettere, guadagno in termini di comprensione
della natura umana. È come se qualcuno avesse una bottiglia piena
d'acqua e ne chiedesse ancora. Non c'è posto in cui metterla. Non vale
la pena investire tempo ed energia per insegnare a queste persone,
perché la loro mente è già piena. Versateci dentro una sola cosa in più
e, senza che essa possa essere di una qualche utilità, traboccherà. Se
la bottiglia fosse vuota, l'acqua potrebbe essere versata da qualche
parte, e ne riceverebbe beneficio sia chi la offre sia chi la riceve.

Quando le persone sono veramente interessate al Dhamma, siedono
tranquillamente e ascoltano con attenzione, e io mi sento più ispirato a
insegnare. Se la gente non presta attenzione è come l'uomo con la
bottiglia piena d'acqua, non c'è più spazio. È a malapena il caso di
parlare con loro. In queste situazioni, non nasce in me alcuna energia
per insegnare. Non potete mettere molta energia nel dare, se nessuno
mette molta energia nel ricevere.

Di questi tempi è così che tendono ad andare le cose quando si offrono
discorsi di Dhamma, e va sempre peggio. La gente non è alla ricerca
della Verità, studia solo per procurarsi il sapere necessario a
guadagnarsi da vivere, formarsi una famiglia e prendersi cura di se
stessi. Studia per sostentarsi. Anche un po' di Dhamma, certo, ma non
troppo. Oggi gli studenti sanno di più degli studenti del passato. Hanno
tutto a loro disposizione, tutto è più a portata di mano. Però, sono
anche molto più confusi e soffrono più di prima. Perché? Perché cercano
solo quella conoscenza che consente di guadagnarsi da vivere.

Perfino i monaci sono così. A volte li sento dire: «~Non sono diventato
monaco per praticare il Dhamma, ho voluto ricevere l'ordinazione
monastica solo per studiare.~» Queste sono le parole di chi ha del tutto
tagliato fuori il Sentiero della pratica. Non c'è alcuna via da seguire,
solo un vicolo cieco. Quando questi monaci insegnano, lo fanno solo
usando la memoria. Se insegnano una cosa, la loro mente è in un posto
completamente diverso. In questi insegnamenti non c'è la Verità. Il
mondo è fatto così. Se cercate di vivere semplicemente, di praticare il
Dhamma e di vivere serenamente, dicono che siete strani e asociali.
Dicono che state ostacolando il progresso della società. Possono anche
ricorrere a intimidazioni. Alla fine può anche succedere che iniziate a
credere a quel che vi dicono e che torniate alle vie del mondo,
sprofondando sempre più in esso fino a quando vi è impossibile uscirne.
Certa gente dice: «~Ormai non posso venirne fuori, ci sono troppo
dentro.~» Così tende a essere la società. Non apprezza il valore del
Dhamma.

Nei libri non si può trovare il valore del Dhamma. I libri sono solo
manifestazioni esteriori del Dhamma, non sono comprensioni del Dhamma in
termini di esperienza personale. Se comprendete il Dhamma, lo
comprendete nella vostra stessa mente, la Verità la vedete lì. Quando la
Verità diventa manifesta, interrompe la corrente dell'illusione.

L'insegnamento del Buddha è Verità immutabile, nel presente e in
qualsiasi altra epoca. Il Buddha rivelò questa Verità 2500 anni fa e, da
allora, è sempre stata la Verità. Niente deve essere aggiunto o tolto.
Il Buddha disse: «~Quel che il \emph{Tathāgata} ha stabilito, non
dovrebbe essere eliminato; quel che il \emph{Tathāgata} non ha
stabilito, non dovrebbe essere aggiunto agli insegnamenti.~» Egli
sigillò gli insegnamenti. Perché il Buddha li sigillò? Perché questi
insegnamenti sono le parole di un Essere privo di contaminazioni. Non
importa quanto il mondo possa mutare, questi insegnamenti non ne
risultano condizionati, non mutano con esso. Se qualcosa è sbagliato,
anche se la gente dice che è giusto, ciò non lo rende meno sbagliato. Se
qualcosa è giusto, ciò non cambia solo perché la gente dice che non lo
è. Una generazione può succedere a un'altra, ma queste cose non
cambiano, perché questi insegnamenti sono la Verità.

Chi creò questa Verità? Fu la Verità stessa a creare la Verità! La creò
il Buddha? No, non la creò. Il Buddha ``scoprì'' solamente la Verità, il
modo in cui sono le cose, e poi si ripromise di dichiararla. La Verità è
costantemente vera, che un Buddha sorga nel mondo o meno. Il Buddha
``possiede'' solo in questo senso il Dhamma, non lo ha effettivamente
creato. Il Dhamma è stato sempre qui. Prima nessuno ha però cercato e
trovato ``Ciò che non muore'' né lo ha poi insegnato come Dhamma. Il
Buddha non lo ha inventato, era già lì.

Con il passare del tempo a un certo punto la Verità viene illuminata e
la pratica del Dhamma fiorisce. Man mano che il tempo trascorre e
passano le generazioni, la pratica degenera fino a che non svanisce
completamente. Dopo un certo periodo, l'insegnamento viene nuovamente
rifondato e fiorisce ancora una volta. Gli adepti del Dhamma man mano si
moltiplicano, la prosperità si afferma e poi, di nuovo, l'insegnamento
comincia a seguire l'oscurità del mondo. E così degenera ancora una
volta, fino a che perde del tutto terreno. Regna di nuovo la confusione.
Giunge allora il tempo di ristabilire la Verità. Nei fatti, la Verità
non va da nessuna parte. Quando i Buddha muoiono, il Dhamma non scompare
con loro.

Così gira il mondo. È come un albero di mango. L'albero giunge a
maturazione, fiorisce, e i frutti appaiono e crescono fino a diventare
maturi. In seguito marciscono e il seme torna nella terra, per poi
diventare un nuovo albero di mango. Il ciclo ricomincia ancora una
volta. Ci possono essere più frutti maturi che cadono, marciscono,
affondano nella terra in forma di semi e diventano di nuovo alberi. Il
mondo è così. Non va molto lontano, gira sempre attorno alle solite
vecchie cose.

La nostra vita, ai nostri giorni, è la stessa cosa. Oggi stiamo
semplicemente facendo le stesse vecchie cose che abbiamo sempre fatto.
La gente pensa troppo. Ci sono così tante cose alle quali interessarsi,
ma nessuna di esse conduce alla Realizzazione. Ci sono scienze come la
matematica, la fisica, la psicologia e così via. Potete addentrarvi in
ognuna di esse, ma solo con la Verità potete arrivare a una conclusione.

Supponete che ci sia un carro trainato da un bue. Finché il bue traina
il carro, i solchi delle ruote lo seguono. Le ruote sono rotonde, ma le
tracce lunghe; le tracce sono lunghe, ma le ruote sono solo dei cerchi.
Guardando il carro mentre sta fermo, non potete vedere nulla di lungo
ma, appena il bue inizia a muoversi, vedete i solchi che si allungano
dietro di voi. Finché il bue si muove, le ruote continuano a girare, ma
arriva il giorno in cui il bue si stanca e si sbarazza del suo giogo. Il
bue se ne va e lascia lì il carro vuoto. Le ruote non girano più. Dopo
un po' il carro cade a pezzi e ciò che lo compone torna a essere terra,
acqua, fuoco e vento, i quattro elementi.

Se cercate la pace nel mondo, i solchi delle ruote del carro si
allungheranno senza fine dietro di voi. Finché seguite il mondo non c'è
sosta, non c'è riposo. Se solo smettete di seguirlo, il carro si ferma e
le ruote non girano più. Seguendo il mondo, le ruote girano
incessantemente. È così che si crea cattivo \emph{kamma}. Finché seguite
le vecchie abitudini, non c'è modo di fermarsi. Se vi fermate, allora
tutto si ferma. È così che pratichiamo il Dhamma.

