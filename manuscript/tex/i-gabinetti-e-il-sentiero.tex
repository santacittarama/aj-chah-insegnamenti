\chapter{I gabinetti e il Sentiero}

\section{Introduzione}

\emph{(di Ajahn Jayasāro)}

Questo discorso,
originariamente offerto in laotiano, è stato tradotto in thailandese per
la biografia di Luang Por Chah, \emph{Upalamani}. È un discorso molto
intenso che ho integralmente accolto con particolare favore nella
biografia in thailandese e, solo in parte, in questa traduzione degli
\emph{Insegnamenti} perché nulla di simile era fruibile nelle lingue
occidentali. La maggior parte del lavoro di raccolta dei discorsi di
Ajahn Chah è stato focalizzato sugli insegnamenti riguardanti la
meditazione e la saggezza: nella vita quotidiana al Wat Nong Pah Pong
questo tipo di discorsi di Dhamma era in realtà piuttosto raro e, quando
veniva offerto, era molto apprezzato. Le quotidiane istruzioni e la
maggior parte dei discorsi riguardavano infatti quello che noi chiamiamo
\emph{korwat}, regolamentazioni monastiche che sottolineano il lato
della pratica concernente \emph{sīla}.

Questo è in parte dovuto al fatto che i monasteri della Tradizione della
Foresta e soprattutto i monasteri di Ajahn Chah di venti anni fa
avevano, in ragione del gran numero di novizi, una composizione e una
natura differenti rispetto a oggi. I novizi adolescenti di allora
tendevano a essere piuttosto energici e chiassosi e, come ben si può
immaginare, ciò influiva in modo piuttosto significativo sull'atmosfera
del monastero. Gli abati avevano il problema di cercare di amministrare
una comunità nella quale circa la metà di coloro che la componevano non
erano poi così interessati a essere dei \emph{bhikkhu}. I monaci della
mia generazione conoscono numerose storie relative a novizi
disubbidienti, difficili, indisciplinati e antipatici. Benché al Wat Pah
Pong la percentuale dei novizi fosse un po' più bassa, essi comunque
esercitavano un influsso insieme ai monaci che ricevevano un'ordinazione
solo temporanea e che potevano perdere tempo senza in realtà sapere il
motivo per cui si trovavano lì, in quanto la loro ordinazione altro non
era che un gesto per mostrare gratitudine ai propri genitori.

Appena giunsi al Wat Pah Pong rimasi sorpreso, perché mi aspettavo che
fosse un luogo di addestramento, un monastero davvero inflessibile.
Certamente lo era, ma fui sorpreso dal numero di monaci e di novizi che
non sembrava apprezzare quel che succedeva, e non si impegnava poi molto
nell'addestramento impartito da Ajahn Chah. Questo significava che molti
degli insegnamenti, invece di essere raffinati discorsi sulla natura del
\emph{samādhi} e dei \emph{jhāna} e così via, sottolineavano piuttosto
il \emph{korwat patipadā}.\footnote{La pratica delle osservanze
  monastiche.} Nei monasteri -- che si trattasse del Wat Pah Pong o di
un monastero affiliato -- vi era in una tempesta di \emph{desana}, ossia
di discorsi di Dhamma, che travolgeva e scuoteva le persone. In seguito,
per qualche giorno tutto procedeva in modo particolarmente rigoroso.
Poi, gradualmente, le cose si deterioravano di nuovo, finché si
verificavano uno o due eventi davvero gravi, e si sapeva che ci sarebbe
stato un altro di questi impetuosi \emph{desana}. Ognuno sarebbe così
nuovamente stato di sostegno per se stesso. Uno schema, questo, che si
ripeteva in continuazione.

Fu Ajahn Chah a offrire i più incisivi e migliori discorsi di tal
genere. Questo ha una forza notevole. È da rimarcare il fatto che non
sia stato pronunciato agli inizi della sua attività, allorché aveva
quaranta o cinquant'anni e ancora molta forza e vigore. Fu invece
pronunciato verso la fine del suo magistero, quando tra i monaci
occidentali ricorreva un'immagine nonnesca dell'\emph{ajahn}: si
trattava, però, di una grande banalizzazione. Le fotografie che si
vedono nei libri e che lo ritraggono sorridente e gentile raffigurano
certo Ajahn Chah, ma la storia non sta tutta qui. Ritengo che questo
discorso trasmetta una buona immagine di tutto ciò. È molto difficile
rendere il tono di uno di questi discorsi. Nei discorsi di Dhamma c'è il
contenuto di quel che viene detto, ma ci sono anche tutta una serie di
segni non verbali, come pure l'intero contesto delle relazioni tra
l'insegnante e i suoi studenti. Sono cose, queste, che non possono
essere riprodotte nel testo a stampa. Quando chi non ha mai vissuto in
un monastero della Tradizione della Foresta con un \emph{kruba
ajahn}\footnote{\emph{Kruba ajahn} (thailandese: \thai{ครูบาอาจารย์}): Si
  tratta di un appellativo per monaci importanti della Tradizione
  Thailandese della Foresta.} sente uno di questi discorsi, gli sembra
quasi di ascoltare delle parole di un bullo insolente, eccessive e sopra
le righe. Perciò si deve davvero fare lo sforzo di mettersi nei panni di
chi vive in un monastero della Tradizione della Foresta quando le cose
iniziano a degenerare un po', e arriva il momento che il maestro rimetta
in riga le persone.

A parte il progetto di Ajahn Liem\footnote{È l'attuale abate del Wat Pah
  Pong, subentrato ad Ajahn Chah dopo la sua morte per volere dello
  stesso Ajahn Chah.} di un'area coperta per la tintura delle vesti
monastiche, non c'è molto lavoro da fare. Quando sarà finita, lavare e
tingere gli abiti sarà più comodo. Quando lui va a lavorare, mi
piacerebbe che tutti andassero a dargli una mano. Quando la nuova
tettoia per la tintura sarà finita, non ci sarà molto altro da fare.
Sarà il momento di tornare alla nostra pratica delle osservanze
monastiche, ai fondamenti del regime monastico. Praticate queste
osservanze come si deve. Se non lo farete, sarà un vero e proprio
disastro. Di questi giorni la pratica delle osservanze connesse agli
alloggi, \emph{senāsana-vatta}, è davvero pessima. Comincio a dubitare
sul fatto che conosciate il significato di queste parole,
\emph{senāsana-vatta}. Non fate finta di non vedere le condizioni in cui
si trovano le \emph{kuṭī} nelle quali vivete e dei gabinetti che
utilizzate. Laici di Bangkok, di Ayutthaya, di tutta la nazione mettono
a disposizione dei fondi per le nostre necessità. Alcuni inviano per
posta del denaro per la cucina del monastero. Siamo monaci, pensateci.
Non venite in monastero per diventare ancor più egoisti di quanto lo
eravate quando vivevate nel mondo: sarebbe una disgrazia. Riflettete con
attenzione sulle cose che usate quotidianamente: i quattro generi di
prima necessità, ossia l'abito, il cibo in elemosina, una dimora e le
medicine. Se non prestate attenzione a come usate questi generi di prima
necessità, non vi comportate da monaci.

La situazione relativa alle dimore è particolarmente grave. Le
\emph{kuṭī} sono in condizioni pessime. È difficile dire in quali di
esse vivano dei monaci e quali non siano abitate. Ci sono termiti che si
arrampicano sui pali di cemento e nessuno fa nulla. È proprio una
disgrazia. Appena tornato sono andato a fare un giro d'ispezione: è
stato straziante. Mi dispiace per i laici che hanno costruito queste
\emph{kuṭī} affinché ci viviate. Tutto quello che volete fare è
andarvene in giro con le vostre ciotole e i vostri \emph{glot} in spalla
alla ricerca di posti in cui meditare, non avete idea di come prendervi
cura delle \emph{kuṭī} e delle proprietà del Saṅgha. È scioccante.
Abbiate un po' di rispetto per i sentimenti dei donatori. Durante
l'ispezione ho visto pezzi di stoffa che sono stati usati nelle
\emph{kuṭī} e poi gettati via benché ancora in buone condizioni. C'erano
delle sputacchiere utilizzate e poi non correttamente riposte. Qualcuno
aveva pure urinato in alcune senza poi svuotarle. È disgustoso, nemmeno
i laici lo fanno. Se voi praticanti di Dhamma non siete neanche in grado
di gestire delle sputacchiere, che speranza avete nella vita?

La gente porta tazze per WC nuove di zecca in offerta. Non so se le
abbiate mai pulite o no, ma ci sono topi e gechi che vanno a cacarci.
Topi, gechi e scimmie, tutti quanti insieme le usano. I gechi non
passano mai la ramazza e neanche le scimmie lo fanno. Siete allo stesso
loro livello. L'ignoranza non può scusare una cosa del genere. Tutto
quello che usate in questa vita è di supporto alla pratica. Il
venerabile Sāriputta teneva il posto nel quale viveva immacolatamente
pulito. Se da qualche parte vedeva della sporcizia, prendeva una scopa e
puliva. Se succedeva durante la questua, usava un piede. Il luogo in cui
un vero monaco praticante vive è diverso da quello di una persona
ordinaria. Se la vostra \emph{kuṭī} è un disastro totale, lo sarà pure
la vostra mente.

Questo è un monastero della Tradizione della Foresta. Durante la
stagione delle piogge al suolo cadono rami e foglie. Al pomeriggio,
prima di spazzare, accatastate i rami secchi o trascinateli nella
foresta. Spazzate per bene i bordi dei sentieri. Se siete sciatti e
spazzate solo in modo superficiale, le \emph{kuṭī} e i sentieri andranno
completamente in rovina. Tempo addietro ho fatto realizzare dei sentieri
per la meditazione camminata separati dai sentieri che conducevano alle
\emph{kuṭī}. Ogni \emph{kuṭī} aveva il proprio sentiero. Dalle
\emph{kuṭī} ognuno usciva sul proprio sentiero, eccetto coloro che
abitavano più lontano, dietro. Ognuno andava e tornava dalla propria
\emph{kuṭī} su un sentiero del quale si prendeva cura. Le \emph{kuṭī}
erano pulite e ordinate. Oggi non è così. Vi invito a fare una camminata
fino all'estremità superiore della terra del monastero per vedere i
risultati del lavoro che ho fatto in una \emph{kuṭī} e nell'area a essa
circostante. È un esempio.

Per quanto concerne le modifiche alle \emph{kuṭī,} non lavorate troppo
per cambiare cose che non necessitano di essere riparate. Si tratta di
dimore del Saṅgha, che il Saṅgha vi ha assegnato. Non è giusto apportare
modifiche sulla base della vostra fantasia. Dovreste prima chiedere il
permesso o consultarvi con un monaco anziano. Alcuni non capiscono cosa
ciò implichi e si sopravvalutano. Pensano che si tratti di una
miglioria, ma quando si arriva al dunque realizzano una cosa brutta e
inopportuna. Alcuni sono solo degli ignoranti. Prendono il martello e
iniziano a piantare chiodi nel legno delle pareti, e prima di rendersene
conto hanno distrutto il muro. Non so chi sia stato, perché a fatto
compiuto i colpevoli sono scappati via. Se qualcuno entra, quel che si
vede ha un aspetto orribile.

Prendete in considerazione con cura il nesso tra la pratica di
meditazione e una dimora pulita, ordinata e piacevole. Se nella vostra
mente c'è solo desiderio o avversione, cercate di concentrarvi proprio
su questo, è su questo che dovete perfezionarvi, meditateci, eliminate
le contaminazioni proprio dove esse si trovano. Sapete che cosa ha un
aspetto piacevole e che cosa non lo ha? Se cercate di fingere di non
saperlo, è una vergogna, siete in difficoltà. Le cose peggioreranno di
giorno in giorno. Pensate alla gente che da ogni provincia del paese
viene a vedere questo monastero.

La dimora di un praticante di Dhamma non è grande, è piccola ma pulita.
Se un Essere Nobile abita in pianura, quel posto diviene fresco e
gradevole. Quando se ne va a vivere sulle alture, quelle alture
diventano fresche e gradevoli. Perché dovrebbe essere così? Ascoltate
bene. Perché il suo cuore è puro. Non segue la sua mente, segue il
Dhamma. È sempre consapevole del suo stato mentale. Però, arrivare a
questo livello è difficile. Quando è il momento di spazzare, vi dico di
spazzare verso l'interno, verso il centro del sentiero, ma voi non lo
fate. Devo starmene lì in piedi e gridare: «~Verso l'interno! Verso
l'interno!~» O forse non lo fate perché non sapete cosa significa
``verso l'interno''? Forse non lo sapete. Forse siete così da quando
eravate piccoli. Mi sono inventato parecchie spiegazioni. Quando ero
bambino passavo davanti alle abitazioni della gente e spesso sentivo i
genitori che dicevano ai loro figli di cacare ben lontani da casa. Non
lo faceva mai nessuno. Appena si trovavano poco distanti da casa, ecco
fatto. Poi quando c'era cattivo odore, tutti si lamentavano. È la stessa
cosa.

Alcuni non si rendono proprio conto di quello che fanno, non portano le
cose fino a una conclusione. Che si tratti di questo o di quello, sanno
cosa c'è bisogno di fare, ma sono troppo pigri per farlo. Lo stesso
avviene con la meditazione. Alcuni non sanno cosa fare, e appena lo
spieghi lo fanno subito bene. Altri, però, anche dopo che lo hai
spiegato continuano a non farlo: è che hanno deciso di non farlo.
Prendete in seria considerazione che cosa significhi per un monaco
addestrare la mente. Distinguetevi dai monaci e dai novizi che non
praticano, siate diversi dai laici. Andate e riflettete su quello che
significa. Non è così facile come mi sembra che pensiate. Fate domande
sulla meditazione, sulla mente tranquilla e sul Sentiero che conduce al
\emph{Nibbāna}, però non riuscite a tenere pulito il sentiero fino al
gabinetto e alla vostra \emph{kuṭī}. È una cosa davvero oscena. Se
continuate così le cose si deterioreranno costantemente. Le osservanze
stabilite dal Buddha a riguardo delle dimore dicono di tenerle pulite.
Un gabinetto è compreso tra i \emph{senāsanā}. È infatti considerato una
piccolissima \emph{kuṭī}, e non dovrebbe essere lasciato sporco e
trasandato. Seguite l'ingiunzione del Buddha e rendetelo un luogo
piacevole da usare, così che da qualsiasi punto di vista lo si guardi
non offenda gli occhi.

Ehi! Quel novizio giovane, laggiù. Perché stai già sbadigliando? È
mattino presto. A quest'ora di solito dormi o che altro? La testa ti
ciondola di qua e di là come se stessi per morire. Che c'è che non va?
Quando devi ascoltare un discorso, sei intontito. Ho notato che non ti
senti così quando è ora di mangiare. Se non fai attenzione, quale
beneficio potrai mai ottenere dall'essere qui? Come riuscirai a
migliorare te stesso?

Chi non pratica è solo un peso per il monastero. Quando vive con
l'insegnante è solo un peso per l'insegnante, rende le cose difficili e
gli appesantisce il cuore. Se volete stare qui fatelo per bene. Oppure
pensate che essere monaci significhi stare a perdere tempo? Portate le
cose fino al limite, scavate fino a quando raggiungete la roccia. Se non
praticate, le cose non andranno meglio da sole. La gente manda soldi per
la cucina da tutta la nazione per pensare alle vostre necessità, e voi
che fate? Lasciate i gabinetti sporchi e non spazzate le vostre
\emph{kuṭī}. Cos'è questa storia? Riponete le cose, prendetevi cura di
esse. Urinate nelle sputacchiere e le lasciate dove le avete usate. Se
avete una zanzariera che non vi piace, la gettate via. Se i laici lo
vedessero, si sentirebbero sfiduciati: «~Per quanto si sia poveri, quali
che siano i nostri stenti, riusciamo a comprare un po' di stoffa da
offrire ai monaci. Ma loro vivono come dei re. Stoffe davvero in ottimo
stato, senza un solo strappo, sono sparse ovunque, gettate via.~»
Perderebbero tutta la loro fede.

Non c'è bisogno di impartire discorsi di Dhamma e di esporre gli
insegnamenti. Quando i laici arrivano e vedono un monastero bello e
curato, sanno che i monaci che stanno qui sono diligenti e conoscono le
osservanze monastiche. Non c'è bisogno di lusingarli o di fare chissà
che cosa. Quando vedono le \emph{kuṭī} e i gabinetti, sanno che tipo di
monaci vive qui. Tenere le cose pulite è un modo di proclamare il
buddhismo.

Quando ero un giovane novizio al Wat Ban Gor fu costruito un
\emph{vihāra}\footnote{\emph{Vihāra}: Un'abitazione, un luogo in cui
  dimorare. Di solito si riferisce al luogo in cui dimorano i monaci,
  ossia un monastero.} e furono acquistate più di cento sputacchiere.
Per l'annuale \emph{Phra Vessandara Ngan}\footnote{Come si dice nel
  testo, il \emph{Phra Vessandara Ngan} (in thailandese \thai{พระเวสสันดร}) è
  una festa che ogni anno commemora l'ultima vita del \emph{bodhisatta},
  un termine utilizzato per descrivere il Buddha prima che diventasse il
  Buddha, dalla prima aspirazione alla buddhità, anche nelle vite
  precedenti, fino al pieno Risveglio.} c'erano molti monaci in visita e
le sputacchiere erano state usate come recipienti per il succo di betel.
Questo \emph{Ngan} è una festa per accumulare meriti e commemorare
l'ultima vita del \emph{bodhisatta}: quando tutto era finito le
sputacchiere sporche erano state accantonate ovunque, nei posti più
bizzarri della sala. Un centinaio di sputacchiere, ognuna di esse piena
di succo di betel, nessuna era stata svuotata. Quando mi sono imbattuto
in queste sputacchiere ho pensato: «~Se questa non è una cosa terribile,
allora nulla lo è.~» Le avevano riempite con il succo di betel e poi le
avevano lasciate lì, fino all'anno successivo. Poi le tiravano fuori,
grattavano via un po' di incrostazioni giusto per far capire che si
trattava di sputacchiere e ci sputavano di nuovo dentro. Questo è il
tipo di \emph{kamma} che ti fa rinascere all'inferno! Una cosa
assolutamente inammissibile. Monaci e novizi che si comportano in questo
modo non hanno alcuna cognizione di ciò che è bene e di ciò che è male,
di corto e lungo, di giusto e sbagliato. Sono pigri e inetti, ritengono
che siccome sono monaci e novizi possono prendere le cose alla leggera,
e, senza rendersene conto, si trasformano in cani.

Li avete visti? Li avete visti quegli anziani con i capelli grigi che vi
rendono omaggio quando alzano i loro contenitori di bambù per mettere il
riso nelle vostre ciotole? Quando vengono qui a offrire il cibo si
prostrano e poi si prostrano ancora. Date un'occhiata a voi stessi. È
quello che mi ha indotto a lasciare il monastero del villaggio: gli
anziani che vengono a offrire del cibo e che si prostrano continuamente.
Me ne stavo seduto a ripensarci. Cosa c'è di così buono in me che induce
la gente a prostrarsi così tanto? Ovunque vada, la gente solleva le mani
in \emph{añjali}. Perché? Che cos'è che mi rende degno di tutto questo?
Quando ci pensavo mi vergognavo, mi vergognavo di affrontare i miei
sostenitori laici. Non era giusto. Se non pensate a questo e non fate
qualcosa ora, quando lo farete? Avete una buona opportunità e non la
state cogliendo. Guardate dentro tutto questo, se non mi credete.
Pensateci su veramente.

In altri miei insegnamenti ho già menzionato Chao Khun Nor del Wat
Tepsirin. Durante il regno di re Vajiravudh era un paggio del sovrano.
Quando nel 1925 il re morì, si fece monaco. L'unica volta che lasciò la
sua \emph{kuṭī} fu per una riunione formale del Saṅgha. Non scendeva al
piano di sotto neanche per ricevere gli ospiti laici.\footnote{Molto
  spesso in Thailandia le \emph{kuṭī} sono capanne a palafitta poste
  all'altezza di un piano dal suolo.} Viveva nella sua \emph{kuṭī}
insieme a una bara. Durante tutta la sua vita da monaco non andò mai in
\emph{tudong}. Non ne aveva bisogno, era irremovibile. Voi andate in
\emph{tudong} fino a quando vi vengono le vesciche ai piedi. Andate su
per le montagne e poi scendete verso il mare e quando ci arrivate non
sapete in quale altro luogo andare. Andate alla cieca in cerca del
\emph{Nibbāna} con la mente confusa, infilando il naso ovunque possiate,
e in qualsiasi posto andiate vi lasciate alle spalle gabinetti sporchi.
Siete troppo presi a cercare il \emph{Nibbāna} per pulirli. Siete ciechi
o che altro? Lo trovo incredibile.

Per arrivare all'Illuminazione e al \emph{Nibbāna} sono necessarie molte
altre cose. La prima è prendersi cura per bene dei luoghi in cui si
dimora. È necessario costringervi a farlo, o che altro? Se non foste
davvero ostinati e recalcitranti non sarebbe stato necessario arrivare
fino a questo punto. In questo momento chi si prende cura delle cose
lavora fino allo sfinimento. Coloro ai quali non potrebbe importare meno
restano indifferenti. Non guardano, non prestano attenzione, non ne
sanno nulla. Che cosa si deve fare con gente come questa?

I problemi che affiorano con i beni di prima necessità quali le dimore,
il cibo in elemosina e l'abito monastico sono come le mosche cavalline.
Puoi scacciarle per qualche tempo, ma dopo che hanno ronzato lì attorno
per un po' tornano a posarsi nello stesso posto. Questi giorni ognuno di
voi sta lasciando l'equivalente di uno o due piatti di avanzi. Non so
perché prendiate una quantità di cibo così grande. Un pezzo di riso
glutinoso è sufficiente a riempire la pancia. Prendete quanto basta.
Invece prendete più di quanto riusciate a mangiare e quel che resta va a
marcire in una fossa. Questi giorni gli avanzi ammontano a una dozzina
di ciotole grandi. È vergognoso che non conosciate la capienza del
vostro stomaco. Prendete quel che riuscite a mangiare. Per quale ragione
prenderne di più? Se gli avanzi di ognuno di voi sono sufficienti per la
colazione di tre o quattro laici, e anche di più, allora è troppo. Com'è
possibile che chi non ha moderazione possa comprendere come addestrare
la propria mente? Quando state praticando la meditazione seduta e la
vostra mente è in subbuglio, dove andrete a prendere la saggezza per
pacificarla? È spaventoso che non siate nemmeno a conoscenza di cose
basilari come la quantità di cibo di cui avete bisogno. Se non conoscete
i vostri limiti, siete come quell'uomo avido di cui si parla in quella
storia, che cercò di portare fuori dalla foresta un tronco talmente
grande da morire sotto il suo peso.

\emph{Bhojanemattaññutā} significa moderazione nell'assunzione di cibo.
\emph{Jāgariyānuyoga} significa sforzarsi senza indulgere nel piacere di
riposare. \emph{Indriyāsamvara} significa contenere gli occhi, gli
orecchi, il naso, la lingua, il corpo e la mente al fine di evitare che
sorgano pensieri di soddisfazione e d'insoddisfazione. Tutte queste
pratiche sono finite fuori dalla finestra. È come se non aveste né
occhi, né orecchi, né bocca, non so di quale genere di spirito famelico
siate il risultato. Non ramazzate le vostre dimore. Se non comprendete
quel che fate, più praticate più degenerate.

Più passa il tempo più diventate golosi. Dovete conoscere i vostri
limiti. Pensate a quella volta, quando stavamo costruendo il
\emph{bòht}\footnote{Tempio per le riunioni formali del Saṅgha (in
  thailandese \thai{โบสถ์}).} e ci portarono del caffé. Ho sentito che qualcuno
si lamentava: «~Oooh! Basta! Basta! Ne ho bevuto così tanto da sentirmi
male.~» Che lo dica un monaco è una cosa assolutamente disgustosa! Berne
così tanto fino ad aver voglia di vomitare. Sette o otto tazze ognuno. A
che cosa stavate pensando? Significa farsi prendere troppo dalle cose.
Pensate di esser diventati monaci per mangiare e bere? Se invece era una
specie di gara, si trattava di una follia. Dopo aver finito, le tazze le
lasciaste lì, allineate in una lunga fila, e altrettanto avvenne con i
bricchi. Nessuno lavò nulla. Solo i cani non rassettano dopo aver
mangiato. Quel che sto dicendo è che se foste stati veri monaci e veri
novizi i bricchi sarebbero stati tutti lavati. Questo comportamento
indica tutte le abitudini malsane che stanno dentro di voi. Chi si
comporta così porta con sé la sua mediocrità ovunque vada.

Vi dico tutto questo affinché sia di nutrimento per le vostre
riflessioni. Osservate davvero il modo in cui state vivendo in questi
giorni. Vedete qualcosa che necessiti di essere migliorato? Se
continuate a essere come ora, i monaci che si dedicano veramente alla
pratica non saranno in grado di sopportarlo. Se ne andranno tutti, e chi
non lo farà non vorrà parlare con voi, e il monastero ne soffrirà.
Quando il Buddha entrò nel \emph{Nibbāna} non portò con sé i modi di
praticare che conosciamo. Li lasciò qui per tutti noi. Non c'è bisogno
di complicare la situazione parlando di cose troppo lontane da noi.
Concentratevi solo su quello che potete vedere qui, sulle cose che
facciamo ogni giorno. Imparate a vivere insieme in armonia e ad aiutarvi
reciprocamente. Sappiate quello che è giusto e quello che è sbagliato.

\emph{Gāravo ca nivāto ca santutthi ca kataññutā.}\footnote{«~Essere
  rispettosi e umili, soddisfatti e riconoscenti.~» È una citazione dal
  \emph{Mangala} \emph{Sutta}, Snp 2.4.} Questo argomento è importante e
deve essere capito. Attualmente le cose sono andate molto al di là
dell'accettabile. Sono l'unico per il quale molti di voi mostrano un po'
di rispetto. Non va bene essere così. E non va bene che abbiate paura di
me. La cosa migliore è venerare il Buddha. Se fate del bene solo perché
temete l'insegnante, allora non c'è speranza. Dovete aver paura degli
errori, riverire il Dhamma insegnato dal Buddha ed essere soggetti al
potere del Dhamma che rappresenta il nostro rifugio. Il Buddha ci ha
insegnato ad accontentarci e ad avere pochi desideri, a essere contenuti
e composti. Non fate il passo più lungo della gamba, osservate quello
che è a portata di mano. I laici pensano che il Saṅgha del Wat Pah Pong
pratichi bene e mandano denaro alla cucina per acquistare del cibo. Lo
date per scontato. A volte sto seduto a rifletterci -- sto criticando i
\emph{bhikkhu} e i novizi che non praticano, non quelli che lo fanno --
mi vergogno quando penso che le cose non stanno come pensano loro. È
come se ci fossero due buoi a tirare un carro. Quello furbo è
imbrigliato più vicino al giogo e lascia che sia l'altro che sta più
avanti a sforzarsi. Il bue vicino al giogo può andare avanti tutto il
giorno senza stancarsi. Può continuare ad andare avanti o può riposarsi,
può fare quello che vuole, perché non sta tirando alcun peso, non sta
impiegando alcuna energia. Con un solo bue a trascinarlo, il carro si
muove lentamente. Il bue che sta dietro si gode il suo sleale vantaggio.

\emph{Supatipanno}: chi pratica bene. \emph{Ujupatipanno}: chi pratica
con integrità. \emph{Ñāyapatipanno}: chi pratica davvero per abbandonare
le contaminazioni. \emph{Sāmīcipatipanno}: chi pratica molto
correttamente. Leggetele spesso queste parole. Descrivono le virtù del
Saṅgha, le virtù dei monaci, le virtù dei novizi, le virtù dei
\emph{pah-kao}, le virtù dei praticanti. Secondo me, se avete lasciato
il mondo per praticare in questo modo avete fatto bene.

Gli abitanti dei villaggi che vengono qui a porgervi omaggio hanno così
tanta fiducia in voi che quando comincia la raccolta del riso novello
non consentono alla famiglia di mangiarlo. La prima parte del raccolto è
messa da parte per il Saṅgha. Quando comincia la stagione dei manghi, ai
bambini non vengono dati quelli grandi. I genitori li fanno maturare e
li conservano per i monaci. Quando ero bambino, mi arrabbiavo con mia
madre e con mio padre per questo motivo. Non capivo la ragione per cui
dovessero avere così tanta fede. Non sapevano cosa succedeva in
monastero. Però, mi capitava spesso di vedere dei novizi che mangiavano
furtivamente alla sera. E se questo non è cattivo \emph{kamma}, di che
cosa si tratta allora? Parlare e comportarsi in più modi malsani e poi
accettare che la gente vi offra del cibo: è un \emph{kamma} che vi
condurrà nel più profondo dei regni infernali. Qual è il bene che può
provenire da questa cosa? Davvero, pensateci sul serio. Attualmente la
vostra pratica è un disastro.

Diffondere il buddhismo non è solo questione di esporre il Dhamma. Si
tratta di ridurre i desideri, accontentarsi, tenere pulite le vostre
dimore. Allora, che cosa sta succedendo? Ogni volta che qualcuno deve
andare in gabinetto è necessario che tenga il naso per aria, rivolto
verso il soffitto. L'odore è talmente cattivo che nessuno osa respirare
normalmente. Che cosa pensate di fare? Non è poi così difficile capire
qual è il vostro problema. Risulta ovvio appena si osservano le
condizioni in cui sono i gabinetti. Provateci. Fate di questo monastero
un buon monastero. Non c'è bisogno di molto. Fate quel che è necessario
fare. Prendetevi cura delle \emph{kuṭī} e dell'area centrale del
monastero. Se lo fate, quando i laici entrano e vedono possono sentirsi
così ispirati da un'emozione religiosa da realizzare il Dhamma proprio
lì, in quello stesso momento. Non provate alcuna benevolenza per loro?
Pensate a quando si entra dentro una montagna, dentro una grotta, a
quell'emozione religiosa che sorge e che fa inclinare in modo naturale
la mente verso il Dhamma. Se la gente entra e vede solo monaci e novizi
che si comportano in modo sciatto, che vivono in \emph{kuṭī} mal tenute
e che usano gabinetti sporchi, da dove può sorgere l'emozione
religiosa?

Quando le persone sagge ascoltano uno che parla, capiscono subito qual è
il punto, basta una sola occhiata. Quando qualcuno comincia a parlare, i
saggi sanno immediatamente se si tratta di un egoista che accumula
contaminazioni, se le sue opinioni sono in contrasto con il Dhamma o con
la Disciplina, oppure se conosce il Dhamma. Se avete già praticato e per
queste cose ci siete passati, sono facili da vedere. Non dovete fare
nulla di originale. Fate solo le cose tradizionali, ravvivate le vecchie
pratiche in declino. Se permettete che si continui a degenerare, tutto
cadrà in pezzi, e non sarete in grado di ripristinare gli antichi
livelli. Siate perciò determinati nella vostra pratica, sia quella
esteriore sia quella interiore. Non siate falsi. Monaci e novizi
dovrebbero vivere in armonia e fare tutto insieme.

Andate in quella \emph{kuṭī} e guardate quello che ho fatto. Ci ho
lavorato per molte settimane. Un monaco, un novizio e un laico mi hanno
aiutato. Andate a guardare. È un lavoro fatto bene? Ha un aspetto
gradevole? Si tratta del modo tradizionale di prendersi cura delle
dimore. Andate a guardare. Dopo aver usato il gabinetto si puliva con
uno spazzolone. Prima non c'era acqua nei gabinetti. I gabinetti che
avevamo non funzionavano bene come quelli che usiamo oggi. Erano però i
monaci e i novizi a funzionare bene, ed eravamo in pochi. Oggi i
gabinetti funzionano bene, ma non la gente che li usa. Sembra che non si
possano avere queste due cose contemporaneamente. Pensateci su davvero.

L'unico problema è che la mancanza di diligenza nella pratica conduce a
un completo disastro. Non conta quanto buono e nobile sia lo scopo, non
lo si può raggiungere se non si capisce qual è il metodo giusto.
Altrimenti è un completo sfacelo. Rammemorate il Buddha e fate in modo
che la vostra mente inclini verso il Dhamma. In esso vedrete il Buddha
stesso. In quale altro luogo potrebbe mai essere? Guardate solo il suo
Dhamma. Leggete gli Insegnamenti. In essi riuscite a trovare qualcosa di
sbagliato? Focalizzate la vostra attenzione sull'insegnamento del Buddha
e lo vedrete. Pensate di poter fare quello che vi pare perché il Buddha
non può vedervi? Che follia! Non state esaminando voi stessi. Se siete
sempre pigri, come pensate di poter praticare? Nulla può essere
paragonato alla scaltrezza delle contaminazioni. Non è facile vederla.
Ovunque sorga la visione profonda, è subito seguita dalla
contaminazione. Non pensiate di poter continuare a perder tempo, a
mangiare e dormire, se nessuno si oppone.

Come potrebbe mai sfuggirvi il Dhamma se vi dedicaste realmente alla
pratica? Non siete né sordomuti né ritardati, siete in possesso di tutte
le vostre facoltà. Che cosa potete aspettarvi, se siete pigri e
distratti? Se foste ancora uguali a quando siete arrivati le cose non
andrebbero così male, temo solo che stiate peggiorando. Riflettete a
fondo su questo. Chiedete a voi stessi: «~Perché sono venuto qui? Qui
che cosa ci sto a fare? Vi siete rasati i capelli, avete indossato
l'abito color ocra. Per quale ragione? Avanti, chiedetevelo. Pensate di
averlo fatto solo per mangiare, dormire ed essere distratti? Se è questo
quello che volete, potete farlo nel mondo. Prendete buoi e bufali,
tornatevene a casa, mangiate e dormite, tutti riescono a farlo. Se in
monastero vi comportate in questo modo sventato e indulgente nei
riguardi delle vostre contaminazioni, non siete degni del nome di monaci
e di novizi.~»

Sollevate il vostro spirito. Non siate assonnati, indolenti e meschini.
Ricominciate a praticare, immediatamente. Sapete quando arriverà la
morte? Anche i giovani novizi possono morire, lo sapete. Non è solo
Luang Por che sta per morire. Anche i \emph{pah-kao}. Tutti stiamo per
morire. Che cosa resterà quando arriverà la morte? Volete scoprirlo? Può
darsi che domani riusciate a fare quel che pensate, ma se vi capita di
morire stanotte? Non conoscete i vostri limiti. Le faccende da sbrigare
servono per imparare a impegnarsi. Non trascurate i doveri del Saṅgha.
Non mancate agli incontri quotidiani. Sostenete sia la vostra pratica
personale sia i vostri doveri nei riguardi della comunità. Potete
praticare sia quando state lavorando sia quando state scrivendo,
innaffiando gli alberi o che altro, perché la pratica è quello che state
facendo. Non credete alle vostre contaminazioni e alla vostra brama.
Hanno già portato alla rovina molta gente. Se credete alle
contaminazioni vi tagliate fuori da ogni bontà. Pensateci. Nel mondo le
persone che si lasciano andare finiscono per assuefarsi a droghe come
l'eroina. Si arriva fino a quel punto, ma la gente non vede il pericolo.

Se praticate con sincerità, il \emph{Nibbāna} vi attende. Non limitatevi
a starvene seduti ad aspettare che venga da voi. Avete mai visto
qualcuno che ci sia riuscito in questo modo? Ovunque vediate di essere
in torto, ponetevi subito rimedio. Se avete fatto una cosa in modo non
corretto, rifatela nel modo giusto. Investigate. Dovete ascoltare, se
volete trovare il bene. Se vi addormentate mentre state ascoltando il
Dhamma, i Guardiani degli Inferi vi afferreranno le braccia e vi
lanceranno nell'Inferno. Proprio all'inizio di un discorso, durante
l'invocazione in pāli, alcuni di voi già cominciano a crollare. Non vi
vergognate? Non vi sentite imbarazzati a sedere in quelle condizioni di
fronte ai laici? E quell'appetito da dove l'avete preso? Siete degli
spiriti famelici o che altro? I cani dopo aver mangiato sono almeno
ancora in grado di abbaiare. Tutto quel che riuscite a fare è starvene
seduti in preda al torpore. Sforzatevi un po'. Non siete soldati di leva
nell'esercito.\footnote{Si intenda: non siete costretti ad ascoltare
  discorsi di Dhamma.} Appena il cappellano inizia a istruirli, la testa
dei soldati inizia a inclinarsi verso il petto: «~Quando la smetterà?~»
Come credete di riuscire a realizzare il Dhamma se pensate come un
soldato di leva?

I cantanti non riescono a cantare bene senza l'accompagnamento di un
flauto. Lo stesso vale per gli insegnanti. Se i discepoli seguono gli
insegnamenti e le istruzioni del loro maestro con tutto il cuore, lui si
sente pieno di energia. Quando però impiega ogni genere di fertilizzanti
ma la terra resta secca e priva di vita, è terribile. Non prova gioia,
perde l'ispirazione, si chiede perché dovrebbe mai preoccuparsi tanto.

Prima di mangiare siate molto cauti. In occasione del \emph{Wan
Phra}\footnote{\emph{Wan Phra} (in thailandese \thai{วันพระ}): Il giorno di
  osservanza lunare; in queste ricorrenze la meditazione si protrae per
  tutta la notte.} e comunque tutte le volte che avete la tendenza a
sentirvi molto assonnati, non date al vostro corpo alcun cibo, lasciate
che sia qualcun altro a mangiare. Dovete reagire. Non mangiate affatto.
«~Se ti comporti male, oggi non mangerai.~» Questo dovete dirgli. Se
lasciate vuoto lo stomaco, la mente può essere davvero serena. È il
Sentiero della pratica. Potete stare là seduti fino al giorno della
morte senza riuscire a distinguere il nord dal sud, intontiti come degli
imbecilli, ma non ne ricaverete nulla, resterete ignoranti come lo siete
ora. Prendete in considerazione attentamente queste cose. Cosa dovete
fare per rendere la vostra pratica una ``buona pratica''? Guardate. C'è
gente che viene da lontano, da altre nazioni, per vedere il modo in cui
pratichiamo qui, vengono qui per ascoltare il Dhamma e per addestrarsi.
La pratica è per loro di beneficio. Il vostro beneficio e il beneficio
degli altri sono interdipendenti. Non si tratta solo di fare le cose per
mostrarle agli altri, ma anche a vostro stesso beneficio. I laici si
sentono ispirati quando vedono che il Saṅgha pratica bene. Che cosa
credete che penserebbero se venissero qui e vedessero che i monaci e i
novizi sono come delle scimmie? In cosa potrebbero mai riporre le loro
speranze?

Per quanto concerne l'esposizione del Dhamma, non dovete fare molto.
Alcuni discepoli del Buddha, come il venerabile Assaji, quasi non
parlavano. Con l'abito sobrio di color ocra, andavano a fare la questua
con calma e serenità, senza camminare né veloci né lenti. Sia che
camminassero sia nei movimenti, che andassero avanti o indietro erano
misurati e composti. Un giorno, quando il venerabile Sāriputta era
ancora discepolo di un brahmano chiamato Sanjaya, intravide il
venerabile Assaji e fu ispirato dal suo comportamento. Chiese al
venerabile Assaji chi fosse il suo maestro e questa fu la risposta che
ricevette: « Il venerabile Gotama.~» «~Che cos'è che insegna e che vi
rende capaci di praticare in questo modo?~» «~Non insegna molte cose.
Dice solo che tutti i \emph{dhamma} sorgono in ragione di cause.
Affinché cessino, sono prima le loro cause a dover cessare.~» Tutto qui.
Era sufficiente. Egli comprese. Questo bastò al venerabile Sāriputta per
realizzare il Dhamma. Molti di voi, invece, quando vanno a fare la
questua sembrano un gruppo di pescatori chiassosi che vanno a pesca. Le
vostre risate e il vostro fare scherzoso si sentono da lontano. La
maggior parte di voi non si rende conto di nulla, sprecate il vostro
tempo pensando a cose inutili e banali.

Tutte le volte che tornate dalla questua potete portare molto Dhamma con
voi, anche stando qui seduti a consumare il vostro pasto. Sorgono molte
sensazioni. Se siete composti e contenuti ne sarete consapevoli. Non c'è
bisogno di sedere in meditazione a gambe incrociate perché queste cose
succedano. Potete ottenere l'Illuminazione nella normale vita
quotidiana. Volete forse discutere a questo proposito? Un pezzo di
carbone ardente non si raffredda subito, appena viene tolto dal fuoco.
In qualsiasi punto lo prendiate, è rovente. La consapevolezza conserva
il suo stato di vigilanza come fa il carbone con il suo calore, la
consapevolezza di sé è sempre presente. Se le cose stanno così, come
potrebbe mai la mente essere preda dell'illusione?

Tenete il vostro sguardo fisso sulla mente. Ciò non significa guardarla
senza nemmeno battere le ciglia come dei matti. Significa monitorare
costantemente le vostre sensazioni. Fatelo molto, concentratevi molto,
sviluppate molto questa attività: questo si chiama progredire. Voi non
sapete che cosa intendo con questo ``tenere lo sguardo fisso sulla
mente'', con questo tipo di sforzo e di sviluppo. Sto parlando di
conoscere lo stato della vostra mente nel presente. Se nella vostra
mente sorgono la brama, il malanimo o qualsiasi altra cosa, allora
dovete sapere ogni cosa in relazione a quegli stati mentali. La mente è
come un bimbo che cammina carponi e il conoscere è come un genitore. Il
bimbo cammina carponi come sanno fare i bambini, e il genitore lo lascia
fare, ma nello stesso tempo lo tiene costantemente d'occhio. Se il bimbo
sta per cadere in una buca, in un pozzo oppure va verso il pericolo
nella giungla, il genitore lo sa. Questo genere di consapevolezza è
chiamata ``Colui che Conosce, Colui che è Chiaramente Consapevole, il
Radioso''.

La mente non addestrata non comprende che cosa stia avvenendo, la sua
consapevolezza somiglia a quella di un bambino. Sapere che nella mente
c'è bramosia e non fare nulla, sapere che vi state approfittando di
qualcun altro, mangiare più del dovuto, sapere come alzare un peso dal
lato più leggero e lasciare che qualcun altro sollevi il lato più
pesante, sapere di aver avuto più di un altro: sono tutti modi malsani
di conoscere. La gente egoista ha questo tipo di conoscenza. Trasforma
il chiarore della consapevolezza in oscurità. Molti di voi hanno la
tendenza ad avere questo genere di conoscenza. Tutto quello che sembra
pesante lo lasciate stare e andate a cercare qualcosa di leggero. Questo
è il genere di conoscenza che possedete!

Noi addestriamo la nostra mente nello stesso modo in cui i genitori si
prendono cura dei loro figli. Si lascia che i figli vadano per la loro
strada, ma se stanno per mettere una mano nel fuoco, se stanno per
cadere in un pozzo oppure vanno verso un pericolo, si è pronti ad
aiutarli. Chi può amare un figlio come i genitori? I genitori amano i
loro figli, per questo li sorvegliano in continuazione. Nella mente
hanno una costante consapevolezza, la sviluppano in continuazione. Un
genitore non trascura un figlio, ma nemmeno gli sta sempre addosso. I
bambini non hanno conoscenza del modo in cui sono le cose, per questo i
genitori devono sorvegliarli, seguire i loro movimenti. Quando sembra
che stiano per cadere nel pozzo, la madre li prende e li porta da
qualche altra parte, lontani dal pericolo. Poi i genitori tornano al
lavoro, ma continuano a tenere d'occhio i bambini e ad addestrare
coscientemente questa conoscenza e questa consapevolezza dei loro
movimenti. Quando corrono di nuovo verso il pozzo, la madre li prende e
li riporta in un luogo sicuro.

Far crescere la mente è la stessa cosa. Se così non fosse, come potrebbe
il Buddha prendersi cura di noi? \emph{Buddho} significa ``Colui che
Conosce, il Risvegliato, il Radioso''. Se la vostra consapevolezza è
quella di un bambino piccolo, come potrete essere risvegliati e radiosi?
Continuerete solo a mettere la mano nel fuoco. Conoscere la propria
mente ma non addestrarla è forse una cosa intelligente? La conoscenza
mondana significa essere astuti, sapere come nascondere i propri errori,
come cavarsela con le cose. Questo è ciò che il mondo ritiene buono. Il
Buddha non è d'accordo. Che senso ha guardare lontano, al di fuori di se
stessi? Guardate invece vicino, proprio qui. Guardate la vostra mente.
Questa sensazione sorge e non è salutare, questo pensiero sorge ed è
salutare. Dovete conoscere quando la mente è in uno stato salutare e
quando non lo è. Abbandonate quel che non è salutare e sviluppate quel
che è salutare. È così che deve essere, se volete conoscere. Avviene
prendendosi cura della pratica, incluse le norme riguardanti la propria
dimora.

Al mattino dovete innanzitutto alzarvi velocemente, appena sentite il
suono della campana. Chiudete la porta e le finestre della \emph{kuṭī} e
prendete parte ai canti del mattino. Svolgete i lavori di gruppo.
Attualmente, invece, che cosa succede? Quando vi alzate dovete
affrettarvi, la porta e le finestre della \emph{kuṭī} le lasciate
aperte, e delle vesti restano sulla corda dei panni. La pioggia vi
coglie del tutto impreparati. Appena comincia a piovere oppure sentite
un tuono, siete costretti a farvi tutta la strada di corsa per tornare
indietro. Ogni volta che uscite dalla \emph{kuṭī}, chiudete la porta e
le finestre. Se la vostra veste è fuori, sulla corda dei panni,
portatela dentro e riponetela ordinatamente. Non vedo molti che lo
fanno. Portate la veste per la balneazione\footnote{In thailandese
  \emph{pahapnamfon} (\thai{ผ้าอาบน้ำฝน}). Si tratta di un pezzo di stoffa
  rettangolare, avvolto alla vita e usato direttamente a contatto con la
  pelle sotto il \emph{sabong}. Questo stesso termine thailandese viene
  utilizzato anche per indicare in pāli il \emph{vassika-sāṭikaṃ}, che
  nel Vinaya è un indumento che, analogo al \emph{pahapnamfon}, viene
  utilizzato solo durante il Ritiro delle Piogge per lavarsi sotto la
  pioggia o in un fiume onde evitare la nudità.} nei pressi della vostra
\emph{kuṭī} affinché si asciughi. Durante la stagione delle piogge
mettetela nella \emph{kuṭī}.

Non sono necessarie molte vesti. Ho visto dei \emph{bhikkhu} andare a
lavare gli abiti monastici mezzi sepolti dalla stoffa. Se non è per
questa ragione, è perché stavano andando ad accendere un qualche falò.
Quando di vesti se ne hanno molte è una seccatura. Avete bisogno solo di
un \emph{jeewon}, di un \emph{sanghati}, di uno o due
\emph{sabong}.\footnote{La veste monastica dei monaci \emph{theravādin}
  che copre la parte superiore del corpo è un ampio rettangolo di stoffa
  (in pāli: \emph{uttarā-saṅgha}; in thailandese \emph{jeewon}, \thai{จีวร})
  che si avvolge attorno al corpo e che spesso viene messo ad asciugare
  dall'umidità e dal sudore al ritorno della questua. Vi è poi la parte
  inferiore della veste, un rettangolo più piccolo indossato dalla vita
  in giù (in pāli: \emph{āntara-vāsaka}; in thailandese: \emph{sabong},
  \thai{สบง}). Oltre alla veste superiore e a quella inferiore vi è una veste
  esterna a doppio strato (in pāli: \emph{saṅghāti}; in thailandese
  \emph{sanghati}: \thai{สังฆาฏ}) che in genere viene portata ripiegata lungo
  la spalla sinistra in situazioni cerimoniali.} Non so che cosa sia
quel gran mucchio disordinato di roba che vi portate in giro. Nei giorni
dedicati al lavaggio dell'abito monastico, alcuni di voi arrivano dopo
tutti gli altri, quando l'acqua già bolle, ovviamente perché così vanno
a lavare solo le loro cose. Quando hanno finito, si affrettano ad
andarsene e non aiutano a riordinare. Ci manca poco che gli altri li
uccidano, lo capite? Quando tutti si aiutano a vicenda per preparare le
schegge del legno dell'albero del pane e per far bollire l'acqua, è una
cosa orribile che qualcuno si nasconda da qualche parte per non essere
visto.

Lavare uno o due pezzi di stoffa a testa non dovrebbe essere un grande
problema. Però, da quei ``steng steng steng'' che sento, pare che stiate
abbattendo un albero enorme per costruire il pilastro di una casa,
invece di scheggiare del legno di albero del pane. Siate moderati. Se le
schegge di legno le usate solo una o due volte e poi le gettate via,
come faremo a procurarci questo legno? E poi la cottura delle ciotole.
Continuate a mettere legna sul fuoco fino a che le ciotole si spaccano,
e poi le gettate via. C'è un mucchio di ciotole scartate ai piedi
dell'albero di manghi. Perché lo fate? Se non sapete come si cuociono le
ciotole, chiedete. Chiedete a un monaco anziano. Consultatelo. Ci sono
stati dei \emph{bhikkhu} che sono andati ugualmente avanti a cuocere le
ciotole in qualche modo, anche se non erano a conoscenza del giusto modo
di farlo. Quando la ciotola si rompeva, andavano a chiederne un'altra.
Come si fa a essere così malaccorti? Sono tutte azioni sbagliate e
\emph{kamma} cattivo.

Badate agli alberi del monastero al meglio delle vostre possibilità. Per
evitare che vadano bruciati rami e foglie, in nessun caso accendete dei
fuochi nei pressi degli alberi. Abbiate cura degli alberi. Non consento
neanche ai laici di accendere dei fuochi per scaldarsi nelle mattinate
della stagione fredda. Una volta, quando alcuni lo fecero comunque,
successe che la loro testa si riempì di pulci. Peggio ancora, la cenere
si sparse ovunque e insudiciò tutto. Solo dei rudi pescatori si
comportano così.

Quando ho fatto un giro per dare un'occhiata intorno al monastero, nella
foresta ho visto lattine, scatole di detersivi e involucri di saponette
gettati al suolo. Sembrava più il cortile di un mattatoio che un
monastero nel quale la gente si reca a rendere omaggio. Non è di buon
auspicio. Se gettate qualcosa, fatelo nel posto giusto, in modo tale che
tutta l'immondizia possa essere portata via e incenerita. Ma ora che
cosa succede? Appena vi trovate fuori dell'area in cui si trova la
vostra \emph{kuṭī}, lanciate la vostra spazzatura fuori, nella foresta.
Siamo monaci, praticanti del Dhamma. Fate le cose in un bel modo, bello
all'inizio, bello nel mezzo, bello alla fine. Bello nel senso in cui ci
insegnò il Buddha. Questa pratica riguarda l'abbandono delle
contaminazioni. Se invece ne accumulate altre, significa che state
percorrendo un sentiero differente da quello del Buddha. Egli le
contaminazioni le elimina, voi ve ne accollate altre. È totale follia.

La ragione non è difficile da capire. È solo che non riflettete con
sufficiente continuità per chiarire le cose. Affinché la riflessione
sulla nascita, sull'invecchiamento e sulla morte abbia qualche reale
effetto, deve essere condotta fino al punto che quando al mattino vi
svegliate, rabbrividite. Riconoscete il dato di fatto che la morte può
arrivare in qualsiasi momento. Potreste morire domani. Potreste morire
oggi. Se è così, non potete limitarvi ad andare avanti spensieratamente.
Dovete svegliarvi. Praticate la meditazione camminata. Se avete paura di
morire, allora dovete cercare di realizzare il Dhamma nel tempo che
avete a disposizione. Se però non meditate sulla morte, vincerà la
paura.

Forse agli incontri del mattino non ci sarebbe nessuno se la campana non
suonasse così forte e tanto a lungo, e non so quand'è che potreste
recitare qualche canto. Alcuni di voi si svegliano all'alba, afferrano
la ciotola e si affrettano subito per fare un corto giro per la questua.
Tutti escono dal monastero solo quando hanno voglia di farlo. Invece
dovete parlare tra voi per stabilire chi, nel giro per la questua, fa un
tragitto e chi ne fa un altro. Per stabilire a che ora dovrebbero
partire coloro che vanno a Ban Glang, a che ora dovrebbero partire
quelli che vanno a Ban Gor, a che ora dovrebbero partire gli altri che
vanno a Ban Bok. Che questi orari siano il vostro punto di riferimento.
Quando la campana suona, avviatevi subito. Ora, coloro che si avviano
per primi restano in piedi ad attendere ai margini del villaggio e
quelli che si avviano dopo devono correre per raggiungerli. A volte un
gruppo ha appena attraversato il villaggio e, quando sta per uscirne,
ecco che arriva un secondo gruppo. Le persone non sanno che cosa mettere
nelle ciotole dei monaci del secondo gruppo. Se si va avanti così è un
disastro. Decidete chi percorre una strada e chi ne percorre un'altra.
Se qualcuno è malato o ha qualche problema e vuole cambiare strada, che
lo dica. Ci si può mettere d'accordo per fare le cose. Che cosa pensate
di fare, di seguire i vostri desideri? È una disgrazia totale! Vi
starebbe bene se durante la questua riceveste solo uno
scalpello.\footnote{Lo scalpello è di solito usato come arma. Uno
  scalpello messo nella ciotola durante la questua verrebbe interpretato
  come una minaccia di violenza nei riguardi di un \emph{bhikkhu}.}

Se avete bisogno di dormire di più non restate svegli fino a tardi la
sera. Quale attività tanto impegnativa svolgete, da necessitare di
dormire così tanto? Sforzarvi a praticare la meditazione seduta e quella
camminata non è che vi faccia perdere così tanto sonno. Il sonno ve lo
fa invece perdere il tempo trascorso a indulgere alla socializzazione.
Quando avete fatto abbastanza meditazione camminata e vi sentite
stanchi, andate a dormire. Dividete il vostro tempo nel giusto modo tra
le attività per il Saṅgha e le attività legate alla pratica individuale
in modo da poter riposare a sufficienza.

Durante la stagione calda, alcuni giorni, ad esempio quando c'è molta
umidità, possiamo prenderci una pausa dai canti della sera. Dopo aver
tirato l'acqua dal pozzo, potete fare il vostro bagno e praticare nel
modo che preferite. Se volete fare la meditazione camminata, arrivate
subito al dunque. Potete farla per tutto il tempo che volete. Provateci.
Anche se fate la meditazione camminata fino alle sette, avete ancora
tutta la notte davanti a voi. Non c'è alcuna ragione per perdere il
sonno. Il problema è che non sapete gestire il vostro tempo. Dipende da
voi. Alzarvi tardi o presto dipende da voi. Come potete riuscire a
realizzare qualcosa se non addestrate e non correggete voi stessi?
L'addestramento è indispensabile. Se vi addestrate, queste piccole cose
non saranno una difficoltà. Con esse ci giocherete. Fate in modo che la
vostra pratica sia di beneficio per voi stessi e per gli altri.

Addestratevi bene nella pratica. Se sviluppate la vostra mente, la
saggezza è destinata a sorgere. Se nella camminata
\emph{jongrom}\footnote{\emph{Jongrom}: Parola thailandese (\thai{จงกรม},
  \thai{เดินจงกรม}) per il termine pāli \emph{cankama}; indica la meditazione
  di solito eseguita andando avanti e indietro su di un sentiero
  prestabilito -- lungo circa 15 metri e largo circa 1 metro, delimitato
  all'inizio e alla fine da un oggetto o da un albero -- mentre si
  focalizza l'attenzione su di un oggetto di meditazione.} ci mettete il
cuore, dopo che avrete percorso il sentiero per tre volte il Dhamma
fluirà con forza. Voi invece vi trascinate su e giù in uno stato di
sonnolenza, con la testa appesa. Statemi a sentire, voi che avete il
collo rotto. Si dice che se andate nella foresta o in montagna gli
spiriti vi prenderanno, lo sapete? Se quando state seduti vi sentite
assonnati, alzatevi! Fate un po' di meditazione camminata, non restate
lì seduti. In piedi, camminando o seduti, dovete vincere la sonnolenza.
Se sorge qualcosa e non fate niente per risolvere il problema o per
migliorare voi stessi, come potrà andare meglio?

Mentre state camminando sul \emph{jongrom}, imparate il
\emph{Pāṭimokkha} ripetendolo a memoria. È davvero piacevole, e rende
pure sereni. Addestratevi. Andate a fare la questua verso Ban Gor,
mantenete la vostra attenzione su voi stessi, tenetevi lontani da coloro
ai quali piace chiacchierare. Lasciateli andare avanti, loro camminano
svelti. Non parlate con i chiacchieroni. Parlate molto con il vostro
cuore, meditate molto. La gente che ama parlare tutto il giorno somiglia
agli uccelli che ciarlano. Non tollerate alcuna sciocchezza. Indossate
l'abito in modo ordinato e avviatevi per il vostro giro per la questua.
Appena cominciate a camminare con la vostra andatura, iniziate a
recitare a memoria il \emph{Pāṭimokkha}. Ciò rende la mente disciplinata
e radiosa. È una specie di manuale. Non è che debba diventare
un'ossessione. Semplicemente, quando lo avrete imparato a memoria, il
\emph{Pāṭimokkha} illuminerà la vostra mente. Quando camminate,
focalizzate l'attenzione su di esso. Lo imparerete presto e sorgerà
automaticamente. Addestratevi in questo modo.

Addestrate voi stessi. Dovete addestrarvi. Non state a perder tempo.
Quando vi comportate così, siete come cani. Un cane vero, però, è
meglio. A notte fonda quando gli si passa vicino abbaia. Voi non fate
neanche questo. «~Perché ti interessa solo dormire? Perché non ti
alzi?~» Dovete insegnare a voi stessi ponendovi queste domande. Durante
la stagione fredda alcuni di voi si avvolgono nell'abito e, nel bel
mezzo della giornata, se ne vanno a dormire. Non fatelo.

Quando andate in gabinetto, prima prostratevi. Prostratevi al mattino
quando suona la campana, prima di andare a fare la questua. Prostratevi
dopo il pasto, dopo aver lavato la vostra ciotola e aver riunito tutte
le vostre cose, prostratevi prima di tornare nella vostra \emph{kuṭī}.
Non lasciatevi sfuggire queste occasioni. La campana suona quando si
deve tirare acqua dal pozzo: prostratevi prima di lasciare la vostra
\emph{kuṭī}. Se ve ne dimenticate, e avete già camminato fino all'area
centrale del monastero prima di esservene resi conto, tornate indietro e
prostratevi. Dovete portare a questo livello il vostro addestramento.
Addestrate il vostro cuore e la vostra mente. Non limitatevi a lasciar
andare. Tutte le volte che vi dimenticate di prostrarvi, tornate
indietro e prostratevi. Come potrete dimenticarvene se siete diligenti
fino a questo punto, e sapete che poi dovete camminare avanti e
indietro? Qual è ora il vostro comportamento? «~Mi sono dimenticato. Non
fa niente. Non importa.~» Ecco perché il monastero è in queste
condizioni. Mi sto riferendo ai vecchi metodi tradizionali. Sembra che
siano svaniti. Non so come le chiamiate queste cose oggigiorno.

Tornate alle vecchie abitudini, alle pratiche ascetiche. Quando sedete
ai piedi di un albero, prostratevi. Anche se lì non c'è un'immagine del
Buddha, prostratevi. Se lo fate, la vostra consapevolezza è lì. Quando
siete seduti, mantenete una postura corretta, non state seduti
aggrappandovi alle ginocchia come dei matti. Stare seduti in questo modo
è l'inizio della fine. Addestrare voi stessi non vi farà morire. È solo
pigrizia, questo è il problema. Non lasciate che vi entri nella testa.
Se siete davvero insonnoliti, allora mettetevi distesi, ma fatelo con
consapevolezza, rammentando a voi stessi di alzarvi quando vi svegliate.
Siate severi con voi stessi: «~Se non lo farò, che possa finire
all'inferno!~» Uno stomaco pieno rende davvero stanchi, e la stanchezza
vi fa sembrare che mettersi distesi sia una cosa meravigliosa. Se siete
sdraiati comodamente e a vostro agio, quando sentite il suono della
campana e vi dovete alzare, vi arrabbiate. Potreste perfino desiderare
di uccidere chi suona la campana. Contate. Dite alla vostra mente: «~Se
arrivo fino al tre e non mi alzo, che io possa finire all'inferno.~»
Dovete dirlo seriamente. Dovete afferrare la contaminazione e ucciderla.
Non basta prendere in giro la propria mente.

Leggete le biografie dei grandi maestri. È gente strana, vero? Sono
diversi. Riflettete con attenzione a questa differenza. Addestrate la
mente in modo corretto. Non dovete fare affidamento su nessun altro.
Scoprite da voi stessi i giusti mezzi abili per addestrare la mente. Se
comincia a pensare a cose mondane, soggiogatela subito. Fermatela.
Alzatevi. Cambiate postura. Dite a voi stessi di non pensare a quelle
cose, ci sono cose migliori alle quali pensare. È essenziale che non si
ceda neanche un po' a quei pensieri. Non pensiate di poterla prendere
alla leggera e che sarà la vostra pratica a occuparsi di queste cose.
Tutto dipende dall'addestramento.

Alcuni animali sono in grado di trovare il cibo di cui hanno bisogno e
di tenersi in vita perché sono veloci e abili. Guardate le lucertole. E
le tartarughe? Le tartarughe sono così lente che potreste chiedervi come
facciano a sopravvivere. Non lasciatevi ingannare. Gli esseri hanno una
loro volontà, hanno i loro metodi. Con la meditazione seduta e la
meditazione camminata è la stessa cosa. I grandi maestri avevano i loro
metodi, ma per loro era difficile comunicarli. Come quell'anziano che
viveva a Piboon. Tutte le volte che qualcuno affogava, era lui a
immergersi per cercare il cadavere. Continuava a immergersi a lungo --
fino a quando il sole non faceva seccare le foglie di un ramo spezzato
-- e trovava sempre i cadaveri. Quando affogava qualcuno, lui era l'uomo
giusto. Quando gli chiesi come riuscisse a farlo, disse che lo sapeva
fare, ma che non riusciva a spiegarlo a parole. Ecco com'è, si tratta di
una questione personale. È difficile da comunicare, dovete imparare a
farlo da soli. Addestrare la mente è la stessa cosa.

Sbrigatevi con questo addestramento! Ve lo dico, ma non vi sto dicendo
che il Dhamma è una cosa da rincorrere, o che possa essere realizzata
solo mediante uno sforzo fisico, andando avanti senza dormire e
digiunando. Non si tratta di arrivare allo sfinimento, si tratta di
rendere la mente ``proprio giusta'' per il Dhamma.

