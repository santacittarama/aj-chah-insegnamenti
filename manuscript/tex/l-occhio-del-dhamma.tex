\chapter{L'Occhio del Dhamma}

\begin{openingQuote}
  Discorso offerto al Wat Pah Pong ai monaci e ai novizi nell'ottobre del 1968.
\end{openingQuote}

Alcuni di noi cominciano con la pratica e dopo un anno o due non hanno
ancora capito che cosa sia. Siamo ancora incerti a proposito della
pratica. Quando siamo ancora incerti, non vediamo che ogni cosa attorno
a noi è puro Dhamma, e così ci rivolgiamo agli insegnamenti degli
\emph{ajahn}. Però, in realtà, quando conosciamo la nostra mente, quando
c'è \emph{sati} a osservare da vicino la mente, c'è saggezza. Ogni
momento e ogni luogo diventano per noi occasioni di ascoltare il Dhamma.

Possiamo imparare il Dhamma dalla natura, dagli alberi ad esempio. Delle
cause fanno nascere un albero ed esso cresce seguendo il corso della
natura. Proprio così l'albero ci sta insegnando il Dhamma, ma non lo
comprendiamo. A tempo debito cresce, e cresce fino a quando appaiono
delle gemme, dei fiori e dei frutti. Tutto quello che vediamo è
l'apparenza, i fiori e i frutti. Non siamo in grado di interiorizzare
queste cose e di contemplarle. È per questo che non sappiamo che
l'albero ci sta insegnando il Dhamma. I frutti appaiono e noi ci
limitiamo a mangiarli senza investigare: dolce, aspro o salato, è la
natura del frutto. E questo è Dhamma, l'insegnamento del frutto. Le
foglie invecchiano, appassiscono, muoiono e poi cadono dall'albero.
Tutto quello che vediamo è che le foglie sono cadute. Ci camminiamo
sopra, le spazziamo, e questo è tutto. Non investighiamo con
accuratezza, e per questo non sappiamo che la natura ci sta insegnando.
In seguito nuove foglie germogliano, e noi vediamo solo questo, senza
andare oltre. Non portiamo queste cose nella nostra mente per
contemplarle.

Se riusciamo a portare tutto ciò dentro di noi per investigarlo, vedremo
che la nascita di un albero e la nostra stessa nascita non sono diverse.
Questo nostro corpo è nato ed esiste dipendendo da condizioni, dagli
elementi terra, acqua, fuoco e vento. Ha il suo cibo, e cresce, cresce.
Ogni parte del corpo cambia e fluisce in accordo con la sua natura. Non
è diverso dall'albero. Capelli, peli, unghie, denti e pelle cambiano. Se
conosciamo le cose della natura, allora conosceremo noi stessi. La gente
nasce. Alla fine muore. Dopo essere morta nasce di nuovo. Capelli, peli,
unghie, denti e pelle muoiono e ricrescono in continuazione. Se
comprendiamo la pratica, allora possiamo capire che un albero non è
diverso da noi. Se comprendiamo gli insegnamenti dei maestri, allora
vediamo che l'esterno e l'interno sono simili. Cose che hanno coscienza
e altre prive di coscienza non differiscono. Sono la stessa cosa. E se
comprendiamo questa uniformità, quando ad esempio vedremo la natura di
un albero, sapremo che esso non è diverso dai nostri cinque
\emph{khandhā}: corpo, sensazione, memoria, pensiero e coscienza. Se
abbiamo questa comprensione, allora comprendiamo il Dhamma. Se
comprendiamo il Dhamma, comprendiamo i cinque \emph{khandhā}, come si
spostano e cambiano in continuazione, senza fermarsi mai.

Perciò sia stando in piedi, camminando, seduti o distesi dovremmo avere
\emph{sati} per badare alla mente e prenderci cura di essa. Quando
vediamo le cose esteriori è come guardare quelle interiori. Quando
vediamo quelle interiori è la stessa cosa che guardare quelle esteriori.
Se lo comprendiamo, possiamo sentire l'insegnamento del Buddha. Se lo
comprendiamo, possiamo dire che la ``Natura del Buddha'' -- ``Colui che
Conosce'' -- si è insediata dentro di noi. Essa conosce l'esterno. Essa
conosce l'interno. Essa comprende tutte le cose che sorgono. Quando
comprendiamo in questa maniera, allora stando seduti ai piedi di un
albero sentiamo l'insegnamento del Buddha. In piedi, camminando, stando
seduti o distesi, sentiamo l'insegnamento del Buddha. Quando vediamo,
sentiamo, odoriamo, assaporiamo, tocchiamo e pensiamo sentiamo
l'insegnamento del Buddha. Il Buddha è solo questo Colui che Conosce che
si trova proprio all'interno di questa mente. Conosce il Dhamma,
investiga il Dhamma. Non è che il Buddha che visse tanto tempo fa venga
a parlare con noi, è questa Natura del Buddha a sorgere, sorge Colui che
Conosce. La mente diviene illuminata.

Se facciamo insediare il Buddha all'interno della nostra mente, allora
vediamo tutto, contempliamo ogni cosa come se non fosse diversa da noi.
Vediamo gli alberi, le montagne, le piante e i vari animali come non
diversi da noi. Vediamo i poveri e i ricchi. Non sono diversi da noi.
Gente di colore e di pelle chiara. Non sono diversi! Hanno tutti le
stesse caratteristiche. Chi comprende le cose in questo modo ovunque si
trovi è contento. Sente sempre l'insegnamento del Buddha. Se non lo
comprendiamo, anche se trascorriamo tutto il nostro tempo ad ascoltare
gli insegnamenti degli \emph{ajahn}, non ne comprenderemo il senso. Il
Buddha disse che l'Illuminazione del Dhamma è solo conoscere la
natura,\footnote{Con ``natura'' qui si intendono tutte le cose, mentali
  e fisiche, non solo alberi, animali e così via.} la realtà che sta
tutt'intorno a noi, la natura che è proprio qui. Se questa natura non la
comprendiamo, sperimentiamo delusione e gioia, ci perdiamo negli oggetti
mentali e così facciamo sorgere dolore e rimpianto. Perdersi negli
oggetti mentali significa perdersi nella natura. Quando ci perdiamo
nella natura non conosciamo il Dhamma. L'Essere Illuminato semplicemente
sottolineò questa natura. Dopo essere sorte, tutte le cose si
trasformano e muoiono. Le cose che fabbrichiamo, come i piatti, le
ciotole, le stoviglie, hanno tutte le stesse caratteristiche. Una
ciotola viene fusa, ottiene una forma e perviene all'esistenza in
ragione di una causa, dell'impulso dell'uomo a crearla, e quando la
usiamo invecchia, si rompe e scompare. Alberi, montagne e piante, come
pure animali e persone: sono la stessa cosa.

Quando Aññā Kondañña, il primo discepolo del Buddha, sentì per la prima
volta il suo Insegnamento, la comprensione che ottenne non fu nulla di
complicato. Egli comprese semplicemente che qualsiasi cosa nasca, quella
cosa deve trasformarsi, invecchiare per ragioni naturali e alla fine
morire. Aññā Kondañña non aveva mai pensato a questo prima di allora, o
se l'aveva fatto non gli era stato del tutto chiaro e per questo non
aveva lasciato andare, si era ancora attaccato ai \emph{khandhā}. Quando
sedette ascoltando con consapevolezza il discorso del Buddha, in lui
sorse la Natura del Buddha. Ricevette una specie di ``trasmissione'' di
Dhamma, che consisteva nella conoscenza che tutti i fenomeni
condizionati sono impermanenti. Tutto quel che nasce deve invecchiare e
morire per natura. Questa sensazione era diversa da tutte le altre che
aveva provato in precedenza. Egli comprese davvero la sua mente, e
perciò il ``Buddha'' sorse dentro di lui. Allora il Buddha dichiarò che
in Aññā Kondañña si era schiuso l'``Occhio del Dhamma''.

Che cos'è che vede questo Occhio del Dhamma? Questo Occhio vede che
qualsiasi cosa nasca deve invecchiare e morire per natura. ``Qualsiasi
cosa nasca'' significa tutto! Materiale o immateriale, tutto è compreso
in quel ``qualsiasi cosa nasca''. Si riferisce a tutta la natura. Come
questo corpo, ad esempio. È nato e procede verso l'estinzione. Quando è
piccolo ``muore'' all'infanzia e va verso la giovinezza. Dopo un po'
``muore'' alla giovinezza e diventa di mezza età. Poi va avanti a
``morire'' alla mezza età e raggiunge la vecchiaia, e al termine
raggiunge la fine. Gli alberi, le montagne, le piante, tutto ha questa
caratteristica. La visione o la comprensione di ``Colui che Conosce''
entrò con chiarezza nella mente di Aññā Kondañña, che stava lì seduto.
La conoscenza che ``qualsiasi cosa nasca'' penetrò a fondo nella sua
mente, rendendolo in grado di sradicare l'attaccamento al corpo. Questo
attaccamento era \emph{sakkāya-diṭṭhi}.\footnote{\emph{Sakkāya-diṭṭhi}:
  Convinzione che induce l'identificazione con il sé, con l'io.} Ciò
significa che egli non considerava più il corpo come se fosse un sé o un
essere, non lo vedeva in termini di ``lui'' o di ``io''. Non si
aggrappava più a esso. Lo vide con chiarezza, e così sradicò
\emph{sakkāya-diṭṭhi}. Allora \emph{vicikicchā}\footnote{\emph{Vicikicchā}:
  Il dubbio.} fu distrutta. Avendo sradicato l'attaccamento al corpo non
ebbe dubbi su ciò che aveva comprenso. Così venne sradicato anche
\emph{sīlabbata parāmāsa}.\footnote{\emph{Sīlabbata parāmāsa}: viene
  tradizionalmente tradotto come attaccamento ai riti e alle
  cerimonie/osservanze. Qui il venerabile Ajahn lo riferisce, assieme al
  dubbio, specificamente al corpo. Queste tre cose,
  \emph{sakkāya-diṭṭhi}, \emph{vicikicchā} e \emph{sīlabbata parāmāsa}
  sono le prime tre delle dieci ``catene'' che vengono abbandonate
  allorché si intravede l'Illuminazione, un evento noto come ``Entrata
  nella corrente''. Con la completa Illuminazione vengono trascese tutte
  e dieci le ``catene''. Si veda il \emph{Glossario} p. \pageref{glossary-samyojana}, alla voce
  \emph{saṃyojana}.} La sua pratica divenne stabile e retta. Anche se il
suo corpo provava dolore o aveva la febbre, egli non si aggrappava a
esso, non dubitava. Non dubitava perché aveva sradicato l'attaccamento.
Questo attaccamento al corpo è chiamato \emph{sīlabbata parāmāsa}.
Quando si sradica l'opinione che il corpo sia ``il sé'', l'attaccamento
e il dubbio sono finiti. Quando questa opinione del corpo come ``il sé''
sorge all'interno della mente, l'attaccamento e il dubbio cominciano
proprio da lì.

Così, quando il Buddha espose il Dhamma, in Aññā Kondañña si schiuse
l'Occhio del Dhamma. Questo Occhio è solo ``Colui che Conosce con
chiarezza''. Vede le cose in modo diverso. Vede proprio la loro natura.
Vedendo la natura con chiarezza, l'attaccamento è sradicato, e nasce
``Colui che Conosce''. In precedenza conosceva, ma c'era ancora
attaccamento. Si potrebbe dire che egli conosceva il Dhamma, ma che non
lo aveva ancora visto, oppure che aveva visto il Dhamma ma che non era
ancora tutt'uno con esso. Allora il Buddha disse: «~Kondañña conosce.~»
Che cos'è che conosceva? Conosceva la natura. Di solito nella natura ci
perdiamo, come pure in questo nostro corpo. Terra, acqua, fuoco e vento
si riuniscono per formare questo corpo. È un aspetto della natura, un
oggetto materiale che possiamo vedere con gli occhi. Esiste in
dipendenza dal cibo, cresce e si trasforma finché, alla fine, giunge
all'estinzione.

Nell'interiorità, ciò che vigila sul corpo è la coscienza: è solo questo
``Colui che Conosce'', è questa sola consapevolezza. Se riceve gli
oggetti per mezzo dell'occhio, si chiama vedere. Se li riceve per mezzo
dell'orecchio, si chiama sentire; per mezzo del naso, odorare; per mezzo
della lingua, assaporare; per mezzo del corpo, toccare; per mezzo della
mente, pensare. Questa coscienza è solo una, ma, quando essa funziona in
posti diversi, la chiamiamo in modo differente. Per mezzo dell'occhio la
chiamiamo in un modo, per mezzo dell'orecchio la chiamiamo in un altro.
Che però entri in funzione con l'occhio, l'orecchio, il naso, la
lingua, il corpo o la mente, si tratta di una sola consapevolezza.
Seguendo le Scritture parliamo delle sei coscienze, ma in realtà è una
sola coscienza che sorge su queste sei differenti basi. Ci sono sei
``porte'', ma una sola consapevolezza, che è proprio questa mente.

La mente è in grado di conoscere la Verità della natura. Se la mente ha
ancora degli impedimenti, diciamo che conosce per mezzo dell'ignoranza.
Conosce erroneamente e vede erroneamente. A conoscere erroneamente e
vedere erroneamente, oppure a conoscere e vedere rettamente è solo
un'unica consapevolezza. Le chiamiamo errata visione e Retta Visione, ma
si tratta di una cosa sola. Retto ed errato sorgono entrambi in questo
unico luogo. Quando c'è errata conoscenza allora c'è errata visione,
errata intenzione, errata azione, errati mezzi di sussistenza. Tutto è
errato! E d'altra parte il Sentiero della Retta Pratica nasce proprio in
questo stesso posto. Quando c'è quel che è giusto, quel che è errato
scompare.

Il Buddha praticò la sopportazione di molti disagi e torturò se stesso
con il digiuno e altre cose ancora, ma poi investigò più a fondo nella
sua mente finché, alla fine, riuscì a sradicare l'ignoranza. Tutti i
Buddha erano Illuminati nella mente, perché il corpo non conosce nulla.
Lo potete far mangiare o no, non importa, può morire in qualsiasi
momento. Tutti i Buddha praticarono con la mente. Erano Illuminati nella
mente. Dopo aver contemplato la sua mente, il Buddha rinunciò ai due
estremi della pratica -- indulgere al piacere e indulgere al dolore -- e
nel suo primo discorso espose la Via di Mezzo tra questi due estremi.
Noi però ascoltiamo il suo Insegnamento ed esso stride con i nostri
desideri. Siamo infatuati del piacere e del benessere, siamo infatuati
della felicità, pensiamo che siamo buoni, che stiamo bene: questo è
indulgere al piacere. Non è il Retto Sentiero. Insoddisfazione,
dispiacere, avversione e rabbia: questo è indulgere al dolore. Questi
sono gli estremi che dovrebbero essere evitati da chi percorre il
Sentiero.

Queste ``vie'' sono semplicemente la felicità e l'infelicità che sorgono
nella mente. ``Chi percorre il Sentiero'' è proprio questa mente,
``Colui che Conosce''. Se a sorgere è uno stato mentale buono, ci
attacchiamo a esso come buono: questo significa indulgere al piacere. Se
a sorgere è uno stato mentale spiacevole, ci attacchiamo a esso per
mezzo dell'avversione: questo significa indulgere al dolore. Si tratta
di sentieri errati, non sono le vie del meditante. Sono le vie degli
esseri mondani, di chi cerca divertimento e felicità ed evita le cose
spiacevoli e la sofferenza. I saggi conoscono i sentieri errati e li
abbandonano, vi rinunciano. Restano impassibili al piacere e al dolore,
alla felicità e alla sofferenza. Queste cose sorgono, ma coloro che
conoscono non si attaccano a esse, le lasciano andare secondo la loro
natura. Questa è Retta Visione. Quando tutto questo è conosciuto in modo
completo, vi è la Liberazione. Felicità e infelicità non hanno
significato per un Essere Illuminato.

Il Buddha disse che gli Esseri Illuminati sono lontani dalle
contaminazioni. Questo non significa che loro scapparono dalle
contaminazioni, non scapparono da nessuna parte. Le contaminazioni erano
lì. Egli paragonò questa situazione a una foglia di loto in uno stagno.
La foglia e l'acqua coesistono, sono in contatto, ma la foglia non si
bagna. L'acqua rappresenta le contaminazioni e la foglia di loto la
mente illuminata. La mente di chi pratica è così. Non scappa da nessuna
parte, resta proprio lì. Bene, male, felicità e infelicità, giusto e
sbagliato sorgono, ed essa li conosce tutti. Il meditante si limita a
conoscerli, non entrano nella sua mente. Non ha attaccamento, è
semplicemente colui che ne fa esperienza. Dire che egli è semplicemente
colui che ne fa esperienza fa parte del nostro linguaggio ordinario. Nel
linguaggio del Dhamma diciamo che egli fa in modo che la sua mente segua
la Via di Mezzo.

Queste attività della felicità, dell'infelicità e così via sorgono in
continuazione perché sono caratteristiche del mondo. Il Buddha fu
Illuminato nel mondo, contemplò il mondo. Se non avesse contemplato il
mondo, se non avesse visto il mondo, non avrebbe potuto trascenderlo.
L'Illuminazione del Buddha fu semplicemente l'Illuminazione proprio in
questo mondo. Il mondo era ancora lì, guadagno e perdita, lode e
biasimo, fama e discredito, felicità e infelicità erano ancora lì. Se
queste cose non fossero esistite, non ci sarebbe stato alcun motivo per
illuminarsi! Quello che conosceva era solo il mondo, ciò che circonda il
cuore della gente. Se la gente segue queste cose, alla ricerca di lode e
fama, di guadagno e felicità, e cerca di evitare i loro opposti, affonda
sotto il peso del mondo. Guadagno e perdita, lode e biasimo, fama e
discredito, felicità e infelicità: questo è il mondo. Chi si perde nel
mondo non ha via d'uscita, e il mondo lo sovrasta. Questo mondo segue la
Legge del Dhamma, ed è per questo che lo chiamiamo \emph{dhamma}
mondano. Chi vive dentro il \emph{dhamma} mondano è chiamato essere
mondano. Vive attorniato dalla confusione.

Per questa ragione il Buddha ci insegnò a sviluppare il Sentiero.
Possiamo suddividerlo in moralità, concentrazione e saggezza. Si
dovrebbe svilupparlo fino alla completezza. Questo è il Sentiero della
Pratica che distrugge il mondo. Dov'è questo mondo? È solo nella mente
degli esseri che ne sono infatuati! L'azione di attaccarsi al guadagno,
alla lode, alla fama, alla felicità e all'infelicità è ``il mondo''.
Quando queste cose sono nella mente, allora il mondo sorge, l'essere
mondano è nato. Il mondo nasce a causa del desiderio. Il desiderio è il
luogo di nascita di tutti i mondi. Porre fine al desiderio significa
porre fine al mondo.

La nostra pratica della moralità, della concentrazione e della saggezza
è altrimenti detta Nobile Ottuplice Sentiero. Questo Nobile Ottuplice
Sentiero e gli otto \emph{dhamma} mondani rappresentano una coppia. Che
cosa significa che rappresentano una coppia? Se parliamo secondo le
Scritture, diciamo che guadagno e perdita, lode e biasimo, fama e
discredito, felicità e infelicità sono gli otto \emph{dhamma} mondani.
Retta Visione, Retta Intenzione, Retta Parola, Retta Azione, Retti Mezzi
di Sussistenza, Retto Sforzo, Retta Consapevolezza e Retta
Concentrazione, questo è il Nobile Ottuplice Sentiero. Queste due
ottuplici vie esistono nello stesso luogo. Gli otto \emph{dhamma}
mondani sono proprio qui, in questa mente, con ``Colui che Conosce'', ma
questo ``Colui che Conosce'' ha degli impedimenti e conosce perciò in
modo errato, e diviene così il mondo stesso. Si tratta proprio di
quest'unico ``Colui che Conosce'', di nient'altro. La Natura del Buddha
non è ancora sorta in questa mente, non ha ancora separato se stessa dal
mondo. La mente fatta così è il mondo. Quando pratichiamo il Sentiero,
quando addestriamo il nostro corpo e la nostra parola, tutto ciò lo si
fa proprio in questa stessa mente. È nello stesso posto che tutto ciò
avviene e perciò si vedono l'uno con l'altro, il Sentiero vede il mondo.
Se pratichiamo con questa nostra mente incontriamo questo attaccamento
al guadagno, alla lode, alla fama e alla felicità, vediamo
l'attaccamento al mondo.

Il Buddha disse: «~Dovreste conoscerlo il mondo. Abbaglia come un carro
reale. Estasia gli stolti, ma non inganna i saggi.~» Non voleva che
girassimo tutto il mondo per osservare e studiare ogni cosa. Voleva solo
che osservassimo questa mente che si attacca al mondo. Quando il Buddha
ci disse di guardare il mondo, non voleva che vi rimanessimo bloccati,
voleva che lo investigassimo, perché il mondo nasce proprio in questa
mente. Stando seduti all'ombra di un albero potete osservare il mondo.
Quando c'è il desiderio, è proprio lì che il mondo perviene a esistere.
Il desiderio è il luogo in cui nasce il mondo. Estinguere il desiderio
significa estinguere il mondo.

Quando sediamo in meditazione vogliamo che la mente diventi serena, ma
non succede. Perché? Non vogliamo pensare, ma pensiamo. È come una
persona che va a sedersi su un formicaio, le formiche continuano a
morderla.\footnote{Nelle zone tropicali asiatiche le formiche possono
  essere molto aggressive e il loro morso urticante e doloroso.} Quando
la mente è il mondo, allora anche seduti immobili con gli occhi chiusi
tutto quel che vediamo è il mondo. Piacere, dolore, ansia, confusione,
tutte queste cose sorgono. Perché? Perché non abbiamo ancora realizzato
il Dhamma. Se la mente è così il meditante non riesce a fronteggiare i
\emph{dhamma} mondani, non investiga. È come se stesse seduto su un
formicaio. Le formiche lo mordono perché lui sta seduto proprio sulla
loro casa! Che cosa dovrebbe fare? Dovrebbe cercare un qualche veleno o
usare il fuoco per stanarle. Però, la maggior parte dei praticanti di
Dhamma non la pensa così. Se sono soddisfatti seguono solo la
soddisfazione, e quando provano insoddisfazione seguono solo quella.
Seguendo i \emph{dhamma} mondani la mente diventa il mondo. Talvolta
potreste pensare: «~Oh, non riesco a farlo, è al di sopra delle mie
forze.~» E così non ci provate neanche. Questo avviene perché la mente è
colma di contaminazioni. I \emph{dhamma} mondani evitano che sorga il
Sentiero. Non abbiamo resistenza nello sviluppo della moralità, della
concentrazione e della saggezza. Proprio come l'uomo che siede sul
formicaio. Non riesce a fare nulla, le formiche lo mordono e gli si
arrampicano addosso, e lui è immerso nella confusione e nell'agitazione.
Non riesce ad alzarsi dal posto pericoloso in cui siede, e così resta lì
e soffre.

Altrettanto avviene con la nostra pratica. I \emph{dhamma} mondani
esistono nella mente degli esseri mondani. Quando gli esseri mondani
desiderano trovare la pace, proprio allora sorgono i \emph{dhamma}
mondani. Quando la mente è ignorante c'è solo oscurità. Quando la
conoscenza sorge, la mente s'illumina, perché ignoranza e conoscenza
nascono nello stesso luogo. Quando è sorta l'ignoranza, la conoscenza
non può entrare, perché la mente ha accettato l'ignoranza. Quando la
conoscenza è sorta, non può rimanere l'ignoranza. Per questo il Buddha
esortò i suoi discepoli a praticare con la mente, perché il mondo nasce
nella mente, gli otto \emph{dhamma} mondani sono lì. Il Nobile Ottuplice
Sentiero, ossia l'investigazione mediante la calma e la meditazione di
visione profonda, lo sforzo diligente e la saggezza da noi sviluppati
sono tutte cose che allentano la morsa del mondo. L'attaccamento,
l'avversione e l'illusione diventano più tenui, ed essendo più tenui li
conosciamo per quello che sono. Se sperimentiamo fama, prosperità
materiale, lode, felicità o sofferenza, ne siamo consapevoli. Dobbiamo
conoscere queste cose prima di trascendere il mondo, perché il mondo è
dentro di noi.

Quando siamo liberi da queste cose è come uscire di casa. Quando
entriamo in casa, qual è la sensazione che abbiamo? Percepiamo di essere
passati per la porta e di essere entrati in casa. Quando lasciamo la
casa percepiamo che l'abbiamo lasciata, giungiamo alla luce del sole,
non è scuro come all'interno. L'azione della mente che entra nei
\emph{dhamma} mondani è come entrare in casa. La mente che ha distrutto
i \emph{dhamma} mondani è come uno che è uscito di casa. Perciò il
praticante di Dhamma deve diventare testimone del Dhamma in se stesso.
Se i \emph{dhamma} mondani sono stati abbandonati o no, lo sa da sé, se
il Sentiero si è sviluppato o no, lo sa da sé. Quando il Sentiero è ben
sviluppato elimina i \emph{dhamma} mondani. Diventa sempre più forte. La
Retta Visione cresce, mentre l'errata visione diminuisce, finché alla
fine il Sentiero distruggerà le contaminazioni. Altrimenti saranno le
contaminazioni a distruggere il Sentiero! Retta Visione ed errata
visione, ci sono solo queste due vie. Anche l'errata visione ha i suoi
trucchi, lo sapete. Ha la sua saggezza, ma si tratta di una saggezza che
conduce fuori strada. Il meditante che inizia a sviluppare il Sentiero
sperimenta una scissione. Alla fine è come se fosse due persone, una nel
mondo e l'altra sul Sentiero. Si dividono, si separano. Tutte le volte
che egli investiga, c'è questa separazione, e questo continua a
succedere fino a quando la mente raggiunge la visione profonda, la
\emph{vipassanā}.

O forse si tratta di \emph{vipassanū}!\footnote{\emph{Vipassanūpakkilesa}:
  ``Contaminazione della visione profonda''.} Cercando di instaurare
risultati benefici nella nostra pratica, li vediamo e ci attacchiamo a
essi. Questo tipo di attaccamento proviene dal nostro voler ottenere
qualcosa dalla pratica. Questo è \emph{vipassanū}, la saggezza delle
contaminazioni, ossia una ``saggezza contaminata''. Alcuni sviluppano la
bontà e vi si attaccano, altri sviluppano la purezza e si attaccano a
quella, oppure sviluppano la conoscenza e vi si attaccano. L'azione di
aggrapparsi a quella bontà o a quella conoscenza è \emph{vipassanū} che
si infiltra nella nostra pratica. Quando perciò si sviluppa
\emph{vipassanā}, state in guardia! Fate attenzione a \emph{vipassanū},
perché sono così vicine che a volte non si riesce a separarle. Con la
Retta Visione, però, è possibile vederle entrambe con chiarezza. Se si
tratta di \emph{vipassanū} il risultato sarà che prima o poi sorgerà la
sofferenza. Non c'è sofferenza, se veramente si tratta di
\emph{vipassanā}. C'è pace. Sia la felicità che l'infelicità tacciono.
Lo potete vedere da voi stessi.

La pratica richiede sopportazione. Alcuni, quando cominciano a
praticare, non vogliono essere disturbati da nulla, non vogliono
attriti. Però, gli attriti ci sono esattamente come prima. Dobbiamo
cercare di porre fine agli attriti mediante gli attriti stessi. Così, se
nella vostra pratica ci sono attriti, va bene. Se non ci sono attriti
non va bene, e mangiate e dormite quanto vi pare. Quando volete andare
da qualche parte o dire qualcosa, seguite i vostri desideri.
L'insegnamento del Buddha irrita. Quel che è sovramondano va contro
quel che è mondano. La Retta Visione si oppone all'errata visione, la
purezza si oppone all'impurità. L'Insegnamento irrita i nostri desideri.
Nelle Scritture c'è un racconto sul Buddha prima dell'Illuminazione.
Allora, dopo aver ricevuto un piatto di riso, Egli lo fece galleggiare
sulla corrente d'acqua di un fiume e nella sua mente disse: «~Se otterrò
l'Illuminazione, che questo piatto possa galleggiare controcorrente
sull'acqua.~» Il piatto risalì la corrente! Quel piatto era la Retta
Visione del Buddha, oppure la Natura del Buddha alla quale Egli si
risvegliò. Essa non seguiva i desideri degli esseri mondani. Galleggiava
contro la corrente della sua mente, era in ogni modo a essa contraria.

Al giorno d'oggi, allo stesso modo, l'insegnamento del Buddha va in
senso contrario rispetto al nostro cuore. La gente vuole indulgere
all'avidità e all'odio, ma il Buddha non lo consente. Vuole le
illusioni, ma il Buddha distrugge le illusioni. Per questo la mente del
Buddha è antitetica rispetto a quella degli esseri mondani. Il mondo
dice che il corpo è bello, Egli dice che non lo è. La gente dice che il
corpo ci appartiene, Egli dice che non è così. Dice che è sostanziale,
Egli dice di no. La Retta Visione è al di là del mondo. Gli esseri
mondani si limitano a seguire il flusso della corrente.

Il Buddha ricevette poi otto manciate d'erba da un brahmano. Le otto
manciate d'erba -- questo è il loro vero significato -- sono gli otto
\emph{dhamma} mondani: guadagno e perdita, lode e biasimo, fama e
discredito, felicità e infelicità. Il Buddha, dopo aver ricevuto l'erba,
decise di sedercisi sopra e di entrare in \emph{samādhi}. L'atto di
sedersi sull'erba era il \emph{samādhi} stesso, la sua mente che stava
al di sopra dei \emph{dhamma} mondani e che sottometteva il mondo per
realizzare la trascendenza. I \emph{dhamma} mondani furono per Lui come
dei rifiuti, persero ogni significato. Ci si sedette sopra, senza che
essi ostruissero in alcun modo la sua mente. Alcuni demoni giunsero per
cercare di sopraffarlo, ma Egli rimase seduto in \emph{samādhi},
soggiogando il mondo, fino a che si illuminò al Dhamma e sconfisse del
tutto Māra. Ossia sconfisse il mondo. La pratica per sviluppare il
Sentiero è ciò che uccide le contaminazioni.

Attualmente la gente ha poca fede. Dopo aver praticato un anno o due
vogliono arrivare a quel punto, vogliono procedere alla svelta. Non
considerano che il Buddha, il nostro Maestro, aveva lasciato casa ben
sei anni prima di diventare Illuminato. Questa è la ragione per cui noi
abbiamo la ``libertà dalla dipendenza''.\footnote{Un giovane monaco deve
  essere ``dipendente'', ossia deve vivere sotto la guida di un monaco
  più anziano per almeno cinque anni.} Secondo le Scritture, un monaco
deve avere almeno cinque Piogge\footnote{``Piogge'' si riferisce qui ai
  ritiri annuali di tre mesi mediante i quali i monaci contano la loro
  anzianità; perciò, un monaco con cinque Piogge ha ricevuto
  l'ordinazione monastica da cinque anni.} prima di poter essere
considerato in grado di vivere da solo. Ha studiato e praticato a
sufficienza, ha un'adeguata conoscenza, ha fede, ha un buon
comportamento. Dico che chi pratica per cinque anni è capace e
competente. Deve però aver praticato davvero, non deve aver solo passato
il tempo a indossare l'abito monastico per cinque anni. Deve essersi
realmente preso cura della pratica, averla svolta davvero. Fino a quando
non raggiungete cinque Piogge, potreste chiedervi: «~Cos'è questa
``libertà dalla dipendenza'' di cui parlò il Buddha?~» Dovreste
veramente cercare di praticare per cinque anni e poi conoscerete da voi
stessi le qualità alle quali faccio riferimento. Dopo di allora dovreste
essere competenti, competenti nella mente, sicuri. Dopo cinque Piogge si
dovrebbe aver raggiunto almeno il primo livello dell'Illuminazione. Ciò
significa non solo cinque Piogge nel corpo, ma pure cinque Piogge nella
mente. Un monaco così teme di essere biasimato, ha un senso di vergogna
e di modestia. Non osa fare cose sbagliate né al cospetto della gente né
alle loro spalle, alla luce del sole o di notte. Perché? Perché ha
raggiunto il Buddha, ``Colui che Conosce''. Prende rifugio nel Buddha,
nel Dhamma e nel Saṅgha.

Per poter fare davvero affidamento sul Buddha, sul Dhamma e sul Saṅgha
dobbiamo vedere il Buddha. Quale utilità avrebbe prendere rifugio senza
conoscere il Buddha? Se non conosciamo ancora il Buddha, il Dhamma e il
Saṅgha, la mente non li ha ancora raggiunti e il rifugiarsi in essi è
solamente un atto del corpo e della parola. Quando la mente li
raggiunge, sappiamo come sono il Buddha, il Dhamma e il Saṅgha. Allora
possiamo davvero prendere rifugio in essi, perché queste cose sorgono
nella nostra mente. Ovunque ci troveremo avremo con noi il Buddha, il
Dhamma e il Saṅgha.

Chi è così non osa compiere cattive azioni. Questa è la ragione per cui
diciamo che chi ha raggiunto il primo stadio dell'Illuminazione non
nascerà più in stati sventurati. La sua mente è certa, è ``entrata nella
Corrente'', per lui non ci sono più dubbi. Se non ottiene oggi
l'Illuminazione, questo certamente avverrà prima o poi, in futuro. Può
sbagliare qualcosa, ma non sarà mai abbastanza per farlo finire
nell'Inferno, ossia egli non regredirà compiendo malvagità con il corpo
e con la parola, ne è incapace. Per questo motivo diciamo che ha fatto
il suo ingresso nella Nobile Nascita. Non può tornare in infime
condizioni d'esistenza. Si tratta di una cosa che dovreste vedere da voi
stessi proprio in questa vita. Al giorno d'oggi coloro fra noi che hanno
ancora dei dubbi sulla pratica ascoltano queste cose e dicono: «~Come
posso farcela?~» A volte siamo felici, altre volte preoccupati,
compiaciuti o dispiaciuti. Per quale motivo? Perché non conosciamo il
Dhamma. Quale Dhamma? Solo il Dhamma della natura, la realtà attorno a
noi, il corpo e la mente.

Il Buddha disse: «~Non attaccatevi ai cinque \emph{khandhā}, lasciateli
andare, rinunciate a essi!~» Perché non riusciamo a lasciarli andare?
Perché non li vediamo e non li comprendiamo del tutto. Li vediamo come
se fossero noi stessi, vediamo noi stessi nei \emph{khandhā}. Vediamo
felicità e sofferenza come noi stessi, vediamo noi stessi nella felicità
e nella sofferenza. Non riusciamo a separare noi stessi da queste cose.
Questo significa che non vediamo il Dhamma, che non vediamo la natura.
Felicità, infelicità, piacere e tristezza: queste cose non sono noi, ma
è così che le consideriamo. Esse entrano in contatto con noi e le
vediamo come un grumo di \emph{attā},\footnote{\emph{Attā}: Io o sé,
  sostanziale, personale; a volte inteso con il senso di anima.} un
grumo del sé. Tutte le volte che c'è il sé, lì troverete felicità,
infelicità e così via. Per questo il Buddha disse di distruggere questo
``grumo'' dell'io, ossia di distruggere \emph{sakkāya-ditthi}. Quando
\emph{attā}, il sé è distrutto, sorge in modo naturale \emph{anattā}, il
non-sé.

Pensiamo che la natura sia noi e che noi stessi siamo la natura, e
perciò la natura non la conosciamo davvero. Se va bene ridiamo, se va
male piangiamo, ma la natura è semplicemente \emph{saṅkhāra}. Mentre
recitiamo i canti diciamo: \emph{Tesam vūpasamo sukho}, la vera felicità
è la pacificazione dei \emph{saṅkhāra}. Come li pacifichiamo?
Semplicemente rimuovendo l'attaccamento e vedendoli per quello che
davvero sono. Così, in questo mondo c'è la Verità. Alberi, montagne e
piante vivono tutti secondo la loro propria verità, nascono e muoiono
seguendo la loro natura. Siamo noi esseri umani a non essere veritieri.
Vediamo tutte queste cose e ci agitiamo a più non posso, ma la natura
resta impassibile, è così com'è. Ridiamo, piangiamo, uccidiamo, ma la
natura resta nella Verità, è la Verità. Non importa quanto si possa
essere felici o tristi, questo corpo segue solo la propria natura. È
nato, cresce e invecchia, cambiando in continuazione. Segue la natura in
questo modo. Chiunque pensa al corpo e se lo porta dietro come se fosse
se stesso, soffrirà.

Per questa ragione Aññā Kondañña riconobbe che ogni cosa rientra in
questo ``qualsiasi cosa nasca'', materiale o immateriale che sia. Il suo
modo di vedere il mondo cambiò. Vide la Verità. Si rialzò dal luogo in
cui stava seduto e portò quella Verità con sé. Il nascere e il morire
continuavano, ma lui si limitava a stare a guardare. Felicità e
infelicità sorgevano e svanivano, ma lui si limitava a notarle. La sua
mente era stabile. Non cadeva più in stati mentali infimi. Non era mai
troppo contento né eccessivamente turbato per queste cose. La sua mente
era saldamente fondata nell'attività della contemplazione. Ecco! In Aññā
Kondañña si era schiuso l'Occhio del Dhamma. Vedeva la natura, che noi
chiamiamo \emph{saṅkhāra}, secondo la Verità. Ciò che conosce la verità
dei \emph{saṅkhāra} è la saggezza. Questa è la mente che conosce e vede
il Dhamma, la mente che si è arresa. Fino a quando non vediamo il Dhamma
dobbiamo essere pazienti e contenuti. Perché dobbiamo essere diligenti?
Perché siamo pigri! Perché dobbiamo sviluppare la sopportazione? Perché
non riusciamo a sopportare! È così. Quando però siamo fondati nella
nostra pratica, ci liberiamo dalla pigrizia e perciò non abbiamo bisogno
di essere diligenti. Se già conosciamo la verità di tutti gli stati
mentali, se non ci fanno diventare felici o infelici, non abbiamo
bisogno di sopportare, perché la mente è già Dhamma. ``Colui che
Conosce'' ha visto il Dhamma, è il Dhamma.

Quando la mente è Dhamma, si ferma. Ha raggiunto la pace. Non c'è più
bisogno di fare nulla di particolare, perché la mente è già Dhamma.
L'esterno è Dhamma, l'interno è Dhamma. ``Colui che Conosce'' è Dhamma.
Lo stato mentale è Dhamma e ciò che conosce lo stato mentale è Dhamma. È
uno. È libero. Questa natura non è nata, non invecchia né si ammala.
Questa natura non muore. Questa natura non è felice né triste, non è
grande né piccola, non è pesante né leggera; non è corta né lunga, né
nera né bianca. Non la si può paragonare a nulla. Non c'è convenzione
che possa raggiungerla. Questa è la ragione per cui diciamo che il
\emph{Nibbāna} non ha colore. Tutti i colori sono solo convenzioni. Lo
stato che è al di là del mondo è al di là della portata delle
convenzioni del mondo.

Per questo il Dhamma è ciò che è al di là del mondo. È quello che ognuno
dovrebbe vedere da se stesso. È al di là del linguaggio. Non è possibile
esprimerlo con le parole, si può solo parlarne in relazione ai modi e ai
mezzi per realizzarlo. Chi lo ha visto da sé ha terminato il suo lavoro.

