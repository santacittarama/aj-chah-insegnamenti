\chapter{Un dono di Dhamma}

\begin{openingQuote}
  \centering

  Discorso pronunciato per i monaci occidentali, i novizi e i discepoli laici
  riuniti nel monastero della foresta Bung Wai a Ubon, il 10 ottobre 1977; il
  discorso fu offerto ai genitori di un monaco, che lo erano venuti a trovare
  dalla Francia.
\end{openingQuote}

Sono felice che abbiate colto quest'opportunità per venire a visitare il
Wat Nong Pah Pong e per vedere vostro figlio, che è monaco qui. Mi
dispiace però di non aver alcun dono da offrirvi. In Francia ci sono già
così tante cose materiali, ma di Dhamma ce n'è proprio poco. Ci sono
stato e ho visto personalmente che lì in verità non c'è alcun Dhamma che
possa condurre alla pace e alla tranquillità. Ci sono solo cose che
rendono la mente sempre confusa e agitata. La Francia è già ricca da un
punto di vista materiale, ha da offrire molte cose allettanti per i
sensi. Cose da vedere e da toccare, suoni, odori e sapori. Tutto questo
può solo confondere chi ignora il Dhamma. Perciò, oggi vi offrirò un po'
di Dhamma da portare in Francia come dono del Wat Nong Pah Pong e del
Wat Pah Nanachat.

Che cos'è il Dhamma? Il Dhamma è ciò che può recidere i problemi e le
difficoltà del genere umano, riducendoli progressivamente fino ad
annullarli. Questo è quel che viene chiamato Dhamma, e questo è quel che
dovrebbe essere studiato durante la nostra vita quotidiana in modo che,
quando in noi sorge un'impressione mentale, possiamo essere in grado di
affrontarla e trascenderla. Che si viva qui in Thailandia oppure in
altre nazioni, abbiamo tutti gli stessi problemi. Se non sappiamo come
risolverli, saremo sempre soggetti alla sofferenza e all'angoscia. Quel
che risolve i problemi è la saggezza, e per avere saggezza dobbiamo
sviluppare e addestrare la mente. L'oggetto della pratica non è molto
lontano, è proprio qui nel nostro corpo e nella nostra mente. Gli
occidentali e i thailandesi sono uguali, gli uni e gli altri hanno un
corpo e una mente. Un corpo e una mente confusa significano una persona
confusa, un corpo e una mente serena significano una persona serena.

In realtà, nella sua condizione naturale la mente è pura, come l'acqua
piovana. Se nella limpida acqua piovana mettiamo delle gocce di colore
verde, essa diverrà verde. Se ne mettiamo di gialle, diverrà gialla. La
mente reagisce in modo simile. Quando un'impressione mentale gradevole
``gocciola'' nella mente, la mente è a suo agio. Quando l'impressione
mentale è sgradevole, essa è a disagio. La mente si tinge, proprio come
l'acqua quando viene colorata. Quando l'acqua limpida entra in contatto
con il giallo, diventa gialla. Quando entra in contatto con il verde,
diventa verde. Cambia sempre colore. In realtà l'acqua, verde o gialla
che sia, è naturalmente pulita e limpida. Così è pure lo stato naturale
della mente. Pulito, puro e non confuso. La mente diventa confusa solo
perché segue le impressioni mentali. Si perde negli stati mentali!

Consentitemi di spiegarlo con maggior chiarezza. Proprio in questo
momento siamo seduti in una tranquilla foresta. Se non c'è vento qui le
foglie restano immobili. Quando il vento soffia, si agitano e
ondeggiano. La mente è come le foglie. Quando entra in contatto con
un'impressione mentale, anch'essa si ``agita e ondeggia'' in sintonia
con la natura di quell'impressione mentale. Meno sappiamo di Dhamma, più
la mente seguirà in continuazione le impressioni mentali. Se prova
felicità, sarà sopraffatta dalla felicità. Se prova sofferenza, sarà
sopraffatta dalla sofferenza. C'è confusione continua!

Alla fine le persone diventano nevrotiche. Perché? Perché non sanno!
Seguono solo i loro stati d'animo e non sanno come prendersi cura della
loro mente. Quando nessuno si prende cura della mente, essa è come un
bimbo privo di una madre o di un padre che lo accudisca. Un orfano non
ha rifugio e, senza un rifugio, si sente molto insicuro. Allo stesso
modo, la mente è davvero ossessiva se non viene sorvegliata, se non vi è
un addestramento o la maturazione di un carattere dotato di retta
comprensione.

Il metodo di addestramento della mente che oggi vi offrirò è il
\emph{kammaṭṭhāna}.\footnote{\emph{Kammaṭṭhāna:} Letteralmente, ``base
  di lavoro'' o ``luogo di lavoro'', metodo meditativo.}
Kamma\footnote{Kamma: Atto intenzionale compiuto per mezzo
  del corpo, della parola o della mente, il quale conduce sempre a un
  effetto (\emph{kamma-vipāka}).} significa ``azione'' e \emph{ṭhāna}
significa ``base''. Nel buddhismo, è il metodo per rendere la mente
serena e tranquilla. Serve ad addestrare la mente e poi a investigare il
corpo con mente addestrata. Il nostro essere è composto da due parti:
una è il corpo, l'altra la mente. Ci sono solo queste due parti. Quel
che è chiamato corpo può essere visto con gli occhi fisici. D'altro
canto, la ``mente'' non ha un aspetto fisico. La mente può essere vista
solo con l'``occhio interno'' o ``occhio della mente''. Queste due cose,
il corpo e la mente, sono in uno stato di costante agitazione.

Che cos'è la mente? In realtà la mente non è affatto una ``cosa''.
Convenzionalmente parlando, è ciò che percepisce e sente. Ciò che
percepisce, riceve e sperimenta tutte le impressioni mentali è chiamata
``mente''. La mente c'è proprio in questo momento. Mentre parlo con voi,
la mente riconosce cosa sto dicendo. I suoni entrano attraverso gli
orecchi e voi sapete quello che viene detto. Ciò che esperisce tutto
questo è chiamata ``mente''. Nella mente non c'è alcun sé o sostanza.
Non ha alcuna forma. Esperisce solo le attività mentali, questo è tutto!
Se insegniamo alla mente la Retta Visione,\footnote{Retta Visione
  (\emph{sammā-diṭṭhi}): La Retta Visione è il primo fattore del Nobile
  Ottuplice Sentiero.} essa non avrà alcun problema. Si sentirà a
proprio agio. La mente è la mente. Gli oggetti mentali sono gli oggetti
mentali. Gli oggetti mentali non sono la mente, la mente non è gli
oggetti mentali. Per comprendere con chiarezza la nostra mente e gli
oggetti mentali al suo interno, diciamo che la mente è ciò che riceve
gli oggetti mentali che in essa emergono. Quando queste due cose, la
mente e i suoi oggetti, entrano in contatto l'una con gli altri sorgono
le sensazioni. Alcune sono buone, altre cattive, alcune fredde, altre
roventi \ldots{} ce n'è di tutti i tipi! La mente sarà ovviamente agitata se
affrontiamo queste sensazioni privi di saggezza.

La meditazione è la via per sviluppare la mente, la base per il sorgere
della saggezza. Il respiro è il fondamento fisiologico. Lo chiamiamo
\emph{ānāpānasati}\footnote{\emph{Ānāpānasati:} Letteralmente,
  ``consapevolezza dell'inspirazione e dell'espirazione''.} o
``consapevolezza del respiro''. Assumiamo quest'oggetto di meditazione
perché è il più semplice e fin dai tempi antichi ha rappresentato il
cuore della meditazione.

Quando vi è una buona occasione per fare la meditazione seduta, sedete a
gambe incrociate: la gamba destra sulla gamba sinistra e la mano destra
sulla mano sinistra. Tenete la schiena diritta, eretta. Dite a voi
stessi: «~Ora lascerò andare tutte le mie preoccupazioni e
responsabilità.~» Dovete volere che non ci sia niente a causarvi
preoccupazione. Per il momento lasciate andare tutte le vostre
responsabilità. Concentrate la vostra attenzione sul respiro. Poi
inspirate ed espirate. Mentre sviluppate la consapevolezza del respiro,
non rendete intenzionalmente il respiro lungo o corto, e neanche pesante
o leggero. Lasciatelo fluire normalmente e naturalmente. Sorgendo dalla
mente, la consapevolezza e la presenza mentale conosceranno
l'inspirazione e l'espirazione.

Siate a vostro agio. Non pensate a nulla. Non c'è bisogno di pensare a
questo o a quello. L'unica cosa che dovete fare è concentrare la vostra
attenzione sull'inspirazione e sull'espirazione. Non dovete fare altro!
Mantenete la vostra consapevolezza fissa sull'inspirazione e
sull'espirazione. Siate consapevoli dell'inizio, della metà e della fine
di ogni respiro. Durante l'inspirazione, l'inizio del respiro è sulla
punta del naso, alla metà è all'altezza del cuore e alla fine
nell'addome. Durante l'espirazione, è il contrario: l'inizio del respiro
è nell'addome, alla metà è all'altezza del cuore e alla fine sulla punta
del naso. Sviluppate la consapevolezza del respiro. Uno, sulla punta del
naso. Due, all'altezza del cuore. Tre, nell'addome. Poi al contrario.
Uno, nell'addome. Due, all'altezza del cuore. Tre, sulla punta del naso.

Focalizzare l'attenzione su questi tre punti allevierà ogni vostra
preoccupazione. Non pensate a nient'altro. Mantenete la vostra
attenzione sul respiro. Altri pensieri entreranno forse nella mente, ed
essa sarà assorta su altre cose che vi distrarranno. Non preoccupatevi.
Tornate nuovamente al respiro quale oggetto della vostra attenzione. La
mente potrebbe essere coinvolta nel giudizio e nell'investigazione del
vostro stato mentale, ma continuate a praticare, a essere costantemente
attenti dell'inizio, della metà e della fine di ogni respiro.

Col passare del tempo, in questi tre punti la mente sarà sempre
consapevole del respiro. Praticando così per un po', la mente e il corpo
si abitueranno a questo lavoro. La fatica sparirà. Il corpo si sentirà
più leggero e il respiro diventerà sempre più sottile. La consapevolezza
e la presenza mentale proteggeranno la mente e la sorveglieranno.
Pratichiamo in questo modo finché la mente è calma e serena, unificata.
``Unificata'' significa che la mente è completamente assorta nel
respiro, che non si separa dal respiro. La mente non sarà confusa e si
sentirà a suo agio. Conoscerà l'inizio, la metà e la fine del respiro e
resterà fermamente fissa su di esso.

Quando la mente è calma, concentriamo la nostra attenzione
sull'inspirazione e sull'espirazione solo sulla punta del naso. Non
dobbiamo seguire il respiro su e giù, fino all'addome e viceversa.
Concentratevi solo sulla punta del naso, dove il respiro entra ed esce.
Questo si chiama ``calmare la mente'', renderla rilassata e serena.
Quando sorge la tranquillità, la mente si ferma, si ferma sul suo unico
oggetto, il respiro. Ciò significa rendere serena la mente per far
sorgere la saggezza.

È l'inizio, il fondamento della nostra pratica. Dovreste cercare di
praticare in questo modo ogni giorno, ovunque siate. A casa, in
automobile, distesi o seduti, dovreste essere consapevoli e presenti a
voi stessi, sorvegliare la mente con costanza. Questo è chiamato
addestramento mentale e dovrebbe essere praticato in tutte e quattro lo
posture. Non solo seduti, bensì anche in piedi, camminando e distesi. Il
punto è che dovremmo sapere in ogni momento qual è il nostro stato
mentale, e per essere in grado di saperlo dobbiamo essere costantemente
consapevoli e mentalmente presenti. La mente è felice o sta soffrendo? È
confusa? È serena? Conoscere la mente in questo modo le consente di
tranquillizzarsi e quando essa si tranquillizzerà, sorgerà la saggezza.

Con la mente tranquilla, investigate l'oggetto di meditazione -- il
corpo -- dalla sommità del capo fino alla punta dei piedi, e poi
all'inverso. Fatelo ripetutamente. Osservate e vedrete i capelli, i
peli, le unghie, i denti e la pelle. Con questa meditazione vedremo che
tutto il corpo è composto di quattro ``elementi'': terra, acqua, fuoco e
vento. L'elemento terra è le parti solide e compatte, l'elemento acqua
quelle liquide, che fluiscono. I venti che scorrono su e giù attraverso
il nostro corpo sono l'elemento vento, e il calore del nostro corpo è
l'elemento fuoco. Presi nel loro insieme, compongono quel che chiamiamo
un ``essere umano''.

Quando le parti che compongono il corpo si disgregano, restano solo
questi quattro elementi. Il Buddha insegnò che di per sé non c'è alcun
``essere'', né umano, né thailandese, né occidentale, non c'è alcuna
persona ma, in ultima analisi, ci sono solo questi quattro elementi, e
questo è tutto! Presumiamo che ci sia una persona o un ``essere'', ma in
realtà non c'è nulla del genere. Presi separatamente come terra, acqua,
fuoco e vento, o etichettati assieme in quanto forma di un ``essere
umano'', sono tutti quanti impermanenti, soggetti alla sofferenza e
non-sé. Sono tutti quanti instabili, incerti e in uno stato di perenne
cambiamento: non sono stabili neanche un attimo!

Il nostro corpo è instabile, si altera e cambia in continuazione. I
capelli e i peli cambiano, le unghie cambiano, i denti cambiano, la
pelle cambia, tutto cambia completamente! Anche la nostra mente cambia
in continuazione. Non è un sé o qualcosa di sostanziale. Sebbene si
possa pensare che sia così, non è realmente ``noi'', non è realmente
``loro''. Forse può pensare al suicidio, oppure alla felicità o alla
sofferenza: a qualsiasi cosa! È instabile. Se non abbiamo saggezza e
crediamo alla nostra mente, ci mentirà in continuazione, e in noi si
alterneranno sofferenza e felicità. Questa mente è una cosa incerta. Il
corpo è una cosa incerta. Entrambi sono impermanenti. Entrambi sono
fonte di sofferenza. Entrambi sono privi di un sé. Il Buddha evidenziò
che non sono un essere e neanche una persona, né un sé, né un'anima, né
``noi'' e nemmeno ``loro''. Sono semplicemente elementi: terra, acqua,
fuoco e vento. Solo elementi!

Quando la mente vedrà tutto questo, si sbarazzerà dell'attaccamento che
le fa ritenere ``io'' sono bello, ``io'' sono buono, ``io'' sono
cattivo, ``io'' sto soffrendo, ``io'' ho, ``io'' questo e ``io'' quello.
Sperimenterete uno stato di unificazione, perché avrete capito che
tutto il genere umano è fondamentalmente uguale. Non c'è alcun ``io''.
Ci sono solo elementi. Quando contemplerete e vedrete l'impermanenza, la
sofferenza e il non-sé, non ci sarà più attaccamento al sé, a un essere,
a un ``io'', un ``lui'' o una ``lei''. Nella mente che li vede sorgerà
\emph{nibbidā}, il disincanto e il distacco. Vedrà tutte le cose solo
come impermanenti, come sofferenza e come prive di un sé. Allora la
mente si fermerà. La mente è Dhamma. Avidità, odio e illusione
diminuiranno e svaniranno poco a poco fino a che, infine, resterà solo
la mente, solo la mente pura. Questo si chiama ``praticare la
meditazione''.

Vi chiedo perciò di prendere questo dono di Dhamma che vi offro, per
studiarlo e contemplarlo nella vostra vita quotidiana. Accettate per
favore questo insegnamento di Dhamma come un'eredità tramandatavi dal
Wat Pah Pong e dal Wat Pah Nanachat. Tutti i monaci qui presenti,
compreso vostro figlio e tutti gli insegnanti, ve lo offrono affinché lo
portiate con voi in Francia. V'indicherà la via per la pace della mente,
renderà la vostra mente calma e non confusa. Forse il vostro corpo potrà
essere in subbuglio, ma non la vostra mente. Gli altri che vivono nel
mondo potranno essere confusi, ma voi non lo sarete. Anche se nella
vostra nazione c'è confusione, voi non sarete confusi, perché la mente
avrà visto. La mente è Dhamma. Questo è il Retto Sentiero, la vera via.

Possiate ricordare questo insegnamento nel futuro. Possiate stare bene
ed essere felici.

