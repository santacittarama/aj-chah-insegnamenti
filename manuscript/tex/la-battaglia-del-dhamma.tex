\chapter{La battaglia del Dhamma}

\begin{openingQuote}
  \centering

  Estratti da un discorso offerto a monaci e novizi al Wat Pah Pong.
\end{openingQuote}

Combattete l'avidità, combattete l'avversione, combattete l'illusione.
Questi sono i nemici. Nella pratica del buddhismo, il Sentiero del
Buddha, combattiamo mediante il Dhamma con paziente tolleranza.
Combattiamo resistendo ai nostri innumerevoli stati mentali.

Il Dhamma e il mondo sono correlati. Dove c'è il Dhamma c'è il mondo,
dove c'è il mondo c'è il Dhamma. Dove ci sono le contaminazioni ci sono
coloro che vincono le contaminazioni, combattiamo contro di esse. Questo
si chiama combattere interiormente. Per combattere esteriormente la
gente getta bombe o spara con i fucili. Vincono e sono vinti. Vincere
gli altri è la via del mondo. Nella pratica del Dhamma non dobbiamo
combattere gli altri, ma vincere la nostra stessa mente, resistendo
pazientemente a tutti i nostri stati mentali.

Quando si tratta della pratica del Dhamma, non nutriamo rancori e
inimicizie, ma nelle nostre azioni e nei nostri pensieri lasciamo andare
tutte le forme di malanimo, liberandoci dalla gelosia, dall'avversione
e dal risentimento. L'odio può essere superato solo non nutrendo
risentimenti e rancori. Azioni nocive e vendette sono cose diverse, ma
strettamente collegate. Le azioni, una volta compiute, sono terminate.
Non c'è bisogno di rispondere con vendette e ostilità. Questa è chiamata
``azione'' (kamma).\footnote{Kamma: Atto intenzionale
  compiuto per mezzo del corpo, della parola o della mente, il quale
  conduce sempre a un effetto (\emph{kamma-vipāka}).} ``Vendetta''
(\emph{vera}) significa continuare ulteriormente quell'azione con
pensieri del tipo: «~Tu lo hai fatto a me e io lo faccio a te.~» Queste
cose non hanno fine. Ciò conduce alla continua ricerca di vendetta, e
così l'odio non è mai abbandonato. Finché ci comportiamo in questo modo,
la catena è ininterrotta, non c'è fine. Non importa dove andiamo, la
faida continua.

Il supremo Maestro\footnote{Il Buddha.} insegnò al mondo. Provò
compassione per tutti gli esseri del mondo. Il mondo, però, continua
ugualmente ad andare avanti così. Il saggio dovrebbe guardare dentro il
mondo e scegliere le cose che davvero hanno valore. Il Buddha si
addestrò nelle varie arti della guerra quando era un principe, ma vide
che non erano veramente utili. Erano limitate al mondo, con le sue
battaglie e aggressioni. Perciò, noi che abbiamo lasciato il mondo,
abbiamo bisogno di addestrare noi stessi. Dobbiamo imparare a rinunciare
a tutte le forme del male, a rinunciare a tutte quelle cose che sono
causa di inimicizia. Conquistiamo noi stessi, non abbiamo bisogno di
conquistare gli altri. Combattiamo, ma combattiamo solo le
contaminazioni. Se c'è avidità, noi la combattiamo. Se c'è avversione,
noi la combattiamo. Se c'è illusione, ci sforziamo di rinunciarvi.
Questa è chiamata ``battaglia del Dhamma''. Questa guerra del cuore è
davvero difficile, è la cosa più difficile di tutte. Diventiamo monaci
per contemplare tutto questo, per imparare l'arte di combattere
l'avidità, l'avversione e l'illusione. Questa è la nostra responsabilità
primaria. Questa è la battaglia interiore, combattere contro le
contaminazioni. Sono però pochissime le persone che combattono in questo
modo. La maggior parte della gente combatte contro altre cose, raramente
combatte le contaminazioni, e di rado perfino le vede.

Il Buddha ci ha insegnato ad abbandonare tutte le forme del male e a
coltivare le virtù. Questo è il Retto Sentiero. Insegnando in questo
modo, è come se il Buddha ci prendesse e ci mettesse all'inizio del
Sentiero. Raggiunto il Sentiero, percorrerlo o meno dipende da noi. Il
lavoro del Buddha finisce lì. Egli indica la strada, quel che è giusto e
quel che è sbagliato. È abbastanza, il resto dipende da noi. Dopo aver
raggiunto il Sentiero non sappiamo ancora nulla, non abbiamo visto
ancora nulla, è per questo che dobbiamo imparare. Per imparare dobbiamo
essere pronti a sopportare alcune difficoltà, proprio come gli studenti
del mondo. È abbastanza difficile acquisire la conoscenza e imparare il
necessario per chi deve fare carriera. Devono resistere. Quando pensano
in modo errato, provando avversione o pigrizia, devono sforzarsi di
continuare prima per laurearsi e, poi, per avere un lavoro. La pratica
di un monaco è simile. Se decidiamo di praticare e di contemplare,
allora vedremo certamente la via.

\emph{Diṭṭhi-māna} è una cosa nociva. \emph{Diṭṭhi} significa
``visione'' o ``opinione''. Tutte le forme di visione sono
\emph{diṭṭhi:} vedere il bene come male, vedere il male come bene, quale
che sia il modo in cui vediamo le cose, il problema non sta qui. Il
problema sta nell'attaccamento a questi modi di vedere, chiamato
\emph{māna},\footnote{\emph{Māna:} Presunzione, orgoglio.}
nell'aggrapparsi a questi punti di vista come se fossero la verità.
Proprio a causa dell'attaccamento ciò conduce a girare in tondo dalla
nascita fino alla morte, senza raggiungere mai un compimento. Perciò il
Buddha ci esortò a lasciar andare i punti di vista.

Quando molte persone vivono insieme come facciamo noi qui, è ancora
possibile praticare agevolmente se i modi di vedere sono in armonia.
Però, anche solo due o tre monaci avrebbero difficoltà a vivere insieme
se i loro punti di vista non fossero buoni o in armonia. Se siamo umili
e lasciamo andare i nostri punti di vista, anche se siamo in molti
arriviamo insieme nel luogo del Buddha, del Dhamma e del
Saṅgha.\footnote{\emph{Tiratana:} La ``Triplice Gemma'', composta dal
  Buddha, dal Dhamma e dal Saṅgha.} Non è giusto dire che l'armonia non
può esserci solo perché siamo in molti. Guardate un millepiedi. Un
millepiedi ha molte zampe, vero? Guardandolo si potrebbe pensare che
possa avere delle difficoltà a camminare, ma in realtà non è così. Ha un
suo assetto e un suo ritmo. Nella nostra pratica è la stessa cosa.

Se pratichiamo come praticò il Nobile Saṅgha del Buddha, allora è
facile. Cioè \emph{supaṭipanno}, coloro che praticano bene;
\emph{ujupaṭipanno}, coloro che praticano rettamente;
\emph{ñāyapaṭipanno}, coloro che praticano per trascendere la
sofferenza; e \emph{sāmīcipaṭipanno}, coloro che praticano propriamente.
Sono queste quattro qualità che, allorché dimorano dentro di noi, ci
rendono veri membri del Saṅgha. Anche se siamo centinaia o migliaia, non
importa quanti siamo, percorriamo tutti lo stesso Sentiero. Proveniamo
da ambienti diversi, ma siamo uguali. I nostri punti di vista possono
essere differenti, ma se pratichiamo correttamente non ci saranno
attriti. Proprio come tutti i fiumi e tutti i corsi d'acqua che scorrono
verso il mare, che appena entrano in mare hanno tutti lo stesso sapore e
lo stesso colore. Con le persone è la stessa cosa. Quando entrano nella
Corrente del Dhamma, è un solo Dhamma. Anche se provengono da luoghi
diversi, si armonizzano, si fondono. Il pensiero che induce tutte le
dispute e tutti i conflitti è \emph{diṭṭhi-māna}. Per questo il Buddha
ci ha insegnato a lasciar andare i punti di vista. Al di là della
rilevanza di questi punti di vista, non consentite a \emph{māna} di
attaccarsi a essi.

Il Buddha insegnò l'importanza di un'ininterrotta \emph{sati},\footnote{\emph{Sati:}
  Consapevolezza, presenza mentale, attenzione.} la rammemorazione. Sia
in piedi, mentre camminiamo, sia seduti o distesi, ovunque siamo,
dovremmo avere questa capacità di rammemorazione. Quando abbiamo
\emph{sati} vediamo noi stessi, vediamo la nostra mente. Vediamo il
``corpo nel corpo'', ``la mente nella mente''. Se non abbiamo
\emph{sati} non conosciamo nulla, non siamo consapevoli di quello che
sta succedendo. È per questo che \emph{sati} è davvero importante. Con
costante \emph{sati} ascoltiamo sempre il Dhamma del Buddha. Ciò avviene
perché ``l'occhio che vede le forme'' è Dhamma, ``l'orecchio che ode i
suoni'' è Dhamma, ``il corpo che prova sensazioni'' è Dhamma. Quando
delle impressioni sorgono nella mente, anche questo è Dhamma. Perciò,
chi ha costante \emph{sati} ascolta sempre l'insegnamento del Buddha. Il
Dhamma è sempre lì. Perché? A causa di \emph{sati}, perché siamo
consapevoli. \emph{Sati} è rammemorazione, \emph{sati-sampajañña} è
consapevolezza di sé. Questa consapevolezza è il vero \emph{Buddho}, il
Buddha.\footnote{Buddha (\emph{Buddho}): Letteralmente, ``Risvegliato'',
  ``Illuminato''. Questa parola viene anche usata per la meditazione,
  recitando interiormente \emph{Bud-} nel corso dell'inspirazione e
  \emph{-dho} durante l'espirazione.} Quando c'è \emph{sati-sampajañña},
seguirà la comprensione. Sappiamo cosa sta succedendo. Quando l'occhio
vede le forme, è opportuno che lo faccia o no? Quando l'orecchio ode un
suono, è appropriato o no? È dannoso? È giusto? È sbagliato? E così via
in questo modo con ogni cosa. Se comprendiamo ascoltiamo sempre il
Dhamma.

Cerchiamo allora tutti di capire che, proprio ora, stiamo imparando nel
bel mezzo del Dhamma. Incontriamo il Dhamma sia che andiamo avanti sia
che facciamo un passo indietro. È tutto Dhamma se abbiamo \emph{sati}.
Anche vedendo gli animali che corrono nella foresta possiamo riflettere,
vedendo che tutti quegli animali sono uguali a noi. Scappano dalla
sofferenza e inseguono la felicità, proprio come fa la gente. Evitano
tutto ciò che a loro non piace. Hanno paura di morire, proprio come la
gente. Se riflettiamo su questo, vediamo che tutti gli esseri del mondo,
anche le persone, con le loro varie istintualità, sono uguali. Pensare
in questo modo è chiamato \emph{bhāvanā},\footnote{\emph{Bhāvanā:}
  Meditazione, sviluppo o coltivazione.} vedere in accordo con la
Verità, vedere che tutti gli esseri sono compagni nella vecchiaia, nella
malattia e nella morte. Se veramente vediamo le cose così come sono, la
nostra mente rinuncerà ad attaccarsi a esse. Per questo è stato detto
che dobbiamo avere \emph{sati}. Se abbiamo \emph{sati} vedremo lo stato
della nostra mente. Dobbiamo conoscere ogni nostro pensiero e ogni
nostra sensazione. Questa conoscenza è chiamata \emph{Buddho}, il
Buddha, ``Colui che Conosce'',\footnote{Colui che Conosce: La qualità
  della presenza mentale, quella facoltà della mente che, se rettamente
  coltivata, conduce alla Liberazione.} che conosce a fondo, che conosce
con chiarezza e completamente. Quando la mente conosce completamente
troviamo la retta pratica. Perciò il modo giusto di praticare è avere
consapevolezza, \emph{sati}. Se siete privi di \emph{sati} per cinque
minuti, siete pazzi per cinque minuti, distratti per cinque minuti.
Tutte le volte che mancate di \emph{sati} siete folli. Per questo
\emph{sati} è essenziale. Avere \emph{sati} significa conoscere se
stessi, conoscere le condizioni della vostra mente e la vostra vita.
Questo è avere comprensione e discernimento, ascoltare sempre il Dhamma.
Dopo aver smesso di sentire il discorso dell'insegnante, sentite ancora
il Dhamma, perché il Dhamma è ovunque.

Per questa ragione tutti voi dovete assicurarvi di praticare ogni
giorno. Che vi sentiate pigri o diligenti, praticate ugualmente. Non si
pratica il Dhamma seguendo gli stati mentali. Se praticate seguendo i
vostri umori allora non è Dhamma. Non fate differenza tra giorno e
notte, tra mente serena e agitata, praticate e basta. È come un bambino
che sta imparando a scrivere. All'inizio non scrive bene: segnacci
grandi e lunghi, scarabocchi. Scrive come un bambino. Dopo un po' la
scrittura migliora grazie alla pratica. La pratica del Dhamma è così.
All'inizio siete impacciati, a volte siete calmi e altre volte no, non
sapete davvero come stiano le cose. Alcuni si scoraggiano. Non battete
la fiacca! Con la pratica dovete perseverare. Vivete con impegno,
proprio come uno scolaro, che diventando più grande scriverà sempre
meglio. La brutta scrittura cresce in bella scrittura, tutto grazie alla
pratica dell'infanzia.

Così è la nostra pratica. Cercate di avere sempre rammemorazione: in
piedi, camminando, seduti o distesi. Quando svolgiamo i nostri vari
compiti per bene e con gentilezza, proviamo la pace della mente. Quando
c'è la pace della mente nel nostro lavoro, è facile avere una
meditazione serena. Si tengono per mano. Fate uno sforzo. Dovreste fare
uno sforzo per seguire la pratica. Questo è l'addestramento.

