\chapter{L'addestramento del cuore}

\begin{openingQuote}
  \centering

  Discorso offerto nel marzo del 1977 a un gruppo di monaci occidentali
  provenienti dal Wat Bovornives di Bangkok.
\end{openingQuote}

Ai tempi di Ajahn Mun\footnote{Ajahn Mun. Probabilmente fu il più
  rispettato e influente maestro di meditazione del secolo scorso in
  Thailandia. Sotto la sua guida l'ascetica Tradizione Thailandese della
  Foresta (\emph{dhutaṅga kammaṭṭhāna}) divenne veramente importante
  nella rinascita della pratica della meditazione buddhista. La
  maggioranza dei grandi maestri di meditazione della Thailandia di
  recente deceduti o ancora viventi sono stati diretti discepoli del
  venerabile Ajahn Mun oppure furono profondamente influenzati dal suo
  insegnamento. Egli morì nel novembre del 1949. Nella traduzione si è
  scelto di lasciare ``Mun'', come di solito si rinviene nei testi
  inglesi. Si avverte il lettore italiano che, però, l'esatta pronuncia
  thailandese è ``Màn''.} e di Ajahn Sao\footnote{Ajahn Sao. Anch'egli
  fu un monaco altamente rispettato della Tradizione Thailandese della
  Foresta, che si pensa fosse un \emph{arahant}; fu il maestro di Ajahn
  Mun.} la vita era molto più semplice, molto meno complicata di oggi.
Allora i monaci avevano pochi doveri e dovevano compiere poche
cerimonie. Vivevano nella foresta, senza fissa dimora. Potevano
dedicarsi completamente alla pratica della meditazione. Raramente si
vedevano le cose lussuose che oggi sono una normalità, è che
semplicemente non c'erano. Si dovevano fare tazze e sputacchiere con il
bambù, e i laici venivano di rado a fare visita. Non si voleva e non ci
si attendeva molto, e si era contenti di quel che si aveva. Si poteva
vivere e respirare la meditazione!

I monaci pativano molte privazioni vivendo in questo modo. Se qualcuno
prendeva la malaria e chiedeva delle medicine, l'insegnante diceva:
«~Non hai bisogno di medicine! Continua a praticare.~» D'altra parte, è
che non c'erano tutti i medicinali che sono ora disponibili. Tutto
quello che si aveva erano erbe e radici che crescevano nella foresta.
L'ambiente richiedeva che i monaci avessero grande pazienza e
sopportazione. Non ci si preoccupava per piccoli problemi di salute.
Oggi basta un po' di dolore che si va in ospedale!

A volte si doveva camminare da dieci a dodici chilometri per la questua.
Si partiva appena faceva giorno e si tornava forse intorno alle dieci o
alle undici. Comunque non si otteneva molto, forse un po' di riso
glutinoso, del sale o qualche peperoncino. Non importava se si aveva o
non si aveva qualcosa da mangiare con il riso. Così era. Nessuno osava
lamentarsi per la fame o per la stanchezza. I monaci non erano inclini a
lamentarsi, ma imparavano a prendersi cura di se stessi. Con numerosi
pericoli che stavano in agguato nei dintorni, praticavano nella foresta
con pazienza e sopportazione. Molti animali selvaggi e feroci vivevano
nella giungla e, per il monaco che dimorava nella foresta, le pratiche
ascetiche \emph{dhutaṅga} prevedevano molte durezze per il corpo e per
la mente. Allora la pazienza e la sopportazione dei monaci erano
eccellenti, perché erano le circostanze a obbligarli a essere così.

Oggigiorno le circostanze ci spingono invece nella direzione opposta.
Anticamente si doveva viaggiare a piedi. Poi arrivò il carro trainato
dai buoi e, infine, l'automobile. Le aspirazioni e le ambizioni
crebbero, così che se al giorno d'oggi una macchina non ha l'aria
condizionata non ci si vuole nemmeno stare seduti dentro. È impossibile
viaggiare se non c'è l'aria condizionata! Le virtù della pazienza e
della sopportazione stanno diventando sempre più deboli. I criteri per
la meditazione e per la pratica si sono rilassati e la situazione
peggiora sempre più, fino al punto che di questi tempi troviamo
meditanti ai quali piace seguire le loro opinioni e i loro desideri.
Quando la gente anziana parla dei tempi passati, è come ascoltare un
mito o una leggenda. Si ascolta con indifferenza, e non si capisce. Non
sono cose toccanti!

Per quanto ci riguarda, secondo l'antica tradizione monastica un monaco
dovrebbe trascorrere almeno cinque anni con il suo insegnante. Certi
giorni non si dovrebbe parlare con nessuno. Non consentite a voi stessi
di parlare molto. Non leggete libri! Leggete il vostro cuore, invece.
Prendiamo ad esempio il Wat Pah Pong. Molti laureati vengono per
l'ordinazione monastica. Cerco di impedire che trascorrano il loro tempo
leggendo libri sul Dhamma, perché queste persone leggono sempre. Hanno
tante opportunità di leggere libri, ma rare sono le opportunità di
leggere i loro cuori. Perciò, quando vengono per essere ordinati
temporaneamente monaci per tre mesi secondo l'usanza thailandese,
cerchiamo di far sì che chiudano i loro libri e manuali. Durante il
periodo di ordinazione monastica hanno questa splendida opportunità di
leggere il loro cuore.

Ascoltare il proprio cuore è davvero molto interessante. Questo cuore
non addestrato corre di qua e di là, seguendo senza freni le sue
abitudini. Salta avanti e indietro in modo eccitato e casuale, perché
non è mai stato addestrato. Addestrate il vostro cuore! La meditazione
buddhista riguarda il cuore, sviluppare il cuore, o mente, sviluppare la
vostra mente. È una cosa molto, molto importante. La priorità va a
questo addestramento del cuore. Il buddhismo è la religione del cuore.
Tutto qui! Chi pratica per sviluppare il cuore è uno che pratica il
buddhismo.

Questo nostro cuore vive in una gabbia, e per di più in questa gabbia
c'è una tigre furiosa. Se questo nostro cuore capriccioso non ottiene
quel che vuole, crea problemi. Dovete disciplinarlo con la meditazione,
con il \emph{samādhi}. Questo si chiama ``addestrare il cuore''.
All'inizio, il fondamento della pratica è impiantare \emph{sīla}, la
disciplina morale. \emph{Sīla} è l'addestramento del corpo e della
parola. Quando non consentite a voi stessi di fare ciò che volete c'è
attrito, da questo derivano conflitto e confusione.

Mangiate poco! Dormite poco! Quali che siano le vostre abitudini
mondane, riducetele, andate contro il loro potere. Non fate solo quel
che vi piace, non indulgete ai vostri pensieri. Bloccate questa
schiavitù. Dovete costantemente andare contro la corrente
dell'ignoranza. Questa si chiama ``disciplina''. Quando disciplinate il
vostro cuore, esso manifesta insoddisfazione e inizia a combattere. Si
sente limitato e oppresso. Quando al cuore s'impedisce di fare quel che
vuole, inizia a vagare e a combattere. La sofferenza,
\emph{dukkha},\footnote{\emph{dukkha.} ``Dis-agio'', ``difficile da
  sopportare'', insoddisfazione, sofferenza, insicurezza, instabilità,
  tensione.} ci diventa palese.

Questo \emph{dukkha}, questa sofferenza, è la prima della Quattro Nobili
Verità. La maggior parte della gente vuole fuggire da essa. Non vuole
avere alcun genere di sofferenza, non ne vuole affatto. In realtà,
questa sofferenza è ciò che ci porta alla saggezza, ci fa contemplare
\emph{dukkha}. La felicità, \emph{sukha},\footnote{\emph{sukha.}
  Piacere, benessere, soddisfazione, felicità.} tende a chiuderci occhi
e orecchi. Non ci consente di sviluppare la pazienza. L'agio e la
felicità ci rendono distratti. Di queste due contaminazioni, la più
facile da vedere è \emph{dukkha}. Per questo dobbiamo far affiorare la
sofferenza, per porre fine alla nostra sofferenza. Prima di conoscere
come praticare la meditazione, dobbiamo conoscere che cos'è
\emph{dukkha}.

All'inizio dovete addestrare il vostro cuore in questo modo. Può darsi
che non comprendiate che cosa sta succedendo o che senso possa avere, ma
quando l'insegnante vi dice di fare qualcosa, dovete farlo. Svilupperete
le virtù della pazienza e della sopportazione. Qualsiasi cosa succeda,
sopportate, perché è così. Ad esempio, quando iniziate a praticare il
\emph{samādhi} volete la pace e la tranquillità. Ma non ne avete. Non ne
avete perché non avete mai praticato in questo modo. Il vostro cuore
dice: «~Siederò fino a quando otterrò la tranquillità.~» Però, quando la
tranquillità non sorge, soffrite. E quando c'è sofferenza, vi alzate e
scappate via! Praticare in questo modo non può chiamarsi ``sviluppare il
cuore''. Si chiama ``disertare''.

Invece di indulgere ai vostri umori, addestrate voi stessi con il Dhamma
del Buddha. Che vi sentiate pigri o diligenti, continuate a praticare e
basta. Non pensate che questa è la strada migliore? L'altra strada, la
strada che consiste nel seguire i vostri umori, non raggiungerà mai il
Dhamma. Se praticate il Dhamma, allora quale che sia il vostro stato
mentale continuate a praticare, praticate costantemente. L'altra strada,
quella dell'auto-indulgenza, non è la via del Buddha. Se seguiamo le
nostre opinioni a proposito della pratica, le nostre opinioni a
proposito del Dhamma, non potremo mai vedere con chiarezza che cosa è
giusto e che cosa è sbagliato. Non conosceremo il nostro cuore. Non
conosceremo noi stessi.

Se indulgete nel seguire le vostre opinioni e cercate di praticare di
conseguenza, inizierete a pensare e a dubitare molto. Penserete: «~Non
ho molti meriti. Non sono fortunato. Ho praticato meditazione per anni e
non sono ancora illuminato. Non ho ancora visto il Dhamma.~» Praticare
con questo atteggiamento non può definirsi ``sviluppare il cuore''. Si
chiama ``sviluppare il disastro''. Se ora vi state comportando in questo
modo, se siete dei meditanti che ancora non conoscono, che non vedono,
se non avete ancora rinnovato voi stessi è perché state praticando in
modo sbagliato. Non avete seguito gli insegnamenti del Buddha. Il Buddha
insegnava in questo modo: «~Ānanda, pratica molto! Sviluppa con costanza
la tua pratica! Allora tutti i tuoi dubbi, tutte le tue incertezze
svaniranno.~» Questi dubbi non svaniranno mai attraverso il pensiero, né
attraverso teorie, né attraverso speculazioni, né attraverso
discussioni. I dubbi non scompariranno nemmeno non facendo nulla. Tutte
le contaminazioni svaniranno attraverso lo sviluppo del cuore, solo
attraverso la retta pratica.

La Via per sviluppare il cuore insegnataci dal Buddha è l'esatto opposto
della via del mondo, perché i suoi insegnamenti provengono da un cuore
puro. Un cuore puro, indipendente dalle contaminazioni, è la Via del
Buddha e dei suoi discepoli. Se praticate il Dhamma, dovete piegare il
vostro cuore al Dhamma. Non dovete far sì che il Dhamma si pieghi a voi.
Quando percorrete questa Via, sorge la sofferenza. Non c'è una sola
persona che può sfuggire a questa sofferenza. Perciò, quando iniziate la
vostra pratica la sofferenza è proprio lì.

I doveri dei meditanti consistono nello sviluppo della consapevolezza,
del raccoglimento e dell'appagamento. Sono queste le cose che ci
fermano. Fermano le abitudini del cuore di chi non s'è mai addestrato.
Perché dovremmo prenderci il fastidio di farlo? Se non vi importa di
addestrare il cuore, esso resterà selvaggio, seguirà le vie della
natura. È possibile addestrare quella natura, in modo che possa essere
usata vantaggiosamente. Un esempio, gli alberi. Se lasciassimo gli
alberi nel loro stato naturale, non saremmo mai in grado di costruire
una casa con essi. Non potremmo fabbricare delle assi o nulla che possa
essere utile per costruire una casa. Se un carpentiere intendesse
costruire una casa, cercherebbe alberi come questi. Prenderebbe questo
materiale grezzo e lo userebbe vantaggiosamente. In breve tempo
riuscirebbe a costruire una casa. La meditazione e lo sviluppo del cuore
sono simili. Dovete prendere questo cuore non addestrato come
prendereste un albero nella sua condizione naturale in una foresta, e
addestrare questo cuore naturale così da renderlo più affinato, più
consapevole di se stesso e più sensibile. Tutto è nella sua condizione
naturale. Quando comprendiamo la natura, possiamo cambiarla, possiamo
distaccarcene, possiamo lasciarla andare. Allora non soffriremo più.

La natura del nostro cuore è tale che tutte le volte che esso si attacca
e si aggrappa, c'è agitazione e confusione. Prima vaga di qua, poi vaga
di là. Quando si riesce a osservare questa agitazione, si può pensare
che sia impossibile addestrare il cuore e, di conseguenza, soffriamo.
Non comprendiamo che questo è il modo il cui il cuore è. Ci può essere
un movimento di pensieri e di sensazioni anche se stiamo praticando, se
stiamo cercando di raggiungere la pace. È così. Quando avremo
contemplato molte volte la natura del cuore, giungeremo a capire che
questo cuore è solo così com'è, e che non può essere altrimenti.
Conosceremo che le vie del cuore sono solo quel che sono. È la sua
natura. Se lo vediamo con chiarezza, possiamo allora distaccarci da
pensieri e sensazioni. E, dicendo continuamente a noi stessi «~è solo
così com'è~», non abbiamo null'altro da aggiungere. Quando il cuore
comprende davvero, lascia andare tutto. Pensieri e sensazioni saranno
ancora lì, ma quegli stessi pensieri e quelle stesse sensazioni saranno
privi di potere.

È come un bimbo al quale piace giocare e saltellare in una maniera che
ci infastidisce, tanto che vorremmo rimproverarlo o sculacciarlo.
Dovremmo capire che è naturale per un bimbo comportarsi così. Allora
lasciamo andare e gli consentiamo di giocare come vuole. Così i nostri
problemi sono finiti. Com'è che sono finiti? Perché accettiamo il modo
di essere dei bambini. La nostra prospettiva cambia e accettiamo la vera
natura delle cose. Lasciamo andare, e il nostro cuore diventa più
sereno. Abbiamo ``Retta Comprensione''. Con l'errata comprensione, anche
se vivessimo in una profonda e buia caverna oppure su in alto, per aria,
sarebbe un caos. Il cuore può essere in pace solo quando c'è ``Retta
Comprensione''. Allora non ci sono più enigmi da risolvere, né sorgono
problemi. È così. Vi staccate. Lasciate andare. Tutte le volte che c'è
una qualsiasi sensazione di attaccamento, ce ne distacchiamo, perché
sappiamo che proprio quella sensazione è solo così com'è. Non è arrivata
da noi proprio per darci fastidio. Potremmo pensare che invece è
arrivata per questa ragione, ma in verità è solamente così com'è. Se
cominciamo a pensarci e a prenderla in maggiore considerazione, anche
questo è solamente così com'è. Se lasciamo andare, allora la forma è
solo forma, il suono è solo suono, l'odore è solo odore, il sapore è
solo sapore, il tatto è solo tatto e la mente è solo mente.

È come l'acqua e l'olio. Se li mettete insieme dentro una bottiglia, a
causa della loro differente natura non si mischiano. L'olio e l'acqua
sono diversi allo stesso modo in cui sono diversi un saggio e un
ignorante. Il Buddha viveva con forma, suono, odore, sapore, tatto e
pensiero. Era un \emph{arahant}\footnote{\emph{arahant.} Letteralmente,
  un ``Meritevole''; una persona la cui mente è libera dalle
  contaminazioni (\emph{kilesa}). È anche un titolo del Buddha e il
  livello più alto dei suoi Nobili Discepoli.} e s'era perciò distolto
da queste cose, non era più rivolto verso di esse. Si distolse e si
distaccò poco a poco, allorché comprese che il cuore è solo il cuore e
che il pensiero è solo il pensiero. Non li confondeva e non li
mescolava. Il cuore è solo il cuore. Pensieri e sensazioni sono solo
pensieri e sensazioni. Lasciate che le cose siano così come sono!
Lasciate che la forma sia solo forma, lasciate che il suono sia solo
suono, lasciate che il pensiero sia solo pensiero. Perché dovremmo
preoccuparci di attaccarci a essi? Se pensiamo e sentiamo in questo
modo, allora c'è distacco, separazione. I nostri pensieri e le nostre
sensazioni staranno da una parte e il nostro cuore starà dall'altra.
Proprio come l'olio e l'acqua: sono nella stessa bottiglia ma sono
separati.

Il Buddha e i suoi discepoli illuminati vivevano con persone ordinarie,
non illuminate. Non solo vivevano con queste persone, ma insegnavano a
questi esseri ordinari, non illuminati e ignoranti come diventare
nobili, illuminati e saggi. Potevano farlo perché sapevano come
praticare. Sapevano, proprio come vi ho spiegato, che è una questione di
cuore.

Perciò, quale che sia il livello della vostra pratica di meditazione,
non impegnatevi a metterla in dubbio. Se andiamo via di casa per
ricevere l'ordinazione monastica, non andiamo via per perderci nelle
illusioni e neanche per vigliaccheria o per paura. Andiamo via per
addestrare noi stessi, per avere padronanza di noi stessi. Se abbiamo
questo genere di comprensione, allora possiamo seguire il Dhamma. Il
Dhamma diventerà sempre più chiaro. Chi comprende il Dhamma comprende se
stesso, e chi comprende se stesso comprende il Dhamma. Oggigiorno solo
sterili resti di Dhamma sono entrati a far parte delle istituzioni. In
realtà, il Dhamma è ovunque. Non c'è bisogno di scappare in nessun altro
posto. Fuggite invece mediante la saggezza. Fuggite mediante
l'intelligenza. Fuggite mediante l'abilità, non mediante l'ignoranza. Se
volete la pace, allora lasciate che sia la pace della saggezza. È
abbastanza!

Ogni volta che vediamo il Dhamma c'è la Retta Via, il Retto Sentiero. Le
contaminazioni sono solo contaminazioni, il cuore è solo il cuore. Ogni
volta che ci distacchiamo e ci separiamo, così che restano solo queste
cose come realmente sono, allora esse sono per noi solo oggetti. Quando
siamo sul Retto Sentiero, siamo impeccabili. Quando siamo impeccabili,
c'è sempre apertura e libertà. Il Buddha disse: «~Monaci, ascoltate. Non
dovete attaccarvi a nessun \emph{dhamma}.~» Cosa sono questi
\emph{dhamma}?\footnote{\emph{dhamma.} È un termine
  difficilmente traducibile e con un notevole numero di significati.
  Indica sia la dottrina del Buddha, la realtà delle cose, l'ordine che
  governa l'universo, la legge morale; sia, in senso tecnico e con la
  lettera minuscola, il fenomeno tanto fisico quanto mentale, oppure
  solo lo stato mentale, l'oggetto mentale, la caratteristica o la
  qualità.} Sono ogni cosa. Non c'è nulla che non sia \emph{dhamma}.
Amore e odio sono \emph{dhamma}, felicità e sofferenza sono
\emph{dhamma}, piacere e dispiacere sono \emph{dhamma}. Tutte queste
cose, non importa quanto poco significative possano essere, sono
\emph{dhamma}. Quando pratichiamo il Dhamma, quando comprendiamo, allora
possiamo lasciar andare. È così possibile ottemperare all'insegnamento
del Buddha che afferma di non attaccarsi a nessun \emph{dhamma}.

Tutti i fenomeni condizionati che nascono nel nostro cuore, tutti i
fenomeni condizionati della nostra mente, tutti i fenomeni condizionati
del nostro corpo sono sempre in cambiamento. Il Buddha insegnò a non
attaccarsi a nessuno di essi. Insegnò ai suoi discepoli a praticare per
staccarsi da tutti i fenomeni condizionati, e non a praticare per
ottenere qualcosa. Se seguiamo gli insegnamenti del Buddha, siamo nel
giusto. Siamo nel giusto ma è anche un problema. Non è che gli
insegnamenti siano problematici, sono le nostre contaminazioni a
esserlo. Le contaminazioni mal comprese ci ostruiscono e ci causano
problemi. Non c'è nulla di realmente problematico nel seguire
l'insegnamento del Buddha. Possiamo infatti dire che attaccarsi al
Sentiero del Buddha non reca sofferenza, perché il Sentiero consiste
semplicemente nel ``lasciar andare'' ogni \emph{dhamma}! Il Buddha
insegnò che la pratica del ``lasciar andare'' è lo scopo principale
della pratica buddhista della meditazione. Non trascinatevi dietro
nulla! Staccatevi! Se vedete la bontà, lasciatela andare. Se vedete la
rettitudine, lasciatela andare. Queste parole, ``lasciar andare'', non
significano che non dobbiamo praticare. Significano che dobbiamo
praticare seguendo il metodo del ``lasciar andare''.

Il Buddha ci insegnò a contemplare tutti i \emph{dhamma}, a sviluppare
il Sentiero attraverso la contemplazione del nostro corpo e del nostro
cuore. Il Dhamma non è da nessun'altra parte. È proprio qui! Non in
qualche luogo lontano. È qui, proprio in questo nostro corpo e in questo
nostro cuore. Un meditante deve perciò praticare con energia. Rendete il
cuore più grande e più luminoso. Rendetelo libero e indipendente. Dopo
aver fatto una buona azione, non portatevela dietro nel vostro cuore,
lasciatela andare. Dopo esservi astenuti dal compiere una cattiva
azione, lasciate andare. Il Buddha ci insegnò a vivere nell'immediatezza
del presente, nel qui e ora. Non perdetevi nel passato o nel futuro.

L'insegnamento che la gente meno comprende e che più è in conflitto con
le loro opinioni è quello del ``lasciar andare'', o del ``lavorare con
una mente vuota''. Questo modo di parlare si chiama ``linguaggio del
Dhamma''. Quando lo concepiamo in termini mondani, diventiamo confusi e
pensiamo di poter fare tutto quel che vogliamo. Può essere interpretato
in questo modo, ma il suo vero significato è più vicino a questo
esempio: è come se stessimo trasportando una pietra pesante. Dopo un po'
iniziamo a sentirne il peso, ma non sappiamo come lasciar andare. Così
sopportiamo per tutto il tempo questo grande peso. Se qualcuno ci dice
di gettarla via, rispondiamo: «~Se la getto via, non mi resterà nulla!~»
Quando ci elencano tutti i benefici provenienti dal gettarla via, non ci
crediamo e continuiamo a pensare: «~Se la getto via, non avrò nulla!~»
Proseguiamo e trasportiamo questa pesante pietra, finché diventiamo
talmente deboli ed esausti da non riuscire più a sopportarne il peso. È
allora che la gettiamo.

Dopo averla gettata, sperimentiamo improvvisamente i benefici del
lasciar andare. Ci sentiamo subito meglio e più leggeri, e sappiamo da
noi stessi quanto sia pesante portare una pietra. Non sarebbe stato
possibile conoscere i benefici del lasciar andare prima di averlo fatto.
Così, se qualcuno dicesse di lasciar andare, un essere non illuminato
non ne capirebbe la ragione. Continuerebbe solo ciecamente a tenersi
stretta la pietra e rifiuterebbe di lasciar andare fino a quando essa
non diventa tanto pesante che lasciarla andare è l'unica possibilità.
Allora potrebbe sentire da sé la leggerezza e il sollievo, e perciò
conoscerebbe da sé i benefici del lasciar andare. In seguito potremo
cominciare di nuovo a trasportare pesi ma, conoscendone il risultato, li
lasciaremo andare con maggior facilità. Comprendere che è inutile
trasportare pesi e che il lasciar andare reca benessere e leggerezza è
un esempio del conoscere se stessi.

Il nostro orgoglio, il senso del sé dal quale siamo dipendenti, è
identico a quella pietra pesante. Come quella pietra, se pensiamo a
lasciar andare la presunzione, temiamo che senza di essa non resterebbe
nulla. Però, quando finalmente la lasciamo andare, comprendiamo da noi
stessi l'agio e il benessere del non attaccamento. Nell'addestrare il
cuore, non dovete attaccarvi né alla lode né al biasimo. Voler essere
solo lodati e non voler essere biasimati è la via del mondo. La Via del
Buddha è accettare la lode quando è appropriata e accettare il biasimo
quando è appropriato. Ad esempio, quando si alleva un bambino è bene non
rimproverarlo in continuazione. Alcuni sgridano troppo i bambini. Una
persona saggia sa quando è il momento giusto di sgridare e quando è il
momento giusto di lodare. Per il nostro cuore è la stessa cosa. Siate
intelligenti nel conoscere il cuore. Siate abili nel prendervene cura.
Allora sarete capaci di addestrarlo. E quando il cuore è abile, possiamo
liberarci dalla nostra sofferenza. La sofferenza esiste proprio qui, nel
nostro cuore. Complica e crea cose in continuazione, e rende pesante il
cuore. Nasce qui, ed è qui che muore.

La via del cuore è così. A volte ci sono buoni pensieri, altre volte
cattivi pensieri. Il cuore è ingannevole. Non fidatevi! Guardate invece
direttamente le condizioni in cui il cuore si trova. Accettatele per
quello che sono. Sono solo così come sono. Che siano buone, cattive, o
quali che siano le sue condizioni, sono così come sono. Se non vi
aggrappate a queste condizioni, non diventeranno nulla di più o nulla di
meno di quel che già sono. Se ci aggrappiamo, saremo morsicati e
soffriremo. Con la ``Retta Visione''\footnote{Retta Visione
  (\emph{sammā-diṭṭhi}). La Retta Visione è il primo fattore del Nobile
  Ottuplice Sentiero.} c'è solo la pace. Il \emph{samādhi} è nato e la
saggezza prende il sopravvento. Ovunque possiate sedere o giacere, c'è
la pace. C'è pace ovunque, non conta dove andate.

Così, oggi avete portato qui i vostri discepoli ad ascoltare il Dhamma.
Potreste comprenderne un po' e un altro po' forse no. Affinché possiate
capire con maggior facilità, ho parlato della pratica della meditazione.
Sia che pensiate che quel che ho detto è giusto sia che pensiate il
contrario, dovreste prenderlo e contemplarlo. Io stesso, come
insegnante, mi sono trovato in una difficile situazione di questo
genere. Anch'io ho desiderato ascoltare discorsi di Dhamma perché,
ovunque sia andato, ho offerto insegnamenti agli altri ma non ho mai
avuto l'opportunità di ascoltare. Così, questa volta siete voi ad
apprezzare l'ascolto di un discorso da un insegnante. Il tempo passa
così veloce quando si è seduti e si ascolta tranquillamente. Siete
affamati di Dhamma, per questa ragione volete ascoltare davvero.
All'inizio offrire insegnamenti agli altri è un piacere, ma dopo un po'
il piacere se ne va. Ci si sente annoiati e stanchi. Allora arriva il
desiderio di ascoltare. Quando si ascolta un discorso da un insegnante,
si è molto ispirati e si comprende con facilità. Quando si diventa
anziani e c'è fame di Dhamma, il sapore è particolarmente delizioso. Se
insegnate agli altri, siete per loro un esempio, siete un modello per
gli altri \emph{bhikkhu}. Siete un esempio per i vostri discepoli. Siete
un esempio per tutti, perciò non dimenticatevi di voi stessi. Però, non
dovete neanche pensare a voi stessi. Se sorge un pensiero di questo
genere, sbarazzatevene. Se fate così, allora siete una persona che
conosce se stessa.

Ci sono mille modi di praticare il Dhamma. Non c'è fine alle cose che si
possono dire sulla meditazione. Sono talmente tante da poterci far
dubitare. I dubbi continuate a spazzarli via, fino a quando non ce ne
sono più! Quando abbiamo questa Retta Comprensione, non importa dove
sediamo o camminiamo, c'è solo pace e benessere. Quale che sia il posto
in cui si fa meditazione, proprio quello è il posto in cui portate la
vostra consapevolezza. Non pensiate che si medita solo quando si è
seduti o si cammina. In tutto e ovunque si trova la nostra pratica. C'è
sempre consapevolezza. C'è sempre presenza mentale. Possiamo sempre
vedere nella mente e nel corpo la nascita e la morte, ma non consentiamo
che ingombrino il nostro cuore. Lasciate andare continuamente. Se arriva
l'amore, lasciate che torni alla sua casa. Se arriva l'avidità, lasciate
che torni a casa. Se arriva la rabbia, lasciate che torni a casa.
Seguitele! Dove vivono? Accompagnatele lì. Non trattenete nulla. Se
praticate in questo modo siete come una casa vuota. Oppure, per
spiegarla in un altro modo, questo è un cuore vuoto, un cuore vuoto e
libero da ogni malvagità. Possiamo chiamarlo ``cuore vuoto'', ma non è
vuoto come se lì non ci fosse nulla, è vuoto di malvagità ma pieno di
saggezza. Allora qualsiasi cosa farete, la farete con saggezza.
Penserete con saggezza. Mangerete con saggezza. Ci sarà solo saggezza.

Questo è l'insegnamento di oggi, ve lo offro. L'ho registrato su un
nastro. Se ascoltare il Dhamma reca pace ai vostri cuori, va abbastanza
bene. Non avete bisogno di ricordare nulla. Qualcuno potrà non crederci.
Se rendete i vostri cuori sereni e ascoltate solamente, lasciando che le
parole scorrano ma contemplando in continuazione, allora siamo come un
registratore. Quando dopo un po' di tempo lo accendiamo, è tutto lì. Non
abbiate timore che non ci sia nulla. Non appena accenderete il vostro
registratore, sarà tutto lì. Questo insegnamento desidero offrirlo a
ogni \emph{bhikkhu}, e a tutti. Alcuni di voi forse conoscono il
thailandese solo un po', ma non importa. Che possiate imparare la lingua
del Dhamma. È abbastanza!

