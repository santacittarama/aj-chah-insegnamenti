\chapter{Con tutto il cuore}

In ogni casa, in tutte le comunità, che si viva in città o in campagna,
oppure nella foresta o sulle montagne, tutti siamo uguali quando
sperimentiamo la felicità e la sofferenza. Molti di noi non hanno un
luogo in cui rifugiarsi, un prato o un giardino nel quale possano essere
coltivate le qualità positive del cuore. Sperimentiamo questa povertà
spirituale perché non c'è reale impegno. Non abbiamo chiara comprensione
della nostra vita e di come dovremmo impiegarla. Dall'infanzia alla
giovinezza, fino all'età adulta, impariamo solo a cercare il piacere e a
trarre soddisfazione dagli oggetti dei sensi. Man mano che procediamo
nella nostra vita, non pensiamo mai al pericolo che incombe su di noi,
ad esempio quando ci facciamo una famiglia, e così via.

Se non abbiamo una terra da coltivare e una casa nella quale vivere,
siamo privi di un rifugio esteriore, e la nostra vita è colma di
difficoltà e di angoscia. Per di più, nella nostra vita manchiamo
interiormente di \emph{sīla} e del Dhamma, e non andiamo a sentire gli
insegnamenti e a praticare il Dhamma. Il risultato è che c'è poca
saggezza e tutto regredisce e degenera. Il Buddha, il nostro Maestro
supremo, aveva \emph{mettā}\footnote{\emph{mettā.} Gentilezza amorevole,
  benevolenza, cordialità, amichevolezza.} per gli esseri viventi.
Condusse figli e figlie di buona famiglia all'ordinazione monastica.
Praticò e realizzò la Verità, per radicare e diffondere il Dhamma al
fine di mostrare alla gente come vivere felicemente la quotidianità.
Insegnò il giusto modo per guadagnarsi da vivere, a essere moderati e
parchi col denaro, a non agire sconsideratamente, in qualsiasi ambito.

Se invece siamo carenti di supporto in entrambi i sensi, esteriormente
in senso materiale e interiormente in senso spirituale, con il passare
del tempo e con l'aumento della popolazione le illusioni, la povertà e
le difficoltà ci inducono ad allontanarci sempre più dal Dhamma. Non
siamo interessati a cercare il Dhamma a causa della nostra difficile
situazione. Anche se c'è un monastero nelle vicinanze, non ce la
sentiamo di andare ad ascoltare gli insegnamenti perché siamo
ossessionati dalla nostra povertà e dai nostri problemi, dalle
difficoltà che incontriamo per la mera sopravvivenza. Il Buddha però ci
insegnò che, per quanto si possa essere poveri, non dovremmo permettere
al nostro cuore di impoverirsi e alla nostra saggezza di morire di fame.
Anche quando un'inarrestabile alluvione inonda i nostri campi, i nostri
villaggi e le nostre case, il Buddha ci insegnò a non consentire che
essa inondi e vinca il nostro cuore. Quando il cuore viene inondato,
perdiamo di vista il Dhamma, non lo conosciamo più. C'è \emph{ogha},
l'inondazione, l'inondazione della sensorialità, del divenire, delle
opinioni e dell'ignoranza. Queste quattro cose oscurano e avvolgono il
cuore degli esseri, sono peggio dell'acqua che inonda i nostri campi, i
nostri villaggi e le nostre città. Anche se nel corso degli anni l'acqua
inonda in continuazione i nostri campi, e pure se il fuoco brucia del
tutto le nostre case, abbiamo ancora la nostra mente. Se \emph{sīla} e
il Dhamma sono nella nostra mente, possiamo far uso della nostra
saggezza e trovare il modo per guadagnarci da vivere. Possiamo di nuovo
acquistare della terra e ricominciare da capo.

Ora, quando abbiamo mezzi di sussistenza, abitazioni e possedimenti, la
nostra mente può essere retta e sentirsi a proprio agio, possiamo avere
la forza di spirito per aiutarci e assisterci l'un l'altro. Se qualcuno
è in grado di fornire cibo e vestiti, e di offrire ricovero a chi è in
stato di necessità, questi sono atti di gentilezza amorevole. Per come
la vedo io, donare cose con spirito di gentilezza amorevole è molto
meglio che venderle per ricavarne profitto. Chi ha \emph{mettā}, per sé
non desidera nulla. Desidera solo che gli altri vivano felicemente.

Se davvero addestrassimo la nostra mente e ci impegnassimo nel giusto
modo, penso che non ci sarebbero gravi difficoltà. Non sperimenteremmo
estrema povertà, non saremmo come dei lombrichi. Abbiamo uno scheletro,
degli occhi, degli orecchi, delle braccia e delle gambe. Possiamo ancora
\mbox{mangiare} cose come la frutta, non abbiamo bisogno di mangiare sporcizia,
come fanno i lombrichi. Se vi lamentate della povertà, se nel profondo
sentite di essere degli sventurati, il lombrico vi dirà: «~Non essere
così dispiaciuto. Non hai forse ancora braccia e gambe, e delle ossa?
Queste cose io non le ho, e tuttavia non mi sento povero.~» Il lombrico
ci farà vergognare.

Una volta un allevatore di maiali venne a trovarmi. Si lamentava:
«~Quest'anno è davvero troppo! Il prezzo del mangime è salito. Quello
dei maiali si è abbassato. Mi dovrò vendere la camicia!~» Ascoltai le
sue lamentele e gli dissi: «~Non rattristarti troppo. Se tu fossi un
maiale, allora sì che avresti una buona ragione per rattristarti. Quando
il prezzo dei maiali sale, i maiali vengono macellati, e quando il loro
prezzo scende, vengono macellati ugualmente. Sono i maiali che hanno
davvero qualcosa di cui lamentarsi. La gente non dovrebbe lamentarsi.
Pensaci su seriamente, per favore.~» Si preoccupava solo dei soldi che
avrebbe guadagnato. I maiali avrebbero molto di cui preoccuparsi, ma non
ci pensiamo. Non è che siamo noi che stiamo per essere uccisi, e perciò
possiamo sempre cercare una via d'uscita.

Penso proprio che se si ascolta il Dhamma, se lo si contempla e
comprende, si può davvero porre fine alla sofferenza. Sai che cos'è
giusto fare, che cosa c'è bisogno di fare, come usare e spendere il
denaro. Si può vivere la propria vita in armonia con \emph{sīla} e con
il Dhamma, applicando la saggezza nei riguardi delle cose del mondo.
Però, la maggioranza di noi è lontana da tutto questo. Nella nostra vita
non abbiamo né moralità né Dhamma, e così la nostra vita è piena di
discordie e attriti. C'è discordia tra marito e moglie, c'è discordia
tra figli e genitori. I figli non ascoltano i loro genitori proprio a
causa della mancanza di Dhamma nelle famiglie. La gente non è
interessata ad ascoltare il Dhamma e a imparare qualcosa e, così, invece
di sviluppare buon senso e abilità, sprofonda nell'ignoranza e il
risultato è innumerevoli vite di sofferenza.

Il Buddha insegnò il Dhamma e indicò la via della pratica. Non cercava
di renderci la vita difficile. Voleva che facessimo progressi, che
diventassimo migliori e più abili. È che non ascoltiamo. Va proprio
male. È come un bimbo che non vuole fare il bagno durante l'inverno
perché fa troppo freddo. Inizia a puzzare talmente che i genitori di
notte non riescono neanche a dormire, e perciò lo prendono e gli fanno
fare il bagno. Questo fa infuriare il bambino, che piange e inveisce
contro i genitori. I genitori e il bambino vedono la situazione in modo
differente. Per il bambino fa troppo freddo per fare il bagno d'inverno.
Per i genitori l'odore del bambino è insopportabile. I due punti di
vista sono inconciliabili. Il Buddha non voleva che rimanessimo come
siamo. Voleva che diventassimo diligenti e che lavorassimo sodo, in modo
benefico, che fossimo entusiasti del Retto Sentiero. Invece di essere
pigri, dobbiamo sforzarci. Il suo insegnamento non ci trasformerà in
folli o in persone inutili. Ci insegna come sviluppare e applicare la
saggezza a tutto quel che facciamo: quando lavoriamo, quando coltiviamo
i campi, quando ci occupiamo della famiglia, quando amministriamo i
nostri beni, sempre consapevoli di ogni aspetto di queste cose. Se
viviamo nel mondo, dobbiamo essere attenti e conoscere le vie del mondo.
Altrimenti si finisce nei guai.

Viviamo in un posto nel quale il Buddha e il suo Dhamma ci sono
familiari. Però, pensiamo che tutto quel che dobbiamo fare è andare a
sentire gli insegnamenti e prenderla alla leggera, vivendo come al
solito. Si tratta di un grave fraintendimento. Come avrebbe mai potuto
il Buddha raggiungere la conoscenza in questo modo? Buddha non sarebbe
mai esistito.

Egli ci insegnò i vari generi di ricchezza: la ricchezza della vita
umana, la ricchezza del regno celeste, la ricchezza del Nibbāna.
Chi ha il Dhamma, anche se vive nel mondo, non è povero. E quando lo è,
non ne soffre. Se viviamo in accordo con il Dhamma, non proviamo dolore
quando ci voltiamo indietro a guardare quel che abbiamo fatto. Generiamo
solo buon kamma. Se stiamo generando un kamma cattivo, in
seguito ne risulterà infelicità. Se non abbiamo generato un cattivo
kamma, non patiremo effetti di questo genere in futuro. Se però
non cerchiamo di modificare le nostre abitudini e di smetterla con le
cattive azioni, le nostre difficoltà continueranno, sia la sofferenza
mentale sia i problemi materiali. È per questo che abbiamo bisogno di
ascoltare e di contemplare, per capire da dove provengono le nostre
difficoltà. Avete mai trasportato nei campi delle cose su un'asta
poggiata sulle spalle? È vero che quando davanti il carico è troppo
pesante il trasporto è difficile? E quando è troppo pesante dietro, non
è ugualmente scomodo? Quando sta in equilibrio? E quand'è che non sta in
equilibrio? Lo potete vedere mentre state trasportando le cose. Con il
Dhamma è uguale. C'è causa ed effetto, si tratta di comune buon senso.
Quando il carico è equilibrato, il trasporto è più facile. Possiamo
gestire la nostra vita in modo equilibrato, con un atteggiamento di
moderazione. Le nostre relazioni in famiglia e il nostro lavoro possono
procedere in armonia e, anche se non siete ricchi, il vostro benessere è
nella mente. Non c'è bisogno di soffrire.

Se i componenti di una famiglia non lavorano sodo, possono trovarsi in
difficoltà. Quando vedono che gli altri hanno di più, iniziano a provare
cupidigia, gelosia e risentimento, e tutto questo può indurre a rubare.
Così, il villaggio in cui vivono diventa un posto infelice. È meglio
lavorare a beneficio di voi stessi e delle vostre famiglie, sia per
questa vita sia per quelle future. Se le vostre necessità materiali
vengono soddisfatte mediante i vostri sforzi, allora la vostra mente si
sente a proprio agio ed è felice, e ciò induce ad ascoltare gli
insegnamenti di Dhamma, a imparare quello che è giusto e quello che è
sbagliato -- virtù e demeriti -- e a continuare a cambiare in meglio la
vostra vita. Potete imparare a riconoscere come le cattive azioni
servano solo a creare difficoltà, e così rinuncerete a questo genere di
azioni e continuerete a migliorarvi. Il vostro modo di lavorare
cambierà, e anche la vostra mente. Prima eravate ignoranti, poi avrete
la conoscenza. Prima avevate delle cattive abitudini, poi sarete di buon
cuore. Potrete insegnare ai vostri figli e ai vostri nipoti quel che
sapete. Ciò significa generare benessere nel futuro facendo quello che è
giusto nel presente. Però, chi è privo di saggezza non fa nulla di
benefico nel presente, e finisce solo per procurarsi dei guai. Se
diventa povero, pensa solo a giocare d'azzardo. E questo lo condurrà a
trasformarsi in un ladro.

Non siamo ancora morti e perciò dobbiamo parlare di queste cose. Se non
ascoltate il Dhamma ora che siete degli esseri umani, non avrete nessuna
altra possibilità. Pensate che agli animali si possa insegnare il
Dhamma? La vita degli animali è molto più dura della nostra. Nascere
come rospo o rana, come maiale o cane, come cobra o vipera, come
scoiattolo o coniglio. Quando la gente li vede, pensa solo a ucciderli o
a bastonarli, a catturarli o ad allevarli per mangiarseli. Abbiamo
questa opportunità in quanto esseri umani. È molto meglio! Siamo ancora
in vita, e perciò è questo il tempo di guardare dentro queste cose e di
ravvedersi. Se le cose si fanno difficili, per il momento cercate di
sopportare i problemi e vivete in modo retto fino a che non riuscirete a
risolverli. Questo è praticare il Dhamma. Mi piacerebbe ricordare a
tutti voi quanto sia importante avere una mente buona e vivere con
moralità la vita. Finora, ovviamente, di cose ne avete forse fatte, e
allora dovreste guardarle da vicino e vedere se vanno bene o no. Se
avete seguito delle strade sbagliate, rinunciate a esse. Rinunciate ai
mezzi di sussistenza non retti. Guadagnarsi da vivere in modo buono e
decente non arreca danno agli altri e non arreca danno a voi stessi e
alla società. Quando i vostri mezzi di sussistenza sono retti, potete
vivere con una mente che è a proprio agio.

Noi, monaci e monache, dipendiamo dai laici per tutti i nostri bisogni
materiali e facciamo affidamento sulla contemplazione, così da essere in
grado di spiegare il Dhamma ai laici affinché comprendano e tutto questo
vada a loro beneficio, rendendoli in grado di migliorare la loro vita.
Potete imparare a riconoscere e a rimuovere tutto quello che per voi è
causa d'infelicità e di conflitto. Sforzatevi di andare d'accordo gli
uni con gli altri, fate in modo che nelle vostre relazioni ci sia
armonia, invece di sfruttarvi e danneggiarvi a vicenda. Oggigiorno le
cose vanno proprio male. Per la gente è difficile andare d'accordo. Non
funziona neanche quando poche persone si incontrano per una piccola
riunione. Basta che si guardino in faccia tre volte che già sono pronti
a uccidersi. Perché? È solo perché nella vita della gente non c'è né
\emph{sīla} né Dhamma. Ai tempi dei nostri genitori era molto diverso.
Anche il solo modo in cui la gente si guardava rivelava amore e
amicizia. Non era come adesso. Se uno straniero arriva in un villaggio
di sera, tutti diventano sospettosi: «~Che cos'è venuto a fare qui di
notte?~» Perché dovremmo avere paura quando una persona arriva nel
villaggio? Se è un cane estraneo ad arrivare, nessuno pensa che abbia un
secondo fine. Una persona è forse peggio di un cane? «~È un estraneo, è
uno straniero!~» Com'è possibile che uno sia uno straniero? Quando
qualcuno arriva in un villaggio, gli dovremmo essere grati. Ha bisogno
di ricovero, e perciò può restare con noi, e noi ci possiamo prendere
cura di lui e aiutarlo. Staremo in compagnia.

Al giorno d'oggi non c'è più né la tradizione dell'ospitalità né la
buona volontà. C'è solo paura e sospetto. Direi che in alcuni villaggi
non ci sono più persone, ma solo animali. Si sospetta di tutto, si è
possessivi per ogni cespuglietto e per qualche centimetro di terra solo
perché non c'è moralità, non c'è spiritualità. Quando non c'è né
\emph{sīla} né Dhamma, la nostra vita è colma di disagio e di paranoia.
Di notte la gente va a dormire e subito si sveglia, preoccupata di quel
che può succedere, oppure perché ha sentito un rumore. Nei villaggi non
si va d'accordo, non c'è fiducia reciproca. Tra genitori e figli non c'è
fiducia reciproca. Tra marito e moglie non c'è fiducia reciproca. Che
cosa sta succedendo? Tutto questo avviene quando si è lontani dal Dhamma
e si vive senza Dhamma. Da qualsiasi parte si guardi è così, e la vita è
dura. Ora se anche solo poche persone si fanno vedere nel villaggio e
chiedono ospitalità per la notte, si dice loro di andarsi a cercare un
albergo. Ora si fa tutto per guadagno. In passato nessuno avrebbe mai
pensato di mandar via la gente in questo modo. Tutto il villaggio si
sarebbe unito per mostrare ospitalità. La gente sarebbe andata a
invitare i vicini, e ognuno avrebbe portato cibo e bevande da dividere
con gli ospiti. Questo adesso non si può fare. Dopo aver cenato, la
gente chiude la porta a chiave. Nel mondo, ovunque ora si guardi, è così
che vanno le cose. Significa che il non-spirituale va proliferando e
prendendo il sopravvento. In genere non siamo molto felici e non abbiamo
molta fiducia negli altri. Adesso succede pure che qualcuno uccida i
propri genitori. Marito e moglie si tagliano la gola a vicenda. Nella
società c'è tutto questo dolore semplicemente perché c'è mancanza di
\emph{sīla} e di Dhamma. Per favore, cercate di comprenderlo e non
gettate via i principi della virtù. Con la virtù e con la spiritualità,
la vita degli uomini può essere felice. Senza di esse diventiamo come
gli animali.

Il Buddha nacque in una foresta. Nacque nella foresta e studiò il Dhamma
nella foresta. Insegnò il Dhamma nella foresta, a cominciare dal
Discorso della Messa in Moto della Ruota del Dhamma. Entrò nel
Nibbāna nella foresta. Per quanti di noi vivono nella foresta, è
importante capire la foresta. Viverci non significa che la nostra mente
diventa selvaggia, come quella degli animali. La nostra mente può
elevarsi e diventare nobile spiritualmente. Questo disse il Buddha.
Vivendo in città viviamo tra distrazioni e disturbi. Nella foresta c'è
quiete e tranquillità. Possiamo contemplare le cose con chiarezza e
sviluppare la saggezza, e così questa quiete ci aiuta e diventa nostra
amica. Siccome questo ambiente è favorevole alla pratica del Dhamma, vi
dimoriamo. Le montagne e le grotte diventano il nostro rifugio. In
questi luoghi la saggezza sopraggiunge osservando i fenomeni naturali.
Impariamo dagli alberi e li comprendiamo, e altrettanto avviene con
qualsiasi altra cosa, e tutto questo reca gioia. I suoni della natura
che sentiamo non ci disturbano. Reca molta gioia anche ascoltare gli
uccelli che si chiamano l'un l'altro a loro piacimento. Non reagiamo con
alcuna avversione né pensiamo di nuocere a qualcuno. Non parliamo con
durezza né siamo aggressivi. Per la mente è piacevole ascoltare i rumori
della foresta e quando li sentiamo la mente è serena.

I suoni e i rumori della gente portano invece agitazione. Anche quando
le persone parlano in modo gentile, ciò non rasserena la mente. I suoni
che piacciono alla gente, come la musica, non sono sereni. Inducono
eccitazione e piacere, ma in essi non c'è pace. Quando le persone stanno
insieme e cercano il piacere, ciò conduce all'irragionevolezza, a
parole aggressive e a contenziosi, e questa situazione di turbamento
continua a crescere. Così sono i suoni della gente. Non recano reale
agio e felicità, a meno che non siano pronunciate parole di Dhamma. In
genere, quando le persone vivono insieme nella società, parlano spinti
dai loro interessi, ognuno procura agitazione all'altro, si offendono e
si accusano a vicenda, e l'unico risultato è la confusione e il
turbamento. Quando è priva del Dhamma, la gente tende a essere così. I
suoni ci conducono verso l'illusione. La musica e le parole delle
canzoni agitano e confondono la mente. Date un'occhiata a questa cosa.
Prendete in considerazione le sensazioni piacevoli che provengono
dall'ascolto della musica. Le persone pensano che sia qualcosa di
grande, che sia molto divertente. Possono stare in piedi sotto il sole
cocente per ascoltare la musica e assistere a spettacoli di danza.
Restano lì in piedi fino a quando sono cotti a puntino, ma continuano a
pensare che si stanno divertendo. Se però qualcuno si rivolge a loro
duramente, li critica o li insulta, sono di nuovo infelici. È così che
sono i normali suoni degli esseri umani.

Quando però i suoni degli esseri umani diventano suoni di Dhamma, se la
mente è Dhamma e noi parliamo il linguaggio del Dhamma, allora vale la
pena di ascoltare, si tratta di qualcosa su cui riflettere, da studiare
e contemplare. Questo tipo di suono va bene, non è eccessivo né manca in
alcun modo d'equilibrio, ma reca felicità e tranquillità. In genere i
suoni degli esseri umani recano solo confusione, agitazione e tormento.
Fanno sorgere bramosia, rabbia e illusione, e incitano la gente a essere
avida e vorace, a voler danneggiare gli altri, a eliminarli. I suoni
della foresta non sono così. Se ascoltiamo il canto di un uccello, ciò
non induce bramosia o rabbia.

Dovremmo usare il nostro tempo per generare benessere, proprio ora, nel
presente. Questa era l'intenzione del Buddha: benessere in questa vita,
benessere nelle vite future. In questa vita, fin dall'infanzia dobbiamo
impegnarci nello studio, per imparare almeno abbastanza per guadagnarci
da vivere, in modo tale da poter mantenere noi stessi e infine essere in
grado di avere una famiglia senza vivere in povertà. In genere, non
abbiamo però un comportamento così assennato. Vogliamo solo divertirci.
Andiamo ovunque ci sia una festa, uno spettacolo o un concerto, anche se
s'avvicina il tempo del raccolto. Gli anziani si trascinano dietro i
nipoti pur di ascoltare cantanti famosi. «~Nonna, dove vai?~» «~Porto i
bambini a sentire il concerto!~» Non si sa se è la nonna che porta i
bambini o se sono i bambini a portare la nonna. Quanto tempo ci voglia
per andarci o quanto sia difficile arrivarci non importa, continuano a
farlo. Dicono che stanno portando i bambini ad ascoltare il concerto, ma
la verità è che sono loro a volerci andare. Per loro questo è
trascorrere bene il tempo. Se li invitate a venire in monastero per
ascoltare il Dhamma e imparare quello che è giusto e quello che è
sbagliato, vi dicono: «~Vai tu. Io preferisco restare a casa e
riposare.~» Oppure: «~Ho un gran mal di testa, mi fanno male la schiena
e le ginocchia, non mi sento affatto bene.~» Se però si tratta di andare
a sentire un cantante oppure di assistere a uno spettacolo avvincente,
si precipitano a prendere i bambini, non hanno dolori.

La gente è così. Si sforza tanto e l'unico risultato che ottiene è
aumentare la sofferenza e le difficoltà. Va in cerca dell'oscurità,
della confusione e s'intossica sul sentiero delle illusioni. Il Buddha
ci insegna a essere di beneficio a noi stessi in questa vita. Il supremo
beneficio, la ricchezza spirituale. Dovreste farlo ora, in questa vita.
Dovreste andare in cerca della conoscenza che vi aiuta in questo, per
poter vivere bene la vostra vita, facendo buon uso delle vostre risorse,
lavorando con diligenza sulla via dei retti mezzi di sussistenza. Dopo
aver ricevuto l'ordinazione monastica, iniziai a praticare -- prima a
studiare e poi a praticare -- e così nacque in me la fiducia. Appena
cominciai a praticare pensai alle vite degli esseri di questo mondo.
Tutto mi sembrò straziante e penoso. Che cosa c'era di così penoso?
Tutti i ricchi sarebbero presto morti e costretti a lasciare le loro
grandi case, con figli e nipoti che combattevano per l'eredità. Quando
nella mia mente ho visto accadere queste cose, ho pensato: «~Mmm \ldots{}
questo mi turba.~» Provai compassione nei riguardi sia dei ricchi che
dei poveri, sia dei saggi che dei folli. Tutti quelli che vivono in
questo mondo sono sulla stessa barca.

Riflettere sul nostro corpo, sulla condizione del mondo e sulla vita
degli esseri senzienti fa nascere un senso di stanchezza e di distacco.
Pensando alla vita da monaci, al fatto che abbiamo scelto questo modo di
vivere, di dimorare e di praticare nella foresta, sviluppando un
costante atteggiamento di disincanto e di distacco, la nostra pratica
farà progressi. Pensando continuamente ai fattori della pratica,
arriverà il rapimento estatico e i peli si drizzeranno. Se confrontiamo
la nostra vita di prima con quella di adesso, c'è una sensazione di
gioia quando si riflette sul modo in cui viviamo. Il Dhamma faceva sì
che sensazioni di questo genere mi riempissero il cuore. Non sapevo come
parlarne. Ero presente, in qualsiasi situazione mi trovassi ero presente
e vigile. Significa che avevo alcune conoscenze del Dhamma. La mia mente
era luminosa, e compresi molte cose. Sperimentai beatitudine, reale
appagamento e vero piacere per il mio modo di vivere. Per dirla
semplicemente, sentivo di essere diverso dagli altri. Ero un uomo
adulto, normale, ma potevo vivere nella foresta. Non avevo alcun
rimpianto né pensavo di essermi perso qualcosa. Quando vedevo altri che
avevano una famiglia, pensavo che fosse una cosa davvero spiacevole. Mi
guardavo intorno e pensavo: «~Quanti riescono a vivere come me?~» Giunsi
ad avere fede e fiducia nel Sentiero della pratica che avevo scelto, e
questa fede mi ha sorretto fino a oggi.

Ai primi tempi del Wat Pah Pong vivevano con me quattro o cinque monaci.
Avevamo molte difficoltà. Per quel che vedo ora, la pratica della
maggior parte dei buddhisti è molto carente. Quando al giorno d'oggi
entrate in un monastero si vedono solo delle \emph{kuṭī}, la sala del
tempio, i terreni del monastero e i monaci. Però, non si vede ciò che
veramente rappresenta il cuore della Via del Buddha
(\emph{Buddhasāsana}).\footnote{\emph{buddhasāsana.} La dottrina del
  Buddha; si riferisce in primo luogo agli insegnamenti, ma anche a
  tutte le infrastrutture religiose, grosso modo alla religione
  buddhista, al buddhismo nel suo complesso.} Ne ho parlato spesso. È
per me ragione di tristezza. In passato avevo un compagno di Dhamma che
iniziò a interessarsi più allo studio che alla pratica. Seguì gli studi
di lingua pāli e di \emph{Abhidhamma},\footnote{\emph{Abhidhamma.} Terza
  parte del Canone in pāli, composta di trattati analitici basati su
  elenchi di categorie estratte dai discorsi del Buddha.} e dopo un po'
andò a vivere a Bangkok. L'anno scorso ha finalmente completato i suoi
studi ed ha ricevuto un certificato e dei titoli adeguati a quel che ha
imparato. Così, ora è etichettato. Io non ho alcuna etichetta. Nel mio
studio non ho avuto modelli, ho contemplato le cose e ho praticato, ho
pensato e praticato. Perciò non sono stato etichettato come gli altri.
In questo monastero abbiamo avuto semplici monaci, gente che non aveva
molta istruzione, ma che era determinata a praticare.

All'inizio sono arrivato qui su invito di mia madre. Dopo la mia nascita
fu lei a prendersi cura di me e a mantenermi, ma non avevo ricompensato
la sua gentilezza, e perciò pensai che questo sarebbe stato il modo per
farlo, venire qui al Wat Pah Pong. Quando ero un bambino, mio padre
diceva che Ajahn Sao era venuto a stare qui. Mio padre andò da lui ad
ascoltare il Dhamma. Ero un bambino, ma questo ricordo è rimasto con me,
sempre impresso nella mia mente. Mio padre non ricevette mai
l'ordinazione monastica, ma mi raccontò che andò a porgere omaggio a
questo monaco dedito alla meditazione. Fu la prima volta che vide un
monaco mangiare dalla ciotola per la questua, metteva tutto assieme in
quell'unica ciotola. Riso, curry, dolciumi e pesce, tutto quanto
insieme. Non aveva mai visto una cosa del genere, e si chiese di che
tipo di monaco si trattasse. Questo mi disse quando ero piccolo. Quello
era un monaco dedito alla meditazione. Poi mi raccontò di aver ascoltato
degli insegnamenti di Dhamma da Ajahn Sao. Non si trattava del solito
modo d'insegnare. L'\emph{ajahn} si limitava a dire le cose che erano
nella sua mente. Quello fu il monaco che praticava la meditazione e che
una volta venne a stare qui. Perciò, quando io stesso sono andato via
per praticare, è rimasta con me una sensazione particolare. Quando
pensavo al mio villaggio, mi veniva sempre in mente questa foresta.
Così, quando giunse il tempo di tornare da queste parti, è qui che venni
a stare.

Invitai a venire qui un monaco d'alto rango del distretto di Piboon.
Disse che non poteva. Rimase per un po' e poi disse: «~Questo luogo non
fa per me.~» Lo disse alla gente del posto. Un altro \emph{ajahn} venne
a stare qui per un po' e poi se ne andò anche lui. Io però restai.
Allora questa foresta era davvero distante dal villaggio. Era lontana da
tutto e vivere qui era veramente difficile. C'erano degli alberi di
mango piantati dagli abitanti del villaggio e spesso i frutti maturavano
e andavano a male. Qui crescevano anche dei taro, ma marcivano sul
terreno. Non avrei mai osato prendere una di queste cose. La foresta era
proprio fitta. Quando si arrivava qui con la ciotola, non c'era posto
per poggiarla. Fui costretto a chiedere agli abitanti del villaggio di
ripulire un po' di spazi nella foresta. Era una foresta nella quale la
gente non osava entrare, avevano davvero paura di questo posto. Nessuno
sapeva che cosa io ci stessi a fare qui. La gente non capiva la vita di
un monaco dedito alla meditazione. Dopo essere rimasto qui per un paio
d'anni, iniziò ad arrivare qualche altro monaco, i primi discepoli.
Allora si viveva in modo davvero semplice e serenamente. Ci ammalammo di
malaria, quasi ne morimmo. Però non andammo mai in ospedale. Avevamo già
il nostro sicuro rifugio, facevamo affidamento sul potere spirituale del
Buddha e dei suoi insegnamenti. Di notte il silenzio era assoluto. Mai
nessuno venne qui. Gli unici rumori che si sentivano erano quelli degli
insetti. Le \emph{kuṭī} erano molto appartate nella foresta.

Una notte, intorno alle nove, ho sentito che qualcuno stava uscendo
dalla foresta. Un monaco era molto malato, aveva la febbre, e temeva di
morire. Non voleva morire solo nella foresta. Dissi: «~Va bene.
Cerchiamo qualcuno che non sia malato e che possa badare a chi lo è. Un
malato come può prendersi cura di un altro malato?~» Tutto qui. Non
avevamo medicinali. Avevamo del \emph{borapet}.\footnote{\emph{borapet}
  (in thailandese \thai{บอระเพ็ด}). Un viticcio medicinale molto amaro.} Lo si
faceva bollire e poi si beveva. Quando nel pomeriggio si parlava di
``preparare una bevanda calda'', non è che ci si dovesse pensare molto
su, si poteva intendere solo del \emph{borapet}. Avevamo tutti la febbre
e bevevamo tutti del \emph{borapet}. Non avevamo nient'altro, e non
chiedevamo niente a nessuno. Se un monaco si ammalava davvero
gravemente, gli dicevo: «~Non aver paura. Non ti preoccupare. Se
morirai, ti cremerò io. Ti cremerò proprio qui in monastero. Non si
dovrà andare da nessun'altra parte.~» È così che affrontavo il problema.
Parlare in questo modo rafforzava la mente. Molta era la paura che si
doveva affrontare.

Le condizioni erano dure. I laici molte cose non le sapevano. Ci
portavano del \emph{plah rah},\footnote{Il pesce in salamoia (in
  thailandese \thai{ปลาร้า})
  è una costante della dieta locale.} ma era fatto con pesce crudo e
perciò non lo mangiavamo. Davo una rimescolata, guardavo per bene per
vedere di cosa si trattasse e mi limitavo a lasciare tutto lì. Era
proprio difficile, nessuno può immaginare come fossero le condizioni di
allora. Attualmente qualche traccia ne è rimasta nella pratica dei
monaci che da allora sono rimasti qui. Dopo il Ritiro delle Piogge
potevamo andare in \emph{tudong}\footnote{\emph{tudong} (in thailandese
  \thai{ธุดงค์}). La pratica ascetica di errare a piedi, nelle campagne, in
  pellegrinaggio o alla ricerca di posti tranquilli per ritiri solitari,
  vivendo di cibo offerto in elemosina.} proprio qui, dentro il
monastero. Andavamo a meditare nella quiete profonda della foresta. Di
tanto in tanto ci riunivamo, impartivo qualche insegnamento e poi
tornavamo tutti nella foresta per continuare con la meditazione
camminata e quella seduta. Nella stagione calda praticavamo in questo
modo. Non andavamo in giro alla ricerca di foreste per praticare, perché
le giuste condizioni le avevamo qui. Le pratiche del \emph{tudong}
potevamo effettuarle proprio qui.

Ora dopo le piogge tutti vogliono andare da qualche parte. Di solito il
risultato è che la pratica viene interrotta. Praticare con costanza e
con sincerità è importante, perché è così che si giunge a conoscere le
proprie contaminazioni. Questo modo di praticare è buono e autentico. In
passato era molto più duro. Vale a dire che pratichiamo per non essere
più una persona: la persona dovrebbe morire e trasformarsi in un monaco.
Aderivamo strettamente al Vinaya e ognuno di noi aveva un reale ritegno
riguardo alle proprie azioni. Quando i monaci sbrigavano le faccende,
tiravano acqua dal pozzo o ramazzavano il suolo, non si sentiva parlare.
Il silenzio era assoluto quando si lavavano le ciotole per la questua.
Ora, certi giorni devo mandare qualcuno che dica loro di smettere di
parlare e scopra il motivo di tutta quella confusione. Mi chiedo se lì
fuori stiano facendo a pugni. C'è così tanto rumore che non riesco a
immaginare cosa stia succedendo. Sono costretto a proibire in
continuazione che si chiacchieri. Non so di cosa abbiano bisogno di
parlare. Quando hanno mangiato fino a essere sazi si distraggono a causa
del piacere che provano. E io continuo a dire: «~Quando tornate dalla
questua, non parlate!~» Se qualcuno vi chiede perché non parlate,
rispondete: «~Non ci sento bene.~» Altrimenti diventate uguali a un
branco di cani che abbaiano. Chiacchierare porta con sé emozioni, e
potete finire per fare a pugni, soprattutto quando siete tutti affamati:
i cani sono affamati e le contaminazioni sono attive.

Ecco che cosa ho notato. La gente non si mette a praticare con tutto il
cuore. Ho visto le cose cambiare nel corso degli anni. In passato chi si
addestrava otteneva dei risultati e poteva prendersi cura di se stesso
ma ora, quando le persone sentono parlare di difficoltà, scappano dallo
spavento. Per loro è inconcepibile. Se le cose le rendete facili, allora
tutti sono interessati, ma il punto qual è? La ragione per cui in
passato siamo stati in grado di ottenere dei benefici sta nel fatto che
ci addestravamo insieme con tutto il cuore. I monaci che vivevano qui
praticavano la sopportazione davvero al massimo grado. Le cose le
capivamo insieme, dall'inizio alla fine. Capivamo la pratica. Dopo aver
praticato insieme per molti anni, pensai che fosse giusto farli tornare
nei villaggi dai quali provenivano per fondare dei monasteri. Chi di voi
è arrivato dopo proprio non riesce a immaginare la nostra situazione di
allora. Non so con chi parlarne. La pratica era estremamente severa.
Pazienza e sopportazione erano le cose più importanti con le quali si
conviveva. Nessuno si lamentava. Se avevamo solo riso bianco da
mangiare, nessuno si lamentava. Si mangiava in assoluto silenzio, senza
discutere a proposito del fatto che il cibo fosse saporito o no. La
nostra bevanda calda era il \emph{borapet}.

Uno dei monaci andò nel centro della Thailandia e lì bevve del caffé.
Qualcuno gliene offrì un po' da portare qui. Così, una volta ci capitò
di avere del caffé. Però non avevamo zucchero. Nessuno se ne lamentò.
Dove saremmo mai potuti andare a prendere lo zucchero? Potevamo perciò
dire di aver bevuto veramente del caffé, ma senza neanche un po' di
zucchero che ne addolcisse il sapore. Dipendevamo dagli altri che ci
mantenevano, volevamo che per la gente mantenerci fosse facile e,
ovviamente, non chiedevamo niente a nessuno. In questo modo, si
continuava ad andare avanti senza le cose e sopportando tutte le
situazioni nelle quali ci trovavamo.

Un anno due laici che ci offrivano il loro sostegno, il signor Puang e
la signora Daeng, vennero qui per ricevere l'ordinazione monastica.
Venivano dalla città e non avevano mai vissuto in questo modo senza
nulla, sopportando le privazioni, mangiando come facciamo noi,
praticando sotto la guida di un \emph{ajahn} e assolvendo ai doveri
previsti dalle regole dell'addestramento. Avevano però saputo che un
loro nipote viveva qui e così decisero di venire per l'ordinazione
monastica. Appena l'ebbero ricevuta, un loro amico portò caffé e
zucchero. Vivevano nella foresta per praticare la meditazione, ma
avevano l'abitudine di svegliarsi presto al mattino e bere un
caffellatte prima di fare qualsiasi altra cosa. Stiparono le loro
\emph{kuṭī} di zucchero e caffè. Qui però c'erano i canti del mattino e
la meditazione, e subito dopo i monaci si preparavano ad andare alla
questua, così che non c'era alcuna possibilità di prepararsi il caffé.
Dopo un po' iniziarono a capire come stavano le cose. Il signor Puang
camminava avanti e indietro pensando al da farsi. Non aveva un posto in
cui prepararsi il caffé, né qualcuno sarebbe arrivato a servirglielo, e
così finì col portare tutto nella cucina del monastero e lasciare lì
ogni cosa.

Venire a stare qui, vedere le condizioni in cui realmente si stava in
monastero e il modo di vivere dei monaci dediti alla meditazione lo
buttò proprio giù. Era un uomo anziano e un mio importante parente.
Quello stesso anno si smonacò. Fu una cosa giusta per lui, perché non
aveva ancora sistemato le sue cose. Abbiamo avuto anche del ghiaccio. E
abbiamo visto anche un po' di zucchero, di tanto in tanto. La signora
Daeng dovette andare a Bangkok. Quando parlava del nostro modo di vivere
cominciava a piangere. La gente che non aveva mai visto la vita dei
monaci dediti alla meditazione non aveva idea di come fosse. Mangiare
una volta al giorno, significa fare progressi o tornare indietro? Non
saprei che dire.

Durante la questua, oltre al riso, la gente preparava pacchettini di
salsa di peperoncino da mettere nelle nostre ciotole. Tutto quel che
ricevevamo lo si portava in monastero, lo dividevamo tra noi e
mangiavamo. Non costituiva argomento di conversazione il fatto che
avessimo cose che alla gente piacciono o che il cibo fosse più o meno
gustoso. Mangiavamo per riempirci, tutto qui. Era proprio semplice. Non
c'erano piatti o altre ciotole, tutto andava a finire nella ciotola per
la questua. Nessuno veniva a farci visita. La sera ognuno di noi tornava
nella propria \emph{kuṭī} per praticare. Di notte neanche i cani avevano
il coraggio di stare qui. Le \emph{kuṭī} era molto appartate e lontane
dal luogo di ritrovo. Alla fine della giornata, dopo che tutto era stato
fatto, ci separavamo ed entravamo nella foresta per recarci nelle nostre
\emph{kuṭī}. Questo faceva sì che i cani temessero di non aver alcun
posto sicuro nel quale stare. Perciò seguivano i monaci nella foresta,
ma quando loro salivano nelle \emph{kuṭī} i cani restavano soli e
avevano paura, e così cercavano di seguire un altro monaco, ma anche lui
spariva nella sua \emph{kuṭī}. Nemmeno i cani potevano vivere qui:
questa era la nostra vita per la pratica della meditazione. A volte ci
penso. Nemmeno i cani potevano sopportare questo genere di vita, ma noi
viviamo ancora qui! Una cosa piuttosto estrema. Questa cosa mi fa venire
anche un po' di malinconia.

Ogni genere di ostacoli \ldots{} si viveva con la febbre, ma abbiamo
affrontato la morte e siamo sopravvissuti tutti. Oltre ad affrontare la
morte, siamo stati costretti a vivere in condizioni difficili, ad
esempio con cibo scarso e povero. Però, non fu mai un problema. Mi volto
indietro a guardare le condizioni in cui allora si viveva e le confronto
con quelle di ora. Sono così diverse. Prima non avevamo vassoi e piatti.
Si metteva tutto insieme nella ciotola. Ora non si può fare. Così, se è
un centinaio di monaci a mangiare, dopo c'è bisogno di cinque persone
per lavare i piatti. Quando è l'ora del discorso di Dhamma a volte
stanno ancora lavando i piatti. Queste cose creano complicazioni. Non so
che fare, e mi limiterò a lasciare che sia la vostra stessa saggezza a
farvi riflettere su tutto questo.

Non c'è fine. Coloro ai quali piace lamentarsi troveranno sempre
qualcosa per farlo, non importa quanto favorevoli siano le circostanze.
Il risultato è che i monaci si sono estremamente attaccati ai sapori e
agli aromi. Senza volerlo, a volte mi capita di sentirli parlare dei
loro pellegrinaggi ascetici: «~Ragazzi, lì il cibo è veramente ottimo!
Sono andato in \emph{tudong} al sud, sulla costa, e ho mangiato
moltissimi gamberetti! Ho mangiato grandi pesci dell'Oceano.~» Ecco di
cosa parlano. Quando la mente è catturata da questo genere di
preoccupazioni, è facile immergersi nel desiderio del cibo e attaccarsi
a esso. Quando la mente è priva di controllo, vaga e resta bloccata
nelle immagini, nei suoni, negli odori, nei sapori, nelle sensazioni
tattili e nei pensieri, e praticare il Dhamma diventa difficile. Quando
la gente è attaccata ai sapori, diventa difficile per un \emph{ajahn}
insegnare la retta via. È come quando si alleva un cane. Se gli date
solo del riso bianco da mangiare, crescerà forte e sano. Provate però a
mettere sul suo riso per un paio di giorni un po' di curry: vedrete che
dopo nemmeno lo guarderà il riso bianco.

Immagini, suoni, odori e sapori sono la rovina della pratica del Dhamma.
Sono molto nocivi. Se ognuno di noi non contemplasse l'uso dei nostri
quattro beni indispensabili -- l'abito, il cibo ricevuto in elemosina,
la dimora e le medicine -- la Via del Buddha non potrebbe fiorire.
Potete guardare e vedere che nel mondo per quanto aumentino il progresso
materiale e lo sviluppo, insieme a essi sono pure la confusione e la
sofferenza degli esseri umani ad aumentare. Dopo che si va avanti così
per un po' di tempo, è quasi impossibile trovare una soluzione. Per
questo dico che quando andate in un monastero vedete i monaci, il tempio
e le \emph{kuṭī}, ma non vedete il \emph{Buddhasāsana}. Il
\emph{sāsana}\footnote{\emph{sāsana.} Insegnamento, dispensazione,
  dottrina ed eredità del Buddha; la scuola spirituale buddhista.} è in
declino. È facile da constatare.

Il \emph{sāsana}, ossia l'insegnamento genuino e diretto che istruisce
la gente a essere retta e onesta, e a nutrire una reciproca gentilezza
amorevole, è andato perduto. Turbamento e tensione hanno preso il loro
posto. Coloro che hanno praticato con me per anni in passato conservano
ancora la loro diligenza, ma qui, dopo venticinque anni, vedo fino a che
punto la pratica s'è affievolita. La gente ora non osa spronare se
stessa né praticare così tanto. Ha paura. Teme di arrivare all'estremo
dell'auto-mortificazione. In passato è proprio questo che volevamo. A
volte i monaci digiunavano per più giorni, anche per una settimana.
Volevano vedere la loro mente, volevano addestrare la loro mente. Se era
testarda, la frustavano. La mente e il corpo lavorano insieme. Quando
non si è esperti della pratica, se il corpo è grasso ed è troppo a
proprio agio, la mente sfugge al controllo. Quando scoppia un incendio e
il vento soffia, le fiamme si diffondono e la casa brucia del tutto. È
così. Quando prima parlavo di mangiare poco, di dormire poco e di
parlare poco, i monaci queste cose le prendevano a cuore. Ora, però,
questo genere di discorsi sembra essere sgradito alla mente dei
praticanti. «~La nostra strada la troveremo. Perché dovremmo soffrire e
praticare in modo così austero? È l'estremo dell'auto-mortificazione,
non è il sentiero del Buddha.~» Non appena qualcuno parla così, sono
tutti d'accordo. Sono affamati. Che cos'è che posso dire a questa gente?
Continuo a cercare di correggere questo comportamento, ma pare che ora
le cose stiano proprio così.

Tutti voi, per favore, rendete le vostre menti forti e stabili. Oggi
siete venuti da vari monasteri affiliati al Wat Pah Pong e siete qui
riuniti per porgermi omaggio in quanto vostro insegnante, vi siete
riuniti qui come amici nel Dhamma. Per questa ragione vi sto offrendo
qualche insegnamento sul Sentiero della Pratica. La Pratica del rispetto
è il Dhamma supremo. Quando c'è rispetto vero, non ci sarà mancanza
d'armonia, la gente non litigherà e le persone non si ammazzeranno a
vicenda. Porgere omaggio a un maestro spirituale, ai nostri precettori e
insegnanti, ci induce a fiorire. Il Buddha ne parlò come di una cosa di
buon auspicio.

Alla gente di città piace mangiare funghi. Chiedono: «~Da dove vengono i
funghi?~» Qualcuno risponde: «~Crescono dalla terra.~» Così questa gente
prende un cestino e se ne va a camminare per la campagna, pensando che i
funghi crescano allineati sul ciglio della strada per farsi raccogliere.
Camminano, camminano e camminano, vanno su per le colline e attraversano
i campi senza vedere un solo fungo. Chi abita in un villaggio ed è già
andato a raccogliere funghi sa dove cercarli, sa in quale parte della
foresta recarsi. Però la gente di città ha fatto solo l'esperienza di
vedere i funghi nei piatti. Sente dire che crescono dalla terra e pensa
che sia facile trovarli, ma non è così. Addestrare la mente al
\emph{samādhi} è la stessa cosa. Ci facciamo l'idea che sia facile.
Però, quando ci sediamo ci fanno male le gambe, ci fa male la schiena,
ci sentiamo stanchi, abbiamo caldo, c'è prurito. Così iniziamo a
scoraggiarci e pensiamo che il \emph{samādhi} sia tanto lontano da noi
quanto il cielo dalla terra. Non sappiamo che cosa fare e le difficoltà
ci sommergono. Però, se riceviamo degli insegnamenti, un po' alla volta
diventerà più facile.

Voi che siete venuti qui per praticare il \emph{samādhi} sapete per
esperienza quanto sia difficile. Anch'io ho avuto i miei problemi con il
\emph{samādhi}. Mi sono addestrato con un \emph{ajahn}, e quando stavamo
seduti aprivo gli occhi e guardavo: «~Oh! L'\emph{ajahn} sta forse per
smettere?~» Chiudevo di nuovo gli occhi e cercavo di sopportare ancora
un po'. Mi sentivo come se stessi morendo, e continuavo ad aprire gli
occhi, ma lui, lì seduto, sembrava così a suo agio. Un'ora, due ore, io
stavo agonizzando, ma l'\emph{ajahn} non si muoveva. Dopo un po' iniziai
ad aver paura delle sedute di meditazione. Quando era il momento di
praticare il \emph{samādhi}, mi spaventavo. Agli inizi addestrarsi al
\emph{samādhi} è difficile. Tutto è difficile quando non sappiamo come
si fa. Questo è il nostro ostacolo. Però, addestrandosi le cose possono
cambiare. Quel che è buono può infine vincere e sovrastare quello che
buono non è. Quando si combatte si ha la tendenza a essere pusillanimi.
È una reazione normale, tutti ci siamo passati. Per questo è importante
addestrarsi per un po' di tempo. È come aprirsi un sentiero nella
foresta. All'inizio camminare è difficoltoso, ci sono un sacco di
ostacoli, ma camminandoci su in continuazione ci apriamo la strada. Dopo
un po' rimuoviamo rami e tronchi, e il terreno diventa solido e levigato
perché ci abbiamo camminato sopra ripetutamente. Così abbiamo un buon
sentiero per attraversare la foresta. Somiglia a quando addestriamo la
mente. Continuandolo a fare, la mente diventa luminosa.

Ad esempio, noi gente di campagna cresciamo mangiando riso e pesce.
Quando poi veniamo a imparare il Dhamma ci viene detto di astenerci dal
nuocere, che non dovremmo uccidere creature viventi. Cosa possiamo fare,
allora? Siamo in un vicolo cieco. I campi sono il nostro mercato. Se
l'insegnante ci dice di non uccidere, non possiamo mangiare. Basta
questo e siamo bloccati, non sappiamo cosa fare. Come ci nutriremo? Per
noi gente di campagna non sembra esserci via d'uscita. I nostri mercati
sono i campi e la foresta. Dobbiamo catturare animali per ucciderli e
mangiarli. Per anni ho cercato di insegnare alla gente come affrontare
questo problema. Le cose stanno così. I contadini mangiano riso. Per la
maggior parte, la gente che lavora nei campi coltiva il riso e lo
mangia. E un sarto in città, che fa? Mangia la macchine da cucire?
Mangia abiti? Cominciate col pensare a questo. Se siete contandini,
mangiate riso. Se qualcuno vi offre un altro lavoro, rifiutate dicendo:
«~Non posso. Come farò senza mangiare riso?~» I fiammiferi che usate a
casa: siete in grado di fabbricarli? No. Come avete fatto per entrare in
possesso dei fiammiferi? Capita forse che abbia dei fiammiferi solo chi
è in grado di fabbricarli? E le ciotole dalle quali mangiate? Qui nei
villaggi vicini c'è qualcuno che sa come fabbricarle? La gente le ha a
casa propria? Dove le avete prese? Ci sono moltissime cose che non siamo
in grado di fabbricare, però possiamo guadagnare del denaro per
acquistarle. Questo significa usare la nostra intelligenza per trovare
una soluzione.

Anche per la meditazione dobbiamo fare una cosa del genere. Troviamo il
modo di evitare le cattive azioni e di praticare quel che è giusto.
Prendete in considerazione il Buddha e i suoi discepoli. Prima erano
degli esseri comuni, ma svilupparono se stessi per progredire attraverso
gli stadi dell'Illuminazione, dall'Entrata nella Corrente\footnote{Entrata
  nella Corrente (\emph{sotāpatti}). Evento tramite il quale si diviene
  \emph{sotāpanna}, il primo livello dell'Illuminazione.} fino alla
condizione di \emph{arahant}. Lo fecero per mezzo dell'addestramento.
La saggezza cresce gradualmente. Sopraggiunge un senso di vergogna nei
riguardi delle cattive azioni. Una volta ho insegnato a una persona
saggia. Era un sostenitore laico che veniva a praticare e a osservare i
precetti nei giorni dell'osservanza lunare, ma continuava ad andare a
pesca. Cercavo di insegnargli a non farlo, ma non riuscivo a risolvere
questo problema. Diceva che non uccideva i pesci: erano loro che
andavano a ingoiare l'amo. Continuai a insegnargli, ed egli iniziò a
provare un po' di rimorso. Se ne vergognava, ma continuava a farlo. Poi
il suo modo di ragionare cambiò. Lanciava l'amo in acqua e diceva: «~Il
pesce che ha esaurito il suo kamma di essere vivente, venga a
ingoiare il mio amo.~Se il tuo tempo non è giunto, non ingoiarlo.~» La
sua giustificazione era mutata, ma i pesci continuavano a ingoiare
l'amo.

Finalmente iniziò a guardarli, con l'amo infilato in bocca, e
cominciò a provare un po' di pietà. Però, la sua mente non riusciva
ancora a decidersi: «~Bene, ho detto ai pesci di non ingoiare l'amo se
il loro tempo non è giunto. Che cosa posso fare se continuano a
venire?~» Poi pensò: «~Però muoiono a causa mia.~» Pensa e ripensa,
infine riuscì a smettere. Poi fu la volta delle rane. Non riusciva a
smettere di catturare le rane per mangiarsele. «~Non farlo!~», gli
dicevo. «~Guardale per bene \ldots{} d'accordo, se non riesci a smettere di
ucciderle non te lo proibirò, ma per favore prima di farlo guardale.~»
Così, prese una rana e la guardò. Guardò il muso, gli occhi, le zampe.
«~Ha braccia e gambe, come mio figlio. Gli occhi sono aperti, mi sta
guardando.~» Si sentì male. Però le uccideva ancora. Le guardava una per
una in questo modo e poi le uccideva, sentendo che stava facendo
qualcosa di male. La moglie diceva che non avrebbero avuto nulla da
mangiare se egli non avesse ucciso le rane, e lo spingeva a farlo.

Alla fine non ci riuscì più. Le catturava, ma non spezzava più le loro
zampe come faceva prima. Prima spezzava le zampe alle rane in modo che
non potessero più saltare via. Però non riusciva ancora a decidersi di
lasciarle andare. «~Bene, mi sto solo prendendo cura di loro, le nutro.
Le allevo e basta. Qualsiasi cosa gli altri vogliano fare alle rane, io
non ne so nulla.~» Ovviamente lo sapeva: gli altri continuavano a
ucciderle per mangiarle. Dopo un po' fu costretto ad ammetterlo. «~Bene,
come che sia, ho ridotto del cinquanta per cento il mio cattivo
kamma. È qualcun altro a ucciderle.~» Questa cosa cominciava a
farlo diventare matto, ma ancora non riusciva a lasciar andare.
Continuava a tenere le rane a casa. Non spezzava più le loro zampe, ma
era la moglie a farlo. «~È colpa mia. Anche se non lo faccio io, gli
altri lo fanno per colpa mia.~» Alla fine rinunciò. Sua moglie però si
lamentava: «~Come faremo? Cosa dovremmo mangiare ora?~» Era proprio in
trappola. Quando andò in monastero, l'\emph{ajahn} gli disse che cosa
avrebbe dovuto fare. Quando tornò a casa, la moglie gli disse che cosa
avrebbe dovuto fare. L'\emph{ajahn} gli disse di smettere di farlo. La
moglie lo incitò a continuare. Che fare? Quanta sofferenza! È così che
dobbiamo soffrire visto che siamo nati in questo mondo.

Infine anche la moglie fu costretta a lasciar andare. Così smisero di
uccidere le rane. Lavorava nei campi e si prendeva cura dei suoi bufali.
Sviluppò l'abitudine di liberare pesci e rane. Quando vedeva un pesce
nella rete, lo liberava. Una volta andò a casa di un amico. In un vaso
vide delle rane e le liberò. Poi giunse la moglie del suo amico per
preparare la cena. Tolse il coperchio al vaso e vide che le rane non
c'erano più. Immaginò che cos'era successo: «~È stato quello lì, che ha
a cuore i meriti.~» La donna riuscì a catturare una rana e la usò per
fare della pasta di peperoncini. Si misero seduti per mangiare e quando
lui stava per affondare un po' del suo riso nella pasta di peperoncini,
lei disse: «~Ehi, tu che hai a cuore i meriti! Non dovresti mangiare
quella roba! In quella pasta di peperoncini c'è una rana.~» Fu troppo.
Quanto dolore solo per essere vivi e cercare di nutrirsi! Quando ci
pensò, non riuscì a trovare una via d'uscita. Era già anziano, e così si
decise per l'ordinazione monastica. Fece i preparativi per
l'ordinazione, si rasò i capelli ed entrò in casa. Appena la moglie
vide che si era rasato i capelli, iniziò a piangere. Lui la supplicò:
«~Da quando sono nato non ho avuto alcuna occasione per ricevere
l'ordinazione monastica. Per favore, dammi la tua benedizione. Voglio
ricevere l'ordinazione, ma poi lascerò l'abito e tornerò a casa.~» La
moglie acconsentì.

Ricevette l'ordinazione nel monastero del suo villaggio e dopo la
cerimonia chiese al suo precettore cosa avrebbe dovuto fare. Il
precettore gli disse: «~Se lo stai facendo davvero seriamente, dovresti
andare a praticare la meditazione. Segui un maestro di meditazione, non
restare qui vicino agli edifici.~» Comprese, e decise di farlo. Dormì
una notte nel tempio, e al mattino si congedò e chiese dove avrebbe
potuto trovare Ajahn Tongrat.\footnote{Durante la giovinezza di Ajahn
  Chah, Ajahn Tongrat era un noto maestro di meditazione.} Questo nuovo
monaco che nemmeno riusciva ancora a indossare per bene il suo abito,
mise in spalla la sua ciotola, andò a cercare Ajahn Tongrat e imboccò la
strada giusta. «~Venerabile \emph{ajahn}, non ho altro scopo nella vita.
Voglio offrirti il mio corpo e la mia vita.~» Ajahn Tongrat rispose:
«~Molto bene! Questo significa molti meriti! Bastava poco e non mi
avresti trovato. Stavo per andarmene. Fai le prostrazioni e siediti
lì.~» Il nuovo monaco chiese: «~Ora che ho ricevuto l'ordinazione
monastica, che cosa dovrei fare?~» Stavano seduti in prossimità di un
vecchio ceppo d'albero. Ajahn Tongrat lo indicò e disse: «~Sii come
questo ceppo d'albero. Non fare nient'altro, sii solo come questo ceppo
d'albero.~» Gli insegnò la meditazione in questo modo.

Ajahn Tongrat se ne andò per la sua strada e il monaco restò lì, a
contemplare le sue parole: «~L'\emph{ajahn} mi ha detto di essere come
un ceppo d'albero. Che devo fare?~» Ci pensò in continuazione, quando
camminava, quando sedeva o era disteso per dormire. Pensava al ceppo che
prima era un seme, poi cresceva e si trasformava in un albero, poi
diventava più grande e vecchio, e alla fine veniva abbattuto e restava
solo il ceppo. Adesso che era un ceppo, non sarebbe più cresciuto e da
esso nulla sarebbe più sbocciato. Continuò a rifletterci nella sua
mente, a pensarci in continuazione, finché questo non divenne il suo
oggetto di meditazione. Lo espanse, lo applicò a tutti i fenomeni e fu
in grado di interiorizzarlo e di applicarlo a se stesso. «~Forse tra un
po' diventerò come questo ceppo, una cosa inutile.~» L'averlo compreso
gli fece prendere la decisione di non lasciare l'abito monastico. A
questo punto la sua mente si era pacificata. In lui si erano realizzate
delle condizioni che si unificarono per condurlo a quel livello. Quando
la mente diventa così, non c'è più nulla che possa fermarla. Siamo tutti
nella stessa barca. Per favore pensate a tutto questo, e cercate di
applicarlo alla vostra pratica. Essendo nati come esseri umani è tutto
pieno di difficoltà. E non è che solo ora per noi ci siano delle
difficoltà, anche in futuro ce ne saranno. I giovani cresceranno, quelli
che sono cresciuti invecchieranno, con l'età si ammaleranno, e la gente
malata morirà. Continuerà ad andare avanti così, il ciclo delle
incessanti trasformazioni non avrà mai termine.

È per questo che il Buddha ci insegnò a meditare. Nella meditazione
dobbiamo prima praticare il \emph{samādhi}, che significa rendere
immobile e serena la mente, come l'acqua in un catino. Se continuiamo a
metterci dentro roba e a sobillarla, sarà sempre torbida. Se permettiamo
alla mente di essere sempre pensierosa e di preoccuparsi continuamente
delle cose, non vedremo mai nulla con chiarezza. Se lasciamo che nel
catino l'acqua si assesti e divenga ferma, in essa sarà possibile vedere
l'immagine riflessa di qualsiasi cosa. Quando la mente è composta e
immobile, la saggezza è in grado di vedere le cose. La luce brillante
della saggezza supera qualsiasi altra luminosità.

Quale fu il consiglio del Buddha a proposito di come praticare? Insegnò
a praticare come la terra. A praticare come l'acqua. A praticare come il
fuoco. A praticare come il vento. A praticare come fanno le ``vecchie
cose'', le cose di cui siamo fatti: l'elemento solido della terra,
l'elemento liquido dell'acqua, l'elemento caldo del fuoco, l'elemento
in movimento dell'aria. Se qualcuno scava nella terra, la terra non se
ne preoccupa. La si può ammucchiare, coltivare o irrigare. Vi si può
seppellire quel che sta marcendo. La terra rimarrà indifferente. L'acqua
può essere bollita, ghiacciata o usata per lavare cose sporche, non ne
risente. Il fuoco può bruciare cose belle e fragranti oppure brutte e
sudicie, non gli importa. Quando il vento soffia, soffia su qualsiasi
cosa, fresca o marcia, bella o brutta, senza preoccuparsene.

Il Buddha usò questa analogia. Ognuno di noi è un aggregato dovuto alla
riunione degli elementi terra, acqua, fuoco e vento. Se in esso cercate
di trovare una persona reale, non potete riuscirci. C'è solo questa
unione di elementi. Però, durante tutta la nostra vita non pensiamo mai
a separarli in questo modo per vedere che cosa veramente ci sia dentro
di noi. Pensiamo solo questo: «~Questo sono io, quello è mio.~» Abbiamo
sempre visto tutto quanto in termini di un ``sé'', senza mai capire che
ci sono solo terra, acqua, fuoco e vento. Però, questo è quello che
insegna il Buddha. Egli parla dei quattro elementi e ci esorta a vedere
che è quello che siamo. Ci sono terra, acqua, fuoco e vento, qui non c'è
nessuna persona. Contemplate questi elementi per vedere che non c'è
alcun essere o individuo, ma solo terra, acqua, fuoco e vento.

È una cosa profonda, vero? È nascosta in profondità. La gente guarda, ma
non la vede. Siamo soliti contemplare tutto nei termini del sé e
dell'altro da sé, lo facciamo sempre. È per questo che la nostra
meditazione non va ancora in profondità. Non raggiunge la Verità e noi
non andiamo al di là del modo in cui le cose sembrano essere. Restiamo
bloccati nelle convenzioni del mondo, e restare bloccati nel mondo
significa restare all'interno del ciclo della trasformazione: ottenere
le cose e perderle, morire e nascere, nascere e morire, soffrire nel
regno della confusione. Qualsiasi cosa desideriamo, qualsiasi cosa alla
quale aspiriamo non si realizza nel modo in cui davvero vogliamo, perché
le cose le vediamo in modo errato. Gli attaccamenti ai quali ci
aggrappiamo sono così. Siamo ancora lontani, proprio molto lontani dal
vero Sentiero del Dhamma. Mettetevi subito al lavoro. Non dite: «~Quando
sarò più vecchio comincerò ad andare in monastero.~» Che cosa significa
invecchiare? I giovani invecchiano come gli anziani. Cominciano a
invecchiare fin dalla nascita. Ci piace dire: «~Quando sarò più vecchio,
quando sarò più vecchio.~» Ehi! I giovani sono più vecchi di quanto non
lo fossero prima. Ecco cosa significa invecchiare. Tutti voi, per
favore, date un'occhiata a questa cosa. Tutti portiamo questo fardello.
Tutti noi abbiamo il compito di lavorarci sopra. Pensate ai vostri
genitori e ai vostri nonni. Sono nati, poi sono invecchiati e alla fine
sono morti. Non sappiamo dove siano andati.

Per questo il Buddha voleva che cercassimo il Dhamma. Questo genere di
conoscenza è la più importante di tutte. Ogni altro genere di conoscenza
o di studio che non concordi con la via buddhista comporta un imparare
che include \emph{dukkha}. La nostra pratica del Dhamma dovrebbe
condurci oltre la sofferenza. Se non riusciamo a trascendere del tutto
la sofferenza, dovremmo almeno essere in grado di trascenderla un po',
ora, nel presente. Ad esempio, se qualcuno ci parla con durezza e non ci
arrabbiamo con lui significa che abbiamo trasceso la sofferenza. Se ci
arrabbiamo, non abbiamo trasceso la sofferenza. Quando qualcuno ci parla
con durezza, se riflettiamo sul Dhamma vedremo che a parlare è solo un
mucchietto di terra. Va bene, mi sta criticando, ma sta criticando solo
un mucchietto di terra. Un mucchietto di terra che critica un altro
mucchietto di terra. Acqua che critica acqua. Fuoco che critica fuoco.
Vento che critica vento. Se però vediamo davvero le cose in questo modo,
probabilmente gli altri diranno che siamo matti. «~Non gli importa di
nulla. È privo di sensibilità.~» Quando qualcuno muore non ci agiteremo
e non piangeremo, e ci diranno di nuovo che siamo matti. Dove possiamo
mai stare?

Si riduce davvero tutto a questo. Dobbiamo praticare per capire da noi
stessi. Andare oltre la sofferenza non dipende da quello che gli altri
pensano di noi, ma dal nostro individuale stato mentale. Non vi
preoccupate di quel che diranno, noi sperimentiamo la Verità da noi
stessi. Così possiamo essere a nostro agio. In genere, però, non
arriviamo fino a questo punto. I giovani si recano in monastero una o
due volte e, poi, quando tornano a casa, i loro amici li prendono in
giro: «~Ehi, Dhamma Dhammo!~» Imbarazzati, non se la sentono più di
tornare qui. Alcuni di loro mi hanno raccontato di essere venuti qui per
ascoltare gli insegnamenti e di averli compresi a sufficienza per
smettere di bere e di star a perder tempo andandosene in giro con altra
gente. Gli amici però li sminuivano: «~Sei andato in monastero e adesso
non vuoi più uscire con noi a bere. Che cosa ti è successo?~» Si
sentivano imbarazzati e finivano di nuovo per fare le solite vecchie
cose. Per le persone è difficile essere coerenti.

Invece di puntare troppo in alto, praticate perciò la pazienza e la
sopportazione. Esercitarsi alla pazienza e al contenimento in famiglia
va già abbastanza bene. Non discutete e non litigate, se ci riuscite va
bene, per il momento avete già trasceso la sofferenza. Quando succede
qualcosa, richiamate alla mente il Dhamma. Pensate a quello che vi hanno
insegnato le vostre guide spirituali. Vi hanno insegnato a lasciar
andare, a rinunciare, ad astenervi, a deporre le cose. Vi hanno
insegnato a sforzarvi e a combattere in questo modo per risolvere i
vostri problemi. Di quali problemi stiamo parlando? Come va in famiglia?
Avete problemi con le vostre mogli o i vostri mariti, con i vostri
amici, al lavoro e così via? Tutte queste cose vi fanno venire il mal di
testa, o no? Stiamo parlando di questi problemi. Gli insegnamenti dicono
che potete risolvere i problemi della vita quotidiana con il Dhamma.

Siamo nati come esseri umani. Dovrebbe essere possibile vivere con una
mente felice. Svolgiamo il nostro lavoro a seconda delle nostre
responsabilità. Quando le cose si fanno difficili pratichiamo la
sopportazione. Guadagnarsi da vivere in modo giusto è una maniera di
praticare il Dhamma, la pratica di vivere in modo morale. Vivere così,
felicemente e armoniosamente, va già abbastanza bene. Però, di solito
perdiamo le occasioni. Non fatelo! Se venite qui nel giorno di
osservanza lunare per prendere i precetti e poi tornate a casa e
litigate, questo è perdere un'occasione. Gente, lo sentite quello che
dico? Comportarsi in questo modo è solo perdere un'occasione. Significa
che il Dhamma non lo vedete neanche un pochettino. Non c'è alcun
beneficio. Per favore, capitelo. Oggi avete ascoltato il Dhamma per un
giusto intervallo di tempo.

