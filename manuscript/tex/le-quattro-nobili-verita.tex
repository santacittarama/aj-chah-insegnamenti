\chapter{Le Quattro Nobili Verità}

Oggi l'abate mi ha invitato a offrirvi un insegnamento e vi chiedo
perciò di sedere in silenzio e di rendere composte le vostre menti. A
causa della barriera indotta dalla lingua, dobbiamo avvalerci di un
traduttore e così, se non prestate la dovuta attenzione, potreste non
capire. Il mio soggiorno qui è stato molto gradevole. Sia il maestro sia
voi, i suoi discepoli, siete stati tutti molto gentili, cordiali e
sorridenti, come s'addice a chi pratica il vero Dhamma. Anche la vostra
tenuta suscita davvero ispirazione, è così grande! Ammiro il vostro
impegno nel ristrutturarla per fondarvi un luogo per la pratica del
Dhamma.

Ho insegnato per molti anni e anch'io ho avuto le mie difficoltà. Ora
sono complessivamente circa quaranta i monasteri\footnote{Al momento
  della stampa dell'edizione inglese (2011) i monasteri affiliati al Wat
  Nong Pah Pong erano, tra grandi e piccoli, più di trecento.} affiliati
al mio, il Wat Nong Pah Pong, ma tuttora ho dei discepoli ai quali è
difficile insegnare. Alcuni sanno, ma non si preoccupano di praticare,
altri né sanno né cercano di praticare. Non so che fare. Perché la mente
degli esseri umani è fatta così? Essere ignoranti non va tanto bene ma,
quando lo dico, continuano a non ascoltare. Non so che cosa potrei fare
di più. Nella pratica la gente è così piena di dubbi, è sempre incerta.
Tutti vogliono entrare nel \emph{Nibbāna}, ma non vogliono percorrere il
Sentiero. È sconcertante. Quando dico loro di fare meditazione hanno
paura o, se non sono preda di essa, lo sono della semplice sonnolenza. A
loro per lo più piace fare cose che io non insegno. Quando mi sono
incontrato con il vostro venerabile abate, gli ho chiesto come fossero i
suoi discepoli. Mi ha detto che sono come i miei. Questo è il dolore di
essere un insegnante.

L'insegnamento che oggi vi offrirò è un modo per risolvere i problemi
nel momento presente, in questa vita. Alcuni dicono di avere così tanto
lavoro da non aver tempo per praticare il Dhamma. Mi chiedono: «~Che
possiamo fare?~» Io dico: «~Respirate mentre lavorate?~» E loro: «~È
ovvio che respiriamo!~» «~Com'è allora che avete il tempo per respirare
se siete così indaffarati?~» Rispondo così, e loro non sanno che cosa
replicare. «~Se mentre lavorate avete \emph{sati}, avrete un sacco di
tempo per praticare.~»

Praticare la meditazione è proprio come respirare. Respiriamo quando
lavoriamo, respiriamo quando dormiamo, respiriamo quando siamo seduti.
Perché abbiamo il tempo di respirare? Siccome comprendiamo l'importanza
del respiro, riusciamo sempre a trovare il tempo per respirare. Allo
stesso modo, se comprendiamo l'importanza della pratica di meditazione,
troveremo sempre il tempo per praticare.

Qualcuno di voi ha mai sofferto? Siete mai stati felici? Proprio lì è la
verità, quello è il luogo in cui dovete praticare il Dhamma. Chi è a
essere felice? La mente. Chi soffre? La mente. Ovunque queste cose
sorgono, è lì che cessano. Avete mai sperimentato la felicità? Avete mai
sperimentato la sofferenza? Questo è il nostro problema. Se conosciamo
\emph{dukkha}, la causa della sofferenza, la fine della sofferenza e il
Sentiero che conduce alla fine della sofferenza, possiamo risolvere il
problema.

Ci sono due tipi di sofferenza. La sofferenza ordinaria e quella non
ordinaria. La sofferenza ordinaria è la sofferenza naturalmente innata
nei fenomeni condizionati: stare in piedi è sofferenza, stare seduti è
sofferenza, stare distesi è sofferenza. Questa è la sofferenza innata in
tutti i fenomeni condizionati. Perfino il Buddha sperimentò queste cose,
sperimentò agio e dolore, ma li riconobbe come condizioni naturali. Egli
seppe come superare queste sensazioni ordinarie, naturali, di agio e di
dolore, mediante la comprensione della loro vera natura. Siccome
comprese questa ``naturale sofferenza'', tali sensazioni non lo
turbarono.

Il secondo tipo di sofferenza è importante, è la sofferenza che,
dall'esterno, si insinua dentro, la ``sofferenza non ordinaria''. Se
siamo malati possiamo aver bisogno che il dottore ci faccia
un'iniezione. Quando l'ago entra e perfora la pelle, c'è un po' di
dolore, è naturale. Quando l'ago viene estratto, quel dolore scompare. È
simile alla sofferenza di tipo ordinario, non è un problema, tutti la
sperimentano. La sofferenza non ordinaria è la sofferenza che sorge da
ciò che chiamiamo \emph{upādāna}, l'aggrapparsi alle cose. È come farsi
un'iniezione con una siringa piena di veleno. Non si tratta più di un
tipo di dolore ordinario, si tratta di un dolore che conduce alla morte.
È simile alla sofferenza che sorge dall'attaccamento.

L'errata visione, la mancata conoscenza della natura impermanente di
tutte le cose condizionate, è un problema di genere differente. Le cose
condizionate costituiscono il regno del \emph{saṃsāra}. Volere che le
cose non cambino: se pensiamo in questo modo dobbiamo soffrire. Quando
pensiamo che il corpo siamo noi o che ci appartiene, abbiamo paura
quando vediamo che cambia. Prendete in considerazione il respiro. Una
volta che è entrato deve uscire, dopo essere uscito deve entrare di
nuovo. È la sua natura, è così che riusciamo a vivere. Le cose non vanno
come vogliamo. I fenomeni condizionati sono così, ma noi non lo
comprendiamo.

Supponiamo di aver perduto qualcosa. Se pensassimo che quell'oggetto è
davvero nostro, ci rimugineremmo sopra. Qualora non riuscissimo a
vederlo come una cosa condizionata che segue le leggi della natura,
sperimenteremmo sofferenza. Se inspirate solamente, riuscite a vivere?
Le cose condizionate devono naturalmente cambiare in questo modo. Vedere
questo è vedere il Dhamma, vedere \emph{aniccā}, il cambiamento. Viviamo
in dipendenza da questo cambiamento. Quando conosciamo il modo in cui
sono le cose, le possiamo lasciar andare.

Praticare il Dhamma significa sviluppare la comprensione del modo di
essere delle cose, e così la sofferenza non sorge. Se pensiamo
erroneamente, siamo in contrasto con il mondo, in contrasto con il
Dhamma e con la Verità. Supponiamo di essere malati e di dover andare in
ospedale. La maggior parte delle persone pensa: «~Per favore, non fatemi
morire, voglio stare meglio.~» Questo è un pensiero errato, condurrà
alla sofferenza. Pensando, dovete dire a voi stessi: «~Se guarisco
guarisco, se muoio muoio.~» Questo è retto pensiero, perché in ultima
analisi non potete controllare i fenomeni condizionati. Se pensate in
questo modo, che si muoia o che si guarisca non potete sbagliare, non
dovete preoccuparvi. La mente che vuole guarire a tutti i costi e che ha
paura quando pensa di morire è una mente che non comprende i fenomeni
condizionati. Dovreste pensare in questo modo: «~Se sto meglio va bene,
se non sto meglio va bene.~» Così non potete sbagliarvi, non dovete aver
paura o piangere, perché siete in sintonia con il modo in cui sono le
cose.

Il Buddha vide con chiarezza. Il suo insegnamento è sempre pertinente,
non è mai obsoleto. Non cambia mai. Oggi è ancora così, non è cambiato.
Prendendo a cuore quest'insegnamento, possiamo ottenere come ricompensa
pace e benessere. Negli insegnamenti vi è la riflessione sul ``non-sé'':
«~Questo non è il mio sé, questo non mi appartiene.~» Alla gente, però,
non piace ascoltare questo genere d'insegnamento, perché è attaccata
all'idea di un sé. Questa è la causa della sofferenza. Dovreste
prenderne atto.

Oggi una donna mi ha chiesto come affrontare la rabbia. Le ho detto che
la prossima volta che si arrabbia dovrebbe caricare la sveglia e
mettersela di fronte. Poi dovrebbe dare alla sua rabbia due ore di tempo
per andarsene. Se fosse davvero la ``sua'' rabbia, potrebbe
probabilmente dirle di andare via in questo modo: «~Vattene entro due
ore!~» Ma la rabbia non è ai nostri ordini. A volte dopo due ore è
ancora lì, altre volte in un'ora se n'è già andata. Attaccarsi alla
rabbia come se fosse un possesso personale causerà sofferenza. Se ci
appartenesse davvero, dovrebbe obbedirci. Se non ci obbedisce, significa
che ci stiamo ingannando. Sia che la mente ami sia che la mente odi, non
ci cascate, è tutto un inganno.

Qualcuno di voi si è mai arrabbiato? Quando siete arrabbiati vi sentite
bene o male? Se ci si sente male, perché non gettate via quella
sensazione? Perché disturbarsi a tenerla? Come si può dire di essere
saggi e intelligenti se ci attacchiamo a cose di questo genere? Da
quando siete nati fino a ora quante volte la mente vi ha ingannato con
la rabbia? A volte la mente può indurre un'intera famiglia a discutere,
o farvi piangere per una notte intera. Tuttavia continuiamo ad
arrabbiarci, continuiamo ad aggrapparci alle cose e a soffrire. Se non
vedete la sofferenza, dovrete continuare a soffrire all'infinito, senza
tregua. Il mondo del \emph{saṃsāra} è così. Se sappiamo com'è, possiamo
risolvere il problema.

L'insegnamento del Buddha afferma che, per vincere la sofferenza, non
c'è nulla di meglio del capire che ``questo non è il mio sé'', che
``questo non è mio''. È il metodo migliore. Di solito, però, non ci
badiamo. Quando la sofferenza sorge ci limitiamo a piangere, senza
imparare da essa. Perché è così? Dobbiamo guardare queste cose in modo
diretto, per sviluppare \emph{Buddho}, Colui che Conosce.

Fate attenzione, alcuni di voi potrebbero non accorgersi che questo è un
insegnamento di Dhamma. Vi sto offrendo un Dhamma che non si trova nelle
Scritture. La maggior parte della gente legge le Scritture ma non vede
il Dhamma. Oggi vi sto offrendo un insegnamento che nelle Scritture non
c'è. Alcuni potrebbero non coglierlo o non essere in grado di
comprenderlo.

Supponiamo che due persone stiano camminando vicine e che vedano
un'anatra e una gallina. Una dice: «~Perché quella gallina non è come
l'anatra, perché l'anatra non è come la gallina?~» Vuole che la gallina
sia un'anatra e che l'anatra sia una gallina. È impossibile. Se è
impossibile, allora anche se quella persona desiderasse per il resto
della vita che l'anatra fosse una gallina e che la gallina fosse
un'anatra, ciò non avverrebbe, perché la gallina è una gallina e
l'anatra è un'anatra. Soffrirebbe per tutto il tempo che si trovasse a
pensare in quel modo. L'altra persona potrebbe invece vedere che la
gallina è una gallina e che l'anatra è un'anatra, e questo è tutto. Non
c'è problema. Vede in modo giusto. Se volete che l'anatra sia una
gallina e che la gallina sia un'anatra, soffrirete davvero.

Allo stesso modo, la legge di \emph{aniccā} afferma che tutte le cose
sono impermanenti. Se volete che le cose siano permanenti state per
soffrire. Ogni volta che l'impermanenza si mostrerà, resterete delusi.
Chi vede che le cose sono per natura impermanenti sarà a proprio agio,
non sarà in conflitto. Chi vuole che le cose siano permanenti, sarà in
conflitto, probabilmente perderà il sonno. Questo significa essere
ignoranti a riguardo di \emph{aniccā}, l'impermanenza, l'insegnamento
del Buddha.

Dove dovreste guardare, se volete conoscere il Dhamma? Dovete guardare
nel corpo e nella mente. Non lo troverete sugli scaffali di una
libreria. Per vedere davvero il Dhamma, dovete guardare all'interno del
vostro corpo e della vostra mente. Ci sono solo queste due cose. La
mente non è visibile con gli occhi fisici, deve essere vista con
``l'occhio della mente''. Prima che il Dhamma possa essere compreso,
dovete sapere dove guardare. Il Dhamma che è nel corpo deve essere visto
nel corpo. E con cosa guardiamo il corpo? Guardiamo il corpo con la
mente. Non troverete il Dhamma guardando in qualsiasi altro posto,
perché sia la felicità sia la sofferenza sorgono proprio lì. Avete visto
la felicità sorgere tra gli alberi? Oppure dai fiumi, o nelle condizioni
meteorologiche? Felicità e sofferenza sono sensazioni che sorgono nei
nostri stessi corpi e nelle nostre stesse menti.

Per questa ragione il Buddha ci dice di conoscere il Dhamma proprio qui.
Il Dhamma è proprio qui, dobbiamo guardare proprio qui. Un insegnante
può dirvi di guardare il Dhamma nei libri, ma se pensate che è lì che
davvero si trova il Dhamma, non lo vedrete mai. Dovete riflettere su
quegli insegnamenti interiormente, dopo aver guardato nei libri. Allora
potete comprendere il Dhamma. Dov'è che il vero Dhamma esiste? Esiste
proprio qui, in questo nostro corpo e in questa nostra mente. Questa è
l'essenza della pratica di contemplazione. Quando faremo così, nelle
nostre menti sorgerà la saggezza. Quando nelle nostre menti c'è
saggezza, ovunque guardiamo c'è il Dhamma, vedremo sempre \emph{aniccā},
\emph{dukkha} e \emph{anattā}. \emph{Aniccā} significa transitorietà. Se
ci attacchiamo alle cose, che sono transitorie, dobbiamo soffrire, sono
\emph{dukkha}, perché esse non sono noi o nostre, sono \emph{anattā}.
Questo però non lo vediamo, vediamo sempre le cose come se fossero noi
stessi e ci appartenessero.

Questo significa che non vediamo la verità della convenzione. Dovreste
capirle le convenzioni. Ad esempio, tutti noi qui seduti abbiamo dei
nomi. I nostri nomi sono nati con noi o ci sono stati assegnati in
seguito? Capite? È una convenzione. Le convenzioni sono utili? Certo che
lo sono. Supponiamo che ci siano quattro uomini, A, B, C e D. Devono
tutti avere dei nomi che li individuino per esigenze di comunicazione e
lavorative. Se volessimo parlare con il signor A, potremmo chiamare il
signor A e lui arriverebbe, non gli altri. Questa è l'utilità della
convenzione. Quando guarderemo in profondità, vedremo che in realtà lì
non c'è nessuno. Vedremo la trascendenza. C'è solo terra, acqua, fuoco e
vento, i quattro elementi. Questo è tutto quel che c'è in questo nostro
corpo. Però, a causa del potere dell'attaccamento, di
\emph{attavādupādāna},\footnote{\emph{Attavādupādāna}: L'attaccamento
  all'idea di un sé. È una delle quattro basi dell'attaccamento; si
  veda \emph{upādāna} nel \emph{Glossario}, p. \pageref{glossary-upadana}.} non vediamo le cose in
questo modo. Se guardassimo con chiarezza vedremmo che non c'è poi molto
in ciò che chiamiamo ``persona''. La parte solida è l'elemento terra, la
parte fluida è l'elemento acqua, la parte che fornisce il calore è
l'elemento fuoco. Se scomponiamo le cose vediamo che c'è solo terra,
acqua, fuoco e vento. Dov'è la persona? Non c'è.

Per questa ragione il Buddha insegnò che non c'è pratica più eccelsa del
vedere che ``questo non è il mio sé e non mi appartiene''. Si tratta di
semplici convenzioni. Se comprendiamo tutto con chiarezza in questi
termini, saremo in pace. Se nel momento presente comprendiamo la verità
dell'impermanenza, che le cose non sono il nostro sé né ci appartengono,
quando si disintegrano siamo in pace, perché non appartengono comunque a
nessuno. Sono semplicemente elementi fatti di terra, acqua, fuoco e
vento. Per la gente è difficile capirlo, anche se non si tratta di una
cosa che va al di là delle nostre capacità. Se riusciamo a capirlo,
troveremo appagamento, non avremo così tanta rabbia, tanta avidità e
tante illusioni. Nei nostri cuori ci sarà sempre il Dhamma. Non avremo
bisogno di essere gelosi e rancorosi, perché il nostro corpo è solo
terra, acqua, fuoco e vento. Non c'è nulla più di questo. Quando
accetteremo questa verità, vedremo la verità dell'insegnamento del
Buddha.

Se potessimo vedere la verità dell'insegnamento del Buddha, non avremmo
bisogno di tanti insegnanti! Non sarebbe necessario ascoltare gli
insegnamenti tutti i giorni. Quando comprendiamo, facciamo semplicemente
quello che ci viene richiesto. Quel che rende così difficile insegnare
alla gente, è che essa non accetta l'insegnamento e discute con gli
insegnanti e con l'insegnamento. Di fronte all'insegnante le persone si
comportano un po' meglio, ma alle sue spalle diventano ladri! È davvero
difficile insegnare alla gente. Le persone in Thailandia sono così, ecco
perché hanno bisogno di tanti insegnanti.

Siate attenti. Se non siete attenti non vedrete il Dhamma. Dovete essere
circospetti, prendere l'insegnamento e considerarlo per bene. È bello
questo fiore? Vedete la bruttezza dentro questo fiore? Per quanti giorni
sarà bello? A cosa somiglierà d'ora in poi? Perché cambia così? Fra tre
o quattro giorni dovrete prenderlo e buttarlo via, vero? Perderà tutta
la sua bellezza. La gente è attaccata alla bellezza, è attaccata alla
bontà. Se qualcosa è buono, ne sono completamente catturati. Il Buddha
ci dice di guardare le cose belle solo come belle; non dovremmo
attaccarci a esse. Se c'è una sensazione piacevole, non dovremmo farci
catturare. La bontà non è una cosa sicura, la bellezza non è una cosa
sicura. Niente è certo. Non c'è nulla in questo mondo che sia una
certezza. Questa è la verità. Le cose che non sono vere sono le cose che
cambiano, come la bellezza. L'unica verità è che cambia costantemente.
Se crediamo che le cose siano belle, quando la loro bellezza svanisce
anche la nostra mente perde la sua bellezza. Quando le cose non sono più
buone, anche la nostra mente perde la sua bontà. Quando si distruggono o
si danneggiano, soffriamo perché ci siamo attaccati a esse come se
fossero nostre. Il Buddha ci dice di vedere che queste cose sono mere
costruzioni della natura. La bellezza appare e dopo non molti giorni
svanisce. Capirlo significa avere saggezza.

È per questo che dovremmo vedere l'impermanenza. Se pensiamo che una
cosa sia bella, dovremmo dire a noi stessi che non lo è, se pensiamo che
una cosa sia brutta, dovremmo dire a noi stessi che non lo è. Cercate di
vedere le cose in questo modo, riflettete continuamente in questo modo.
Allora vedremo la Verità dentro le cose non vere, e vedremo la certezza
dentro le cose incerte.

Oggi vi ho spiegato come comprendere la sofferenza, quel che causa la
sofferenza, la cessazione della sofferenza e il Sentiero che conduce
alla cessazione della sofferenza. Quando conoscete la sofferenza,
dovreste gettarla via. Conoscendo la causa della sofferenza, dovreste
gettarla via. Praticate per vedere la cessazione della sofferenza. Se
vedete \emph{aniccā}, \emph{dukkha} e \emph{anattā}, la sofferenza
cesserà.

Quando la sofferenza cessa, dove andiamo? Per cosa stiamo praticando?
Stiamo praticando per abbandonare, non per ottenere qualcosa. Oggi
pomeriggio una donna mi ha detto che sta soffrendo. Le ho chiesto che
cosa vorrebbe essere, e mi ha risposto che vuole essere un'illuminata.
«~Finché vuoi essere un'illuminata non lo diventerai mai. Non volere
nulla~», le ho detto.

Se conosciamo la verità della sofferenza, gettiamo via la sofferenza.
Quando conosciamo la causa della sofferenza, allora non creiamo più
quelle cause e pratichiamo per condurre la sofferenza alla cessazione.
La pratica che conduce alla cessazione della sofferenza consiste nel
vedere che ``questo non è un sé'', ``questo non sono io o loro''. Vedere
in questo modo consente alla sofferenza di cessare. È come raggiungere
la nostra destinazione e fermarci. È la cessazione. È andare vicini al
\emph{Nibbāna}. Per metterla in altro modo, andare avanti è sofferenza,
tornare indietro è sofferenza e fermarsi è sofferenza. Non andare
avanti, non tornare indietro e non fermarsi: resta qualcosa? Qui il
corpo e la mente cessano. Questa è la cessazione della sofferenza.
Difficile da capire, vero? Se ci applichiamo con diligenza e costanza a
questo insegnamento, trascenderemo le cose e raggiungeremo la
comprensione. Ci sarà la cessazione. Questo è l'insegnamento ultimo del
Buddha, il punto di arrivo. L'insegnamento del Buddha termina nel punto
del totale abbandono.

Oggi questo insegnamento lo offro a voi e anche al vostro venerabile
maestro. Se vi è qualcosa di sbagliato in esso, vi chiedo di perdonarmi.
Però, non siate frettolosi nel giudicare se è giusto o sbagliato,
innanzitutto limitatevi ad ascoltarlo. Se stessi per dare a tutti voi un
frutto e vi dicessi che è delizioso, prendereste atto delle mie parole
ma non mi credereste immediatamente, perché non lo avete ancora
assaggiato. Per l'insegnamento di oggi è la stessa cosa. Se volete
sapere se il ``frutto'' è dolce o aspro, dovete tagliarne un pezzetto e
assaggiarlo. Allora conoscerete la sua dolcezza o asprezza. Allora
potreste credermi, perché siete stati voi stessi a provarlo. Perciò, per
favore, non buttate via questo ``frutto'', conservatelo e assaggiatelo,
conoscete da voi stessi il suo sapore.

Il Buddha non ebbe un maestro, lo sapete. Una volta un asceta gli chiese
chi fosse il suo maestro, e il Buddha rispose che non ne
aveva.\footnote{\emph{Vinaya, Mahāvagga} 1.6.} L'asceta se ne andò
scuotendo la testa. Il Buddha era stato troppo sincero. Aveva parlato a
uno che non poteva conoscere o accettare la verità. Per questo vi dico
di non credermi. Il Buddha disse che limitarsi a credere agli altri è
stolto, perché in ciò non vi è chiara conoscenza. Ecco perché il Buddha
disse: «~Non ho maestro.~» Questa è la verità. Dovreste però guardarla
nel modo giusto. Se la fraintendete, mancherete di rispetto al vostro
insegnante. Non andate in giro dicendo: «~Non ho maestro.~» Dovete
fidarvi del vostro insegnante, quando vi dice ciò che è giusto e ciò che
è sbagliato, e poi praticare di conseguenza.

Oggi per noi è un giorno fortunato. Ho avuto l'opportunità di incontrare
tutti voi e il vostro venerabile maestro. Viviamo così lontani: non
dovreste pensare che ci si sia potuti incontrare così. Credo che ci
debba essere una qualche ragione speciale che ci abbia consentito di
incontrarci in questo modo. Il Buddha insegnò che tutto ciò che sorge
deve avere una causa. Non dimenticatelo. Ci deve essere una qualche
causa. Forse in un'esistenza precedente eravamo fratelli e sorelle in
una stessa famiglia. È possibile. Un altro insegnante non è arrivato, ma
io sì. Perché? Forse stiamo creando delle cause in questo stesso
momento. Anche questo è possibile.

Vi lascio tutti con questo insegnamento. Vi auguro di essere diligenti e
infaticabili nella pratica. Non c'è niente di meglio della pratica del
Dhamma. Il Dhamma sostiene tutto il mondo. Oggigiorno la gente è confusa
perché non conosce il Dhamma. Se abbiamo il Dhamma con noi, saremo
appagati. Sono felice di aver avuto questa opportunità di aiutare voi e
il vostro venerabile insegnante a sviluppare la pratica del Dhamma. Vi
lascio con un augurio che nasce dal profondo del cuore. Domani partirò,
non so esattamente per dove. È naturale. Quando c'è un arrivo ci deve
essere una partenza, quando c'è una partenza ci deve essere un arrivo.
Così è il mondo. I cambiamenti nel mondo non dovrebbero farci gioire
troppo o turbarci. C'è la felicità e poi la sofferenza; c'è la
sofferenza e poi la felicità; c'è il guadagno e poi la perdita; c'è la
perdita e poi il guadagno. È così che vanno le cose.

Ai suoi tempi, il Buddha stesso non piaceva ad alcuni suoi discepoli,
perché li esortava a essere diligenti, a essere attenti. Quelli che
erano pigri lo temevano e provavano risentimento nei suoi riguardi.
Quando Egli morì, alcuni discepoli piansero e si afflissero perché non
ci sarebbe più stato il Buddha a guidarli. Non si trattava comunque di
discepoli intelligenti. Altri erano compiaciuti e sollevati perché non
avrebbero più portato sul groppone il Buddha che diceva loro cosa fare.
Altri ancora erano equanimi. Riflettevano sul fatto che quanto sorge
deve, per naturale conseguenza, svanire. Tre erano i gruppi di
discepoli. Con quale gruppo vi identificate? Volete appartenere a quello
dei compiaciuti, o a quale altro? I discepoli che facevano parte del
gruppo di chi pianse quando il Buddha morì non avevano ancora compreso
il Dhamma. Il secondo gruppo era composto da chi era risentito con il
Buddha. Egli proibiva sempre a costoro di fare quel che volevano.
Vivevano nella paura del suo sdegno e dei suoi rimproveri, e così quando
morì si sentirono sollevati.

Oggigiorno le cose non sono molto diverse. È possibile che l'insegnante
che sta qui abbia dei discepoli che nutrono risentimento nei suoi
riguardi. Potrebbero anche non mostrarlo, ma è nelle loro menti. È
normale per chi ha ancora delle contaminazioni provare questo tipo di
sentimenti. Anche il Buddha aveva persone che lo odiavano. Io stesso ho
discepoli che sono risentiti con me. Io dico loro di rinunciare alle
cattive azioni, ma loro se le tengono care. Perciò mi odiano. Un sacco
di gente è così. Tutti coloro che fra voi sono intelligenti possano
diventare stabili nella pratica del Dhamma.

