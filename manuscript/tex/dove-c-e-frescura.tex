\chapter{Dove c'è frescura}

\begin{openingQuote}
  Discorso tenuto per l'assemblea dei monaci e dei novizi al Wat Pah Nanachat
  durante il Ritiro delle Piogge del 1978.
\end{openingQuote}

La pratica del Dhamma va contro le nostre abitudini. La Verità va contro
i nostri desideri. È per questo che praticare è difficile. Alcune cose
che riteniamo sbagliate possono essere giuste, mentre cose che riteniamo
giuste possono essere sbagliate. Perché? Le nostre menti sono
nell'oscurità e, perciò, non vediamo la Verità con chiarezza. Non
conosciamo proprio nulla e per questa ragione le menzogne della gente ci
ingannano. Indicano ciò che è giusto come sbagliato, e noi ci crediamo,
quel che è sbagliato dicono che è giusto, e noi ci crediamo. Questo
succede perché non siamo ancora gli insegnanti di noi stessi. I nostri
stati mentali ci mentono in continuazione. Questa mente con le sue
opinioni non dovrebbe essere la nostra guida, perché non conosce la
Verità.

Alcuni gli altri non vogliono assolutamente ascoltarli, ma questo non è
il modo d'essere di un saggio. Un saggio ascolta tutti. Chi ascolta il
Dhamma deve ascoltare allo stesso modo quello che gli piace e quello che
non gli piace, ed evitare sia di credere ciecamente sia di non credere.
Deve restare a mezza strada, sul punto medio, non essere avventato. Egli
ascolta solo e poi contempla, facendo sorgere di conseguenza i giusti
effetti. Il saggio dovrebbe contemplare e comprendere da sé causa ed
effetto prima di credere a ciò che ascolta. Anche se un insegnante dice
la verità, non credetegli semplicemente, perché ancora non conoscete da
voi stessi la verità di quel che dice. È lo stesso per tutti noi, me
incluso. Ho praticato più a lungo di voi, ho già visto molte menzogne.
Ad esempio: «~Questa pratica è davvero difficile, è dura davvero.~»
Perché è difficile la pratica? Lo è solo perché pensiamo in modo
sbagliato, abbiamo errata visione.

In precedenza ho vissuto con altri monaci, ma non mi sentivo a mio agio.
Sono scappato nella foresta e sulle montagne per evitare la folla, i
monaci e i novizi. Pensavo che non fossero come me, che non praticassero
così duramente come io facevo, pensavo che fossero sciatti. Quella
persona era così, quell'altra cosà. Era una cosa che davvero mi metteva
in agitazione, era la ragione delle mie continue fughe. Però, che
vivessi da solo o con gli altri, non avevo pace. Da solo non ero
contento, in compagnia non ero contento. Pensavo che questa scontentezza
fosse dovuta ai miei compagni, ai miei umori, al posto in cui vivevo, al
cibo, al tempo, pensavo che fosse dovuta ora a questo e ora a quello.
Ero alla continua ricerca di qualcosa che si confacesse alla mia mente.

In quanto monaco \emph{dhutaṅga} girovagavo,\footnote{Ossia un monaco
  errante che si addestra mediante le tredici pratiche ascetiche
  volontarie che vanno sotto il nome di \emph{dhutaṅga}, delle quali si
  parla più dettagliatamente nel \emph{Glossario}, p. \pageref{glossary-dhutanga}.} ma le cose non
andavano bene. Perciò contemplavo: «~Che cosa posso fare per sistemare
le cose? Che cosa posso fare?~» Ero insoddisfatto se vivevo con molta
gente, ero insoddisfatto se vivevo con poche persone. Qual era il
motivo? Non riuscivo a capirlo. Perché ero insoddisfatto? Perché avevo
errata visione, solo per questo, perché ero ancora attaccato al
\emph{dhamma} sbagliato. Ovunque andassi ero scontento e pensavo: «~Qui
non va bene, non c'è nulla di buono.~» E continuavo in questo modo. Davo
la colpa agli altri. Davo la colpa al tempo, al caldo e al freddo,
biasimavo ogni cosa! Come un cane matto, un cane che morde tutto ciò che
incontra, proprio perché è matto. Quando la mente è così, non si assesta
mai. Oggi ci sentiamo bene, domani no. È sempre così. Non riusciamo a
essere appagati e in pace.

Una volta il Buddha vide uno sciacallo, un cane selvatico che era
fuggito dalla foresta nella quale viveva. Stava fermo per un po', poi
correva nei cespugli e di nuovo ne usciva. Quindi correva nella cavità
di un albero, e poi via di nuovo. Stava fermo per un minuto, e il minuto
successivo correva, si sdraiava e poi saltava su. Quello sciacallo aveva
la rogna. Quando stava fermo la rogna gli mangiava la pelle, e così
correva. Mentre correva si sentiva ancora a disagio, e perciò si
fermava. Era a disagio quando stava sulle zampe, e così si metteva a
giacere. Poi saltava su di nuovo, correva tra i cespugli, nella cavità
di un albero, non stava mai fermo. Il Buddha disse: «~Monaci, questo
pomeriggio avete visto quello sciacallo? Soffriva quando stava sulle
zampe, soffriva quando correva, soffriva quando giaceva. Soffriva tra i
cespugli, nella cavità di un albero o in una tana. Per il suo disagio
incolpava lo stare sulle zampe o seduto, incolpava il correre o il
giacere; incolpava l'albero, i cespugli e la tana. Nei fatti il problema
non stava in nessuna di quelle cose. Lo sciacallo aveva la rogna. Il
problema era la rogna.~»

Noi monaci siamo proprio come quello sciacallo. La nostra
insoddisfazione è dovuta all'errata visione. Siccome non esercitiamo il
contenimento dei sensi, attribuiamo all'esterno la colpa della nostra
sofferenza. Non siamo soddisfatti sia che viviamo al Wat Pah Pong, in
America oppure a Londra. Se andiamo a vivere al Bung Wai o in qualsiasi
altro monastero affiliato non siamo soddisfatti. Perché? Dentro di noi
c'è ancora errata visione, ecco perché. Ovunque andiamo non siamo
contenti. Così, per noi è proprio come per quello sciacallo, che se
guarisce dalla rogna è contento ovunque vada. Rifletto spesso su questo
e ve l'insegno spesso, perché è una cosa davvero importante. Se
conosciamo la verità dei nostri stati mentali giungiamo
all'appagamento. Siamo soddisfatti sia quando fa caldo sia quando fa
freddo, siamo soddisfatti sia quando viviamo con molta gente sia con
poca. L'appagamento non dipende dal numero di persone con cui stiamo,
proviene solo dalla Retta Visione. Se abbiamo Retta Visione, allora
siamo contenti ovunque ci troviamo.

La maggior parte di noi, però, ha errata visione. Proprio come succede
ai vermi. Il posto in cui un verme vive è sudicio e il suo cibo è
sudicio, ma si addicono al verme. Se prendete un bastoncino e cercate di
allontanarlo dal suo grumo di sterco, lotterà strisciando per tornarvi.
Avviene la stessa cosa quando l'\emph{ajahn} ci insegna a vedere
rettamente. Ci fa sentire a disagio e opponiamo resistenza. Torniamo
correndo al nostro ``grumo di sterco'' perché è lì che ci sentiamo a
casa. Siamo proprio così. Se non comprendiamo le conseguenze dannose di
tutti i nostri modi di pensare errati, non possiamo abbandonarli. La
pratica è difficile. È per questo che dovremmo ascoltare. La pratica non
è nient'altro. Se abbiamo Retta Visione, siamo contenti ovunque andiamo.
Lo ho già praticato e compreso. Questi giorni ci sono molti monaci,
novizi e laici che vengono a trovarmi. Se ancora non sapessi, se avessi
ancora errata visione, ormai sarei morto! Per i monaci il luogo giusto
nel quale dimorare, il luogo in cui vi è frescura, è solo la Retta
Visione stessa. Non dovremmo cercare nient'altro.

Perciò, sebbene vi possa capitare di essere infelici, non importa:
quella infelicità è incerta. Quell'infelicità è il vostro ``sé''? C'è
qualcosa di sostanziale in essa? Io non la ritengo affatto reale.
L'infelicità è solo il balenare di una sensazione che appare e poi se ne
va. La felicità è la stessa cosa. Ha una qualche consistenza? È
veramente un'entità? Si tratta semplicemente del balenare di una
sensazione che all'improvviso se ne va. Ecco! Nasce e muore. L'amore
compare per un momento e poi svanisce. Dov'è la consistenza dell'amore,
dell'odio o del risentimento? In verità non c'è alcuna entità
sostanziale lì, sono solo impressioni mentali che infiammano la mente e
poi muoiono. Ci ingannano in continuazione, la certezza non esiste da
nessuna parte. È proprio come disse il Buddha: quando l'infelicità
sorge, resta per un po' e poi svanisce. Quando l'infelicità svanisce, la
felicità sorge, indugia per un po' e poi muore. Quando la felicità
svanisce, sorge di nuovo l'infelicità, e così via, sempre in questo
modo.

Alla fine possiamo dire solo questo: non c'è nulla oltre alla nascita,
alla vita e alla morte della sofferenza. È tutto qui. Però, noi che
siamo ignoranti afferriamo in continuazione tutto questo. Non vediamo
mai la Verità, non vediamo che c'è solo questo continuo cambiamento. Se
lo comprendiamo, allora non abbiamo bisogno di pensare tanto, ma abbiamo
molta saggezza. Se non lo sappiamo, allora avremo più pensieri che
saggezza, e forse di saggezza non ne avremo affatto! Solo quando vediamo
veramente i risultati dannosi delle nostre azioni possiamo rinunciare a
esse. Allo stesso modo, è solo quando vediamo gli effetti benefici della
pratica che possiamo iniziare ad attenerci a essa e a lavorare per
rendere ``buona'' la mente.

Se tagliamo un tronco di legno e lo gettiamo nel fiume ed esso non
affonda né marcisce, e neanche si arena su una delle due sponde, quel
tronco raggiungerà certamente il mare. Possiamo paragonarvi la nostra
pratica. Se praticate seguendo il Sentiero tracciato dal Buddha, se lo
seguite rettamente, trascenderete due cose. Quali? Proprio quei due
estremi che il Buddha affermò non essere il Sentiero di un vero
meditante: indulgere al piacere e indulgere al dolore. Sono le due
sponde del fiume. Una delle due sponde di quel fiume è l'odio, l'altra è
l'amore. Oppure, potete dire che una è la felicità, l'altra
l'infelicità. Il ``tronco'' è la mente. Mentre ``galleggia giù per il
fiume'' sperimenta felicità e infelicità. Se la mente non si attacca a
quella felicità o a quella infelicità raggiungerà l'``oceano'' del
\emph{Nibbāna}. Dovreste capire che non c'è nient'altro, ci sono solo
felicità e infelicità che sorgono e svaniscono. Se non vi ``arenate'' su
queste cose, allora siete sul Sentiero del vero meditante.

Questo è l'insegnamento del Buddha. Felicità, infelicità, amore e odio
sono semplicemente presenti nella natura in accordo con la sua costante
legge. Il saggio non li segue né li incoraggia, non vi si attacca.
Questa è la mente che lascia andare l'indulgenza al piacere e
l'indulgenza al dolore. È la retta pratica. Proprio come quel tronco che
col tempo galleggia fino al mare, così la mente che non si attacca a
questi due estremi raggiungerà inevitabilmente la pace.

