\chapter{Comprendere il Vinaya}

Questa nostra pratica non è facile. Possiamo sapere alcune cose, ma c'è
ancora molto che non sappiamo. Quando ad esempio ascoltiamo insegnamenti
come ``conosci il corpo, poi conosci il corpo nel corpo'', oppure
``conosci la mente, poi conosci la mente nella mente''. Se queste cose
non le abbiamo già praticate, quando le ascoltiamo possiamo sentirci
sconcertati. Il Vinaya\footnote{\emph{Vinaya}: Il codice della disciplina
  monastica buddhista.} è così. In passato ero un insegnante,\footnote{In
  riferimento ai primi anni da monaco del venerabile Ajahn Chah, prima
  che egli cominciasse a praticare intensamente.} ma solo un ``piccolo
insegnante''. Perché dico un ``piccolo insegnante''? Perché non
praticavo. Insegnavo il Vinaya, ma non lo praticavo. È questo che io
chiamo un piccolo insegnante, un insegnante scadente. Dico un
``insegnante scadente'' perché nella pratica ero manchevole. La mia
pratica era per lo più molto lontana dalla teoria, come se non avessi
imparato affatto il Vinaya.

Ci terrei a precisare che è praticamente impossibile conoscere il Vinaya
completamente. E alcune cose, che le si conosca o no, sono pur sempre
trasgressioni. Tuttavia si sottolinea che se non abbiamo ancora capito
una qualche regola particolare dell'addestramento o dell'insegnamento,
dobbiamo studiare quella regola con entusiasmo e rispetto. Se non
sappiamo, dovremmo fare uno sforzo per imparare. Se non ci sforziamo,
questa è di per sé una trasgressione. Ad esempio quando dubitate.
Supponiamo che ci sia una donna e che, senza sapere se si tratta di una
donna o di un uomo, la tocchiate.\footnote{La seconda trasgressione
  \emph{sanghādisesa} riguarda il toccare una donna con intenzioni
  lascive.} Non ne siete sicuri, ma comunque andate avanti, e toccate.
Già questo è sbagliato. Ero solito chiedermi perché fosse sbagliato, ma
quando presi in considerazione la pratica compresi che un meditante deve
avere \emph{sati}, deve essere circospetto. Che si tratti di parlare, di
toccare o di avere delle cose, egli deve prima valutare a fondo. In
questo caso l'errore è che non c'è \emph{sati} o che \emph{sati} è
insufficiente, oppure sta in una mancanza di preoccupazione al riguardo
in quel momento.

Facciamo un altro esempio. Sono solo le undici del mattino, ma in quel
momento il cielo è nuvoloso, non possiamo vedere il sole e non abbiamo
l'orologio. Supponiamo che sia pomeriggio -- pensiamo davvero che sia
pomeriggio -- e tuttavia andiamo avanti, e mangiamo qualcosa. Iniziamo a
mangiare, le nuvole si fanno da parte e dalla posizione del sole vediamo
che sono appena passate le undici. Si tratta comunque di una
trasgressione.\footnote{In riferimento a \emph{pācittiya}, trasgressione
  n. 36, mangiare al di fuori del tempo consentito, che è dall'alba a
  mezzogiorno.} Di solito mi chiedevo: «~Eh? Non è ancora passato
mezzogiorno, perché è una trasgressione?~» Si commette una trasgressione
per negligenza, per trascuratezza. C'è una mancanza di contenimento. Se
c'è un dubbio e noi agiamo quando siamo in dubbio, è una trasgressione
\emph{dukkata}\footnote{\emph{Dukkata}: Trasgressione di ``azione
  sbagliata'', il genere di trasgressione meno grave nel Vinaya, ove ne
  sono elencate un gran numero.} solo per il fatto di aver agito in
presenza del dubbio. Pensiamo che sia pomeriggio, mentre nei fatti non è
così. L'azione di mangiare non è sbagliata in sé, ma qui c'è una
trasgressione perché siamo distratti e negligenti. Se è davvero
pomeriggio e pensiamo che lo sia, allora si tratta di una trasgressione
più grave, \emph{pācittiya}. Se agiamo quando siamo in dubbio, che
l'azione sia sbagliata o meno, incorriamo comunque in una trasgressione.
Se l'azione non è sbagliata in sé, la trasgressione è più lieve, se è
sbagliata, si incorre allora in una trasgressione più grave. Per questo
il Vinaya può risultare piuttosto sconcertante.

Una volta sono andato a far visita al venerabile Ajahn Mun. Allora avevo
appena cominciato a praticare. Avevo letto il
\emph{Pubbasikkhā}\footnote{\emph{Pubbasikkhā Viṇṇanā} (\emph{Elementi
  di addestramento}): Un commentario thailandese al Dhamma-Vinaya basato
  sui Commentari in pāli.} ed ero in grado di comprenderlo piuttosto
bene. Andai avanti a leggere il \emph{Visuddhimagga},\footnote{\emph{Visuddhimagga}
  (\emph{Il sentiero della purezza}): L'ampio commento di Buddhaghosa al
  Dhamma-Vinaya.} e i tre libri che lo compongono: il \emph{Sīlaniddesa}
(Libro sulla Moralità), il \emph{Samādhiniddesa} (Libro sulla
Concentrazione) e il \emph{Paññāniddesa} (Libro sulla Saggezza). Pensavo
che la testa mi stesse per esplodere! Dopo aver letto, ebbi la
sensazione che praticare fosse al di là delle possibilità di un essere
umano. Però, ragionai sul fatto che il Buddha non avrebbe insegnato una
cosa impossibile da praticare. Non l'avrebbe insegnata né l'avrebbe
proclamata, perché sarebbero state cose inutili per lui e per gli altri.
Il \emph{Sīlaniddesa} è estremamente meticoloso, il
\emph{Samādhiniddesa} lo è di più, e il \emph{Paññāniddesa} di più
ancora! Mi sedetti e pensai: «~Bene, non posso andare più avanti. Non
c'è una strada da percorrere.~» Era come se fossi finito in un vicolo
cieco.

In questa fase stavo combattendo con la mia pratica, ero bloccato.
Avvenne che ebbi l'opportunità di andare a far visita al venerabile
Ajahn Mun, e così gli chiesi: «~Venerabile \emph{ajahn},\footnote{\emph{ajahn}
  (in thailandese, \href{http://www.thai2english.com/dictionary/1453955.html}{\thai{อาจารย์}}).
  Il termine deriva da \emph{ācariya}, in pāli, letteralmente
  ``insegnante''; spesso viene utilizzato per un monaco o per una monaca
  con più di dieci anni di vita monastica.} che cosa devo fare? Ho
appena iniziato a praticare, ma ancora non conosco la retta via. Ho
moltissimi dubbi e non riesco a trovare un fondamento per la pratica.~»
Mi domandò: «~Qual è il problema?~» «~Durante la mia pratica ho preso il
\emph{Visuddhimagga} e l'ho letto, ma mi pare impossibile metterlo in
pratica. I contenuti del \emph{Sīlaniddesa}, del \emph{Samādhiniddesa} e
del \emph{Paññāniddesa} paiono essere del tutto impraticabili. Non penso
che ci sia qualcuno al mondo che possa farlo, tanto è dettagliato e
meticoloso. Memorizzare ogni regola mi sarebbe impossibile, è al di là
delle mie forze.~» Mi rispose così: «~Venerabile, ci sono molte cose, è
vero, ma in realtà è poco. Sarebbe difficile se dovessimo tener conto di
ogni regola per l'addestramento presente nel \emph{Sīlaniddesa}, è vero.
In realtà, quel che chiamiamo \emph{Sīlaniddesa} si è sviluppato dalla
mente umana. Se addestriamo la mente ad avere un senso di vergogna e il
timore per la trasgressione, allora saremo contenuti, saremo cauti \ldots{}
Questo ci indurrà ad accontentarci di poco, ad avere pochi desideri,
perché è impossibile badare a troppe cose. Quando ciò avverrà, la nostra
\emph{sati} diventerà più forte. Saremo sempre in grado di sostenere
\emph{sati}. Ovunque ci troveremo faremo uno sforzo meticoloso per
mantenere \emph{sati}. Si svilupperà la cautela. Di qualsiasi cosa tu
possa dubitare, non parlare e non agire quando sei in dubbio. Se c'è
qualcosa che non comprendi, chiedi all'insegnante. Cercare di praticare
ogni regola dell'addestramento sarebbe certamente gravoso, ma dovremmo
esaminare se siamo pronti o no ad ammettere i nostri errori. Li
accettiamo?~»

È un insegnamento molto importante. Se sappiamo come addestrare la
nostra mente, non è necessario conoscere ogni singola regola. «~Tutta
questa roba che hai letto sorge dalla mente. Se non hai ancora
addestrato la mente a essere sensibile e chiara, dubiterai sempre.
Dovresti cercare di portare gli insegnamenti del Buddha dentro la mente.
Che la tua mente sia composta. Qualsiasi dubbio sorga, rinunciaci e
basta. Se non sai con certezza, allora non dire o non fare. Ad esempio,
se ti chiedi: ``È giusto o no?'' -- se non sei cioè davvero sicuro --
allora non dirlo, non farlo, non mettere da parte il tuo contenimento.~»

Mentre stavo lì seduto e ascoltavo, pensai che questo insegnamento era
conforme agli otto modi di misurare il vero insegnamento del Buddha.
Ogni insegnamento che diminuisca le contaminazioni, che conduca fuori
dalla sofferenza, che parli di rinuncia (ai piaceri dei sensi) e di
accontentarsi di poco, di umiltà e di disinteresse per il rango e lo
status sociale, di distacco e solitudine, di sforzo diligente, e che
conduca a sentirsi sollevati. Queste otto qualità caratterizzano il vero
Dhamma-Vinaya, l'insegnamento del Buddha. Non lo è tutto quello che è in
contraddizione con esse. Se siamo genuinamente sinceri avremo un senso
di vergogna e timore per la trasgressione. Se sappiamo che nella nostra
mente c'è il dubbio, non agiremo né parleremo. Il \emph{Sīlaniddesa} è
fatto di sole parole. Ad esempio, \emph{hiri-ottappa}\footnote{\emph{Hiri-ottappa}:
  Coscienza morale e timore di compiere cattive azioni.} nei libri è una
cosa, ma nella nostra mente è un'altra cosa. Studiando il Vinaya con il
venerabile Ajahn Mun imparai molte cose. Mentre sedevo e ascoltavo,
sorse la conoscenza.

Il Vinaya l'ho studiato in modo considerevole. Durante il Ritiro delle
Piogge alcuni giorni studiavo dalle sei del pomeriggio fino all'alba. Lo
comprendevo a sufficienza. Annotai in un taccuino che tenevo nella mia
borsa tutti i fattori dell'\emph{āpatti}\footnote{\emph{Āpatti}: I vari
  generi di trasgressione di un monaco buddhista o di una monaca.}
elencati nel \emph{Pubbasikkhā}. Ci misi impegno davvero, ma in seguito,
gradualmente, lasciai andare. Era troppo. Non sapendo cosa fosse
essenziale e cosa no, prendevo tutto. Quando compresi più pienamente
lasciai cadere ogni cosa perché era troppo pesante. Rivolsi la mia
attenzione all'interno della mia mente e gradualmente mi staccai dai
testi.

Quando insegno ai monaci, ovviamente la base è il \emph{Pubbasikkhā}.
Per molti anni qui al Wat Pah Pong fui io stesso a leggerlo
all'assemblea dei monaci. In quei giorni salivo sulla sedia del Dhamma
e andavo avanti almeno fino alle undici o mezzanotte, qualche volta
anche fino all'una o alle due del mattino. Erano interessati. E noi ci
addestravamo. Dopo aver ascoltato la lettura del Vinaya avremmo preso in
considerazione quel che avevamo ascoltato. Non si può comprendere il
Vinaya solo ascoltandolo. Dopo aver ascoltato devi esaminarlo e scavarci
dentro ulteriormente.

Benché abbia studiato queste cose per molti anni la mia conoscenza non è
ancora completa, perché nei testi moltissime sono le ambiguità. Ora che
è passato così tanto tempo da quando studiavo sui libri, il mio ricordo
delle varie regole per l'addestramento si è un po' sbiadito, ma nella
mia mente non vedo mancanze. C'è un criterio, qui dentro. Non ci sono
dubbi, c'è comprensione. Ho messo da parte i libri e mi sono concentrato
a sviluppare la mente. Non ho dubbi a proposito di nessuna regola. La
mente ha stima della virtù, non osa fare nulla di sbagliato, sia in
pubblico sia in privato. Non uccido animali, nemmeno quelli piccoli. Se
qualcuno mi chiedesse di uccidere intenzionalmente una formica o una
termite, di schiacciarla con la mano, ad esempio, non potrei farlo
neanche se mi offrissero migliaia di baht.\footnote{Il bhat è la moneta
  thailandese.} Nemmeno una formica o una termite! La vita di una
formica per me avrebbe più valore.

Può certamente succedere che io sia causa della morte di una formica, ad
esempio quando avviene che un qualcosa mi si arrampica sulla gamba e io
lo scanso via. Forse muore, ma quando guardo nella mia mente non c'è
senso di colpa. Non c'è incertezza o dubbio. Perché? Perché non c'era
alcuna intenzione. L'intenzione è l'essenza dell'addestramento morale:
\emph{cetanāham bhikkhave sīlam vadāmi}. Guardando la cosa in questo
modo, vedo che non ho ucciso intenzionalmente. A volte, mentre cammino,
può capitare che io metta il piede su un insetto e lo uccida. In
passato, prima che capissi veramente, soffrivo davvero per cose come
questa. Pensavo di aver commesso una trasgressione. «~E allora? Non
c'era alcuna intenzione.~» «~Non c'era alcuna intenzione, ma non ho
prestato sufficiente attenzione!~» E andavo avanti così, affliggendomi e
preoccupandomi. Per questa ragione il Vinaya è una cosa che può turbare
i praticanti di Dhamma ma, in linea con quello che dicono gli
insegnanti, che ha pure il suo valore: «~Qualsiasi regola
dell'addestramento che non conoscete ancora, dovreste impararla. Se non
sapete, dovreste chiedere a coloro che sanno.~» Lo sottolineano proprio.

Se non conosciamo le regole dell'addestramento, quando trasgrediamo non
ne siamo consapevoli. Prendiamo come esempio un venerabile
\emph{Thera}\footnote{\emph{Thera}: Letteralmente ``anziano''; chi è
  monaco da almeno dieci anni.} del passato, Ajahn Pow del Wat Kow Wong
Got nella provincia di Lopburi. Un giorno, un certo
\emph{Mahā},\footnote{\emph{Mahā}: Titolo acquisito da un \emph{bhikkhu}
  dopo aver sostenuto determinati esami in lingua pāli.} uno dei suoi
discepoli, stava seduto con lui quando alcune donne arrivarono e
chiesero: «~Luang Por!\footnote{Luang Por (in thailandese \thai{หลวงพ่อ}):
  ``Venerabile padre''; è un'espressione che i thailandesi utilizzano
  per rivolgersi ai monaci anziani.} Desideriamo invitarti a venire con
noi ad una escursione. Verrai?~» Luang Por Pow non rispose. Il
\emph{Mahā} che stava seduto vicino a lui pensò che il venerabile Ajahn
Pow non avesse sentito, e così gli disse: «~Luang Por, Luang Por! Non
hai sentito! Queste donne ti hanno invitato a una gita.~» «~Ho
sentito~», rispose. Le donne chiesero nuovamente: «~Luang Por, verrai o
no?~» Restò seduto lì senza dire nulla, e così l'invito rimase senza
risposta. Quando se ne furono andate, il \emph{Mahā} disse: «~Luang Por,
perché non hai risposto a quelle donne?~» Rispose: «~Oh, non conosci
questa regola?~Erano tutte donne. Se delle donne t'invitano a viaggiare
con loro non dovresti acconsentire. Se loro organizzano tutto per conto
loro, va bene. Se voglio posso andare, perché non ho preso parte ai
preparativi.~» Il \emph{Mahā} rimase seduto e pensò: «~Ho fatto proprio
una figuraccia.~» Il Vinaya afferma che organizzare e poi viaggiare
insieme con delle donne, anche se non si è in coppia, è una
trasgressione \emph{pācittiya}.

Facciamo un altro esempio. Dei laici portarono del denaro su un vassoio
per offrirlo al venerabile Ajahn Pow. Egli distese il suo panno per
ricevere le offerte,\footnote{Un ``panno per ricevere le offerte'' è
  utilizzato dai monaci thailandesi per ricevere cose dalle donne, dalle
  quali non possono riceverle direttamente. Il venerabile Ajahn Pow
  allontanò le mani dal panno per ricevere le offerte per indicare che
  lui in realtà non stava ricevendo il denaro.} tenendolo per un capo.
Però quando ebbe davanti il vassoio, ritrasse le mani dal panno. Poi,
semplicemente, lasciò il denaro là dov'era. Sapeva che era là, ma non
gli interessava. Si alzò e se ne andò via perché nel Vinaya si dice che
se non si acconsente a ricevere del denaro, non è necessario proibire ai
laici di offrirlo. Se lo avesse desiderato, avrebbe dovuto dire:
«~Capofamiglia, questo non è consentito a un monaco.~» Avrebbe dovuto
dirlo. Se lo desideri, devi proibire di offrire ciò che non è
consentito. Ovviamente, se davvero non lo desideri, non è necessario.
Basta lasciarlo là e andarsene. Benché l'\emph{ajahn} e i suoi discepoli
avessero vissuto assieme per molti anni, alcuni di loro ancora non
comprendevano la pratica di Ajahn Pow. Si tratta di una situazione
pessima. Per quanto mi concerne, ho osservato e contemplato molti dei
punti più sottili della pratica di Ajahn Pow.

Il Vinaya può perfino indurre alcuni a lasciare l'abito monastico.
Quando lo studiano sorgono i dubbi. Questo mi riporta indietro nel
tempo~\ldots{} «~La mia ordinazione monastica, era stata condotta nel modo
giusto?\footnote{Le regole che governano l'ordinazione monastica sono
  molto precise e dettagliate, e la mancata aderenza a esse può
  invalidarla.} Il mio precettore era puro? Nessuno dei monaci presenti
alla mia ordinazione sapeva qualcosa del Vinaya. Erano seduti alla
giusta distanza? I canti erano corretti?~» I dubbi continuavano ad
arrivare. «~La sala nella quale ho ricevuto l'ordinazione era adatta?
Era così piccola \ldots{}~» Si dubita di tutto, e così si va a finire
nell'inferno. Perciò, fino a quando non si sa come dare un fondamento
alla propria mente è davvero difficile. Bisogna essere davvero composti,
non ci si può semplicemente gettare nelle cose. Però, essere composti
fino al punto di non preoccuparsi di guardare nelle cose è ugualmente
sbagliato. Poiché vedevo molte mancanze nella mia pratica e in quella di
alcuni dei miei insegnanti ero così confuso che quasi lasciai l'abito
monastico. Ero in fiamme e, a causa di questi dubbi, non riuscivo a
dormire.

Più dubitavo, più meditavo, più praticavo. Tutte le volte che sorgeva un
dubbio, praticavo proprio a quel proposito. Sorse la saggezza. Le cose
cominciarono a cambiare. È difficile descrivere il cambiamento che ebbe
luogo. La mente cambiò, finché non ebbi più dubbi. Non so come cambiò.
Se dovessi dirlo a qualcuno, probabilmente non capirebbe. Perciò
riflettei sull'insegnamento \emph{Paccatam veditabbo viññūhi}: «~il
saggio deve conoscere da sé~». Deve trattarsi di una conoscenza che
sorge attraverso l'esperienza diretta. Studiare il Dhamma-Vinaya è
certamente giusto, ma lo studio da solo è insufficiente. Se ti cali
davvero nella pratica, inizi a dubitare di tutto. Prima d'iniziare a
praticare non ero interessato alle trasgressioni minori, ma quando
cominciai a farlo anche le trasgressioni \emph{dukkata} divennero
importanti al pari delle trasgressioni \emph{pārājika}.\footnote{\emph{Pārājika}:
  Si tratta delle trasgressioni di ``sconfitta'', che sono quattro; si
  tratta delle trasgressioni più gravi, che comportano l'espulsione dal
  Saṅgha.} Prima le trasgressioni \emph{dukkata} sembravano essere
nulla, solo delle sciocchezze. Così le consideravo. La sera puoi
confessarle e hai sistemato le cose. Poi puoi commetterle di nuovo.
Questo genere di confessione è impura, perché non ti fermi, non decidi
di cambiare. Non c'è contenimento, continui semplicemente a fare quello
che facevi. Non c'è percezione della verità, non c'è lasciar andare.

Nei termini della Realtà Ultima, non è proprio necessario attraversare
la procedura di confessare le trasgressioni. Se vediamo che la nostra
mente è pura e che non c'è traccia di dubbio, allora quelle
trasgressioni terminano proprio lì. Non siamo puri se ancora dubitiamo,
se ancora esitiamo. Non siamo ancora davvero puri e così non possiamo
lasciar andare. Non vediamo noi stessi, questo è il punto. Questo nostro
Vinaya è come un recinto che ci protegge dagli errori, ed è per questa
ragione che dobbiamo essere scrupolosi al riguardo. Se non vedete il
reale valore del Vinaya da voi stessi, le cose si fanno difficili.

Molti anni prima di venire al Wat Pah Pong decisi di rinunciare al
denaro. Ci avevo pensato per gran parte del tempo, durante un Ritiro
delle Piogge. Alla fine presi il mio portafoglio e andai da un certo
\emph{Mahā} che allora viveva con me, e lo posai a terra di fronte a
lui. « Ecco \emph{Mahā}, prendi questo denaro. Da oggi in poi, come
monaco non accetterò né possiederò del denaro. Mi sei testimone.~»
«~Tienilo, venerabile, può esserti utile per i tuoi studi.~» Il
venerabile \emph{Mahā} non era molto incline a prendere il denaro, era
imbarazzato. «~Perché vuoi liberarti di tutto questo denaro?~» «~Non
devi preoccuparti per me. Ho preso la mia decisione. Ho deciso questa
notte.~» Dal giorno in cui prese quel denaro fu come se qualcuno avesse
scavato un fossato tra noi. Non fummo più in grado di capirci. Per quel
giorno egli mi è ancora testimone. Da allora non ho più fatto uso di
denaro né sono stato coinvolto in acquisti o vendite. Mi sono contenuto
in ogni modo al riguardo del denaro. Ero sempre guardingo per evitare di
sbagliare, sebbene non avessi fatto mai nulla d'errato. Interiormente
sostenevo la pratica meditativa. Non avevo più bisogno di ricchezze, le
consideravo come un veleno. Se si dà del veleno a un essere umano,
oppure a un cane o a un qualsiasi altro essere vivente, esso
invariabilmente causa la morte o induce sofferenza. Se le cose le
vediamo con chiarezza in questo modo, staremo costantemente in guardia
per non prendere quel ``veleno''. Se lo vediamo con chiarezza come
nocivo, non è difficile rinunciare.

Per quanto concerne il cibo e i pasti offerti, se avevo dubbi non li
accettavo. Non importava quanto il cibo potesse essere delizioso o
raffinato, non lo mangiavo. Prendiamo ad esempio il pesce crudo
marinato. Supponi di vivere nella foresta, di andare a fare il giro per
la questua e di ricevere solo riso bianco e qualche pesce marinato
avvolto nelle foglie. Quando torni nel luogo in cui dimori, apri il
pacchetto di foglie e vedi che si tratta di pesce marinato.\footnote{Il
  Vinaya proibisce ai \emph{bhikkhu} di mangiare carne o pesce crudo.}
Gettalo via e basta! Piuttosto che trasgredire i precetti, è meglio
mangiare solo riso bianco. Deve essere così prima che tu possa dire di
avere davvero capito, e allora il Vinaya diventa più semplice.

Se altri monaci volevano darmi dei generi di prima necessità, come la
ciotola per la questua, il rasoio o qualsiasi altra cosa, non accettavo,
a meno che non li conoscessi come compagni di pratica con un livello di
osservanza del Vinaya simile al mio. Perché no? Come fai a fidarti di
qualcuno che non ha contenimento? Può fare cose di ogni genere. I monaci
privi di contenimento non comprendono il valore del Vinaya, ed è perciò
possibile che abbiano ottenuto quelle cose in modi impropri. Ecco quanto
ero scrupoloso. Il risultato fu che alcuni miei compagni monaci mi
guardavano di traverso. «~Non socializza, non si mescola con gli altri
\ldots{}~» Restavo impassibile. Pensavo: «~È certo che potremo mischiarci
quando moriremo \ldots{} Quando si tratta della morte siamo tutti nella
stessa barca.~» Vivevo sopportando. Ero uno che parlava poco. Se gli
altri criticavano la mia pratica restavo impassibile. Perché? Perché
anche se avessi spiegato non mi avrebbero capito. Non sapevano nulla
della pratica. Come quelle volte che venivo invitato a una cerimonia
funebre e qualcuno diceva: «~Non dargli retta! Metti il denaro nella sua
borsa senza dirgli nulla, non farglielo sapere.~»\footnote{Sebbene per i
  monaci costituisca una trasgressione accettare del denaro, molti lo
  fanno. Alcuni possono accettarlo facendo intendere di no, un fatto che
  probabilmente ci consente in questo caso di capire come i laici
  considerassero il rifiuto del venerabile Ajahn Chah nei riguardi del
  denaro. Credevano che egli lo avrebbe accettato se non glielo
  offrivano apertamente, facendolo semplicemente scivolare nella borsa.}
Rispondevo così: «~Ehi! Pensate che sia morto o qualcosa del genere? Già
lo sapete, solo perché alcuni chiamano l'alcol profumo, questo non è
sufficiente a farlo diventare profumo. Voi, gente, quando però volete
bere alcol lo chiamate profumo, e andate avanti a bere. Dovete essere
pazzi!~»

Allora il Vinaya può essere difficile. Dovete accontentarvi di poco,
dovete essere distaccati. Dovete vedere, e vedere rettamente. Una volta,
mentre viaggiavo e stavo attraversando Saraburi, il mio gruppo andò a
stare per un po' in un villaggio nei pressi di un tempio. L'abate aveva
lo stesso mio grado di anzianità monastica. Al mattino andavamo tutti
insieme a fare il giro per la questua, e poi tornavamo in monastero e
deponevamo le nostre ciotole. I laici portavano piatti di cibo nella
sala e li poggiavano. I monaci li prendevano, li scoprivano e li
allineavano affinché fossero formalmente offerti. Dalla parte opposta,
un monaco metteva la sua mano sul piatto. E questo era tutto! Dopo di
che i monaci li portavano agli altri e distribuivano il cibo per il
pasto. Erano cinque i monaci che allora viaggiavano insieme a me, ma
nessuno di noi toccò quel cibo. Nel giro per la questua avevamo ricevuto
solo riso bianco, e così sedemmo assieme a loro e mangiammo unicamente
riso bianco. Nessuno di noi avrebbe osato prendere il cibo da quei
piatti. Le cose andarono avanti in questo modo per alcuni giorni, fino a
quando iniziai ad avere la sensazione che l'abate fosse turbato per il
nostro comportamento. Forse uno dei suoi monaci era andato da lui e gli
aveva detto: «~Quei monaci che sono venuti in visita da noi non prendono
il cibo. Non so quali intenzioni abbiano.~» Dovevo restare lì ancora per
qualche giorno, e così andai dall'abate.

Gli dissi: «~Venerabile, puoi dedicarmi un momento del tuo tempo per
favore? Degli impegni mi hanno indotto a chiedere la tua ospitalità per
qualche giorno, ma temo che ci siano una o due cose che per te e per i
tuoi monaci sono incomprensibili. Precisamente, si tratta del fatto che
non mangiamo il cibo offerto dai laici. Vorrei spiegarmi al riguardo,
venerabile. Non si tratta di nulla di importante, è solo che, a
proposito del ricevere le offerte, noi abbiamo imparato a praticare in
questo modo, venerabile, che quando i laici poggiano il cibo, e i monaci
scoprono i piatti e li sistemano per l'offerta formale, si tratta di una
trasgressione \emph{dukkata}. Per la precisione, maneggiare o toccare il
cibo che non è ancora stato formalmente offerto nelle mani dei monaci,
``guasta'' il cibo. Secondo il Vinaya, qualsiasi monaco che mangi quel
cibo incorre in una trasgressione. Questo è tutto. Non voglio criticare
nessuno, e nemmeno cercare di forzare te o i tuoi monaci a smettere di
praticare in questo modo, assolutamente. Voglio solo che tu conosca le
mie buone intenzioni, perché ciò è indispensabile per consentirmi di
restare qui ancora per qualche giorno.~»

Alzò le sue mani in \emph{añjali}:\footnote{\emph{Añjali}: È un gesto di
  rispetto consistente nel congiungere le mani al petto al cospetto di
  qualcuno; oggigiorno è ancora diffuso nei paesi buddhisti e in India.}
«~\emph{Sadhu}!\footnote{\emph{Sadhu}: Espressione che in pāli indica
  soprattutto approvazione e assenso, e che può essere tradotta in vari
  modi (bene, opportuno, retto, giusto, proficuo, salutare).}
Eccellente! Fino a ora non ho mai visto a Saraburi un monaco che osservi
le regole minori del Vinaya. Oggigiorno non ce ne sono più. Se ancora ci
sono monaci di questo genere, vivono fuori Saraburi. Consentitemi di
lodarvi. Non ho alcuna obiezione. Molto bene.~» Quando il mattino
seguente tornammo dal giro per per la questua, nessuno dei monaci si
avvicinò a quei piatti. Furono i laici stessi a prepararli e offrirli,
perché temevano che i monaci non avrebbero mangiato. Da quel giorno in
poi i monaci e i novizi sembravano stare davvero sulle spine, e perciò
tentai di spiegare loro le cose, di rasserenare le loro menti. Penso che
avessero timore di noi, se ne andavano a chiudersi nelle loro stanze, in
silenzio. Si vergognarono così tanto che per due o tre giorni cercai di
farli sentire a loro agio. Non avevo davvero nulla contro di loro. Non
avevo detto cose di questo genere: «~Non c'è cibo a sufficienza.~»
Oppure: «~Prendi questo o quel cibo.~» Perché non l'avevo fatto? Perché
in precedenza avevo digiunato, talora per sette o otto giorni. Lì avevo
riso bianco, sapevo che non sarei morto. Ricevevo la mia forza dalla
pratica, dall'aver studiato e praticato di conseguenza. Il Buddha era il
mio esempio. Ovunque andassi, qualsiasi cosa gli altri facessero, non mi
immischiavo. Mi dedicavo unicamente alla pratica, perché mi preoccupavo
di me stesso, mi preoccupavo della pratica.

Chi non osserva il Vinaya, chi non pratica la meditazione e chi pratica
rettamente non possono vivere insieme, devono percorrere strade diverse.
In passato io stesso non lo capivo. In quanto insegnante, insegnavo agli
altri ma non praticavo. È davvero una cosa pessima. Quando guardai in
profondità dentro tutto ciò, la mia pratica e la mia conoscenza erano
separate come lo sono la terra e il cielo. Per questo, chi vuole
organizzare centri di meditazione nella foresta, non lo faccia. Se già
non avete veramente la conoscenza, non ci provate, farete solo disastri.
Alcuni monaci pensano che andando a vivere nella foresta troveranno la
pace, ma non comprendono ancora i punti essenziali della pratica.
Tagliano l'erba da sé,\footnote{Si tratta di un'altra trasgressione dei
  precetti, una trasgressione \emph{pācittiya}.} fanno tutto da sé. Chi
conosce davvero la pratica non è interessato a posti come questi, sa che
non riuscirà. Comportarsi in questo modo non conduce al progresso. Non
importa quanto sereno un posto possa essere, non si possono fare
progressi se si fanno cose sbagliate.

Vedono i monaci della foresta vivere nella foresta e vanno a vivere
nella foresta come loro, ma non è la stessa cosa. Gli abiti monastici
non sono gli stessi, le abitudini a riguardo del cibo non sono le
stesse, tutto è diverso. Per la precisione, non addestrano se stessi,
non praticano. Il posto è sprecato, non funziona veramente. Se funziona,
funziona solo come un luogo per mettersi in mostra o per farsi
pubblicità, proprio come in una fiera per i medicinali. Non si va al di
là di questo. Chi ha praticato solo un po' e va a insegnare agli altri
non è ancora maturo, non capisce veramente. In breve tempo rinunciano e
tutto cade in pezzi. Porta solo problemi. Dobbiamo perciò studiare un
po', guardate il \emph{Novakovāda},\footnote{\emph{Navakovāda}: Una
  sinossi semplificata dell'elementare Dhamma-Vinaya.} cosa dice?
Studiatelo, memorizzatelo, fino a quando capite. Di tanto in tanto
chiedete al vostro insegnante per i punti più difficili, ve li
spiegherà. Studiate in questo modo fino a quando il Vinaya lo capite
davvero.

