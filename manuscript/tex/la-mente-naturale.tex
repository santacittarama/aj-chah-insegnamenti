\chapter{La Mente Naturale}

Il nostro modo di praticare consiste nel guardare le cose da vicino per
renderle chiare. Siamo persistenti e costanti, non precipitosi né
frettolosi, ma nemmeno troppo lenti. Si tratta di percepire gradualmente
la nostra direzione e di progredire. Ovviamente, progredire significa
lavorare in direzione di qualcosa: c'è un punto d'arrivo nella nostra
pratica.

Quando cominciamo a praticare, per la maggior parte di noi non c'è altro
che desiderio. Iniziamo a praticare perché lo vogliamo. In questa fase,
volere è volere in modo sbagliato. È illusione. È un volere frammisto di
errata comprensione. Se il volere non è frammisto di errata
comprensione, diciamo che è volere con saggezza (\emph{paññā}). Non è
illusione, è volere con retta comprensione. In questo caso diciamo che è
dovuto alla \emph{pāramī}\footnote{\emph{Pāramī}: ``Perfezione''. Per
  l'elenco delle dieci relative qualità, si veda il \emph{Glossario}.}
di una persona, o ai meriti accumulati in passato. Certamente non è così
per tutti.

Alcuni non vogliono avere desideri o vogliono non avere desideri, poiché
pensano che la nostra pratica vada nella direzione del non volere. Se
non c'è desiderio, ovviamente non c'è possibilità di praticare. Lo
possiamo vedere da noi stessi. Il Buddha e tutti i suoi discepoli
praticarono per porre fine alle contaminazioni. Dobbiamo voler praticare
e dobbiamo voler porre fine alle contaminazioni. Dobbiamo voler avere la
pace nella mente e voler non avere la confusione. Se questo volere è in
noi frammisto di errata comprensione, ciò sicuramente comporterà
maggiori difficoltà. Se a questo proposito siamo onesti, non sappiamo
assolutamente nulla. Oppure, quel che sappiamo non ha alcun effetto
perché non siamo in grado di utilizzarlo propriamente.

Tutti, incluso il Buddha, hanno iniziato in questo modo, con il
desiderio di praticare, volendo avere la pace nella mente e volendo non
avere la confusione e la sofferenza. Questi due tipi di desiderio hanno
lo stesso identico valore. Se non compresi, allora sia voler essere
liberi dalla confusione sia non volere la sofferenza sono
contaminazioni. Sono un modo folle di volere, significano desiderare
senza saggezza. Nella nostra pratica consideriamo questo desiderio o
come indulgere ai sensi oppure come auto-mortificazione. Il nostro
Maestro, il Buddha, fu catturato proprio da questo conflitto, da questo
dilemma. Seguì molti modi di praticare che sfociavano solo in due
estremi. Ai nostri giorni siamo esattamente nella stessa situazione.
Siamo ancora afflitti da questo dualismo, ed è a causa di esso che
continuiamo a smarrire il Sentiero.

È così che dobbiamo incominciare, ovviamente. Incominciamo come esseri
mondani, come esseri con contaminazioni, vogliamo senza saggezza,
desideriamo senza retta comprensione. Se manchiamo di retta
comprensione, entrambi questi desideri lavoreranno contro di noi. Che si
tratti di volere o di non volere, è ancora brama (\emph{tanhā}). Se non
comprendiamo queste due cose, non sapremo come affrontarle quando
sorgeranno. Avremo la percezione che andare avanti è sbagliato e che
tornare indietro è sbagliato, ma non ci potremo fermare. Qualsiasi cosa
faremo, incontreremo solo volere ancora di più. Ciò a causa della
mancanza di saggezza, e a causa della brama.

È proprio qui, con questo volere e questo non volere, che possiamo
comprendere il Dhamma. Il Dhamma che stiamo cercando esiste proprio qui,
ma non lo vediamo. Insistiamo, invece, nei nostri sforzi per smettere di
volere. Vogliamo che le cose siano in un modo, piuttosto che in
qualsiasi altro. Oppure, vogliamo che non siano in un certo modo, ma in
un altro. In verità queste due cose sono identiche. Fanno parte dello
stesso dualismo.

Possiamo forse non aver capito che il Buddha e tutti i suoi discepoli
ebbero questo tipo di volizione. Il Buddha ovviamente comprese volere e
non volere. Comprese che sono solo attività della mente, che cose di tal
genere in un baleno appaiono e poi semplicemente scompaiono. Questi tipi
di desiderio si verificano in continuazione. Quando c'è saggezza, non ci
identifichiamo con essi, siamo liberi dall'attaccamento. Che si tratti
di volere o di non volere, li vediamo semplicemente come tali. In
realtà, è solo l'attività della mente naturale. Quando guardiamo più da
vicino, vediamo con chiarezza che è così che stanno le cose.

\section{La saggezza dell'esperienza quotidiana}

È qui che la nostra pratica di contemplazione ci condurrà alla
comprensione. Facciamo un esempio. Un pescatore tira in barca la sua
rete con dentro un grande pesce. Come pensate che si senta, mentre lo
tira in barca? Ha paura che il pesce scappi, si affretterà e inizierà a
combattere con la rete, afferrandola e tirandola. Prima che se ne possa
rendere conto, il pesce è fuggito. Stava tirando con troppa forza.
Anticamente avrebbero parlato così. Insegnavano che avremmo dovuto
tirare gradualmente, tirarlo in barca con attenzione senza perderlo. È
come nella nostra pratica. Percepiamo gradualmente quel che è giusto, la
tiriamo in barca con attenzione senza perderla. A volte succede che non
ce la sentiamo di praticare. Forse non abbiamo voglia di osservare noi
stessi o forse non vogliamo sapere, ma continuiamo a praticare.
Continuiamo a farcela piacere. Questa è la pratica: se ce la sentiamo di
praticare lo facciamo, e se non ce la sentiamo di praticare lo facciamo
ugualmente. Continuiamo, semplicemente.

Se siamo entusiasti della nostra pratica, la forza della nostra fiducia
darà energia a quello che stiamo facendo. A questo livello, però, siamo
ancora privi di saggezza. Benché energici, non otterremo grande
beneficio dalla pratica. Continuiamo a lungo e, poi, sorge la sensazione
che stiamo per trovare la Via. Possiamo avere la sensazione che non
riusciamo a trovare pace e serenità, o che non siamo sufficientemente
dotati per la pratica. Oppure possiamo avere la sensazione che questa
Via non è più possibile percorrerla. E così rinunciamo! A questo punto
dobbiamo stare attenti, davvero attenti. Dobbiamo essere molto pazienti
e resistenti. È proprio come tirare in barca un grande pesce, percepiamo
gradualmente quel che è giusto. Lo tiriamo in barca con attenzione. La
battaglia non sarà poi così difficile e perciò, senza fermarci,
continuiamo a tirare. Alla fine, dopo un po' di tempo, il pesce si
stancherà, smetterà di combattere e saremo in grado di catturarlo con
facilità. Di solito succede così, tiriamo gradualmente in barca la
nostra pratica.

È in questo modo che facciamo contemplazione. Se non abbiamo alcuna
conoscenza particolare o istruzione a riguardo degli aspetti teorici
degli insegnamenti, contempliamo sulla base della nostra esperienza
quotidiana. Usiamo la conoscenza che già abbiamo, la conoscenza che
deriva dalla nostra esperienza quotidiana. Per la mente questa è una
conoscenza naturale. In verità, che la si studi o no, la realtà della
mente è già qui. La mente è la mente, che la si studi o meno. Questa è
la ragione per cui diciamo che pure se il Buddha non fosse nato, tutto
nel mondo sarebbe rimasto così com'è. Ogni cosa esiste in accordo con la
sua stessa natura. Questa condizione naturale non cambia e nemmeno va da
qualche parte. È solo così com'è. Questo è chiamato
\emph{saccadhamma}.\footnote{\emph{Saccadhamma}: Verità ultima.}
Ovviamente, se non comprendiamo questo \emph{saccadhamma}, non saremo in
grado di riconoscerlo.

È così che pratichiamo la contemplazione. Se non siamo particolarmente
versati nelle Scritture, prendiamo la mente stessa come oggetto di
studio e la leggiamo. Se contempliamo in continuazione, la comprensione
a riguardo della natura della mente sorgerà gradualmente. Non dobbiamo
forzare nulla.

\section{Sforzo costante}

Fino a quando non saremo in grado di fermare la nostra mente, fino a
quando non raggiungeremo la tranquillità, la mente continuerà come
prima. È per questa ragione che gli insegnanti dicono: «~Continua
semplicemente, continua con la pratica!~» Forse pensiamo: «~Se non ho
capito, come posso farlo?~» Fino a quando non saremo in grado di
praticare in modo giusto, la saggezza non sorgerà. Se pratichiamo senza
fermarci, inizieremo a pensare a cosa stiamo facendo. Inizieremo a
prendere in considerazione la nostra pratica.

Nulla succede immediatamente. Per questo all'inizio non riusciamo a
vedere alcun risultato della nostra pratica. È come l'esempio che vi ho
spesso offerto, quello dell'uomo che cerca di accendere un fuoco
sfregando due bastoncini di legno. Dice a se stesso: «~Dicono che qui ci
sia del fuoco.~» E comincia a sfregarli energicamente. È molto irruente.
Sfrega e sfrega, ma la sua impazienza è troppa. Continua a volere il
fuoco, ma il fuoco non arriva. Perciò si ferma per riposare un po'.
Ricomincia di nuovo, ma i progressi sono lenti, e così si riposa di
nuovo. Nel frattempo il calore è scomparso; non ha continuato abbastanza
a lungo. Sfrega e sfrega finché si stanca e si ferma del tutto. Non solo
è stanco, ma pure sempre più scoraggiato, così che rinuncia
completamente. «~Non c'è alcun fuoco qui!~» In effetti stava facendo
quello che era necessario, ma non c'era abbastanza calore per accendere
un fuoco. Il fuoco era lì per tutto il tempo, ma lui non ha portato a
termine il lavoro.

Questo genere di esperienza induce scoraggiamento nella pratica del
meditante, e così egli passa senza sosta da un tipo di pratica a un
altro. Un'esperienza simile l'abbiamo nella nostra stessa pratica. È
uguale per tutti. Perché? Perché siamo ancora radicati nelle
contaminazioni. Anche il Buddha aveva contaminazioni, ma pure molta
saggezza al riguardo. Quando erano ancora esseri mondani, il Buddha e
gli \emph{arahant} erano esattamente come noi. Se siamo ancora esseri
mondani, non pensiamo correttamente. Perciò, quando la volizione sorge
non la vediamo, e quando non sorge non la vediamo. A volte sentiamo
dentro un rimescolamento, altre volte ci sentiamo contenti. Quando non
c'è volizione proviamo una sorta di contentezza, ma anche una certa qual
confusione. Quando c'è volizione, ci possono essere sia contentezza sia
confusione. In questo modo è tutto mescolato.

\section{Conoscere se stessi, conoscere gli altri}

Ad esempio, il Buddha ci insegnò a contemplare il nostro corpo: capelli,
peli, unghie, denti, pelle \ldots{} è tutto quanto corpo. Dateci
un'occhiata! Ci viene detto di investigare proprio qui. Se non vediamo
queste cose con chiarezza come sono dentro di noi, non capiremo gli
altri. Non vedremo gli altri con chiarezza né vedremo noi stessi. Se
comprendiamo e vediamo con chiarezza la natura dei nostri corpi,
ovviamente i nostri dubbi e interrogativi in relazione agli altri
scompariranno. È perché il corpo e la mente (\emph{rūpa}\footnote{\emph{Rūpa}:
  Fenomeno fisico; dato sensoriale.} e \emph{nāma}\footnote{\emph{Nāma}:
  Fenomeno mentale.}) sono uguali per tutti. Non è necessario andare a
esaminare tutti i corpi del mondo per sapere che sono tutti uguali al
nostro, che noi siamo uguali a loro. Se abbiamo questo tipo di
comprensione, il nostro fardello diventa più leggero. Senza questo tipo
di comprensione, tutto quello che facciamo conduce solo a un fardello
più pesante. Se per conoscere gli altri dovessimo andare a esaminare
tutti quanti in tutto il mondo, sarebbe davvero difficile. Ci
scoraggeremmo presto.

Per il nostro Vinaya\footnote{\emph{Vinaya}: Il codice della disciplina
  monastica buddhista.} avviene una cosa simile. Quando guardiamo il
Vinaya abbiamo la sensazione che sia veramente difficile. Dobbiamo
seguire ogni regola, studiare ogni regola e riconsiderare la nostra
pratica alla luce di ogni regola. Al solo pensarci, diciamo: «~Oh, è
impossibile!~» Leggiamo il senso letterale di tutte quelle numerose
regole e, se solo seguiamo i nostri pensieri al riguardo, potremmo ben
decidere che osservarle tutte è al di là delle nostre capacità. Tutti
coloro che hanno avuto tale atteggiamento nei riguardi del Vinaya hanno
avuto questa stessa sensazione: ci sono così tante regole!

Le Scritture ci dicono che dobbiamo esaminare noi stessi in relazione a
ogni singola regola e osservarle tutte rigorosamente. Dobbiamo
conoscerle tutte e osservarle alla perfezione. Ciò equivale a dire che
per capire gli altri dobbiamo assolutamente esaminarli tutti. È un
atteggiamento molto pesante. Succede così perché prendiamo alla lettera
quel che viene detto. Se seguiamo i libri di testo, la strada che
dobbiamo percorrere è questa. Alcuni maestri insegnano in questo modo:
aderenza stretta a quel che dicono i testi. Così non può
funzionare.\footnote{In un'altra occasione il venerabile Ajahn completò
  questa analogia dicendo che se sappiamo come guardare nelle nostre
  menti, ciò equivale a osservare tutte quante le numerose regole del
  Vinaya.} In verità, la nostra pratica non si svilupperà affatto se
studiamo la teoria in questo modo. La nostra fiducia scomparirà, la
nostra fiducia nel Sentiero andrà distrutta. Ciò avviene perché non
abbiamo ancora capito. Quando ci sarà la saggezza, capiremo che tutti
nel mondo intero sono davvero riconducibili a un'unica persona, sono
uguali proprio a questo nostro essere. È per questa ragione che studiamo
e contempliamo il nostro corpo e la nostra mente. Vedendo e comprendendo
la natura del nostro stesso corpo e della nostra stessa mente giungiamo
a comprendere il corpo e la mente di tutti. Così il peso della nostra
pratica diventa più leggero.

Il Buddha disse che dovremmo istruire noi stessi, insegnare a noi
stessi, nessun altro può farlo al nostro posto. Quando studieremo e
comprenderemo la natura della nostra stessa esistenza, comprenderemo la
natura di tutta l'esistenza. Tutti sono uguali. Siamo tutti lo stesso
``prodotto'' e veniamo tutti dalla stessa fabbrica, solo le sfumature
sono diverse, questo è tutto! Proprio come il ``Bort-hai'' e il
``Tum-Jai''.\footnote{Sono medicinali thailandesi.} Sono entrambi
antidolorifici e hanno gli stessi effetti, ma uno è chiamato
``Bort-hai'' e l'altro ``Tum-Jai''. In realtà non sono diversi.

Vedrete che questo modo di considerare le cose diverrà sempre più facile
man mano che, gradualmente, riuscirete a unificare tutti gli aspetti
della pratica. Lo chiamiamo ``percepire la nostra Via'', ed è così che
cominciamo a praticare. Diverremo abili nel farlo. Continuiamo finché
non giungiamo a comprendere e, quando sorgerà questa conoscenza, vedremo
la realtà con chiarezza.

\section{Teoria e pratica}

Continuiamo a praticare in questo modo fino a quando diventiamo abili.
Dopo un po' di tempo, a seconda delle nostre particolari tendenze e
capacità, sorgerà un nuovo genere di comprensione. Si chiama
investigazione dei \emph{dhamma} (\emph{dhammavicaya}), ed è così che
nella mente sorgono i Sette Fattori dell'Illuminazione\footnote{O anche
  Sette Fattori del Risveglio; si veda anche \emph{bojjhaṅga} nel
  \emph{Glossario}.}. L'investigazione dei \emph{dhamma} è uno di essi.
Gli altri sono consapevolezza, energia, gioia estatica, tranquillità,
concentrazione (\emph{samādhi}) ed equanimità.

I Sette Fattori dell'Illuminazione li abbiamo studiati e sappiamo cosa
ci dicono i libri, ma in realtà non abbiamo visto i reali fattori
dell'Illuminazione. Essi sorgono nella mente. Per questo il Buddha ci
diede i vari insegnamenti. Tutti gli Illuminati hanno insegnato la via
d'uscita dalla sofferenza e i loro insegnamenti registrati per iscritto
li chiamiamo insegnamenti teorici. Originariamente questa teoria derivò
dalla pratica, ma ora è diventata solo imparare libri o parole. I
fattori reali dell'Illuminazione sono scomparsi perché non li conosciamo
dentro di noi, non li vediamo all'interno della nostra mente. Se
sorgono, sorgono dalla pratica. Se sorgono dalla pratica, allora sono i
fattori che conducono all'Illuminazione del Dhamma, e possiamo avvalerci
del loro sorgere come un indicatore del fatto che la nostra pratica è
corretta. Se non stiamo praticando correttamente, queste cose non
appariranno.

Se pratichiamo nel modo giusto, possiamo vedere il Dhamma. Per questo vi
dico di continuare a praticare, di percepire la vostra Via gradualmente
e d'investigare in continuazione. Non pensiate che quel che state
cercando non possa essere trovato proprio qui, ma in un qualche altro
posto. Prima di arrivare in questo monastero, uno dei miei discepoli più
anziani era stato altrove a imparare la lingua pāli. Non ebbe molto
successo con i suoi studi e così, siccome pensava che i monaci che
praticano la meditazione fossero in grado di vedere e di capire ogni
cosa solo stando seduti, è venuto qui per tentare questa strada. È
arrivato qui, al Wat Pah Pong, con l'intenzione di sedere in
meditazione. Così, pensava, sarebbe stato capace di tradurre le
Scritture in pāli. Aveva quest'idea della pratica. Allora gli ho
spiegato il nostro modo di praticare. Aveva frainteso tutto. Aveva
pensato che fosse una cosa facile: solo stare seduto affinché tutto si
chiarisca.

Se parliamo di comprensione del Dhamma, sia i monaci dediti allo studio
sia quelli dediti alla pratica usano gli stessi termini. Però, la
comprensione che proviene dallo studio teorico e quella che giunge dalla
pratica del Dhamma non sono esattamente le stesse. Può sembrare che lo
siano, ma una è più intensa. Una è più profonda dell'altra. Il tipo di
comprensione che proviene dalla pratica conduce alla resa, alla
rinuncia. Persistiamo nella nostra contemplazione, perseveriamo fino a
quando la capitolazione non è totale. Se il desiderio o la collera
sorgono nella nostra mente, restiamo indifferenti. Non ci limitiamo a
non considerarle, bensì le prendiamo e le investighiamo per vedere come
e da dove sorgono. Se questi stati mentali sono già nella nostra mente,
allora li contempliamo per vedere come lavorano contro di noi. Li
vediamo con chiarezza, e comprendiamo le difficoltà che causiamo a noi
stessi quando crediamo a questi stati mentali e li seguiamo. Questo
genere di comprensione non può essere rintracciato in un posto che non
sia la nostra mente pura.

Per questo chi studia la teoria e chi pratica la meditazione non si
capiscono. Di solito chi enfatizza lo studio dice cose di questo genere:
«~I monaci che praticano solo la meditazione seguono unicamente le loro
opinioni. Non sono fondati nell'Insegnamento.~» In un certo senso, in
verità, le due vie dello studio e della pratica sono esattamente uguali.
Per capire, possiamo pensare a tale questione come alla palma e al dorso
della mano. Se mettiamo la mano con la palma rivolta verso l'alto,
sembra che il dorso sia scomparso. In realtà non è scomparso, è solo
nascosto, è sotto. Quando diciamo che non possiamo vederlo, non
significa che sia scomparso completamente, significa solo che è
nascosto. Quando voltiamo la mano, succede la stessa cosa alla palma.
Non va da nessuna parte, è solo nascosta.

È questo che dovremmo tenere a mente, quando consideriamo la pratica. Se
pensiamo che sia ``scomparsa'', ci metteremo a studiare sperando di
ottenere risultati. Però, non importa quanto studiate il Dhamma, non
capirete mai perché non conoscerete secondo Verità. Se comprendiamo la
reale natura del Dhamma, ciò diventa lasciar andare. Questo è
arrendersi, rimuovere l'attaccamento (\emph{upādāna}),\footnote{\emph{Upādāna}:
  Attaccamento, aggrapparsi, aderire; è il sostegno per il divenire e la
  nascita.} non aggrapparsi più o, se c'è ancora attaccamento, farlo
diminuire sempre più. Questa è la differenza tra le due strade dello
studio e della pratica.

Quando si parla di studio, noi lo intendiamo così: i nostri occhi sono
oggetto di studio, i nostri orecchi sono oggetto di studio, tutto è
oggetto di studio. Possiamo sapere che la forma sia in questo o in quel
modo, ma ci attacchiamo alla forma senza conoscere una via d'uscita.
Differenziamo i suoni, ma ci attacchiamo a essi. Forme, suoni, odori,
sapori, sensazioni corporee e impressioni mentali: sono come lacci che
intrappolano tutti gli esseri.

Il nostro modo di praticare il Dhamma consiste nell'investigare queste
cose. Quando sorgono alcune sensazioni, se siamo competenti dal punto di
vista teorico, immediatamente ci rivolgiamo alla teoria per vedere come
questa o quella cosa avvenga in questo o in quel modo per poi
trasformarsi in altro \ldots{} e così via. Se non abbiamo imparato la teoria,
per lavorare abbiamo solo lo stato naturale della nostra mente. Questo è
il nostro Dhamma. Se c'è saggezza, saremo in grado di esaminare questa
nostra mente naturale e di usarla come oggetto di studio. È esattamente
la stessa cosa. La nostra mente naturale è la teoria. Il Buddha disse di
prendere ogni pensiero e ogni sensazione che sorgono e di investigarli.
Usate la realtà della nostra mente naturale come vostra teoria. È su
questa realtà che facciamo affidamento.

\section{La meditazione di visione profonda (vipassanā)}

% TODO footnote
% \footnote{\emph{vipassanā}. Visione profonda di natura intuitiva dei fenomeni
%   fisici e mentali, e del loro sorgere e scomparire.}

Se avete fiducia, non conta se avete studiato o meno la teoria. Se la
nostra mente fiduciosa ci conduce a sviluppare la pratica, se ci conduce
a sviluppare costantemente l'energia e la pazienza, allora lo studio non
conta. Abbiamo la consapevolezza quale fondamento della nostra pratica.
Siamo consapevoli in tutte le posture del corpo, seduti, in piedi,
camminando o giacendo. E se c'è consapevolezza, essa sarà accompagnata
da chiara comprensione. Consapevolezza e chiara comprensione sorgeranno
insieme. Possono sorgere così rapidamente che è impossibile
distinguerle. Però, quando c'è consapevolezza ci sarà anche chiara
comprensione.

Quando la nostra mente sarà ferma e stabile, la consapevolezza sorgerà
velocemente e con facilità, e sarà allora che avremo anche saggezza.
Tuttavia, a volte la saggezza è insufficiente o non sorge al momento
giusto. Ci possono essere consapevolezza e chiara comprensione, ma esse,
da sole, non sono sufficienti per controllare la situazione. In genere,
se la consapevolezza e la chiara comprensione sono un fondamento della
mente, allora la saggezza sarà lì ad assisterle. Ovviamente, dobbiamo
costantemente sviluppare questa saggezza mediante la pratica di
meditazione di visione profonda. Questo significa che qualsiasi cosa
sorga nella mente può essere oggetto di consapevolezza e di chiara
comprensione. Dobbiamo però comprendere in accordo con \emph{aniccā},
\emph{dukkha} e \emph{anattā}. La base è l'impermanenza (\emph{aniccā}).
\emph{Dukkha} si riferisce alla qualità dell'insoddisfazione e
\emph{anattā} afferma che quel che sorge è privo di un'entità
individuale. Vediamo che è solo una sensazione che è sorta, che non ha
alcun sé, alcuna identità, e che essa scompare per conto suo. Tutto qui!
Chi è illuso, chi non ha saggezza, perderà questa occasione e non sarà
in grado di utilizzare queste cose a proprio vantaggio.

Se la saggezza è presente, allora la consapevolezza e la chiara
comprensione saranno proprio lì, assieme a essa. In questa fase iniziale
la saggezza ovviamente può non essere perfettamente limpida. La
consapevolezza e la chiara comprensione non sono perciò in grado di
catturare ogni oggetto, ma la saggezza giunge in loro aiuto. Può vedere
che genere di consapevolezza è presente e il tipo di sensazione che è
sorta. Oppure, più in generale, quali che siano la consapevolezza o la
sensazione presenti, che tutto è Dhamma.

Il Buddha assunse come fondamento la pratica della meditazione di
visione profonda. Comprese che la consapevolezza e la chiara
comprensione erano entrambe incerte e instabili. Tutto ciò che è
instabile e che vogliamo abbia stabilità, ci causa sofferenza. Vogliamo
che le cose si accordino con i nostri desideri, ma soffriamo perché le
cose non vanno in quel modo. Questo è l'influsso di una mente impura,
l'influsso di una mente che manca di saggezza.

Quando pratichiamo, abbiamo la tendenza a desiderare che la pratica sia
facile, che sia nel modo in cui piace a noi. Non dobbiamo andare molto
lontano per comprendere un atteggiamento del genere. Basta guardare
questo corpo! È mai veramente come lo vogliamo? Ora ci piace che sia in
un modo, e un momento dopo vogliamo che sia in un altro. È mai davvero
stato come piace a noi? La natura del nostro corpo e della nostra mente
è esattamente la stessa. È semplicemente così com'è.

Nella nostra pratica, questo aspetto può essere facilmente trascurato.
Di solito, se sentiamo che qualcosa non è in sintonia con noi, la
gettiamo via, gettiamo via tutto quel che non ci aggrada. Non ci
fermiamo a pensare se quel modo d'essere, secondo il quale le cose ci
piacciono e non ci piacciono, sia corretto o meno. Pensiamo solo che le
cose che troviamo sgradevoli debbano essere sbagliate e che quelle che
ci risultano gradevoli debbano essere giuste. È da qui che giunge la
brama. Quando riceviamo degli stimoli per mezzo dell'occhio,
dell'orecchio, del naso, della lingua, del corpo o della mente, sorge
una sensazione di piacere o di dispiacere. Questo indica che la mente è
colma di attaccamento. Perciò, il Buddha ci diede quest'insegnamento
sull'impermanenza. Ci offrì un modo per contemplare le cose. Se ci
attacchiamo a qualcosa che non è permanente, sperimenteremo la
sofferenza. Non c'è ragione per volere che le cose siano in sintonia con
quello che ci piace e che non ci piace. Non possiamo far sì che le cose
vadano in questo modo. Non abbiamo quest'autorità o questo potere.
Indipendentemente da come ci piacerebbe che le cose siano, ogni cosa è
già nel modo in cui è. Questo tipo di desiderio non è la via d'uscita
dalla sofferenza.

È qui che possiamo vedere come la mente illusa capisca in un modo, e la
mente che non lo è capisca in un altro. Quando ad esempio la mente
dotata di saggezza riceve una sensazione, la vede come qualcosa alla
quale non attaccarsi, con la quale non identificarsi. Ciò indica
saggezza. Se non c'è alcuna saggezza, ci limitiamo a seguire la nostra
stupidità. Questa stupidità consiste nel non vedere l'impermanenza,
l'insoddisfazione e il non-sé. Quel che ci piace lo vediamo come buono e
giusto. Quello che invece non ci piace lo vediamo come non buono. Così
non possiamo giungere al Dhamma, la saggezza non può sorgere. Se
riusciamo a capirlo, allora sorge la saggezza.

Il Buddha impiantò stabilmente la pratica della meditazione di visione
profonda nella sua mente e la usò per investigare tutte le varie
impressioni mentali. Qualsiasi cosa sorgesse nella sua mente, Egli la
investigava così: benché mi piaccia, è incerto. È sofferenza, perché la
mente non può esercitare alcun influsso su queste cose che sorgono e
svaniscono in continuazione. Tutte queste cose non sono un essere o un
sé, non ci appartengono. Il Buddha ci insegnò a vederle così come sono.
Nella nostra pratica ci atteniamo a questo principio. Allora
comprendiamo che non siamo in grado di determinare i vari stati mentali
in base ai nostri desideri. Ne stanno per nascere sia di buoni che di
cattivi. Alcuni sono salutari, altri no. Se non comprendiamo
correttamente queste cose, non saremo in grado di valutarle
correttamente. Rincorreremo invece la brama o scapperemo dai nostri
desideri. A volte ci sentiamo felici e altre volte tristi, ma questo è
naturale. A volte ci sentiamo compiaciuti, altre volte delusi. Quel che
ci piace lo riteniamo buono e quel che non ci piace lo riteniamo
cattivo. In questo modo ci allontaniamo sempre più dal Dhamma. Quando
ciò avviene, non siamo in grado di capire o di riconoscere il Dhamma e,
perciò, ci sentiamo confusi. I desideri aumentano perché nelle nostre
menti c'è solo illusione.

È così che noi parliamo della mente. Non è necessario andare lontano da
noi stessi per comprendere. Osserviamo semplicemente che questi stati
mentali non sono permanenti. Vediamo che sono insoddisfacenti e che non
hanno un sé permanente. Continuiamo a sviluppare la nostra pratica in
questo modo e la chiamiamo pratica di \emph{vipassanā} o meditazione di
visione profonda. Diciamo che questo è riconoscere i contenuti della
nostra mente e, così, sviluppiamo la saggezza.

\section{La meditazione di tranquillità (samatha)}

Questa è ad esempio la nostra pratica di
\emph{samatha}.\footnote{\emph{Samatha}: Calma
  concentrata, tranquillità.} Instauriamo la consapevolezza
sull'inspirazione e sull'espirazione quale fondamento, quale mezzo per
controllare la mente. La mente diviene salda, calma e immobile perché
segue il fluire del respiro. Questa pratica per calmare la mente è
chiamata ``meditazione di \emph{samatha}''. È necessario praticare molto
in questo modo perché la mente è colma di turbamenti. È molto confusa.
Non è possibile dire per quanti anni o per quante vite sia stata così.
Se ci sediamo e contempliamo, vedremo che c'è molto che non conduce alla
pace e alla calma, molto che conduce alla confusione!

Per questa ragione il Buddha insegnò che si deve trovare un oggetto di
meditazione adatto alle nostre specifiche tendenze, un modo di praticare
giusto per il nostro carattere. Ad esempio passare continuamente in
rassegna le parti del corpo -- capelli, peli, unghie, denti e pelle --
può essere davvero tranquillizzante. Con questa pratica la mente diviene
davvero serena. Se contemplare queste cinque cose conduce alla calma, è
perché si tratta di oggetti di contemplazione appropriati, che si
accordano alle nostre tendenze. Qualsiasi cosa risulti appropriata a
tale scopo, possiamo considerarla parte della nostra pratica e
utilizzarla per domare le contaminazioni.

La rammemorazione della morte è un altro esempio. In coloro nei quali
sono ancora forti l'avidità, l'avversione e l'illusione e vi è perciò
difficoltà nel contenerle, è utile assumere la propria morte quale
oggetto di meditazione. Giungeremo a vedere che tutti devono morire, sia
ricchi sia poveri. Vedremo che muoiono buoni e cattivi. Tutti devono
morire! Quando sviluppiamo questa pratica, vediamo sorgere un
atteggiamento di disincanto. Più pratichiamo e più facilmente si produce
la calma durante le nostre sedute di meditazione. Se questo avviene è
perché si tratta di una pratica a noi adatta e appropriata. Se questa
pratica della meditazione di tranquillità non si accorda con le nostre
particolari tendenze, non produrrà quest'atteggiamento di disincanto.
Se l'oggetto è davvero adatto a noi, vedremo che tale atteggiamento
sorge regolarmente, senza molta difficoltà, e ci ritroveremo a pensarci
spesso.

A questo proposito possiamo addurre un esempio tratto dalla nostra vita
quotidiana. Quando i laici portano vassoi con molti e differenti generi
di cibo da offrire ai monaci, li assaggiamo tutti per capire quel che ci
piace. Quando li abbiamo assaggiati, possiamo dire quale cibo è per noi
il più gradevole. È solo un esempio. Mangiamo ciò che è gradevole per il
nostro palato. Non ci preoccupiamo degli altri vassoi.

La pratica di concentrare la nostra attenzione sull'inspirazione e
sull'espirazione è un esempio di un tipo di meditazione adatto a tutti.
Pare che quando ce ne andiamo in giro a praticare in molti modi
differenti non è che poi ci sentiamo molto bene. Però, appena ci sediamo
a osservare il nostro respiro proviamo una bella sensazione, possiamo
constatarlo con chiarezza. Non c'è bisogno di andare a cercare lontano,
possiamo utilizzare quel che ci è vicino. È meglio. Solo osservare il
respiro. Esce ed entra, fuori e dentro, lo osserviamo in questo modo.
Continuiamo a osservare il nostro respiro che entra ed esce a lungo e,
lentamente, la nostra mente si assesta. Sorgeranno altre attività
mentali, ma le sentiremo come se fossero distanti da noi. Proprio come
quando si vive distanti uno dall'altro e non ci sentiamo più tanto
vicini. Non siamo più in stretto contatto o, forse, non siamo più in
contatto per nulla.

La pratica di consapevolezza del respiro è più facile quando inizia a
diventarci più familiare. Se continuiamo con questa pratica, diverremo
esperti e abili nel conoscere la natura del respiro. Sapremo come ci si
sente quando il respiro è lungo e come ci si sente quando è breve.
Possiamo parlare del respiro come di un nutrimento. Quando siamo seduti
respiriamo, quando dormiamo respiriamo, quando siamo svegli respiriamo.
Se non si respira, si muore. Se ci pensiamo, comprenderemo che possiamo
vivere solo con l'aiuto del cibo. Se non mangiamo cibo ordinario per
dieci minuti, un'ora o anche un giorno, non importa. Si tratta di un
genere di nutrimento grossolano. Se anche per breve tempo non
respiriamo, moriremo. Se non si respira per cinque o dieci minuti, si
muore. Provateci!

Chi pratica la consapevolezza del respiro dovrebbe avere questo genere
di comprensione. La conoscenza che giunge da questa pratica è davvero
meravigliosa. Se non contempliamo, non vediamo il respiro come cibo. In
verità, però, ``mangiamo'' aria in continuazione, dentro, fuori, dentro,
fuori \ldots{} sempre. Inoltre, vedrete che più contemplate in questo modo,
maggiori saranno i benefici che ricaverete dalla pratica, e il respiro
diverrà sempre più sottile. Può perfino succedere che si fermi. È come
se non respirassimo affatto. In realtà, la respirazione avviene
attraverso i pori della pelle. Questa è chiamata ``respirazione
sottile''. Quando la mente è perfettamente tranquilla, la respirazione
normale può cessare. Non è necessario sorprendersi o spaventarsi. Se non
c'è respirazione, cosa dovremmo fare? Solo saperlo! Sapere che non c'è
respirazione, questo è tutto. Questa è retta pratica.

Stiamo parlando di come si effettua la pratica di \emph{samatha}, la
pratica per lo sviluppo della tranquillità. Se l'oggetto di
contemplazione che stiamo utilizzando è giusto e appropriato, ci
condurrà a questo tipo di esperienza. È l'inizio di questa pratica, ma
essa può condurci fino al termine del cammino, o almeno fino al punto in
cui possiamo comprendere chiaramente e continuare con grande fiducia. Se
continuiamo in questo modo con la contemplazione, in noi ci sarà
energia. È come versare acqua in un vaso. Versiamo acqua all'interno di
esso e continuiamo a rabboccare. Continuiamo a riempire d'acqua il vaso,
così gli insetti che vivono nell'acqua non muoiono. Sforzarsi e svolgere
quotidianamente la nostra pratica è così. Tutto questo va a vantaggio
della pratica. Ci sentiamo molto bene, siamo sereni.

Questa serenità proviene da uno stato mentale unificato. Ovviamente
questo stato mentale unificato può diventare molto problematico, perché
non vogliamo che altri stati mentali ci disturbino. In realtà, altri
stati mentali arrivano e, se ci pensiamo, è proprio così che ci può
essere uno stato mentale unificato. È come quando vediamo tanti uomini e
tante donne, ma non nutriamo nei loro riguardi gli stessi sentimenti che
abbiamo per nostro padre e nostra madre. In realtà tutti gli uomini sono
uomini proprio come nostro padre e tutte le donne sono donne proprio
come nostra madre, ma non nutriamo per loro gli stessi sentimenti.
Sentiamo che i nostri genitori sono più importanti. Attribuiamo loro più
valore. Così dovrebbe essere per il nostro stato di unificazione
mentale. Nei riguardi di esso dovremmo avere lo stesso atteggiamento che
abbiamo nei riguardi di nostra madre e di nostro padre. Apprezziamo
tutte le altre attività mentali che sorgono come facciamo con gli uomini
e con le donne in generale. Non smettiamo di vederli, riconosciamo
semplicemente la loro presenza, ma non attribuiamo ad essa lo stesso
valore di quella dei nostri genitori.

\section{Sciogliere il nodo}

Quando la nostra pratica di \emph{samatha} giungerà alla calma, la mente
sarà chiara e luminosa. L'attività della mente diminuirà sempre più. Le
varie impressioni mentali sorgeranno sempre meno. Quando questo avverrà
sorgerà una grande pace, una grande felicità, ma a questa felicità
potremmo attaccarci. Dovremmo contemplarla come incerta. Dovremmo
contemplare anche l'infelicità come incerta e impermanente.
Comprenderemo che tutte le varie sensazioni non sono durevoli e, così,
non ci aggrapperemo a esse. Se vedremo le cose in questo modo, sarà
perché c'è saggezza. Comprenderemo che le cose stanno in questo modo, in
accordo con la loro natura.

Avere questo genere di comprensione è come tenere in mano una corda con
un nodo. Se tiriamo nella direzione giusta, il nodo si allenterà e
inizierà a sciogliersi. Non sarà più così stretto e teso. È come
comprendere che le cose non devono essere sempre così. Prima, abbiamo
avuto la sensazione che sarebbero sempre state così com'erano e, perciò,
il nodo si serrava sempre più. Questo serrarsi è la sofferenza. Vivere
così comporta una grande tensione. Possiamo invece allentare un po' il
nodo e rilassarci. Perché lo allentiamo? Perché è stretto! Se non ci
attacchiamo a esso, possiamo allentarlo. Non è una condizione
permanente, che deve restare sempre così.

Assumiamo l'insegnamento dell'impermanenza quale nostro fondamento.
Vediamo che sia la felicità sia l'infelicità non sono permanenti. Le
consideriamo non affidabili. Non vi è assolutamente nulla che sia
permanente. Con questo tipo di comprensione smettiamo gradualmente di
credere ai vari stati mentali e alle varie sensazioni che affiorano
nella mente. Diminuirà allo stesso modo l'errata comprensione, e
smetteremo di credervi. Sciogliere il nodo significa questo. Continua a
diventare sempre meno stretto. L'attaccamento sarà gradualmente
sradicato.

\section{Disincanto}

Quando giungiamo a vedere l'impermanenza, l'insoddisfazione e il non-sé
in noi stessi, nel nostro corpo, nella nostra mente e in questo mondo,
noteremo sorgere un certo tedio. Non è quella noia quotidiana che ci fa
pensare di non voler sapere, vedere o dire qualcosa, di non voler avere
niente a che fare con nessuno. Quella non è noia vera e propria, c'è
ancora attaccamento, non abbiamo ancora compreso. Ci sono ancora
sentimenti di invidia e di risentimento, siamo ancora attaccati alle
cose che ci causano sofferenza.

Il tipo di noia di cui parlò il Buddha è una condizione priva di collera
o di brama. Sorge dall'aver visto che tutto è impermanente. Quando una
sensazione piacevole sorge nella mente, vediamo che non è durevole.
Questo è il tipo di noia al quale mi riferisco. Si chiama
\emph{nibbidā}, o disincanto. Significa che si è lontani dalle bramosie
e dalle passioni sensoriali. Vediamo che non c'è nulla che sia degno
d'essere desiderato. Non conta che le cose siano o meno in accordo con
quello che ci piace o che non ci piace, non ci identifichiamo con tutto
questo. Non vi attribuiamo alcun valore particolare.

Praticando in questo modo non diamo alle cose l'opportunità di crearci
delle difficoltà. Abbiamo visto la sofferenza e sappiamo che
identificarsi con gli stati mentali non fa sorgere alcuna reale
felicità. Causa attaccamento alla felicità e all'infelicità e
attaccamento a piacere e dispiacere, che sono proprio le cause della
sofferenza. Quando vi è attaccamento non abbiamo ancora un atteggiamento
equanime verso le cose. Alcuni stati mentali ci piacciono, e altri non
ci piacciono. Se proviamo ancora piacere e dispiacere, allora sia
felicità sia infelicità sono sofferenza. È questo genere di attaccamento
che causa sofferenza. Il Buddha insegnò che ogni cosa che ci causa
sofferenza è di per sé insoddisfacente.

\section{Le Quattro Nobili Verità}

Partendo da qui comprendiamo che l'insegnamento del Buddha consiste nel
conoscere la sofferenza e nel conoscere ciò che causa il sorgere di
essa. Dovremmo poi conoscere la Libertà dalla sofferenza e il Sentiero
della pratica che conduce alla Libertà. Egli ci insegnò a conoscere solo
queste quattro cose. Quando comprenderemo queste quattro cose, saremo in
grado di riconoscere la sofferenza quando sorge e di sapere che essa ha
una causa. Sapremo che essa non è giunta così per conto suo, alla
deriva! Quando desidereremo essere liberi da questa sofferenza, saremo
in grado di eliminarne la causa.

Perché proviamo questa sensazione di sofferenza, questa sensazione
d'insoddisfazione? Capiremo che è perché ci stiamo attaccando a quello
che ci piace o a quello che non ci piace. Perveniamo a conoscere che
stiamo soffrendo a causa delle nostre stesse azioni. Soffriamo perché
attribuiamo valore alle cose. Per questo motivo parliamo di conoscere la
sofferenza, di conoscere la causa della sofferenza, di conoscere la
libertà dalla sofferenza e conoscere il Sentiero che conduce a questa
libertà. Quando c'è conoscenza della sofferenza, iniziamo a sciogliere
il nodo. Dobbiamo però essere certi di scioglierlo tirando la corda
nella giusta direzione. Ciò è per dire che dobbiamo sapere che così
stanno le cose. L'attaccamento sarà sradicato. Questa è la pratica che
pone fine alla nostra sofferenza.

Conoscere la sofferenza, conoscere la causa della sofferenza, conoscere
la libertà dalla sofferenza e conoscere il Sentiero che conduce fuori
dalla sofferenza. Questo è il \emph{magga}.\footnote{\emph{Magga}:
  Sentiero. Più specificamente il Sentiero verso la cessazione della
  sofferenza e della tensione.} Avanza in questo modo: Retta Visione,
Retta Intenzione, Retta Parola, Retta Azione, Retto Modo di Vivere,
Retto Sforzo, Retta Presenza Mentale, Retta Concentrazione. Quando
abbiamo retta comprensione a riguardo di queste cose, allora siamo sul
Sentiero. Sono cose che possono porre fine alla sofferenza. Ci conducono
alla moralità, alla concentrazione e alla saggezza (\emph{sīla},
\emph{samādhi}, \emph{paññā}).

Dobbiamo comprendere con chiarezza queste Quattro Verità. Dobbiamo voler
capire. Dobbiamo volerle vedere in termini di realtà. Quando vediamo
queste Quattro Verità, questo è \emph{saccadhamma}. Se guardiamo dentro,
di fronte, a destra o a sinistra, tutto ciò che vediamo è
\emph{saccadhamma}. Ogni cosa la vediamo semplicemente così com'è. Per
chi è giunto al Dhamma, per chi davvero comprende il Dhamma, ovunque
vada tutto sarà Dhamma.

