\chapter{Il Sentiero in armonia}

\begin{openingQuote}
  \centering

  Fusione di due discorsi offerti in Inghilterra rispettivamente\\
  nel 1979 e nel~1977.
\end{openingQuote}

Oggi vorrei chiedere a tutti voi una cosa. «~Siete sicuri, siete certi
della vostra pratica di meditazione?~» Ve lo chiedo perché oggigiorno ci
sono molte persone che insegnano meditazione, sia monaci sia laici, e
temo che possiate vacillare, essere dubbiosi. Se comprendiamo con
chiarezza, saremo in grado di rendere la nostra mente serena e stabile.
Dovreste intendere il Nobile Ottuplice Sentiero come moralità,
concentrazione e saggezza. Il Sentiero si unifica in questo semplice
modo. La nostra pratica consiste nel far sorgere questo Sentiero dentro
di noi.

Quando sediamo in meditazione ci viene detto di chiudere gli occhi, di
non guardare nient'altro, perché stiamo per osservare in modo diretto la
mente. Quando chiudiamo gli occhi, l'attenzione si rivolge verso
l'interno. Fissiamo la nostra attenzione sul respiro, incentriamo lì le
nostre sensazioni, la nostra consapevolezza. Quando i fattori del
Sentiero saranno in armonia, saremo in grado di vedere il respiro, le
sensazioni, la mente e gli oggetti mentali per quello che sono. Vedremo
lì il ``punto focale'', ove il \emph{samādhi} e gli altri fattori del
sentiero convergono armonicamente.

Allorché sedete in meditazione e seguite il respiro, dite a voi stessi
che state sedendo da soli. Non c'è nessun altro seduto attorno a voi,
non c'è assolutamente nient'altro. Sviluppate questa sensazione di
essere seduti da soli finché la mente lascia andare tutte le cose
esteriori e si concentra solo sul respiro. Se state pensando: «~Quella
persona sta seduta là, quell'altra sta seduta lì~» non c'è alcuna pace,
la mente non va verso l'interno. Mettete tutto questo da parte, fino a
quando avete la percezione che non ci sia nessuno seduto attorno a voi,
fino a che non c'è assolutamente nulla, fino a che la mente non vacilla
e non nutrite alcun interesse per l'ambiente circostante.

Lasciate che il respiro continui con naturalezza, non forzatelo a essere
corto o lungo o in qualsiasi altro modo, state solo seduti e osservatelo
mentre entra ed esce. Quando la mente lascerà andare tutte le
impressioni mentali esterne, il rumore delle automobili e altre cose di
questo genere non vi disturberanno più. Nulla, che si tratti di oggetti
visivi o di suoni, vi disturba, perché la mente non li riceve. La vostra
attenzione si concentrerà sul respiro. Se la mente è confusa e non si
concentra sulla respirazione, fate un respiro lungo e profondo, più
profondo che potete, e lasciate uscire tutta l'aria fino a che non ce
n'è più. Fatelo per tre volte, poi ripristinate la vostra attenzione. La
mente si calmerà. È naturale che sia calma per un po', e che in seguito
sorgano di nuovo irrequietezza e confusione. Quando succede,
concentratevi, respirate ancora una volta profondamente e riportate la
vostra attenzione sul respiro. Andate avanti così. Quando lo farete
molte volte, diverrete abili. La mente lascerà andare tutte le
manifestazioni esteriori. Le impressioni esterne non raggiungeranno la
mente. \emph{Sati} si instaurerà con saldezza.

Quando la mente si affina di più, altrettanto avviene con il respiro. Le
sensazioni diventeranno sempre più sottili, il corpo e la mente saranno
leggeri. La nostra attenzione si rivolgerà solo all'interno. Vedremo con
chiarezza le inspirazioni e le espirazioni, vedremo con chiarezza tutte
le impressioni mentali. Vedremo riunirsi qui moralità, concentrazione e
saggezza. Questo si chiama Sentiero in armonia. Quando ci sarà
quest'armonia la nostra mente sarà libera dalla confusione, si
unificherà. Questo si chiama \emph{samādhi}.

Dopo aver osservato il respiro a lungo, esso diverrà davvero sottile.
Gradualmente la consapevolezza del respiro cesserà e rimarrà la nuda
consapevolezza. Il respiro si può assottigliare fino al punto di
scomparire! Forse ``sediamo solo'', come se il respiro non ci fosse
affatto. In realtà il respiro c'è, ma sembra che non ci sia. È perché la
mente ha raggiunto la sua condizione più affinata, vi è solo la nuda
consapevolezza. È andata oltre il respiro. S'instaura la conoscenza che
il respiro è scomparso. Ora che cosa assumeremo come oggetto di
meditazione? Assumeremo come nostro oggetto di meditazione proprio
questa conoscenza, la consapevolezza che il respiro non c'è.

A questo punto possono accadere cose inattese. Alcuni le sperimentano,
altri no. Se sorgono, dovremmo essere saldi e avere una forte
consapevolezza. Alcuni vedono che il respiro è scomparso e si
spaventano, temono di morire. Ora dovremmo conoscere la situazione così
com'è. Notiamo semplicemente che non c'è il respiro ed è questo che
assumiamo come oggetto della nostra consapevolezza. Possiamo dire che
questo è il tipo più stabile e sicuro di \emph{samādhi:} un solo stabile
e immobile stato della mente. Forse il corpo diverrà così leggero che
sarà come se non ci sia affatto. Avremo la sensazione di stare seduti
nello spazio vuoto, vuoto del tutto. Benché ciò possa sembrare insolito,
dovreste capire che non c'è nulla di cui preoccuparsi e, così, rendere
la vostra mente stabile.

Quando la mente è unificata con fermezza, senza che ci siano impressioni
provenienti dai sensi a disturbarla, si può rimanere in questo stato per
tutto il tempo che si vuole. Non saremo disturbati da sensazioni
dolorose. Quando il \emph{samādhi} raggiunge questo livello, possiamo
uscirne quando vogliamo, ma se ne usciamo lo facciamo con un senso di
benessere, non perché siamo annoiati o stanchi. Ne usciamo perché per il
momento è sufficiente così, ci sentiamo a nostro agio, non abbiamo alcun
problema. Se riusciamo a sviluppare questo tipo di \emph{samādhi},
allora se sediamo in meditazione, diciamo, per una trentina di minuti o
per un'ora, la mente sarà tranquilla e serena per molti giorni. Quando
la mente è tranquilla e serena in questo modo, è pulita. Qualsiasi cosa
sperimenteremo, la mente la prenderà e la investigherà. Questo è un
frutto del \emph{samādhi}.

La moralità ha una funzione, la concentrazione ne ha un'altra e la
saggezza un'altra ancora. È come se questi fattori rappresentassero un
ciclo. Possiamo vederli tutti all'interno della mente pacificata. Quando
la mente è serena, è raccolta e contenuta a causa della saggezza e
dell'energia della concentrazione. Quando diventa più raccolta, diventa
anche più affinata, ciò che a sua volta dà alla moralità la forza di
crescere in purezza. Quando la nostra moralità diverrà più pura, ciò
aiuterà lo sviluppo della concentrazione. Quando la concentrazione si
insedierà con fermezza, ciò aiuterà la saggezza a sorgere. Moralità,
concentrazione e saggezza si assistono a vicenda, sono correlate in
questo modo. Infine il Sentiero si unifica, ed è sempre in funzione.
Dovremmo custodire l'energia che sorge dal Sentiero, perché è l'energia
che conduce alla visione profonda e alla saggezza.

\section{I pericoli del samādhi}

Per il meditante il \emph{samādhi} può essere molto dannoso o molto
benefico. Non è possibile dire che sia solo una o solo l'altra cosa. Per
chi non ha saggezza è dannoso, per chi ha saggezza può essere di reale
beneficio e può condurre alla visione profonda. Quel che può essere
dannoso per il meditante è l'assorbimento meditativo del \emph{samādhi}
(\emph{jhāna}), il \emph{samādhi} con profonda e sostenuta tranquillità.
Questo \emph{samādhi} porta molta serenità. Quando c'è serenità, c'è
felicità, e quando c'è felicità sorge l'attaccamento, l'aggrapparsi a
tale felicità. Il meditante non vuole contemplare nient'altro, vuole
solo indulgere a questa piacevole sensazione. Quando abbiamo praticato
per lungo tempo, possiamo avere l'abilità di entrare in questo
\emph{samādhi} molto velocemente. Appena iniziamo a percepire il nostro
oggetto di meditazione, la mente entra nella calma e non vogliamo
uscirne per investigare alcunché. Restiamo solo bloccati in quella
felicità. Per chi pratica la meditazione, questo è un pericolo.

Dobbiamo utilizzare l'\emph{upacāra samādhi:}\footnote{\emph{upacāra samādhi.}
  ``Concentrazione di accesso''; un livello di concentrazione precedente
  il \emph{jhāna}.} entriamo nella calma e quando la mente è
sufficientemente serena, ne usciamo e osserviamo l'attività
esteriore.\footnote{Con ``attività esteriore'' si intende qualsiasi
  modalità d'impressione sensoriale. È utilizzato in contrapposizione
  all'``inattività interiore'' nell'assorbimento del \emph{samādhi}
  (\emph{jhāna}), nella quale la mente non ``esce'' verso le impressioni
  esterne dei sensi.} Osservare l'esterno con la mente calma fa sorgere
la saggezza. È difficile da comprendere, perché è abbastanza simile al
modo ordinario di pensare e di immaginare. Quando il pensiero è
presente, possiamo pensare che la mente non sia serena, ma in verità
quel pensiero ha luogo nel contesto della calma. C'è contemplazione, ma
ciò non disturba la calma. Possiamo scegliere un pensiero per
contemplarlo. In questo caso scegliamo un pensiero per investigarlo, non
è che vaghiamo nei pensieri senza meta, non ci allontaniamo con le
congetture. Si tratta di una cosa che sorge in una mente serena. Si
chiama ``presenza mentale nella calma e calma nella presenza mentale''.
Se si trattasse di pensiero e di immaginazione ordinari, la mente non
sarebbe calma, sarebbe turbata. Ora non sto parlando del pensiero
ordinario, ma di una sensazione che sorge dalla mente serena. È detta
``contemplazione''. È proprio qui che nasce la saggezza.

Ci può essere un \emph{samādhi} giusto e un \emph{samādhi} errato. Il
\emph{samādhi} è errato quando la mente entra nella calma ma senza che
ci sia alcuna consapevolezza. Si può stare seduti per due ore o anche
per tutto il giorno senza che la mente sappia dov'è stata o cosa sia
avvenuto. Non sa nulla. C'è calma, ma questo è tutto. È come un coltello
ben affilato che non ci importa di usare in alcun modo. Si tratta di un
genere ingannevole di calma, perché non c'è molta consapevolezza di sé.
Il meditante può pensare di aver già raggiunto la meta suprema, e così
non si preoccupa di cercare altro. A questo livello il \emph{samādhi}
può essere un nemico. La saggezza non può sorgere, perché non c'è
consapevolezza di quello che è giusto e di quello che è sbagliato. Con
il corretto \emph{samādhi}, quale che sia il livello di calma raggiunto,
c'è consapevolezza. C'è piena consapevolezza e chiara comprensione.
Questo è il \emph{samādhi} che può far sorgere la saggezza, non ci si
può perdere in esso. I praticanti dovrebbero capirlo bene. Non si può
fare a meno di questa consapevolezza, deve essere presente dall'inizio
alla fine. Non ci sono pericoli in questo tipo di \emph{samādhi}.

Potreste chiedervi: dov'è che sorge il giovamento, come sorge la
saggezza dal \emph{samādhi}? Quando si è sviluppato il giusto
\emph{samādhi}, la saggezza può sorgere in qualsiasi momento. Quando gli
occhi vedono una forma, gli orecchi odono un suono, il naso sente degli
odori, la lingua sperimenta dei sapori, il corpo esperisce un contatto e
la mente sperimenta delle impressioni mentali, in tutte le posture la
mente rimane con la piena conoscenza della vera natura di queste
impressioni dei sensi, non le segue. Quando la mente ha saggezza non
``prende e sceglie''. In ogni postura si è completamente consapevoli
della nascita della felicità e dell'infelicità. Noi queste cose le
lasciamo andare entrambe, non ci attacchiamo. Questa è chiamata retta
pratica, quella presente in tutte le posture. Tali parole, ``tutte le
posture'', non si riferiscono alle sole posture del corpo, si
riferiscono alla mente, che ha sempre consapevolezza e chiara
comprensione della Verità. Quando il \emph{samādhi} è stato rettamente
sviluppato, sorge una saggezza come questa. È detta ``visione
profonda'', conoscenza della Verità.

Ci sono due generi di pace, una grossolana e l'altra sottile. La pace
che proviene dal \emph{samādhi} è di genere grossolano. Quando la mente
è serena, c'è felicità. La mente ritiene allora che questa felicità sia
la pace. Però, felicità e infelicità sono divenire e nascita. Non c'è
possibilità di fuga dal \emph{saṃsāra} se ci attacchiamo ancora a esse.
Per questo la felicità non è pace, l'infelicità non è pace. L'altro
genere di pace proviene dalla saggezza. In questo caso non confondiamo
la pace con la felicità. Conosciamo la mente che contempla e conosce la
felicità e l'infelicità come pace. La pace che sorge dalla saggezza non
è felicità, ma è ciò che vede la verità sia nella felicità sia
nell'infelicità. Non sorge l'attaccamento a quegli stati, la mente va al
di sopra di essi. Questo è il vero scopo della pratica buddhista.

