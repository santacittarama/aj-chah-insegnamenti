\chapter{Vivere con un cobra}

Questo breve discorso è a beneficio di una nuova discepola che presto
tornerà a Londra. Che possa aiutarti a comprendere l'insegnamento
studiato qui al Wat Pah Pong. Più semplicemente, questa è la pratica per
essere liberi dalla sofferenza del ciclo della nascita e della morte.

Per svolgere questa pratica, ricordati di fare attenzione a tutte le
varie attività della mente, tutte quelle che ti piacciono e tutte quelle
che non ti piacciono, nello stesso modo in cui faresti attenzione a un
cobra. Il cobra è un serpente estremamente velenoso, abbastanza velenoso
da causare la morte, se ci morde. È la stessa cosa con i nostri stati
mentali. Gli stati mentali che ci piacciono sono velenosi, gli stati
mentali che non ci piacciono sono pure velenosi. Impediscono alla nostra
mente di essere libera e ostacolano la nostra comprensione della Verità
così come essa fu insegnata dal Buddha.

È perciò necessario cercare di mantenere la nostra consapevolezza
sempre, di giorno e di notte. Qualsiasi cosa tu stia facendo, che tu
stia in piedi, seduta o distesa, che tu stia parlando o qualsiasi altra
cosa tu stia facendo, dovresti farla con consapevolezza. Quando sarai in
grado di instaurare questa presenza mentale, vedrai che assieme a essa
sorgerà la chiara comprensione, e queste due condizioni mentali
porteranno saggezza. Così, consapevolezza, chiara comprensione e
saggezza lavoreranno insieme, e tu sarai come chi è ``sveglio'' sia di
giorno che di notte.

Questi insegnamenti lasciati dal Buddha non devono essere solo
ascoltati, oppure assimilati a livello intellettuale. Sono insegnamenti
che, attraverso la pratica, possono essere fatti sorgere ed essere
conosciuti nei nostri cuori. Ovunque andiamo, qualsiasi cosa facciamo,
dovremmo avere questi insegnamenti con noi. E con ``avere questi
insegnamenti con noi'' o ``avere la verità con noi'' intendiamo che
qualsiasi cosa facciamo o diciamo, la facciamo e la diciamo con
saggezza. Quando pensiamo e contempliamo, lo facciamo con saggezza.
Diciamo che chiunque possiede consapevolezza e chiara comprensione fuse
in questo modo con la saggezza, è vicino al Buddha. Quando lascerai
questo posto, dovresti praticare riconducendo tutto alla tua stessa
mente. Osserva la tua mente con questa consapevolezza e con questa
chiara comprensione, e sviluppa questa saggezza. Queste tre condizioni
faranno sorgere un ``lasciar andare''. Conoscerai il costante sorgere e
svanire di tutti i fenomeni.

Dovresti sapere che ciò che sorge e svanisce è solo l'attività della
mente. Quando qualcosa sorge, poi svanisce, ed è seguita da ulteriore
sorgere e svanire. Sulla Via del Dhamma chiamiamo questo sorgere e
svanire ``nascita e morte''. E questo è tutto, tutto quel che c'è!
Quando la sofferenza è sorta, svanisce, e quando la sofferenza è
svanita, poi sorge di nuovo.\footnote{In questo contesto, la sofferenza
  è intesa come l'implicito carattere insoddisfacente di ogni fenomeno
  composto esistente, un carattere differente dalla sofferenza intesa
  quale semplice opposto della felicità.} C'è solo sofferenza che sorge
e svanisce. Quando vedrai le cose così, sarai in grado di conoscere
continuamente questo sorgere e svanire. Quando la tua conoscenza sarà
costante, vedrai che questo è davvero tutto quel che c'è. Tutto è solo
nascita e morte. Non è che ci sia un responsabile di tutto questo. C'è
solo questo sorgere e svanire così com'è, questo è quanto.

Questo modo di vedere farà nascere una sensazione di sereno disincanto
nei riguardi del mondo. Una sensazione di questo genere sorge quando
comprendiamo che, in verità, non c'è nulla che valga la pena di
desiderare. C'è solo sorgere e svanire, un essere nati e poi morire. È
così quando la mente giunge al ``lasciar andare'', lasciar andare tutto
secondo la sua propria natura. Le cose sorgono e svaniscono nella nostra
mente, e lo sappiamo. Quando la felicità sorge, lo sappiamo. Quando la
scontentezza sorge, lo sappiamo. E questo ``conoscere la felicità''
significa che non ci identifichiamo con essa come se fosse nostra. Allo
stesso modo non ci identifichiamo con la scontentezza e l'infelicità
come se fossero nostre. Quando non ci identifichiamo più con la felicità
e con la sofferenza e non ci attacchiamo a esse, stiamo semplicemente
con il modo naturale in cui sono le cose.

Per questo diciamo che l'attività mentale è come un cobra mortalmente
velenoso. Se non lo ostacoliamo, il cobra si limiterà ad andarsente per
la sua strada. Benché sia estremamente velenoso, non ne risentiamo. Se
non ci avviciniamo né lo afferriamo, non ci morderà. Il cobra fa quello
che è naturale fare per un cobra. Così stanno le cose. Se siete
intelligenti, lo lascerete stare. Lasciate che quel che non è buono sia,
lasciate che sia secondo la sua natura. Lasciate che pure quel che è
buono sia. Lasciate esistere ciò che vi piace e ciò che non vi piace,
allo stesso modo in cui non ostacoliamo un cobra.

Chi è intelligente avrà questo atteggiamento nei riguardi dei vari stati
mentali che sorgono nella mente. Quando sorge il benessere, lasciamo che
tale sia, ma lo sappiamo. Ne comprendiamo la natura. Così, allo stesso
modo, lasciamo che ci sia ciò che buono non è, lo lasciamo esistere
secondo la sua natura. Non lo afferriamo perché non vogliamo alcunché.
Non vogliamo il male e non vogliamo neanche il bene. Non vogliamo né
pesantezza né leggerezza, né felicità né sofferenza. In questo modo,
quando i desideri giungono al termine, la pace è stabilmente insediata.

Quando questo tipo di pace si è stabilmente insediata nella nostra
mente, possiamo farvi affidamento. Questa pace, diciamo, è sorta dalla
confusione. La confusione è terminata. Il Buddha definì il conseguimento
dell'Illuminazione definitiva una ``estinzione''. È come estinguere un
fuoco. Estinguiamo il fuoco nello stesso luogo in cui esso compare.
Quale che sia il luogo in cui divampa, è proprio lì che possiamo
raffreddarlo. È così anche con l'Illuminazione. Il \emph{Nibbāna} si
trova nel \emph{saṃsāra}.\footnote{\emph{Saṃsāra}: Flusso del Divenire o
  dell'Esistenza; un vagare perpetuo, il continuo processo del nascere,
  invecchiare e morire.} Illuminazione e illusione si trovano nello
stesso posto, proprio come avviene per il caldo e il freddo. È caldo
dove era freddo ed è freddo dove era caldo. Quando il calore sorge,
scompare la frescura, e quando lì c'è frescura, allora non c'è più
calore. In questo senso il \emph{Nibbāna} e il \emph{saṃsāra} sono
uguali.

Ci è stato detto di porre fine al \emph{saṃsāra}, il che significa
fermare il ciclico cerchio, sempre in moto, della confusione. Si pone
fine alla confusione e si estingue il fuoco. Quando il fuoco esteriore è
estinto, c'è la frescura. Quando il fuoco interiore della bramosia dei
sensi, dell'avversione e dell'illusione sono spenti, anche questa è
frescura. Questa è la natura dell'Illuminazione, è l'estinzione del
fuoco, il raffreddamento di ciò che arde. Questa è la pace. Questa è la
fine del \emph{saṃsāra}, il ciclo della nascita e della morte. Quando si
giunge all'Illuminazione, ecco com'è. È la fine del continuo girare in
tondo e del continuo cambiamento, nella nostra mente è la fine
dell'avidità, dell'avversione e dell'illusione. Ne parliamo in termini
di felicità perché è così che la gente pensa che sia, ma in realtà si è
andati al di là. È al di là sia della felicità sia della sofferenza. È
la pace perfetta.

Quando te ne sarai andata dovresti prendere questo insegnamento che ti
ho dato e contemplarlo con attenzione. La tua permanenza qui non è stata
facile e ho avuto poche opportunità per darti istruzioni, ma in questo
periodo sei stata in grado di studiare il vero significato della nostra
pratica. Che questa pratica possa condurti alla felicità e aiutarti a
crescere nella Verità, e che possa liberarti dalla sofferenza della
nascita e della morte.

