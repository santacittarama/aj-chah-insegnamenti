\chapter{Foreword}

Gli insegnamenti di Ajahn Chah erano disarmanti per la loro immediatezza
e stimolanti per la loro rilevanza. Egli avrebbe detto: «~Se lasciate
andare un po', avrete un po' di pace. Se lasciate andare molto, avrete
molta pace. E se lasciate andare del tutto, avrete una pace totale.~»

Stare vicino a lui significava essere vicino al miglior amico possibile.
Quando eravamo maldestri o sbagliavamo non rideva di noi, rideva con
noi. Quando stavamo soffrendo per i dubbi, non ci rimproverava per
mancanza di fede, ma ci parlava dei tempi in cui lui stesso aveva
dubitato così tanto da pensare che la sua testa sarebbe scoppiata. E se
voleva ispirarci diligenza nella pratica, si sedeva in meditazione con
noi, recitava i canti con noi e lavorava con noi. Il nostro inciampare e
annaspare non erano mai giudicati, ma considerati in un modo che dava
dignità ai nostri sforzi, non disperazione. Gli incoraggiamenti di Ajahn
Chah a lasciar andare non erano né una tecnica né un toccasana; si
trattava piuttosto di condividere la luce che aveva trovato nella sua
pratica, per far sì che anche noi potessimo trovare la direzione verso
la libertà dalla sofferenza.

Osservando la mole di questa edizione, i lettori potrebbero
meravigliarsi del fatto che, sebbene gli insegnamenti fossero stati così
semplici, siano state necessarie così tante parole per esprimerli.
Questo è dovuto al fatto che siamo in grado di generare confusione in
numerosissimi modi. Ajahn Chah conosceva il luogo della pace perfetta ed
era contento di dimorarvi. Egli, però, era anche instancabile nei suoi
sforzi di guidare gli altri. Vivendo con lui, a volte pareva che ci
fosse indicato quel luogo di benessere, un invito a godere i frutti
della pratica. Più spesso era come se lui stesse percorrendo la strada
al nostro fianco.

Come vedrete, questi insegnamenti non sono un manuale di buddhismo. Qui
non troverete neanche le soluzioni a tutti i vostri problemi. Gli
insegnamenti di Ajahn Chah mirano a metterci in contatto con le nostre
domande più profonde e ad aiutarci ad ascoltarle, pazientemente e
gentilmente, fino a quando non si rivela la via da seguire.

I discorsi presenti in questa raccolta sono stati registrati, trascritti
e tradotti molti anni fa e sono in un certo qual modo lontani dalla loro
fonte. Ovviamente, se letti con un cuore ricettivo e con una mente
raccolta, queste ``indicazioni'' verso la Verità forniranno ispirazione
e istruzioni preziose. L'umiltà, la gioia e la saggezza di Ajahn Chah
risplendono nelle sue parole, illuminando il Sentiero mentre lo
percorriamo. Sono trascorsi quasi vent'anni da quando Ajahn Chah è
morto. Grazie al supporto di sponsor generosi, abbiamo colto questa
opportunità di mettere insieme tutti i discorsi disponibili per
distribuzione gratuita e di presentarli in una forma che speriamo possa
essere facilmente accessibile a tutti coloro che si sentono attratti
dalla pace.

\bigskip

{\raggedleft
  Ajahn Munindo,\\
  aprile 2011
\par}

