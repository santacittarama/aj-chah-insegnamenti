\chapter{Seduta serale}

Mi piacerebbe chiedervi della vostra pratica. Tutti siete stati qui a
praticare meditazione, ma siete sicuri della vostra pratica? Chiedetelo
a voi stessi. Siete fiduciosi nei riguardi della vostra pratica?
Oggigiorno vanno in giro insegnanti di meditazione di ogni genere, sia
monaci sia laici, e temo che ciò possa indurvi ad avere molti dubbi e
incertezze su quel che state facendo. È per questa ragione che ve lo
chiedo. Per quanto concerne la pratica buddhista, non vi è nulla di più
grande o elevato di questi insegnamenti del Buddha grazie ai quali avete
praticato qui. Se li comprendete con chiarezza, questo farà sorgere nel
vostro cuore e nella vostra mente una pace assolutamente stabile e
incrollabile.

Rendere la mente serena è una pratica nota come meditazione, come
\emph{samādhi}. La mente è estremamente mutevole e inaffidabile. Se
osservate come avete praticato fino ad ora, riuscite a notarlo? Alcuni
giorni praticate la meditazione seduta e in un attimo la mente è calma,
altri giorni vi mettete seduti e, qualsiasi cosa facciate, non si calma:
la mente si dibatte costantemente per scappare via, fino a che non ci
riesce. Alcuni giorni va bene, altri giorni è terribile. È cosi che la
mente mostra le sue differenti condizioni affinché le possiate vedere.
Dovete capire che gli otto fattori del Nobile Ottuplice
Sentiero\footnote{Nobile Ottuplice Sentiero: Gli otto fattori che
  conducono alla fine della sofferenza; tali fattori sono elencati nel
  \emph{Glossario}, p. \pageref{glossary-ottuplice}.} si fondono in \emph{sīla}
(moralità), \emph{samādhi} (concentrazione) e \emph{paññā} (saggezza). Non si incontrano da
nessun'altra parte. Questo significa che quando mettete insieme i
fattori della vostra pratica, ci deve essere \emph{sīla}, ci deve essere
\emph{samādhi} e ci deve essere \emph{paññā}; devono essere presenti
insieme nella vostra mente. Questo significa che praticando la
meditazione proprio qui e ora, state generando in modo davvero diretto
le cause per far sorgere il Sentiero.

Per la meditazione seduta vi è stato insegnato a chiudere gli occhi,
così che non passiate il vostro tempo a guardare cose di vario genere.
Il Buddha insegnò che dovreste conoscere la vostra stessa mente.
Osservate la mente. Se chiudete gli occhi, la vostra attenzione si
rivolgerà naturalmente all'interno, verso la mente, la fonte di
differenti generi di conoscenza. Questo è un modo di addestrare la mente
che fa sorgere il \emph{samādhi}.

Appena vi sedete a occhi chiusi, fissate la vostra presenza mentale sul
respiro, rendete la consapevolezza del respiro la cosa più importante.
Questo significa che applicate la vostra presenza mentale nel seguire il
respiro e, continuando in questo modo, conoscerete il punto in cui si
focalizza \emph{sati},\footnote{\emph{Sati.} Consapevolezza, presenza
  mentale, attenzione; il termine, molto importante nella pratica
  meditativa buddhista, può significare anche ``memoria''.} il punto
focale del conoscere e della consapevolezza della mente. Ogni volta che
questi fattori del Sentiero lavorano assieme sarete in grado di
osservare e vedere il vostro respiro, le vostre sensazioni, la vostra
mente e l'\emph{ārammaṇa},\footnote{\emph{Ārammaṇa.} Oggetto mentale,
  oggetto di riferimento di un metodo meditativo.} così come sono nel
momento presente. Conoscerete infine sia il punto focale del
\emph{samādhi} sia il punto di unificazione dei fattori del Sentiero.

Quando sviluppate il \emph{samādhi}, fissate l'attenzione sul respiro e
immaginate che stiate sedendo da soli, assolutamente senza niente e
nessuno attorno che possa disturbarvi. Sviluppate questa percezione
nella mente, sostenendola fino a che la mente non lascia completamente
andare il mondo esteriore, e resta solo la conoscenza del respiro che
entra ed esce. La mente deve mettere da parte il mondo esterno. Non
consentitevi di cominciare a pensare a quella persona seduta lì o a
quell'altra seduta là. Non lasciate spazio ad alcun pensiero che faccia
sorgere confusione o agitazione nella mente, è meglio sbatterli fuori e
farla finita con essi. Qui non c'è nessun altro, state sedendo da soli.
Sviluppate questa percezione fino a che tutti gli altri ricordi, tutte
le altre percezioni e tutti gli altri pensieri su persone e cose
cessano, e non vi sono più questioni e domande su persone e cose che vi
stanno attorno. Potete allora fissare la vostra attenzione unicamente
sull'inspirazione e sull'espirazione. Respirate normalmente. Consentite
che l'inspirazione e l'espirazione continuino con naturalezza, senza
forzarle a essere più lunghe o più corte, più forti o più deboli del
normale. Consentite che il respiro continui normale ed equilibrato, e
poi continuate a stare seduti a osservarlo mentre entra nel corpo e
mentre lo lascia.

Quando la mente avrà lasciato andare gli oggetti mentali esterni, non
sarete più disturbati dal traffico o da altri rumori. Non sarete
irritati da alcuna cosa esterna. Che si tratti di forme, suoni o altro,
non vi saranno fonti di disturbo, perché la mente non vi presterà
attenzione: sarà centrata sul respiro.

Se la mente è agitata da varie cose e non riuscite a concentrarvi,
provate a inspirare molto profondamente fino a che i polmoni non sono
completamente pieni, e poi espirate tutta l'aria fino a quando non ne
contengono più. Fatelo molte volte, poi ripristinate la presenza mentale
e continuate a sviluppare la concentrazione. Riacquistata la
consapevolezza, è normale che per un po' la mente resti calma, ma poi
cambierà e si agiterà nuovamente. Quando ciò avviene, fermate la mente,
prendete un altro respiro profondo e poi espellete l'aria dai polmoni.
Riempite di nuovo tutta la loro capacità per un attimo e ristabilite la
consapevolezza sul respiro. Fissate \emph{sati} sulle inspirazioni e
sulle espirazioni, e continuate a mantenere la consapevolezza in questo
modo.

La pratica tende a essere così, e per questo dovrete fare molte sedute e
tanti sforzi prima di diventare esperti. Quando lo sarete, la mente
lascerà andare il mondo esterno e resterà indisturbata. Dall'esterno gli
oggetti mentali non saranno in grado di penetrare all'interno e di
disturbare la mente. Quando non saranno in grado di penetrare
all'interno, vedrete la mente. Vedrete la mente quale oggetto di
consapevolezza, il respiro come un altro oggetto e i contenuti mentali
come un altro ancora. Saranno tutti presenti all'interno del campo della
vostra consapevolezza, centrati sulla punta del vostro naso. Una volta
che \emph{sati} è stabilizzata sull'inspirazione e sull'espirazione,
potete continuare a praticare a vostro agio. Quando la mente diventa
calma, anche il respiro, che all'inizio era grossolano, diventa più
leggero e affinato. Pure il contenuto mentale diviene sempre più sottile
e affinato. Il corpo si sente più leggero e, progressivamente, anche la
mente si sente più leggera e meno gravata. La mente lascia andare gli
oggetti mentali esterni e voi continuate a osservare internamente.

Da questo momento in poi la vostra consapevolezza si terrà lontana dal
mondo esterno e sarà indirizzata all'interno, per focalizzarsi sulla
mente. Quando la mente si è raccolta ed è giunta alla concentrazione,
mantenete la consapevolezza sul punto in cui la mente si è focalizzata.
Mentre respirate, vedrete con chiarezza il respiro quando entra ed esce,
\emph{sati} diverrà acuta e la consapevolezza degli oggetti mentali e
dell'attività mentale sarà più chiara. A quel punto capirete le
caratteristiche di \emph{sīla}, \emph{samādhi} e \emph{paññā}, e il modo
in cui si fondono assieme. Si tratta della ``unificazione dei fattori
del Sentiero''. Quando questa unificazione si verificherà, la vostra
mente sarà libera da ogni forma di agitazione e di confusione. Diventerà
unificata, e questo è il \emph{samādhi}.

Focalizzando la vostra attenzione su una sola cosa, il respiro, la
chiarezza e la consapevolezza crescono a causa dell'ininterrotta
presenza di \emph{sati}. Man mano che continuate a vedere il respiro con
chiarezza, \emph{sati} diverrà più forte e la mente più sensibile in
molti modi diversi. Vedrete la mente unificata al centro di quel punto
-- il respiro -- e la consapevolezza sarà focalizzata all'interno,
piuttosto che rivolta all'esterno, verso il mondo. Il mondo scomparirà
gradualmente dalla vostra consapevolezza e la mente non andrà più a
svolgere alcuna attività esteriore. È come se foste entrati nella vostra
``casa'', dove tutte le facoltà dei vostri sensi si sono riunite per
formare un'unità compatta. Siete a vostro agio e la mente è libera da
tutti gli oggetti esterni. La consapevolezza rimane con il respiro e,
col tempo, penetrerà sempre più in profondità all'interno, diventando
progressivamente più raffinata.

Alla fine la consapevolezza del respiro è così raffinata che la
sensazione di esso sembra scomparire. Potreste dire sia che la
consapevolezza della sensazione del respiro è scomparsa sia che è
scomparso il respiro stesso. Sorge allora un nuovo tipo di
consapevolezza, la consapevolezza che il respiro è scomparso. In altre
parole, la consapevolezza del respiro diventa così raffinata che è
difficile definirla. Potrebbe perfino succedere che ve ne stiate lì
seduti e che il respiro sia assente. In realtà il respiro è ancora lì,
ma è diventato così sottile che sembra essere scomparso. Perché? Perché
la mente è al suo livello più sottile, con uno speciale genere di
conoscenza. Resta solo il conoscere. Sebbene il respiro sia svanito, la
mente è ancora concentrata sulla conoscenza che il respiro non c'è più.
Quando continuate, che cosa dovreste assumere quale oggetto di
meditazione? Prendete questa stessa conoscenza come oggetto di
meditazione -- ossia la conoscenza che non c'è respiro -- e mantenetela.
Potreste dire che nella mente si è instaurato uno specifico genere di
conoscenza.

A questo punto, in alcune persone potrebbero manifestarsi dei dubbi,
perché è ora che possono sorgere dei \emph{nimitta}.\footnote{%
  \emph{Nimitta:}
  Segno mentale, immagine o visione che può sorgere durante la
  meditazione.} Possono essere di vari tipi, sia forme sia suoni.
Durante la pratica può a questo punto sorgere ogni genere di cose
impreviste. Se sorge un \emph{nimitta} -- ad alcuni succede, ad altri no
-- dovete capirlo in accordo con la verità. Non dubitate, né
consentitevi di allarmarvi.

In questo stadio della pratica dovreste rendere la vostra mente
imperturbabile quanto a concentrazione e molto consapevole. Alcuni si
sorprendono quando notano che il respiro è scomparso, perché sono
abituati ad averlo con sé. Quando sembra che il respiro se ne sia andato
potreste entrare nel panico, spaventarvi, pensando che stiate per
morire. Adesso dovete avvalervi della comprensione che progredire in
questo modo è nella natura della pratica. Ora che cosa dovreste
osservare come oggetto di meditazione? Osservate proprio questa
sensazione, che non c'è il respiro, e sostenetela come oggetto di
consapevolezza mentre continuate a meditare. Il Buddha la descrisse come
la più stabile, la più imperturbabile forma di \emph{samādhi}. C'è solo
un fisso e saldo oggetto mentale. Quando la vostra pratica di
\emph{samādhi} raggiungerà questo punto, nella mente si verificheranno
molti cambiamenti strani e raffinati, delle trasformazioni di cui
potrete essere consapevoli. La sensazione del corpo sarà la più lieve
possibile, o potrà perfino sparire del tutto. Potreste sentirvi
fluttuare a mezz'aria, come se foste completamente privi di peso.
Potrebbe essere come trovarsi nello spazio e le vostre facoltà
sensoriali, ovunque siano indirizzate, non sembreranno registrare
assolutamente nulla. Pur sapendo che il vostro corpo è ancora seduto lì,
sperimenterete una vacuità totale. Questa sensazione di vacuità può
essere piuttosto strana.

Man mano che continuate a praticare, comprendete che non c'è nulla di
cui preoccuparsi. Rendete salda e stabile nella mente questa sensazione
di rilassamento e di assenza di preoccupazioni. Quando la mente sarà
concentrata e unificata, nessun oggetto mentale sarà in grado di
penetrare in essa o di disturbarla, e voi sarete in grado di sedere così
per tutto il tempo che vorrete. Potrete mantenere la concentrazione
senza alcuna sensazione di dolore o disagio. Dopo aver sviluppato il
\emph{samādhi} a questo livello, sarete in grado di entrarvi o uscirvi a
vostro piacimento. Quando lo abbandonerete, sarà perché pare e conviene
a voi. Ne uscirete perché lo volete, non in quanto pigri o stanchi.
Recederete dal \emph{samādhi} perché sarà il tempo giusto per farlo, e
ne uscirete per vostra volontà.

Questo è il \emph{samādhi}. Siete rilassati e a vostro agio. Ne entrate
e ne uscite senza alcun problema. La mente e il cuore sono a proprio
agio. Se avete davvero un \emph{samādhi} come questo, ciò significa che
fare meditazione seduta ed entrare in \emph{samādhi} solo per una
trentina di minuti o per un'ora vi consentirà di restare calmi e sereni
per molti giorni. Sperimentare gli effetti del \emph{samādhi} in questo
modo per più giorni purifica la mente e qualsiasi cosa sperimenterete
diverrà un oggetto di contemplazione. È da questo momento che la pratica
comincia davvero. È il frutto che sorge quando matura il \emph{samādhi}.

La funzione del \emph{samādhi} è calmare la mente. \emph{Samādhi} svolge
una funzione, \emph{sīla} ne svolge un'altra e \emph{paññā} ne svolge
un'altra ancora. Queste caratteristiche, sulle quali state focalizzando
l'attenzione e che state sviluppando nella pratica, sono legate, formano
un cerchio. È il modo in cui si manifestano nella mente. \emph{Sīla,}
\emph{samādhi} e \emph{paññā} sorgono e maturano nello stesso posto.
Quando la mente si calmerà, diverrà progressivamente più contenuta e
composta per la presenza di \emph{paññā} e per l'energia del
\emph{samādhi}.

Allorché la mente diviene più composta e sottile, sorge un'energia che
agisce purificando \emph{sīla}. La maggior purezza di \emph{sīla}
facilita lo sviluppo di un \emph{samādhi} più forte e raffinato, e ciò a
sua volta supporta la maturazione di \emph{paññā}. Si assistono a
vicenda in questo modo. Ogni aspetto della pratica agisce quale fattore
di supporto per gli altri, e alla fine questi termini diventano
sinonimi. Tali tre fattori continuano a maturare assieme fino a formare
un cerchio completo, che infine fa sorgere \emph{magga}.\footnote{%
  \emph{Magga:}
  Sentiero. Più specificamente, il Sentiero verso la cessazione della
  sofferenza e della tensione.} \emph{Magga} è una sintesi di queste tre
funzioni della pratica che lavorano insieme, in modo lieve ma costante.
Quando praticate dovete preservare questa energia. È l'energia che farà
sorgere \emph{vipassanā}\footnote{\emph{Vipassanā.} Visione profonda di
  natura intuitiva dei fenomeni fisici e mentali del loro sorgere e
  scomparire.} o \emph{paññā}. Raggiunto questo stadio,
indipendentemente dal fatto che ci sia serenità o meno, \emph{paññā}
sarà già attiva nella mente e darà alla pratica un'energia costante e
indipendente. Se noterete che la mente non è serena, non dovreste
attaccarvi a questo stato, e altrettanto dovreste fare anche se lo è.
Avendo lasciato andare il fardello delle preoccupazioni, di conseguenza
il cuore si sentirà molto più leggero. Vi sentirete a vostro agio
sperimentando oggetti mentali sia piacevoli sia spiacevoli. La mente
resterà serena.

Un'altra cosa importante è capire che quando avete terminato di svolgere
la pratica meditativa formale, se non c'è saggezza in funzione nella
mente, interromperete del tutto la pratica e non ci sarà alcuna
ulteriore contemplazione né alcuno sviluppo della consapevolezza o
dell'interesse a proposito del lavoro che ancora resta da fare. Quando
uscite dal \emph{samādhi}, nella mente sapete con chiarezza che ne siete
usciti. Essendone usciti, dovreste continuare a comportarvi normalmente.
Conservate sempre la presenza mentale e la consapevolezza.
\emph{Samādhi} significa che la mente è salda e incrollabile, e non è
che la meditazione si pratica solo da seduti. Quando continuate la
vostra vita quotidiana, rendete la mente ferma e stabile e mantenete
sempre questo senso di saldezza come oggetto mentale. Dovete praticare
\emph{sati} e \emph{sampajañña}\footnote{\emph{Sampajañña.} ``Chiara
  comprensione'', consapevolezza di sé, autorammemorazione, attenzione,
  consapevolezza, presenza mentale, comprensione profonda.}
continuamente. Dopo esservi alzati dalla meditazione formale seduta,
quando svolgete le vostre attività -- camminare, guidare l'automobile e
così via -- se i vostri occhi vedono una forma o i vostri orecchi odono
qualcosa, mantenete la presenza mentale. Allorché sperimentate oggetti
mentali che fanno sorgere piacere o dispiacere, impegnatevi
costantemente a conservare la consapevolezza del fatto che questi stati
mentali sono impermanenti e incerti. In tal modo la mente resterà calma
e in una condizione di ``normalità''.

Finché la mente è calma usatela per contemplare gli oggetti mentali.
Contemplate l'insieme di questa forma, il corpo fisico. Potete farlo
sempre e in qualsiasi postura: mentre praticate la meditazione formale,
quando vi rilassate a casa, fuori al lavoro o in qualsiasi situazione vi
troviate. Mantenete sempre l'attitudine meditativa e riflessiva. Mentre
fate una passeggiata, anche vedere le foglie morte sul terreno ai piedi
di un albero può offrire l'opportunità di contemplare l'impermanenza.
Tra noi e le foglie non c'è differenza: quando si diventa vecchi si
avvizzisce e si muore. È lo stesso per tutti. Questo significa elevare
la mente al livello della \emph{vipassanā}, contemplare la verità di
come stanno le cose, sempre. Sia che camminiamo sia che stiamo in piedi,
seduti o distesi, \emph{sati} è ugualmente e costantemente supportata.
Questo è praticare correttamente la meditazione, dovete seguire la mente
da vicino, controllarla sempre.

Praticando qui e ora alle sette di sera, siamo stati seduti e abbiamo
fatto meditazione insieme per un'ora, e adesso abbiamo smesso. Potrebbe
essere che la vostra mente abbia completamente cessato di praticare e
che non abbia continuato a riflettere. È un modo di fare sbagliato.
Quando smettiamo, dovrebbero cessare solo l'incontro formale e la
meditazione seduta. Dovreste continuare a praticare e a sviluppare
costantemente la presenza mentale, senza mollare.

Insegno spesso che se non si pratica con costanza, si tratta solo di
gocce d'acqua. Gocce d'acqua perché la pratica non è un continuo e
ininterrotto fluire. \emph{Sati} non è sostenuta in modo uniforme. Il
punto importante è che la mente pratichi e non faccia altro. Il corpo
non pratica. È la mente a farlo, è la mente che pratica. Se lo
comprendete con chiarezza, vedrete che non dovete necessariamente
sedervi in meditazione formale perché la mente conosca il
\emph{samādhi}. È la mente che pratica. Dovete farne esperienza e
comprenderlo da voi stessi, nella vostra mente.

Non appena lo capirete da voi stessi, inizierete a sviluppare la
consapevolezza nella mente sempre e in ogni postura. Se state facendo
fluire \emph{sati} in modo costante e ininterrotto, è come se le gocce
assumessero la forma di un flusso d'acqua corrente dolce e continuo.
\emph{Sati} sarà presente nella mente di momento in momento e, di
conseguenza, ci sarà sempre consapevolezza degli oggetti mentali. Se
rendiamo la mente contenuta e composta senza interruzioni mediante
\emph{sati}, ogni volta che sorgono stati mentali salutari o non
salutari saprete quali sono gli oggetti mentali che li causano.
Conoscerete la mente che è calma e la mente che è confusa e agitata.
Praticherete in questo modo ovunque andiate. Se addestrerete così la
mente, la vostra meditazione maturerà velocemente e con profitto.

Per favore, ora non fraintendetemi. Di questi tempi è normale che la
gente vada a frequentare corsi di \emph{vipassanā} per tre o sette
giorni, durante i quali non si deve parlare né fare nessun'altra cosa
che non sia meditazione. Forse siete stati in un ritiro di meditazione
silenziosa per una settimana o due, dopo di che siete tornati alla
vostra normale vita quotidiana. Potreste essere andati via pensando di
aver ``fatto \emph{vipassanā}'' e, poiché vi sembrava ormai di sapere di
cosa si trattasse, avete continuato ad andare a feste, in discoteche e a
indulgere a diverse forme di piaceri sensoriali. Che cosa succede quando
vi comportate in questo modo? Alla fine, non resterà alcun frutto della
\emph{vipassanā}. Se andate a compiere ogni genere di azioni maldestre,
che disturbano e agitano la mente, sprecate i vostri precedenti sforzi.
L'anno successivo tornate di nuovo a fare un altro ritiro per sette
giorni o poche settimane, e poi andate via e continuate con feste,
discoteche e alcol. Questa non è vera pratica. Non è
\emph{paṭipadā},\footnote{\emph{Paṭipadā.} Strada, via, sentiero; i
  mezzi per raggiungere lo scopo o la destinazione finale, il
  Nibbāna.} il Sentiero per il progresso spirituale.

Dovete fare uno sforzo di rinuncia. Dovete contemplare fino a quando
capite gli effetti dannosi di tale comportamento. Capire il danno che
arrecano le bevande alcoliche e andare fuori, in città. Riflettete e
vedete il pericolo insito in tutti i vari tipi di comportamento
maldestro ai quali indulgete, finché questo pericolo diviene del tutto
evidente. Ciò dovrebbe spingervi a fare un passo indietro e a cambiare i
vostri modi di essere. Allora potreste trovare un po' di vera pace. Per
sperimentare la pace della mente dovete capire con chiarezza gli
svantaggi e i pericoli di questi comportamenti. Questo è praticare in
modo corretto. Se andate in ritiro silenzioso per sette giorni, dove non
si può parlare o essere coinvolti da nessuno, e poi vi mettete a
chiacchierare, spettegolare ed eccedere per altri sette mesi, come
potete ottenere un qualche reale o durevole beneficio da quei sette
giorni di pratica?

Vorrei incoraggiare tutti i laici che stanno praticando qui a sviluppare
la consapevolezza e la saggezza per comprendere tutto questo. Cercate di
praticare in modo costante. Guardate gli svantaggi della pratica priva
di costanza e di sincerità, e cercate di sostenere uno sforzo più mirato
e continuo. Tutto qui. Allora può esserci una realistica possibilità che
possiate eliminare i \emph{kilesa}.\footnote{\emph{Kilesa.}
  Contaminazione; inquinante mentale; fattore mentale che oscura e
  contamina la mente.} Ma quel modo di vivere senza parlare e senza
divertirsi per sette giorni, seguito da sette mesi di totale indulgenza
ai piaceri sensoriali, senza alcuna consapevolezza o moderazione,
condurrà solo a sprecare ogni progresso ottenuto dalla meditazione, non
resterà niente. È come andare a lavorare per un giorno e guadagnare
trenta euro e poi, nello stesso giorno, uscire e spenderne quaranta in
cibo e altre cose. Resterà qualche risparmio? Se ne andrà tutto. Lo
stesso avviene con la meditazione.

Questa è una sollecitazione per tutti voi, e perciò vi chiedo di
perdonarmi. È necessario parlare in questo modo, così che gli aspetti
sbagliati della pratica vi siano chiari e, di conseguenza, possiate
essere in grado di abbandonarli. Potreste dire che la ragione per cui
siete venuti a praticare è imparare come evitare di sbagliare in futuro.
Che cosa succede quando fate delle cose sbagliate? Fare cose sbagliate
vi porta ad agitazione e sofferenza, nella mente non c'è bontà. Non è la
via per la pace della mente. Così stanno le cose. Se praticate in un
ritiro, non parlate per sette giorni e poi andate a indulgere per un po'
di mesi, non conta quanto rigorosamente abbiate praticato quei sette
giorni, non otterrete alcun durevole vantaggio da quella pratica.
Praticando così, non andrete proprio da nessuna parte. In molti luoghi
in cui si insegna meditazione questo problema non viene preso in reale
considerazione oppure lo si trascura del tutto. Davvero, nella vostra
vita quotidiana dovete comportarvi sempre con calma e in modo contenuto.

Nella meditazione dovete costantemente rivolgere la vostra attenzione
alla pratica. È come piantare un albero. Se piantate un albero in un
posto e dopo tre giorni lo sradicate e lo piantate altrove, e poi dopo
altri tre giorni lo sradicate di nuovo e lo ripiantate in un altro posto
ancora, l'albero morirà senza produrre nulla. Nemmeno praticare
meditazione in questo modo sarà fruttuoso. Si tratta di una cosa che
dovete capire da soli. Contemplatela. Provate voi stessi quando andate a
casa. Prendete un alberello, piantatelo in un punto e sradicatelo dopo
pochi giorni, per poi piantarlo in un punto differente. Morirà senza
aver prodotto alcun frutto. È come fare un ritiro di meditazione per
sette giorni seguiti da sette mesi di comportamento sfrenato, consentire
ancora alla mente di macchiarsi e poi andare a fare un altro ritiro per
un breve periodo, praticando rigorosamente senza parlare e, quindi,
uscire ed essere di nuovo sfrenati. Come succede per l'albero, la
meditazione muore, nessun frutto salutare è serbato. L'albero non
cresce, la meditazione non cresce. Vi dico che praticare in questo modo
non reca frutti.

In verità, non mi piace molto fare discorsi di questo genere. È perché
mi dispiace parlarvi in modo critico. Quando state facendo cose
sbagliate, è mio dovere dirvelo, ma se sto parlando è perché provo
compassione per voi. Alcuni potrebbero sentirsi a disagio e pensare che
li sto solo rimproverando. Davvero, non vi sto rimproverando in modo
fine a se stesso, sto aiutandovi a osservare dove sbagliate, così che lo
sappiate. Alcuni potrebbero pensare: «~Luang Por\footnote{Luang Por (in
  thailandese \thai{หลวงพ่อ}). ``Venerabile padre''; è un'espressione che viene
  utilizzata in Thailandia per rivolgersi ai monaci anziani.} ci sta
solo sgridando.~» Non è così. In molto tempo sono potuto venire una sola
volta per tenere un discorso: se dovessi tenere discorsi di questo
genere ogni giorno, vi arrabbiereste davvero! Ma la verità è che non
sareste voi ad arrabbiarvi, ad arrabbiarsi sarebbero solo i
\emph{kilesa}. Dico solo questo, per ora.

