\chapter{Perché siamo qui?}

\begin{openingQuote}
  \centering

  Discorso pronunciato per un gruppo di laici al Wat Tham\\
  Saeng Phet (il monastero della Caverna della Luce di Diamante)\\
  durante il Ritiro delle Piogge del 1981, poco prima che\\
  la sua salute peggiorasse.
\end{openingQuote}

Non ho molte forze durante questo Ritiro delle Piogge, non sto molto
bene e, così, per respirare un po' d'aria fresca sono venuto su questa
collina. La gente viene a trovarmi, ma non posso riceverli come ero
solito fare, perché la mia voce sta per andarsene e il mio respiro non
c'è quasi più. Che questo corpo stia ancora qui seduto e lo possiate
vedere, questa cosa potete ritenerla una benedizione. È di per sé una
benedizione. Presto non lo vedrete più. Il respiro finirà, la voce se ne
andrà. Se ne andranno, in accordo con i fattori che li sostengono, come
succede per tutte le cose composte. Il Buddha lo chiamò
\emph{khaya-vayaṃ}, decadenza e dissoluzione di tutti i fenomeni
condizionati.

Come decadono? Pensate a un blocco di ghiaccio. Inizialmente è acqua. La
gente la congela e diventa ghiaccio. Però, non ci vuole molto tempo
prima che si sciolga. Prendete un grande blocco di ghiaccio, diciamo
grande come questo registratore che sta qui, e lasciatelo fuori al sole.
Potete vedere come decade, in modo molto simile a quello del corpo. Si
disintegra gradualmente. Dopo non molte ore o minuti, tutto quel che
resta è una pozza d'acqua. Ciò è detto \emph{khaya-vayaṃ}, decadenza e
dissoluzione di tutte le cose composte. È stato così da lungo tempo, fin
dall'inizio dei tempi. Quando si nasce, veniamo al mondo portando innata
in noi questa natura, non possiamo evitarla. Al momento della nascita
portiamo con noi vecchiaia, malattia e morte.

Questa è la ragione per cui il Buddha disse \emph{khaya-vayaṃ},
decadenza e dissoluzione di tutte le cose composte. Tutti noi che ora
stiamo seduti in questa sala, monaci e novizi, laici e laiche, siamo
tutti, senza eccezione alcuna, ``blocchi in decomposizione''. Ora come
ora il blocco è solido, proprio come il blocco di ghiaccio. Inizia come
acqua, diventa ghiaccio per un po' e poi si scioglie di nuovo. Potete
vedere questa decadenza in voi stessi? Guardate questo corpo. Invecchia
ogni giorno, i capelli stanno invecchiando, le unghie stanno
invecchiando, tutto sta invecchiando!

Prima non eravate così, o no? Probabilmente eravate molto meno grandi di
ora. Adesso siete cresciuti e diventati maturi. D'ora in poi, per voi ci
sarà il declino, secondo la via della natura. Il corpo decade proprio
come il blocco di ghiaccio. Presto, proprio come il blocco di ghiaccio,
se ne sarà andato del tutto. Tutti i corpi sono composti dai quattro
elementi: terra, acqua, fuoco e vento. Un corpo è quella convergenza di
terra, acqua, vento e fuoco che noi poi chiamiamo persona. In origine è
difficile dire come potremmo chiamarla, ma ora noi la chiamiamo
``persona''. Ce ne infatuiamo, diciamo che è un maschio, una femmina,
attribuiamo a essa dei nomi, signor, signora e così via, così da poterci
identificare a vicenda con maggior facilità. In verità lì non c'è
nessuno. C'è terra, acqua, fuoco e vento. Quando si riuniscono in questa
forma, il risultato lo chiamiamo ``persona''. Non è il caso di agitarsi
tanto. Se davvero ci guardate dentro, lì non c'è nessuno.

Nel corpo quel che è solido -- carne, pelle, ossa e così via -- è
l'elemento terra. Le parti liquide sono l'elemento acqua. La qualità del
calore è l'elemento fuoco, mentre i venti che lo attraversano sono
l'elemento vento. Al Wat Pah Pong abbiamo un corpo che non è né maschio
né femmina: è lo scheletro appeso nella sala principale. Guardandolo non
si riceve la sensazione che si tratti di un uomo o di una donna. Le
persone si chiedono se è una donna o un uomo, ma tutto quello che
possono fare è guardarsi a vicenda privi di espressione. È solo uno
scheletro, la pelle e la carne non ci sono più.

A questo proposito la gente è ignorante. Alcuni vanno al Wat Pah Pong,
vedono lo scheletro nella sala principale e poi scappano subito via! Non
ce la fanno a guardare. Hanno paura, hanno paura degli scheletri.
Immagino che, prima di allora, costoro non abbiano mai visto se stessi.
Siccome hanno paura degli scheletri, non riflettono sul grande valore di
uno scheletro. Per andare al monastero sono dovuti arrivare in macchina
o a piedi. Se non avessero avuto le ossa, come avrebbero potuto farlo?
Sarebbero stati in grado di camminare? Però, sono andati in macchina al
Wat Pah Pong, hanno camminato fin nella sala principale, hanno visto lo
scheletro, e poi sono subito scappati! Prima non avevano mai visto una
cosa del genere. Sono nati con uno scheletro, ma non l'hanno mai visto.
È una vera fortuna che abbiano avuto un'opportunità per vederlo. Anche
la gente anziana vede lo scheletro e si spaventa. Che cos'è tutto questo
trambusto? Ciò dimostra che non sono affatto in contatto con se
stessi, che non conoscono veramente se stessi. Forse tornano a casa e
non riescono a dormire per tre o quattro giorni, e tuttavia stanno
dormendo con uno scheletro! Si vestono con lo scheletro, mangiano e
fanno tutto con lo scheletro, eppure ne hanno paura.

Questo dimostra quanto siano in contatto con se stessi. È penoso! Stanno
sempre a guardare l'esterno, gli alberi, l'altra gente, gli oggetti
esteriori. «~Questo è grande~» e «~quello è piccolo~», «~questo è
corto~» e «~quello è lungo~», dicono. Sono così occupati a guardare le
altre cose che non hanno mai visto se stessi. Per essere onesti, la
gente fa davvero compassione, non ha alcun rifugio.

Nelle cerimonie di ordinazione monastica, gli ordinandi devono imparare
i cinque temi basilari di meditazione: \emph{kesā}, i capelli;
\emph{lomā}, i peli; \emph{nakhā}, le unghie; \emph{dantā}, i denti;
\emph{taco}, la pelle. Qualche studente e le persone istruite
ridacchiamo quando ascoltano questa parte della cerimonia di
ordinazione. «~Cosa sta cercando di insegnarci l'\emph{ajahn}? Ci sta
insegnando che esistono i capelli, che abbiamo da anni. Non deve
insegnarcelo, lo sappiamo già. Perché preoccuparsi di insegnarci una
cosa che già sappiamo?~» La gente vana è così, pensa di poter già vedere
i capelli. Io faccio loro notare che, quando dico di ``guardare i
capelli'', intendo vedere i capelli ``come sono in realtà''. Guardate i
peli come sono veramente; guardate le unghie, i denti e la pelle come
sono veramente. Questo è quel che io chiamo ``vedere'', non vedere in
modo superficiale, ma vedere in accordo con la Verità. Non saremmo
immersi nelle cose fino al collo se le potessimo vedere come realmente
sono. Capelli, peli, unghie, denti, pelle. Che cosa sono realmente? Sono
belli? Sono puliti? Hanno una qualche reale sostanzialità? Sono stabili?
In essi non c'è niente. Non sono belli, ma immaginiamo che lo siano. Non
hanno sostanzialità, ma immaginiamo che ne abbiano.

Capelli, peli, unghie, denti, pelle. La gente è davvero attaccata a
queste cose. Il Buddha disse che sono temi basilari per la meditazione,
ci insegnò a conoscerle queste cose. Sono transitorie, imperfette e
senza proprietario: non sono ``io'' o ``loro''. Siamo nati con queste
cose ed esse ci illudono, anche se sono davvero sudicie. Supponete che
non ci si lavi per una settimana. Potremmo sopportare di stare vicini
l'uno all'altro? Avremmo davvero un pessimo odore. Quando la gente suda
molto, quando ad esempio molte persone lavorano sodo assieme, l'odore è
terribile. Torniamo a casa e ci strofiniamo con acqua e sapone, e
l'odore diminuisce un po', la fragranza del sapone lo sostituisce.
Quando l'odore del sapone se ne va, torna di nuovo l'odore del corpo.

Abbiamo la tendenza a considerare questi corpi belli, piacevoli,
durevoli e forti. Tendiamo a pensare che non invecchieremo, che non ci
ammaleremo e che non moriremo. Siamo ammaliati e incantati dal corpo, e
perciò siamo così ignoranti a proposito del vero rifugio che si trova in
noi stessi. Il vero luogo di rifugio è la mente. La mente è il nostro
vero rifugio. Questa sala è abbastanza grande, ma non può essere un vero
rifugio. Qui si rifugiano i piccioni, qui si rifugiano i gechi, qui si
rifugiano le lucertole. Possiamo anche pensare che questa sala ci
appartenga, ma non è così. Qui viviamo insieme con tutto il resto. È
solo un ricovero temporaneo, presto lo dovremo lasciare. La gente invece
considera posti come questi dei rifugi.

Per questa ragione il Buddha ci disse di trovare il nostro rifugio. Ciò
significa trovare il vostro vero cuore, e questo cuore è davvero
importante. La gente di solito non prende in considerazione le cose
importanti, trascorre la maggior parte del tempo a guardare cose
irrilevanti. Ad esempio, quando le persone rassettano casa possono
essere propensi a pulirla, a lavare i piatti e così via, ma non riescono
a osservare i loro cuori. Il loro cuore può essere corrotto, possono
essere arrabbiate, e lavare i piatti con un'espressione truce in volto.
Non riescono a vedere che il loro cuore non è affatto pulito. Questo è
ciò che io chiamo ``considerare un ricovero temporaneo come un
rifugio''. Abbelliscono casa e famiglia, ma non pensano ad abbellire il
loro cuore. Non esaminano la sofferenza. Il cuore è una cosa importante.
Il Buddha insegnò a trovare un rifugio all'interno del nostro stesso
cuore: \emph{attā hi attano nātho}, rendi te stesso un rifugio per te
stesso. Chi altri può essere il vostro rifugio? Il vero rifugio è il
cuore, nient'altro. Potete cercare di fare affidamento su altro, ma non
è cosa sicura. Potete fare affidamento su altre cose solo se già avete
un rifugio dentro voi stessi. Dovete prima avere un vostro vero rifugio,
per poter far affidamento su qualsiasi altra cosa, sia essa un
insegnante, la famiglia, gli amici o i parenti.

Tutti voi che siete oggi venuti a trovarmi, sia laici sia monaci,
prendete per favore in considerazione questo insegnamento. Chiedete a
voi stessi: «~Chi sono, perché sono qui?~» Chiedete a voi stessi:
«~Perché sono nato?~» Alcuni non lo sanno. Vogliono essere felici, ma la
sofferenza non cessa mai. Ricchi e poveri, giovani e anziani, soffrono
tutti allo stesso modo. È tutta sofferenza. Perché? Perché non hanno
saggezza. I poveri sono infelici perché non hanno abbastanza, e i ricchi
sono infelici perché hanno troppo di cui prendersi cura.

In passato, quando ero un giovane novizio, ho tenuto un discorso di
Dhamma. Ho parlato della felicità di avere ricchezze e possedimenti,
domestici e così via \ldots{} un centinaio di inservienti uomini, un
centinaio di inservienti donne, un centinaio di elefanti, un centinaio
di mucche, un centinaio di bufali \ldots{} tutto a centinaia! I laici se la
bevevano proprio. Potete immaginare cosa significhi prendervi cura di un
centinaio di bufali? O di un centinaio di mucche, di centinaia di
inservienti? Riuscite a immaginarlo? Sarebbe divertente? La gente non
considera quest'aspetto delle cose. Desidera possedere, avere mucche,
bufali, inservienti, a centinaia. Vi assicuro che cinquanta bufali
sarebbero già troppi. Solo intrecciare la fune per tutte quelle bestie
sarebbe già troppo! La gente però queste cose non le considera, pensa
solo al piacere di possedere. Non considera i problemi connessi.

Se non abbiamo saggezza, tutto quello che ci sta attorno sarà fonte di
sofferenza. Se siamo saggi, occhi, orecchi, naso, lingua, corpo e mente
ci condurranno fuori dalla sofferenza. Gli occhi non sono
necessariamente una buona cosa, lo sapete. Se siete di cattivo umore,
solamente vedere altra gente può farvi arrabbiare o perdere il sonno.
Oppure potete innamorarvi. Anche l'amore è sofferenza, se non ottenete
ciò che volete. Amore e odio sono entrambi sofferenza a causa del
desiderio. Voler avere è sofferenza, non voler avere è sofferenza.
Ottenere delle cose: anche se ci riuscirete sarà ugualmente sofferenza,
perché avrete paura di perderle. C'è solo sofferenza. Come pensate di
conviverci? Potete avere una casa grande e lussuosa ma, se il vostro
cuore non è buono, le cose non andranno come credete.

Perciò dovreste tutti guardare voi stessi. Perché siamo nati? Otterremo
mai davvero qualcosa in questa vita? Qui in campagna si comincia a
piantare riso fin da bambini. Quando arrivano a diciassette o diciotto
anni si precipitano a sposarsi, col timore di non aver abbastanza tempo
per fare fortuna. Iniziano a lavorare giovanissimi, pensando che, così,
diventeranno ricchi. Piantano riso fino a quando hanno settanta,
ottanta, perfino novant'anni. Io chiedo: «~Hai lavorato fin dalla
nascita. Ora è quasi giunto il tempo di andare. Che cosa intendi portare
con te?~» Non sanno cosa rispondere. Tutto ciò che sanno dire è: «~Non
saprei!~» Abbiamo un detto da queste parti: «~Non attardarti a
raccogliere bacche lungo il cammino, la notte scenderà prima che tu
possa rendertene conto.~» Tutto a causa di questo ``non saprei''. Non
sono né qui né lì, si accontentano di un ``non saprei'' e siedono tra i
cespugli a rimpinzarsi di bacche. «~Non saprei, non saprei.~»

Quando siete ancora giovani pensate che non avere relazioni sentimentali
non vada bene; vi sentite un po' soli. Così, trovate un compagno o una
compagna per viverci insieme. Mettete due persone insieme, e ci saranno
attriti! Vivendo da soli c'è troppa tranquillità, vivendo con gli altri
ci sono attriti.

Quando i figli sono piccoli i genitori pensano: «~Quando saranno più
grandi, andrà meglio.~» Si occupano dei loro figli e li fanno crescere.
Tre, quattro, cinque figli, e pensano che quando saranno grandi il
fardello sarà più leggero. Però, i figli crescono e il fardello diventa
ancor più pesante. Come due pezzi di legno, uno grande e l'altro
piccolo. Gettate via quello piccolo e tenete quello più grande, pensando
che tutto diverrà più leggero, ma non è così, ovviamente. Quando i figli
sono piccoli non danno poi così tanto fastidio, hanno bisogno solo di un
po' di riso e di una banana di tanto in tanto. Quando crescono, vogliono
la moto o l'automobile! Bene, non potete dire di no, perché i vostri
figli li amate. Così, cercate di dare loro ciò che desiderano. A volte i
genitori discutono. «~Non comprargli l'automobile, non abbiamo
abbastanza denaro!~» Però, siccome amate i vostri figli, chiedete un
prestito da qualche parte. Forse, per accontentare i figli, i genitori
devono restare senza denaro. Poi c'è l'istruzione. «~Quando avranno
finito di studiare sarà tutto a posto.~» Di studiare non si finisce mai!
Cosa stanno per portare a compimento? Solo nella scienza del buddhismo
c'è un punto di arrivo, tutte le altre conoscenze girano in tondo. Alla
fine c'è solo un gran mal di testa. Se in una casa ci sono quattro o
cinque figli, i genitori discutono tutti i giorni.

La sofferenza che ci attende nel futuro non riusciamo a vederla,
pensiamo che non arriverà mai. Quando arriva, allora sì che la
conosciamo. Quel genere di sofferenza, la sofferenza insita nei nostri
corpi, è difficile da prevedere. Quando ero un ragazzino che badava ai
bufali, prendevo un pezzetto di carbone e me lo sfregavo sui denti per
sbiancarli. Tornavo a casa, mi guardavo allo specchio e li vedevo così
belli e bianchi. Ero ingannato dalle mie stesse ossa, ecco. Quando
arrivai all'età di cinquanta o sessant'anni i denti iniziarono ad
allentarsi. Quando cominciarono a cadere, faceva davvero male. Ci sono
passato. Così andai dal dentista per farmeli togliere tutti. Ora ho la
dentiera. I miei denti mi davano così tanti problemi che volli
togliermeli tutti, sedici in una sola volta. Il dentista era riluttante
a farlo, ma io gli dissi: «~Toglili e basta, mi assumo ogni
responsabilità.~» Così li tolse tutti in una sola volta. Alcuni erano
ancora buoni, almeno cinque. Li tolse tutti. Fu però una cosa davvero
rischiosa. Dopo che me li tolse non riuscii a mangiare per due o tre
giorni.

Prima, quando ero un ragazzino che badava ai bufali, ero solito pensare
che lustrarsi i denti fosse una cosa importante. Amavo i miei denti,
pensavo che fossero un qualcosa di positivo. Alla fine, però, per i
denti giunse il tempo di andare. Il dolore mi ha quasi ucciso. Ho
sofferto di mal di denti per mesi, anni. A volte le gengive erano
dappertutto completamente gonfie. Prima o poi alcuni di voi potrebbero
avere l'opportunità di sperimentare tutto questo. Se i vostri denti sono
ancora buoni e ve li lavate tutti i giorni per conservarli belli e
bianchi, fate attenzione! In seguito potrebbero farvi qualche
scherzetto.

Vi sto solo informando di queste cose, della sofferenza che sorge da
dentro, dall'interno del nostro stesso corpo. Nel corpo non c'è nulla su
cui possiate fare affidamento. Quando si è ancora giovani non va così
male ma, quando si diventa anziani, le cose cominciano a collassare.
Tutto inizia a cadere in pezzi. I fenomeni condizionati seguono il loro
corso naturale. Che si rida o si pianga, vanno per la loro strada. Non
importa come si viva o si muoia, per essi non fa alcuna differenza. E
non c'è sapere o scienza che possa impedire il naturale corso degli
eventi. Potete far in modo che un dentista dia un'occhiata ai vostri
denti ma, anche se li sistema, continueranno a seguire il loro corso
naturale. Può succedere che pure il dentista abbia questo stesso
problema. Alla fine tutto cade in pezzi.

Sono cose che dovremmo contemplare quando abbiamo ancora un po' di
vigore. Dovremmo praticare quando siamo giovani. Se volete accumulare
meriti, sbrigatevi allora, e fate come vi ho detto, non aspettate
d'essere anziani. La maggior parte della gente aspetta quando è anziana
prima di andare in monastero e cercare di praticare il Dhamma. Donne e
uomini dicono la stessa cosa: «~Aspetto di essere anziano.~» Non so
perché dicano così. Un anziano ha molto vigore? Fatelo gareggiare a
correre con un giovane, e guardate la differenza. Perché per praticare
vogliono attendere di essere anziani? È come se non dovessero mai
morire. Poi arrivano a cinquanta, sessant'anni o più. «~Ehi, nonna,
andiamo al monastero!~» «~Andate voi, non ci sento più tanto bene.~»
Capite quel che vi sto dicendo? Quando il suo udito era buono, che cosa
ascoltava? «~Non saprei!~» Stava perdendo tempo con le bacche. Alla
fine, quando l'udito se n'è andato, va in monastero. È inutile. Ascolta
il discorso, ma non ha la benché minima idea di cosa si stia dicendo.
Prima di pensare di praticare il Dhamma, la gente attende di non avere
più energie.

\enlargethispage{-\baselineskip}

Il discorso di oggi può essere utile a chi, fra voi, è in grado di
comprenderlo. Si tratta di cose che dovreste cominciare a osservare, le
abbiamo ricevute in eredità. Diventeranno sempre più pesanti, sempre di
più, un fardello che ognuno di noi dovrà portare. In passato le mie
gambe erano forti, potevo correre. Adesso, anche solo per camminare qui
attorno le sento pesanti. Prima erano le mie gambe a portarmi. Adesso
sono io a doverle portare. Quando ero bambino vedevo gli anziani che si
alzavano dal posto in cui sedevano: «~Oh!~» Alzandosi, gemevano: «~Oh!~»
C'era sempre questo ``oh''. Però, non sanno che cos'è che li fa gemere
in questo modo. La gente non vede che il corpo va in rovina nemmeno
quando arriva a questo punto. Non potete sapere quando vi separerete da
esso. Sono semplicemente i fenomeni condizionati che seguono il loro
corso naturale a causare tutto quel dolore.

La gente parla di artrite, di reumatismi, di gotta e così via; il
dottore prescrive medicinali, ma non si guarisce mai completamente. Alla
fine il corpo cade a pezzi, e pure il dottore! Sono le condizioni che
seguono il loro corso naturale. Sono così, è la loro natura. Dateci
un'occhiata. Se lo vedete in anticipo, ve la caverete meglio. È come
vedere lungo la via, di fronte a voi, un serpente velenoso. Se vedete
che è lì, potete tenervi lontani e non essere morsicati. Se non lo
vedete, potreste continuare a camminare e calpestarlo, e allora vi
morderà.

La gente non sa che fare quando sorge la sofferenza. Dove andare per
curarla? Vogliono evitare di soffrire, vogliono essere liberi da essa,
ma quando sorge non sanno come curarla. E si continua a vivere in questo
modo fino a quando non si diventa anziani e malati, e si muore.
Anticamente si diceva che se qualcuno era mortalmente malato, uno dei
parenti più stretti avrebbe dovuto sussurrargli nelle orecchie
«~Bud-dho, Bud-dho.~» Che utilità ha \emph{Buddho} per chi sta per
finire sulla pira funeraria? Perché non hanno imparato \emph{Buddho}
quando erano giovani e sani? Quando il respiro si è fatto irregolare, vi
avvicinate e dite: «~Mamma, \emph{Buddho}, \emph{Buddho}!~» Perché
perdete tempo? La confondete solamente, lasciatela andare in pace.

\enlargethispage{-\baselineskip}

La gente non sa come risolvere i problemi nel proprio cuore, non ha un
rifugio. Le persone si arrabbiano facilmente e hanno un sacco di
desideri. Perché è così? Perché non hanno un rifugio. Quando sono
sposate da poco, vanno d'amore e d'accordo, ma dopo i cinquant'anni o
giù di lì non si capiscono più. Qualsiasi cosa la moglie dica, per il
marito è intollerabile. Qualsiasi cosa il marito dica, la moglie non
l'ascolta. Si voltano le spalle a vicenda. Non sto parlando perché non
ho mai avuto una famiglia. Volete sapere per quale ragione non ho mai
avuto una famiglia? Solo guardando questa parola,
``famiglia'',\footnote{In thailandese c'è un gioco di parole legato alla
  parola famiglia, \emph{khrâwp-khrua} (\thai{ครอบครัว}), che letteralmente
  significa ``struttura per cucinare'' o ``cerchio per arrostire''. Nel
  testo inglese, ove si è opportunamente evitata la traduzione letterale
  dal thailandese, si ha ``household'', ciò che subito dopo ha però
  consentito di innescare un altro gioco di parole prima in riferimento
  a ``hold'' (tenere, mantenere, ossia nel nostro caso ``tenere
  insieme'') e poi a ``house'': nella traduzione italiana si è dovuto
  rinunciare anche a questo.\\[8pt]
  Quel che Ajahn Chah afferma a proposito della famiglia può risultare
  molto forte e, forse, di estrema durezza per il lettore occidentale,
  per lo più abituato a un approccio diverso. Si deve però tener conto
  del fatto che il Maestro sta tentando di compensare gli usuali
  attaccamenti che abbiamo la tendenza di alimentare in modo eccessivo.}
già sapevo di cosa si trattava. Che cos'è una ``famiglia''? È un
``tenere insieme''. Se qualcuno ci legasse con una corda mentre stiamo
qui seduti, come stareste? Questo è detto ``essere tenuti insieme''. A
qualsiasi cosa possa somigliare, ``essere tenuti insieme'' è così. Vi è
un cerchio, all'interno del quale si è confinati. L'uomo vive dentro il
suo cerchio di confino e la donna vive dentro il suo cerchio di confino.

Questa parola, ``famiglia'', la ritengo una parola pesante. Non è una
cosa insignificante, è davvero una cosa assassina. La parola ``tenere''
è un simbolo della sofferenza. Potete andare ovunque, ma dovete restare
all'interno del vostro cerchio di confino. Famiglia significa ``ciò che
dà fastidi''. Avete mai tostato peperoncini? In casa tutti tossiscono e
starnutiscono. La parola ``famiglia'' significa confusione. Non ne vale
la pena. A causa di questa parola mi feci ordinare monaco e, poi, non ho
più lasciato l'abito. ``Famiglia'' è terrificante. Sei bloccato e non
puoi andare da nessuna parte. Problemi con i figli, con il denaro e con
tutto il resto. Dov'è che si può andare? Si è legati. Ci sono figli e
figlie, discussioni a profusione fino al giorno della morte e non c'è
alcun altro posto in cui andare, non importa quanto grande sia la
sofferenza. Lacrime e poi lacrime, in continuazione. Le lacrime non
finiranno mai con la ``famiglia''. Solo se non c'è famiglia si può
essere in grado di farla finita con le lacrime, in nessun altro modo.

\enlargethispage{-\baselineskip}

Considerate questo dato di fatto. Se non ci siete già passati finora,
potrebbe succedervi in futuro. Alcuni l'hanno in parte sperimentato,
altri sono già arrivati all'esasperazione: «~Resto o me ne vado?~» Al
Wat Pah Pong ci sono circa settanta, ottanta \emph{kuṭī}.\footnote{\emph{kuṭī.}
  Capanna nella foresta che funge da piccola dimora per i monaci e per i
  praticanti laici.} Quando sono quasi tutte occupate, dico al monaco
incaricato di tenerne alcune libere nel caso ci fosse una discussione
tra coniugi. È per lo più certo che, poco dopo tempo, arrivi una signora
con le valigie. «~Sono stufa del mondo, Luang Por.~» «~Ehi! Non dirlo.
Sono parole grosse davvero.~» Poi arriva il marito, e anche lui dice di
essere stufo. Dopo due o tre giorni in monastero la stanchezza per il
mondo scompare. Dicono di essere stufi del mondo, ma si stanno solo
prendendo in giro. Quando se ne stanno tranquillamente da soli seduti
fuori dalla \emph{kuṭī}, dopo un po' cominciano a pensare: «~Quando
arriverà mia moglie per chiedermi di tornare a casa?~» Proprio non
capiscono cosa stia succedendo. Che cos'è questa loro ``stanchezza del
mondo''? Si arrabbiano per qualcosa e arrivano correndo in monastero. A
casa pareva che tutto andasse male. Il marito aveva torto, la moglie
aveva torto, ma dopo tre giorni di serena riflessione \ldots{} «~Mmm, mia
moglie in fin dei conti aveva ragione, ero io ad avere torto.~» «~Mio
marito aveva ragione, non avrei dovuto arrabbiarmi così tanto.~» Si
scambiano le parti. Così è, questa è la ragione per cui non prendo il
mondo troppo sul serio. Già ne conosco i retroscena, per questo ho
scelto di vivere da monaco.

Vorrei offrire a tutti voi il discorso di oggi come un compito per casa.
Che lavoriate nei campi o in città, prendetele in considerazione queste
parole. «~Perché sono nato?~» «~Che cosa posso portare con me?~»
Chiedetevelo in continuazione. Se vi ponete spesso queste domande,
presto diverrete saggi. Se non riflettete su queste cose, rimarrete
ignoranti. Ascoltando il discorso di oggi potreste comprendere qualcosa.
Se non ora, magari quando tornerete a casa. Oppure questa sera. Mentre
state ascoltando tutto vi risulta oscuro, ma forse un po' di
comprensione vi sta aspettando nell'automobile. Quando vi entrerete, può
succedere che quel po' di comprensione entri in voi e poi, a casa, tutto
divenga chiaro: «~Oh, ecco cosa voleva dire Luang Por. Prima non
riuscivo a capirlo.~»

Penso che per oggi sia abbastanza. Se parlo troppo, questo vecchio corpo
si stanca.

