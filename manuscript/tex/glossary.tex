\chapter{Glossario}

Si è cercato quanto più possibile di rispettare la forma corretta sia
dal punto di vista grammaticale che tecnico di termini e concetti in
italiano; i sostantivi in lingua pāli sono in genere allo stato
tematico, mentre la forma nominativa, singolare o plurale, è indicata
tra parentesi tonde qualora essa corrisponda a quella di solito
ricorrente. Per alcune integrazioni ci si è avvalsi del glossario
contenuto in \emph{La Rivelazione del Buddha,} I: \emph{I testi
antichi,} a cura di R. Gnoli, Milano 2007\textsuperscript{4} (I
Meridiani. Classici dello Spirito).



\emph{Abhidhamma}. (1) Nei discorsi del Canone in pāli questo termine
indica semplicemente il ``Dhamma più elevato'', nonché un tentativo
sistematico di definire gli insegnamenti del Buddha e di comprendere le
loro correlazioni. (2) Terza parte del Canone in pāli, composta di
trattati analitici basati su elenchi di categorie estratte dai discorsi
del Buddha.

\emph{ācariya}. Insegnante, mentore, maestro; → \emph{ajahn}; →
\emph{kalyāṇamitta}.

\emph{adhiṭṭhāna}. Determinazione, decisione, risolutezza, impegno o
intento che rivolge la mente in una certa direzione. È una delle dieci
perfezioni; → \emph{pāramī}.

\emph{ajahn} (in thailandese,
\href{http://www.thai2english.com/dictionary/1453955.html}{อาจารย์}). Il
termine deriva da \emph{ācariya}, in pāli, letteralmente ``insegnante'';
spesso viene utilizzato per un monaco o per una monaca con più di dieci
anni di vita monastica.

\emph{ājīvaka}. Una scuola di contemplativi contemporanea del Buddha, i
cui seguaci ritenevano che la volontà degli esseri non fosse in grado di
indirizzare le loro azioni e che l'universo fosse guidato dalla sorte.

\emph{akusala}. Non salutare, nocivo, maldestro, non meritorio; →
\emph{kusala}.

ālāra Kālāma. Il maestro che insegnò al \emph{bodhisatta} la meditazione
nella sfera del ``senza forma'' sulla ``base del nulla'' quale più alta
fruizione della vita santa.

\emph{anāgāmin} (\emph{anāgāmī}). ``Chi è senza ritorno'', ossia chi ha
divelto tutte e cinque le catene inferiori (→ \emph{saṃyojana}) che
legano la mente al ciclo della rinascita, e che dopo la morte apparirà
in uno dei mondi di Brahmā, per poi entrare nel → \emph{Nibbāna}, senza
mai tornare in questo mondo.

\emph{anāgārika} (in thailandese \emph{pah-kao}: ผ้าขาว; ปะขาว).
Letteralmente, ``non cittadino'', ossia ``senza casa''. Un postulante
che ha assunto gli Otto → Precetti e spesso vive con i → \emph{bhikkhu},
oltre a sostenere la sua pratica di meditazione, li aiuta in alcuni
lavori che il Vinaya impedisce loro di svolgere.

\emph{ānāpānasati}. Letteralmente, ``consapevolezza dell'inspirazione e
dell'espirazione'' o consapevolezza del respiro. Questa pratica di
meditazione consiste nel mantenere l'attenzione e la consapevolezza
sulle sensazioni del respiro.

\emph{anatta} (\emph{anattā}). Non-sé, non sostanziale, impersonale; →
\emph{tilakkhaṇa}.

\emph{anicca} (\emph{aniccā}). Incostante, instabile, impermanente; →
\emph{tilakkhaṇa}.

\emph{añjali}. È un gesto di rispetto consistente nel congiungere le
mani al petto al cospetto di qualcuno; oggigiorno è ancora diffuso nei
paesi buddhisti e in India.

\emph{āntara-vāsaka} → veste monastica.

\emph{anupubbī-kathā}. Istruzione graduale. Il metodo d'insegnamento del
Dhamma da parte del Buddha che conduce progressivamente i suoi
ascoltatori per mezzo di argomenti via via più avanzati: la generosità
(→ \emph{dāna}), la virtù o moralità (→ \emph{sīla}), i paradisi, gli
svantaggi dei piaceri sensoriali, la rinuncia (→ \emph{nekkhamma}) e le
Quattro Nobili Verità (→ \emph{ariya-sacca}).

\emph{anusaya}. Predisposizione; tendenza latente. Ci sono sette
tendenze maggiori latenti, verso le quali la mente torna in
continuazione: verso la passione sensoriale
(\emph{kāma}-\emph{rāganusaya}), l'avversione (\emph{patīghānusaya}),
le visioni errate (\emph{dhiṭṭhānusaya}), il dubbio
(\emph{vicikicchānusaya}), l'orgoglio (\emph{mānusaya}), la passione
per il divenire (\emph{bhava}-\emph{rāganusaya}), l'ignoranza
(\emph{avijjānusaya}); → \emph{saṃyojana.}

\emph{arahat} (\emph{arahant}). Letteralmente, un ``Meritevole''; una
persona la cui mente è libera dalle contaminazioni (→ \emph{kilesa}),
che ha abbandonato tutte e dieci le catene (→\emph{saṃyojana}), sia le
cinque inferiori sia le cinque superiori che legano la mente al ciclo
della rinascita, il cui cuore è libero dagli influssi impuri (→
\emph{āsava}), e che perciò non è destinato a un'altra rinascita. È
anche un titolo del Buddha e il livello più alto dei suoi Nobili
Discepoli.

\emph{ārammaṇa}. Oggetto mentale, oggetto di riferimento di un metodo
meditativo.

\emph{ariya}. Nobile; chi ha ottenuto la visione trascendente in uno dei
quattro livelli dell'Illuminazione, il più alto dei quali è quello
dell'→ \emph{arahat}; i tre precedenti stadi sono: → \emph{sotāpanna}; →
\emph{sakadāgāmin}; → \emph{anāgāmin}. Tutti insieme costoro formano la
categoria delle Nobili Persone; → \emph{ariya-puggala}.

\emph{ariya-puggala}. Letteralmente, ``Nobile Persona''; chi ha percorso
almeno il primo sentiero inferiore dei quattro Nobili Sentieri (→
\emph{magga}) o conseguito il Frutto (→ \emph{phala}) di essi. Si
paragoni quanto detto in relazione a → \emph{putthujjana}.

\emph{ariya-sacca} (\emph{ariya-saccāni}). Nobile Verità. Le Quattro
Nobili Verità costituiscono il primo e centrale insegnamento del Buddha
riguardo alla sofferenza, alla sua origine, alla sua cessazione e al
Sentiero che conduce a tale cessazione
(\emph{dukkha-nirodha-gāminī-paṭipadā}). La completa comprensione della
Quattro Nobili Verità equivale alla fruizione del \emph{Nibbāna}.

\emph{asaṅkhata-dhamma}. Si veda il suo opposto →
\emph{saṅkhata-dhamma.}

\emph{āsava}. Influsso impuro, macchia, fermentazione o effluenza. Le
quattro qualità che macchiano la mente: brama sensoriale, visioni
errate, divenire e ignoranza.

\emph{asekha.} Una persona (\emph{puggala}) oltre l'addestramento, ossia
un → \emph{arahat}.

\emph{asubha}. Non bello, da intendersi come repulsivo, ripugnante e
sporco. Il Buddha raccomandò la contemplazione di questi aspetti del
corpo come antidoto alla lussuria.

\emph{atta} (\emph{attā}). Io o sé, sostanziale, personale; a volte con
il senso di anima; si veda il suo opposto (→ \emph{anatta}).

\emph{avijjā}. Non conoscenza, ignoranza; consapevolezza offuscata;
confusione (→ \emph{moha}) sulla natura della mente. La principale
radice del male e della continua rinascita.

\emph{āyatana}. Le basi sensoriali. Le basi interne sono gli organi dei
sensi: occhi, orecchi, naso, lingua, corpo e mente. Le basi esterne sono
i loro rispettivi oggetti.

\emph{bala}. Forza, potere. Si riferisce a cinque facoltà: fede/fiducia
(→ \emph{saddhā}), energia (→ \emph{viriya}), consapevolezza (→
\emph{sati}), concentrazione (→ \emph{samādhi}), saggezza (→
\emph{paññā}); queste facoltà vengono coltivate per spezzare le cinque
catene secondarie (→ \emph{saṃyojana}).

\emph{bhante}. Epiteto, ``venerabile signore''; viene spesso utilizzato
quando ci si rivolge a un monaco buddhista.

\emph{bhava}. Esistenza; divenire; una ``vita''. Stati dell'esistenza
che si sviluppano nella mente e possono essere sperimentati come mondi
interiori e/o come mondi a livello esterno. Tre sono i livelli del
divenire: il livello dei sensi, il livello della forma e il livello
dell'assenza di forma.

\emph{bhāvanā}. Meditazione, sviluppo o coltivazione. Termine spesso
utilizzato per far riferimento a \emph{citta-bhāvanā}, allo sviluppo
della mente, o a \emph{paññā-bhāvanā}, sviluppo della saggezza; →
\emph{kammaṭṭhāna}.

\emph{bhava-taṇhā}. Bramosia di divenire, di essere.

\emph{bhikkhu}. Un monaco buddhista; un uomo che ha rinunciato al suo
ruolo in famiglia per vivere a un livello più alto di virtù (→
\emph{sīla}) in accordo con il → Vinaya in generale, e con le regole del
→ \emph{Pātimokkha}; → \emph{parisā}; → Saṅgha; → \emph{upasampadā}.

\emph{bhikkhunī}. Una monaca buddhista; una donna che ha rinunciato al
suo ruolo in famiglia per vivere a un livello più alto di virtù (→
\emph{sīla}) in accordo con il → Vinaya in generale, e con le regole del
→ \emph{Pātimokkha}; → \emph{parisā}; → Saṅgha; → \emph{upasampadā}.

\emph{bhikkhu-saṅgha}. La comunità dei monaci buddhisti; → Saṅgha.

\emph{bodhi.} Risveglio; → Illuminazione.

\emph{bodhi-pakkhiya-dhamma.} Le parti del Risveglio, ossia i
trentasette fattori che contribuiscono al Risveglio: (1) i quattro
fondamenti della consapevolezza (→ \emph{satipaṭṭhāna}); (2) i quattro
tipi di retto sforzo (→ \emph{sammappadhāna}); (3) le quattro basi del
potere psichico (→ \emph{iddhipadā}); (4) le cinque facoltà spirituali
(→ \emph{indriya}); (5) i cinque poteri (→ \emph{bala}); (6) i sette
fattori del Risveglio (→ \emph{bojjhaṅga}); (7) il → Nobile Ottuplice
Sentiero (\emph{magga}); → \emph{magga}.

\emph{bodhisatta} (sanscrito: \emph{bodhisattva}). Un essere che si
impegna per raggiungere il Risveglio; è il termine utilizzato per
descrivere il Buddha prima dell'Illuminazione, dall'iniziale aspirazione
alla Buddhità fino al pieno Risveglio.

\emph{bojjhaṅga}. I sette fattori del Risveglio: consapevolezza (→
\emph{sati}); investigazione dei \emph{dhamma} o stati mentali (→
\emph{dhamma-vicaya}); energia (→ \emph{viriya}): gioia (→ \emph{pīti});
tranquillità (\emph{passaddhi}); concentrazione o raccoglimento (→
\emph{samādhi}); equanimità (→ \emph{upekkhā}).

\emph{brahmacariyā}. Letteralmente, comportamento di Brahmā, ``condotta
divina'', vita pura; il termine è di solito riferito alla vita monastica
per enfatizzare il voto del celibato.

\emph{brāhmaṇa}. (1) Brahmano, membro della casta dei brahmani,
``sacerdote''. (2) La casta dei brahmani la quale in India ha per molto
tempo ritenuto che, per nascita, i suoi componenti fossero degni del più
alto rispetto. Il Buddha utilizzò il termine ``brahmano'' per applicarlo
a coloro che fossero riusciti a raggiungere il fine della vita
religiosa, la Liberazione, per mostrare che il rispetto non è frutto
della nascita, dell'appartenenza a una razza o a una casta, bensì di una
conquista spirituale. In senso buddhista, è sinonimo di → \emph{arahat}.

\emph{brahma-vihāra}. Le quattro dimore ``divine'' o ``sublimi'' che si
ottengono per mezzo dello sviluppo di un'illimitata → \emph{mettā}
(benevolenza, gentilezza amorevole), → \emph{karuṇā} (compassione), →
\emph{muditā} (gioia empatica e di apprezzamento) e → \emph{upekkhā}
(equanimità).

\emph{Buddha-sāsana}. La dottrina del Buddha; si riferisce in primo
luogo agli insegnamenti, ma anche a tutte le infrastrutture religiose,
grosso modo alla religione buddhista, al buddhismo nel suo complesso.

Buddha (\emph{Buddho}). Letteralmente, ``Risvegliato'', ``Illuminato''.
Questa parola viene anche usata per la meditazione, recitando
interiormente \emph{Bud-} nel corso dell'inspirazione e \emph{-dho}
durante l'espirazione.

\emph{caṅkama}. Letteralmente, ``camminata avanti e indietro'', per
indicare la meditazione di solito eseguita andando avanti e indietro su
di un sentiero prestabilito -- lungo circa 15 metri e largo circa 1
metro, delimitato all'inizio e alla fine da un oggetto o da un albero --
mentre si focalizza l'attenzione su di un oggetto di meditazione.

\emph{cetanā}. Intenzione, volizione. È l'atto mentale che precede
l'azione e che ha conseguenze sul → \emph{kamma}.

\emph{cetasika}. Fattore mentale che accompagna il → \emph{citta} o
mente; → \emph{vedanā}; → \emph{saññā}; → \emph{saṅkhāra.}

\emph{ceto-vimutti}. Liberazione della mente-cuore; → \emph{vimutti.}

\emph{chanda}. Desiderio, aspirazione, preferenza. È un termine neutro,
che può riferirsi a desideri sia salutari sia non salutari.

Cinque Precetti → Precetti.

\emph{citta}. Mente-cuore; stato di coscienza.

Colui che Conosce. La qualità della presenza mentale, quella facoltà
della mente che, se rettamente coltivata, conduce alla Liberazione.
Sotto l'influsso dell'ignoranza indotta dalle contaminazioni, le cose si
conoscono in modo erroneo. Addestrando Colui che Conosce per mezzo della
pratica del → Nobile Ottuplice Sentiero, si ottiene la conoscenza
risvegliata del Buddha.

\emph{dāna}. L'atto di donare, liberalità, generosità; fare offerte,
elemosine. Specificamente, offrire ai monaci i quattro beni di prima
necessità (cibo, abiti, riparo e medicinali). Più in generale, la
tendenza a donare, senza attendersi alcun genere di ricompensa da chi ha
ricevuto. \emph{Dāna} è il primo tema del sistema di addestramento
graduale del Buddha (→ \emph{ānupubbī-kathā}), la prima delle dieci →
\emph{pāramī}, uno dei sette tesori (→ \emph{dhana}) e la prima delle
tre basi delle azioni meritorie (→ \emph{sīla}; → \emph{bhāvanā}).

\emph{danta} (\emph{dantā}). Dente, una delle → trentadue parti del
corpo.

\emph{devadūta}. ``Messaggero divino''; nome simbolico per la vecchiaia,
la malattia e la morte e per il → \emph{samaṇa} (asceta mendicante).

Dhamma, \emph{dhamma}. È un termine difficilmente traducibile e con un
notevole numero di significati. Indica sia la dottrina del Buddha, la
realtà delle cose, l'ordine che governa l'universo, la legge morale;
sia, in senso tecnico e con la lettera minuscola, il fenomeno tanto
fisico quanto mentale, oppure solo lo stato mentale, l'oggetto mentale,
la caratteristica o la qualità.

\emph{dhamma} mondani. Le otto condizioni mondane di guadagno e perdita,
lode e biasimo, felicità e sofferenza, fama e discredito.

\emph{dhamma-savaṇa}. L'ascolto o lo studio del Dhamma.

\emph{dhamma-vicaya}. Investigazione dei \emph{dhamma} o stati mentali.

Dhamma-Vinaya. ``Dottrina e Disciplina'', il nome attribuito dal Buddha
a ciò che insegnava.

\emph{dhana.} Tesoro. I \emph{dhana} sono le sette qualità della fiducia
o fede (→ \emph{saddhā}), della virtù o moralità (→ \emph{sīla}), della
consapevolezza (→ \emph{sati}), del fervore ascetico (\emph{tapo}),
dell'apprendimento (\emph{ajjhesanā}), della generosità (\emph{cāga} o →
\emph{dāna}); della saggezza (\emph{mati} o → \emph{paññā}).

\emph{dhātu}. Elemento, proprietà. Terra (nel senso di solidità), acqua
(liquidità), fuoco (calore) e vento (movimento). I sei elementi
comprendono, oltre ai quattro appena menzionati, anche lo spazio e la
coscienza.

\emph{dhutaṅga}. Pratica ascetica volontaria che i praticanti possono
intraprendere di tanto in tanto, oppure come impegno a lungo termine, al
fine di coltivare l'accontentarsi e purificare il → \emph{sīla}. Per i
monaci le pratiche di questo genere sono tredici; due riguardano
l'abito, cinque il cibo, cinque la dimora, e una la postura (conosciuta
come il \emph{dhutaṅga} dello sforzo): (1) usare solo vesti abbandonate
(\emph{paṃsukūla}); (2) usare un solo gruppo delle tre vesti
che compongono l'abito monastico (\emph{tecīvarika}); (3) fare la
questua (\emph{piṇḍapāta}); (4) non saltare neanche un donatore
o un'abitazione che si trovano sulla via della questua
(\emph{sapadānacārika}); (5) non mangiare più di una volta al giorno
(\emph{ekāsanika}); (6) mangiare solo dalla ciotola dell'elemosina,
mettendo tutto assieme (\emph{pattapiṇḍika}); (7) non accettare altro
cibo dopo la questua (\emph{khalupacchābhattika}); (8) vivere nella
foresta (\emph{āraññika}); (9) abitare sotto un albero
(\emph{rukkhamūla}); (10) vivere a cielo aperto, senza un riparo
(\emph{abbhokāsika}); (11) abitare nei cimiteri (\emph{susānika}); (12)
accontentasi del luogo in cui si dimora (\emph{yathāsantatika}); (13)
rinunciare a stare distesi (\emph{nesajjika}).

Dieci Precetti → Precetti

\emph{diṭṭhi}. Visione, opinione, convinzione, concezione. In generale
il termine è associato a una ``visione errata''; infatti, nel Canone in
pāli la parola \emph{diṭṭhi} da sola ha quasi sempre questo significato.
I principali tipi di \emph{diṭṭhi} sono due: \emph{sammā-diṭṭhi}, la
Retta Visione, il primo fattore del → Nobile Ottuplice Sentiero, e
\emph{micchā-diṭṭhi}, la visione errata.

\emph{dosa}. Avversione, odio. Uno dei principali inquinanti; →
\emph{kilesa}.

\emph{dukkha}. ``Dis-agio'', ``difficile da sopportare'',
insoddisfazione, sofferenza, insicurezza, instabilità, tensione; è una
delle Tre Caratteristiche di tutti i fenomeni condizionati; →
\emph{tilakkhaṇa}.

\emph{ekaggatā}. Unificazione mentale; il quinto fattore
dell'assorbimento meditativo. È la qualità mentale che consente
all'attenzione di rimanere raccolta e focalizzata sull'oggetto scelto
per la meditazione. Raggiunge la completa maturazione con lo sviluppo
del quarto livello dei → \emph{jhāna}.

Entrata nella Corrente (\emph{sotāpatti}). Evento tramite il quale si
diviene → \emph{sotāpanna}, il primo stadio dell' → Illuminazione.

fondamento della consapevolezza; → \emph{satipaṭṭhāna}.

\emph{glot} (in thailandese กลค). Ombrello con una zanzariera
tutt'intorno all'estremità, utilizzato dai monaci che intraprendono i →
\emph{dhutaṅga} sia per la meditazione sia come riparo; viene appeso ai
rami degli alberi così da potercisi sedere sotto, al riparo dagli
insetti; è un termine diverso rispetto a quello utilizzato per
l'ombrello dei laici, \emph{rom} (in thailandese ร่ม).

\emph{gotrabhū}. ``Cambio di lignaggio''. Il passaggio da una condizione
di essere ordinario (→ \emph{puthujjana}) a quella di Nobile Persona (→
\emph{ariya}-\emph{puggala}).

\emph{gotrabhū-citta}. Stato di coscienza del cambio di lignaggio (→
\emph{gotrabhū}).

\emph{gotrabhū-ñāṇa}. ``Conoscenza del cambio di lignaggio'':
intravedere il → \emph{Nibbāna} con la transizione dalla condizione di
essere ordinario (→ \emph{puthujjana}) a quella di Nobile Persona (→
\emph{ariya}-\emph{puggala}).

\emph{hiri-ottappa}. Coscienza morale e timore di compiere cattive
azioni. Questi due stati mentali, detti ``i guardiani del mondo'', sono
associati a tutte le azioni abili e salutari. \emph{Hiri} è un freno
interiore il quale ci fa astenere dal compiere azioni che metterebbero a
rischio il rispetto per noi stessi. \emph{Ottappa} è una salutare paura
di compiere azioni non salutari che potrebbero recare danno a noi stessi
o agli altri; → \emph{kamma}.

\emph{iddhipāda} (\emph{iddhipādā}). Base del potere psichico; sentiero
del successo spirituale. I quattro \emph{iddhipāda} sono lo zelo (→
\emph{chanda}), lo sforzo (→ \emph{viriya}), l'applicazione della mente
(→ \emph{citta}) e l'investigazione (→ \emph{vīmaṃsā}).

Illuminazione (\emph{bodhi}). L'Illuminazione o Risveglio si realizza
quando le Quattro Nobili Verità (→ \emph{ariya-sacca}) vengono
completamente comprese e realizzate. Sono trentasette i fattori che
contribuiscono al Risveglio (→ \emph{bodhi-pakkhiya-dhamma}). Quattro
sono le Nobili Persone negli stadi dell'Illuminazione: →
\emph{sotāpanna}, → \emph{sakadāgāmin}, → \emph{anāgāmin}, →
\emph{arahat}. Solo chi raggiunge l'ultimo stadio dell'Illuminazione ha
divelto tutte le catene (→ \emph{saṃyojana}) che lo legano a ulteriori
rinascite; → \emph{saṃsāra}; → \emph{vaṭṭa}.

\emph{indriya}. Le cinque facoltà spirituali: la fiducia (→
\emph{saddhā}), lo sforzo (→ \emph{viriya}), la consapevolezza (→
\emph{sati}), la concentrazione (→ \emph{samādhi}) e la saggezza (→
\emph{paññā}). Nei \emph{sutta} questo termine può essere riferito anche
alle sei basi sensoriali (\emph{āyatana}); →
\emph{bodhi-pakkhiya-dhammā}.

\emph{jhāna}. Assorbimento mentale; uno stato di forte concentrazione
focalizzata su una singola sensazione fisica (che conduce a un
\emph{rūpajjhāna}), oppure su di una nozione mentale (che conduce a un
\emph{arūpajjhāna}). Lo sviluppo dei \emph{jhāna} sorge dalla temporanea
sospensione dei cinque impedimenti (→ \emph{nīvaraṇa}) attraverso lo
sviluppo di cinque fattori mentali: applicazione dell'attenzione (→
\emph{vitakka}), mantenimento dell'attenzione (→ \emph{vicāra}), gioia
(→ \emph{pīti}), felicità (→ \emph{sukha}) e unificazione della mente (→
\emph{ekaggatā}).

\emph{jongrom}. Parola thailandese (จงกรม, เดินจงกรม) per il termine
pāli → \emph{caṅkama}.

\emph{kalyāṇajana}. Una persona buona, un essere virtuoso.

\emph{kalyāṇamitta}. Amico spirituale, maestro che consiglia o insegna
il Dhamma.

\emph{kāmacchanda}. Desiderio sensoriale; uno dei cinque impedimenti o
ostacoli (→ \emph{nīvaraṇa}) per il progresso spirituale.

\emph{kāmataṇhā}. Bramosia sensoriale.

\emph{kamma}. Atto intenzionale compiuto per mezzo del corpo, della
parola o della mente, il quale conduce sempre a un effetto
(\emph{kamma-vipāka}).

\emph{kammaṭṭhāna}. Letteralmente, ``base di lavoro'' o ``luogo di
lavoro'', metodo meditativo. Il termine si riferisce all'``occupazione''
di un meditante: più precisamente la contemplazione di alcuni temi di
meditazione, per mezzo della quale si può sradicare il potere esercitato
sulla mente dagli inquinanti (\emph{kilesa}), dalla brama (\emph{taṇhā})
e dall'ignoranza (\emph{avijjā}). Nella procedura dell'ordinazione
monastica, a ogni nuovo monaco è insegnata la contemplazione di cinque
parti del corpo: capelli (\emph{kesā}), peli (\emph{lomā}), unghie
(\emph{nakhā}), denti (\emph{dantā}) e pelle (\emph{taco}). In modo
esteso, questo \emph{kammaṭṭhāna} comprende trentadue temi di
meditazione (→ trentadue parti del corpo), i quali includono le ossa, il
sangue, alcuni organi interni del corpo e vari altri liquidi umorali; →
\emph{kāyagatāsati}.

\emph{kāmupādāna} → \emph{upādāna}.

\emph{karuṇā}. Compassione; empatia; l'aspirazione a trovare una via che
sia davvero salutare per se stessi e per gli altri. È una delle quattro
dimore divine; → \emph{brahmavihāra.}

\emph{kasiṇa}. Oggetto esterno di meditazione utilizzato per sviluppare
il \emph{samādhi}, ad esempio un piatto con dell'acqua, la fiamma di una
candela o un disco colorato.

\emph{kāyagatāsati}. Consapevolezza immersa nel corpo. Si tratta di un
termine che, a seconda dei \emph{sutta}, può essere riferito a varie
pratiche meditative, per esempio mantenere la mente attenta al respiro,
essere consapevoli della postura del corpo, essere consapevoli di quel
che si sta facendo con il corpo, analizzare il corpo nelle sue varie
parti, analizzare il corpo nelle sue varie componenti fisiche (→
\emph{dhātu}), contemplare il dato di fatto che il corpo è
inevitabilmente soggetto alla morte e alla dissoluzione.

\emph{kesa} (\emph{kesā}) → trentadue parti del corpo.

\emph{khandha} (\emph{khandhā}). Aggregato, insieme di elementi col
quale ci si identifica; le componenti fisiche e mentali della
personalità e dell'esperienza sensoriale in generale. I \emph{khandhā}
sono le cinque basi dell'attaccamento (→ \emph{upādāna}): forma (→
\emph{rūpa}), sensazione (→ \emph{vedanā}), percezione (→ \emph{saññā}),
formazioni mentali (→ \emph{saṅkhāra}) e coscienza (→ \emph{viññāna}).

\emph{khanti.} Pazienza, sopportazione. È una delle Dieci Perfezioni (→
\emph{pāramī}).

\emph{kilesa} (\emph{kilesā}). Contaminazione; inquinante mentale;
fattore mentale che oscura e contamina la mente. L'avidità (→
\emph{lobha}), l'avversione (→ \emph{dosa}) e la confusione (→
\emph{moha}) sono le tre principali radici (→ \emph{mūla}) non salutari
le quali si esprimono sotto varie forme che includono l'attaccamento, la
malevolenza, la rabbia, il rancore, l'ipocrisia, l'arroganza, l'invidia,
l'avarizia, la disonestà, la vanagloria, l'ostinazione, la violenza,
l'orgoglio, la presunzione, la fissazione, l'ipocrisia.

\emph{kusala}. Salutare, abile, buono, meritorio. Un'azione
caratterizzata da questa qualità morale (\emph{kusala-kamma}) produce
alla fine risultati felici e favorevoli. L'azione caratterizzata dal suo
opposto (\emph{akusala-kamma}) conduce al dolore. Entrambe sono azioni
intenzionali; → \emph{kamma}.

\emph{kuṭī}. La piccola dimora del monaco buddhista; una capanna.

\emph{lobha.} Avidità, desiderio non salutare. Una delle tre radici (→
\emph{mūla}) non salutari presenti nella mente (→ \emph{kilesa}).

\emph{lokavidū}. ``Conoscitore del mondo'', un epiteto del Buddha.

\emph{loma} (\emph{lomā}) → trentadue parti del corpo.

Luang Por (in thailandese หลวงพ่อ). ``Venerabile padre''; è
un'espressione utilizzata in Thailandia per rivolgersi ai monaci
anziani.

\emph{magga}. Sentiero. Più specificamente, il Sentiero che conduce alla
cessazione della sofferenza e della tensione. I quattro sentieri
trascendenti -- o meglio il sentiero con quattro livelli di affinamento
-- sono i sentieri di ``Chi è entrato nella Corrente'' (→
\emph{sotāpanna}), di ``Chi torna una sola volta'' (→
\emph{sakadāgāmin}), di ``Chi è senza ritorno'' (→ \emph{anāgāmin}) e
del ``Meritevole'' (→ \emph{arahat}); → Nobile Ottuplice Sentiero; →
\emph{phala}; → \emph{Nibbāna}.

\emph{mahat} (\emph{mahā-}). ``Grande, importante, superiore''. Nella
gerarchia ecclesiastica thailandese è il titolo acquisito dopo aver
sostenuto determinati esami in lingua pāli, avendo completato un
programma di studi almeno fino al terzo anno.

\emph{mahāthera}. ``Grande anziano''; titolo onorifico automaticamente
conferito a un \emph{bhikkhu} con almeno venti anni di vita monastica; →
\emph{thera.}

\emph{majjhimā}-\emph{paṭipadā}. Via di Mezzo; → Nobile Ottuplice
Sentiero.

\emph{māna}. Presunzione, orgoglio. È una delle cinque catene superiori
(→ \emph{saṃyojana}).

Māra. Letteralmente, ``Colui che fa morire'', divinità che cerca di
indurre il Buddha e i meditanti alla distrazione.

\emph{mettā}. Gentilezza amorevole, benevolenza, cordialità,
amichevolezza. Una delle quattro dimore divine (→ \emph{brahmavihāra}) e
una delle dieci perfezioni (→ \emph{pāramī}).

\emph{moha}. Confusione; ignoranza (→ \emph{avijjā}). Una delle tre
radici (→ \emph{mūla}) non salutari della mente.

\emph{muditā}. Gioia empatica e di apprezzamento. Il provare piacere per
la felicità e il successo degli altri. È una delle quattro dimore divine
(→ \emph{brahmavihāra}).

\emph{mūla}. Letteralmente, ``radice''. Le condizioni fondamentali della
mente che determinano la qualità morale, ossia salutare (→
\emph{kusala}) o non salutare (→ \emph{akusala}), delle azioni
intenzionali (→ \emph{kamma}). Le tre radici non salutari o
contaminazioni (→ \emph{kilesa}) sono l'avidità (→ \emph{lobha}),
l'avversione (→ \emph{dosa}) e la confusione (→ \emph{moha}); le tre
radici salutari della generosità (→ \emph{dāna}), amorevolezza (→
\emph{mettā}) e saggezza (→ \emph{paññā}) sono i tre contrari di esse.

\emph{nāga}. Categoria di esseri non umani dalle fattezze serpentine;
elefanti; uno degli epiteti del Buddha.

\emph{nakha} (\emph{nakhā}) → trentadue parti del corpo.

\emph{nāma}. Fenomeno mentale. È un termine che può riferirsi alla
sensazione (→ \emph{vedanā}), alla percezione (→ \emph{saññā}),
all'intenzione o volizione (→ \emph{cetanā}) e all'attenzione
(\emph{manasikāra}). Alcuni commentatori utilizzano questo termine per
far riferimento ai quattro aggregati (→ \emph{khandha}) mentali.

\emph{nāma-dhamma} (\emph{nāma-dhammā}). Fenomeno mentale; →
\emph{nāma}.

\emph{ñāṇadassana}. Conoscenza e visione, anche all'interno delle
Quattro Nobili Verità (→ \emph{ariya-sacca}).

\emph{ñāyapaṭipanna} (\emph{ñāyapaṭipanno}). Coloro la cui pratica è
connotata dalla conoscenza della Verità.

\emph{nekkhamma}. Rinuncia; libertà dal desiderio sensoriale. Una delle
dieci perfezioni (→ \emph{pāramī}).

\emph{Nibbāna} (sanscrito \emph{Nirvāṇa}). La Liberazione finale da ogni
sofferenza, lo scopo della pratica buddhista. La libertà della mente
dagli influssi impuri (→ \emph{āsava}), dagli inquinanti mentali o
contaminazioni (→ \emph{kilesa}) e dal ciclo della rinascita e del
divenire (→ \emph{vaṭṭa}), come pure da tutto quello che può essere
descritto e definito. Siccome questo termine indica anche lo spegnimento
di un fuoco, esso reca con sé pure il senso di acquietamento,
raffreddamento e pace. Secondo i principi della fisica insegnata ai
tempi del Buddha, un fuoco aderisce al suo combustibile; quando si è
spento, è libero. In alcuni contesti il ``\emph{Nibbāna} totale'' (→
\emph{Parinibbāna}) indica l'esperienza del → Risveglio; in altri,
addita il transito finale di un → \emph{arahat}.

\emph{nibbidā}. Disincanto, stanchezza; voltare abilmente le spalle al
mondo condizionato del → \emph{saṃsāra} per volgersi verso
l'incondizionato, il trascendente, il → \emph{Nibbāna}.

\emph{nimitta} (\emph{nimittaṃ}). Segno mentale, immagine o visione che
può sorgere durante la meditazione. \emph{Uggaha-nimitta} si riferisce
alle immagini che sorgono spontaneamente durante la meditazione.
\emph{Paribhāga}-\emph{nimitta} è un'immagine riflessa che può essere
soggetta a una manipolazione mentale.

\emph{nirodha}. Cessazione, dispersione, arresto.

\emph{nīvaraṇa}. Impedimento o ostacolo alla pratica meditativa della
concentrazione e al progresso spirituale. Essi sono cinque: il desiderio
sensoriale (→ \emph{kāmacchanda}), la malevolenza (\emph{vyāpāda},
\emph{abhijjhā}), la pigrizia e il torpore (\emph{thīna}-\emph{middha}),
l'agitazione e l'ansia (\emph{uddhacca-kukkucca}), il dubbio
(\emph{vicikicchā}).

Nobile Ottuplice Sentiero. Gli otto fattori della pratica spirituale che
conducono alla cessazione della sofferenza: Retta Visione, Retta
Intenzione, Retta Parola, Retta Azione, Retti Mezzi di Sussistenza,
Retto Sforzo, Retta Consapevolezza e Retta Concentrazione. È anche detto
``Via di Mezzo'' (\emph{majjhimā}-\emph{paṭipadā}) insegnata dal Buddha;
→ \emph{magga}.

Nobili Verità → \emph{ariya-sacca}.

\emph{ogha}. ``Inondazione''; un altro termine per indicare i quattro
influssi impuri (→ \emph{āsava}) della brama sensoriale (→ \emph{kāma}),
del divenire (→ \emph{bhava}), della visione errata (→ \emph{diṭṭhi}) e
dell'ignoranza (→ \emph{avijjā}).

\emph{opanayika} (\emph{opanayiko}). ``Che conduce all'interno'', degno
di essere realizzato e condotto all'interno della mente; un attributo
del Dhamma.

Otto Precetti → Precetti.

Ottuplice Sentiero → Nobile Ottuplice Sentiero.

\emph{pabbajjā}. Nei testi buddhisti in pāli indica il passaggio dalla
vita laica a quella di monaco privo di dimora, e può essere reso con
l'``abbandono'' della vita laica. \emph{Pabbajjā} è, appunto, un termine
utilizzato nella prima ordinazione d'ingresso nel Saṅgha, tramite la
quale si diventa novizi o \emph{sāmaṇera}; → \emph{upasampadā}.

\emph{paccatta} (\emph{paccattaṃ}). Da sperimentare individualmente e
personalmente (\emph{veditabba}) da parte dei saggi (\emph{viññūhi}).

\emph{Pacceka-buddha}. Un Buddha solitario. Una persona che, come il
Buddha, ha conseguito il → Risveglio senza beneficiare
dell'insegnamento di un maestro, ma che non possiede sufficienti →
\emph{pāramī} per insegnare agli altri la pratica che conduce
all'Illuminazione e, dopo averla realizzata, vive in solitudine.

\emph{pahkao} → \emph{anāgārika}.

\emph{paññā}. Saggezza, discernimento, visione profonda. Una delle dieci
perfezioni (→ \emph{pāramī}).

\emph{paññā-vimutti} → \emph{vimutti}.

\emph{paramattha-dhamma}. ``Verità o Realtà Ultima'', il Dhamma o i
\emph{dhamma} descritti in termini di significato ultimo, non di mera
convenzione.

\emph{pāramī}. ``Perfezione''. Un gruppo di dieci qualità sviluppate
attraverso molte vite da un → \emph{bodhisatta}: generosità (→
\emph{dāna}), virtù (→ \emph{sīla}), rinuncia (→ \emph{nekkhamma}),
discernimento/saggezza (→ \emph{paññā}), energia/costanza (→
\emph{viriya}), pazienza/sopportazione (→ \emph{khanti}), sincerità (→
\emph{sacca}), determinazione (→ \emph{adhiṭṭhāna}), gentilezza
amorevole (→ \emph{mettā}) ed equanimità (→ \emph{upekkhā}).

\emph{parinibbāna}. \emph{Nibbāna} completo o definitivo, un termine
associato alla morte fisica del Buddha.

\emph{pariyatti}. Comprensione teorica del Dhamma, ottenuta mediante la
lettura, lo studio, l'apprendimento; → \emph{paṭipatti}, →
\emph{paṭivedha}.

\emph{paṭiccasamuppāda}. Coproduzione condizionata, genesi
interdipendente. Una tabella che descrive il modo in cui i cinque
aggregati (\emph{khandha}) e le sei basi sensoriali (\emph{āyatana})
interagiscono dopo il contatto (\emph{phassa}) con l'ignoranza
(\emph{avijjā}) e con la brama (\emph{taṇhā}) per condurre alla tensione
e alla sofferenza (\emph{dukkha}).

\emph{Pātimokkha}. Il codice fondamentale della disciplina monastica,
che viene recitato ogni due settimane in lingua pāli e che consiste di
227 regole o precetti per i → \emph{bhikkhu} e di 331 per le →
\emph{bhikkhunī}; → \emph{Vinaya}.

\emph{paṭipadā}. Strada, via, sentiero; i mezzi per raggiungere lo scopo
o la destinazione finale, il → \emph{Nibbāna}. Di solito in riferimento
alla ``Via di Mezzo'' (\emph{majjhimā}-\emph{paṭipadā}), il → Nobile
Ottuplice Sentiero che conduce alla cessazione della sofferenza, in
quanto Sentiero della pratica descritto dalle Quattro Nobili Verità
(\emph{dukkha-nirodha-gāminī-paṭipadā}); → \emph{ariya-sacca}.

\emph{paṭipatti}. La pratica del Dhamma, opposta alla mera conoscenza
teorica; → \emph{pariyatti}, → \emph{paṭivedha}.

\emph{paṭivedha}. Realizzazione diretta, in prima persona, del Dhamma; →
\emph{pariyatti}, → \emph{paṭipatti}.

\emph{phala}. Frutto. Più specificamente, il Frutto di uno dei quattro
Sentieri della trascendenza o livelli dell' → Illuminazione; →
\emph{magga.}

\emph{phassa.} Contatto sensoriale; → \emph{paṭiccasamuppāda}.

\emph{pīti}. Gioia. Il terzo fattore dell'assorbimento meditativo (→
\emph{jhāna}).

Precetti. Le linee guida morali (→ \emph{sīla}) per azioni e pensieri
salutari. I Cinque Precetti per i laici consistono nell'astenersi da
uccidere altri esseri (I); astenersi dal prendere ciò che non è dato
(II); astenersi da una condotta sessuale scorretta (III); astenersi dal
mentire (IV); astenersi dall'assunzione di sostanze intossicanti (V).
Per gli → \emph{anāgārika}, si hanno gli Otto Precetti; oltre a quelli
appena menzionati, fermo restando che il precetto relativo ai costumi
sessuali si trasforma in astensione da qualsiasi attività sessuale
consapevole, si aggiungono i seguenti: astensione dall'assunzione di
cibo dopo mezzogiorno (VI); astensione dal danzare, cantare o comunque
da intrattenimenti e distrazioni, nonché dall'uso di ogni genere di
abbellimenti del corpo quali collane, orecchini e anelli, come pure da
cosmetici e profumi (VII); astensione dal dormire in letti lussuosi o
comunque ampi e troppo comodi (VIII). Per il → \emph{sāmaṇera} si hanno
Dieci Precetti: ai suddetti Otto se ne assommano altri due, o meglio
uno, relativo all'astensione dall'usare oro e argento o comunque valori
in genere e denaro (X); l'altro è il risultato della suddivisione del
VII precetto: qui il VII contempla l'astensione dal danzare, cantare o
comunque da intrattenimenti e distrazioni, mentre l'VIII comporta
l'astensione dall'uso di ogni genere di abbellimenti del corpo quali
collane, orecchini e anelli, come pure da cosmetici e profumi; il IX
corrisponde all'VIII, l'astensione dal dormire in letti lussuosi o
comunque ampi e troppo comodi. Per i → \emph{bhikkhu} i precetti o
regole sono 227 e per le → \emph{bhikkhunī} sono 331; tali precetti sono
dettagliatamente esposti nel → \emph{Pātimokkha}.

\emph{puthujjana}. Una persona comune, ordinaria, non illuminata; un
essere ``mondano'' che non ha ancora realizzato alcuna → Illuminazione;
→ \emph{ariya-puggala}, → \emph{magga}.

quattro basi dell'attaccamento → \emph{upādāna}.

quattro fondamenti della consapevolezza → \emph{satipaṭṭhāna.}

Quattro Nobili Verità → \emph{ariya-sacca}

Retta Visione → \emph{sammā-diṭṭhi}.

Risveglio → \emph{bodhi}.

\emph{rūpa}. Fenomeno fisico; dato sensoriale. Il significato basilare
di questo termine è ``forma''. È usato in vari contesti differenti, in
ognuno dei quali assume sfumature diverse. Nell'elenco degli oggetti dei
sensi, è indicato come oggetto del senso della vista. Come uno dei →
\emph{khandha}, è riferito ai fenomeni fisici, in quanto la visibilità o
la forma sono le caratteristiche che definiscono i fenomeni fisici.
Quest'ultimo è pure il significato che esso veicola quando viene usato
in opposizione ai fenomeni mentali (→ \emph{nāma}).

\emph{rūpa-dhamma}. Il mondo fisico, opposto a \emph{nāma-dhamma}; →
\emph{rūpa}, \emph{nāma}.

\emph{sabhāva}. Letteralmente, ``natura propria''. Principio o
condizione della natura, qualcosa che è come veramente è; →
\emph{sabhāva}-\emph{dhamma}.

\emph{sabhāva}-\emph{dhamma}. Fenomeno condizionato della natura;
qualsiasi fenomeno, proprietà o qualità in quanto sperimentata in se
stessa e di per se stessa. Nella Tradizione Thailandese della Foresta si
riferisce ai fenomeni naturali e alla visione profonda che sorge durante
lo sviluppo della pratica del Dhamma; → \emph{sabhāva}.

\emph{sacca}. Verità, sincerità. Una delle dieci perfezioni (→
\emph{pāramī}).

\emph{sacca-dhamma}. Verità Ultima; → \emph{sacca}.

\emph{saddhā}. Fiducia, fede. Una fiducia nel Buddha che fa sorgere la
volontà di mettere in pratica il suo insegnamento. La fede diviene
incrollabile allorché si raggiunge la condizione di → \emph{sotāpanna},
coincidente con il primo stadio dell' → Illuminazione.

\emph{sādhu}. È un'esclamazione che significa ``Bene!'' e che esprime
apprezzamento o che si è d'accordo.

\emph{sakadāgāmin} (\emph{sakadāgāmī}). Il secondo stadio
dell'Illuminazione, ``Chi torna una sola volta'' a esistere in forma
umana prima di conseguire l'Illuminazione, dopo aver distrutto le prime
tre catene inferiori e attenuato le altre due (→ \emph{saṃyojana}) che
legano la mente al ciclo della rinascita.

\emph{sakkāya-diṭṭhi}. Convinzione che induce l'identificazione con il
sé, con l'io. L'opinione che erroneamente identifica ogni →
\emph{khandha} come ``sé''. È la prima delle dieci catene (→
\emph{saṃyojana}); l'abbandono di \emph{sakkāya-diṭṭhi} è una delle
caratteristiche di ``Chi è entrato nella Corrente'' (→
\emph{sotāpanna}).

\emph{samādhi}. Concentrazione, unificazione della mente, stabilità
mentale; stato di calma concentrata che risulta dalla pratica di
meditazione.

\emph{samaṇa}. Un contemplativo. Letteralmente, chi abbandona gli
obblighi convenzionali della vita sociale per un modo di vivere più in
sintonia con la natura.

\emph{sāmaṇera}. Letteralmente, ``piccolo → \emph{samaṇa}'', un monaco
novizio che osserva Dieci → Precetti ed è candidato per l'ammissione
nell'Ordine dei → \emph{bhikkhu}; → \emph{pabbajjā}, →
\emph{upasampadā}.

\emph{sāmañña-lakkhaṇa}. Indica che tutto è identico nei termini delle
Tre Caratteristiche (→ \emph{tilakkhaṇa}), ossia impermanenza (→
\emph{anicca}), carattere insoddisfacente (→ \emph{dukkha}) e non-sé (→
\emph{anatta}).

\emph{samāpatti}. ``Ottenimento''. Termine che indica i quattro
assorbimenti immateriali, o i Frutti del Sentiero nei vari stadi dell' →
Illuminazione.

\emph{samatha}. Calma concentrata, tranquillità; → \emph{samādhi}, →
\emph{jhāna}.

\emph{sāmīcipaṭipanna} (\emph{sāmīcipaṭipanno}). Colui la cui pratica è
connotata da completa rettitudine o integrità.

\emph{sammā-diṭṭhi}. Retta Visione, il primo fattore del → Nobile
Ottuplice Sentiero, il Sentiero che conduce al \emph{Nibbāna}. Nel suo
significato più alto, avere Retta Visione significa comprendere le
Quattro Nobili Verità (→ \emph{ariya-sacca}).

\emph{sammuti}. Realtà convenzionale, convenzione, verità relativa,
supposizione; tutto quello che viene condotto a esistenza da parte della
mente.

\emph{sammuti-sacca}. Realtà convenzionale, dualistica o nominale; la
realtà dei nomi, delle determinazioni.

\emph{sampajañña}. ``Chiara comprensione'', consapevolezza di sé,
autorammemorazione, attenzione, consapevolezza, presenza mentale,
comprensione profonda. \emph{Sampajañña} è spesso usato in coppia con
\emph{sati}. Si potrebbe dire che \emph{sati} assiste come testimone con
una consapevolezza che osserva, ma è esente da ogni constatazione,
osserva semplicemente; \emph{sampajañña} è invece un genere di
consapevolezza che constata, è più circostanziata, meno immediata, più
``dialogata''; → \emph{sati}.

\emph{saṃsāra}. Flusso del Divenire o dell'Esistenza; un vagare
perpetuo, il continuo processo del nascere, invecchiare e morire. Ciclo
dei fenomeni condizionati, sia mentali sia materiali; → \emph{vaṭṭa}.

\emph{samudaya}. Origine, originazione, il sorgere; causa.

\emph{saṃyojana}. Catena che lega la mente alla ruota della rinascita (→
\emph{vaṭṭa}). Le cinque catene inferiori sono la convinzione che
conduce all'identificazione con il sé, con l'io (→
\emph{sakkāya-diṭṭhi}); il dubbio (→ \emph{vicikicchā}); l'attaccamento
ai riti e alle cerimonie/osservanze (→ \emph{sīlabbata-parāmāsa}); il
desiderio per gli oggetti dei sensi (→ \emph{kāma-rāga}); la malevolenza
(→ \emph{vyāpāda}). Le cinque catene superiori: il desiderio per la
forma (\emph{rūpa}-\emph{rāga}); il desiderio per i fenomeni privi di
forma (\emph{arūpa}-\emph{rāga}); la presunzione (→ \emph{māna});
l'agitazione (\emph{uddhacca}); l'ignoranza (→ \emph{avijjā}). Il →
\emph{sotāpanna} ha sradicato le tre catene \emph{sakkāya-diṭṭhi},
\emph{vicikicchā} e \emph{sīlabbata-parāmāsa}; il → \emph{sakadāgāmin}
ha solo attenuato le due catene \emph{kāma-rāga} e \emph{vyāpāda}; l'→
\emph{anāgāmin} ha del tutto distrutte queste ultime due catene; un →
\emph{arahat} ha eliminato le restanti cinque catene superiori; →
\emph{anusaya}, → \emph{saṃsāra}, → \emph{vaṭṭa}.

Saṅgha. A livello convenzionale (→ \emph{sammuti}), questo termine
indica la comunità dei monaci buddhisti e delle monache; a livello
ideale, indica quei seguaci del Buddha che, laici o monaci, hanno
raggiunto almeno l'``Entrata nella Corrente'' (→ \emph{sotāpanna}), il
primo dei sentieri (→ \emph{magga}) trascendenti che culminano nel →
\emph{Nibbāna}, e costituiscono così l' → \emph{ariya} Saṅgha\emph{.}

\emph{saṅghāti} → veste monastica.

\emph{saṅkhāra}. Formazione; fenomeno condizionato. Il termine può
essere riferito più specificamente alle formazioni di pensiero
all'interno della mente, uno dei cinque → \emph{khandha}.

\emph{saṅkhata-dhamma}. Fenomeno condizionato, realtà convenzionale, in
contrapposizione con l'incondizionato (\emph{asaṅkhata-dhamma}), ossia
il → \emph{Nibbāna}.

\emph{saññā}. Percezione; atto del riconoscere in base a un ricordo; →
\emph{khandha}.

\emph{sāsana}. Insegnamento, dispensazione, dottrina ed eredità del
Buddha; la scuola spirituale buddhista; → Dhamma-Vinaya.

\emph{sati}. Consapevolezza, presenza mentale, attenzione; il termine,
molto importante nella pratica meditativa buddhista, può significare
anche ``memoria''.

\emph{satipaṭṭhāna}. Fondamento della consapevolezza. I quattro
\emph{satipaṭṭhāna} sono esposti dettagliatamente nel
\emph{Mahāsatipaṭṭhāna-suttanta} (\emph{Dīgha-nikāya}, 22). Essi
consistono nella contemplazione del corpo (→ \emph{kāya}), la
contemplazione delle sensazioni (→ \emph{vedanā}), la contemplazione
della mente (→ \emph{citta}), la contemplazione degli oggetti mentali (→
\emph{dhamma}). Tali fondamenti vanno visti in sé e per sé man mano che
si presentano.

\emph{sekha}. Chi si sottopone all'addestramento spirituale; il termine
si riferisce ai sette \emph{ariya-sāvaka} o → \emph{ariya-puggala} che
non sono ancora diventati → \emph{arahat}. Tutti gli esseri non nobili
sono classificati come \emph{n'eva sekha nasekha}, ossia né in
addestramento né non addestrati; → \emph{asekha}.

Sette Fattori del Risveglio; → \emph{bojjhaṅga}.

Siddhattha Gotama. Il nome proprio del Buddha storico; nei testi
canonici più antichi si menziona il Buddha solo con il nome di Gotama.

\emph{sīla}. Virtù, moralità; precetto. Purezza morale la quale evita
che si compiano azioni non salutari. Si riferisce pure ai precetti
dell'addestramento che consentono di astenersi da azioni nocive.
\emph{Sīla} è il secondo argomento nell'addestramento graduale (→
\emph{anupubbī-kathā}), una delle dieci → \emph{pāramī}, il secondo dei
sette tesori (→ \emph{dhana}) e il primo dei tre livelli delle azioni
meritorie.

\emph{sīlabbata-parāmāsa} → \emph{saṃyojana}.

\emph{sīla-dhamma}. Un altro modo per indicare gli insegnamenti morali
del buddhismo. A livello personale, ``virtù (e conoscenza) della
verità''.

\emph{sotāpanna}. ``Chi è entrato nella Corrente'' e ha così conseguito
il primo livello dell'Illuminazione; il \emph{sotāpanna} ha abbandonato
le prime tre catene (→ \emph{saṃyojana}) che legano la mente al ciclo
della rinascita ed è perciò ``Entrato nella Corrente'' che
inesorabilmente fluisce verso il → \emph{Nibbāna}; egli non rinascerà
più di sette volte, e solo nel regno umano o in altri più elevati.

\emph{sukha}. Piacere; benessere; soddisfazione, felicità. Durante la
meditazione, una qualità della mente che raggiunge piena maturità con lo
sviluppo dei → \emph{jhāna}.

\emph{supaṭipanna} (\emph{supaṭipanno}). Colui che pratica bene.

\emph{sutta}. Letteralmente, ``filo''. Un discorso o sermone del Buddha
o dei discepoli suoi contemporanei. Dopo la morte del Buddha i
\emph{sutta} furono trasmessi oralmente per vari secoli e infine messi
per iscritto nello Sri Lanka. Secondo le cronache singalesi, il Canone
in pāli fu redatto nel periodo in cui regnò il sovrano Vaṭṭagamaṇi, tra
il 29 e il 17 a.C. Più di 10.000 \emph{sutta} sono presenti nel
\emph{Sutta-Piṭaka}, una delle principali raccolte scritte del buddhismo
del Therāvada (→ \emph{Tipiṭaka}). I \emph{sutta} in lingua pāli sono
considerati come le più antiche testimonianze degli insegnamenti del
Buddha.

\emph{Sutta-Piṭaka} → \emph{sutta}.

\emph{taca} (\emph{taco}) → trentadue parti del corpo.

\emph{taṇhā}. Letteralmente, ``sete''. Bramosia per gli oggetti dei
sensi, per l'esistenza o per la non esistenza; → \emph{bhava}, →
\emph{lobha}.

\emph{Tathāgata}. Letteralmente, ``così andato'', ``così venuto''. Un
termine utilizzato nell'antica India per una persona che ha realizzato
il più alto scopo spirituale. Nel buddhismo indica di solito il Buddha,
anche se talvolta può essere riferito ai suoi discepoli divenuti →
\emph{arahat}.

\emph{thera}. Letteralmente, ``anziano''; chi è monaco da almeno dieci
anni.

\emph{tilakkhaṇa}. Letteralmente, ``Tre Caratteristiche''. Le qualità di
tutti i fenomeni: impermanenza (→ \emph{anicca}), carattere
insoddisfacente (→ \emph{dukkha}) e non-sé (→ \emph{anatta}).

\emph{Tipiṭaka}. Il Canone buddhista in pāli. Letteralmente, i ``tre
canestri'', in riferimento alle tre principali suddivisioni del Canone:
il → \emph{Vinaya-Piṭaka} (le regole disciplinari), il →
\emph{Sutta-Piṭaka} (i discorsi) e l' → \emph{Abhidhamma-Piṭaka} (i
trattati filosofici).

\emph{tiratana}. La ``Triplice Gemma'', composta dal Buddha, dal Dhamma
e dal Saṅgha, ai quali tutti i buddhisti si rivolgono come a dei rifugi;
→ \emph{tisaraṇa}.

\emph{tisaraṇa}. Il ``Triplice Rifugio'', il Buddha, il Dhamma e il
Saṅgha; → \emph{tiratana}.

Tre Caratteristiche → \emph{tilakkhaṇa}.

trentadue parti del corpo. Un tema di meditazione il quale prevede che
si investighino le parti del corpo, quali i capelli (\emph{kesā}), i
peli (\emph{lomā}), le unghie (\emph{nakhā}), i denti (\emph{dantā}), la
pelle (\emph{taco}) e così via, in rapporto al loro essere non attraenti
(→ \emph{asubha}) e insoddisfacenti (→ \emph{dukkha}). La contemplazione
di queste cinque parti del corpo costituisce la prima tecnica meditativa
insegnata a un monaco o a una monaca appena ordinati dal loro
precettore.

\emph{tudong} (in thailandese ธุดงค์). La pratica ascetica di errare a
piedi, nelle campagne, in pellegrinaggio o alla ricerca di posti
tranquilli per ritiri solitari, vivendo di elemosina.

\emph{ujupaṭipanna} (\emph{ujupaṭipanno}). Colui la cui pratica è retta
o diretta.

\emph{upacāra-samādhi}. ``Concentrazione di accesso''; un livello di
concentrazione precedente i → \emph{jhāna}.

\emph{upādāna}. Attaccamento, aggrapparsi, aderire; è il sostegno per il
divenire e la nascita. Le quattro basi dell'attaccamento sono
\emph{kāmupādāna}, l'attaccamento agli oggetti dei sensi;
\emph{sīlabbatupādāna}, l'attaccamento a riti e osservanze;
\emph{diṭṭhupādāna}, l'attaccamento alle opinioni; e
\emph{attavādupādāna}, l'attaccamento all'idea del sé.

\emph{upāsaka}. Un fedele laico.

\emph{upasampadā}. Accettazione; ordinazione piena di un →
\emph{bhikkhu} o di una → \emph{bhikkhunī}; → \emph{pabbajjā}.

\emph{upāsikā}. Una fedele laica.

\emph{upekkhā}. Equanimità. È una delle quattro dimore divine (→
\emph{brahmavihāra}) e una delle dieci perfezioni (→ \emph{pāramī}).

\emph{uposatha}. Giorno di osservanza lunare, corrispondente alle fasi
lunari, durante il quale i laici buddhisti si riuniscono per ascoltare
il → Dhamma e per osservare gli Otto → Precetti. Negli \emph{uposatha}
di luna piena e di luna nuova i monaci si riuniscono per recitare le
regole del → \emph{Pātimokkha}.

\emph{uttarā-saṅgha} → veste monastica.

\emph{vaṭṭa}.~``Ciò che gira'', quel che va avanti, o è consueto, ossia
dovere, servizio, consuetudine. In contesto buddhista si riferisce al
ciclo della nascita, della morte e della rinascita. Ciò indica sia la
morte sia la rinascita degli esseri viventi sia la morte e la rinascita
degli inquinanti (→ \emph{kilesa}) all'interno della mente; →
\emph{saṃsāra}.

\emph{vedanā}. Sensazione. Può essere dolorosa
(\emph{dukkha}-\emph{vedanā}), piacevole (\emph{sukha}-\emph{vedanā}), o
né dolorosa né piacevole (\emph{adukkham-asukha-vedanā}); →
\emph{khandha}.

veste monastica. La veste monastica dei monaci \emph{theravādin} che
copre la parte superiore del corpo è un ampio rettangolo di stoffa (in
pāli \emph{uttarā-saṅgha}; in thailandese \emph{jeewon}, จีวร) che si
avvolge attorno al corpo e che spesso viene messo ad asciugare
dall'umidità e dal sudore al ritorno della questua. Vi è poi la parte
inferiore della veste, un rettangolo più piccolo indossato dalla vita in
giù (in pāli \emph{āntara-vāsaka}; in thailandese \emph{sabong}, สบง).
Oltre alla veste superiore e a quella inferiore vi è una veste esterna a
doppio strato (in pāli \emph{saṅghāti}; in thailandese \emph{sanghati},
สังฆาฏิ) che in genere viene portata ripiegata lungo la spalla sinistra
in situazioni cerimoniali.

Via di Mezzo → Nobile Ottuplice Sentiero.

\emph{vibhavataṇhā}. Bramosia per la non esistenza; desiderio di non
divenire, di non essere.

\emph{vicāra}. Mantenimento dell'attenzione. Nella meditazione il
\emph{vicāra} è il fattore mentale che consente all'attenzione di
muoversi intorno all'oggetto di meditazione prescelto e di esplorarlo.
Il \emph{vicāra}, assieme al fattore che a esso si accompagna (→
\emph{vitakka}), raggiunge la piena maturità con lo sviluppo dei →
\emph{jhāna}.

\emph{vihāra}. Un'abitazione, un luogo in cui dimorare. Di solito si
riferisce al luogo in cui dimorano i monaci, ossia un monastero.

\emph{vijjā}. Conoscenza genuina, più specificamente facoltà cognitiva
sviluppata tramite la pratica di meditazione e il discernimento.

\emph{vīmaṃsā}. Investigazione, indagine; → \emph{iddhipādā}.

\emph{vimutti}. Liberazione, libertà dalle formazioni e dalle
convenzioni della mente. Nei \emph{sutta} si parla di Liberazione per
mezzo del discernimento o saggezza (\emph{paññā-vimutti}), quando si
descrive la mente di un → \emph{arahat}, che è libera da → \emph{āsava},
nonché di \emph{ceto-vimutti} (Liberazione per mezzo della
consapevolezza), che viene utilizzata per descrivere la soppressione
mondana dei → \emph{kilesa} durante la pratica dei → \emph{jhāna} e
delle quattro dimore divine (→ \emph{brahmavihāra}).

Vinaya. Il codice della disciplina monastica buddhista; letteralmente,
``che conduce fuori'', perché l'osservanza delle regole ``conduce
fuori'' dagli stati non salutari della mente. Si può aggiungere che esso
``conduce fuori'' anche dalla vita famigliare e dall'attaccamento al
mondo. L'essenza delle regole per i monaci è contenuta nel →
\emph{Pātimokkha}. L'unione tra il Dhamma e il Vinaya rappresenta il
cuore del buddhismo: ``Dhamma-Vinaya'', ``la Dottrina e la Disciplina'',
è la definizione attribuita dal Buddha al suo stesso insegnamento.

\emph{viññāna}. Coscienza, cognizione; l'atto di conoscere i dati
sensoriali e gli stati mentali che si presentano; → \emph{khandha}.

\emph{vipassanā}. Visione profonda di natura intuitiva dei fenomeni
fisici e mentali, e del loro sorgere e scomparire, vedendoli per quello
che in realtà sono in sé e per sé, nei termini delle Tre Caratteristiche
(→ \emph{tilakkhaṇa}) e in termini di sofferenza (→ \emph{dukkha}), di
origine della sofferenza e di cessazione della sofferenza (→
\emph{ariya-sacca}).

\emph{vipassanūpakkilesa}. ``Contaminazione della visione profonda''.
Esperienza intensa che può verificarsi durante la meditazione e che può
indurre a pensare che si sia raggiunta la fine del Sentiero. L'elenco
tradizionale comprende dieci elementi: l'aura (\emph{obhāsa}), la
conoscenza (\emph{ñāṇa}), la gioia (\emph{pīti}), la tranquillità
(\emph{passaddhi}), la felicità (\emph{sukha}), la risolutezza
(\emph{adhimokkha}), lo spronare la mente (\emph{paggaha}), l'evidenza
(\emph{upaṭṭhāna}), l'equanimità (\emph{upekkhā}), l'attaccamento alle
apparenze (\emph{nikanti}); → \emph{vipassanā}.

\emph{viriya}. Perseveranza, energia. È una delle dieci perfezioni (→
\emph{pāramī}), dei cinque poteri (→ \emph{bala}) e delle cinque facoltà
(→ \emph{indriya}); → \emph{bodhi-pakkhiya-dhamma}.

\emph{vitakka}. Applicazione dell'attenzione. Nella meditazione il
\emph{vitakka} è il fattore mentale per mezzo del quale l'attenzione
viene condotta sull'oggetto di meditazione prescelto. Il \emph{vitakka},
assieme al fattore che a esso si accompagna (→ \emph{vicāra}), raggiunge
la piena maturità con lo sviluppo dei → \emph{jhāna}.

\emph{yarm} (in thailandese ย่าม). Borsa tipica utilizzata dai monaci.



\begin{glossarydescription}

% === A ===

\item[anicca] (Pali) Impermanence: one of the \emph{three characteristics of
    existence} along with not-self (\emph{anattā}) and unsatisfactoriness
  (\emph{dukkha}).

% === B ===

\item[borapet] (Thai) Tinospora crispa. Heart-shaped moonseed or guduchi.
  An extremely bitter vine used as a prophylactic and treatment for malaria.

% === C ===

% === D ===

% === E ===

% === F ===

% === G ===

% === H ===

% === I ===

% === J ===

% === K ===

% === L ===

% === M ===

% === N ===

% === O ===

% === P ===

% === Q ===

% === R ===

% === S ===

% === T ===

% === U ===

% === V ===

% === W ===

\end{glossarydescription}

