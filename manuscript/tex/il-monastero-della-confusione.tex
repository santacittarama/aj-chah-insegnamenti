\chapter{Il monastero della confusione}

Non è importante restare o andare, lo è quel che pensiamo. Perciò, per
favore, tutti voi lavorate insieme, collaborate e vivete in armonia.
Questa dovrebbe essere l'eredità che lasciate al Wat Pah Nanachat Bung
Wai, il Monastero Internazionale della Foresta del distretto di Bung
Wai. Non lasciate che diventi il Wat Pah Nanachat Woon Wai, Monastero
Internazionale della Foresta della Confusione e del
Turbamento.\footnote{È uno dei giochi di parole preferiti di Ajahn Chah.}
Chiunque giunga qui dovrebbe aiutare a creare questa eredità.

Per come io vedo le cose, è per la loro fede che i laici ci stanno
fornendo stoffe per le vesti monastiche, cibo in elemosina, un luogo in
cui dimorare e medicine in giusta quantità. È vero, sono persone
semplici, di campagna, ma ci aiutano come meglio possono. Non vi
lasciate trasportare dalle vostre idee su come dovrebbero essere. Ad
esempio: «~Oh, cerco di insegnare a questi laici, ma mi fanno
inquietare. Oggi è il giorno di osservanza lunare e loro vengono a
prendere i precetti. Però, domani andranno a gettare le loro reti da
pesca. Berranno whisky. Fanno queste cose dove tutti possono vederli.
Poi, il prossimo giorno di osservanza lunare, verranno di nuovo qui.
Prenderanno i precetti e ascolteranno ancora il Dhamma, e poi andranno
nuovamente a gettare le loro reti, uccideranno ancora animali e berranno
di nuovo whisky.~»

Potete arrabbiarvi parecchio se pensate in questo modo. Riterrete che il
vostro impegno con i laici non sia affatto di beneficio. Oggi prendono i
precetti e domani gettano le reti da pesca. Un monaco senza molta
saggezza può scoraggiarsi, pensare che ha fallito e ritenere che il suo
lavoro non rechi frutti. Però, non è che i suoi sforzi non abbiano
risultati: sono quei laici che non hanno risultati. Ovviamente, un
qualche buon risultato c'è, quando ci si sforza di accrescere la virtù.
Così, che cosa dovremmo fare quando ci troviamo in una situazione del
genere e cominciamo a soffrire?

Contempliamo dentro noi stessi per notare come le nostre buone
intenzioni abbiano arrecato qualche beneficio e abbiano avuto un senso.
È solo che le facoltà spirituali di quelle persone non sono sviluppate.
Non sono ancora forti. Per adesso è così, ed è per questo che dobbiamo
continuare a offrire consigli. Se con questa gente rinunciamo e basta, è
probabile che diventino peggiori di quel che ora sono. Se continuiamo,
un giorno potrebbero maturare e riconoscere che le loro azioni sono
maldestre. Poi, proveranno un po' di rimorso e inizieranno a vergognarsi
di fare cose di quel genere. Ora la fede consente loro di offrirci
sostegno mediante offerte materiali, sostengono la nostra vita con
generi di prima necessità. Ho valutato tutto questo, si tratta di una
cosa importante. Non di una piccola cosa. Ci donano il cibo, le nostre
dimore, le medicine che curano le nostre malattie, non è una cosa di
poco conto. Noi stiamo praticando per conseguire il \emph{Nibbāna}.
Questo sarebbe molto difficile se non avessimo cibo per nutrirci. Come
potremmo sedere in meditazione? Come avremmo potuto costruire questo
monastero?

Dovremmo sapere quando le facoltà spirituali delle persone non sono
ancora mature. Che cosa dovremmo fare, allora? Siamo come chi vende le
medicine. Forse li avrete visti o sentiti, mentre guidano qui attorno e
i loro altoparlanti fanno pubblicità alle medicine contro varie
malattie. È la gente che ha forti mal di testa o la digestione difficile
che va a comprarle. Si può accettare denaro da chi acquista le medicine.
Non si può prendere denaro da chi non compra nulla. Si può essere essere
contenti se la gente compra qualcosa. Se gli altri restano nelle loro
case e non escono a comprare, non ci si dovrebbe arrabbiare con loro.
Non dovremmo criticarli. Se insegniamo alle persone, ma loro non
praticano correttamente, non dovremmo arrabbiarci con loro. Non fatelo!
Non criticateli, continuate invece a istruirli e a guidarli. Quando le
loro facoltà saranno sufficientemente mature, lo faranno. È proprio come
vendere medicine: dobbiamo continuare a svolgere il nostro lavoro.
Quando la gente avrà dei malanni che danno problemi, comprerà le
medicine. Coloro che non sentono la necessità di acquistarle, forse non
sono affetti da condizioni che li fanno soffrire. Perciò, non
preoccupatevi. Se continuate con questo atteggiamento, avrete risolto il
problema. Situazioni come queste esistevano anche ai tempi del Buddha.

Vogliamo farlo bene, ma per qualche ragione ora non possiamo ottenere
ciò che vogliamo. Sono le nostre stesse facoltà a non essere mature a
sufficienza. Le nostre \emph{pāramī} non sono ancora complete. È come un
frutto che sta ancora maturando sull'albero. Non potete costringerlo a
essere dolce. È ancora acerbo, è ancora piccolo e aspro semplicemente
perché non ha ancora terminato di crescere. Non potete costringerlo a
essere più grande, a essere dolce, a essere maturo: dovete lasciarlo
maturare secondo quella che è la sua natura. Man mano che il tempo passa
e che le cose cambiano, le persone possono raggiungere la maturità
spirituale. Man mano che il tempo passa il frutto crescerà, maturerà e
diventerà dolce da sé. Con questo atteggiamento potete sentirvi a vostro
agio. Se invece sarete impazienti e insoddisfatti, continuerete a
chiedervi: «~Perché questo mango non è ancora dolce? Perché è aspro?~» È
ancora aspro solo perché non è maturo. È la natura del frutto.

Nel mondo la gente è così. Questo mi fa pensare all'insegnamento del
Buddha sui quattro tipi di loto. Alcuni sono nel fango, altri ne sono
venuti fuori ma si trovano ancora sott'acqua, altri ancora sono sulla
superficie e, infine, ve ne sono di cresciuti fuori dall'acqua e
sbocciati. Il Buddha fu in grado di impartire i suoi insegnamenti a
persone tanto differenti perché comprese i loro differenti livelli di
sviluppo spirituale. Dovremmo riflettere su questo, e non sentirci
oppressi da quello che avviene qui. Pensate di essere come chi vende le
medicine. La vostra responsabilità consiste nel pubblicizzarle e nel
metterle a disposizione. Se qualcuno s'ammala, probabilmente verrà a
comprarle. Allo stesso modo, se le facoltà spirituali delle persone
matureranno a sufficienza, è probabile che la loro fede un giorno si
sviluppi. Non si tratta di una cosa che possiamo forzarli a fare.
Vedendo le cose in questo modo, tutto sarà a posto.

Vivere qui, in questo monastero, ha indubbiamente senso. Non è privo di
benefici. Tutti voi, per favore, praticate insieme in armonia e
amicizia. Allorché sperimentate degli ostacoli e incontrate la
sofferenza, rammentate le virtù del Buddha. Quale conoscenza realizzò il
Buddha? Cosa insegnò il Buddha? Cosa evidenziò il Buddha? Come praticò
il Saṅgha? Rammentare continuamente le qualità dei Tre Gioielli reca
molto beneficio. Che siate thailandesi o che proveniate da altre
nazioni, non è questo che conta. È importante essere in armonia gli uni
con gli altri e lavorare assieme. Viene gente da ogni parte del mondo
per visitare questo monastero. Quando le persone vengono al Wat Pah
Pong, le esorto a venire qui, a vedere il monastero, a praticare qui. È
l'eredità che state creando. Sembra che la gente abbia fede e ne sia
allietata. Perciò, non dimenticatevi di voi stessi. Dovreste guidare la
gente, non essere guidati dalla gente. Fate del vostro meglio per
praticare bene, date un saldo fondamento a voi stessi, e i buoni
risultati arriveranno.

Avete qualche dubbio sulla pratica che ora sentite il bisogno di
risolvere?

\section{Domande e risposte}

Domanda: Quando la mente non pensa molto, ma si trova in uno stato
oscuro e offuscato, c'è qualcosa che possiamo fare per rischiararla?
Oppure dovremmo solamente sederci in meditazione?

Risposta: Succede sempre o solo quando siedi in meditazione? Com'è
esattamente quest'oscurità? È una mancanza di saggezza?

D.: Quando siedo in meditazione non mi viene sonnolenza, la mia mente è
cupa, una specie di densità, di opacità.

R.: Così vorresti rendere la tua mente saggia, giusto? Cambia posizione
e fai molta meditazione camminata. Questa è una cosa da fare. Puoi
camminare per tre ore in una sola volta, fino a che non sei davvero
stanco.

D.: Faccio meditazione camminata per un paio d'ore al giorno, e di
solito quando la faccio ho un sacco di pensieri. Però, quel che proprio
mi preoccupa è lo stato di oscurità di quando siedo in meditazione.
Dovrei cercare solo di esserne consapevole e lasciar andare, oppure c'è
qualche rimedio al quale dovrei ricorrere per contrastarlo?

R.: Credo che le tue posture non siano equilibrate. Quando cammini hai
molti pensieri. Allora dovresti contemplare molto il pensiero
discorsivo, così la mente può retrocedere dal pensare. Non vi resterà
incollata. Non preoccuparti, però. Per ora aumenta il tempo che dedichi
alla meditazione camminata. Concentrati su questo. Se poi la mente vaga,
tirala fuori da questo stato e pratica qualche contemplazione, ad
esempio l'investigazione del corpo. L'hai mai praticata di continuo,
piuttosto che come una riflessione occasionale? Quando sperimenti questo
stato di oscurità, ne soffri?

D.: Per questo stato della mia mente mi sento frustrato. Non sto
sviluppando il \emph{samādhi} o la saggezza.

R.: Quando la mente è in questo stato, la sofferenza proviene dalla non
conoscenza. C'è un dubbio sul perché la mente è così. Il principio
importante nella meditazione è che qualsiasi cosa si presenti, non
bisogna dubitarne. Il dubbio aumenta la sofferenza. Se la mente è chiara
e sveglia, non dubitarne. È una condizione della mente. Se è cupa e
offuscata, non dubitarne. Continua solo a praticare con diligenza, senza
farti catturare dalle reazioni a quella condizione. Prendi atto e sii
consapevole dello stato della tua mente, non dubitarne. È solo quello
che è. Quando accogli dei dubbi e inizi ad aggrapparti e ad attribuire a
essi un significato, è allora che arriva l'oscurità. Quando pratichi,
questi stati mentali sono cose che s'incontrano man mano che si
progredisce. Non c'è bisogno di avere dubbi in proposito. Notali con
consapevolezza e continua a lasciar andare. E la sonnolenza? Durante la
tua seduta di meditazione è di più il tempo durante il quale sei
assonnato o quello durante il quale sei sveglio?

(Nessuna risposta.)

Forse ti è difficile ricordare che ti sei addormentato! Se avviene,
medita con gli occhi aperti. Non chiuderli. Puoi focalizzare il tuo
sguardo su un punto, come la luce di una candela. Non chiudere gli
occhi! Questo è un modo per rimuovere l'impedimento della sonnolenza.
Quando sei seduto, di tanto in tanto puoi chiudere gli occhi e, se la
mente è chiara e priva di sonnolenza, continua a sedere a occhi chiusi.
Se è cupa e assonnata, apri gli occhi e metti a fuoco un punto. È simile
alla meditazione con i \emph{kasiṇa}.\footnote{\emph{Kasiṇa}: Oggetto
  esterno di meditazione utilizzato per sviluppare la concentrazione.}
Facendo così, puoi rendere la mente sveglia e tranquilla. La mente
assonnata non è tranquilla, è oscurata da un impedimento e si trova
nell'oscurità.

Dovremmo parlare anche del sonno. Non si può andare avanti senza
dormire. È la natura del corpo. Se stai meditando e ti senti
insopportabilmente e assolutamente assonnato, consentiti di andare a
dormire. È un modo di sedare l'impedimento, quando esso ti travolge.
Altrimenti vai avanti a praticare, e tieni gli occhi aperti se hai la
tendenza a sentirti assonnato. Dopo un po' chiudi gli occhi, e controlla
il tuo stato mentale. Se la mente è limpida, puoi praticare a occhi
chiusi. Dopo un po', riposati. Alcuni lottano in continuazione contro il
sonno. Si costringono a non dormire, e il risultato è che quando fanno
meditazione il sonno li trascina via, si piegano su se stessi e siedono
in uno stato di incoscienza.

D.: Possiamo focalizzare l'attenzione sulla punta del naso?

R.: Va benissimo. Qualsiasi cosa sia appropriata per te va bene,
focalizzati su tutto quello con cui ti senti a tuo agio e che ti aiuta a
fermare la mente.

Succede così: può essere difficile capire se ci attacchiamo troppo agli
ideali e prendiamo le istruzioni che ci vengono date troppo alla
lettera. Quando pratichiamo una meditazione standard come la
consapevolezza del respiro, all'inizio dovremmo dire a noi stessi con
determinazione che ora stiamo per effettuare questa pratica meditativa,
e che stiamo per assumere quale nostro fondamento la consapevolezza del
respiro. Focalizziamo il respiro in tre punti: quando attraversa le
narici, nel torace e nell'addome. Quando l'aria entra, prima passa per
il naso, poi attraversa il torace e infine va verso l'addome. Quando
lascia il corpo, l'inizio è l'addome, la metà è il torace e la fine è il
naso. Ne prendiamo solo atto. Legare la consapevolezza a questi punti --
all'inizio, alla metà e alla fine delle inspirazioni e delle
inspirazioni -- è un modo per iniziare a controllare la mente.

Prima di cominciare, dovremmo sederci e lasciare che la mente si
rilassi. È come confezionare un abito con una macchina da cucire a
pedale. Quando stiamo imparando a usare la macchina da cucire,
all'inizio ci sediamo di fronte a essa solo per sentirci a nostro agio,
per familiarizzare. Nel nostro caso, ci sediamo e respiriamo. Non
fissiamo la nostra consapevolezza su nulla, stiamo respirando, prendiamo
solo atto di questo. Prendiamo atto di quanto il respiro sia rilassato,
di quanto sia lungo o corto. Dopo averlo notato, iniziamo a focalizzare
l'attenzione sull'inspirazione e sull'espirazione in questi tre punti.
Continuiamo in questo modo finché diventiamo abili e la pratica procede
con fluidità. La fase successiva consiste nel focalizzare la
consapevolezza solo sulla sensazione del respiro sulla punta del naso o
sul labbro superiore. A questo punto non ci interessa più se il respiro
è lungo o corto, ci focalizziamo solo sulla sensazione del respiro che
entra ed esce.

Vari sono i fenomeni che possono entrare in contatto con i sensi, oppure
possono sorgere pensieri. Questo è chiamato pensiero iniziale
(\emph{vitakka}). La mente richiama un'idea relativa alla natura dei
fenomeni composti (\emph{saṅkhāra}), al mondo oppure a qualsiasi altra
cosa. Appena la mente la richiama, vorrà esserne coinvolta e fondersi
con essa. Se si tratta di un oggetto salutare, lascia che la mente
assuma tale oggetto. Se si tratta di un oggetto non salutare, fermala
immediatamente. Se è un oggetto salutare, lascia che la mente lo
contempli, e seguiranno letizia, appagamento e felicità. La mente è
luminosa e chiara quando il respiro entra ed esce e quando la mente
assume come oggetti di contemplazione questi pensieri iniziali. In
seguito il pensiero iniziale diventa pensiero discorsivo
(\emph{vicāra}). La mente sviluppa familiarità con l'oggetto,
esercitandosi e fondendosi con esso. A questo punto non c'è sonnolenza.

Dopo un adeguato lasso di tempo riporta l'attenzione sul respiro. Man
mano che continui, ci sarà pensiero iniziale e pensiero discorsivo,
pensiero iniziale e pensiero discorsivo. Se stai contemplando abilmente
un oggetto, come la natura dei \emph{saṅkhāra}, la mente sperimenterà
una tranquillità più profonda e nascerà il rapimento. Ci sono
\emph{vitakka} e \emph{vicāra}, e questo conduce la mente alla felicità.
Ora non ci sarà pesantezza alcuna né sonnolenza. La mente non sarà cupa
se si pratica in questo modo. Sarà lieta ed estaticamente rapita. Questo
rapimento inizierà a diminuire e dopo un po' svanirà, e potrai così
tornare di nuovo al pensiero iniziale. La mente non si distrarrà da
esso, diverrà stabile e determinata. Puoi allora passare di nuovo al
pensiero discorsivo, e la mente si fonderà con esso. Quando stai
praticando una meditazione adatta al tuo temperamento e lo stai facendo
bene, qualunque sia l'oggetto da te scelto sopraggiungerà il rapimento
estatico: i peli del corpo si drizzeranno e la mente sarà rapita e
sazia. Quando è così, non c'è alcun torpore né sonnolenza. Non avrai
alcun dubbio. Avanti e indietro fra pensiero iniziale e pensiero
discorsivo, pensiero iniziale e discorsivo, numerose volte, e arriva il
rapimento estatico. Poi c'è \emph{sukha}.

Questo succede durante la meditazione seduta. Dopo averla praticata per
un po', puoi alzarti e fare la meditazione camminata. La mente può
sperimentare le stesse cose durante la meditazione camminata. Non è
assonnata, ha \emph{vitakka} e \emph{vicāra}, \emph{vitakka} e
\emph{vicāra}, e poi il rapimento. Non ci sarà alcun
\emph{nīvarana},\footnote{\emph{Nīvaraṇa}: Impedimento o ostacolo alla
  pratica meditativa della concentrazione e al progresso spirituale.} e
la mente sarà senza macchia. Qualsiasi cosa succeda, non ti preoccupare.
Quale che sia l'esperienza che possa capitarti di avere -- luce,
beatitudine o altro -- non c'è bisogno di dubitarne. Non aver dubbi a
proposito di queste condizioni della mente. Se la mente è cupa, se è
luminosa, non fissarti su queste condizioni, non attaccarti a esse.
Lasciale andare, disfatene. Continua a camminare, continua a notare cosa
sta succedendo senza attaccamenti o infatuazioni. Non soffrire per
queste condizioni della mente. Non aver dubbi su esse. Sono solo quel
che sono, seguono la strada dei fenomeni mentali. A volte la mente sarà
gioiosa, altre volte sarà triste. Ci può essere felicità o sofferenza,
possono essere impedimenti. Invece di dubitare, comprendi che le
condizioni della mente sono così. Qualsiasi cosa si manifesti, si
verifica per il maturarsi delle sue cause. In questo momento si sta
manifestando questa condizione: questo è quel che dovresti riconoscere.
Anche se la mente è cupa, non c'è bisogno di turbarsi. Se diventa
luminosa, non te ne rallegrare eccessivamente. Non aver dubbi su queste
condizioni della mente o sulle tue reazioni a esse.

Fai la tua meditazione camminata fino a quando sei davvero stanco, poi
pratica la meditazione seduta. Quando siedi, rendi la tua mente
determinata a sedersi, non stare a perdere tempo. Se ti senti assonnato,
apri gli occhi e metti a fuoco qualche oggetto. Cammina fino a che la
mente si separa dai pensieri ed è serena, poi siedi in meditazione. Se
sei sveglio e sereno, puoi chiudere gli occhi. Se ti senti di nuovo
assonnato, apri gli occhi e guarda un oggetto. Non cercare di farlo per
tutto il giorno e per tutta la notte. Quando hai bisogno di dormire,
fallo. Proprio come con il cibo: mangiamo una volta al giorno. Quando
arriva il momento, diamo del cibo al corpo. Per il bisogno di dormire è
lo stesso. Quando arriva il momento, consentiti di riposare un po'. Dopo
aver riposato per un tempo appropriato, alzati. Non lasciar che la mente
languisca nel torpore, ma alzati e lavora, inizia a praticare. Fai molta
meditazione camminata. Se cammini lentamente e la mente diventa opaca,
allora cammina velocemente. Impara a trovare l'andatura giusta per te.

D.: \emph{Vitakka} e \emph{vicāra} sono la stessa cosa?

R.: Se sei seduto in meditazione e all'improvviso il pensiero di
qualcuno affiora nella tua mente, questo è \emph{vitakka}, il pensiero
iniziale. Prendi il pensiero di quella persona e inizia a osservarlo
dettagliatamente. \emph{Vitakka} prende l'idea, \emph{vicāra} la
investiga. Ad esempio, prendiamo l'idea della morte e poi cominciamo a
riflettere su di essa: «~Io morirò, gli altri moriranno, ogni essere
vivente morirà; quando muoiono, dove vanno?~» Poi fermati! Fermati e
riporta di nuovo indietro la mente. Vai avanti per un po' in questo
modo, poi fermala di nuovo e torna alla consapevolezza del respiro. A
volte il pensiero discorsivo vagherà senza tornare indietro, perciò
dovrai fermarlo. Continua fino a quando la mente è luminosa e chiara. Se
pratichi \emph{vicāra} con un oggetto che ti è adatto, è possibile che
ti si rizzino i peli del corpo, che gli occhi ti lacrimino o che tu
possa provare una gioia estrema. Molte e diverse sono le cose che
capitano quando arriva il rapimento estatico.

D.: Questo può capitare con qualsiasi tipo di pensiero, oppure avviene
solo quando c'è uno stato di tranquillità?

R.: Avviene quando la mente è tranquilla. Non è l'ordinaria
proliferazione mentale. Ci si siede con la mente serena e poi arriva il
pensiero iniziale. Ad esempio, penso a mio fratello che è appena morto.
Oppure posso pensare ad alcuni altri parenti. Succede quando la mente è
tranquilla, la tranquillità non è una cosa certa, ma almeno per il
momento la mente è tranquilla. Dopo che questo pensiero iniziale è
arrivato, entro nel pensiero discorsivo. Si tratta di un filo di
pensieri abili e salutari, che induce la mente a essere felice e a
sentirsi a suo agio, e vi è l'estasi con le sue esperienze concomitanti.
Questo rapimento proviene dal pensiero iniziale e discorsivo che si
verifica in una condizione di quiete. Non dobbiamo assegnargli nomi come
primo \emph{jhāna}, secondo \emph{jhāna} e così via. Parliamo solo di
tranquillità.

Il fattore successivo è il piacere (\emph{sukha}). Quando infine la
tranquillità diviene più intensa, lasciamo cadere il pensiero iniziale e
quello discorsivo. Perché? Lo stato mentale diviene più raffinato e
sottile. \emph{Vitakka} e \emph{vicāra}, che sono relativamente
grossolane, svaniranno. Resterà solo il rapimento estatico accompagnato
da beatitudine e unificazione mentale. Quando si raggiunge il punto
massimo non ci sarà più nulla, la mente sarà vuota. Questa è la
concentrazione di assorbimento.

Non c'è bisogno di fissarsi su nessuna di queste esperienze né di
dimorare in esse. Si procederà naturalmente da una a quella successiva.
Si comincia con il pensiero iniziale e quello discorsivo, poi ci sono il
rapimento, la beatitudine, e l'unificazione mentale. Il rapimento viene
eliminato,\footnote{Nelle Scritture di solito si dice: «~con lo svanire
  del rapimento.~»} poi lo stesso avviene con la beatitudine e, infine,
restano solo l'unificazione mentale e l'equanimità. Significa che la
mente diventa sempre più tranquilla e i suoi oggetti diminuiscono
costantemente fino a che rimangono l'unificazione e l'equanimità. È
questo che può succedere quando la mente è tranquilla e focalizzata. È
l'energia della mente, lo stato mentale di quando si raggiunge la
tranquillità. Quando si è in questa condizione non c'è nessuna
sonnolenza. Non può entrare nella mente, si dileguerà. Non saranno
presenti nemmeno gli altri impedimenti: il desiderio sensoriale,
l'avversione, il dubbio, l'irrequietezza e l'agitazione. Benché possano
essere ancora latenti nella mente del meditante, ora non si presentano.

D.: Dovremmo chiudere gli occhi per tagliare fuori l'ambiente esterno,
oppure entrare in rapporto con le cose così come le vediamo? Gli occhi è
importante tenerli aperti o chiusi?

R.: Appena cominciamo ad addestrarci è importante evitare troppi stimoli
sensoriali, ed è perciò meglio chiudere gli occhi. Non vedendo oggetti
che possano distrarci e esercitare un influsso su di noi, incrementiamo
la forza della mente. Quando la mente è forte possiamo aprire gli occhi
e, qualsiasi cosa vediamo, non ne siamo dominati. Non è importante
tenere gli occhi aperti o chiusi. Di norma, quando ci si riposa si
chiudono gli occhi. Sedere in meditazione con gli occhi chiusi è la
dimora di un praticante. In ciò proviamo piacere e riposo. Per noi è un
fondamento importante. Però, quando non sediamo in meditazione saremo in
grado di affrontare le cose? Sediamo con gli occhi chiusi e ne ricaviamo
beneficio. Quando apriamo gli occhi e abbandoniamo la meditazione
formale, dobbiamo essere in grado di gestire tutto quel che incontriamo.
Le cose non ci sfuggiranno di mano. Non saremo disorientati. In fondo,
stiamo solo gestendo le cose. È quando torniamo a praticare la nostra
meditazione seduta che potenziamo la nostra saggezza.

Sviluppiamo la pratica in questo modo. Quando essa raggiunge la
completezza, non importa se teniamo gli occhi aperti o chiusi, è uguale.
La mente non cambierà né sbanderà. In tutti i momenti della giornata --
al mattino, a mezzogiorno o di notte -- lo stato della mente sarà
uguale. Noi dimoriamo in questo modo. Non c'è nulla che possa scuotere
la mente. Quando sorge la felicità, prendiamo atto che ``non è cosa
certa'', ed essa passa. Sorge l'infelicità, prendiamo atto che ``non è
cosa certa'', e questo è tutto. Vi viene l'idea che volete lasciare
l'abito monastico. Non è cosa certa. Però pensate che sia cosa certa.
Prima volevate essere ordinati monaci, ed eravate sicuri al riguardo.
Adesso siete sicuri di voler lasciare l'abito monastico. Tutto è
incerto, ma non lo capite a causa dell'oscurità presente nella vostra
mente. La vostra mente vi sta mentendo: «~Se resto qui, spreco solo
tempo.~» Se lasciate l'abito monastico e tornate nel mondo, lì non
sprecherete tempo? A questo non pensate. Lasciando l'abito monastico per
lavorare in campi e orti, per far crescere fagioli o allevare maiali e
capre, questo non sarà una perdita di tempo?

C'era un grande stagno che brulicava di pesci. Col passare del tempo, la
pioggia diminuì e nello stagno rimase poca acqua. Un giorno sulla riva
si presentò un uccello. Disse ai pesci: «~Mi dispiace davvero per voi.
L'acqua riesce a malapena a bagnarvi il dorso. Sapete che non molto
lontano c'è un lago grande e profondo molti metri, dove i pesci possono
allegramente nuotare?~» Quando i pesci nello stagno in secca sentirono
queste cose, si entusiasmarono. Dissero all'uccello: «~Che bello! Come
facciamo ad arrivare lì?~» L'uccello rispose: «~Non ci sono problemi, vi
posso trasportare uno alla volta nel mio becco.~» I pesci discussero fra
loro la cosa. «~Qui non si sta più bene. L'acqua non ci copre nemmeno la
testa. Dovremmo andare.~» Così, si misero in fila per essere
trasportati. L'uccello prese un pesce per volta. Volò via, e appena non
poté più essere visto dallo stagno, atterrò e mangiò il pesce. Poi tornò
allo stagno e disse ai pesci: «~Proprio in questo momento il vostro
amico nuota felice nel lago e mi ha detto di chiedervi quando lo
raggiungerete!~»

Ai pesci parve meraviglioso. Non riuscivano ad aspettare e perciò
cominciarono a spingere per arrivare in cima alla fila. L'uccello se la
cavò in questo modo con i pesci. Poi tornò allo stagno per vedere se
poteva racimolare ancora qualcosa. C'era solo un granchio. Ricominciò
con i suoi discorsi da imbonitore a proposito del lago. Il granchio era
scettico. Chiese all'uccello come sarebbe potuto giungervi. L'uccello
gli rispose che l'avrebbe trasportato nel suo becco. Però, il granchio
aveva un po' di saggezza. Disse all'uccello: «~Facciamo così: starò
sulla tua schiena e ti metterò le zampe attorno al collo. Se farai
brutti scherzi, ti strangolerò con le mie chele.~» L'uccello restò
deluso, ma decise di tentare, pensando che in qualche modo sarebbe
riuscito a mangiarsi il granchio. Così, il granchio salì sulla sua
schiena e si levarono in volo. L'uccello girò lì attorno alla ricerca di
un posto per atterrare. Però, non appena provava ad atterrare, il
granchio iniziava a stringergli la gola con le chele. L'uccello non
riusciva neanche a gridare. Poteva solo emettere un suono secco e
gracchiante. Così, alla fine fu costretto a rinunciare e a riportare il
granchio allo stagno.

Spero che abbiate la saggezza del granchio! Se siete come quei pesci,
ascolterete quelle voci che vi dicono quanto sarebbe meraviglioso
tornare nel mondo. Si tratta di un ostacolo che i monaci incontrano. Per
favore, fate attenzione.

D.: Perché succede che gli stati mentali spiacevoli sono difficili da
vedere con chiarezza mentre quelli piacevoli si vedono con facilità?
Quando provo felicità o piacere posso vedere che si tratta di cose
impermanenti, ma quando sono infelice è più difficile.

R.: Cercando di capire pensi in termini di attrazione e avversione, ma
in realtà la radice predominante è l'illusione. Percepisci l'infelicità
come difficile da vedere e la felicità come facile da vedere. Si tratta
solo del modo in cui funzionano le tue afflizioni. L'avversione è
difficile da lasciar andare, vero? È una sensazione forte. Dici che la
felicità è facile da lasciar andare. In realtà, facile non è. È solo
meno insopportabile. Piacere e felicità sono cose che piacciono alla
gente, sono cose con le quali si sente a suo agio. Non sono così facili
da lasciar andare. L'avversione è dolorosa, ma la gente non sa come
lasciarla andare. La verità è che sono uguali. Quando contempli a fondo
e arrivi a un certo punto, riconosci subito e con chiarezza che sono
uguali. Se ci fosse una bilancia per pesarle, il loro peso sarebbe
uguale. Però noi incliniamo verso quel che è piacevole.

Stai dicendo che puoi lasciar andare facilmente la felicità e che
l'infelicità è invece difficile da lasciar andare? Pensi che sia facile
rinunciare alle cose che ci piacciono, ma ti stai chiedendo come mai sia
difficile rinunciare alle cose che non ci piacciono. Se però non sono
buone, perché è difficile rinunciarvi? Non è così. Pensa in un'altra
maniera. Sono del tutto uguali. È solo che non abbiamo la stessa
propensione nei loro riguardi. Quando c'è infelicità ci sentiamo
turbati, vogliamo fuggire velocemente e perciò sentiamo che è difficile
liberarsene. La felicità di solito non ci turba, e per questo facciamo
amicizia con essa e abbiamo la sensazione di riuscire a lasciarla andare
facilmente. Non è così. È che non opprime e non strizza il nostro cuore.
Questo è tutto. L'infelicità ci opprime. Pensiamo che una abbia più
valore o peso dell'altra, ma in realtà sono uguali. È come per il caldo
e il freddo. Possiamo essere bruciati a morte dal fuoco. Però anche il
freddo ci può congelare e fare morire ugualmente. Nessuno dei due è più
grande. Così è per la felicità e per la sofferenza, ma col pensiero
attribuiamo a esse differenti valori.

Prova a prendere in considerazione lode e biasimo. Pensi che la lode sia
facile da lasciar andare e che il biasimo sia difficile da lasciar
andare? In realtà sono uguali. Però, quando veniamo lodati non ci
sentiamo turbati. Siamo compiaciuti, non è un sentimento pungente. Il
biasimo è doloroso, e perciò pensiamo che sia difficile da lasciar
andare. Anche il compiacimento è difficile da lasciar andare, però lo
accogliamo con favore ed è per questo che non abbiamo lo stesso
desiderio di sbarazzarcene in fretta. Il piacere che proviamo
nell'essere lodati e il bruciore che sentiamo quando siamo criticati
sono uguali. Identici. Però, quando la nostra mente incontra queste
cose, reagiamo a esse in modo diverso. Non ci dispiace essere in
contatto con alcune di esse. Comprendilo, per favore. Nella nostra
meditazione sorgerà ogni genere di afflizioni mentali. La giusta
prospettiva consiste nell'essere pronti a lasciarle andare tutte quante,
sia quelle piacevoli sia quelle dolorose. Benché la felicità sia
qualcosa che desideriamo e la sofferenza qualcosa che non desideriamo,
riconosciamo che hanno lo stesso valore. Queste sono cose che
sperimenteremo. La gente nel mondo desidera la felicità. Non desidera la
sofferenza. Il \emph{Nibbāna} è al di là del desiderare e del non
desiderare. Capisci? Non c'è alcun desiderio legato al \emph{Nibbāna}.
Voler ottenere la felicità, voler essere liberi dalla sofferenza, voler
trascendere la felicità e la sofferenza: non c'è niente di tutto questo.
È pace.

Per come la vedo io, non succede che la Verità possa essere realizzata
facendo affidamento sugli altri. Dovreste capire che tutti i dubbi
dovranno essere risolti mediante i vostri stessi sforzi, per mezzo di
una pratica energica e costante. Non ci libereremo dal dubbio chiedendo
agli altri. Porremo fine al dubbio mediante i nostri stessi, inesorabili
sforzi. Ricordatevelo! Si tratta di un principio importante nella
pratica. L'impegno effettivo e concreto vi istruirà. Giungerete a
conoscere tutto quello che è giusto e tutto quello che è sbagliato. «~Il
brahmano porrà fine al dubbio mediante la pratica incessante.~» Non
importa dove andiamo: tutto può essere risolto per mezzo dei nostri
sforzi incessanti. Però non riusciamo a perseverare. Non riusciamo a
sopportare le difficoltà che incontriamo. Ci risulta difficile
affrontare la nostra sofferenza senza scappare. Se la affrontiamo e la
sopportiamo, la nostra conoscenza crescerà, e la pratica inizierà
automaticamente a istruirci, a insegnarci quello che è giusto e quello
che è sbagliato, e il modo in cui le cose realmente sono. La nostra
pratica ci mostrerà gli errori e i risultati nocivi del modo errato di
pensare. Succede davvero così. È però difficile trovare persone in grado
di capirlo. Tutti vogliono immediatamente il Risveglio. Scappare di qua
e di là seguendo i vostri impulsi vi farà solo sentire peggio. Stateci
attenti.

Ho spesso insegnato che la tranquillità è immobilità e che il fluire è
saggezza. Pratichiamo meditazione per calmare la mente e renderla
immobile, poi essa può fluire. Inizialmente impariamo com'è l'acqua
ferma, in seguito com'è l'acqua che scorre. Dopo aver praticato per un
po' vedremo come queste due cose siano di supporto l'una all'altra.
Dobbiamo rendere calma la mente, come acqua ferma. Poi essa scorre. È
ferma e scorre al tempo stesso. Non è una cosa facile da contemplare.
Possiamo capire che l'acqua ferma non scorre. Possiamo capire che
l'acqua che scorre non è ferma. Però, quando pratichiamo, succedono
entrambe le cose. La mente di un vero praticante è come acqua ferma che
scorre, oppure come acqua che scorre da ferma. Qualsiasi cosa succeda
nella mente di un praticante di Dhamma, è come l'acqua che scorre da
ferma. Dire che scorre soltanto non è corretto. Dire solamente che è
ferma non è corretto. Di solito, l'acqua ferma è ferma e l'acqua che
scorre, scorre. Però, quando avremo esperienza della pratica, la nostra
mente sarà in questa condizione: acqua ferma che scorre.

È una cosa che non abbiamo mai visto. Quando vediamo l'acqua che scorre,
scorre soltanto. Quando vediamo l'acqua ferma, non scorre. Però,
all'interno della nostra mente sarà proprio così, come acqua che scorre
da ferma. Nella nostra pratica del Dhamma abbiamo insieme
\emph{samādhi}, o tranquillità, e saggezza. Abbiamo moralità,
meditazione e saggezza. Ovunque sediamo, la mente è ferma e scorre.
Acqua ferma che scorre. Con la stabilità meditativa e con la saggezza,
con la tranquillità e la visione profonda, è così. Il Dhamma è così. Se
avete raggiunto il Dhamma, allora avrete sempre questa esperienza.
Essere tranquilli e avere saggezza. Scorre, ma è ferma. È ferma, ma
scorre. Ogni volta che nella mente di chi pratica avviene questo, si
tratta di qualcosa di strano e diverso. È una cosa diversa dalla mente
ordinaria che tutti conosciamo. Prima quando scorreva, scorreva. Quando
era ferma, non scorreva, era solo ferma: è così che la mente può essere
paragonata all'acqua. Ora è entrata in una condizione simile all'acqua
che scorre da ferma. In piedi, camminando, seduti o distesi, è come
acqua che scorre, ma è ferma. Se facciamo in modo che la nostra mente
sia così, vi è sia tranquillità sia saggezza.

Qual è il fine della tranquillità? Perché dovremmo avere saggezza? Al
solo scopo di liberarci dalla sofferenza, nient'altro. Attualmente
stiamo soffrendo, stiamo vivendo con \emph{dukkha}\footnote{\emph{Dukkha}:
  ``Dis-agio'', ``difficile da sopportare'', insoddisfazione,
  sofferenza, insicurezza, instabilità, tensione.} senza comprendere
\emph{dukkha} e, perciò, aggrappandoci a esso. Però, se la mente è nel
modo che vi ho descritto, ci saranno molti tipi di conoscenza. Si
conoscerà la sofferenza, si conoscerà la causa della sofferenza, si
conoscerà la cessazione della sofferenza e si conoscerà il Sentiero
della pratica per raggiungere la fine della sofferenza. Queste sono
Nobili Verità. Appariranno da sé quando vi sarà acqua ferma che scorre.

Quando succederà non ci distrarremo più, indipendentemente da cosa si
stia facendo. L'abitudine alla distrazione s'indebolirà e scomparirà.
Non cadremo nella distrazione, quale che sia la cosa di cui faremo
esperienza, perché la mente aderirà in modo naturale e serrato alla
pratica. Avrà timore di perdere la pratica. Quando continueremo a
praticare e a imparare dall'esperienza, ci abbevereremo sempre più al
Dhamma, e la nostra fiducia continuerà a crescere. Per chi pratica deve
essere così. Non dovremmo essere come quelli che si limitano a seguire
gli altri: se i nostri amici non stanno praticando, neanche noi lo
facciamo, perché ciò potrebbe metterci in imbarazzo. Se loro si fermano,
noi ci fermiamo. Se loro praticano, noi pratichiamo. Se l'insegnante ci
dice di fare qualcosa, lo facciamo. Se smette di dirlo, smettiamo di
farlo. Non è certo una via per raggiungere celermente la Realizzazione.

Qui, com'è che dobbiamo addestrarci? Quando siamo soli, siamo in grado
di continuare con la pratica. Ora, mentre viviamo qui insieme, quando ci
ritroviamo al mattino e alla sera per praticare, ci riuniamo e
pratichiamo con gli altri. Costruiamo un'abitudine, così che la via
della pratica venga interiorizzata nei nostri cuori, e allora saremo in
grado di vivere ovunque e di continuare a praticare nello stesso modo. È
come avere un certificato di garanzia. Se il re sta per
arrivare,\footnote{L'espressione risulta comprensibile qualora si tenga
  conto del fatto che da molti secoli la Thailandia è una monarchia; il
  nome ufficiale della nazione è ``Regno di Thailandia''(\emph{Ratcha
  Anachak Thai}: \thai{ราชอาณาจักรไทย}) e molti dei suoi sovrani sono stati
  benefattori e sostenitori del buddhismo.} prepariamo tutto come meglio
possiamo. Egli resta per un po', e poi se ne va per la sua strada, ma ci
dà il suo regio sigillo per attestare che qui è tutto in ordine. Ora
molti di noi stanno praticando insieme, ed è il momento di imparare bene
la pratica, per comprenderla e interiorizzarla in modo che ognuno di voi
sia il testimone di se stesso. È come quando si diventa maggiorenni.

