\chapter{Essere attenti}

Il Buddha insegnò a vedere il corpo nel corpo. Che cosa significa? A
tutti risultano familiari parti del corpo come i capelli, le unghie, i
denti e la pelle. Com'è allora che si vede il corpo nel corpo? Se
riconosciamo tutte queste cose come impermanenti, insoddisfacenti e
prive di un sé, è questo che chiamiamo ``vedere il corpo nel corpo''.
Non è necessario entrare nei dettagli e meditare sulle singole sue
parti. È come avere della frutta in una cesta. Se abbiamo già contato la
frutta sappiamo che cosa c'è nella cesta e, quando ne abbiamo bisogno,
possiamo prenderla e portarla via. Con essa verrà via anche tutta la
frutta. Sappiamo che la frutta è tutta lì, e perciò non abbiamo bisogno
di contarla di nuovo.

Dopo aver meditato sulle ``trentadue parti del corpo''\footnote{%
  ``Trentadue parti del corpo''. Un tema di meditazione il quale prevede che si
  investighino le parti del corpo, quali i capelli (\emph{kesā}), i peli
  (\emph{lomā}), le unghie (\emph{nakhā}), i denti (\emph{dantā}), la
  pelle (\emph{taco}) e così via, in rapporto al loro essere non
  attraenti (\emph{asubha}) e insoddisfacenti (\emph{dukkha}).}
e averle riconosciute come un qualcosa di non stabile o permanente, non abbiamo
più bisogno di stancarci separandole in questo modo per meditare così
dettagliatamente, così come non dobbiamo svuotare la cesta per ricontare
ogni volta la frutta. Però, portiamo la cesta verso la nostra
destinazione, camminando con consapevolezza e presenza mentale, facendo
attenzione a non inciampare e cadere.

Quando vediamo il corpo nel corpo, significa che in esso vediamo il
Dhamma e che, conoscendo il nostro stesso corpo e quello degli altri
come fenomeni impermanenti, non abbiamo bisogno di spiegazioni
dettagliate. Seduti qui, abbiamo la consapevolezza costantemente sotto
controllo, conosciamo le cose così come sono. Allora la meditazione
diventa piuttosto semplice. È come quando meditiamo su
\emph{Buddho}:\footnote{Buddha (\emph{Buddho}). Letteralmente,
  ``Risvegliato'', ``Illuminato''. Questa parola viene anche usata per
  la meditazione, recitando interiormente \emph{Bud-} nel corso
  dell'inspirazione e \emph{-dho} durante l'espirazione.} dopo averne
davvero compreso il significato -- piena conoscenza e ferma
consapevolezza -- non abbiamo bisogno di ripetere la parola ``Buddho''.
Questa è meditazione.

In genere, la meditazione non è ancora ben compresa. Pratichiamo in un
gruppo, ma spesso non sappiamo che cosa sia. Alcuni pensano che la
meditazione sia proprio difficile. «~Sono venuto in monastero, ma non
riesco a stare seduto. Non ho molta resistenza. Mi fanno male le gambe,
mi duole la schiena, sono dolorante dappertutto.~» Così ci rinunciano e
non vengono più, pensano di non riuscirci.

Nei fatti, però, il \emph{samādhi} non consiste nello stare seduti. Non
è camminare. Non è stare distesi né stare in piedi. Sedersi, camminare,
chiudere o aprire gli occhi sono semplici azioni. Tenere gli occhi
chiusi non necessariamente significa che stiate praticando
\emph{samādhi}. Potrebbe voler solo dire che siete assonnati e
intorpiditi. Se sedete a occhi chiusi ma state per addormentarvi, se la
testa ciondola qua e là e la mascella si spalanca come se fosse appesa,
questo non è sedersi in \emph{samādhi}. È stare seduti a occhi chiusi.
\emph{Samādhi} e occhi chiusi sono due cose diverse. Il vero
\emph{samādhi} può essere praticato sia con gli occhi aperti sia con gli
occhi chiusi. Potete essere seduti, camminare, stare in piedi o distesi.

\emph{Samādhi} significa che la mente è stabilmente focalizzata con
onnicomprensiva presenza mentale, con contenimento e circospezione.
Siete costantemente consapevoli di ciò che è giusto e di quello che è
sbagliato, osservate costantemente tutte le condizioni che sorgono nella
mente. Quando scatta un pensiero su qualcosa e provate uno stato d'animo
di avversione o di brama, ne siete consapevoli. Alcuni si scoraggiano.
«~Non ci riesco. Non appena mi siedo, la mente inizia a pensare a casa
mia. È un peccato.\footnote{\thai{บาป} (\emph{bààp}), in thailandese.}~» Ehi!
Se questa piccolezza fosse un peccato, il Buddha non sarebbe mai
diventato il Buddha. Egli passò cinque anni a lottare con la sua mente,
pensando a casa sua e alla sua famiglia. Fu solo dopo sei anni che
ottenne il Risveglio.

Altri ritengono che quest'improvviso emergere del pensiero sia una cosa
sbagliata, un male. Può capitarvi di sentire l'impulso di uccidere
qualcuno. Ma un istante dopo ne siete consapevoli, comprendete che
uccidere è sbagliato, perciò vi fermate e vi astenete dal farlo. Questo
è fare del male? Che ne pensate? Oppure pensate di rubare qualcosa, un
pensiero subito seguito dalla più forte rammemorazione che farlo è
sbagliato, e così vi astenete da quest'azione. È cattivo kamma?
Non è che ogni volta che avete un impulso accumulate istantaneamente un
kamma negativo. Come potrebbe esserci altrimenti una via per la
Liberazione? Gli impulsi sono solo impulsi. I pensieri sono solo
pensieri. Nel primo stadio del processo, non avete creato ancora nulla.
Nel secondo stadio, se agite su di esso col corpo, la parola o la mente,
allora state creando qualcosa. \emph{Avijjā}\footnote{%
  \emph{avijjā.} Non conoscenza, ignoranza; consapevolezza offuscata.}
ha assunto il
controllo. Se avete l'impulso di rubare e poi siete consapevoli sia di
voi stessi sia del fatto che ciò sarebbe sbagliato, questa è saggezza, e
al suo posto vi è \emph{vijjā}.\footnote{%
  \emph{vijjā.} Conoscenza
  genuina, più specificamente facoltà cognitiva sviluppata tramite la
  pratica di meditazione e il discernimento.}
L'impulso mentale non è attuato.

Questa è consapevolezza tempestiva, la saggezza che sorge e informa la
nostra esperienza. Se vi è il primo momentaneo pensiero di voler rubare
qualcosa e poi agiamo in base ad esso, questo è il \emph{dhamma}
dell'illusione; le azioni del corpo, della parola e della mente che
seguono l'impulso porteranno risultati negativi. Così stanno le cose.
Avere solo dei pensieri non è cattivo kamma. Se non abbiamo alcun
pensiero, come si svilupperà la saggezza? Alcuni vogliono solo stare
seduti con la mente vuota. Questa è errata comprensione.

Sto parlando del \emph{samādhi} accompagnato dalla saggezza. Infatti il
Buddha non pensava che fosse necessario molto \emph{samādhi}. Non voleva
\emph{jhāna}\footnote{%
  \emph{jhāna.} Assorbimento mentale; uno stato di
  forte concentrazione focalizzata su una singola sensazione fisica (che
  conduce a un \emph{rūpa-jhāna}), oppure su di una nozione mentale (che
  conduce a un \emph{arūpa-jhāna}).}
e \emph{samāpatti}.\footnote{%
  \emph{samāpatti.} ``Ottenimento''. Termine che indica i quattro assorbimenti
  immateriali, o i Frutti del Sentiero nei vari stadi
  dell'Illuminazione.}
Egli considerava il \emph{samādhi} come uno dei
fattori che compongono il Sentiero. \emph{Sīla}, \emph{samādhi} e
\emph{paññā} sono componenti o ingredienti, come gli ingredienti usati
per cucinare. Cucinando, si utilizzano delle spezie per insaporire il
cibo. Il punto non sono le spezie, ma il cibo che mangiamo. Praticare
\emph{samādhi} è lo stesso. I maestri del Buddha, Uddaka e ālāra,
enfatizzarono fortemente la pratica dei \emph{jhāna} e l'ottenimento di
vari generi di poteri, come la chiaroveggenza. Se andate così lontano,
tornare indietro è difficile. In alcuni posti si insegna questa profonda
tranquillità, a sedere deliziandosi nella quiete. Così i meditanti si
intossicano con il loro stesso \emph{samādhi}. Se hanno \emph{sīla}, si
intossicano con il loro \emph{sīla}. Se percorrono il Sentiero, si
intossicano con il Sentiero e, abbagliati dalle bellezze e dalle
meraviglie che sperimentano, non raggiungono la vera destinazione.

Il Buddha disse che si tratta di un errore sottile. Tuttavia, è
comprensibile per chi si trova a un livello grossolano. In verità, il
Buddha voleva che noi avessimo \emph{samādhi} in giusta misura, senza
restare bloccati lì. Dopo che ci esercitiamo e sviluppiamo il
\emph{samādhi}, allora il \emph{samādhi} dovrebbe sviluppare la
saggezza.

Il \emph{samādhi} che è al livello di \emph{samatha}, della
tranquillità, è come un sasso che copre l'erba. Nel \emph{samādhi}
sicuro e stabile la saggezza è presente anche quando gli occhi sono
aperti. Quando la saggezza è nata, comprende e conosce -- ``regola'' --
tutte le cose. È per questo motivo che il Maestro non volle quei
raffinati livelli di concentrazione e cessazione, perché essi si
trasformano in un diversivo che fa dimenticare il Sentiero. È necessario
non attaccarsi alla meditazione seduta né a qualsiasi altra postura in
particolare. Il \emph{samādhi} non significa tenere gli occhi chiusi o
aperti, oppure essere seduti, in piedi, camminare o stare distesi. Il
\emph{samādhi} pervade tutte le posture e ogni attività.

I più anziani, che spesso non possono stare seduti in modo perfetto,
riescono invece a contemplare particolarmente bene e a praticare il
\emph{samādhi} con facilità. Possono anche sviluppare molta saggezza.
Com'è che possono svilupparla? Tutto li travolge. Quando aprono gli
occhi non vedono le cose con la stessa chiarezza alla quale erano
abituati in passato. I denti danno loro problemi, e cadono. Il loro
corpo è per la maggior parte del tempo dolorante. Già questo rappresenta
un buon punto d'osservazione per indagare. Per questo la meditazione è
davvero facile per gli anziani. La meditazione è dura per i più giovani.
I loro denti sono forti, e perciò possono godersi il cibo. Dormono
profondamente. Hanno tutte le facoltà intatte e per loro il mondo è
divertente ed eccitante, e così s'illudono alla grande. Quando gli
anziani masticano qualcosa di duro provano subito dolore. Proprio in
quel momento i \emph{devadūta},\footnote{\emph{devadūta.} ``Messaggero divino''; nome simbolico per la
  vecchiaia, la malattia e la morte e per il \emph{samaṇa} (asceta
  mendicante).} che insegnano tutti i giorni agli anziani, stanno
parlando con loro. Quando aprono gli occhi, la loro vista è sfocata. Al
mattino duole loro la schiena. Di sera fanno male le gambe. È così!
Questo è davvero un eccellente oggetto di studio. I più anziani tra voi
diranno di non poter meditare. Su cosa volete meditare? Da chi
imparerete la meditazione?

Questo è vedere il corpo nel corpo e la sensazione nella sensazione. Le
state vedendo queste cose o state scappando da esse? Dire che non potete
praticare perché siete troppo vecchi è solo errata comprensione. La
questione è: le cose vi sono chiare? I più anziani hanno un sacco di
pensieri, un sacco di sensazioni, un sacco di disagio e di dolore. Tutto
si manifesta! Se meditano, lo possono testimoniare davvero. Per questo
affermo che la meditazione è facile per gli anziani. Possono meditare al
meglio. Dicono: «~Quando sarò anziano, andrò in monastero.~» Se lo
capite, è davvero tutto a posto. Lo avete visto dentro voi stessi.
Quando sedete, è vero. Quando vi alzate in piedi, è vero. Quando
camminate, è vero. Tutto è un fastidio, tutto presenta ostacoli, tutto
insegna. Non è così? Ora riuscite ad alzarvi subito e ad andarvene
facilmente? Se vi alzate: «~Ohi!~» O forse non lo avete notato? Ed è
«~ohi!~» pure quando camminate. Sono incitamenti.

Quando siete giovani potete alzarvi e camminare, andarvene per la vostra
strada. In realtà, però, non sapete nulla. Quando siete anziani, ogni
volta che vi alzate è un «~ohi!~» Non è così che dite? «~Ohi! Ohi!~»
Ogni volta che vi muovete, imparate qualcosa. Come potete allora dire
che è difficile meditare? Cos'altro c'è da osservare? È tutto giusto. I
\emph{devadūta} vi stanno dicendo qualcosa. È chiarissimo. I
\emph{saṅkhāra} vi stanno dicendo che non sono né stabili né permanenti,
né sono voi né vi appartengono. Ve lo stanno dicendo ogni momento. Noi,
però, non la pensiamo così. Pensiamo che non è giusto. Ospitiamo
l'errata visione e le nostre idee sono molto lontane dalla verità. In
realtà gli anziani possono vedere l'impermanenza, la sofferenza e
l'assenza di un sé e far sorgere il distacco e il disincanto, perché le
prove sono proprio lì, sempre dentro di loro. Questo è bene, penso.

Avere quella sensibilità interiore che è sempre consapevole del giusto e
dello sbagliato è detto \emph{Buddho}. Non è necessario ripetere
continuamente «~Buddho.~» Avete contato la frutta nella vostra cesta.
Ogni volta che sedete, non dovete prendervi il fastidio di tirar fuori
la frutta per contarla di nuovo. Potete lasciarla nella cesta. Qualcuno
però, con erroneo attaccamento, continuerà a contarla. Si fermerà sotto
un albero, la tirerà fuori, la conterà e la rimetterà nella cesta. Poi
camminerà fino al successivo luogo di sosta e lo farà di nuovo. Ma sta
solo contando la stessa frutta. Questa è solo bramosia. Ha paura che se
non la conterà, ci sarà un errore. Solo chi non sa quanta frutta c'è, ha
bisogno di contare. Quando lo sapete, potete stare tranquilli e
limitarvi a lasciarla nella cesta. Quando state sedendo, sedete e basta.
Quando siete distesi, giacete e basta, perché la vostra frutta è tutta
quanta con voi.

Praticando la virtù e creando meriti diciamo \emph{Nibbāna paccayo
hotu}, ossia che questo possa essere una condizione per realizzare il
Nibbāna. Per realizzare il Nibbāna fare offerte è bene.
Osservare i precetti è bene. Praticare meditazione è bene. Ascoltare gli
insegnamenti di Dhamma è bene. Possono diventare condizioni per
realizzare il Nibbāna. Ad ogni modo, che cos'è il Nibbāna?
Nibbāna significa non aggrapparsi. Nibbāna significa non
attribuire significato alle cose. Nibbāna significa lasciar
andare. Fare offerte e compiere azioni meritevoli, osservare i precetti
morali e meditare sulla gentilezza amorevole: tutto questo serve per
sbarazzarsi delle contaminazioni e della brama, per non desiderare
alcunché, per non desiderare di essere o di diventare qualcosa, per
rendere la mente vuota, vuota di auto-gratificazione, vuota dei concetti
del sé e di altro da sé.

\emph{Nibbāna paccayo hotu.} Che questo possa essere una condizione per
realizzare il Nibbāna. Praticare la generosità è rinunciare,
lasciar andare. Ascoltare gli insegnamenti serve ad acquisire la
conoscenza della rinuncia e del lasciar andare, per sradicare
l'attaccamento a ciò che è bene e a ciò che è male. Inizialmente
meditiamo per diventare consapevoli di quello che è sbagliato, di ciò
che è male. Quando lo comprendiamo, vi rinunciamo e pratichiamo ciò che
è bene. Poi, quando raggiungiamo un po' di bene, non ci attacchiamo a
quel bene. Restate a mezza strada nel bene o al di sopra del bene, non
dimorate sotto il bene. Se stiamo sotto il bene, il bene ci domina e noi
diveniamo suoi schiavi. Diventiamo schiavi, ed esso ci costringe a
creare ogni genere di kamma e di demeriti. Ci può condurre in
situazioni il cui risultato sarà lo stesso tipo di infelicità e di
circostanze sventurate nelle quali ci trovavamo in precedenza.

Rinunciare al male e sviluppare meriti, rinunciare a ciò che è negativo
e sviluppare ciò che è positivo. Sviluppare meriti e restare al di sopra
dei meriti. Restare al di sopra di merito e demerito, al di sopra di
bene e male. Continuate a praticare con una mente che sta rinunciando,
che sta lasciando andare e diventando libera. Vale sempre,
indipendentemente da quello che state facendo: se lo fate con una mente
che lascia andare, è una causa per la realizzazione del Nibbāna.
Quel che fate liberi dal desiderio, liberi dalle contaminazioni, liberi
dalla bramosia, tutto si fonde con il Sentiero e significa Nobile
Verità, è \emph{saccadhamma}.\footnote{\emph{saccadhamma.} Verità
  ultima.} Le Quattro Nobili Verità significano avere la saggezza che
conosce \emph{taṇhā},\footnote{\emph{taṇhā.} Letteralmente ``sete''.
  Bramosia per gli oggetti dei sensi, per l'esistenza o per la non
  esistenza.} la fonte di \emph{dukkha}.\footnote{\emph{dukkha.}
  ``Dis-agio'', ``difficile da sopportare'', insoddisfazione,
  sofferenza, insicurezza, instabilità, tensione.}
\emph{Kāmataṇhā} (bramosia sensoriale),
\emph{bhavataṇhā} (bramosia di divenire),
e \emph{vibhavataṇhā} (bramosia per la non esistenza) sono
l'origine, la fonte della sofferenza. Se state desiderando qualcosa o di
voler diventare qualcosa, state nutrendo \emph{dukkha}, state portando
\emph{dukkha} verso l'esistenza, perché è questo che fa nascere
\emph{dukkha}. Queste sono le cause. Se creiamo le cause per
\emph{dukkha}, allora \emph{dukkha} esisterà. La causa è
\emph{vibhavataṇhā}, questa brama inquieta e ansiosa. Così diventiamo
schiavi del desiderio e creiamo ogni tipo di kamma e di cattive
azioni, ed ecco che nasce la sofferenza. In parole semplici,
\emph{dukkha} è il figlio del desiderio. Il desiderio è il genitore di
\emph{dukkha}. Quando ci sono i genitori, \emph{dukkha} può nascere.
Quando non ci sono genitori, \emph{dukkha} non può esistere, non c'è
discendenza.

È su questo che si dovrebbe focalizzare la meditazione. Dovremmo vedere
tutte le forme di \emph{taṇhā} che ci inducono ad avere desideri.
Parlare del desiderio può indurre confusione. Alcuni pensano che ogni
genere di desiderio sia \emph{taṇhā}, anche il desiderio per il cibo o
per quanto è necessario per vivere. Possiamo però avere questo genere di
desideri in modo normale e naturale. Quando siete affamati e desiderate
del cibo, potete avere un pasto e farla finita. È del tutto normale.
Questo è un desiderio che non sconfina, è privo di cattivi effetti.
Questo tipo di desiderio non è bramosia sensoriale. Se è bramosia
sensoriale, allora diventa qualcosa di più del desiderio. Ci sarà
desiderio per molte cose da consumare, si andrà alla ricerca di sapori,
alla ricerca di piaceri in modi che inducono disagi e problemi, come
bere liquori e birra.

Alcuni turisti mi hanno raccontato di un posto nel quale la gente mangia
cervelli di scimmie vive. Mettono una scimmia al centro del tavolo e gli
aprono il cranio. Poi tirano fuori il cervello a cucchiaiate per
mangiarlo. Questo significa mangiare come demoni o spiriti affamati. Non
è mangiare in modo naturale, normale. Facendo cose di questo genere,
mangiare diviene \emph{taṇhā}. Dicono che il sangue della scimmia li
rende forti. Così, catturano questi animali e, quando li mangiano,
bevono pure liquori e birra. È il modo in cui fantasmi e demoni restano
infangati nella brama sensoriale. È mangiare carboni ardenti, mangiare
fuoco, mangiare di tutto ovunque. Questo genere di desiderio è
\emph{taṇhā}. Non c'è moderazione. Parlare, pensare, mangiare, vestirsi,
tutto ciò che questo tipo di persone fa è un eccesso. Se il nostro
mangiare, dormire e le nostre altre attività indispensabili sono svolte
con moderazione, esse non sono dannose. Dovreste perciò essere
consapevoli di voi stessi in relazione a queste cose, così non
diventeranno fonte di sofferenza. Se sappiamo essere moderati e
parsimoniosi nelle nostre necessità, potremo essere a nostro agio.

Praticare meditazione e generare meriti e virtù non sono poi cose così
difficili da fare, sempre che li si intenda bene. Che cosa significa
compiere azioni sbagliate? Che cos'è il merito? Merito è ciò che è buono
e bello, merito è non arrecare danno a noi stessi o agli altri con i
nostri pensieri, con le nostre parole e azioni. Se lo facciamo, vi è
felicità. Non si sta creando nulla di negativo. Il merito è così. Questa
è l'abilità.

Lo stesso avviene con le offerte e fare la carità. Quando diamo, che
cos'è che stiamo cercando di dare via? Dare serve a distruggere
l'auto-gratificazione, la credenza nell'esistenza di un sé e l'egoismo.
L'egoismo è una sofferenza potente ed estrema. Gli egoisti vogliono
sempre essere meglio degli altri e ottenere di più. Un semplice esempio:
dopo aver mangiato, non vogliono lavare i loro piatti. Lasciano che a
farlo sia qualcun altro. Se mangiano in gruppo, lo lasciano fare al
gruppo. Dopo aver mangiato, se ne vanno. Questo è egoismo, significa
essere irresponsabili e pesare sulle spalle degli altri. Ciò in realtà
equivale a non avere cura di se stessi, significa essere una persona che
non aiuta gli altri e che davvero non ama se stesso. Quando pratichiamo
la generosità stiamo cercando di purificare i nostri cuori da questa
attitudine. Questo è quel che si dice creare merito mediante il dare, al
fine di avere una mente compassionevole e che si occupa di tutti gli
esseri viventi, senza eccezioni.

Se riusciamo a liberarci anche solo da questa cosa, l'egoismo, saremo
come il Buddha. Egli non stava nel mondo per se stesso, cercava il bene
di tutti. Se il Sentiero e il suo Frutto sorgono nei nostri cuori, non
possiamo che progredire. Con questa libertà dall'egoismo tutte le
attività legate alle azioni virtuose, alla generosità e alla meditazione
condurranno alla Liberazione. Chiunque pratichi in questo modo diventerà
libero e andrà oltre, oltre ogni convenzione e apparenza.

I principi basilari della pratica non sono al di là della nostra
comprensione. Se ad esempio manchiamo di saggezza, quando pratichiamo la
generosità non ci sarà alcun merito. Se non lo comprendiamo, riteniamo
che la generosità significhi solo dare cose. «~Quando avrò voglia di
dare, darò. Se avrò voglia di rubare qualcosa, ruberò. Se poi mi sentirò
generoso, darò qualcosa.~» È come avere un barile pieno d'acqua. Ne
togliamo una secchiata e, con lo stesso secchio, ne riversiamo un'altra
dentro. Di nuovo una secchiata fuori e una dentro, ancora una fuori e
poi un'altra dentro, e così via. Quando si svuoterà il barile? Potete
vedere la fine? Potete pensare che un comportamente del genere divenga
una causa per la realizzazione del Nibbāna? Il barile si
svuoterà? Una secchiata fuori, una dentro. Riuscite a vedere quando sarà
finita?

\enlargethispage{\baselineskip}

Andare avanti e indietro in questo modo è \emph{vaṭṭa},\footnote{\emph{vaṭṭa.}
  ``Ciò che gira'', quel che va avanti, o è consueto, ossia dovere,
  servizio, consuetudine. In contesto buddhista si riferisce al ciclo
  della nascita, della morte e della rinascita.} la ciclicità stessa. Se
davvero stiamo parlando di lasciar andare, di rinunciare sia al bene sia
al male, vi è solo l'atto di svuotare. Anche se ce n'è solo un po',
svuotate. Non mettete dentro niente, continuate a svuotare. Pure se
avete solo una piccola paletta, fate tutto quello che potete e in questo
modo arriverà il momento che il barile sarà vuoto. Se ne togliete una
secchiata e poi ne riversate un'altra dentro, poi una fuori e di nuovo
un'altra dentro, bene, pensateci: quando vedrete il barile vuoto? Questo
Dhamma non è una cosa lontana. È proprio qui, nel barile. Potete farlo a
casa. Provateci. Potete svuotare un barile d'acqua in questo modo?
Domani fatelo per tutto il giorno e guardate che cosa accade.

«~Rinunciare a tutto il male, praticare ciò che è bene, purificare la
mente.~» Prima rinunciamo alle azioni sbagliate e poi cominciamo a
sviluppare il bene. Che cosa è bene e meritorio? È come un pesce
nell'acqua, per dirla in modo semplice. Se togliamo tutta l'acqua,
prenderemo il pesce. Se togliamo l'acqua e poi ne versiamo altra il
pesce resta nel barile. Se non eliminiamo le azioni sbagliate in ogni
loro forma, non vedremo il merito e non vedremo cos'è vero e giusto.
Togliendo e riversando, togliendo e riversando, restiamo solo come
siamo. Andando avanti e indietro in questo modo sprechiamo solo il
nostro tempo e tutto quel che facciamo non ha senso. Ascoltare gli
insegnamenti non ha senso. Fare offerte non ha senso. Tutti i nostri
sforzi di praticare sono inutili. Non comprendiamo i principi del
Sentiero del Buddha, e così le nostre azioni non producono i frutti
desiderati.

Quando il Buddha insegnò la pratica, non si rivolse solo a chi aveva
ricevuto l'ordinazione monastica. Stava parlando di praticare bene,
praticare correttamente. \emph{Supaṭipanno} significa coloro che
praticano bene. \emph{Ujupaṭipanno} significa coloro che praticano in
modo retto. \emph{Ñāyapaṭipanno} significa coloro che praticano per la
realizzazione del Sentiero, per la Fruizione e per il Nibbāna.
\emph{Sāmīcipaṭipanno} sono coloro che praticano in direzione della
Verità. Potrebbe trattarsi di chiunque. Costoro compongono il
\emph{Saṅgha} dei veri discepoli -- \emph{sāvaka} -- del Buddha.
\emph{Sāvaka} possono essere le donne laiche che vivono a casa.
\emph{Sāvaka} possono essere i laici. Portare queste qualità a
compimento è ciò che rende \emph{sāvaka}. Tutti possono essere veri
discepoli del Buddha e realizzare l'Illuminazione.

La maggior parte dei buddhisti non ha una comprensione così completa. La
nostra conoscenza non va così lontano. Intraprendiamo varie attività
pensando che otterremo un certo tipo di merito grazie a esse. Pensiamo
che ascoltare gli insegnamenti o fare offerte sia meritorio. Questo è
quel che ci viene detto. Ma chi fa offerte per ``ottenere'' meriti sta
producendo un cattivo kamma.

Non riuscite a comprenderlo del tutto. Chi dona per ottenere merito
accumula istantaneamente un cattivo kamma. Donare per lasciar
andare e per rendere la mente libera, questo porta merito. Se lo fate
per ottenere qualcosa, è cattivo kamma. È difficile ascoltare gli
insegnamenti e comprendere davvero la via del Buddha. Il Dhamma diventa
difficile da capire quando la pratica della gente -- osservare i
precetti, sedere in meditazione, donare -- mira a ottenere qualcosa in
cambio. Vogliamo i meriti, vogliamo qualcosa. Bene, se qualcosa può
essere ottenuto, chi lo ottiene? Noi. Quando quel qualcosa va perduto,
di chi è ciò che va perduto? E quando una cosa va perduta, chi ne
soffre? Chi non ha nulla, non perde nulla.

Pensate che vivere la vostra vita per ottenere cose non porti
sofferenza? Se è così, basta che andiate avanti come prima, cercando di
avere tutto. Se però rendiamo la mente vuota, otteniamo tutto. I più
alti regni, il Nibbāna e tutte le realizzazioni connesse:
otteniamo tutto questo. Nel fare offerte non abbiamo alcuno scopo o
attaccamento, la mente è vuota e rilassata. Possiamo lasciar andare. È
come trasportare un ceppo e lamentarci che è pesante. Se qualcuno vi
dice di posarlo a terra, dite: «~Se lo poso, non avrò nulla.~» Bene, ora
avete qualcosa, avete la pesantezza. Non la leggerezza. Allora, volete
la leggerezza o volete continuare a trasportarlo? Uno dice mettilo giù,
l'altro risponde che teme di non avere nulla. È un dialogo tra sordi.

Vogliamo la felicità, vogliamo il benessere, vogliamo la tranquillità e
la pace. Significa che vogliamo la leggerezza. Trasportiamo il ceppo e
poi, mentre lo trasportiamo, qualcuno ci vede e ci dice di lasciarlo
cadere. Rispondiamo che non possiamo. Altrimenti che cosa ci resterebbe?
L'altro però ci dice che se lo lasciamo cadere, otterremo qualcosa di
meglio. Hanno difficoltà a comunicare.

Se facciamo offerte e pratichiamo buone azioni per ottenere qualcosa,
non funziona. Ottenere significa divenire e nascita. Non è una causa per
la realizzazione del Nibbāna. Rinuncia e lasciar andare è
Nibbāna. Il Buddha voleva che guardassimo qui, in direzione di
questo spazio vuoto del lasciar andare. Questo è merito. Questa è
abilità. Quando abbiamo praticato ogni genere di merito e virtù dovremmo
sentire che abbiamo fatto la nostra parte. Non dovremmo portarli ancora
con noi. Lo facciamo allo scopo di abbandonare le contaminazioni e la
brama. Non lo facciamo per generare contaminazioni, brama e
attaccamento. Dove andremo, poi? Da nessuna parte. La nostra pratica è
corretta e vera.

Pur seguendo le forme della pratica e dell'apprendimento, alla maggior
parte di noi buddhisti risulta difficile comprendere questo discorso. È
perché \emph{Māra} -- che significa ignoranza e brama, ossia desiderio
di ottenere, di avere e di essere -- oscura la mente. Troviamo solo
felicità temporanea. Ad esempio, quando siamo pieni di odio nei riguardi
di qualcuno, esso s'impossessa della nostra mente e non ci dà pace.
Pensiamo sempre a quella persona, a come possiamo colpirla. Il pensiero
non si ferma mai. Forse un giorno avremo l'opportunità di andare a casa
sua, d'imprecargli contro e di rimproverarlo. Questo ci darà un po' di
sollievo. Ciò porrà forse fine alle nostre contaminazioni? Avremo
trovato un modo per scaricare la pressione e per questo ci sentiremo
meglio. Non avremo però liberato noi stessi dall'afflizione della
rabbia. O no? È possibile avere una certa qual felicità con le
contaminazioni e con la brama, ma in questo modo le cose vanno così:
stiamo ancora immagazzinando le contaminazioni dentro di noi e quando ci
saranno le condizioni opportune, esse divamperanno peggio di prima.
Allora vorremo di nuovo un sollievo temporaneo. In questo modo quando
finiranno mai le contaminazioni?

È come quando la moglie o i figli di qualcuno muoiono, o le persone
patiscono una grande perdita finanziaria. Per alleviare il dolore,
bevono. Per alleviare il dolore, vanno a vedere un film. Questo allevia
davvero il dolore? In realtà il dolore aumenta. Possono però
momentaneamente dimenticare quel che è successo, e ritengono che si
tratti di un modo di curare la loro infelicità. È come se aveste un
taglio sotto un piede che rende doloroso il camminare. Qualsiasi cosa
entri in contatto con il taglio fa male, e così zoppicate, lamentandovi
del disagio. Se però vedete una tigre sulla vostra strada, mettete le
ali e cominciate a correre senza pensare più al taglio. La paura della
tigre è molto più potente del dolore al piede, così è come se il dolore
se ne fosse andato. La paura l'ha rimpicciolito.

Al lavoro oppure a casa potete avere problemi che sembrano davvero
grandi. Allora vi ubriacate e, in questo stato di ubriachezza e di ancor
più potente illusione, quei problemi non vi turbano più tanto. Pensate
di aver risolto i vostri problemi, di aver alleviato la vostra
infelicità. Ma quando tornate sobri, i vecchi problemi tornano. Che ne è
della vostra soluzione? Continuate a reprimere i problemi bevendo, e
loro continuano a tornare. Potreste finire con l'ammalarvi di cirrosi
senza esservi liberati dei vostri problemi. Poi, un giorno arriva la
morte.

Si prova un certo qual benessere e una certa qual felicità con questi
comportamenti, è la felicità dei folli. È il modo in cui i folli fermano
la loro sofferenza. Non c'è saggezza. Queste differenti condizioni di
confusione sono mescolate nel cuore insieme a una sensazione di
benessere. Se consentiamo alla mente di seguire i suoi umori e le sue
tendenze, essa prova una certa felicità. Ma questa felicità accumula
continuamente dentro di sé dell'infelicità. Ogni volta che essa scoppia,
la nostra sofferenza e la nostra disperazione sono più forti. È come
avere una ferita. Se la curiamo solo in superficie, ma all'interno è
ancora infetta, non l'abbiamo ben curata. Sembra a posto per un po', ma
quando l'infezione si diffonde dobbiamo cominciare a tagliare. Se
internamente l'infezione non viene mai curata, possiamo farci operare
superficialmente in continuazione senza che sia possibile intravedere
una fine. Quel che si vede dall'esterno per un po' può sembrare a posto,
ma dentro è come prima.

Questa è la via del mondo. Le questioni del mondo non finiscono mai.
Così, nelle varie società le leggi risolvono problemi in continuazione.
Ne sono varate sempre di nuove per affrontare situazioni e problemi di
vario genere. Qualcosa viene sistemato per un po', ma sono sempre
necessarie ulteriori leggi e altre soluzioni. Non vi è mai una
risoluzione interna definitiva, solo dei miglioramenti alla superficie.
L'infezione esiste ancora all'interno, vi è perciò sempre bisogno di
altri tagli. La gente è buona solo in superficie, nelle loro parole e
nel loro aspetto. Le loro parole sono buone e i loro volti hanno un
aspetto gentile, ma le loro menti non sono tanto buone.

Saliamo sul treno, incontriamo un conoscente e diciamo: «~Oh! Che bello
vederti! Recentemente ti ho pensato molto! Volevo venirti a trovare!~»
Sono solo parole. Non lo pensiamo veramente. Siamo buoni in superficie,
ma dentro non poi così tanto. Diciamo quelle parole, ma dopo aver fumato
una sigaretta e preso una tazza di caffé, ci separiamo. Se in futuro, un
giorno, di nuovo incontreremo quel conoscente, diremo le stesse cose:
«~Ehi! Che bello vederti! Come stai? Ho pensato di venire a trovarti, ma
è che non ho avuto tempo.~» Così stanno le cose. La gente è buona in
superficie, ma di solito dentro non è poi tanto buona.

Il grande Maestro insegnò il Dhamma e il Vinaya.\footnote{%
  Vinaya. Il codice della disciplina monastica buddhista.}
Sono perfetti e
completi. Niente li supera, nulla necessita di essere modificato o
adattato, perché sono definitivi. Sono completi: perciò è il posto nel
quale possiamo fermarci. Non vi è niente da aggiungere o da eliminare,
perché sono cose che per natura non possono essere aumentate o
diminuite. È giusto così. È vero.

È per questa ragione che noi buddhisti veniamo ad ascoltare gli
insegnamenti di Dhamma e studiamo per imparare queste verità. Se le
conosciamo, le nostre menti entreranno nel Dhamma. Il Dhamma entrerà
nelle nostre menti. Quando la mente di una persona entra nel Dhamma,
quella persona prova benessere, la sua mente è in pace. La mente ha un
modo per risolvere le difficoltà e non può in alcun modo degenerare.
Quando il dolore e la malattia affliggono il corpo, la mente ha molte
maniere per sciogliere la sofferenza. Può scioglierla naturalmente,
comprendendola come fatto naturale senza cadere nella depressione o
nella paura. Se otteniamo qualcosa non ci perdiamo nella gioia. Se la
perdiamo non ci agitiamo troppo, piuttosto comprendiamo che è nella
natura di tutte le cose di declinare e scomparire dopo essere apparse.
Con quest'attitudine possiamo percorrere la nostra strada nel mondo.
Siamo \emph{lokavidū}, conoscitori del mondo, lo conosciamo con
chiarezza. \emph{Samudaya},\footnote{%
  \emph{samudaya.} Origine, originazione, il sorgere; causa.}
la causa della sofferenza, non viene
creata, e \emph{taṇhā} non nasce. C'è \emph{vijjā}, conoscenza delle
cose come sono in realtà, e ciò illumina il mondo. Illumina lode e
biasimo. Illumina guadagno e perdita. Illumina fama e discredito.
Nascita, vecchiaia, malattia e morte sono illuminate con chiarezza nella
mente del praticante.

Questo avviene a chi ha raggiunto il Dhamma. Costoro non lottano più con
la vita e non sono più continuamente alla ricerca di soluzioni.
Risolvono quel che può essere risolto, agendo in modo appropriato. Ciò
corrisponde agli insegnamenti del Buddha: insegnò a chi era possibile
insegnare. Abbandonò le persone alle quali non era possibile insegnare,
li lasciò andare: così, li lasciò cadere. Potreste perciò farvi l'idea
che il Buddha doveva mancare di \emph{mettā},\footnote{\emph{mettā.}
  Gentilezza amorevole, benevolenza, cordialità, amichevolezza.} per
abbandonare così le persone. Ehi! Se scartate un mango marcio, mancate
di \emph{mettā}? Non serve a nulla, questo è tutto. Con quella gente non
era possibile comunicare in alcun modo. Il Buddha è elogiato come essere
di suprema saggezza. Non metteva semplicemente insieme tutti e tutto, in
confuso disordine. Aveva l'Occhio Divino, poteva vedere con chiarezza
ogni cosa com'è in realtà. Era il Conoscitore del mondo.\footnote{\emph{lokavidū.}
  ``Conoscitore del mondo'', un epiteto del Buddha.}

In quanto Conoscitore del mondo, vedeva il pericolo insito nella ruota
del \emph{saṃsāra}. Per noi, suoi discepoli, è lo stesso. Conoscere
tutte le cose così come sono ci condurrà al benessere. Dove sono
esattamente le cose che ci causano felicità e sofferenza? Pensateci
bene. Sono solo create da noi stessi. Quando creiamo l'idea che qualcosa
siamo noi o che è nostra, soffriamo. Le cose possono arrecarci danno o
beneficio, dipende dalla nostra comprensione. Perciò il Buddha ci
insegnò a prestare attenzione a noi stessi, alle nostre azioni e alle
creazioni della nostra mente. Tutte le volte che proviamo amore o
avversione estremi nei riguardi di qualcuno o qualcosa, tutte le volte
che siamo particolarmente ansiosi, questo ci porterà verso una grande
sofferenza.

È importante, perciò osservate bene. Investigate questi sentimenti,
forte amore o forte avversione, e poi fate un passo indietro. Se vi
avvicinate troppo, vi morderanno. Avete sentito? Quando le afferrate e
le accarezzate, queste cose mordono e scalciano. Quando date l'erba al
vostro bufalo, dovete stare attenti. Se state attenti, quando scalcia
non vi colpirà. Dovete nutrirlo e prendervene cura, ma dovreste essere
abbastanza intelligenti da farlo senza essere morsicati. L'amore per i
figli, per i parenti, per i beni e per i possessi vi morderà. Lo capite?
Quando lo nutrite, non avvicinatevi troppo. Se gli date dell'acqua, non
avvicinatevi troppo. Tirate la fune, quando c'è bisogno di farlo. Questa
è la via del Dhamma: riconoscere l'impermanenza, l'insoddisfazione e la
mancanza di un sé, riconoscere il pericolo e applicare cautela e
moderazione in modo consapevole.

Ajahn Tongrat non insegnò molto. Ci diceva sempre: «~State davvero
attenti! State davvero attenti!~» È così che insegnava. «~State davvero
attenti! Se non state davvero attenti, vi arriverà sul mento!~» Così
stanno in realtà le cose. Anche se non lo avesse detto, le cose
starebbero ancora così. State davvero attenti! Se non state davvero
attenti, vi arriverà sul mento! Per favore, capitelo. Non si tratta del
problema di qualcun altro. Il problema non sta nel fatto che le altre
persone ci amano o ci odiano. Gli altri, quelli che stanno lontani, da
qualche altra parte, non ci fanno creare kamma e non ci arrecano
sofferenza. Sono i nostri possedimenti, le nostre case, le nostre
famiglie, è lì che dobbiamo fare attenzione.

Che ne pensate? Questi giorni, dove avete provato sofferenza? Dove siete
stati coinvolti dall'amore, dall'odio e dalla paura? Controllate voi
stessi, prendetevi cura di voi stessi. Attenzione a non essere
morsicati. Se non mordono, potrebbero scalciare. Non pensiate che queste
cose non mordano o non scalcino. Se venite morsicati, fate in modo che
il morso sia piccolo. Non fatevi scalciare e mordere fino a farvi
ridurre in pezzi. Non cercate di dire a voi stessi che non c'è pericolo.
Possessi, salute, fama, le persone che amiamo, tutte queste cose
scalciano e mordono, se non siete consapevoli. Se siete consapevoli,
sarete a vostro agio. Siate cauti e moderati. Quando la mente comincia
ad aggrapparsi alle cose e a ingigantirle, dovete fermarla. Essa
discuterà con voi, ma dovete metterci un piede sopra. Restate nel mezzo,
quando la mente va e viene. Da un lato mettete da parte l'indulgenza ai
piaceri dei sensi, dall'altro mettete da parte il tormentarsi da soli.
Mettete da una parte l'amore, dall'altra l'odio. Mettete da una parte la
felicità, dall'altra la sofferenza. Restate nel mezzo, senza consentire
alla mente di andare in nessuna delle due direzioni.

Come questi nostri corpi: terra, acqua, fuoco e vento. Dov'è la persona?
Non c'è nessuna persona. Queste poche e diverse cose sono messe insieme
e vengono chiamate ``persona''. È falso. Non è reale. Lo è solo in senso
convenzionale. Quando è giunto il tempo, gli elementi tornano al loro
stato precedente. Noi siamo venuti a stare con loro solo per un po', e
perciò dobbiamo lasciarli tornare a essere quel che erano. La parte che
è terra, fatela tornare terra. La parte che è acqua, fatela tornare
acqua. La parte che è fuoco, fatela tornare fuoco. La parte che è vento,
fatela tornare vento. Oppure cercherete di andare con loro, e di
trattenere qualcosa? Possiamo fare affidamento su di essi per un po', ma
quando è giunto il tempo che vadano, lasciateli andare. Quando arrivano,
lasciateli arrivare. Tutti questi fenomeni, \emph{sabhāva},\footnote{\emph{sabhāva.}
  Letteralmente, ``natura propria''. Principio o condizione della
  natura, qualcosa che è come veramente è.} appaiono e, poi, scompaiono.
Questo è tutto. Comprendiamo che tutte queste cose sono fluttuanti,
appaiono e scompaiono costantemente.

Fare offerte, ascoltare gli insegnamenti, praticare meditazione, tutto
quel che facciamo dovrebbe essere fatto per sviluppare la saggezza.
Sviluppare la saggezza serve alla Liberazione, alla libertà da queste
condizioni e fenomeni. Quando siamo liberi, non importa quale sia la
nostra situazione, non siamo costretti a soffrire. Se abbiamo figli, non
dobbiamo soffrire. Se lavoriamo, non dobbiamo soffrire. È come un loto
nell'acqua. «~Io cresco nell'acqua, ma non soffro a causa dell'acqua.
Non posso annegare né bruciare, perché nell'acqua ci vivo.~» Quando
l'acqua sale o scende, non influisce sul loto. L'acqua e il loto possono
vivere insieme, senza conflitti. Stanno insieme, ma sono separati.
Qualsiasi cosa si trovi nell'acqua nutre il loto e lo aiuta a diventare
bello.

Per noi è la stesso. Salute, casa, famiglia e tutte le contaminazioni
della mente non ci inquinano più, ci aiutano invece a sviluppare le
\emph{pāramī}, le perfezioni spirituali. In un boschetto di bambù le
foglie morte si ammucchiano attorno agli alberi e, quando la pioggia
cade, si decompongono e diventano concime. I germogli crescono e gli
alberi si sviluppano grazie al concime, e noi abbiamo una fonte di cibo
e di reddito. Però, non è affatto bello a vedersi. Fate perciò
attenzione: nella stagione secca, se accendete dei fuochi nella foresta,
essi bruceranno tutto il futuro concime, il quale si trasformerà in
fuoco e brucerà il bambù. Allora non avrete più germogli di bambù da
mangiare. Se bruciate la foresta, bruciate il concime del bambù. Se
bruciate il concime, bruciate gli alberi e il boschetto muore.

Lo capite? Voi e le vostre famiglie potete vivere in felicità e armonia
con le vostre case e i vostri possessi, liberi dal pericolo di alluvioni
e incendi. Se una famiglia subisce un'alluvione o un incendio, è solo a
causa delle persone di quella famiglia. È proprio come il concime del
bambù. Il boschetto può bruciare, oppure crescere bellissimo.

Le cose crescono e diventano bellissime e dopo non lo sono più, per poi
diventare belle di nuovo. Crescita e degenerazione, poi ancora crescita
e di nuovo degenerazione, così sono i fenomeni del mondo. Se conosciamo
crescita e degenerazione per quello che sono, possiamo porvi fine. Le
cose crescono e raggiungono i loro limiti. Le cose degenerano e
raggiungono i loro limiti. Ma noi rimaniamo stabili. È come quando a
Ubon ci fu un incendio. La gente lamentò le distruzioni e versò fiumi di
lacrime. Dopo l'incendio, però, tutto fu ricostruito e i nuovi edifici
sono certamente più grandi e migliori di com'erano prima, e ora la gente
si gode di più la città.

Così sono i cicli di perdita e guadagno. Tutto ha dei limiti. Perciò il
Buddha voleva che noi contemplassimo sempre. Mentre siamo ancora in vita
dovremmo pensare alla morte. Non consideratela come una cosa lontana. Se
siete poveri, non cercate di nuocere agli altri o di sfruttarli.
Affrontate la situazione e lavorate sodo per esser d'aiuto a voi stessi.
Se siete benestanti, non fate distrarvi dall'agio e dal benessere. Non è
poi così difficile che tutto vada perduto. Un ricco può diventare povero
in un paio di giorni. Un povero può diventare ricco. È tutto dovuto al
fatto che si tratta di condizioni impermanenti e instabili. Per questa
ragione il Buddha disse \emph{pamādo maccuno padaṃ}, la distrazione è la
via per la morte. I distratti sono come morti. Non siate distratti!
Tutti gli esseri e tutti i \emph{saṅkhāra} sono instabili e
impermanenti. Non attaccatevi in alcun modo a essi! Felice o triste che
sia, sulla via del progresso o del disfacimento alla fine tutto va a
finire nello stesso posto. Comprendetelo, per favore.

Vivendo nel mondo con questa prospettiva, possiamo essere liberi dai
pericoli. Tutto quello che otteniamo o che realizziamo in questo mondo
grazie al nostro buon kamma appartiene pur sempre al mondo ed è
soggetto a decadimento e perdita: perciò non fatevi trasportare troppo
lontano da tutto questo. È come un coleottero che gratta la terra. Può
grattarne un mucchio che è molto più grande di lui, ma è pur sempre un
mucchio di sporcizia. Se lavora sodo, può fare un buco profondo nel
terreno, ma è pur sempre un buco nella sporcizia. Se un bufalo produce
lì il suo letame, questo sarà più grande del mucchio di terra del
coleottero, ma non si tratterà di nulla che possa raggiungere il cielo.
È tutta sporcizia. Le realizzazioni mondane sono così. Non conta quanto
sodo il coleottero possa lavorare, sta solo a fare buchi nella sporcizia
e ad ammucchiarla.

Le persone con un buon kamma mondano hanno un'intelligenza che
consente loro di riuscire bene nel mondo. Però, non importa quanto bene
riescano, stanno ancora vivendo nel mondo. Tutte le cose che fanno sono
mondane e hanno dei limiti, come il coleottero che gratta la terra. Il
buco può andare in profondità, ma è nella terra. Il mucchio può
diventare grande, ma è solo un mucchio di sporcizia. Riuscire bene,
ottenere molto: stiamo solo riuscendo bene e ottenendo molto nel mondo.

Per favore comprendetelo, e cercate di sviluppare il distacco. Se non
ottenete molto, siate contenti e capite che sono solo cose del mondo. Se
ottenete molto, capite che sono solo cose del mondo. Contemplate queste
verità e non siate distratti. Guardate entrambi i lati delle cose, non
restate bloccati su un lato solo. Quando qualcosa vi procura gioia,
trattenetevi, perché quella gioia non durerà. Quando siete felici, non
andate completamente da quella parte, perché dopo non molto vi troverete
dall'altra parte, l'infelicità.

