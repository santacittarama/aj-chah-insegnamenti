\chapter{Comprendere la sofferenza}

Punge la pelle ed entra nella carne, e dalla carne passa nelle ossa. È
come un insetto che mangia e attraversa la corteccia di un albero, poi
il legno e il midollo, finché l'albero muore. Siamo cresciuti in questo
modo. Ha messo radici nel profondo. I nostri genitori ci hanno insegnato
ad aggrapparci e ad attaccarci, a dare importanza alle cose, credendo
fermamente che esistiamo in quanto entità dotate di un sé, e che le cose
ci appartengano. Ecco cosa ci è stato insegnato fin dalla nascita. Lo
abbiamo sentito in continuazione. Ci è penetrato nel cuore ed è rimasto
lì come sensazione abituale. Ci è stato insegnato a ottenere cose, ad
accumularle e ad aggrapparci a esse, a considerarle importanti e nostre.
Questo è quanto i nostri genitori sanno, e ce lo hanno insegnato. Tutto
ciò è penetrato nella nostra mente e nelle nostre ossa.

Quando ci interessiamo alla meditazione, gli insegnamenti di una guida
spirituale li ascoltiamo, ma non sono facili da capire. Non ci
coinvolgono. Ci viene detto di non considerare le cose nello stesso modo
di prima, di non comportarci come prima, ma quando lo sentiamo, gli
insegnamenti non entrano nella mente, li ascoltiamo solo con gli
orecchi. È che la gente non conosce se stessa. Così ci sediamo e
ascoltiamo gli insegnamenti, ma è solo del suono che entra negli
orecchi. Non arrivano dentro, non ci toccano. È come se fossimo
protagonisti di un incontro di pugilato e continuassimo a colpire
l'altro senza che vada giù. Restiamo bloccati nella nostra presunzione.
I saggi hanno detto che è più facile eliminare una montagna che
rimuovere la presunzione della gente.

Per livellare una montagna possiamo usare degli esplosivi e poi spostare
la terra. Ma per il serrato aggrapparsi della nostra presunzione, cari
miei, non è così! I saggi possono insegnarcelo fino alla fine dei nostri
giorni, ma riusciranno a farcela. La presunzione resta immobile. Così,
le nostre idee sbagliate e le cattive tendenze rimangono salde e
immutate senza che nemmeno ce ne rendiamo conto. Per questo i saggi
hanno detto che rimuovere la presunzione e trasformare l'errata
comprensione in retta comprensione è la cosa più difficile da farsi. È
così difficile per noi \emph{puthujjana}\footnote{\emph{Puthujjana}: Una
  persona comune, ordinaria, non illuminata; un essere ``mondano''.}
progredire e divenire \emph{kalyāṇajana}!\footnote{\emph{Kalyāṇajana}:
  Una persona buona, un essere virtuoso.} \emph{Puthujjana} significa
persone oscurate, profondamente bloccate nel buio e nelle tenebre. Il
\emph{kalyāṇajana} ha reso le cose più leggere. Noi insegniamo ad
alleggerirsi, ma la gente non vuole farlo perché non comprende la
situazione, la condizione di oscuramento. Così, le persone continuano a
vagare nel loro stato confusionale.

Se ci imbattiamo in un cumulo di sterco di bufalo non pensiamo che ci
appartenga, non vogliamo prenderlo. Lo lasciamo semplicemente lì dov'è,
perché sappiamo di cosa si tratta. Quel cumulo di sterco è, per chi è
impuro, il bene. Il cibo dei malvagi è il male. Se insegnate loro a fare
il bene, non sono interessati e preferiscono restare come sono, perché
non pensano sia dannoso. Se non si vede il pericolo, non v'è modo che le
cose possano essere corrette. Se capite, allora pensate: «~Oh! Tutto il
mio cumulo di sterco non vale quanto un piccolo pezzetto d'oro!~» Così,
al posto dello sterco vorrete l'oro, lo sterco non lo vorrete più. Se
non lo capite, restate proprietari di un cumulo di sterco. Non sarete
interessati neanche se vi offrono un diamante o un rubino.

Questo è il ``bene'' di chi è impuro. Oro, gioielli e diamanti sono
considerati qualcosa di buono nel reame degli esseri umani. Sporcizia e
marciume vanno bene per le mosche e per altri insetti. Se sopra ci
mettete del profumo, scapperanno via. Questo considera ``bene'' chi ha
errata visione. Questo è il ``bene'' per chi ha un'errata visione, per i
contaminati. Non ha un buon odore, ma se diciamo loro che puzza,
risponderanno che è fragrante. Non possono capovolgere con facilità
questo modo di vedere. È per questo che non è facile insegnare a loro.

Se raccogliete fiori freschi, le mosche non se ne interesseranno. Anche
se cercate di pagarle, non verranno. Però, ovunque vi sia un animale
morto, ovunque vi sia qualcosa di marcio, è lì che andranno. Non c'è
bisogno di chiamarle, arrivano. Così è l'errata visione. Si delizia in
questo genere di cose. Ciò che puzza e ciò che è guasto ha un buon
profumo. È impantanata e immersa in tutto questo. Quello che ha un dolce
profumo per un'ape, non lo ha per una mosca.

Nella pratica incontriamo difficoltà, ma in ogni cosa che intraprendiamo
dobbiamo attraversare delle difficoltà per raggiungere uno stato di
benessere. Nella pratica del Dhamma iniziamo con la verità di
\emph{dukkha}, il pervasivo carattere insoddisfacente dell'esistenza.
Appena lo sperimentiamo, però, ci perdiamo d'animo. Non vogliamo
guardarlo. \emph{Dukkha} è davvero la Verità, ma vogliamo girarci
attorno in un qualche modo. Così, non ci piace neanche guardare gli
anziani, preferiamo guardare i giovani.

Se non vogliamo guardare \emph{dukkha} non importa quante volte
rinasceremo, \emph{dukkha} non lo comprenderemo mai. \emph{Dukkha} è una
nobile verità. Se permettiamo a noi stessi di affrontarla, cominceremo a
cercare una via d'uscita. Se stiamo cercando di andare da qualche parte
e la strada è bloccata, penseremo a come realizzare un viottolo.
Lavorandoci giorno dopo giorno, potremo superare l'ostacolo. La saggezza
si sviluppa quando incontriamo dei problemi. Senza vedere \emph{dukkha}
non guardiamo nell'interiorità e non risolviamo i nostri problemi, ci
passiamo solo accanto con indifferenza.

Il mio modo di addestrare la gente comporta qualche sofferenza, perché
la sofferenza è il Sentiero del Buddha verso l'Illuminazione. Egli
volle che vedessimo la sofferenza, che ne vedessimo l'origine, la
cessazione e il Sentiero. Questa è la via d'uscita per tutti gli
\emph{ariya},\footnote{\emph{Ariya}: Nobile; chi ha ottenuto la visione
  trascendente in uno dei quattro stadi dell'Illuminazione.} i
Risvegliati. Se non percorrete questo Sentiero, non c'è via d'uscita.
L'unica via è conoscere la sofferenza, conoscere la causa della
sofferenza, conoscere la cessazione e conoscere il Sentiero della
pratica che conduce alla cessazione della sofferenza. Questo è il modo
in cui gli \emph{ariya}, cominciando con l'``Entrata nella
Corrente'',\footnote{Entrata nella Corrente (\emph{sotāpatti}): Evento
  tramite il quale si diviene \emph{sotāpanna}, il primo livello
  dell'Illuminazione.} sono riusciti a fuggire. È necessario conoscerla
la sofferenza.

Se conosciamo la sofferenza, la vedremo in tutto quello che
sperimentiamo. Alcuni pensano di non soffrire molto. Nel buddhismo, la
pratica serve a liberarci dalla sofferenza. Cosa dovremmo fare per non
soffrire più? Quando \emph{dukkha} sorge dovremmo investigare per vedere
le cause del suo sorgere. Una volta che le conosciamo, possiamo
praticare per rimuoverle. Sofferenza, origine, cessazione: per condurla
alla cessazione, dobbiamo comprendere il Sentiero della pratica. Poi,
quando percorriamo il Sentiero fino il compimento, \emph{dukkha} non
sorgerà più. Nel buddhismo questa è la via d'uscita.

Contrastando le nostre abitudini creiamo un po' di sofferenza. Di solito
abbiamo paura di soffrire. Se una cosa ci fa soffrire, non la vogliamo.
Siamo interessati a quello che ci pare essere buono e bello, ma pensiamo
che sia male tutto ciò che ha a che fare con la sofferenza. Non è così.
La sofferenza è \emph{saccadhamma}, verità. Se c'è sofferenza nel cuore,
ciò diviene la causa che vi fa pensare a una via di fuga. Vi induce a
contemplare. Non dormirete tranquilli perché sarete intenti a
investigare per scoprire cosa stia in realtà succedendo, per cercare di
vedere le cause e i loro effetti.

Chi è felice non sviluppa saggezza. È addormentato. È come un cane che
mangia a sazietà. Dopo non ha voglia di fare nulla. Può dormire tutto il
giorno. Se arriva un ladro, non abbaia; così sazio è troppo stanco. Se
però gli date meno cibo, starà all'erta e sveglio. Se qualcuno prova ad
aggirarsi furtivamente, salterà su e inizierà ad abbaiare. Lo capite?

Noi esseri umani siamo intrappolati e imprigionati in questo mondo,
abbiamo problemi in abbondanza, siamo sempre pieni di dubbi, di
confusione e di preoccupazioni. Non è un gioco. È una situazione davvero
difficile e problematica. C'è qualcosa di cui dobbiamo sbarazzarci.
Secondo la via dell'addestramento spirituale, dovremmo rinunciare al
nostro corpo, rinunciare a noi stessi. Dobbiamo decidere di dare la
nostra vita. Possiamo vedere gli esempi dei grandi rinuncianti, come il
Buddha. Egli era un nobile di casta guerriera, ma fu in grado di
lasciarsi tutto alle spalle, senza voltarsi indietro. Era erede di
ricchezze e potere, ma riuscì a rinunciarvi.

Se esponiamo il Dhamma sottile la maggior parte della gente si spaventa.
Non osa accedervi. Anche solo dicendo ``non fare il male'', non riesce a
praticarlo. Così stanno le cose. Ho cercato con ogni mezzo di superare
questa situazione. Spesso dico che indipendentemente dal fatto che siamo
lieti o agitati, felici o sofferenti, che versiamo lacrime o che
cantiamo, in questo mondo viviamo in una gabbia. Non andiamo al di là di
questa condizione, quella di essere in gabbia. Anche se siete ricchi non
importa, state vivendo in una gabbia. Se siete poveri, state vivendo in
una gabbia. Se cantate e ballate, state cantando e ballando in una
gabbia. Se guardate un film, lo state guardando in una gabbia.

Che cos'è questa gabbia? È la gabbia della nascita, la gabbia della
vecchiaia, la gabbia della malattia, la gabbia della morte. In questo
modo, siamo imprigionati nel mondo. «~Questo è mio.~» «~Questo mi
appartiene.~» Non sappiamo cosa siamo in realtà o cosa stiamo facendo.
Tutto quello che stiamo davvero facendo è accumulare sofferenza. Non si
tratta di una cosa lontana da noi a causare la nostra sofferenza, è che
non guardiamo noi stessi. Ovviamente, per quanta felicità e benessere
possiamo avere, essendo nati non possiamo evitare la vecchiaia, dovremo
ammalarci e morire. Questo è \emph{dukkha} stesso, qui e ora.

Possiamo essere sempre colpiti dal dolore e dalla malattia. Può
succedere in qualsiasi momento. È come se avessimo rubato qualcosa.
Potrebbero sempre venire ad arrestarci, perché quell'azione l'abbiamo
compiuta. Questa è la nostra situazione. Pericolo e tormento. Esistiamo
tra cose nocive. Nascita, invecchiamento e malattia governano le nostre
vite. Per sfuggire a esse non possiamo scappare da nessuna parte.
Possono venire a prenderci in ogni momento, ci sono sempre ottime
opportunità. Dobbiamo concederglielo e accettare la situazione. Dobbiamo
dichiararci colpevoli. Se lo facciamo, la sentenza non sarà così
gravosa. Se non lo facciamo, soffriremo enormemente. Se ci dichiariamo
colpevoli, ci andranno piano con noi. Non staremo in carcere troppo a
lungo.

Quando il corpo nasce, non appartiene a nessuno. È come la nostra sala
di meditazione. Dopo che è stata costruita sono arrivati i ragni, vi
alloggiano. Le lucertole vengono a viverci. Ogni tipo d'insetti e cose
striscianti viene ad alloggiarci. Possono venire a viverci pure i
serpenti. Tutto può venire a viverci. Non è solo la nostra sala. È la
sala di tutti. Per questo corpo è lo stesso. Non è nostro. Le persone
vengono a stare nel corpo e dipendono da esso. Anche malattia, dolore e
vecchiaia vengono a risiedere nel corpo e noi ci limitiamo ad abitare in
esso insieme a loro. Quando il dolore e la malattia giungono a
compimento in questo corpo, esso infine si disgrega e muore. Non siamo
noi, però a morire. Perciò, non aggrappatevi a nulla di tutto questo.
Dovete invece contemplare la questione e il vostro attaccamento si
esaurirà gradualmente. Se vedete correttamente, l'errata comprensione
cesserà.

È stata la nascita a darci questo fardello. In genere, però, non
l'accettiamo e pensiamo che non essere nati sarebbe stato il male
maggiore, e che morire e non rinascere sarebbe la cosa peggiore in
assoluto. È così che vediamo le cose. Di solito pensiamo solo a tutto
quello che vogliamo in futuro. Poi andiamo ancora più in là con i
desideri: «~Nella prossima vita, che io possa nascere tra gli dèi, o che
io possa nascere benestante.~» Stiamo chiedendo un fardello ancor più
pesante! Però, pensiamo che ci porterà felicità. Questo modo di pensare
è su una strada del tutto differente da ciò che insegna il Buddha. È una
strada pesante. Il Buddha disse di lasciarla andare e di rifiutarla. Noi
invece pensiamo: «~Non posso lasciar andare!~» Così, continuiamo a
trasportare un fardello che continua ad appesantirsi. Siccome siamo
nati, abbiamo questa pesantezza. È per questo che comprendere il Dhamma
in modo puro è davvero difficile. Dobbiamo far affidamento su una seria
investigazione.

Facciamo un altro passo. Che cosa pensate? Pensate che la brama abbia
dei limiti? Quand'è che sarà soddisfatta? Esiste qualcosa in grado di
soddisfarla? Se prendete in considerazione \emph{taṇhā}, la brama cieca,
capirete che non può essere soddisfatta. Continua a desiderare sempre di
più. Anche se conduce a una sofferenza tale da farci quasi morire,
continuerà a volere, perché è impossibile soddisfare \emph{taṇhā}.

È una cosa importante. Se solo riuscissimo a pensare in modo equilibrato
e moderato! Bene, parliamo di indumenti. Di quanti vestiti abbiamo
bisogno? A proposito di cibo. Quanto mangiamo? Al massimo, durante un
pasto potremmo mangiarne due piatti. Dovrebbe essere abbastanza. Se
conosciamo la moderazione, saremo felici e a nostro agio, ma non è una
cosa molto comune. Il Buddha impartì degli ``insegnamenti ai ricchi''.
Dicono di accontentarsi di quello che si ha. Chi si accontenta è ricco.
Penso che questo tipo di conoscenza sia davvero degna di essere
studiata. La conoscenza insegnata sulla via del Buddha è una cosa degna
d'essere imparata e sulla quale vale la pena riflettere.

Il puro Dhamma della pratica va oltre. È molto più profondo. Alcuni di
voi potrebbero non essere in grado di capire. Basta solo prendere in
considerazione le parole del Buddha che dicono che qui non c'è più
nascita per Lui, che nascita e divenire sono finiti. Ascoltare questo vi
fa sentire a disagio. Una formulazione più diretta: il Buddha disse che
non saremmo dovuti nascere, perché la nascita è sofferenza. Anche solo
questo, la nascita, il Buddha la mise a fuoco, la contemplò e ne
comprese la gravità. \emph{Dukkha} arriva tutto con l'essere nati.
Succede simultaneamente alla nascita. Quando arriviamo in questo mondo,
riceviamo gli occhi, una bocca, un naso. Tutto arriva solo a causa della
nascita. Se invece sentiamo parlare di morire e di non nascere di nuovo,
questo ci sembra un disastro totale. Là dove la nascita non avviene, non
ci vogliamo andare. Questo è però l'insegnamento più profondo del
Buddha.

Perché ora stiamo soffrendo? Perché siamo nati. Così, ci viene insegnato
di porre fine alla nascita. Non si tratta di parlare solo del corpo che
nasce e del corpo che muore. È troppo semplice. Lo può capire anche un
bambino. Il respiro termina, il corpo muore e poi giace qui. Questo è
ciò che normalmente intendiamo quando parliamo della morte. Ma un morto
che respira? È una cosa che non comprendiamo. Un morto che può
camminare, parlare e sorridere è una cosa alla quale non abbiamo
pensato. Conosciamo solo il cadavere che non respira più. Questo è quel
che chiamiamo morte.

Lo stesso vale per la nascita. Quando diciamo che è nato qualcuno,
intendiamo che una donna è andata in ospedale e ha partorito. Però, il
momento in cui nella mente nasce qualcosa, ad esempio quando a casa vi
arrabbiate, l'avete notato? A volte nasce l'amore. Altre volte
l'avversione. Essere contenti, essere scontenti: ogni genere di stato
mentale. Non è nient'altro che nascita. Soffriamo solo per questa
ragione. Quando gli occhi vedono una cosa spiacevole, nasce
\emph{dukkha}. Quando gli orecchi sentono qualcosa che vi piace davvero,
pure in quel caso nasce \emph{dukkha}. C'è solo sofferenza. Il Buddha
riassunse il tutto dicendo che c'è solo una massa di sofferenza. La
sofferenza nasce e la sofferenza cessa. Questo è tutto quello che c'è.
Noi ci balziamo sopra e l'afferriamo in continuazione, balziamo sul
sorgere, balziamo sul cessare, senza mai comprenderli davvero.

Quando \emph{dukkha} sorge, parliamo di sofferenza. Quando cessa,
parliamo di felicità. È tutta roba vecchia: sorgere e cessare. Ci viene
insegnato a osservare il corpo e la mente che sorgono e cessano. Oltre a
questo non c'è niente. Per riassumere, non c'è felicità, c'è solo
\emph{dukkha}. La vediamo e la definiamo in questo modo, ma non c'è. È
solo \emph{dukkha} che cessa. \emph{Dukkha} sorge e cessa, sorge e
cessa, e noi ci balziamo sopra e l'afferriamo. Appare la felicità e
siamo contenti. Appare l'infelicità e siamo sconvolti. In realtà è la
stessa cosa, mero sorgere e cessare. Quando c'è il sorgere, c'è
qualcosa, quando c'è il cessare, quel qualcosa se n'è andato. È qui che
dubitiamo. Per questo ci viene insegnato che \emph{dukkha} sorge e
cessa, e che oltre a questo non c'è nulla. Quando si arriva al dunque,
c'è solo sofferenza. Ma non lo vediamo con chiarezza.

La verità è che in questo nostro mondo non c'è niente che faccia
qualcosa a qualcuno. Non c'è niente per cui essere ansiosi. Non c'è
nulla per cui valga la pena di piangere, nulla per cui ridere. Niente è
di per sé tragico o piacevole. Questo è però il modo consueto della
gente di sperimentare le cose. Il nostro linguaggio può essere
ordinario, possiamo relazionarci agli altri secondo la maniera consueta
di vedere le cose. Tutto questo va bene. Però, se pensiamo in modo
ordinario, ciò conduce alle lacrime. In verità, se davvero conosciamo il
Dhamma e lo vediamo continuamente, tutto è assolutamente nulla. C'è solo
sorgere e svanire. Non c'è reale felicità o sofferenza. È allora che il
cuore è in pace, quando non c'è felicità o sofferenza. Quando c'è
felicità e sofferenza, c'è divenire e nascita.

Normalmente creiamo un tipo di \emph{kamma} che consiste nel tentare di
fermare la sofferenza e produrre la felicità. È quello che vogliamo.
Quello che vogliamo, però, non è vera pace: è felicità e sofferenza. Lo
scopo dell'insegnamento del Buddha è praticare per creare un
\emph{kamma} che conduce oltre la felicità e la sofferenza e che ci darà
la pace. Non siamo capaci di pensare in questo modo. Riusciamo solo a
pensare che se siamo felici avremo la pace. Pensiamo che sia sufficiente
avere la felicità. Per questo noi esseri umani desideriamo cose in
abbondanza. Se otteniamo molto, è bene. È così che pensiamo, in genere.
Si suppone che fare del bene porti buoni risultati e che, se li
otteniamo, saremo felici. Riteniamo che sia necessario fare solo questo,
e ci fermiamo lì. Ma dov'è che il bene giunge a una conclusione? Il bene
non si conserva. Continuiamo ad andare avanti e indietro, a sperimentare
bene e male, cercando giorno e notte di cogliere ciò che pensiamo sia
buono. L'insegnamento del Buddha è che, primo, dovremmo rinunciare al
male e poi praticare ciò che è bene; secondo, che dovremmo rinunciare
pure al bene, senza attaccarci a esso, perché pure il bene è un tipo di
combustibile. Alla fine il combustibile va in fiamme. Il bene è un
combustibile. Il male è un combustibile.

Parlare a questo livello uccide la gente. Non è in grado di seguire.
Così, dobbiamo tornare all'inizio e insegnare la moralità. Non
danneggiatevi l'un l'altro. Nel vostro lavoro siate responsabili e non
nuocete né sfruttate gli altri. Il Buddha insegnò anche questo, ma ciò
non è sufficiente per fermarsi. Perché ci troviamo qui, in questa
condizione? Perché siamo nati. Il Buddha nel suo primo insegnamento, il
Discorso della Messa in Moto della Ruota del Dhamma, disse: «~La nascita
è finita. Questa è la mia ultima esistenza. Non c'è più altra nascita
per il \emph{Tathāgata}.~»\footnote{\emph{Tathāgata}: Letteralmente,
  ``così andato'', ``così venuto''.} Non sono molte le persone che
tornano su questo punto per contemplarlo e comprendere in accordo con i
principi della via del Buddha. Però, se abbiamo fiducia nella via del
Buddha, essa ci ripagherà. Se le persone confidano sinceramente nei Tre
Gioielli, praticare è semplice.

