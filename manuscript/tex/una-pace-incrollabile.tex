\chapter{Una pace incrollabile}

L'unica ragione per studiare il Dhamma, gli insegnamenti del Buddha,
consiste nella ricerca di un modo per trascendere la sofferenza e per
raggiungere la pace e la felicità. Anche se studiamo i fenomeni fisici o
mentali, la mente (\emph{citta}) o i suoi fattori psicologici
(\emph{cetasikā}),\footnote{\emph{cetasikā}. Fattore mentale che
  accompagna il \emph{citta} o mente.} siamo sul giusto Sentiero solo
quando il nostro scopo principale è la Liberazione dalla sofferenza,
nient'altro. La sofferenza ha una causa e delle condizioni che le
consentono di esistere.

Per favore, comprendi con chiarezza che la mente è nel suo normale stato
naturale quando è immobile. Appena la mente si muove, diventa
condizionata (\emph{saṅkhāra}). Quando la mente è attratta da qualcosa,
diventa condizionata. Quando sorge l'avversione, diventa condizionata.
Il desiderio di muoversi qua e là sorge dal suo essere condizionata. Se
la nostra consapevolezza non sta al passo con queste proliferazioni
mentali, la mente le inseguirà e ne sarà condizionata. Tutte le volte
che la mente si muove, proprio in quel momento, diventa una realtà
convenzionale. Per questo il Buddha ci insegnò a contemplare queste
condizioni oscillanti della mente. Tutte le volte che la mente si muove,
diventa instabile e impermanente (\emph{aniccā}), preda
dell'insoddisfazione (\emph{dukkha}) e non può essere considerata come
un sé (\emph{anattā}). Queste sono le Tre Caratteristiche universali di
tutti i fenomeni condizionati. Il Buddha ci insegnò a osservare e
contemplare questi movimenti della mente. Altrettanto avviene con
l'insegnamento dell'originazione dipendente
(\emph{paṭiccasamuppāda}).\footnote{\emph{paṭiccasamuppāda}.
  Coproduzione condizionata, genesi interdipendente. Una tabella che
  descrive il modo in cui i cinque aggregati (\emph{khandhā}) e le sei
  basi sensoriali (\emph{āyatana}) interagiscono dopo il contatto
  (\emph{phassa}) con l'ignoranza (\emph{avijjā}) e con la brama
  (\emph{taṇhā}) per condurre alla tensione e alla sofferenza
  (\emph{dukkha}).} La comprensione illusoria (\emph{avijjā})\footnote{\emph{avijjā}.
  Non conoscenza, ignoranza.} è la causa e la condizione del sorgere
delle formazioni volizionali del \emph{kamma} (\emph{saṅkhāra}), ciò è
la causa e la condizione del sorgere della coscienza (\emph{viññāna}),
ciò è la causa e la condizione del sorgere degli oggetti mentali e della
materialità (\emph{nāma} e \emph{rūpa}), e così via. Proprio come
abbiamo studiato nelle Scritture. Il Buddha separò ogni anello della
catena per renderlo più semplice da studiare. Si tratta di un'accurata
descrizione della realtà, ma quando questo processo si verifica nella
vita reale, chi l'ha studiato non è in grado di tenere il passo con quel
che succede.

È come cadere dalla cima di un albero e finire al suolo. Non abbiamo
alcuna idea di quanti siano stati i rami che abbiamo spezzato mentre
cadevamo. Allo stesso modo, quando la mente è di colpo toccata da
un'impressione mentale, si crogiola in essa deliziandosene, e poi spicca
il volo verso uno stato mentale buono. Lo considera buono senza essere
consapevole della catena delle condizioni che l'hanno condotta fino a
quel punto. Il processo ha luogo in accordo con quanto è descritto nella
teoria, ma nello stesso tempo supera i limiti di quella teoria. Non vi è
nulla che proclami: «~Questa è un'illusione. Sono formazioni volizionali
del \emph{kamma}, e questa è la coscienza.~» Il processo non dà allo
studioso la possibilità di leggere l'elenco della catena mentre il tutto
sta avvenendo. Benché il Buddha analizzò e spiegò nei dettagli la
sequenza dei momenti mentali, per me è più come se si cadesse da un
albero. Quando cadiamo rovinosamente a terra, non vi è possibilità di
valutare per quanti metri e per quanti centimetri siamo caduti. Quel che
sappiamo è che siamo caduti al suolo con un tonfo e che fa male!

Con la mente è la stessa cosa. Quando si cade per qualcosa, ciò di cui
siamo consapevoli è il dolore. Da dove è venuta tutta questa sofferenza,
tutto questo dolore, tutta questa afflizione e disperazione? Non è
venuta dalla teoria di un libro. I dettagli della nostra sofferenza non
stanno scritti da nessuna parte. Il nostro dolore non corrisponde
esattamente alla teoria, ma entrambi percorrono lo stesso itinerario.
Perciò la dottrina non può da sola tenere il passo con la realtà. Ecco
perché il Buddha ci insegnò a coltivare la chiara conoscenza da noi
stessi. Qualsiasi cosa sorga, sorge in questa conoscenza. Quando ``Colui
che Conosce'' conosce in accordo con la Verità, allora vediamo che la
mente e i suoi fattori psicologici non ci appartengono. In ultima
analisi, tutti questi fenomeni devono essere scartati e gettati via come
se fossero spazzatura. Non dovremmo attaccarci o dare a essi alcun
significato.

\textbf{Teoria e realtà}

Il Buddha non ci insegnò il funzionamento della mente e dei fattori
psicologici affinché ci attaccassimo ai concetti. La sua unica
intenzione era di farceli riconoscere come impermanenti, insoddisfacenti
e non-sé. E poi lasciarli andare. Metterli da parte. Essere consapevoli
e conoscerli appena sorgono. La mente è già stata condizionata. È stata
addestrata e condizionata ad allontanarsi e a uscire da uno stato di
pura consapevolezza. Quando la mente turbina, crea fenomeni condizionati
che esercitano un ulteriore influsso su di essa, e la proliferazione
mentale continua. Questo processo fa nascere il bene, il male e tutto
quello che sta sotto il sole. Il Buddha ci ha insegnato ad abbandonare
tutto. All'inizio, ovviamente, si deve familiarizzare con la teoria per
essere in grado, nella fase successiva, di abbandonare tutto. Si tratta
di un processo naturale. La mente è proprio così. I fattori psicologici
sono proprio così.

Prendi ad esempio il Nobile Ottuplice Sentiero. Quando la saggezza
(\emph{paññā}) con la visione profonda vede le cose in modo corretto,
questa Retta Visione conduce alla Retta Intenzione, alla Retta Parola,
alla Retta Azione e così via. Tutto ciò coinvolge condizioni
psicologiche sorte da quella pura consapevolezza che conosce. Questa
conoscenza è come una lanterna che in una notte buia diffonde la luce
sul Sentiero davanti a te. Se la conoscenza è retta, se è in accordo con
la Verità, a sua volta pervaderà e illuminerà tutti gli altri passi sul
Sentiero. Qualsiasi cosa di cui si abbia esperienza, tutto sorge
dall'interno di questa conoscenza. Se questa mente non esistesse, non
esisterebbe neanche la conoscenza. Sono tutti fenomeni della mente. Come
dice il Buddha, la mente è solo la mente. Non è un essere, una persona,
un sé, o te stesso. Non è nemmeno noi o loro. Il Dhamma è semplicemente
il Dhamma. È un processo naturale, privo del sé. Non appartiene a noi o
a qualcun altro. Non è un qualcosa. Qualsiasi cosa un individuo
sperimenti, rientra tutto all'interno delle cinque fondamentali
categorie di aggregati (\emph{khandhā}): corpo, sensazione,
memoria/percezione, pensieri e coscienza. Il Buddha disse di lasciar
andare tutto.

La meditazione è come un bastone di legno. La visione profonda
(\emph{vipassanā}) è un'estremità e la tranquillità (\emph{samatha})
l'altra. Se lo prendiamo, solleviamo solo un'estremità o entrambe? Se
qualcuno prende un bastone, alza entrambe le estremità. Allora quale
delle due estremità è \emph{vipassanā}, e quale \emph{samatha}? Dove
finisce una e dove comincia l'altra? Entrambi sono la mente. Quando la
mente diviene serena, all'inizio la pace sorge dalla serenità di
\emph{samatha}. Focalizziamo e unifichiamo la mente in stati di
tranquillità meditativa (\emph{samādhi}). Ovviamente, se la pace e la
tranquillità del \emph{samādhi} svaniscono, al loro posto sorge la
sofferenza. Perché succede questo? Perché la pace procurata dalla sola
meditazione \emph{samatha} è ancora basata sull'attaccamento. Questo
attaccamento può essere causa di sofferenza. La serenità non è la fine
del Sentiero. Il Buddha comprese, grazie alla sua stessa esperienza, che
questo genere di pace non è la pace suprema. Le cause soggiacenti al
processo dell'esistenza (\emph{bhava})\footnote{\emph{bhava}. Esistenza;
  divenire; una ``vita''.} non erano ancora state condotte alla
cessazione (\emph{nirodha}). Le condizioni per la rinascita esistevano
ancora. Il suo lavoro spirituale non aveva ancora raggiunto la
perfezione. Perché? In quanto la sofferenza c'era ancora. Così, sulla
base di quella serenità \emph{samatha}, Egli continuò a contemplare, a
investigare e analizzare la natura condizionata della realtà fino a che
si liberò da ogni attaccamento, anche dall'attaccamento alla serenità.
La serenità fa ancora parte del mondo dell'esistenza condizionata e
della realtà convenzionale. Attaccarsi a questo genere di pace è
attaccarsi alla realtà convenzionale, e finché ci aggrappiamo resteremo
impantanati nell'esistenza e nella rinascita. Deliziarsi nella pace
\emph{samatha} conduce ancora a ulteriori esistenze e rinascite. Quando
l'irrequietezza e l'agitazione della mente si calmano, ci si attacca
alla pace che ne consegue.

Il Buddha esaminò le cause e le condizioni soggiacenti all'esistenza e
alla rinascita. Finché non riuscì a capire del tutto la questione e
comprese la verità, con la mente serena continuò a esplorare sempre più
in profondità, a riflettere su come tutte le cose -- serene o meno che
fossero -- pervengono a esistere. La sua investigazione proseguì finché
gli fu chiaro che tutto quel che giunge all'esistenza è come un blocco
di ferro incandescente. Le cinque categorie esperienziali di un essere
(\emph{khandhā}) sono tutte quante blocchi di ferro incandescente.
Quando un blocco di ferro è incandescente, dov'è che potete toccarlo
senza bruciarvi? Ha una parte fredda? Provate a toccarlo sulla sommità,
ai lati o di sotto. C'è un solo posto che sia freddo? Impossibile.
Questo blocco di ferro rovente è incandescente dappertutto. Non possiamo
attaccarci nemmeno alla serenità. Se ci identifichiamo con quella pace,
pensando che c'è qualcuno che è calmo e sereno, ciò rinforza la
sensazione che ci sia un io indipendente o un'anima. Questa sensazione
del sé fa parte della realtà convenzionale. «~Io sono sereno.~» «~Io
sono agitato.~» «~Io sono buono.~» «~Io sono cattivo.~» «~Io sono
felice.~» «~Io sono infelice.~» Così pensiamo, ed ecco che siamo
catturati da altre esistenze e rinascite. C'è più sofferenza. Se la
nostra felicità svanisce, allora siamo infelici. Quando il nostro
dispiacere svanisce, siamo di nuovo felici. Prigionieri di questo ciclo
infinito, turbiniamo ripetutamente attraverso paradiso e inferno.

Prima dell'Illuminazione, il Buddha comprese questo schema nel suo
cuore. Seppe che le condizioni per l'esistenza e per la rinascita non
erano ancora cessate. Il suo lavoro non era ancora finito. Focalizzando
la condizionalità della vita, contemplò in accordo con la natura: «~Per
questa causa c'è la nascita, a causa della nascita c'è la morte, e tutto
questo movimento di andare e venire.~» Questi sono i fattori che il
Buddha contemplò per comprendere la verità sui cinque aggregati
(\emph{khandhā}). Tutto quello che è mentale e fisico, tutto ciò che si
concepisce e che si può pensare è senza eccezioni condizionato. Quando
lo capì, ci insegnò a deporre tutto. Quando lo capì, ci insegnò ad
abbandonare tutto. Incoraggiò gli altri a comprendere in accordo con
questa Verità. Se non lo faremo, soffriremo. Non saremo in grado di
lasciar andare queste cose.

Ovviamente, quando vedremo la Verità, saremo in grado di vedere come
queste cose ci illudono. Come il Buddha insegnò: «~La mente non ha
sostanza, non è un qualcosa.~» La mente non nasce appartenendo a
qualcuno. Non muore come se fosse di qualcuno. La mente è libera,
brillante e radiosa, non è irretita da alcun problema o questione. La
ragione per cui i problemi sorgono è perché la mente è ingannata dalle
cose condizionate, è ingannata da questa errata percezione del sé.
Perciò il Buddha insegnò a osservare la mente. In principio cosa c'è
nella mente? In verità lì non c'è nulla. Non sorge con le cose
condizionate, e non muore con esse. Quando la mente incontra qualcosa di
buono, non cambia per diventare buona. Quando la mente incontra qualcosa
di cattivo, nemmeno diventa cattiva. Ecco com'è, quando c'è chiara
visione profonda nella propria natura. Si comprende che vi è uno stato
di fatto essenzialmente non sostanziale.

La visione profonda del Buddha vide tutto come impermanente,
insoddisfacente e privo di un sé. Volle che comprendessimo appieno le
cose in questo modo. È solo allora che la conoscenza conosce in accordo
con la Verità. Quando conosce la felicità o il dolore, resta immobile.
L'emozione della felicità è una forma di nascita. La tendenza a essere
tristi è una forma di morte. Quando c'è morte, c'è nascita, e ciò che è
nato deve morire. Quel che sorge e scompare viene catturato in questo
incessante ciclo del divenire. Quando la mente del meditante giunge a
questo stato di comprensione, non restano dubbi sul fatto che ci sia
ulteriore divenire e rinascita. Non c'è bisogno di chiedere a nessun
altro.

Il Buddha investigò in modo completo i fenomeni condizionati e fu così
in grado di lasciar andare tutto. I cinque \emph{khandhā} dovevano
essere lasciati andare, e la conoscenza doveva essere portata avanti
unicamente come un imparziale osservatore del processo. Se Egli
sperimentava una cosa positiva, non diventava positivo insieme a essa.
Osservava solo, e restava consapevole. Se Egli sperimentava una cosa
negativa, non diventava negativo. E perché? La sua mente era stata
tagliata fuori, s'era liberata da queste cause e da queste condizioni.
Aveva penetrato la Verità. Le condizioni che portavano alla rinascita
non esistevano più. Questa è la conoscenza certa e affidabile. Questa è
una mente davvero in pace. Questo è ciò che non è nato, non invecchia,
non s'ammala e non muore. Questo non è né causa né effetto, e nemmeno
dipende da causa ed effetto. È indipendente dal processo dei
condizionamenti causali. Le cause allora cessano senza che rimanga
alcunché di condizionante. La mente è al di sopra e al di là di nascita
e morte, al di sopra e al di là di felicità e tristezza, al di sopra e
al di là sia del bene sia del male. Che si può dire? Descriverlo va
oltre le possibilità del linguaggio. Tutte le condizioni di supporto
sono cessate e qualsiasi tentativo di descrizione condurrebbe solo
all'attaccamento. Le parole che si usano diventerebbero la teoria della
mente.

Le descrizioni teoriche della mente e delle sue creazioni sono accurate,
ma il Buddha comprese che questo genere di conoscenza è relativamente
inutile. Comprendiamo qualcosa da un punto di vista intellettuale e poi
ci crediamo, ma questo non reca alcun vero beneficio. Non conduce alla
pace della mente. La conoscenza del Buddha porta al lasciar andare. Ha
come risultato l'abbandono e la rinuncia, perché è proprio questa mente
che ci fa essere coinvolti sia con ciò che è giusto sia con ciò che è
sbagliato. Se siamo intelligenti, restiamo coinvolti con le cose giuste.
Se siamo stupidi, restiamo coinvolti con le cose sbagliate. Una mente
del genere è il mondo, e il Beato prese le cose del mondo ed esaminò
proprio questo mondo. Essendo pervenuto alla conoscenza del mondo quale
esso in realtà è, fu noto come ``Colui che comprende il mondo con
chiarezza''.

Per quanto concerne la questione di \emph{samatha} e \emph{vipassanā}, è
importante sviluppare questi stati nel nostro cuore. Solo quando li
coltiveremo sinceramente sapremo cosa sono davvero. Possiamo studiare
tutto quello che i libri dicono sui fattori psicologici della mente, ma
questo genere di comprensione intellettuale è inutile per eliminare del
tutto il desiderio egoistico, la collera e l'illusione. In relazione al
desiderio egoistico, alla collera e all'illusione studiamo solo la
teoria, che si limita a descrivere le varie caratteristiche di queste
contaminazioni mentali: «~Il desiderio egoistico ha questo significato;
la collera significa quello; l'illusione è definita in questo modo.~»
Conoscendo solo le loro caratteristiche teoriche, possiamo parlarne solo
a questo livello. Sappiamo, siamo intelligenti, ma quando queste
contaminazioni compaiono davvero nella nostra mente, corrispondono o no
alla teoria? Ad esempio, quando sperimentiamo qualcosa di
indesiderabile, reagiamo e ci ritroviamo di cattivo umore? Ci
attacchiamo? Riusciamo a lasciar andare? Se affiora l'avversione e la
riconosciamo, ci aggrappiamo ancora a essa? Oppure, quando l'abbiamo
vista la lasciamo andare? Se notiamo che stiamo vedendo una cosa che non
ci piace e tratteniamo l'avversione nel nostro cuore, sarebbe meglio
tornare indietro e cominciare a studiare di nuovo. Non va ancora bene.
La pratica non è ancora perfetta. Quando la pratica raggiunge la
perfezione, avviene il lasciar andare. Considerate le cose sotto questa
luce.

Dobbiamo veramente guardare in profondità nel nostro cuore se vogliamo
sperimentare i frutti della pratica. Tentare di descrivere la psicologia
della mente nei termini di numerosi momenti di coscienza separati e
delle loro varie caratteristiche significa, secondo me, non prendere sul
serio la pratica. C'è ben di più nella pratica. Se vogliamo studiare
queste cose, allora è necessario conoscerle assolutamente, con chiarezza
e penetrante comprensione. Senza la chiarezza della visione profonda,
quando termineremo mai di studiarle? Non c'è fine. Non completeremo mai
i nostri studi. Per questo motivo praticare il Dhamma è di estrema
importanza. Praticando, è così che studiavo. Non sapevo nulla di momenti
mentali e fattori psicologici. Osservavo solo la qualità del conoscere.
Se sorgeva un pensiero di odio, mi chiedevo perché. Se sorgeva un
pensiero d'amore, mi chiedevo perché. È così che si fa. Che sia definito
come pensiero oppure chiamato fattore psicologico, che cosa cambia?
Approfondisci solo questo punto, fino a quando sei in grado di risolvere
questi sentimenti di amore e di odio, finché scompaiono completamente
dal cuore. Quando fui in grado di smettere di amare e di odiare in
qualsiasi circostanza, riuscii a trascendere la sofferenza. Poi, quel
che succede non conta. Il cuore e la mente sono liberi e a proprio agio.
Non rimane nulla. Tutto si ferma.

Pratica in questo modo. Se la gente vuole parlare molto di teoria, è
affar suo. Non conta quanto si discuta, la pratica giunge al dunque
proprio qui. Quando qualcosa sorge, sorge proprio qui. Che sia tanto o
solo un po', ha origine proprio qui. Quando cessa, la cessazione è
proprio qui. E dove, altrimenti? Il Buddha la chiamò ``Conoscenza''.
Allorché essa conoscerà le cose con accuratezza, in linea con la Verità,
comprenderemo il significato di ``mente''. Le cose ci ingannano in
continuazione. Mentre le studi, nello stesso tempo ti stanno ingannando.
In quale altro modo potrei esprimermi? Anche se le conosci, le cose ti
ingannano esattamente proprio là, dove le conosci. Questa è la
situazione. Questo è il problema. Secondo me, il Buddha non voleva che
sapessimo come le cose sono chiamate. Lo scopo degli insegnamenti del
Buddha è comprendere il modo di liberarci da queste cose per mezzo della
ricerca delle cause a esse soggiacenti.

\textbf{\emph{Sīla}, \emph{samādhi} e \emph{paññā}}

Ho praticato il Dhamma senza sapere granché. Sapevo solo che il Sentiero
per la Liberazione inizia con la virtù (\emph{sīla}). La virtù è il
bell'inizio del Sentiero. La pace profonda del \emph{samādhi} è la bella
metà. La saggezza (\emph{paññā}) è la bella fine. Sebbene possano essere
separate come tre singoli aspetti dell'addestramento, quando le
osserviamo più in profondità, queste tre qualità convergono fino a
diventare una sola. Per sostenere la virtù, si deve essere saggi. Di
solito consigliamo alla gente di iniziare sviluppando i fondamenti
etici, osservando i Cinque Precetti, in modo tale che la loro virtù
diventi solida. Ovviamente, la perfezione della virtù richiede molta
saggezza. Dobbiamo prendere in considerazione le nostre parole e le
nostre azioni, e analizzarne le conseguenze. È un lavoro che solo la
saggezza può svolgere. Dobbiamo far affidamento sulla nostra saggezza
per coltivare la virtù.

Secondo la teoria, viene prima la virtù, poi il \emph{samādhi} e poi la
saggezza, ma quando ho esaminato tale questione ho capito che la
saggezza è il fondamento di qualsiasi altro aspetto della pratica. Per
comprendere pienamente le conseguenze di quel che diciamo e facciamo --
soprattutto le conseguenze nocive -- è necessario utilizzare la guida e
la supervisione della saggezza, sottoporre a esame le relazioni tra
causa ed effetto. Questo purificherà le nostre azioni e le nostre
parole. Allorché avremo capito quali sono i comportamenti etici e quali
i non etici, vedremo il luogo della pratica. Poi abbandoneremo ciò che è
male e coltiveremo quel che è bene. Abbandonare quello che è sbagliato e
coltivare quello che è giusto. Questa è virtù. Quando facciamo così, il
cuore diventa sempre più stabile e risoluto. Un cuore risoluto e
incrollabile è libero dall'apprensione, dal rimorso e dalla confusione a
proposito delle proprie azioni e delle proprie parole. È il
\emph{samādhi}.

Questa stabile unificazione della mente genera nella nostra pratica del
Dhamma una seconda e ancor più potente fonte di energia, che consente
una più profonda contemplazione di ciò che vediamo, sentiamo e così via,
di ciò che sperimentiamo. Quando la mente dimora in una consapevolezza e
in una pace stabili e incrollabili, possiamo impegnarci in una costante
indagine della realtà del corpo, della sensazione, della percezione, del
pensiero, della coscienza, degli oggetti visivi, dei suoni, degli odori,
dei sapori, delle sensazioni tattili e degli oggetti mentali. Man mano
che sorgono in continuazione, in continuazione li investighiamo con la
sincera determinazione di non perdere la nostra consapevolezza. Allora
conosceremo come sono veramente queste cose. Pervengono a esistere
seguendo la loro propria, naturale verità. Allorché la nostra
comprensione cresce costantemente, ecco che nasce la saggezza. Appena vi
è chiara comprensione del modo in cui sono le cose, le nostre vecchie
percezioni vengono sradicate e la nostra conoscenza concettuale si
trasforma in saggezza. Ecco come la virtù, il \emph{samādhi} e la
saggezza si fondono e operano come una sola cosa.

Quando la saggezza cresce in forza e in audacia, il \emph{samādhi} si
evolve e diventa sempre più stabile. Più irremovibile è il
\emph{samādhi}, più irremovibile e onnicomprensiva diventa la virtù.
Quando la virtù si perfeziona, nutre il \emph{samādhi}, e
quest'aggiuntivo rafforzamento del \emph{samādhi} porta a maturazione la
saggezza. Questi tre aspetti dell'addestramento si intersecano e
armonizzano. Uniti, formano il Nobile Ottuplice Sentiero, la Via del
Buddha. Allorché virtù, \emph{samādhi} e saggezza raggiungono l'apice,
questo Sentiero ha l'energia per sradicare quelle cose che contaminano
(\emph{kilesa}) la purezza della mente. Quando sorge il desiderio dei
sensi, quando la collera e l'illusione mostrano il loro volto, questo
Sentiero è l'unica cosa in grado di bloccarli immediatamente.

Le Quattro Nobili Verità sono il contesto della pratica del Dhamma: la
sofferenza (\emph{dukkha}), l'origine della sofferenza
(\emph{samudaya}), la cessazione della sofferenza (\emph{nirodha}) e il
Sentiero che conduce alla cessazione della sofferenza (\emph{magga}).
Questo Sentiero è fatto di virtù, \emph{samādhi} e saggezza, che a loro
volta sono il contesto nel quale si svolge l'addestramento del cuore. La
loro verità non deve essere cercata nel significato delle parole, ma
dimora nel profondo del nostro cuore. Così sono la virtù, il
\emph{samādhi} e la saggezza. Si alternano in continuazione. Il Nobile
Ottuplice Sentiero avvolgerà qualsiasi cosa si veda, si ascolti, si
odori, si assapori, come pure ogni sensazione tattile o oggetto mentale
che sorga.

Ovviamente, se i fattori dell'Ottuplice Sentiero sono deboli e timidi,
le contaminazioni si impossesseranno delle nostre menti. Se il Nobile
Sentiero è forte e coraggioso, conquisterà e distruggerà le
contaminazioni. Se le contaminazioni sono potenti e audaci, ma il
Sentiero è tenue e fragile, saranno le contaminazioni a conquistare il
Sentiero. Conquisteranno il nostro cuore. Se la conoscenza non è
abbastanza veloce e agile quando vengono sperimentate forme, sensazioni,
percezioni e pensieri, questi si impossesseranno di noi e ci
devasteranno. Il Sentiero e le contaminazioni procedono di pari passo.
Quando nel cuore si sviluppa la pratica del Dhamma, queste due forze si
combattono a ogni passo del cammino. È come se nella mente ci fossero
due persone che discutono, ma si tratta solo del Sentiero del Dhamma e
delle contaminazioni che combattono per conquistare il cuore. Il
Sentiero guida e incoraggia la nostra abilità a contemplare. Finché
siamo in grado di contemplare con accuratezza, le contaminazioni
perderanno terreno. Se però vacilliamo le contaminazioni si
riorganizzeranno e riacquisteranno vigore, il Sentiero sarà sbaragliato
e le contaminazioni prenderanno il suo posto. I due rivali continueranno
a combattere fino a quando ci sarà un vincitore, e tutta la faccenda
sarà risolta.

Se concentriamo i nostri sforzi per sviluppare la via del Dhamma, le
contaminazioni saranno sradicate, in modo graduale ma persistente. Se
coltivate appieno, le Quattro Nobili Verità risiederanno nel nostro
cuore. Quale che sia la forma che assume, la sofferenza esiste sempre in
ragione di una causa. È la Seconda Nobile Verità. Qual è la causa? Una
virtù debole. Un \emph{samādhi} debole. Una saggezza debole. Quando il
Sentiero non è stabile, le contaminazioni dominano la mente. Quando la
dominano, entra in gioco la Seconda Nobile Verità, e sorge ogni genere
di sofferenza. Quando stiamo soffrendo, le qualità in grado di domare la
sofferenza scompaiono. Le condizioni che fanno sorgere il Sentiero sono
la virtù, il \emph{samādhi} e la saggezza. Quando hanno raggiunto piena
forza, il Sentiero del Dhamma è inarrestabile, avanza incessantemente
per sopraffare l'attaccamento e l'aggrapparsi, che ci procurano così
tanta angoscia. La sofferenza non può sorgere perché il Sentiero sta
distruggendo le contaminazioni. È a questo punto che si verifica la
cessazione della sofferenza. Perché il Sentiero è in grado di condurre
alla cessazione della sofferenza? Perché la virtù, il \emph{samādhi} e
la saggezza stanno raggiungendo il culmine della perfezione, e il
Sentiero ha guadagnato uno slancio inarrestabile. Tutto si unifica
proprio qui. Per chi pratica in questo modo, direi che non contano le
idee teoriche sulla mente. Se la mente se n'è liberata, allora è
assolutamente affidabile e sicura. Quale che sia il cammino che
intraprende, non dobbiamo spronarla molto per farla continuare a
procedere diritta.

Considera le foglie di un albero di mango. Come sono? Lo sappiamo quando
ne esaminiamo pure solo una. Anche se ce ne sono decine di migliaia,
sappiamo come sono tutte quante. Guarda solo una foglia. Le altre sono
essenzialmente uguali. Lo stesso avviene con il tronco. Dobbiamo
guardare il tronco di un solo albero di mango per conoscere le
caratteristiche di tutti gli altri. Guarda un solo albero. Tutti gli
altri alberi di mango non riveleranno differenze sostanziali. Anche se
di alberi ce ne fossero centomila, se ne conoscessi uno li conoscerei
tutti. Questo insegnò il Buddha.

Virtù, \emph{samādhi} e saggezza costituiscono il Sentiero del Buddha.
Però, la Via non è l'essenza del Dhamma. Il Sentiero non è di per sé un
fine, non è lo scopo ultimo del Beato. È la Via che conduce verso
l'interiorità. È proprio come quando hai viaggiato da Bangkok fino al
mio monastero, il Wat Nong Pah Pong. Non volevi la strada. Quel che
volevi era raggiungere il monastero, ma per viaggiare avevi bisogno
della strada. La strada sulla quale hai viaggiato non è il monastero. È
solo il modo per arrivarci. Se però volevi arrivare al monastero, dovevi
seguire la strada. Avviene la stessa cosa con la virtù, il
\emph{samādhi} e la saggezza. Potremmo dire che non sono l'essenza del
Dhamma, ma la strada per arrivarci. Quando si ha padronanza della virtù,
del \emph{samādhi} e della saggezza, il risultato è la pace profonda
della mente. Quella è la destinazione. Quando siamo giunti a questa
pace, anche se sentiamo un rumore la mente rimane imperturbata. Quando
abbiamo raggiunto questa pace, non resta nient'altro da fare. Il Buddha
insegnò ad abbandonare tutto. Qualsiasi cosa succeda, non c'è ragione di
preoccuparsi. Allora conosciamo davvero e indiscutibilmente da noi
stessi. Non crediamo più semplicemente a quello che gli altri dicono.

Il principio essenziale del buddhismo è privo di qualsiasi fenomeno. Non
dipende dal manifestarsi di miracolosi poteri psichici, da abilità
paranormali o da qualsiasi altra cosa, mistica o bizzarra che sia. Il
Buddha non sottolineò l'importanza di queste cose. Questi poteri
ovviamente esistono ed è possibile svilupparli, ma questo aspetto del
Dhamma è ingannevole, e perciò il Buddha non li difese né incoraggiò ad
acquisirli. Lodò solo coloro che furono in grado di liberare se stessi
dalla sofferenza. Realizzare questo richiede addestramento, e lo
strumentario e l'attrezzatura per svolgere questo lavoro sono la
generosità, la virtù, il \emph{samādhi} e la saggezza. Dobbiamo
adottarli e addestrarci in essi. Insieme costituiscono un Sentiero che
va nella direzione dell'interiorità, e la saggezza rappresenta il primo
passo. Questo Sentiero non può maturare se la mente è incrostata dalle
contaminazioni. Solo se siamo intrepidi e forti il Sentiero eliminerà
queste impurità. Se sono le contaminazioni a essere intrepide e forti,
saranno ovviamente loro a distruggere il Sentiero. La pratica del Dhamma
coinvolge queste due sole forze che lottano in continuazione, fino a che
si raggiunge la fine della strada. È una battaglia incessante fino alla
fine.

\textbf{I pericoli dell'attaccamento}

Utilizzare gli strumenti della pratica comporta disagi e sfide ardue.
Facciamo affidamento sulla pazienza, sulla sopportazione e sulla
rinuncia. Dobbiamo farlo da noi stessi, sperimentarlo da noi stessi,
realizzarlo da noi stessi. Ovviamente gli studiosi hanno la tendenza a
essere molto confusi. Ad esempio, quando siedono in meditazione, appena
la loro mente sperimenta anche solo un po' di tranquillità iniziano a
pensare: «~Ehi, questo deve essere il primo \emph{jhāna}.~» È così che
lavora la loro mente. E appena sorgono questi pensieri, la tranquillità
che hanno sperimentato va in pezzi. Presto cominciano a pensare che quel
che hanno raggiunto è il secondo \emph{jhāna}. Non pensarci su, non
specularci sopra. Non ci sono cartelli che annuncino quale livello di
\emph{samādhi} si stia sperimentando. La realtà è del tutto diversa. Non
c'è una segnaletica stradale che ti dice: «~Questa strada conduce al Wat
Nong Pah Pong.~» Non è così che si legge la mente. Essa non ci notifica
nulla.

Benché numerosi e stimati studiosi abbiano descritto il primo, il
secondo, il terzo e il quarto \emph{jhāna}, quel ch'è stato scritto è
solo un'informazione esterna. Se la mente entra davvero in questi stati
di pace profonda, non sa nulla di queste descrizioni. Conosce, ma ciò
che conosce non è la stessa cosa della teoria che studiamo. Se gli
studiosi cercano di afferrare questa teoria e di trascinarla nella loro
meditazione, pensando e soppesando: «~Hmm ... questo cosa potrebbe
essere? È già il primo \emph{jhāna}?~» Ecco, la pace va in pezzi e loro
non sperimentano nulla che abbia un reale valore. Perché? Perché c'è
desiderio, e quando c'è la brama che succede? La mente si ritrae dalla
meditazione. Per tutti noi è perciò necessario abbandonare i pensieri e
le speculazioni. Abbandonale del tutto. Prendi il corpo, la parola e la
mente e addentrati nella pratica. Osserva come lavora la mente, ma
mentre lo fai non trascinarti dietro i libri di Dhamma, altrimenti ogni
cosa diventa un gran pasticcio, perché in quei libri nulla corrisponde
con precisione alla realtà del modo in cui sono veramente le cose.

Di solito la gente che studia molto, chi è pieno di conoscenze teoriche
non riesce nella pratica del Dhamma. Si resta impantanati al livello
delle informazioni. La verità è che il cuore e la mente non possono
essere misurati mediante criteri esteriori. Se la mente ottiene la pace,
consentile solo di stare in pace. I livelli più intensi di pace profonda
esistono. Per quanto mi concerne, non sapevo molto di teoria. Ero stato
monaco per tre anni e avevo ancora molti interrogativi su cosa fosse in
realtà il \emph{samādhi}. Mentre meditavo continuavo a cercare di
pensarci e di capirlo, ma la mia mente diventava ancora più inquieta e
distratta di prima! Il numero di pensieri in realtà aumentava. Quando
non stavo meditando era più serena. Ragazzi, se era difficile! Era
esasperante! Però, benché io abbia incontrato così tanti ostacoli, non
ho mai gettato la spugna. Ho continuato a meditare e basta. Quando non
cercavo di fare nulla di particolare, la mia mente era abbastanza a suo
agio. Tutte le volte che ero determinato a unificare la mente nel
\emph{samādhi}, me ne sfuggiva il controllo. Mi chiedevo: «~Che cosa sta
succedendo? Perché succede tutto questo?~» In seguito iniziai a capire
che la meditazione era paragonabile al processo della respirazione. È
molto difficile forzare il respiro a essere lieve, profondo o anche solo
normale. Ovviamente, è invece molto rilassante se andiamo a fare una
passeggiata senza nemmeno essere consci di quando inspiriamo o
espiriamo. Mi misi a riflettere: «~Ah! Forse è così che funziona.~»
Quando durante la giornata una persona cammina normalmente, senza
focalizzare l'attenzione sul respiro, il respiro induce sofferenza? No,
ci si sente rilassati e basta. Quando però ci sediamo e con
determinazione decidiamo di rendere serena la mente, entrano in gioco
l'aggrapparsi e l'attaccamento. Quando cercavo di controllare il respiro
affinché fosse lieve o profondo, ciò comportava una tensione maggiore di
prima. Perché? Perché la forza di volontà che stavo usando era
contaminata dall'aggrapparsi e dall'attaccamento. Non sapevo che cosa
stesse succedendo. Tutta quella frustrazione e quel disagio nascevano
perché portavo la brama nella meditazione.

\textbf{Una pace incrollabile}

Una volta mi trovavo in un monastero della foresta che distava meno di
un chilometro da un villaggio. Una sera gli abitanti del villaggio
stavano rumorosamente festeggiando mentre facevo la meditazione
camminata. Dovevano essere all'incirca le undici, e mi sentivo un po'
strano. Mi sentivo così da dopo mezzogiorno. La mia mente era serena.
Non c'era quasi nessun pensiero. Ero davvero rilassato e a mio agio.
Feci la meditazione camminata fino a quando fui stanco, e poi andai a
sedere nella mia capanna col tetto impagliato. Appena mi misi seduto
ebbi a mala pena il tempo d'incrociare le gambe prima che, in modo
sorprendente, la mia mente volle solo addentrarsi in uno stato di pace
profonda. Avvenne da sé. Appena mi misi seduto, la mente divenne proprio
serena. Era solida come una roccia. Era come se non potessi sentire i
rumori degli abitanti del villaggio che cantavano e ballavano, o meglio,
potevo, ma potevo anche tagliare completamente fuori i rumori.

Strano. Quando non prestavo attenzione ai rumori, c'era un silenzio
perfetto, non sentivo nulla. Se però volevo sentire, potevo farlo senza
che ciò mi arrecasse disturbo. Era come se nella mia mente ci fossero
due oggetti affiancati, ma che non si toccavano. Potevo vedere che la
mente e il suo oggetto di consapevolezza erano separati e distinti,
proprio come la sputacchiera e il bollitore che stanno qui. Allora
compresi: quando la mente si unifica nel \emph{samādhi}, se si dirige
l'attenzione all'esterno si può udire, ma se la lasci dimorare nella sua
vacuità, allora c'è un silenzio perfetto. Quando percepivo il suono,
potevo percepire con chiarezza che la conoscenza e il suono erano
diversi. Contemplavo: «~Se questo non è il modo in cui è, come potrebbe
altrimenti essere?~» Era così. Queste due cose erano del tutto separate.
Continuai a investigare in questa maniera fino a che la mia comprensione
divenne ancor più profonda: «~Ah, questa è una cosa importante. Quando
si interrompe la percezione della continuità dei fenomeni, il risultato
è la pace.~» La precedente illusione della continuità (\emph{santati})
si trasformò nella pace della mente (\emph{santi}). Così continuai a
sedere, misi energia nella meditazione. Allora la mia mente era
focalizzata solo sulla meditazione, era indifferente a qualsiasi altra
cosa. Se a questo punto avessi smesso di fare meditazione, ciò sarebbe
potuto avvenire solo perché essa era stata portata a termine. Avrei
potuto prendermela con calma, ma ciò non sarebbe mai potuto verificarsi
per pigrizia, stanchezza o noia. Assolutamente no. Queste cose erano
assenti dal mio cuore. Vi era solo un perfetto bilanciamento ed
equilibrio interiore, del tutto perfetto.

Tutt'al più mi prendevo una pausa, ma era solo la postura seduta a
cambiare. Il mio cuore restava costante, incrollabile e instancabile.
Presi un cuscino, volevo riposare. Mentre mi sdraiavo la mia mente restò
serena, proprio come lo era in precedenza. Poi, giusto prima che la
testa toccasse il cuscino, la consapevolezza della mente iniziò a fluire
verso l'interno, non sapevo dove fosse diretta, ma continuava a fluire
sempre più in profondità, verso l'interno. Era come una corrente
elettrica che scorreva in un cavo verso un interruttore. Quando colpì
l'interruttore, il mio corpo esplose con un botto assordante. Per quel
lasso di tempo la conoscenza fu estremamente lucida e sottile. Superato
quel punto, la mente fu libera di penetrare all'interno, in profondità.
Arrivò in quel punto ove non c'è proprio nulla. Dal mondo esterno
assolutamente niente poteva raggiungere quel posto. Proprio nulla poteva
raggiungerlo. Dopo aver dimorato internamente per un po' di tempo, la
mente si ritrasse, per fluire di nuovo all'esterno. Ovviamente, quando
dico che si ritrasse, non voglio intendere che la feci fluire di nuovo
all'esterno. Ero solo un osservatore, conoscevo e testimoniavo, non
facevo nient'altro. La mente uscì sempre più all'esterno, per tornare
infine alla normalità.

Quando tornai al mio normale stato di coscienza, sorse una domanda.
«~Che cosa è stato?!~» La risposta giunse immediatamente. «~Queste cose
avvengono da sé. Non devi cercare una spiegazione.~» Una risposta
sufficiente a soddisfare la mia mente. Dopo poco tempo la mente iniziò
di nuovo a fluire verso l'interno. Non stavo compiendo alcuno sforzo
cosciente per dirigere la mente. Decollò da sola. Mentre si muoveva
sempre più in profondità verso l'interno, colpì nuovamente lo stesso
interruttore. Questa volta il mio corpo andò in frantumi, scomponendosi
nelle più piccole particelle e frammenti. La mente fu di nuovo libera di
penetrare in profondità dentro se stessa. Silenzio assoluto. Ancor più
della prima volta. Assolutamente nulla di esterno poteva raggiungerla.
La mente dimorò qui per un po', per tutto il tempo che volle, e poi si
ritrasse per fluire all'esterno. Stava seguendo il suo stesso impulso e
tutto accadde da sé. Non stavo esercitando alcun influsso particolare
sulla mia mente per farla scorrere all'interno o farla ritrarre per
uscire all'esterno. Ero solo colui che conosce e osserva.

La mia mente tornò di nuovo al suo normale stato di coscienza, ma io non
mi posi domande né feci congetture a proposito di quello che stava
avvenendo. Quando meditai, la mente si diresse ancora una volta verso
l'interno. Ora fu l'intero cosmo a frantumarsi e a disintegrarsi in
minuscole particelle. La terra, il suolo, le montagne, i campi e le
foreste -- tutto il mondo -- si disintegrarono nell'elemento spazio. La
gente era svanita. Era tutto scomparso. Non rimase assolutamente nulla,
questa terza volta.

La mente, dopo essersi diretta all'interno, vi restò per tutto il tempo
che volle. Non posso dire di aver capito esattamente come vi restò. È
difficile descrivere cosa avvenne. Non riesco a fare paragoni. Non c'è
similitudine adatta. Questa volta la mente restò all'interno molto più a
lungo di prima, e solo dopo un po' di tempo uscì da quello stato. Quando
dico che uscì, non intendo che fui io a farla uscire o che stavo tenendo
sotto controllo quel che avveniva. La mente lo fece da sé. Io ero solo
un osservatore. Alla fine tornò al suo normale stato di coscienza. Come
si potrebbe attribuire un nome a quel che avvenne queste tre volte? Chi
lo sa? Quale termine useresti per etichettarlo?

\textbf{Il potere del \emph{samādhi}}

Tutto quello che ti ho raccontato riguarda la mente che segue la via
della natura. Non è stata una descrizione teorica né della mente né di
stati psicologici. Non ce n'è bisogno. Quando c'è fede e fiducia, lì ci
arrivi e lo fai davvero. Non ci girare attorno, metti in gioco la tua
vita. E quando la tua pratica raggiunge lo stadio che ti ho descritto,
dopo è tutto il mondo che risulta capovolto. La tua comprensione della
realtà è del tutto diversa. Il tuo modo di vedere si trasforma
completamente. Se qualcuno in quel momento ti vedesse, potrebbe pensare
che sei matto. Se qualcuno che non sa controllarsi facesse questa
esperienza, potrebbe impazzire, perché niente è più come prima. La gente
del mondo appare in modo diverso rispetto a come è di solito. Però, lo
vedi solo tu. Tutto cambia, assolutamente. I tuoi pensieri sono
trasformati: ora gli altri pensano in un modo, tu in un altro. Loro
parlano delle cose in un modo, tu in un altro. Loro scendono in una
direzione, tu sali in un'altra. Non sei più come gli altri esseri umani.
Questa maniera di sperimentare le cose non si deteriora. Persiste e
prosegue. Provaci. Se è proprio così come ti dico, non dovrai andare a
cercare molto lontano. Guarda solo dentro il tuo cuore. Questo cuore è
strenuamente coraggioso, incrollabilmente audace. Questo è il potere del
cuore, la sua fonte di forza e di energia. Potenzialmente il cuore ha
questa forza. Questo è il potere e la forza del \emph{samādhi}.

A questo punto c'è solo la forza e la purezza che la mente ottiene dal
\emph{samādhi}. Ora il \emph{samādhi} è al suo massimo livello. La mente
ha raggiunto il vertice del \emph{samādhi}. Non è semplice
concentrazione momentanea. Se a questo punto attivassi la meditazione
\emph{vipassanā}, la contemplazione sarebbe ininterrotta e penetrante.
Oppure potresti prendere quest'energia focalizzata e utilizzarla in
altri modi. Da questo punto in poi potresti sviluppare poteri psichici,
operare atti miracolosi oppure avvalertene nel modo che preferisci. Gli
asceti e gli eremiti hanno utilizzato l'energia del \emph{samādhi} per
produrre acqua santa e talismani o per fare incantesimi. Sono tutte cose
possibili a questo livello, e a loro modo possono essere di un qualche
beneficio. Ma è come il beneficio dell'alcol. Lo bevi e ti ubriachi.

Questo livello di \emph{samādhi} è una sosta per riposare. Il Buddha si
fermò qui e si riposò. È il fondamento per la contemplazione e per la
\emph{vipassanā}. Non è ovviamente indispensabile avere un
\emph{samādhi} così profondo per osservare i fenomeni condizionati che
stanno attorno a noi, perciò continua a contemplare costantemente i
processi di causa ed effetto. Per farlo, focalizziamo la pace e la
chiarezza della nostra mente per analizzare ciò che vediamo, sentiamo,
odoriamo e tocchiamo, le sensazioni fisiche, i pensieri e gli stati
mentali che sperimentiamo. Esaminiamo i nostri umori e le nostre
emozioni, sia positivi sia negativi, sia felici sia tristi. Esaminiamo
tutto. È come se qualcun altro fosse salito su un albero di manghi e lo
scuotesse per far cadere i frutti, mentre noi aspettiamo sotto per
raccoglierli. Quelli marci non li prendiamo. Raccogliamo solo i manghi
buoni. Non è faticoso, perché non abbiamo bisogno di arrampicarci
sull'albero. Aspettiamo solo sotto per raccogliere i frutti.

Capisci il significato di questa similitudine? Tutto ciò che
sperimentiamo con mente serena conduce a una maggior comprensione. Non
creiamo più proliferazioni e interpretazioni attorno a quello che
sperimentiamo. Ricchezza, fama, lode, biasimo, felicità e infelicità
vanno e vengono da sé. E noi siamo in pace. Siamo saggi. In verità è
divertente. Diventa divertente vagliare queste cose e distinguerle l'una
dall'altra. Quello che gli altri chiamano buono, bene, male, qui, lì,
felicità, infelicità e così via. Tutto viene a nostro profitto. Qualcun
altro è salito sull'albero di manghi e sta scuotendo i rami per far
cadere i manghi per noi. Noi ci divertiamo solo a raccogliere i frutti
senza timore. Cosa c'è da temere, a ogni modo? È qualcun altro a far
cadere i manghi per noi. Ricchezza, fama, lode, biasimo, felicità,
infelicità e tutto il resto non sono altro che manghi che cadono, e noi
li esaminiamo con cuore sereno. Allora sapremo quali sono buoni e quali
marci.

\textbf{Lavorare in accordo con la natura}

Quando iniziamo a utilizzare la pace e la serenità che abbiamo
sviluppato nella meditazione per contemplare queste cose, sorge la
saggezza. Questo è ciò che chiamo saggezza. Questo è \emph{vipassanā}.
Non si tratta di un qualcosa di inventato o di costruito. Se siamo
saggi, la \emph{vipassanā} si svilupperà naturalmente. Non c'è bisogno
di etichettare quel che avviene. Se c'è anche solo poca chiara visione
profonda, la chiamiamo ``piccola \emph{vipassanā}''. Quando la chiara
visione cresce un po', la chiamiamo ``\emph{vipassanā} moderata''. Se la
conoscenza è del tutto in sintonia con la Verità, la chiamiamo
``\emph{vipassanā} suprema''. Personalmente, invece di \emph{vipassanā}
preferisco usare la parola \emph{paññā} (saggezza). Se pensiamo di
sederci di tanto in tanto per praticare la meditazione \emph{vipassanā},
stiamo per andare incontro a momenti molto difficili. La visione
profonda deve giungere dalla pace e dalla tranquillità. L'intero
processo avverrà naturalmente, da sé. Non possiamo forzarlo.

Il Buddha insegnò che questo processo matura seguendo un proprio ritmo.
Raggiunto questo livello della pratica, le consentiamo di svilupparsi
secondo le nostre innate capacità, le nostre attitudini spirituali e i
meriti accumulati in passato. Però, non smettiamo mai di impegnarci
nella pratica. Che il progresso sia celere o lento, è al di là del
nostro controllo. È proprio come piantare un albero. L'albero sa quanto
velocemente deve crescere. Se vogliamo che cresca più velocemente, si
tratta di mera illusione. Se vogliamo che cresca più lentamente,
riconosciamo che anche questa è un'illusione. Se svolgiamo il lavoro, i
risultati verranno, proprio come quando piantiamo un albero. Quando ad
esempio vogliamo un cespuglietto di peperoncini piccanti, il nostro
compito consiste nello scavare una buca, interrare la piantina,
innaffiarla, concimarla e proteggerla dagli insetti. Questo è il nostro
lavoro, qui si conclude per noi la questione. È qui che entra in ballo
la fiducia. Non dipende da noi che la pianta di peperoncini cresca o
meno. Non è affar nostro. Non possiamo strattonare la pianta, cercare di
allungarla per farla crescere più velocemente. Non è così che lavora la
natura. Nostra responsabilità è innaffiarla e concimarla. Praticare il
Dhamma in questo stesso modo mette il nostro cuore a suo agio.

Se realizziamo l'Illuminazione in questa vita, va benissimo. Se dobbiamo
aspettare fino alla prossima, non importa. Abbiamo fiducia e crediamo
risolutamente nel Dhamma. Progredire velocemente o lentamente dipende
dalle nostre capacità, dalle nostre attitudini spirituali e dai meriti
che abbiamo accumulato fino a ora. Praticare in questo modo mette il
cuore a proprio agio. È come stare su un carretto trainato da un
cavallo. Non mettiamo il carretto davanti al cavallo. Oppure, è come se
cercassimo di arare una risaia camminando davanti al nostro bufalo
d'acqua piuttosto che dietro. Quel che ti sto dicendo è che la mente
cammina davanti a se stessa. È impaziente di ottenere celeri risultati.
Non è così che si fa. Devi camminare dietro il bufalo d'acqua.

È proprio come quella pianta di peperoncini che stiamo coltivando.
Innaffiala e concimala, ed essa svolgerà il lavoro di assorbire il
nutrimento. Quando le formiche o le termiti la infestano, le cacciamo
via. Fare questo è sufficiente perché i peperoncini crescano
magnificamente da sé, quando pensiamo che dovrebbero fiorire non
cerchiamo di forzarli a farlo. Non è affar nostro. Creiamo solo inutile
sofferenza. Consenti ai fiori di sbocciare da sé. E quando i fiori sono
sbocciati, non pretendiamo che producano immediatamente peperoncini. Non
fare affidamento sulla coercizione. Causa davvero sofferenza! Una volta
che l'abbiamo capito, comprendiamo quali sono le nostre responsabilità e
quali no. Ognuno ha il suo compito specifico da svolgere. La mente sa
qual è il suo ruolo nel lavoro da fare. Se la mente non lo comprende,
cercherà di forzare la pianta a produrre peperoncini nel giorno stesso
in cui l'abbiamo piantata. La mente insisterà che cresca, fiorisca e
produca peperoncini, tutto in un giorno.

Questo non è nient'altro che la Seconda Nobile Verità: la brama è la
causa che fa sorgere la sofferenza. Se siamo consapevoli di questa
Verità e la prendiamo in considerazione, capiremo che cercare di forzare
i risultati nella nostra pratica del Dhamma è pura illusione. È
sbagliato. Comprendendo come funziona, lasciamo andare e consentiamo
alle cose di maturare in accordo con le nostre capacità innate, le
nostre attitudini spirituali e i meriti accumulati. Continuiamo a fare
la nostra parte. Non preoccuparti del fatto che potrebbe volerci molto
tempo. Anche se ci volessero cento o mille vite per ottenere
l'Illuminazione, e allora? Per quanto numerose siano le vite che ci
vogliono, continueremo solo a praticare con la nostra andatura, con il
cuore sereno e a proprio agio. Una volta che la mente è entrata nella
``Corrente'', non c'è nulla da temere. È andata al di là perfino della
più piccola cattiva azione. Il Buddha disse che la mente di un
\emph{sotāpanna},\footnote{\emph{sotāpanna}. ``Chi è entrato nella
  Corrente'' e ha così conseguito il primo livello dell'Illuminazione.}
di chi ha raggiunto il primo stadio dell'Illuminazione, è entrata nella
Corrente del Dhamma che scorre verso l'Illuminazione. Queste persone non
dovranno più sperimentare i truci regni inferiori dell'esistenza, non
cadranno di nuovo nell'inferno. Com'è possibile che cadano nell'inferno
se le loro menti hanno abbandonato il male? Hanno visto il pericolo che
risiede nel produrre cattivo \emph{kamma}. Anche se tu cercassi di
costringerli a fare o a dire qualcosa di male, ne sarebbero incapaci, e
perciò non è possibile che precipitino di nuovo nell'inferno o nei
regni inferiori dell'esistenza. La loro mente scorre con la Corrente del
Dhamma.

Quando sei nella Corrente, sai quali sono le tue responsabilità. Conosci
il lavoro che hai da fare. Sai come praticare il Dhamma. Sai quando
sforzarti e quando rilassarti. Comprendi il tuo corpo e la tua mente,
questo processo fisico e mentale, e rinunci alla cose alle quali si
dovrebbe rinunciare, abbandonandole, continuamente, senza alcuna ombra
di dubbio.

\textbf{Modificare il nostro modo di vedere}

Nella mia vita di pratica del Dhamma non ho tentato di padroneggiare
un'ampia gamma di questioni. Solo una. Affinare questo cuore. Diciamo
che stiamo guardando un corpo. Se notiamo che ne siamo attratti, allora
analizziamolo. Guarda per bene: capelli, peli, unghie, denti e
pelle.\footnote{capelli (\emph{kesā}), peli (\emph{lomā}), unghie
  (\emph{nakhā}), denti (\emph{dantā}) e pelle (\emph{taco}). La
  contemplazione di queste cinque parti del corpo costituisce la prima
  tecnica meditativa insegnata dal precettore a un monaco o a una monaca
  appena ordinati.} Il Buddha ci insegnò a contemplare accuratamente e
continuamente queste parti del corpo. Visualizzale separatamente,
staccale, togli la pelle al corpo e brucia il tutto con la mente. È così
che si fa. Attieniti a questa meditazione fino a quando essa si è
insediata in modo fermo e incrollabile. Vedi tutti allo stesso modo. Ad
esempio, quando al mattino i monaci e i novizi vanno al villaggio per la
questua, chiunque vedano -- che si tratti di un altro monaco o di un
abitante del villaggio, uomo o donna che sia -- lo immaginano come un
corpo morto, come un cadavere che cammina barcollando davanti a loro
lungo la strada. Devono restare concentrati su questa percezione. È così
che ci si sforza. Ciò conduce alla maturità e allo sviluppo spirituale.
Quando vedi una giovane donna che trovi attraente, immaginala come un
cadavere che cammina, con il corpo putrido e maleodorante per la
decomposizione. Vedi tutti in questo modo. E non lasciare che ti si
avvicinino troppo! Non consentire che l'infatuazione persista nel tuo
cuore. Se percepisci gli altri come putridi e maleodoranti, ti posso
assicurare che l'infatuazione non persisterà.

Contempla fino a quando sei sicuro di ciò che stai vedendo, fino a che
la visione non è definita, fino a quando diventi esperto. Su qualsiasi
strada poi camminerai, non ti perderai. Mettici tutto il cuore. Ogni
volta che vedi qualcuno, è come osservare un cadavere. Che si tratti di
un uomo o di una donna, guarda quella persona come un corpo morto. E non
dimenticare di vedere te stesso come un corpo morto. Alla fine è tutto
quel che resta. Cerca di sviluppare questo modo di vedere con la maggior
accuratezza possibile. Se lo fai davvero, ti assicuro che è molto
divertente. Se invece ti preoccupi di leggerlo nei libri, avrai momenti
difficili. Devi farlo, e fallo con la massima sincerità. Fallo finché
questa meditazione diventa parte di te. Fai che il tuo scopo sia la
realizzazione della Verità. Se sarai motivato dal desiderio di
trascendere la sofferenza, sarai sul giusto Sentiero.

Di questi tempi c'è molta gente che insegna \emph{vipassanā} e un'ampia
gamma di tecniche di meditazione. Ti dico questo: fare \emph{vipassanā}
non è facile. Non possiamo saltarci dentro direttamente. Non funzionerà,
se non si parte da un alto livello di moralità. Scoprilo da solo. La
disciplina morale e l'addestramento nei precetti sono cose necessarie,
perché se il nostro comportamento, le nostre azioni e le nostre parole
non sono impeccabili non saremo mai in grado di stare dritti sulle
nostre gambe. Fare meditazione senza avere la virtù è come cercare di
saltare un tratto essenziale del Sentiero. Allo stesso modo, di tanto in
tanto si sente la gente che dice: «~Non c'è bisogno di sviluppare la
tranquillità. Saltatela e andate direttamente alla meditazione di
visione profonda, nella \emph{vipassanā.}~» È la gente sciatta, quella
che gradisce le scorciatoie, a parlare in questo modo. Dice che non
bisogna preoccuparsi della disciplina morale. Sostenere e affinare la
virtù è impegnativo, non significa trastullarsi. Se potessimo saltare
tutti gli insegnamenti riguardanti l'etica comportamentale, sarebbe
molto facile, vero? Tutte le volte che incontreremmo una difficoltà, la
eviteremmo saltandola. A tutti piace saltare le difficoltà.

Una volta incontrai un monaco che mi disse di essere un vero meditante.
Mi chiese il permesso di restare qui, con me, e mi chiese del programma
e del livello di disciplina monastica. Gli spiegai che in questo
monastero viviamo seguendo il Vinaya, il codice di disciplina monastica
del Buddha, e che se lui voleva venire ad addestrarsi con me avrebbe
dovuto rinunciare al suo denaro e agli altri suoi beni personali. Mi
rispose che la sua pratica consisteva nel ``non attaccamento a tutte le
convenzioni''. Gli dissi che non sapevo di cosa stesse parlando.
«~Potrei restare qui -- affermò -- e tenere il mio denaro senza
attaccarmi a esso. Il denaro è solo una convenzione.~» «~Certamente --
gli dissi -- non c'è problema. Se puoi mangiare del sale senza trovarlo
salato, allora puoi usare il denaro senza esservi attaccato.~» Stava
dicendo solo cose senza senso. In verità, era solo troppo pigro per
seguire il Vinaya nei dettagli. Te lo sto dicendo, è difficile. «~Se
puoi mangiare del sale e assicurarmi in tutta onestà che non è salato,
allora ti prenderò in seria considerazione. E se mi dici che non è
salato, ti darò allora un intero sacco di sale da mangiare. Provaci.
Davvero non sarà salato? Il non attaccamento alle convenzioni non
consiste solo nel parlare in modo ingegnoso. Se parli così, non puoi
restare con me.~» Se ne andò.

Dobbiamo cercare di mantenere la pratica della virtù. I monaci
dovrebbero addestrarsi con l'esperienza delle pratiche ascetiche
(\emph{dhutanga}), mentre i laici che praticano a casa dovrebbero
osservare i Cinque Precetti. Tentate di essere impeccabili in tutto quel
che dite e fate. Dovremmo coltivare la bontà al meglio delle nostre
capacità, continuando a farlo gradualmente.

Quando inizi a coltivare la tranquillità della meditazione di
\emph{samatha}, non fare l'errore di tentare una o due volte per poi
rinunciare perché la mente non è serena. Non è la strada giusta. Devi
coltivare la meditazione per un lungo periodo di tempo. Perché ci vuole
così tanto? Pensaci. Per quanti anni abbiamo consentito alla nostra
mente di vagare e di smarrirsi? Per quanti anni non abbiamo praticato la
meditazione di \emph{samatha}? Tutte le volte che la mente ci ha
ordinato di seguirla in un suo particolare percorso, ci siamo affrettati
a farlo. Per calmare questa mente vagabonda, per indurla a fermarsi, per
renderla serena, non saranno sufficienti un paio di mesi di meditazione.
Riflettici su.

Quando ci impegniamo ad addestrare la mente affinché sia tranquilla in
ogni situazione, per favore comprendi che all'inizio, quando sorgerà
un'emozione contaminata, la mente non sarà serena. Sarà distratta e
incontrollabile. Perché? Perché c'è brama. Non vogliamo che la nostra
mente pensi. Non vogliamo sperimentare alcun stato mentale o emozione
che ci distragga. Non volere è brama, brama per la non-esistenza. Più
brama abbiamo di non sperimentare certe cose, più le invitiamo a entrare
in noi. «~Non voglio queste cose, ma allora perché continuano a venire
da me? Se desidero che non sia così, perché è così?~» Ecco qua! Bramiamo
che le cose esistano in un certo modo perché non comprendiamo la nostra
mente. Può volerci un tempo incredibilmente lungo prima di capire che
trastullarsi con queste cose è un errore. Alla fine, quando valutiamo il
tutto con chiarezza, comprendiamo: «~Oh, queste cose arrivano perché le
chiamo.~»

Essere bramosi di non sperimentare qualcosa, di essere sereni, di non
essere distratti e agitati. È tutta bramosia. È tutto uguale a un pezzo
di ferro incandescente. Però non preoccupartene. Vai solamente avanti
con la pratica. Tutte le volte che sperimentiamo uno stato mentale o
un'emozione, li esaminiamo nei termini dell'impermanenza, del loro
carattere insoddisfacente e privo di un sé, e li gettiamo in una di
queste tre categorie. Poi rifletti e investiga. Queste emozioni
contaminate sono per lo più accompagnate da troppi pensieri. Ovunque uno
stato mentale sia diretto, il pensiero ci vagabonda dietro. Il pensiero
e la saggezza sono due cose molto diverse. Il pensiero reagisce ai
nostri stati mentali e li segue, e i pensieri continuano senza che sia
possibile intravedere una fine. Però, se la saggezza è attiva, condurrà
la mente alla quiete. La mente si ferma e non va da nessuna parte. C'è
solo il conoscere e il riconoscere ciò che si sta sperimentando. Quando
questa emozione arriva, la mente è in questo modo. Quando arriva quello
stato mentale, è in quell'altro modo. Sosteniamo la ``conoscenza''. Alla
fine ecco cosa ci viene in mente: «~Ehi, tutti questi pensieri, tutto
questo chiacchiericcio mentale privo di uno scopo, queste preoccupazioni
e giudizi. È tutto insostanziale e privo di senso. È tutto impermanente,
insoddisfacente, questi pensieri non sono io né sono miei.~» Gettali in
una di queste tre categorie onnicomprensive, e acquieta il loro
insorgere. Li elimini alla fonte. In seguito, quando sediamo di nuovo in
meditazione, torneranno di nuovo. Guardali da vicino. Spiali.

È proprio come allevare bufali d'acqua. C'è un contadino, ci sono alcune
piante di riso e un bufalo d'acqua. Ora, succede che il bufalo d'acqua
voglia mangiare quelle piante di riso. Ai bufali d'acqua piace mangiare
le piante di riso, vero? La tua mente è un bufalo d'acqua. Le emozioni
contaminate sono come le piante di riso. La conoscenza è il contadino.
La pratica del Dhamma è proprio così. Non c'è differenza. Paragonali tu
stesso. Quando ti prendi cura di un bufalo d'acqua, che fai? Lo lasci
libero e gli permetti di andare in giro liberamente, ma lo tieni
d'occhio. Se gironzola troppo vicino alle piante di riso, gridi. Quando
il bufalo sente, si allontana. Non essere distratto, ignaro di cosa il
bufalo stia facendo. Se hai un bufalo d'acqua testardo che non obbedisce
ai tuoi avvertimenti, prendi una bacchetta e dagli un bel colpo sul di
dietro. Poi non oserà avvicinarsi alle piante di riso. Non farti
prendere dalla voglia di schiacciare un pisolino. Se ti sdrai e ti
appisoli, quelle piante di riso saranno solo un ricordo. La pratica del
Dhamma è la stessa cosa: sorveglia la mente, la conoscenza si prende
cura della mente.

«~Chi continua a sorvegliare la propria mente da vicino sarà libero dai
lacci di Māra.~» Però, se questa mente che conosce è pur sempre la
mente, chi è allora che osserva la mente? Questo genere di domande
possono confonderti molto. La mente è una cosa, la conoscenza un'altra
cosa, ma la conoscenza si origina proprio in questa stessa mente. Che
cosa significa conoscere la mente? Come ci si sente ad affrontare stati
mentali ed emozioni? Come ci si sente a essere privi di qualsiasi
emozione contaminata? Quel che conosce ciò che queste cose sono, è
quanto intendiamo con ``conoscere''. Il conoscere segue la mente
osservandola, ed è da questo conoscere che nasce la saggezza. La mente è
ciò che pensa e resta intrappolato nelle emozioni, una dopo l'altra,
proprio come il nostro bufalo d'acqua. Quale che sia la direzione verso
la quale vaga, continua a tenerla d'occhio con attenzione. Come potrebbe
mai scappare? Se comincia ad andare nella direzione delle piante di
riso, grida. Se non ascolta, prendi una bacchetta e giù un colpo.
«~Stack!~» È così che frusti la brama.

Addestrare la mente non è diverso. Quando la mente sperimenta
un'emozione e immediatamente la afferra, il compito della conoscenza è
insegnare alla mente. Esamina lo stato mentale per vedere se è buono o
cattivo. Spiega alla mente come funzionano causa ed effetto. E quando
afferra di nuovo qualcosa che essa pensa sia adorabile, la conoscenza
deve di nuovo insegnare alla mente, spiegarle di nuovo causa ed effetto,
fino a che la mente non è in grado di mettere da parte quella cosa. Ciò
conduce alla pace della mente. Dopo aver capito che qualsiasi cosa
prenda e afferri è intrinsecamente indesiderabile, la mente non fa altro
che fermarsi. Non può più essere disturbata da quelle cose, perché è
costantemente incorsa in una raffica di sgridate e rimproveri. Ostacola
con determinazione la bramosia della mente. Sfidala direttamente, fino a
che gli insegnamenti penetrano nel cuore. È così che addestri la mente.

Ho praticato in questo modo fin dal tempo in cui mi ritirai nella
foresta per fare meditazione. Quando istruisco i miei discepoli, dico
loro di praticare in questo modo, perché voglio che vedano la Verità
invece di leggere soltanto quel che c'è nelle Scritture. Voglio che
capiscano se il loro cuore si è liberato dal pensiero concettuale.
Quando arriva la Liberazione, lo sai, e quando la Liberazione non è
ancora avvenuta, contempli il processo di come una cosa ne generi
un'altra e conduca a un'altra cosa ancora. Contempla fino a quando lo
capisci e lo comprendi a fondo. Appena sarà penetrato dalla visione
profonda, questo processo svanirà da sé. Quando sul tuo cammino arriva
qualcosa che ti blocca, investiga quella cosa. Non arrenderti fino a
quando non molla la presa. Investiga ripetutamente proprio lì. Per
quanto mi riguarda, è in questo modo che ho affrontato l'addestramento,
perché il Buddha insegnò che devi conoscere da te stesso. Tutti i saggi
conoscono la Verità da se stessi. Devi scoprirla nel profondo del tuo
cuore. Conosci te stesso.

Se fai affidamento su quel che conosci e hai fiducia in te stesso, sarai
rilassato sia che ti critichino sia che ti lodino. Qualsiasi cosa gli
altri dicano, sarai a tuo agio. Perché? Perché conosci te stesso. Se
qualcuno ti appoggia lodandoti, ma tu sai di non meritarlo veramente,
gli credi davvero? Ovviamente no. Continui solo con la tua pratica del
Dhamma. Quando facciamo affidamento su quel che sappiamo e gli altri ci
lodano, veniamo risucchiati dalla lode e questo deforma la nostra
percezione. Allo stesso modo quando qualcuno ti critica, dai
un'occhiata, esamina te stesso. «~No, quel che dicono non è vero. Mi
accusano di avere torto, ma in verità non è così. La loro accusa non
vale.~» Se è così, a che servirebbe arrabbiarsi con loro? Le loro parole
non sono vere. Ovviamente, se siamo in torto come dicono, allora le loro
accuse sono giuste. Se è così, a che servirebbe arrabbiarsi con loro?
Quando sei in grado di pensare in questo modo, la vita è veramente
serena e agiata. Nulla di quel che succede è sbagliato. Allora tutto è
Dhamma. Io ho praticato così.

\textbf{Seguire la Via di Mezzo}

È la Via più corta e diretta. Potresti venire da me a discutere su
questioni di Dhamma, ma io non mi unirei a te. Invece di controbattere,
ti offrirei solo alcune riflessioni da prendere in considerazione. Per
favore, comprendi quel che insegnò il Buddha: lascia andare tutto.
Lascia andare con conoscenza e consapevolezza. Senza conoscenza e senza
consapevolezza, il lasciar andare non è diverso da quello delle mucche e
dei bufali d'acqua. Senza metterci il cuore, il lasciar andare non è
corretto. Lascia andare perché comprendi la realtà convenzionale. Questo
è non attaccamento. Il Buddha insegnò che negli stadi iniziali della
pratica del Dhamma dovresti lavorare davvero duro, sviluppare le cose a
fondo e attaccarti molto. Attaccarti al Buddha. Attaccarti al Dhamma.
Attaccarti al Saṅgha. Attaccarti con fermezza e a fondo. Questo insegnò
il Buddha. Attaccarti con sincerità e persistenza, e tenerti forte.

Nella mia personale ricerca ho provato quasi ogni mezzo possibile di
contemplazione. Ho dedicato la mia vita al Dhamma, perché avevo fiducia
nella realtà dell'Illuminazione e nel Sentiero che ad essa conduce. Sono
cose che esistono veramente, esistono proprio come disse il Buddha.
Però, per realizzarle ci vuole la pratica, la retta pratica. Bisogna
spronarsi fino al limite. Ci vuole il coraggio di addestrarsi, di
riflettere e di cambiare radicalmente. È necessario il coraggio di fare
davvero quel che è necessario fare. E come farlo? Addestra il cuore. I
pensieri nella nostra testa ci dicono di andare in una direzione, ma il
Buddha ci dice di prenderne un'altra. Perché è necessario addestrarsi?
Perché il cuore è totalmente incrostato e ricoperto di contaminazioni.
Così è il cuore che non è stato ancora trasformato mediante
l'addestramento. È inaffidabile, e perciò non credergli. Non è ancora
virtuoso. Come possiamo avere fiducia in un cuore che manca di purezza e
di trasparenza? È per questo che il Buddha ci mise in guardia dal
riporre la nostra fiducia in un cuore contaminato. Inizialmente il cuore
è solo il bracciante delle contaminazioni, e se stanno insieme a lungo
il cuore si deforma e diviene tutt'uno con le contaminazioni. Ecco
perché il Buddha ci insegnò a non fidarci del nostro cuore.

Se guardiamo per bene la nostra disciplina per l'addestramento
monastico, vedremo che è tutta incentrata sull'addestramento del cuore.
Tutte le volte che addestriamo il cuore ci sentiamo accaldati,
infastiditi. Appena ci sentiamo accaldati e infastiditi iniziamo a
lamentarci: «~Ragazzi, questa pratica è incredibilmente difficile! È
impossibile.~» Il Buddha, però, non la pensava così. Egli riteneva che
quando l'addestramento causava fastidio e resistenza, questo significava
che si era sulla strada giusta. Noi non la pensiamo in questo modo.
Pensiamo che sia un segno di qualcosa che non va. Quando andiamo
controcorrente rispetto alle contaminazioni e sfidiamo le nostre
bramosie, è naturale che soffriamo. Ci sentiamo accaldati, urtati e
infastiditi, e così smettiamo. Pensiamo di essere sul sentiero
sbagliato. Il Buddha, ovviamente, direbbe che stiamo facendo bene. Ci
stiamo confrontando con le nostre contaminazioni, e sono loro a sentirsi
accaldate e infastidite. Noi invece pensiamo che siamo noi a sentirci
accaldati e infastiditi. Il Buddha insegnò che sono le contaminazioni a
sentirsi concitate e agitate. È lo stesso per tutti.

Ecco perché la pratica del Dhamma è così impegnativa. La gente non
esamina le cose con chiarezza. In genere, si perde il Sentiero sul lato
dell'auto-indulgenza o su quello dell'auto-mortificazione. Resta
bloccata in questi due estremi. Da una parte alle persone piace essere
indulgenti nei riguardi dei desideri del loro cuore. Qualsiasi cosa a
loro piaccia, la fanno e basta. A loro piace sedersi comodi. Amano
sdraiarsi e stendersi comodamente. Qualsiasi cosa facciano, cercano di
farlo tra gli agi. Questo è quel che intendo con auto-indulgenza. Essere
attaccati a sentirsi bene. Con questo genere di indulgenza com'è mai
possibile fare progressi nella pratica del Dhamma? Se non possiamo più
indulgere alla comodità, alla sensorialità e al benessere, ci irritiamo.
Ci agitiamo e arrabbiamo, e ne soffriamo. Questo è cadere fuori dal
Sentiero dal lato dell'auto-mortificazione. Questo non è il Sentiero
dell'uomo saggio e sereno, non è la via di uno che è sereno. Il Buddha
ci ammonì a non smarrirci lungo queste vie collaterali
dell'auto-indulgenza e dell'auto-mortificazione. Quando sperimentate
piacere, sappiatelo con consapevolezza. Quando sperimentate rabbia,
cattiva volontà e irritazione sappiate che non state seguendo le orme
del Buddha. Questi non sono i sentieri di chi cerca la pace, ma le vie
dei comuni abitanti dei villaggi. Un monaco in pace non percorre queste
strade. Egli cammina diritto nel mezzo, con l'auto-indulgenza alla sua
sinistra e l'auto-mortificazione alla sua destra. Questa è la corretta
pratica del Dhamma.

Se stai per intraprendere questo addestramento monastico, devi camminare
nella Via di Mezzo, senza arrovellarti né per la felicità né per
l'infelicità. Deponile. Altrimenti ci si sente come se ci stessero
prendendo a calci. Prima ci danno un calcio da una parte -- «~Ohi!~» --
e poi dall'altra -- «~Ohi!~» Ci sentiamo sbattuti avanti e indietro, da
una parte all'altra, come il batacchio di una campana di legno. La Via
di Mezzo implica lasciar andare la felicità e l'infelicità, e la retta
pratica è la pratica nel mezzo. Quando la bramosia per la felicità ci
colpisce e noi non la soddisfiamo, proviamo dolore. Percorrere la Via di
Mezzo del Buddha è arduo e impegnativo. Ci sono solo questi due estremi
del bene e del male. Se crediamo a quel che essi ci dicono, dobbiamo
seguire i loro ordini. Se ci infuriamo con qualcuno, andiamo subito alla
ricerca di un bastone per colpirlo. Non abbiamo paziente sopportazione.
Se amiamo qualcuno vogliamo accarezzarlo dalla testa ai piedi. Ho
ragione? Queste due vie collaterali evitano del tutto il mezzo. Questo
non è quel che il Buddha raccomandò. Il suo Insegnamento consisteva nel
deporre gradualmente queste cose. La sua pratica era un Sentiero che
conduceva fuori dall'esistenza, lontano dalla rinascita. Un Sentiero
libero dal divenire, dalla nascita, dalla felicità, dall'infelicità, dal
bene e dal male.

Le persone che bramano l'esistenza sono cieche nei riguardi di quel che
c'è nel mezzo. Cadono dal Sentiero dal lato della felicità e poi
oltrepassano il mezzo nel loro cammino verso l'altro lato,
l'insoddisfazione e l'irritazione. Tralasciano sempre il centro. Mentre
si affrettano andando avanti e indietro, per loro il luogo sacro è
invisibile. Non si fermano in quel posto, nel posto in cui non c'è né
esistenza né nascita. A loro non piace, e così non si fermano. Si
trovano comunque fuori casa, e vanno in giro per essere morsi da un
cane, oppure volano in alto per essere beccati da un avvoltoio. Questa è
l'esistenza.

L'umanità è del tutto cieca nei riguardi di quel che è libero
dall'esistenza e privo di rinascita. Il cuore degli uomini è cieco al
riguardo, e perciò l'ignora e lo tralascia. La Via di Mezzo percorsa dal
Buddha, il Sentiero della corretta pratica del Dhamma, trascende
l'esistenza e la rinascita. La mente è libera quando va oltre sia quel
che è salutare sia quel che è malsano. Questo è il Sentiero dell'uomo
saggio e sereno. Se non lo percorriamo, non saremo mai saggi e sereni.
La pace non avrà mai la possibilità di fiorire. Perché? A causa
dell'esistenza e della rinascita. Perché c'è nascita e morte. Il
Sentiero del Buddha è privo di nascita e di morte. Non c'è né basso né
alto. Non c'è né felicità né sofferenza. Non c'è né bene né male. Questo
è il Retto Sentiero. Questo è il Sentiero della pace e della quiete. È
serenamente libero da piacere e dolore, da felicità e tristezza. È così
che si pratica il Dhamma. Sperimentando questo, la mente può fermarsi.
Può smettere di fare domande. Non c'è più alcun bisogno di cercare
risposte. Ecco! Per questo il Buddha disse che il Dhamma è una cosa che
il saggio conosce direttamente, da sé. Non c'è bisogno di chiedere a
nessuno. Comprendiamo con chiarezza da noi stessi senza ombra di dubbio
che le cose sono esattamente come disse il Buddha.

\textbf{Dedizione alla pratica}

Ti ho raccontato qualche breve episodio su come ho praticato. Non avevo
molte conoscenze. Non ho studiato molto. Ho studiato il mio cuore e la
mia mente, e ho imparato in modo naturale mediante esperimenti, prove ed
errori. Quando mi piaceva qualcosa, esaminavo cosa succedeva e dove mi
avrebbe condotto. Inevitabilmente, mi avrebbe trascinato in qualche
successiva sofferenza. La mia pratica consisteva nell'osservarmi. Man
mano che la comprensione e la visione profonda diventavano più intense,
gradualmente giunsi a conoscere me stesso.

Pratica con inflessibile dedizione! Se vuoi praticare il Dhamma, per
favore non pensare troppo. Se stai meditando e ti sorprendi mentre
cerchi di forzare risultati specifici, allora è meglio fermarti. Quando
la tua mente si assesta per diventare serena e tu pensi: «~Ecco! Ecco,
non è vero? È questo?~» Fermati. Prendi tutte le tue conoscenze
analitiche e teoriche, impacchettale e riponile lontano, in una cassa. E
non tirarle fuori per discutere o insegnare. Non è un tipo di conoscenza
che penetra dentro. Sono generi diversi di conoscenza. Quando si vede
realmente qualcosa, non è come le descrizioni scritte. Ad esempio,
scriviamo le parole ``desiderio sensoriale''. Quando il desiderio
sensoriale travolge effettivamente il cuore, è impossibile che le parole
scritte possano esprimere lo stesso significato della realtà. Ugualmente
avviene con ``collera''. Possiamo scrivere le lettere di questa parola
su una lavagna, ma quando siamo davvero arrabbiati l'esperienza non è la
stessa. Il cuore è già inghiottito dalla rabbia ancor prima di riuscire
a leggere quelle lettere.

Questo è un punto di estrema importanza. Gli insegnamenti teorici sono
accurati, ma è essenziale portarli dentro il nostro cuore. Devono essere
interiorizzati. Se il Dhamma non viene portato nel cuore, non lo si
conosce davvero. Non è veramente compreso. Io non ero diverso. Non ho
studiato moltissimo, solo abbastanza per superare alcuni esami di teoria
buddhista. Un giorno ho avuto l'opportunità di ascoltare un discorso di
Dhamma da un maestro di meditazione. Quando ascoltavo, sorsero alcuni
pensieri irriverenti. Non sapevo come ascoltare un vero discorso di
Dhamma. Non riuscivo a capire di cosa stesse parlando quel monaco
errante e maestro di meditazione. Quel che insegnava era come se
provenisse dalla sua esperienza diretta, era come se lui seguisse la
Verità. Con il passare del tempo ebbi qualche esperienza della pratica
in prima persona, e vidi da me la verità di ciò che insegnava quel
monaco. Compresi come comprendere. Sulla scia di tutto questo seguì la
visione profonda. Il Dhamma stava mettendo le radici nel mio cuore e
nella mia mente. Ci volle ancora molto, molto tempo prima di capire che
tutto ciò che quel monaco errante aveva insegnato proveniva da quanto
egli aveva visto da sé. Il Dhamma che insegnava proveniva direttamente
dalla sua stessa esperienza, non da un libro. Parlò secondo la sua
comprensione e la sua visione profonda. Quando fui io stesso a
percorrere il Sentiero, incontrai ogni particolare da lui descritto e
dovetti ammettere che aveva ragione. Perciò continuai.

Cerca di cogliere ogni possibile opportunità per impegnarti nella
pratica del Dhamma. Che ci sia serenità o meno, a questo punto non devi
preoccupartene. La priorità assoluta è di mettere in moto la ruota della
pratica e creare le cause per la futura Liberazione. Se hai svolto il
lavoro, non c'è bisogno di preoccuparti dei risultati. Non essere in
ansia pensando che non stai ottenendo risultati. L'ansia non è serenità.
Ovviamente, se non svolgi il lavoro come puoi attenderti dei risultati?
Come puoi mai sperare di vedere? Chi cerca trova. Chi mangia si sazia.
Tutto quel che è intorno a noi ci mente. Riconoscerlo anche dieci volte
al giorno va già abbastanza bene. Però, il solito vecchio folle continua
a raccontarci le solite vecchie bugie e fandonie. Se sappiamo che sta
mentendo, non va poi così male, ma ci può volere un tempo troppo lungo
prima di accorgercene. Il vecchio arriva, e cerca di raggirarci con
l'inganno in continuazione.

Praticare il Dhamma significa sostenere la virtù, sviluppare il
\emph{samādhi} e coltivare la saggezza nel nostro cuore. Ricorda e
rifletti sulla Triplice Gemma: il Buddha, il Dhamma e il Saṅgha.
Abbandona assolutamente tutto, senza eccezione. Le nostre azioni sono le
cause e le condizioni che matureranno proprio in questa vita. Impegnati
perciò con sincerità. Anche se dobbiamo stare seduti su una sedia per
meditare, è ancora possibile mettere a fuoco la nostra attenzione.
All'inizio non dobbiamo focalizzare molte cose: solo il nostro respiro.
Se lo preferiamo, possiamo ripetere mentalmente a ogni respiro le parole
``Buddha'', ``Dhamma'' o ``Saṅgha''. Mentre focalizzi l'attenzione,
decidi di non controllare il respiro. Se il respiro sembra affaticato o
poco agevole, ciò indica che non ci stiamo relazionando a esso nel
giusto modo. Finché non saremo a nostro agio con il respiro, ci sembrerà
troppo flebile o troppo profondo, troppo sottile o troppo pesante.
Quando ovviamente ci rilassiamo nel respiro, lo troviamo piacevole e
accogliente, e siamo chiaramente consapevoli di ogni inspirazione ed
espirazione. Significa che stiamo imparando come fare. Se non lo stiamo
facendo in modo giusto, perderemo il respiro. Se questo avviene, è
meglio fermarsi un momento e rimettere a fuoco la consapevolezza.

Se durante la meditazione ti viene il desiderio di sperimentare fenomeni
psichici o la mente diventa luminosa e radiosa, oppure hai visioni di
palazzi celestiali e così via, non c'è bisogno di aver paura. Sii
semplicemente consapevole di tutto quello che ti trovi a sperimentare, e
continua a meditare. Dopo un po' di tempo può succedere che il respiro
sembri rallentare fino a fermarsi. La sensazione del respiro pare
svanire e ci si allarma. Non preoccuparti, non c'è nulla da temere. Si
pensa solo che la respirazione si sia fermata. In realtà il respiro è
ancora lì, ma sta funzionando a un livello molto più sottile del solito.
Col tempo il respiro tornerà normale da sé.

Inizialmente concentrati solo per rendere la mente calma e serena. Che
tu stia seduto su una sedia o viaggiando in automobile, oppure facendo
un giro in barca, ovunque ti capiti di trovarti dovresti essere
abbastanza abile nella tua meditazione da riuscire a entrare a tuo
piacimento in uno stato di pace. Quando sali sul treno e ti siedi, porta
in fretta la tua mente in uno stato di pace. Ovunque tu sia, puoi sempre
metterti a sedere. Questo livello di abilità indica che stai
familiarizzando con il Sentiero. Poi investiga. Utilizza il potere di
questa mente serena per investigare quello che sperimenti. A volte si
tratta di quello che vedi. Altre volte di quello che ascolti, odori,
assapori, percepisci con il tuo corpo, oppure pensi o senti nel tuo
cuore. Quale che sia l'esperienza sensoriale che si presenta -- che ti
piaccia o no -- assumila come oggetto di contemplazione. Semplicemente
conosci quel che stai sperimentando. Non proiettare significati o
interpretazioni sulla consapevolezza di questi oggetti dei sensi. Se è
bene, sappi solo che è bene. Se è male, sappi solo che è male. Questa è
realtà convenzionale. Bene o male, tutto è impermanente, insoddisfacente
e non-sé. È tutto inaffidabile. Non vale la pena di aggrapparsi o di
attaccarsi a nulla di tutto ciò.

Se riesci a mantenere questa pratica di pace e investigazione, la
saggezza si genererà automaticamente. Tutto quello che si percepisce con
i sensi e che si sperimenta cade allora nelle tre fosse
dell'impermanenza, dell'insoddisfazione e del non-sé. Questa è la
meditazione di \emph{vipassanā}. La mente è già serena, e tutte le volte
che stati impuri della mente salgono in superficie, gettali in una di
queste tre fosse per l'immondizia. Questa è l'essenza della
\emph{vipassanā}: scartare tutto come impermanente, insoddisfacente e
non-sé. Buono, cattivo oppure orribile: di qualsiasi cosa si tratti,
gettala via. In breve tempo, comprensione e visione profonda, ossia una
tenue visione profonda, fiorirà nel mezzo delle tre caratteristiche
universali.

In questo stadio iniziale la saggezza è ancora debole, ma cerca di
mantenere questa pratica con coerenza. È difficile da esprimere a
parole, ma è come se qualcuno volesse conoscermi. Dovrebbe venire e
vivere qui. Alla fine, entrando quotidianamente in rapporto l'uno con
l'altro potremmo conoscerci.

\textbf{Rispettare la tradizione}

È giunto il momento di cominciare a meditare. Meditare per capire, per
abbandonare, per lasciare andare e per essere in pace. Sono
prevalentemente stato un monaco errante. Ho viaggiato a piedi per
recarmi da maestri e per cercare la solitudine. Non me ne sono andato in
giro a tenere discorsi di Dhamma. Andavo ad ascoltare i discorsi di
Dhamma dei grandi maestri buddhisti di allora. Non sono andato io a
insegnare a loro. Ascoltavo ogni consiglio che avevano da offrirmi.
Ascoltavo pazientemente anche quando erano monaci giovani o ordinati da
poco tempo a cercare di dirmi cosa fosse il Dhamma. Di rado prendevo
parte a discussioni sul Dhamma. Non riuscivo a vedere la ragione per cui
dovessi venire coinvolto in lunghe discussioni. Qualsiasi insegnamento
accettassi, lo adottavo subito, direttamente, quando puntava sulla
rinuncia e sul lasciar andare. Non è necessario diventare esperti nelle
Scritture. Diventiamo più anziani ogni giorno che passa, e ogni giorno
balziamo su un miraggio lasciandoci sfuggire la realtà. Praticare il
Dhamma è del tutto differente dallo studiarlo.

Dell'ampia varietà di stili e tecniche di meditazione, non ne critico
nemmeno uno. Fino a quando comprendiamo il loro vero scopo e
significato, non sono sbagliati. Secondo me non riusciremo mai, se ci
definiamo meditanti buddhisti senza seguire strettamente il codice di
disciplina monastica (Vinaya). Perché? Perché cerchiamo di eludere una
sezione di vitale importanza del Sentiero. Tralasciando la virtù, il
\emph{samādhi} o la saggezza non funzioneranno. Alcuni potrebbero dirti
di non attaccarti alla serenità della meditazione di \emph{samatha}:
«~Non preoccuparti di \emph{samatha}. Punta diritto alla saggezza e alla
pratica di visione profonda di \emph{vipassanā}.~» Per come la vedo io,
se cerchiamo di deviare direttamente verso \emph{vipassanā}, è
impossibile completare il percorso con successo.

Non abbandonare il modo di praticare e le tecniche di meditazione degli
eminenti maestri della Tradizione della Foresta, come i venerabili Ajahn
Sao, Ajahn Mun, Ajahn Tongrat e Ajahn Upāli. Il Sentiero che insegnano,
se lo percorriamo nel modo in cui fecero loro, è del tutto affidabile e
vero. Se seguiamo le loro orme, otterremo la vera visione profonda
dentro noi stessi. Ajahn Sao si prese cura della sua virtù in modo
impeccabile. Non disse che dovremmo eluderla. Questi grandi maestri
della Tradizione della Foresta raccomandarono in modo particolare di
praticare la meditazione e la disciplina monastica e, per il profondo
rispetto che nutriamo nei loro riguardi, dovremmo applicare quel che
insegnarono. Se loro dissero di farlo, allora facciamolo. Se dissero di
fermarsi perché è sbagliato, fermiamoci. Lo facciamo per fiducia. Lo
facciamo con incrollabile sincerità e determinazione. Lo facciamo finché
non vediamo il Dhamma nei nostri cuori, finché non siamo Dhamma. Questo
è ciò che i maestri della Tradizione della Foresta insegnarono. Di
conseguenza i loro discepoli svilupparono profondo rispetto, timore
reverenziale e affetto per loro, poiché fu seguendo il loro Sentiero che
videro quel che i loro maestri videro.

Provaci. Fai proprio come ti dico. Se lo fai veramente, vedrai il
Dhamma, sarai il Dhamma. Se intraprendi davvero la ricerca, che cosa
potrebbe fermarti? Le contaminazioni della mente saranno sopraffatte, se
ti accosti a esse con la strategia giusta. Sii uno che rinuncia, di
poche parole, che s'accontenta di poco e che abbandona tutti i modi di
vedere e le opinioni che provengono dalla presunzione e dalla boria.
Allora sarai in grado di ascoltare pazientemente tutti, anche se stanno
dicendo cose sbagliate. Sarai in grado di ascoltare pazientemente pure
chi ha ragione. Esamina te stesso in questo modo. Te lo assicuro, è
possibile se ci provi. Raramente gli studiosi vengono per mettere in
pratica il Dhamma. Qualcuno c'è, ma sono pochi. È una vergogna. Quello
che hai fatto fino a ora, e che tu sia venuto qui in visita è già degno
di lode. Mostra forza interiore. In alcuni monasteri si incoraggia solo
lo studio. I monaci studiano e studiano, senza sosta, senza che sia
possibile intravedere la fine, e non eliminano mai quel che deve essere
eliminato. Studiano solo la parola ``pace''. Però, se riesci a fermarti,
scoprirai una cosa che ha realmente valore. È così che si fa ricerca.
Questa ricerca davvero preziosa la si fa stando completamente immobili.
Conduce direttamente a quel che hai letto. Se invece gli studiosi non
praticano la meditazione, la loro conoscenza ha poca comprensione.
Quando mettono in pratica gli insegnamenti, allora quelle cose che hanno
studiato diventano vivide e chiare.

Perciò, comincia a praticare! Sviluppa questo tipo di comprensione.
Prova a vivere nella foresta, vieni e resta in una di queste piccole
capanne. Provare per un po' questo addestramento e sperimentarlo da te
stesso ha un valore di gran lunga maggiore della sola lettura dei libri.
Così puoi discutere con te stesso. Mentre osservi la mente è come se
essa lasciasse andare e riposasse nel suo stato naturale. Quando dalla
sua immobilità, dal suo stato naturale essa s'increspa e ondeggia nella
forma di pensieri e concetti, il processo condizionante del
\emph{saṅkhāra} viene messo in moto. Sii molto attento e mantieni
l'occhio vigile su questo processo condizionante. Allorché si muove e
viene snidata da questo stato naturale, la pratica del Dhamma non è più
sulla giusta strada. Scende in quella dell'auto-indulgenza o
dell'auto-mortificazione. Proprio lì. È ciò che fa sorgere questa rete
di condizionamento mentale. Se lo stato mentale è buono, ciò crea un
condizionamento positivo. Se è cattivo, il condizionamento è negativo.
Queste cose si originano nella tua stessa mente.

Ti sto dicendo che è molto divertente osservare da vicino come lavora la
mente. Potrei parlarne allegramente tutto il giorno. Quando riuscirai a
comprendere le vie della mente, vedrai come funziona questo processo e
come esso procede a causa del lavaggio cerebrale al quale siamo
sottoposti dalle impurità della mente. Considero la mente solo come un
posto a sedere. Gli stati psicologici sono ospiti che vengono a visitare
questo posto. A volte lo vanno a trovare dietro invito, altre volte
arrivano per conto loro. Arrivano al centro d'accoglienza. Addestra la
mente a osservarli e conoscerli tutti con uno sguardo di vigile
consapevolezza. È così che ti prendi cura del tuo cuore e della tua
mente. Tutte le volte che un visitatore si avvicina, gli fai cenno di
allontanarsi. Se consenti loro di entrare, dove siederanno? C'è soltanto
una sedia, e ci sei seduto tu. Passa tutto il giorno seduto in questo
posto.

Questa è la ferma e incrollabile consapevolezza del Buddha che sorveglia
e protegge la mente. Tu sei seduto proprio lì. Fin dal momento in cui
sei emerso dal grembo, tutti i visitatori che sono venuti a trovarti
sono arrivati proprio lì. Non importa quanto spesso arrivino, arrivano
sempre in questo stesso posto, proprio lì. Conoscendoli tutti, la
consapevolezza del Buddha siede solitaria, ferma e incrollabile. Questi
visitatori viaggiano fino qui per cercare di esercitare un influsso, per
condizionare e governare la tua mente in vari modi. Quando riescono a
far sì che la mente sia coinvolta nelle loro questioni, sorgono degli
stati psicologici. Quale che sia la questione, ovunque sembri condurre,
dimenticala e basta: non ha importanza. Conosci semplicemente chi sono
gli ospiti, appena arrivano. Quando saranno giunti, troveranno una sola
sedia, e per tutto il tempo che la occuperai non avranno un posto per
sedersi. Arrivano pensando di riempirti gli orecchi di chiacchiere, ma
questa volta per loro non c'è spazio per sedersi. Anche la prossima
volta che verranno non troveranno sedie libere. Non importa quante volte
questi visitatori si presentino, troveranno sempre lo stesso tipo seduto
nello stesso punto. Da quella sedia non ti sei mosso. Per quanto tempo
pensi che continueranno in questo modo? Solo parlando con loro riesci a
conoscerli a fondo. Tutti coloro e tutte le cose che hai conosciuto da
quando hai cominciato a sperimentare il mondo verranno a farti visita.
Osservarli semplicemente, ed essere consapevole, proprio lì, è
abbastanza per vedere il Dhamma appieno. Discuti, osserva e contempla da
te stesso.

È così che si discute il Dhamma. Non so come parlare di nient'altro.
Posso continuare a parlare in questo modo, ma alla fine non c'è
nient'altro che parlare e ascoltare. Ti raccomanderei di andare a
praticare realmente.

\textbf{Padroneggiare la meditazione}

Se guardi da te, è certo che andrai incontro ad alcune esperienze. C'è
un Sentiero a guidarti e a indicarti la direzione. Man mano che
prosegui, la situazione cambia e devi modificare il modo in cui lo
percorri per porre rimedio ai problemi che nascono. Ci può volere molto
tempo prima che tu veda un indicatore chiaro. Se stai per percorrere lo
stesso mio Sentiero, il viaggio deve indubbiamente aver luogo nel tuo
cuore. Altrimenti incontrerai numerosi ostacoli.

È proprio come sentire un suono. Il sentire è una cosa, il suono è
un'altra, e siamo coscienti e consapevoli di entrambi senza complicare
la questione. Facciamo affidamento sulla natura che, nella ricerca della
Verità, ci fornisce il materiale grezzo per l'investigazione. Alla fine
la mente disseziona e separa i fenomeni da sé. Per dirla semplicemente,
la mente non viene coinvolta. Quando gli orecchi captano un suono,
osserva cosa succede nel cuore e nella mente. Restano legati,
intrappolati e sono trasportati via da esso? Si irritano? Sappi almeno
questo. Quando un suono verrà poi archiviato, non disturberà la mente.
Stando qui, prendiamo quelle cose che sono a portata di mano, non quelle
lontane. Anche se ci piacerebbe evitare il suono, non è possibile
fuggire. L'unica possibile via di fuga è addestrare la mente a essere
ferma in presenza del suono. Posa il suono. Possiamo ancora udire i
suoni che lasciamo andare. Sentiamo il suono ma lo lasciamo andare,
perché già lo abbiamo posato. Non è che dobbiamo separare in modo
forzato il sentire dal suono. Si separano automaticamente in ragione
dell'abbandono e del lasciar andare.

Anche se volessimo attaccarci a un suono, la mente non ci si
attaccherebbe. Perché quando abbiamo compreso la vera natura di quel che
vediamo, sentiamo, odoriamo, assaporiamo e di tutto il resto, il cuore
vede con chiara visione profonda che tutto quello che viene percepito
dai sensi rientra senza eccezione nell'ambito delle universali
caratteristiche dell'impermanenza, dell'insoddisfazione e del non‑sé.
Tutte le volte che sentiamo un suono, esso è compreso nei termini di
queste caratteristiche universali. Tutte le volte che si verifica un
contatto sensoriale con l'orecchio, sentiamo, ma è come se non
sentissimo. Questo non significa che la mente non funziona più. La
consapevolezza e la mente si intrecciano e si fondono per controllarsi a
vicenda sempre e senza sosta. Quando la mente è addestrata a questo
livello, non importa quale sia il percorso lungo il quale avremo scelto
di incamminarci per fare ricerca. Coltiveremo l'analisi dei
fenomeni,\footnote{\emph{dhammavicaya}. Investigazione dei \emph{dhamma}
  o stati mentali.} uno dei fattori essenziali dell'Illuminazione, e
questa analisi proseguirà con uno slancio suo proprio.

Discuti il Dhamma con te stesso. Dipana e libera sensazioni, ricordi,
percezioni, pensieri, intenzioni e coscienza. Nulla sarà in grado di
entrare in contatto con essi man mano che continuano a svolgere le loro
funzioni per conto loro. Per chi padroneggia la propria mente, questo
processo di riflessione e di investigazione fluisce automaticamente. Non
è più necessario indirizzarlo intenzionalmente. Quale che sia l'ambito
verso il quale la mente inclini, la contemplazione avviene all'istante
in modo abile.

Se la pratica del Dhamma raggiunge questo livello, si ottiene un altro
interessante e benefico aspetto. Quando si dorme, non ci saranno più
russare, parlare, digrignare i denti, agitarsi e rivoltarsi nel letto.
Anche se il sonno è stato profondo, quando ci sveglieremo non saremo
assonnati. Ci sentiremo pieni di energia e vigili, come se fossimo stati
svegli per tutto il tempo. Di solito russavo ma, quando la mia mente
iniziò a restare sempre sveglia, smisi di farlo. Come si può russare se
si è svegli? È solo il corpo che si ferma e dorme. La mente è del tutto
sveglia giorno e notte, ventiquattr'ore su ventiquattro. Questa è la
pura e intensa consapevolezza del Buddha: Colui che Conosce, il
Risvegliato, il Gioioso, il Radioso. Questa chiara consapevolezza non
dorme mai. La sua energia si sostiene da sé, e non diviene mai smorta o
sonnolenta. A questo livello potete stare senza dormire per due o tre
giorni. Quando il corpo inizia a manifestare segni di stanchezza, ci
sediamo per meditare e immediatamente si entra in profondo
\emph{samādhi} per cinque o dieci minuti. Quando usciamo da questo
stato, ci sentiamo freschi e rinvigoriti come se avessimo dormito per
una notte intera. Se abbiamo superato le preoccupazioni per la salute
del corpo, dormire ha un'importanza minima. Ci prendiamo cura del corpo
adottando le giuste misure, ma non siamo ansiosi per le condizioni
fisiche. Lasciamo che il corpo segua le sue leggi naturali. Non dobbiamo
dire al corpo cosa fare. Lo decide da sé. È come se qualcuno ci
pungolasse, ci spingesse a impegnarci e sforzarci. Anche se ci sentiamo
pigri, dentro c'è una voce che sprona continuamente la nostra diligenza.
Il ristagno è impossibile a questo punto, perché lo sforzo e il
progresso hanno guadagnato uno slancio inarrestabile. Per favore,
verificalo tu stesso. Hai studiato e imparato a lungo. Ora è tempo di
studiare e imparare in relazione a te stesso.

Nelle fasi iniziali di pratica del Dhamma l'isolamento del corpo è di
vitale importanza. Quando vivrai da solo, isolato, ricorderai le parole
del venerabile Sāriputta: «~L'isolamento del corpo è la causa e la
condizione affinché sorgano l'isolamento della mente e degli stati di
\emph{samādhi} profondo liberi dal contatto sensoriale esterno. Questo
isolamento della mente è a sua volta la causa e la condizione per
l'isolamento dalle contaminazioni mentali, per l'Illuminazione.~» Alcuni
tuttavia dicono ancora che la solitudine non è importante: «~Se il tuo
cuore è sereno, non importa dove sei.~» È vero, ma nella fase iniziale
dovremmo ricordare che l'isolamento fisico in un ambiente adatto è di
primaria importanza. Oggi, o comunque presto, cerca un solitario luogo
di cremazione in una foresta, lontano da ogni abitazione. Sperimenta la
vita completamente solo. Oppure cerca la cima di una montagna che ispiri
timore. Va, e vivi da solo, va bene? Ti divertirai un sacco tutta la
notte. Solo allora conoscerai da te. Anch'io una volta pensavo che
l'isolamento fisico non fosse di particolare importanza. È così che
pensavo, ma poi sono partito e l'ho fatto, ed ho potuto riflettere su
quel che il Buddha ha insegnato. Il Beato incoraggiò i suoi discepoli a
praticare in luoghi remoti, molto lontani dalla vita sociale. All'inizio
questo costituisce un fondamento per l'isolamento interiore della mente
che poi supporta l'incrollabile isolamento dalle contaminazioni.

Ad esempio, mettiamo che tu sia un laico, con casa e famiglia. A quale
solitudine puoi aspirare? Quando torni a casa, appena entri sono la
confusione e le complicazioni a colpirti. Non c'è solitudine. E così te
la svigni per un ritiro in un luogo remoto e l'atmosfera è del tutto
differente. È necessario comprendere l'importanza dell'isolamento
fisico e della solitudine nelle fasi iniziali di pratica del Dhamma. Poi
si cerca un maestro di meditazione per ricevere istruzioni. Lui o lei ti
guidano, ti consigliano e ti indicano i punti in cui la tua comprensione
è errata, perché il fraintendimento sta proprio lì dove tu pensi di
essere nel giusto. È proprio dove hai torto che sei sicuro di avere
ragione. Quando l'insegnante spiega, capisci quel che è sbagliato, e
proprio dove l'insegnante ti dice che sei in errore è precisamente lì
che pensavi di essere nel giusto.

Da quel che ho sentito, c'è un buon numero di monaci buddhisti studiosi
che cercano e ricercano basandosi sulle Scritture. Non c'è ragione per
cui non si dovrebbe far ricorso all'esperienza. Quando è il momento di
aprire i nostri libri e di studiare, noi impariamo in questo modo. Però,
quando è il momento d'iniziare a combattere e di dare battaglia, noi
combattiamo in un modo che può non corrispondere alla teoria. Se un
guerriero entra in battaglia e combatte seguendo quel che ha letto, non
riuscirà a tener testa al nemico. Quando il guerriero è autentico e la
battaglia è reale, egli deve combattere in un modo che va al di là della
teoria. È così. Le parole del Buddha nelle Scritture sono soltanto
regole generali ed esempi da seguire, e lo studio può indurre talvolta
alla disattenzione.

La via dei Maestri della Foresta è la via della rinuncia. Su questo
Sentiero c'è solo l'abbandono. Noi sradichiamo i punti di vista che
provengono dalla presunzione del sé. Sradichiamo proprio l'essenza del
nostro senso del sé. Te lo assicuro, questa pratica ti sfiderà al centro
del tuo cuore, però, indipendentemente da quanto difficile possa essere,
non abbandonare i Maestri della Foresta e i loro insegnamenti. Senza una
guida appropriata la mente e il \emph{samādhi} sono potenzialmente
davvero ingannevoli. Cose che non dovrebbero essere possibili iniziano
ad accadere. Mi sono sempre avvicinato a questo genere di fenomeni con
cautela e attenzione. Quando ero solo un giovane monaco, durante i miei
primi anni di pratica non potevo ancora far affidamento sulla mia mente.
Appena acquisii una considerevole esperienza e potei riporre piena
fiducia nel funzionamento della mia mente, ovviamente nulla rappresentò
un problema. Anche se si manifestavano fenomeni insoliti, li lasciavo
perdere. Se sappiamo come funzionano queste cose, cessano da sé. È tutto
combustibile per la saggezza. Con il passare del tempo, ci sentiamo
completamente a nostro agio.

Nella meditazione, possono essere sbagliate anche cose che di solito non
lo sono. Ad esempio, sediamo con determinazione a gambe incrociate e
decidiamo: «~Bene! Questa volta non si tergiversa. Concentrerò la mente.
Stai a vedere.~» Non c'è possibilità che questo sistema funzioni! Tutte
le volte che ci ho provato, la mia meditazione non è andata da nessuna
parte. Noi però amiamo queste bravate. Da quel che ho notato, la
meditazione si sviluppa a un ritmo suo proprio. Molte sere, appena mi
sono seduto a meditare ho detto a me stesso: «~Bene! Questa notte non mi
muoverò da qui almeno fino all'una.~» Anche già solo con questo pensiero
stavo creando del cattivo \emph{kamma}, perché non passava molto tempo
prima che il mio corpo fosse attaccato da ogni parte da dolori che mi
opprimevano fino al punto di farmi sentire come se fossi sul punto di
morire. Ovviamente, le occasioni nelle quali la meditazione andava bene
erano quelle in cui non ponevo alcun limite alle sedute. Non mi
prefissavo l'obiettivo delle 7, delle 8 o delle 9 né qualsiasi altro,
semplicemente continuavo a restare seduto, andando avanti con costanza e
lasciando andare con equanimità. Non forzare la meditazione. Non tentare
d'interpretare quel che sta avvenendo. Non costringere il tuo cuore,
pretendendo non realisticamente che entri in uno stato di
\emph{samādhi}, altrimenti esso sarà ancor più agitato e imprevedibile
del normale. Consenti solo al cuore e alla mente di rilassarsi, di
sentirsi bene e a proprio agio.

Fai con agio fluire il respiro al giusto ritmo, né troppo corto né
troppo lungo. Non cercare di trasformarlo in qualcosa di speciale.
Lascia che il corpo stia bene, sia rilassato e a proprio agio. Continua
a fare così. La tua mente ti chiederà: «~Per quanto tempo mediteremo
questa notte? Quando smetteremo?~» Brontola in continuazione, e così
devi ringhiarle contro un sommesso rimprovero: «~Ascolta. Lasciami in
pace e basta.~» Quest'intrigante ficcanaso deve essere costantemente
sottomesso, perché non si tratta nient'altro che di contaminazioni che
vengono a infastidirti. Non prestarvi alcuna attenzione. Con loro devi
essere inflessibile. «~Se intendo smettere presto o andare avanti fino a
tarda notte, non sono proprio fatti tuoi! Se voglio stare seduto tutta
la notte, non fa differenza per nessuno. E allora perché vieni a
infilare il tuo naso nella mia meditazione?~» È così che devi tagliare
fuori quest'impiccione. Così puoi andare avanti a meditare per tutto il
tempo che vuoi, secondo quel che senti essere giusto.

La mente diventa serena quando le consenti di rilassarsi e di essere a
proprio agio. Allorché sperimenti questa pace, riconosci la forza
dell'attaccamento e ti rendi conto di essa. Quando riuscirai a sedere
senza sosta per un tempo davvero lungo, oltre la mezzanotte, a tuo agio
e rilassato, saprai che hai preso confidenza con la meditazione.
Comprenderai in che modo l'attaccamento e l'aggrapparsi contaminano la
mente. Alcuni accendono davanti a loro un bastoncino d'incenso quando
siedono a meditare e fanno questo voto: «~Non mi alzerò fino a che
questo bastoncino d'incenso non si consumerà.~» Poi si siedono. Dopo che
a loro sembra essere trascorsa un'ora, aprono gli occhi e capiscono che
sono passati solo cinque minuti. Delusi, perché il bastoncino è ancora
troppo lungo, fissano l'incenso. Chiudono di nuovo gli occhi e
continuano. Presto però li riaprono di nuovo per controllare quel
bastoncino d'incenso. Questa gente non va da nessuna parte con la
meditazione. Non fare così. Stando solo seduti a immaginare quel
bastoncino d'incenso -- «~Mi chiedo se sia quasi finito~» -- la
meditazione non porta da nessuna parte. Non dare importanza a queste
cose. La mente non deve fare nulla di speciale.

Se stai per intraprendere il compito di sviluppare la mente nella
meditazione, non lasciare che la contaminazione della brama conosca le
regole fondamentali o gli obiettivi. Ti chiede: «~Come mediterai,
venerabile?~» «~Per quanto tempo? Quanto tardi pensi che si farà?~» La
brama continua ad assillarci fino a che non si giunge a un accordo.
Quando dichiariamo che staremo seduti fino a mezzanotte, inizia
immediatamente ad assillarci. Ancor prima che sia trascorsa un'ora, ci
sentiamo così irrequieti e impazienti da non riuscire a continuare. Un
numero ancor maggiore di impedimenti ci attaccano se ci rimproveriamo:
«~Sono senza speranza! Forse che stare seduto mi ucciderà? Anche se ho
detto che ero in procinto di renderla incrollabile nel \emph{samādhi},
la mente è ancora inaffidabile e vaga ovunque. Ho fatto un voto, ma non
l'ho mantenuto.~» Quando pensieri di auto-svalutazione e scoraggiamento
assalgono la mente, sprofondiamo nell'odio verso noi stessi. Tutto
questo rende le cose più difficili. Non c'è nessuno a cui dare la colpa
o con cui arrabbiarsi. Quando facciamo un voto, però, dobbiamo
rispettarlo, lo si rispetta o si muore per mantenerlo. Se facciamo voto
di stare seduti per un certo periodo di tempo, non dovremmo infrangere
il voto e fermarci. Nel frattempo, ovviamente, pratichiamo e sviluppiamo
la pratica in modo graduale. Non c'è alcun bisogno di fare voti
spettacolari. Cerca di addestrare la mente con costanza e perseveranza.
Saltuariamente la meditazione sarà serena e tutti i dolori e disagi del
corpo svaniranno. Il male alle caviglie e alle ginocchia cesserà da sé.

Se quando proviamo a coltivare la meditazione iniziano a sorgere strane
immagini, visioni o percezioni sensoriali, la prima cosa da fare è
controllare il nostro stato mentale. Non trascurare questo principio
basilare. Perché sorga questo genere di immagini, la mente deve essere
relativamente tranquilla. Non desiderare che appaiano e non desiderare
che non appaiano. Se sorgono, esaminale, ma non permettere che ti
ingannino. Ricorda solamente che non sono nostre. Sono impermanenti,
insoddisfacenti e non-sé, proprio come qualsiasi altra cosa. Anche se
sono reali, non dimorare in esse e non prestarci molta attenzione. Se
rifiutano con ostinazione di svanire, metti di nuovo a fuoco la
consapevolezza sul respiro con più energia. Prendi almeno tre respiri
lunghi e profondi, e ogni volta espira tutta l'aria lentamente. Questo
può essere un trucco. Continua a rimettere a fuoco l'attenzione.

Non essere possessivo nei confronti di questo genere di fenomeni. Non
sono nulla di più di quel che sono, e quel che sono è potenzialmente
ingannevole. Sia che ci piacciano e ce ne innamoriamo sia che avvelenino
la mente con la paura, non sono affidabili. Possono non essere veri o
non essere ciò che sembrano. Se ne fai esperienza, non cercare di
interpretarne il senso né di proiettare significati su di essi. Ricorda
che non sono nostri, perciò non correre dietro queste visioni o
sensazioni. Torna invece immediatamente indietro e controlla lo stato
mentale di quel momento. È una regola generale. Se abbandoniamo questo
principio basilare e restiamo irretiti da quel che crediamo di vedere,
possiamo dimenticarci di noi stessi e cominciare a parlare a vanvera o
perfino diventare matti. Possiamo uscir di senno fino al punto di non
riuscire più a relazionarci con le persone in modo normale. Riponi la
tua fiducia nel tuo stesso cuore. Qualsiasi cosa avvenga, vai
semplicemente avanti osservando il cuore e la mente. Strane esperienze
meditative possono essere benefiche per chi ha saggezza, ma dannose per
chi non ne ha. Qualsiasi cosa succeda non esaltarti e non allarmarti. Se
hai un'esperienza, ciò avviene e basta.

Un altro modo per avvicinarsi alla pratica del Dhamma consiste nel
contemplare ed esaminare tutto quel che vediamo, facciamo e
sperimentiamo. Non abbandonare mai la meditazione. Alcuni, quando
terminano la meditazione seduta o quella camminata, pensano che sia il
momento di fermarsi e di riposare. Smettono di mettere a fuoco la mente
sull'oggetto della meditazione o sul tema della contemplazione. Li
lasciano cadere del tutto. Non praticare in questo modo. Qualsiasi cosa
tu veda, indagala per quello che veramente è. Contempla le brave persone
nel mondo. Contempla anche quelle cattive. Osserva in modo penetrante
l'uomo ricco e potente, e quello indigente colpito dalla povertà. Quando
vedi un bambino, una persona anziana, un uomo o una donna giovane,
investiga il significato dell'età. Tutto è combustibile per l'indagine.
È così che si coltiva la mente.

La contemplazione che conduce al Dhamma è la contemplazione della
causalità, il processo di causa ed effetto in tutte le sue varie
manifestazioni: maggiore e minore, bianco e nero, bene e male. In breve,
tutto. Quando pensi, riconosci che si tratta di pensieri e contempla che
sono solo questo, nulla di più. Tutte queste cose vanno a finire nel
cimitero dell'impermanenza, dell'insoddisfazione e del non-sé, e per
questa ragione non ti aggrappare in modo possessivo a nessuna di esse.
Questo è il campo crematorio di tutti i fenomeni. Seppelliscili e
cremali per sperimentare la Verità.

Avere visione profonda nell'impermanenza significa non consentire a se
stessi di soffrire. Si tratta d'investigare con saggezza. Ad esempio,
otteniamo una cosa che consideriamo buona e piacevole, e perciò siamo
felici. Osserva da vicino e intensamente questa positività e questo
piacere. A volte, dopo aver posseduto una cosa a lungo ce ne stufiamo.
Vogliamo darla via o venderla. Se non c'è nessuno che vuole comprarla,
siamo pronti a gettarla via. Perché? Quali sono le ragioni soggiacenti a
questa dinamica? Tutto è impermanente, incostante e mutevole, ecco
perché. Se non possiamo vendere quella cosa e nemmeno gettarla via,
iniziamo a soffrire. L'intera questione è proprio così, e quando un
episodio è del tutto compreso, non importa quante situazioni simili
sorgano, sono tutte comprese alla stessa maniera. Le cose stanno
semplicemente così. Come dice il proverbio: «~Se ne hai visto uno, li
hai visti tutti.~»

A volte vediamo cose che non ci piacciono. Altre volte sentiamo dei
rumori spiacevoli o fastidiosi, e ci irritiamo. Esaminali e ricordatelo,
perché in futuro quei rumori potrebbero piacerti. Potremmo davvero
provare piacere proprio per quelle cose che in precedenza abbiamo
detestato. È possibile. Allora ci sovviene con chiarezza e visione
profonda: «~Ah! Tutte le cose sono impermanenti, incapaci di soddisfarci
pienamente e non-sé.~» Gettale nella fossa comune delle Tre
Caratteristiche universali. L'attaccamento alle cose piacevoli che
otteniamo e che abbiamo, e a quello che siamo, cesserà. Tutto ciò che
sperimentiamo genera visione profonda nel Dhamma.

Ho detto tutto questo soltanto perché tu lo possa ascoltare e pensarci
su. Sono solo parole, questo è tutto. Quando la gente viene a trovarmi,
parlo. Questo genere di argomenti non serve per sedersi in circolo e
chiacchierare per ore. Fallo e basta. Entraci dentro e fallo. È come
quando chiamiamo un amico per andare da qualche parte. Lo invitiamo.
Otteniamo una risposta. Poi si va, senza troppe storie. Diciamo solo
quel che serve e lasciamo le cose come stanno. Posso dirti una cosa o
due sulla meditazione, perché ho fatto il lavoro che era da fare. Però,
sai, posso sbagliare. Il tuo lavoro consiste nell'investigare e nello
scoprire da te se quel che dico è vero.

