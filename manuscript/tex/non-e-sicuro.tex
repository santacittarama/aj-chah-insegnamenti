\chapter{Non è sicuro}

\begin{openingQuote}
  \centering

  Discorso informale offerto presso la kuṭī di Ajahn Chah ad alcuni
  monaci e novizi una sera dell'anno 1980.
\end{openingQuote}

Qualche tempo fa un monaco occidentale, un mio allievo, tutte le volte
che vedeva monaci e novizi thailandesi che lasciavano l'abito diceva:
«~Che peccato! Perché lo fanno? Perché sono così tanti i monaci e i
novizi thailandesi che lasciano l'abito?~» Era turbato. Gli dispiaceva
quando i monaci e i novizi thailandesi lasciavano l'abito perché era
appena giunto in contatto con il buddhismo. Si sentiva ispirato, deciso.
Per lui farsi monaco era l'unica cosa giusta da fare, pensava che non si
sarebbe mai smonacato. Chiunque lasciava l'abito era un folle. Aveva
visto che i thailandesi all'inizio del Ritiro delle Piogge prendevano
l'abito monastico, e alla fine del Ritiro si smonacavano. Diceva: «~Che
tristezza! Mi dispiace così tanto per quei monaci e per quei novizi
thailandesi. Come possono fare una cosa del genere?~» Col passare del
tempo iniziarono a lasciare l'abito anche degli occidentali ed egli
iniziò a pensare che dopo tutto non si trattava di una cosa così
rilevante. All'inizio, quando aveva appena cominciato a praticare, era
entusiasta. Pensava che diventare monaco fosse davvero importante.
Pensava che fosse una cosa facile.

Quando la gente si sente ispirata, tutto sembra essere buono e giusto.
Non avendo termini di confronto, vanno avanti e decidono per conto loro.
Ma non sanno proprio cosa significhi praticare. Chi invece sa, ha un
solido e stabile fondamento all'interno del proprio cuore, ma non per
questo sente il bisogno di sbandierarlo ai quattro venti. Per quanto mi
riguarda, appena ricevetti l'ordinazione monastica non praticavo molto,
ma avevo molta fede. Non so perché, forse era lì fin dalla mia nascita.
I monaci e i novizi che erano diventati monaci insieme a me si
smonacarono tutti alla fine delle piogge. Tra me e me pensai: «~Eh? Che
succede a questa gente?~» Ovviamente a loro non osai dire nulla, perché
non ero ancora sicuro di me stesso, ero troppo turbato. Dentro di me,
però, sentivo che erano tutti folli. «~È difficile diventare monaci,
smonacarsi è facile. Questa gente non ha molti meriti, pensa che la via
del mondo sia migliore della via del Dhamma.~» È così che pensavo, ma
non dissi nulla, osservavo la mia mente e basta.

Ho visto coloro che erano diventati monaci con me lasciare l'abito, uno
dopo l'altro. A volte si rivestivano di tutto punto e tornavano in
monastero per mettersi in mostra. Li guardavo e pensavo che erano matti,
ma loro credevano di essere eleganti. Quando si lascia l'abito, si sente
di dover fare questo e quell'altro ancora. Tra me e me pensavo che era
un modo di pensare sbagliato. Tuttavia non lo dicevo, perché io stesso
non ero ancora completamente sicuro. Non ero sicuro neanche di quanto
potesse durare la mia fede. Quando i miei amici avevano tutti lasciato
l'abito monastico smisi di preoccuparmi, perché non c'era più nessuno di
cui dovermi preoccupare. Presi il \emph{Pāṭimokkha}\footnote{\emph{Pāṭimokkha}:
  Il codice fondamentale della disciplina monastica.} e cominciai a
studiarlo con determinazione. Non c'era più nessuno a distrarmi e a
farmi perdere tempo, e così mi dedicai con tutto il cuore alla pratica.
Tuttavia non dicevo ancora nulla, perché mi pareva una cosa alquanto
difficile che per tutta una vita -- settant'anni forse, ottanta o
perfino novanta -- si riuscisse a praticare sforzandosi continuamente,
senza impigrirsi.

Chi aveva lasciato la propria casa per diventare monaco era diventato
monaco, chi si era smonacato si era smonacato. Osservavo tutto questo e
basta. Che se ne fossero andati o che fossero rimasti la cosa non mi
riguardava. Osservavo i miei amici andarsene, ma la sensazione che avevo
era che quella gente non vedesse con chiarezza. Forse la pensava così
anche quel monaco occidentale. Vedeva che le persone diventavano monaci
solo per un Ritiro delle Piogge, e restava sconcertato. In seguito
raggiunse quello stadio che noi chiamiamo della noia, noia per la Vita
Santa. Lasciò andare la pratica e infine si smonacò. «~Perché lasci
l'abito? Prima, quando vedevi i monaci thailandesi che lasciavano
l'abito dicevi ``Che peccato! Che tristezza, che pena''. Ora che sei tu
a voler lasciare l'abito, perché non provi dispiacere?~» Non rispose. Si
limitò a sorridere imbarazzato.

Quando si tratta di addestrare la mente, non è facile raggiungere un
buon livello, se dentro di noi non abbiamo ancora sviluppato un
``testimone''. Per la maggior parte delle questioni esteriori si può far
affidamento sugli altri, ci sono regole e precedenti. Quando però si
tratta di usare il Dhamma come punto di riferimento, il Dhamma lo
abbiamo già? Stiamo pensando rettamente o no? E ammesso che si tratti di
un pensiero giusto, sappiamo come lasciar andare questa giustezza oppure
ci attacchiamo ancora ad essa? Dovete contemplare fino a quando
raggiungete il punto in cui riuscite a lasciar andare, questa è la cosa
importante, fino a quando raggiungete il punto in cui non resta nulla,
dove non c'è né bene né male. Li gettate via. Questo significa che
gettate via tutto. È tutto andato, non rimane nulla. Se rimane ancora
qualcosa, significa che non è andato via tutto.

Perciò, per quanto concerne l'addestramento della mente, a volte
possiamo dire che è facile. È facile dirlo, ma è difficile farlo, molto
difficile. È difficile perché l'addestramento non è in linea con i
nostri desideri. A volte pare quasi che siano gli angeli ad aiutarci.
Tutto va bene, tutto quel che pensiamo o diciamo sembra essere giusto.
Poi succede che ci attacchiamo a quella giustezza e dopo un po'
prendiamo una strada sbagliata e tutto va male. Ecco dov'è difficile.
Non abbiamo un punto di riferimento per valutare le cose.

Le persone che hanno molta fede, che credono e che hanno fiducia ma
mancano di saggezza possono riuscire molto bene nel \emph{samādhi}, ma
può succedere che non abbiano molta visione profonda. Vedono solo un
lato di ogni cosa, e si limitano a seguire quello. Non riflettono. Si
tratta di una fede cieca. Nel buddhismo la chiamiamo
\emph{saddhā-adhimokkha}, fede cieca. Hanno fede, ma questa fede non è
nata dalla saggezza. Però non lo capiscono. Credono di essere saggi, e
così non capiscono dove sbagliano. È per questo che negli insegnamenti
si parla delle cinque facoltà (\emph{balā}):
\emph{saddhā}, \emph{viriya}, \emph{sati}, \emph{samādhi}, \emph{paññā}.
\emph{Saddhā} è salda fiducia, \emph{viriya} è sforzo diligente,
\emph{sati} è rammemorazione, \emph{samādhi} è stabilità della mente,
\emph{paññā} è conoscenza onnicomprensiva. Non si dica che è semplice
conoscenza, \emph{paññā} è conoscenza onnicomprensiva, perfetta.

Il Saggio ci ha dato queste cinque cose affinché le utilizzassimo
inizialmente come oggetti di studio, poi come punti di riferimento per
valutare le condizioni della nostra pratica. Ad esempio \emph{saddhā},
la salda fiducia. L'abbiamo già sviluppata? \emph{Viriya}, applichiamo
uno sforzo diligente o no? Il nostro è un Retto Sforzo oppure si tratta
di uno sforzo errato? È questo che dobbiamo prendere in considerazione.
Tutti impiegano un certo qual sforzo, ma il nostro sforzo contiene
saggezza o no? Per \emph{sati} è la stessa cosa. Perfino un gatto ha
\emph{sati}. Quando vede un topo, \emph{sati} è presente. Gli occhi del
gatto sono fissi sul topo. Questa è la \emph{sati} del gatto. Tutti
hanno \emph{sati}: gli animali, i delinquenti, i saggi. \emph{Samādhi},
la stabilità della mente, tutti hanno anche questa facoltà. Il gatto ce
l'ha quando la sua mente è fissa per afferrare il topo e mangiarselo. Ha
un suo stabile intento. Quella \emph{sati} del gatto è una specie di
\emph{sati.} S\emph{amādhi}, un fisso intento per quel che sta facendo,
è pure presente. \emph{Paññā}, la conoscenza, è però propria degli
esseri umani. Il gatto conosce come può conoscere un animale, ha una
conoscenza sufficiente per catturare i topi, il suo cibo.

Queste cinque cose sono chiamate qualità. Queste cinque qualità sono
sorte da \emph{sammā-diṭṭhi}, o no? \emph{Saddhā}, \emph{viriya},
\emph{sati}, \emph{samādhi}, \emph{paññā} sono sorte dalla Retta
visione? Che cos'è la Retta Visione? Qual è il nostro criterio per
valutare la Retta Visione? Dobbiamo capirlo con chiarezza. Retta Visione
significa comprendere che tutte queste cose sono incerte. È per questo
che il Buddha e gli Esseri Nobili non si attaccavano a esse. Le
tenevano, ma non saldamente. Non permettevano a quel tenere di diventare
un'identità. Il modo di tenere che non conduce al divenire non è
macchiato dal desiderio. Senza desiderare di diventare questo o quello,
c'è solo la pratica stessa. Quando vi attaccate a una cosa in
particolare, c'è piacere o dispiacere? Se c'è piacere, vi attaccate al
piacere? Se c'è dispiacere, vi attaccate a quel dispiacere?

Alcuni modi di vedere possono essere utilizzati come principi per
valutare più accuratamente la nostra pratica. Ad esempio, sapere che
quei modi di conoscere che ci dicono che una persona è meglio degli
altri, o uguale agli altri, oppure più sciocca degli altri, sono tutti
riconducibili all'errata visione. Possiamo percepirle queste cose, ma
dobbiamo conoscerle con saggezza, sapere che semplicemente sorgono e
cessano. Pensare che siamo meglio degli altri non è giusto. Pensare che
siamo uguali agli altri non è giusto. Pensare che siamo inferiori agli
altri non è giusto.

La Retta Visione taglia le cose, attraversandole tutte. In quale
direzione andiamo, allora? Se pensiamo di essere meglio degli altri
sorge l'orgoglio. È lì, ma non lo vediamo. Se pensiamo di essere uguali
agli altri, quando è il momento manchiamo di rispetto e di umiltà. Se
pensiamo di essere inferiori agli altri ci deprimiamo, riteniamo di
essere nati sotto una cattiva stella, e così via. Ci stiamo ancora
aggrappando ai cinque \emph{khandhā}, è tutto solo divenire e nascita.
Questo è un criterio per valutare noi stessi.

Eccone un altro. Se ci capita una cosa piacevole siamo felici, se ci
capita una brutta cosa siamo infelici. Siamo in grado di osservare sia
le cose che ci piacciono sia quelle che non ci piacciono come se
avessero lo stesso valore? Valutate voi stessi con questo criterio.
Nella nostra vita quotidiana, nelle varie esperienze che ci capita di
avere, se ascoltiamo una cosa che ci piace, il nostro umore cambia? Se
ci capita una cosa che non ci piace, il nostro umore cambia? Oppure la
mente rimane stabile? Proprio guardando queste cose abbiamo il nostro
criterio di valutazione. Conoscere se stessi, tutto qui, questo è il
vostro testimone. Non prendete decisioni basandovi sulla forza dei
vostri desideri. I desideri possono indurci a pensare di essere ciò che
in realtà non siamo. Dobbiamo essere molto circospetti. Ci sono
moltissime angolazioni e aspetti da prendere in considerazione, ma Retta
Visione è seguire la Verità, non i desideri. Dovremmo conoscere sia il
bene sia il male e, dopo che li abbiamo conosciuti, lasciarli andare. Se
non li lasciamo andare siamo ancora qui, ``esistiamo'' ancora,
``abbiamo'' ancora. Se ancora ``siamo'' allora è rimasto qualcosa, ci
sono ancora in serbo divenire e nascita. Per questa ragione il Buddha
disse di giudicare solo se stessi. Non giudicate gli altri, non importa
quanto buoni o malvagi possano essere. Il Buddha si limita a indicare la
via, dicendo: «~La Verità è fatta così.~» Per quanto ci riguarda, la
nostra mente è così, o no?

Supponiamo ad esempio che un monaco prenda delle cose che appartengono a
un altro monaco, il quale poi lo accusa: «~Hai rubato le mie cose.~»
«~Non le ho rubate, le ho solamente prese.~» Così si chiede a un terzo
monaco di giudicare la questione. Come potrebbe riuscirci? Dovrebbe
chiedere al monaco in torto di comparire di fronte al Saṅgha. «~Sì, le
ho prese, ma non le ho rubate.~» Oppure, in relazione ad altre regole, a
mancanze relative a \emph{pārājika} o \emph{sanghādisesa}:\footnote{\emph{Pārājika}
  o \emph{sanghādisesa}: Trasgressioni gravi che determinano
  l'esclusione dall'ordine oppure una procedura comunitaria per essere
  reintegrati nel Saṅgha.} «~Sì l'ho fatto, ma non ne avevo
l'intenzione.~» È possibile crederci? Si tratta di una cosa complicata.
Se non ci credete, tutto quello che potete fare è lasciare il peso di
quanto è accaduto sulle spalle del responsabile. Dovreste però sapere
che non possiamo nasconderci le cose che sorgono nella nostra mente. Non
potete insabbiare né le cattive azioni né quelle buone. Che le azioni
siano buone o cattive, non potete eliminarle semplicemente ignorandole,
perché queste cose tendono a rivelarsi da sole. Manifestano se stesse,
rivelano se stesse, esistono in se stesse e di per se stesse. Sono tutte
automatiche. Ecco come funzionano le cose.

Non cercate di indovinare o di speculare. Finché c'è ancora
\emph{avijjā}\footnote{\emph{Avijjā}: Non conoscenza, ignoranza;
  consapevolezza offuscata.} non la farete mai finita. Una volta il
primo ministro del nostro governo mi chiese: «~Luang Por, la mente di un
\emph{anāgāmī}\footnote{\emph{Anāgāmī}: ``Chi è senza ritorno'' dopo la
  morte apparirà in uno dei mondi di Brahmā, per poi entrare nel
  \emph{Nibbāna}, senza mai tornare in questo mondo.} è pura?~» «~È pura
solo in parte.~» «~Eh? Un \emph{anāgāmī} ha rinunciato al desiderio
sensoriale, com'è che la sua mente non è pura?~» «~Può anche aver
lasciato andare il desiderio sensoriale, ma resta ancora qualcosa, o no?
C'è ancora \emph{avijjā}. Se lì è rimasto ancora qualcosa, allora lì c'è
ancora qualcosa. È come per le ciotole per la questua dei
\emph{bhikkhu}. Ce n'è di diverse misure: grandissime, grandi medie e
grandi piccole; medie grandi, medie medie, medie piccole; infine,
piccole grandi, piccole medie e piccole piccole. Non conta quanto
piccola possa essere, lì c'è ancora comunque una ciotola, vero? Ecco
come stanno le cose con \emph{sotāpanna},\footnote{\emph{Sotāpanna}:
  ``Chi è entrato nella corrente'' e ha così conseguito il primo livello
  dell'Illuminazione.} \emph{sakadāgāmī}\footnote{\emph{Sakadāgāmī}: Il
  secondo stadio dell'Illuminazione, ``Chi torna una sola volta'' a
  esistere in forma umana prima di conseguire l'Illuminazione.} e
\emph{anāgāmī}. Hanno tutti rinunciato ad alcune contaminazioni, ma solo
in relazione ai loro rispettivi livelli. Qualsiasi cosa rimanga, quei
Nobili Esseri non possono vederla. Se potessero vederla sarebbero tutti
\emph{arahant}.\footnote{\emph{Arahant}: Letteralmente, un
  ``Meritevole''; una persona la cui mente è libera dalle contaminazioni
  (\emph{kilesa}). È anche un titolo del Buddha e il livello più alto
  dei suoi Nobili Discepoli.} Non riescono ancora a vedere tutto. Quel
che non vedono è \emph{avijjā}. Se la mente di un \emph{anāgāmī} fosse
del tutto raddrizzata, non sarebbe un \emph{anāgāmī}, ma un essere
completamente realizzato. È che è rimasto ancora qualcosa.~» «~La sua
mente è pura?~» «~Lo è solo un po', ma non al cento per cento.~» In
quale altro modo avrei potuto rispondere? Disse che sarebbe tornato per
farmi altre domande. Poteva comunque contemplare quel che gli avevo
detto, il parametro è quello.

Non siate distratti. Siate attenti. Il Buddha ci esortò a essere
attenti. Per quanto concerne l'addestramento del cuore, lo sapete,
anch'io ho avuto le mie tentazioni. Ho avuto spesso la tentazione di
provare molte cose, ma mi è sempre sembrato che mi portassero fuori dal
Sentiero. In realtà si tratta solo di una specie di spavalderia della
mente, una specie di presunzione. \emph{Diṭṭhi} (opinione) e \emph{māna}
(orgoglio) sono presenti. È già abbastanza difficile essere consapevoli
di queste due cose.

Qualche tempo fa qui c'era uno che voleva farsi monaco. Si era procurato
l'abito monastico, era deciso a farsi monaco in memoria della madre
defunta. Venne in monastero, poggiò in terra il suo abito e, senza
neanche porgere omaggio ai monaci, iniziò a fare la meditazione
camminata proprio di fronte alla sala principale, avanti e indietro,
avanti e indietro, come se volesse davvero mettere in mostra quel che
sapeva fare. Pensai: «~Oh, c'è anche gente fatta così!~» Questa si
chiama fede cieca, \emph{saddhā adhimokkha}. Probabilmente aveva deciso
di conseguire l'Illuminazione prima del tramonto, o qualcosa del genere,
pensò che sarebbe stato facile. Non guardò nessun altro, si mise a testa
bassa e camminò come se la sua vita dipendesse da questo. Lo lasciai
continuare, ma pensai: «~Credi che sia così facile?~» Non so quanto
tempo restò, credo che non abbia neanche ricevuto l'ordinazione
monastica.

Appena la mente pensa qualcosa, le diamo sfogo, le diamo sfogo in
continuazione. Non comprendiamo che si tratta solo dell'abituale
proliferazione mentale. Essa si traveste di saggezza e si mette a
cianciare in modo minuzioso e dettagliato. Questa proliferazione mentale
sembra molto intelligente. Se non lo sappiamo, la scambiamo per
saggezza. Quando però si arriva alla resa dei conti, vediamo che non è
così. E quando sorge la sofferenza proprio laddove si trova quella
cosiddetta saggezza, che si fa? Ha una qualche utilità? Dopo tutto, è
solo proliferazione mentale. Rimanete perciò con il Buddha. Come vi ho
già detto molte volte, nella nostra pratica dobbiamo volgerci verso
l'interiorità e trovare il Buddha. Dov'è il Buddha? Il Buddha è ancora
vivo proprio adesso, andate dentro e trovatelo. Dov'è? Con
\emph{aniccā}, andate dentro e inchinatevi a Lui, \emph{aniccā},
l'incertezza. Per cominciare, è proprio lì che potete fermarvi. Se la
mente prova a dirvi: «~Sono un \emph{sotāpanna}~», andate a prostrarvi a
un \emph{sotāpanna}. Lui stesso vi dirà: «~È tutto incerto.~» Se
incontrate un \emph{sakadāgāmī}, prestategli omaggio. Quando vi vedrà,
vi dirà semplicemente: «~Non è cosa certa!~» Se c'è un \emph{anāgāmī},
andate a prostrarvi a lui. Vi dirà solo una cosa: «~Incerto.~» Anche se
incontrate un \emph{arahant}, andate a prostrarvi a lui ed egli vi darà
con maggior fermezza: «~Tutto è ancor più incerto!~» Sentirete così le
parole degli Esseri Nobili: «~Tutto è incerto, non attaccarti a nulla.~»

Non guardate al Buddha come farebbero dei sempliciotti. Non attaccatevi
alle cose, aggrappandovi a esse senza lasciarle andare. Guardate alle
cose come a funzioni dell'apparenza e poi collocatele sul piano della
trascendenza. È così che dovete essere. Ci deve essere l'apparenza e ci
deve essere la trascendenza. Perciò dico: «~Vai dal Buddha.~» Dov'è il
Buddha? Il Buddha è il Dhamma. Tutti gli insegnamenti di questo mondo
possono essere racchiusi in un solo insegnamento: \emph{aniccā}.
Pensateci. Come monaco ho cercato per più di quarant'anni, e questo è
tutto quello che sono riuscito a trovare. Questo, e la paziente
sopportazione. Così bisogna avvicinarsi all'insegnamento del Buddha su
\emph{aniccā}: è tutto incerto. Non importa quanto piaccia alla mente
avere delle certezze, ditele solo: «~Non è sicuro!~» Ogni volta che la
mente vuole aggrapparsi a qualcosa come si trattasse di una cosa sicura,
ditele solo: «~Non è sicuro, è transitorio.~» Impacchettatela in questo
modo. Usando il Dhamma del Buddha tutto si riduce a questo. E non si
tratta di un fenomeno solo momentaneo. Sia stando in piedi sia
camminando sia stando seduti o distesi vedete tutto in questo modo. Ogni
volta che sorge piacere o avversione, vedete tutto nella stessa maniera.
Questo significa essere vicini al Buddha, vicini al Dhamma. Ora sento
che questo è il modo più valido di praticare. Tutta la mia pratica
dall'inizio a oggi è stata così. In verità non ho fatto affidamento
sulle Scritture, e nemmeno le ho trascurate. Non ho fatto affidamento su
un maestro, e nemmeno ho ``fatto tutto da me''. La mia pratica è stata
tutta ``né questo né quello''.

Sinceramente, si tratta di ``smaltire tutto''. Ossia di praticare fino
alla fine, facendo propria la pratica e portandola a compimento, vedendo
contemporaneamente apparenza e trascendenza. Di questo ho già parlato,
ma ad alcuni di voi può forse interessare sentirlo di nuovo: se si
pratica con costanza e si considerano le cose accuratamente, alla fine
raggiungerete questo punto. All'inizio vi affrettate ad andare avanti,
vi affrettate a tornare indietro e vi affrettate a fermarvi. Continuate
a praticare in questa maniera finché arrivate al punto in cui non si
tratta di andare avanti, non si tratta di andare indietro, e nemmeno di
fermarsi! È finita. Questa è la fine. Non aspettatevi che questo, la
pratica finisce proprio qui. \emph{Khīṇāsavo}: colui che ha raggiunto la
completezza. Non va avanti, non retrocede e non si ferma. Non c'è
fermarsi, non c'è andare avanti né andare indietro. È la fine.
Pensateci, comprendetelo con chiarezza nella vostra mente. Troverete che
proprio lì non c'è assolutamente nulla.

Che si tratti di una cosa vecchia o nuova dipende da voi, dalla vostra
saggezza e dal vostro discernimento. Chi non ha né saggezza né
discernimento non sarà in grado neanche di immaginarlo. Date anche solo
un'occhiata agli alberi, all'albero di mango o all'albero del pane. Se
crescono ammassati, un albero può diventare più alto degli altri, così
che questi si piegano verso l'esterno per crescere. Perché succede? Chi
dice loro di fare così? È la natura. La natura contiene sia il bene sia
il male, sia il giusto sia lo sbagliato. Può inclinare sia verso quello
che è giusto sia verso quello che è sbagliato. Quale che sia il genere
di alberi che piantiamo vicini, quelli che cresceranno per ultimi si
allontaneranno dal più grande. Com'è che avviene? Chi è che fa in modo
che sia così? È la natura, o Dhamma.

Allo stesso modo \emph{tanhā}\footnote{\emph{Taṇhā}: Letteralmente
  ``sete''. Bramosia per gli oggetti dei sensi, per l'esistenza o per la
  non esistenza.} ci porta verso la sofferenza. Se la contempliamo,
questo ci condurrà fuori dal desiderio, diventeremo più grandi di
\emph{tanhā}. Investigando \emph{tanhā} la scuoteremo, la renderemo
sempre più leggera, fino a quando se ne andrà completamente. Come
avviene con gli alberi: c'è forse qualcuno che ordina a essi il modo in
cui devono crescere? Non possono parlare né andarsene in giro, però
sanno come crescere lontani dagli ostacoli. Dove lo spazio è angusto e
risulta difficile crescere, si piegano all'infuori. Proprio lì si trova
il Dhamma, non c'è bisogno di guardare un sacco di cose. Chi è astuto,
in questo scorgerà il Dhamma. Per natura gli alberi non sanno nulla,
agiscono sulla base delle leggi naturali, tuttavia sanno abbastanza per
crescere lontani dai pericoli, per piegarsi verso un luogo adatto.

Così è la gente che riflette. Diventiamo monaci perché vogliamo
trascendere la sofferenza. Che cos'è che ci fa soffrire? Se seguiamo il
Sentiero verso l'interiorità, lo scopriremo. Quello che ci piace e
quello che non ci piace sono sofferenza. Se sono sofferenza, allora non
avvicinatevi troppo. Volete innamorarvi dei fenomeni condizionati oppure
odiarli? Sono tutti quanti incerti. Quando incliniamo verso il Buddha
tutto giunge alla fine. Non dimenticatelo. E paziente sopportazione.
Queste due sole cose sono sufficienti. Se avete questo genere di
comprensione va molto bene.

In realtà nella mia pratica non ho avuto un maestro che mi impartisse
tanti insegnamenti, tanti quanti voi ne ricevete da me. Non ho avuto
molti insegnanti. Sono stato ordinato monaco in un piccolo monastero nei
pressi di un villaggio e ho vissuto per parecchi anni in monasteri di
questo genere. Nella mia mente prese forma il desiderio di praticare.
Volevo diventare abile, volevo addestrarmi. In quei monasteri non c'era
nessuno a darmi degli insegnamenti, ma sorse l'ispirazione per la
pratica. Ho viaggiato e mi sono guardato attorno. Avevo orecchi e perciò
sentivo, avevo occhi e perciò vedevo. Qualsiasi cosa sentissi dalla
gente, dicevo a me stesso: «~Non è sicuro.~» Qualsiasi cosa vedessi,
dicevo a me stesso: «~Non è sicuro.~» Oppure, quando la lingua entrava
in contatto con dei sapori -- dolci, aspri, salati, piacevoli o
spiacevoli che fossero -- o allorché sensazioni di agio o di dolore
sorgevano nel corpo, dicevo a me stesso: «~Non si tratta di una cosa
sicura!~» E così vivevo con il Dhamma. In verità tutto è incerto, ma i
nostri desideri vogliono che le cose siano certe. Che cosa possiamo
fare? Dobbiamo essere pazienti. La cosa più importante è
\emph{khanti},\footnote{\emph{Khanti}: Pazienza, sopportazione. È una
  delle Dieci Perfezioni.} una paziente sopportazione. Non rifiutate il
Buddha, quel che io chiamo ``incertezza''. Non rifiutatelo.

Talvolta sono andato a visitare alcuni siti religiosi del passato, nei
quali si trovano degli antichi edifici monastici disegnati da architetti
e costruiti da artigiani. In alcuni vi erano delle crepe. Uno dei miei
amici rimarcava: «~Ha delle crepe, che peccato, vero?~» Rispondevo: «~Se
non fosse così, allora non ci sarebbero cose come il Buddha, non ci
sarebbe il Dhamma. Ha delle crepe perché è perfettamente coerente con
l'insegnamento del Buddha.~» Veramente, in fondo anche a me dispiaceva
vedere delle crepe in quegli edifici, ma mettevo da parte il mio
sentimentalismo e cercavo di dire qualcosa che potesse essere utile ai
miei amici e a me stesso. Sebbene provassi del dispiacere, tendevo verso
il Dhamma. «~Se non avesse delle crepe in quella maniera, lì non ci
sarebbe alcun Buddha!~» Lo dicevo con molta convinzione a beneficio dei
miei amici che forse non ascoltavano, ma io sì.

È un modo di considerare le cose molto, molto utile. Immaginiamo ad
esempio che uno arrivi di corsa e dica: «~Luang Por! Sai che cosa quel
tale ha appena detto di te?~» Oppure: «~Di te ha detto così e cosà.~»
Può succedere che uno s'incollerisca. Appena si sentono delle critiche,
si comincia sempre a provare stati d'animo di questo genere. Appena
ascoltiamo parole come queste, iniziamo a essere pronti a controbattere.
Se però osserviamo la questione in modo più veritiero, possiamo notare
che non è così, e che magari è stata detta una cosa diversa da quel che
abbiamo pensato. Ed eccoci a un altro caso di ``incertezza''. Perché
allora dovremmo affrettarci a credere alle cose? Perché dovremmo riporre
fino a questo punto la nostra fede in quel che dicono gli altri?
Qualsiasi cosa si senta, la si dovrebbe notare, essere pazienti, e
osservare la questione in modo attento e retto.

Non è che scriviamo tutto quel che ci passa per la testa come se fosse
verità. Ogni discorso che non tenga conto dell'incertezza non è il
discorso di un saggio. Ricordatevelo. Qualsiasi cosa vediate o
ascoltiate, sia essa piacevole o dolorosa, dite solo: «~Non è sicuro!~»
Ditelo con convinzione a voi stessi, affossate tutto così. Quelle cose
non fatele diventare di primaria importanza, rimettetele al loro posto
in questo modo. Si tratta di una cosa importante. È a questo punto che
le contaminazioni muoiono. I praticanti non dovrebbero trascurarlo. Se
lo trascurate, aspettatevi solo sofferenze ed errori. Se non lo fate
diventare un fondamento della vostra pratica, sbaglierete. Vi potrete
però rimettere sulla giusta strada, perché questo è un principio davvero
ottimo.

In realtà il vero Dhamma, l'essenza di quel che vi ho detto oggi non è
poi così misteriosa. Qualsiasi cosa sperimentiate è semplicemente forma,
semplicemente sensazione, semplicemente percezione, semplicemente
volizione e semplicemente coscienza. Ci sono solo queste qualità
basilari. Dov'è che al loro interno c'è una qualche certezza? Se
riusciamo a comprendere la vera natura delle cose in questo modo,
svaniranno passioni, infatuazioni e attaccamenti. Perché svaniscono?
Perché comprendiamo. Conosciamo. Passiamo dall'ignoranza alla
comprensione. La comprensione nasce dall'ignoranza, la conoscenza nasce
dalla mancanza di conoscenza, la purezza nasce dalle contaminazioni. È
così che funziona. Non accantonate \emph{aniccā}, il Buddha. Ciò
equivale a dire che il Buddha è ancora in vita. Dire che il Buddha è
andato nel \emph{Nibbāna} non è necessariamente vero. In senso più
profondo, il Buddha è ancora vivo. Somiglia molto al senso che diamo
alla parola \emph{bhikkhu}. Se la definiamo come ``colui che
chiede'',\footnote{Ossia uno che vive dipendendo dalla generosità
  altrui.} il senso è molto ampio. Possiamo darle questo senso, ma
avvalersene non va molto bene: non ci viene detto quando smettere di
chiedere! Se volessimo dare a questa parola un senso più profondo,
diremmo: «~\emph{Bhikkhu}, colui che vede il pericolo del
\emph{saṃsāra}.~» Non è più profondo? Non va nella stessa direzione
della prima definizione, va molto più in profondità. Così è per la
pratica del Dhamma. Se non la comprendete appieno, diventa un'altra
cosa. Quando è completamente compresa, allora diventa inestimabile,
diventa fonte di pace.

Quando abbiamo \emph{sati}, siamo vicini al Dhamma. Se abbiamo
\emph{sati}, vedremo \emph{aniccā}, la transitorietà di tutte le cose.
Vedremo il Buddha e, in futuro se non proprio adesso, trascenderemo la
sofferenza del \emph{saṃsāra}. Se eliminiamo questa caratteristica degli
Esseri Nobili, del Buddha e del Dhamma, la nostra pratica diventerà
arida e infruttuosa. La nostra pratica dev'essere costante, sia che
lavoriamo sia che sediamo, anche quando stiamo solo distesi. Quando
l'occhio vede una forma, l'orecchio ode un suono, il naso sente un odore
e la lingua un sapore, o quando il corpo sperimenta una sensazione. Per
qualsiasi cosa, non eliminate il Buddha, non allontanatevi dal Buddha. È
così che si può essere vicini al Buddha, che lo si può costantemente
riverire. Abbiamo delle cerimonie per riverirlo, come succede nei canti
del mattino, \emph{Araham Sammā Sambuddho Baghavā} \ldots{} Si tratta di un
modo per riverire il Buddha, ma lo si può fare in un modo più profondo,
quello di cui vi ho appena parlato. È la stessa cosa che con la parola
\emph{bhikkhu}. Se la definiamo come ``colui che chiede'', allora
succede che si continui a chiedere perché quella è la definizione. Per
definirla meglio potremmo dire: «~\emph{Bhikkhu}, colui che vede il
pericolo del \emph{saṃsāra}.~» Lo stesso avviene quando riveriamo il
Buddha. Riverire il Buddha solo recitando frasi in pāli per le cerimonie
del mattino e della sera lo si può paragonare a definire la parola
\emph{bhikkhu} come ``colui che chiede''. Se ci volgiamo verso
\emph{aniccā}, \emph{dukkha} e \emph{anattā},\footnote{Transitorietà,
  carattere insoddisfacente e non sostanzialità, le Tre Caratteristiche
  universali (\emph{tilakkhaṇa}) di tutti i fenomeni.} tutte le volte
che l'occhio vede una forma, che l'orecchio ode un suono, il naso sente
un odore e la lingua un sapore, il corpo sperimenta una sensazione o la
mente conosce delle impressioni mentali, sempre, questo lo si può
paragonare a definire la parola \emph{bhikkhu} come ``colui che vede il
pericolo del \emph{saṃsāra}''. Va molto più a fondo, attraversa
moltissime cose. Se comprendiamo questo insegnamento, la nostra saggezza
e la nostra comprensione cresceranno.

Tutto questo si chiama \emph{paṭipadā}. Se sviluppate questo
atteggiamento nella pratica sarete sul Retto Sentiero. Se pensate e
riflettete in questo modo, anche se sarete lontani dal vostro
insegnante, gli sarete vicini. Se vivete vicini fisicamente
all'insegnante ma la vostra mente non lo ha ancora incontrato,
trascorrerete il vostro tempo o guardando i suoi errori oppure
adulandolo. Se fa qualcosa che a voi va bene, dite che è perfetto, e la
vostra pratica non va più in là di questo. Sprecando il vostro tempo a
guardare qualcun altro non arriverete da nessuna parte. Però, se
comprendete questo insegnamento potete diventare un Essere Nobile in
questo stesso momento. Ecco perché quest'anno\footnote{Il 2522 dell'era
  buddhista, il 1979 in Occidente.} ho preso le distanze dai miei
discepoli, sia vecchi che nuovi, e non ho insegnato molto: in modo tale
che tutti voi riusciate il più possibile a guardare dentro le cose da
voi stessi. Per i monaci che sono arrivati da poco tempo ho già scritto
il programma e le regole del monastero, ad esempio: «~Non parlate
troppo.~» Non trasgredite le norme già esistenti, trasgredirle significa
trasgredire nei riguardi del Sentiero verso la Realizzazione, la
Fruizione e il \emph{Nibbāna}. Chiunque le trasgredisca non è un vero
praticante, non è uno che ha un'intenzione davvero pura di praticare.
Uno così, quale speranza può avere di vedere la Verità? Anche se
dormisse tutti i giorni vicino a me, non mi vedrebbe. Qualora non
praticasse, anche se dormisse vicino al Buddha non vedrebbe il Buddha.

Conoscere il Dhamma o vedere il Dhamma dipende dalla pratica. Abbiate
fiducia, purificate il vostro cuore. Se nella mente di tutti i monaci di
questo monastero vi fosse consapevolezza, non ci sarebbe bisogno né di
rimproverare né di lodare nessuno. Non ci sarebbe bisogno di essere
sospettosi né di fare eccezioni per qualcuno. Se sorgono rabbia o
avversione, tenetevele nella mente, ma guardatele con chiarezza!
Continuate a osservare queste cose. Fino a quando c'è ancora qualcosa,
significa che è proprio lì che dobbiamo ancora scavare e impegnarci.
Alcuni dicono: «~Non riesco a eliminarla, proprio non ci riesco.~» Se
cominciamo a dire cose di questo genere, qui ci saranno solo
delinquenti, perché nessuno eliminerà le proprie contaminazioni. Dovete
provarci. Se non riuscite a eliminarle, scavate più a fondo. Scavate
nelle contaminazioni e sradicatele. Strappatele via anche se sembrano
tenaci e salde. Il Dhamma non è una cosa che potete raggiungere seguendo
i vostri desideri. La mente può essere in un modo, la Verità in un
altro. Fate attenzione guardando avanti, ma guardatevi pure le spalle. È
per questo che dico: «~Tutto è incerto, transitorio.~»

Questa verità dell'incertezza, questa breve e semplice verità, è nello
stesso tempo talmente profonda e inconfutabile che la gente tende a
ignorarla. Ha la tendenza a vedere le cose in modo differente. Non
attaccatevi alla bontà, non attaccatevi alla malvagità. Sono attributi
del mondo. Noi stiamo praticando per esseri liberi dal mondo, per
portare queste cose all'esaurimento. Il Buddha insegnò a lasciarle
andare, a rinunciare a esse, perché causano soltanto sofferenza.

