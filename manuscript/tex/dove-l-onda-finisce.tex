\chapter{Dove l'onda finisce}

\begin{openingQuote}
  \centering

  Estratti da una conversazione tra Ajahn Chah e un laico buddhista.
\end{openingQuote}

\emph{Domanda:} Ci sono dei periodi in cui al nostro cuore succede di essere
riassorbito dalle cose, ed esso si macchia e si offusca anche se siamo
ancora consapevoli di noi stessi. Come quando, ad esempio, affiorano
alcune forme di avidità, di odio o di illusione. Benché sappiamo che
queste sono cose riprovevoli, non siamo in grado di evitare che sorgano.
Si può dire che, benché siamo consapevoli di esse, ciò rappresenta la
base per un attaccamento e un aggrapparsi maggiori che forse ci fa
tornare indietro rispetto al punto dal quale siamo partiti?

\emph{Risposta:} Proprio così! Si deve continuare a conoscerle in quel punto,
questo è il metodo della pratica.

\emph{D.:} Voglio dire che ne siamo consapevoli e, contemporaneamente, proviamo
repulsione per esse, ma siccome non abbiamo l'abilità di resistere, non
fanno altro che continuare a sorgere.

\emph{R.:} Quando succede questo, fare qualcosa è già al di là delle nostre
capacità. A quel punto ci si deve ricomporre e continuare a contemplare.
Non rinunciare immediatamente. Quando si vede che sorgono delle cose in
quel modo, si ha la tendenza ad agitarsi o a provare rimorso, ma è
possibile dire che sono incerte e soggette al cambiamento. Quel che
succede è che si vede che si tratta di cose sbagliate, ma non si è
ancora pronti o non si è in grado di affrontarle. È come se fossero
entità indipendenti, sono le restanti tendenze del kamma che
stanno ancora creando e condizionando lo stato del cuore. Non si vuole
consentire al cuore di essere così, ma il cuore si comporta così e ciò
indica che la conoscenza e la consapevolezza non sono ancora né
sufficienti né abbastanza veloci per tenere il passo con le cose.

Si deve praticare e sviluppare la consapevolezza quanto più si può, al
fine di farla aumentare e diventare più penetrante. Non ha importanza se
il cuore in qualche modo si è macchiato o sporcato, si dovrebbe
contemplare l'impermanenza e l'incertezza di qualsiasi cosa sorga.
Mantenendo questa contemplazione ogni volta che sorge qualcosa, dopo un
po' di tempo si vedrà la natura impermanente insita in tutti gli oggetti
sensoriali e in tutti gli stati mentali. Siccome li si vede in questo
modo, gradualmente essi perdono la loro importanza, e l'attaccamento e
l'aggrapparsi a quella macchia del cuore continuerà a diminuire. Ogni
volta che sorge la sofferenza, si sarà in grado di lavorarci sopra e di
ricomporsi, ma non si dovrebbe rinunciare a questo lavoro o metterlo da
parte. Si dovrebbe continuare sia a sforzarsi in modo costante sia a
cercare di rendere la consapevolezza sufficientemente veloce per restare
in contatto con il cambiamento delle condizioni mentali. Si potrebbe
dire che fino a quel momento lo sviluppo del Sentiero manca ancora di
sufficiente energia per vincere le contaminazioni mentali. Il cuore si
offusca tutte le volte che sorge la sofferenza, ma bisogna continuare a
sviluppare questa conoscenza e questa comprensione del cuore offuscato.
È su questo che si deve riflettere.

Bisogna davvero prendere questo cuore offuscato e contemplare
ripetutamente che sofferenza e scontentezza non sono cose sicure. Sono
cose fondamentalmente impermanenti, insoddisfacenti e non-sé.
Focalizzando queste Tre Caratteristiche, avendone fatta esperienza in
precedenza, tutte le volte che queste condizioni di sofferenza sorgono,
le si conoscerà in modo diretto. Gradualmente, un po' alla volta, la
pratica dovrebbe guadagnare slancio e, con il passare del tempo,
qualsiasi oggetto dei sensi o stato mentale che si troverà a sorgere
perderà valore. Il cuore li conoscerà per quello che sono e, di
conseguenza, li lascerà andare. Il Sentiero sarà maturato interiormente
quando, dopo aver raggiunto il punto in cui si è in grado di conoscere
le cose e di lasciarle andare con facilità, si avrà l'abilità di
esercitare pressione sulle contaminazioni velocemente. Da allora in poi
ci sarà solo il sorgere e lo svanire, come le onde che s'infrangono
sulla riva del mare. Quando un'onda arriva e raggiunge la battigia, si
disintegra e svanisce. Arriva un'altra onda, e succede di nuovo: l'onda
non va oltre il limite della battigia. Allo stesso modo, nulla sarà in
grado di oltrepassare i limiti fissati dalla nostra consapevolezza.

Questo è il luogo in cui si incontreranno l'impermanenza,
l'insoddisfazione e il non-sé, e si giungerà a comprendere. È qui che le
cose svaniscono. Le Tre Caratteristiche dell'impermanenza,
dell'insoddisfazione e del non-sé sono uguali alla riva del mare, e a
tutti gli oggetti dei sensi e stati mentali che sperimentiamo succede la
stessa cosa che avviene alle onde. La felicità è incerta, è già sorta
molte volte in precedenza. La sofferenza è incerta, è già sorta molte
volte in precedenza. È così che sono. Nel cuore si saprà che è così che
sono, sono solo ``così come sono''. Il cuore sperimenterà tali
condizioni in questa maniera ed esse continueranno progressivamente a
perdere valore e importanza. Questo significa parlare delle Tre
Caratteristiche del cuore, del modo in cui esso è. È così per tutti,
anche per il Buddha e per i suoi discepoli era così

Quando la pratica del Sentiero matura, diventa automatica e non dipende
più da alcunché di esteriore. Quando sorge una contaminazione, si è
immediatamente consapevoli di essa e, di conseguenza, si è in grado di
contrastarla. Ovviamente, la fase in cui il Sentiero non è ancora
abbastanza maturo né sufficientemente veloce per vincere le
contaminazioni deve essere sperimentata da tutti. È inevitabile. Però, è
a quel punto che si deve riflettere abilmente. Non andate a investigare
altrove né cercate di risolvere il problema in qualche altro posto. La
cura sta proprio lì. Applicate la cura nel posto in cui le cose sorgono
e svaniscono. La felicità sorge e poi svanisce, vero? La sofferenza
sorge e poi svanisce, vero? Si sarà continuamente in grado di vedere il
processo del sorgere e dello svanire, e di vedere nel cuore quello che è
bene e quello che è male. Questi sono fenomeni che esistono e che fanno
parte della natura. Non attaccatevi con forza a essi, né create alcunché
partendo da essi.

Se si ha questo genere di consapevolezza, anche se si entrerà in
contatto con le cose, non ci sarà alcun rumore. In altre parole, si
vedrà il sorgere e lo svanire dei fenomeni in modo veramente naturale e
normale. Si vedranno solo le cose sorgere e poi svanire. Si comprenderà
il processo del sorgere e dello svanire alla luce dell'impermanenza,
dell'insoddisfazione e del non-sé. La natura del Dhamma è così. Quando
si possono vedere le cose per ``quello che sono'', allora restano
``quello che sono''. Non ci sarà alcun attaccarsi o aggrapparsi.
L'attaccamento sparirà appena si diverrà consapevoli di esso. Ci saranno
solo sorgere e svanire, e questa è la serenità. È serenità non perché
non si sente nulla. Si sente, ma se ne comprende la natura e non ci si
attacca né ci si aggrappa a nulla. Questo è ciò che s'intende per
serenità: il cuore fa ancora esperienza degli oggetti dei sensi, ma non
li segue né ne viene catturato. Si realizza una separazione tra gli
oggetti dei sensi e le contaminazioni. Quando il cuore entra in contatto
con un oggetto dei sensi e c'è una reazione emotiva di piacere, ciò fa
nascere la contaminazione. Se però si comprende il processo del sorgere
e dello svanire, non c'è nulla che possa veramente sorgere. Finirà lì.

\emph{D.:} C'è necessità di praticare e di ottenere il \emph{samādhi} prima di
poter contemplare il Dhamma?

\emph{R.:} Si può dire che è giusto da un punto di vista, ma se si parla
facendo riferimento alla pratica, allora deve venire prima
\emph{paññā}.\footnote{\emph{Paññā}: Saggezza, discernimento, visione
  profonda.} Seguendo però lo schema convenzionale, ci devono essere
\emph{sīla}, \emph{samādhi} e poi \emph{paññā}. Se si sta davvero
praticando il Dhamma, \emph{paññā} viene per prima. Se fin dall'inizio
c'è \emph{paññā}, ciò significa che si conosce quello che è giusto e
quello che è sbagliato, e che si conosce il cuore calmo e il cuore
turbato e agitato. Parlando sulla base delle Scritture, si deve dire che
la pratica del contenimento e della compostezza fanno sorgere un senso
di vergogna e di timore per ogni azione sbagliata che possa
potenzialmente trovarsi a sorgere. Una volta che il timore di quello che
sbagliato si è insediato, e non si agisce né ci si comporta
erroneamente, quello che è sbagliato non sarà presente nel praticante.
Quando in lui non c'è più nulla di sbagliato, ciò genera le condizioni
dalle quali sorgerà la calma. Questa calma rappresenta il fondamento dal
quale col passare del tempo il \emph{samādhi} crescerà e si svilupperà.

Quando il cuore è calmo, la conoscenza e la comprensione che sorgono da
questa calma sono chiamate \emph{vipassanā}. Significa che di momento in
momento c'è una conoscenza che è in accordo con la Verità, e all'interno
di tutto questo sono presenti differenti fattori. Se uno dovesse
scriverli su un pezzo di carta, scriverebbe \emph{sīla}, \emph{samādhi}
e \emph{paññā}. Parlandone, possono essere messi insieme e si può dire
che questi tre \emph{dhamma} formano un'unità, che sono inseparabili.
Però, qualora se ne debba parlare come di fattori differenti, sarebbe
più corretto dire \emph{sīla}, \emph{samādhi} e \emph{paññā}.

Ovviamente, quando si agisce in modo non salutare, al cuore è
impossibile calmarsi. È perciò più esatto vederli in uno sviluppo
congiunto, e sarebbe giusto dire che questo è il modo in cui il cuore
diviene calmo. La pratica del \emph{samādhi} implica la preservazione di
\emph{sīla}, che include la sorveglianza sulla sfera delle azioni del
corpo e della parola, per non fare nulla di non salutare che possa
condurre al rimorso o alla sofferenza. Ciò costituisce il fondamento per
la pratica della calma, e quando si è fondati nella calma, questo
costituisce a sua volta un fondamento che sostiene il sorgere di
\emph{paññā}. Nell'insegnamento formale si enfatizza l'importanza di
\emph{sīla}. La pratica dovrebbe essere bella all'inizio, bella nel
mezzo e bella alla fine: \emph{ādikalyāna}, \emph{majjhekalyāna},
\emph{pariyosānakalyāna}. Ecco com'è. Hai mai praticato il
\emph{samādhi}?

\emph{D.:} Sto ancora imparando. Il giorno dopo essere andato a trovare Tan
Ajahn al Wat Keuan, mia zia mi ha portato un libro che conteneva alcuni
tuoi insegnamenti. Quella mattina ho cominciato a leggere dei passi che
contenevano domande e risposte a vari problemi. In essi dicevi che per
il cuore la cosa più importante è sorvegliare e osservare il processo di
causa ed effetto che ha luogo al suo interno. Solo osservare e mantenere
la conoscenza delle varie cose che affiorano.

Nel pomeriggio stavo praticando la meditazione e durante la seduta fu
come se il mio corpo fosse scomparso. Non ero in grado di sentire le
mani e le gambe, e non c'erano sensazioni corporee. Sapevo che il corpo
era ancora là, ma non potevo sentirlo. La sera ho avuto l'opportunità di
andare a rendere omaggio a Tan Ajahn Tate e gli ho descritto in
dettaglio la mia esperienza. Disse che queste sono le caratteristiche
del cuore quando si unifica nel \emph{samādhi}, e che avrei dovuto
continuare a praticare. Quest'esperienza l'ho fatta una volta sola. In
occasioni successive mi è capitato di non essere talvolta in grado di
sentire solo alcune parti del corpo, ad esempio le mani, mentre altrove
c'erano ancora sensazioni. Durante la pratica, a volte mi domando se
stare solo seduti e consentire al cuore di lasciar andare tutto sia il
giusto modo di praticare. Forse dovrei pensare, e dedicarmi a problemi
vari, oppure a mie questioni irrisolte riguardanti il Dhamma.

\emph{R.:} Non è necessario continuare a esaminare o aggiungere altre cose in
questa fase. È a questo che Tan Ajahn Tate si riferiva. Non si deve
ribadire quel che già c'è o aggiungervi altro. Quando quel particolare
genere di conoscenza è presente, significa che il cuore è calmo, ed è
quello stato di calma che deve essere osservato. Qualsiasi sensazione si
abbia, che si provi come se ci fosse un corpo, un sé o no, non è questo
il punto. Dovrebbe essere tutto accolto all'interno della
consapevolezza. Questi fenomeni condizionati indicano che il cuore è
calmo e che si è unificato nel \emph{samādhi}.

Quando il cuore si è unificato per un lungo periodo, in alcune occasioni
c'è un cambiamento nelle condizioni, e ci si ritrae da questa
unificazione. Questo stato è chiamato \emph{appanā} \emph{samādhi}
(assorbimento meditativo) e, dopo esserci entrato, il cuore si ritrae da
esso. Infatti, sebbene non sia scorretto dire che il cuore si ritrae, in
realtà non si ritrae. Un altro modo di dirlo è che si capovolge
all'indietro, o che cambia, ma l'espressione usata dalla maggior parte
dei maestri è che, quando il cuore ha raggiunto lo stato di calma, si
ritrae. La gente resta ovviamente intrappolata in discussioni sulle
parole impiegate. Ciò può indurre difficoltà e ci si potrebbe iniziare a
chiedere: «~Come diavolo può ritrarsi? Questa faccenda del ritrarsi mi
confonde solo!~» Tutto questo può condurre a molte sciocchezze e
fraintendimenti, solo a causa delle parole.

Quel che è necessario capire è che la via della pratica consiste
nell'osservare queste condizioni con \emph{sati-sampajañña}.\footnote{\emph{Sampajañña}:
  ``Chiara comprensione'', consapevolezza di sé, autorammemorazione,
  attenzione, consapevolezza, presenza mentale, comprensione profonda.}
In accordo con la caratteristica dell'impermanenza, il cuore tornerà
indietro e si ritrarrà al livello dell'\emph{upacāra-samādhi}.\footnote{\emph{Upacāra-samādhi}:
  ``Concentrazione di accesso''; un livello di concentrazione precedente
  i \emph{jhāna}.} Se si ritrae a questo livello, si può ottenere una
certa qual conoscenza e comprensione, perché a un livello più profondo
non c'è conoscenza e comprensione. Se a questo punto c'è conoscenza e
comprensione, somiglierà al pensiero, a \emph{saṅkhāra}.\footnote{\emph{Saṅkhāra}:
  Formazione, fenomeno condizionato.}

È come quando due persone sono in conversazione e discutono assieme di
Dhamma. Ci si può rammaricare del fatto che il loro cuore non sia
veramente sereno, ma nei fatti il dialogo ha luogo all'interno dei
confini della calma e della moderazione che ha sviluppato. Queste sono
le caratteristiche del cuore quando si è ritirato al livello di
\emph{upacāra}: ci sarà l'abilità di conoscere e comprendere varie cose.

Il cuore resterà in questo stato per un certo tempo e poi si rivolgerà
di nuovo all'interno. In altre parole, si volgerà e tornerà indietro in
uno stato di calma più profondo di quello di prima. Oppure, è anche
possibile che raggiunga livelli più puri e calmi di energia concentrata
rispetto a quelli sperimentati in precedenza. Se raggiunge questo
livello di concentrazione, si dovrebbe solamente notare il dato di fatto
e continuare a osservare, fino a quando il cuore si ritrae di nuovo.
Quando l'avrà fatto, sarà possibile sviluppare conoscenza e comprensione
al sorgere di problemi di vario genere. È a questo punto che dovrebbero
essere investigati ed esaminati i differenti argomenti e le varie
questioni che riguardano il cuore, per comprenderli a fondo. Quando
avremo terminato con questi problemi, il cuore si sposterà di nuovo
all'interno, verso un più profondo livello di concentrazione. Il cuore
resterà lì e maturerà, libero da ogni altro lavoro e conflitto esterno.
Ci sarà solo conoscenza unificata e ciò preparerà e rafforzerà la
consapevolezza fino a quando giungerà il momento di riemergere.

Queste condizioni di entrare e uscire appariranno nel cuore durante la
pratica, ma è una cosa di cui è difficile parlare. Non è nocivo o
dannoso per la pratica. Dopo un certo periodo di tempo il cuore si
ritrarrà e in quel posto inizierà il dialogo interiore, che prenderà la
forma di \emph{saṅkhāra}, di formazioni mentali che condizionano il
cuore. Se non si sa che questa attività è \emph{saṅkhāra}, si potrebbe
pensare che è \emph{paññā}, o che \emph{paññā} sta sorgendo. Si deve
comprendere che questa attività modella e condiziona il cuore, e la cosa
più importante al riguardo è che essa è impermanente. Si deve
continuamente mantenere il controllo, e non consentire al cuore di
cominciare a seguire e a credere in tutte le varie creazioni e storie
che inventa. Tutto questo è solo \emph{saṅkhāra}, non diventa
\emph{paññā}.

\emph{Paññā} si sviluppa quando si ascolta e si conosce il cuore,
allorché il processo delle creazioni e dei condizionamenti porta il
cuore stesso in varie direzioni. Allora esso riflette sull'instabilità e
sull'incertezza del tutto. La comprensione dell'impermanenza delle
creazioni sarà, a quel punto, la causa per poter lasciar andare le cose.
Quando il cuore lascerà andare le cose e le deporrà, diverrà
gradualmente sempre più sereno e stabile. Si deve continuare a entrare e
a uscire dal \emph{samādhi} in questo modo affinché sorga \emph{paññā}.
Si otterranno conoscenza e comprensione.

Vari generi di problemi e di difficoltà tendono a sorgere nel cuore
quando si continua a praticare; però, quali che siano i problemi che il
mondo o perfino l'universo fa affiorare, si sarà in grado di affrontarli
tutti. La saggezza li seguirà e troverà le risposte per ogni questione e
dubbio. Ovunque si mediti, qualsiasi problema affiori, qualsiasi cosa
succeda, tutto sarà causa per il sorgere di \emph{paññā}. Si tratta di
un processo che avrà luogo da sé, libero da influssi esterni. È in
questo modo che sorgerà \emph{paññā}, ma quando ciò avverrà si dovrà
fare attenzione a non ingannarsi, considerandola come \emph{saṅkhāra}.
Tutte le volte che si riflette sulle cose e le si vede come impermanenti
e incerte, non ci si dovrebbe attaccare o aggrappare a esse in alcun
modo. Se si continua a sviluppare questo stato mentale, allorché
\emph{paññā} sarà presente nel cuore, essa prenderà il posto del normale
modo di pensare e di reagire, e il cuore diverrà più pieno e luminoso,
al centro di tutto. Quando ciò avviene, si conoscono e si comprendono le
cose come veramente sono, e il cuore sarà in grado di progredire con la
meditazione in modo corretto, senza essere tratto in inganno. Così
dovrebbe essere.

