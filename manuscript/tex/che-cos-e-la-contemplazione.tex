\chapter{Che cos'è la contemplazione?}

\begin{openingQuote}
  \centering

  Questo insegnamento è tratto da una sessione di domande e risposte intercorse
  tra un gruppo di discepoli di lingua inglese e il venerabile Ajahn Chah che
  ebbe luogo al monastero Wat Gor Nork durante il Vassa del 1979. Sono stati
  necessari alcuni ritocchi nella sequenza della conversazione per facilitare la
  comprensione.
\end{openingQuote}

\emph{Domanda:} Quando insegni l'importanza della contemplazione, parli di
quando ci si siede in meditazione e si pensa ad argomenti particolari,
ad esempio alle ``trentadue parti del
corpo''?\footnote{``trentadue parti del corpo'': Un
  tema di meditazione il quale prevede che si investighino le parti del
  corpo, quali i capelli (\emph{kesa}), i peli (\emph{loma}), le unghie
  (\emph{nakha}), i denti (\emph{danta}), la pelle (\emph{taco}) e così
  via, in rapporto al loro essere non attraenti (\emph{asubha}) e
  insoddisfacenti (\emph{dukkha}).}

\emph{Risposta:} Se la mente è davvero serena non è necessario. Il giusto
oggetto d'investigazione diviene ovvio quando la tranquillità si è
insediata in modo opportuno. Quando la contemplazione è vera non c'è
discriminazione tra ``giusto'' e ``sbagliato'', ``buono'' e ``cattivo'',
non c'è nulla di tutto questo. Non è che ti siedi e pensi: «~Oh, questo
è così e quell'altro è cosà~», e via di seguito. È una forma grossolana
di contemplazione. Con la contemplazione meditativa non è in questione
il pensiero. Si tratta piuttosto di quel che chiamiamo ``contemplazione
silenziosa''. Durante la nostra routine quotidiana prendiamo
consapevolmente in considerazione la reale natura dell'esistenza
mediante paragoni. Anche in questo caso è un genere grossolano
d'investigazione, ma conduce alla Verità.

\emph{D.:} Quando dici di contemplare il corpo e la mente, però, ci serviamo
del pensiero? Il pensiero può produrre vera visione profonda? Questa è
\emph{vipassanā}?\footnote{\emph{Vipassanā:} Visione
  profonda di natura intuitiva dei fenomeni fisici e mentali, del loro
  sorgere e scomparire.}

\emph{R.:} All'inizio abbiamo bisogno di lavorare usando il pensiero, ma in
seguito andremo al di là di esso. Se stiamo facendo vera contemplazione,
tutto il pensiero dualistico è cessato, ma abbiamo bisogno di iniziare
considerando le cose in modo dualistico. Alla fine tutto il pensiero e
tutte le considerazioni cessano.

\emph{D.:} Dici che ci deve essere una sufficiente tranquillità
(\emph{samādhi})\footnote{\emph{Samādhi:}
  Concentrazione, unificazione della mente, stabilità mentale.} per
contemplare. Quanta tranquillità intendi?

\emph{R.:} Sufficientemente tranquilli affinché ci sia presenza mentale.

\emph{D.:} Intendi stare con il ``qui e ora'', senza pensare al passato e al
futuro?

\emph{R.:} Pensare al passato e al futuro va bene, se comprendi cosa siano in
realtà queste cose, ma non devi restarci intrappolato. Trattale nello
stesso modo in cui tratteresti ogni altra cosa, ma non restarci
intrappolato. Quando vedi il pensiero solo come pensiero, questa è
saggezza. Non credere a nessun pensiero! Riconosci che tutto è solo
qualcosa che è sorto e che cesserà. Semplicemente, vedi tutto nel modo
in cui è, è quel che è, la mente è la mente, di per sé non è qualcosa o
qualcuno. Quando lo comprenderai, andrai al di là del dubbio.

\emph{D.:} Continuo a non capire. La vera contemplazione e il pensiero sono la
stessa cosa?

\emph{R.:} Utilizziamo il pensiero come strumento, ma la conoscenza che sorge a
causa di esso è al di sopra e al di là del processo del pensiero, e
conduce a non essere più ingannati dai nostri pensieri. Riconosci che
tutti i pensieri sono solo un movimento della mente, e pure che il
conoscere né nasce né muore. Da dove pensi che provenga tutto questo
movimento chiamato ``mente''? Tutto quel che chiamiamo mente -- tutta
l'attività -- è solo la mente convenzionale. Non è affatto la mente
reale. Ciò che è reale ``è'' solamente, né sorge né cessa.

Però, cercare di capire queste cose solo parlandone, non funziona.
Dobbiamo prendere veramente in considerazione l'impermanenza,
l'insoddisfazione e l'impersonalità (\emph{aniccā}, \emph{dukkha},
\emph{anattā}),\footnote{Le qualità di tutti i fenomeni; impermanenza
  (\emph{anicca}), carattere insoddisfacente (\emph{dukkha}) e non-sé
  (\emph{anatta}). Nel \emph{Glossario} p. \pageref{glossary-tilakkhana}, si veda Tre Caratteristiche
  (\emph{tilakkhaṇa}).} ossia dobbiamo usare il pensiero per contemplare
la natura della realtà convenzionale. Da questo lavoro nasce la saggezza
e, se si tratta di saggezza reale, tutto è completato, finito,
riconosciamo la vacuità. Benché ci possano essere pensieri, sono vuoti,
non ne siamo influenzati.

\emph{D.:} Come possiamo arrivare a questo livello della mente reale?

\emph{R.:} Lavorando con la mente che già avete, ovviamente! Vedete che tutto
quel che sorge è incerto, che non c'è nulla di stabile o di sostanziale.
Vedetelo con chiarezza e comprendete che davvero in nessun luogo c'è
qualcosa a cui aggrapparsi: è tutto vuoto. Quando vedrete le cose che
sorgono nella mente per quello che sono, non dovrete lavorare più con il
pensiero. Non avrete più alcun dubbio al riguardo.

Parlare della ``mente reale'' e così via, può essere solo di relativa
utilità per aiutarci a capire. Inventiamo parole allo scopo di studiare,
ma in realtà la natura è solo così com'è. Star seduti qui sul pavimento
di pietra del piano inferiore, ad esempio. Il pavimento è la base, non
si muove e non va da nessuna parte. Il piano superiore, sopra di noi, è
ciò che è sorto da essa. Il piano superiore è come tutto quello che
vediamo nella nostra mente: forma, sensazione, memoria, pensiero. In
realtà, essi non esistono nel modo in cui presumiamo che esistano. Sono
unicamente la mente convenzionale. Appena sorgono, svaniscono di nuovo,
di per sé non esistono realmente.

Nelle Scritture c'è una storia sul venerabile Sāriputta che esamina un
\emph{bhikkhu}\footnote{\emph{Bhikkhu:} Un monaco buddhista.} prima di
consentirgli di andare a svolgere le pratiche ascetiche itineranti
(\emph{dhutaṅga}\footnote{\emph{Dhutaṅga:} Pratica ascetica volontaria
  che i praticanti possono intraprendere di tanto in tanto, oppure come
  impegno a lungo termine, al fine di coltivare l'accontentarsi e
  purificare il \emph{sīla}. Se ne parla più dettagliatamente nel
  \emph{Glossario} p. \pageref{glossary-dhutanga}.} \emph{vaṭṭa}). Gli chiese come avrebbe risposto se
gli fosse stato chiesto: «~Cosa succede al Buddha dopo la morte?~» Il
\emph{bhikkhu} rispose: «~La forma, la sensazione, la percezione, il
pensiero e la coscienza dopo essere sorti, svaniscono.~» Il venerabile
Sāriputta accettò la risposta.

La pratica non è però solo questione di parole sul sorgere e sul
cessare. Dovete vederlo da voi. Quando siete seduti in meditazione,
vedete semplicemente cosa stia in realtà succedendo. Non seguite nulla.
Contemplazione non significa essere catturati dai pensieri. Il pensiero
contemplativo di chi è sul Sentiero non è uguale al pensiero del mondo.
Se non comprendete in modo giusto cosa s'intende con ``contemplazione'',
più pensate più sarete confusi.

La ragione per cui sottolineiamo l'importanza di coltivare la
consapevolezza è perché abbiamo bisogno di vedere con chiarezza cosa
succede. Dobbiamo comprendere il modo in cui funziona il nostro cuore.
Quando sono presenti questa consapevolezza e questa comprensione, allora
tutto è a posto. Per quale motivo pensi che ``Colui che Conosce la Via''
non agisca mai mosso dall'ira o dall'illusione? Le cause che fanno
sorgere queste cose sono assenti. Da dove potrebbero giungere? La
consapevolezza ha avvolto tutto.

\emph{D.:} La mente di cui stai parlando è chiamata ``Mente Originaria''?

\emph{R.:} Che cosa vuoi dire?

\emph{D.:} È come se tu stessi dicendo che c'è qualcos'altro al di fuori del
convenzionale corpo-mente (i cinque \emph{khandhā}).\footnote{\emph{Khandhā:}
  Aggregato, insieme di elementi col quale ci si identifica; le
  componenti fisiche e mentali della personalità e dell'esperienza
  sensoriale in generale.} C'è qualcos'altro? Come lo chiami?

\emph{R.:} Non c'è nulla e non lo chiamo affatto. Questo è tutto quel che ci
deve essere. Fatela finita con tutte queste cose. Perfino la conoscenza
non appartiene a nessuno, perciò facciamola finita, facciamola finita
con tutto! Non c'è nulla che valga la pena di volere! Tutto questo è
solo un mucchio di problemi. Quando si vede con chiarezza in questo
modo, allora tutto è finito.

\emph{D.:} Non potremmo chiamarla ``Mente Originaria''?

\emph{R.:} Visto che insisti, puoi chiamarla in questo modo. Puoi chiamarla in
qualsiasi modo ti piaccia in rapporto alla realtà convenzionale. Devi
però comprenderlo bene questo punto. È davvero importante. Se non
facessimo uso della realtà convenzionale, non avremmo alcuna parola o
concetto mediante il quale prendere in considerazione la realtà
effettiva: il Dhamma. È davvero importante capirlo.

\emph{D.:} Di quale grado di tranquillità parli a questo livello? E quale
genere di consapevolezza è richiesta?

\emph{R.:} Non c'è bisogno di pensare in questo modo. Se non si avesse la
giusta dose di tranquillità, non si sarebbe affatto in grado di
affrontare questi argomenti. C'è bisogno di sufficiente stabilità
mentale e di concentrazione sia per sapere quel che sta succedendo sia
perché sorgano lucidità e comprensione.

Se fai domande di questo genere significa che stai ancora dubitando. C'è
bisogno di una tranquillità mentale sufficiente per non restare più
intrappolati nei dubbi su quello che stiamo facendo. Se avessi
praticato, queste cose le comprenderesti. Più vai avanti con questo tipo
di domande, più rendi le cose confuse. Va bene se aiuta la
contemplazione, ma parlare non ti mostrerà il modo in cui sono davvero
le cose. Questo Dhamma non si comprende perché qualcun altro te ne
parla, devi vederlo da te, \emph{paccattaṃ}.\footnote{\emph{Paccattaṃ:}
  Da sperimentare individualmente e personalmente (\emph{veditabba}) da
  parte dei saggi (\emph{viññūhi}).} Se hai la capacità di comprendere
di cosa abbiamo parlato, allora diciamo che il tuo dovere di fare
qualcosa è finito, e questo significa che non fai nulla. Se c'è ancora
qualcosa da fare, allora è tuo dovere farlo.

Deponi semplicemente qualsiasi cosa, e sappi che è questo che stai
facendo. Non c'è sempre bisogno di starsi a controllare, preoccupandosi
di cose del tipo ``quanto \emph{samādhi}'' \ldots{} sarà sempre la giusta
quantità. Qualsiasi cosa sorga nella tua pratica, lasciala andare. Sappi
che tutto è incerto, impermanente. Ricordatelo! È tutto incerto. Falla
finita con tutto. Questa è la Via che ti porterà alla fonte, alla tua
Mente Originaria.

