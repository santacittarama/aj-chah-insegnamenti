\chapter{Chiara visione profonda}

\begin{openingQuote}
  \centering

  Discorso tenuto nell'aprile del 1979 a Bangkok per un gruppo di meditanti
  laici.
\end{openingQuote}

Meditate recitando ``Buddho, Buddho'', fino a che questa parola non
penetra profondamente nel cuore della vostra coscienza
(\emph{citta}).\footnote{\emph{Citta}: Mente-cuore; stato di coscienza.}
\emph{Buddho} rappresenta la consapevolezza e la saggezza del Buddha. In
pratica, dovete fare più affidamento su questa parola che su qualsiasi
altra cosa. La consapevolezza che essa reca vi condurrà a comprendere la
verità in relazione alla vostra stessa mente. È un vero rifugio, il che
significa che vi è sia consapevolezza sia visione profonda del momento
presente.

Gli animali possono avere una specie di presenza mentale. Hanno
consapevolezza quando inseguono la loro preda e si preparano ad
attaccare. Perfino un predatore necessita di una salda presenza mentale
per trattenere la preda che ha catturato, che resiste e lotta per
sfuggire alla morte. È una forma di presenza mentale. Per questa ragione
dovete essere in grado di distinguere tra i vari generi di
consapevolezza. Il Buddha insegnò a meditare recitando \emph{Buddho}
come un modo di concentrare la mente. Quando concentrate la mente su un
oggetto in maniera cosciente, essa si desta. È la consapevolezza a
destarla. Quando per mezzo della meditazione sorge questa conoscenza,
potete vedere la mente con chiarezza. Appena le mente è priva della
consapevolezza di \emph{Buddho}, anche se è presente la normale
consapevolezza mondana, la mente non è risvegliata ed è priva di visione
profonda. Non vi condurrà verso quel che è veramente benefico.

\emph{Sati}, la consapevolezza, dipende dalla presenza di \emph{Buddho},
la conoscenza. Deve essere una chiara conoscenza che conduce la mente a
diventare più luminosa e radiosa. L'effetto illuminante che questa
chiara conoscenza ha sulla mente è simile al rischiaramento prodotto da
una luce in una stanza oscurata. Fino a quando nella stanza c'è buio
pesto, ogni oggetto che si trova all'interno di essa è difficile da
distinguere, oppure è impossibile da vedere perché è completamente
oscurato dalla mancanza di luce. Però, quando iniziate a rendere più
intenso il chiarore della luce, essa inizierà a penetrare in tutta la
stanza, rendendovi in grado di vedere di momento in momento più
chiaramente e, così, consentendovi di conoscere sempre meglio i dettagli
di tutti gli oggetti che stanno lì dentro.

Potreste paragonare l'addestramento mentale a quando si insegna a un
bambino. Sarebbe impossibile costringere dei bambini che non hanno
ancora imparato a parlare, ad accumulare conoscenze a un ritmo
innaturalmente veloce, che è al di là delle loro possibilità. Non potete
essere troppo duri con loro né tentare di insegnare in una sola volta
più parole di quanto non riescano ad apprendere, semplicemente perché
non sarebbero in grado di mantenere abbastanza a lungo la loro
attenzione su quel che state dicendo. La mente è simile ai bambini. A
volte è opportuno lodarvi un po' e incoraggiarvi. Altre volte è più
opportuno essere critici. Come con i bambini: se li sgridate troppo
spesso e siete troppo infervorati quando avete a che fare con loro, non
faranno progressi nel giusto modo, benché intenzionati a fare bene. Se
li forzate troppo, saranno influenzati in modo sfavorevole, perché
mancano ancora di conoscenza e di esperienza, e il risultato sarà che
perderanno la giusta direzione. Se fate così con la vostra mente, non è
\emph{sammā-paṭipadā},\footnote{\emph{Sammā-paṭipadā}: Retta pratica}
non è un modo di praticare che conduce all'Illuminazione.
\emph{Paṭipadā} o pratica si riferisce all'addestramento e alla guida
del corpo, della parola e della mente. Ora mi riferisco in modo
specifico all'addestramento della mente.

Il Buddha insegnò che addestrare la mente implica sapere come insegnare
a se stessi e come andare controcorrente rispetto ai desideri. Per
insegnare alla mente si devono utilizzare dei mezzi abili, perché essa
resta costantemente intrappolata nei suoi stati di depressione o di
euforia. Questa è la natura della mente non illuminata: è proprio come
un bambino. I genitori possono insegnare a un bambino che non ha ancora
imparato a parlare perché hanno una migliore conoscenza del linguaggio e
sanno come rivolgersi a lui. I genitori sono sempre in grado di vedere
in quali cose il figlio manca di comprensione, perché sanno di più. Così
è addestrare la mente. Quando avete la consapevolezza di \emph{Buddho},
la mente è più saggia e ha un livello di conoscenza più raffinato del
normale. Questa consapevolezza consente di vedere sia le condizioni
della mente sia la mente stessa. Riuscite a vedere lo stato della mente
nel mezzo di tutti i fenomeni. Se è così, sarete in grado di impiegare
con naturalezza tecniche abili per addestrare la mente. Che siate preda
del dubbio o di qualsiasi altra contaminazione, per voi si tratta di un
fenomeno mentale che sorge nella mente e che deve essere investigato e
affrontato con la mente.

Quella consapevolezza che chiamiamo \emph{Buddho} la possiamo paragonare
ai genitori del bambino. I genitori sono gli insegnanti incaricati della
sua formazione, ed è perciò abbastanza naturale che tutte le volte che
lo lasciano libero debbano nel contempo tenerlo d'occhio, essere
consapevoli di cosa stia facendo e della direzione verso la quale stia
correndo o andando a carponi. A volte potete essere troppo intelligenti
e avere troppe buone idee. Per insegnare ai bambini, potreste pensare
talmente tanto a quello che è meglio per loro da arrivare al punto in
cui più metodi escogitate per insegnare, più loro si allontanano dagli
obiettivi che vorreste raggiungano. Più tentate e più insegnate, più
loro s'allontanano, finché non vanno davvero fuori strada e non riescono
a svilupparsi nel modo giusto.

Nell'addestrare la mente, è fondamentale superare il dubbio scettico.
Dubbio e incertezza sono dei potenti ostacoli, e devono essere
affrontati. L'investigazione delle tre catene -- la catena della
concezione dell'io (\emph{sakkāya-diṭṭhi}), la catena del cieco
attaccamento a regole e pratiche ritualistiche
(\emph{sīlabbata-parāmāsa}) e, infine, la catena del dubbio
(\emph{vicikicchā}) -- è la via d'uscita dall'attaccamento praticata
dagli Esseri Nobili (\emph{ariyapuggalā}). Però, all'inizio queste
contaminazioni le imparate solo dai libri. Manca ancora la visione
profonda del modo in cui sono davvero le cose. Investigare la concezione
dell'io è la maniera per andare al di là dell'illusione che identifica
il corpo con il sé. Ciò comprende l'attaccamento al vostro corpo come sé
o a quello degli altri come un sé concreto. \emph{Sakkāya-diṭṭhi}, o
concezione dell'io, si riferisce a quelle cose che voi pensiate siano
voi stessi. Significa attaccamento alla concezione che il corpo è un sé.
Dovete investigare questa concezione fino a che raggiungete una nuova
comprensione e riuscite a vedere la Verità: l'attaccamento al corpo è
una contaminazione che impedisce alla mente di tutti gli esseri umani di
vedere il Dhamma.

È per questo motivo che, prima di fare qualsiasi altra cosa, il
precettore istruisce ogni nuovo aspirante all'ordinazione monastica a
investigare i cinque oggetti di meditazione: capelli (\emph{kesā}), peli
(\emph{lomā}), unghie (\emph{nakhā}), denti (\emph{dantā}) e pelle
(\emph{taco}). È per mezzo della contemplazione e dell'investigazione
che sviluppate la visione profonda nella concezione dell'io. Questi
oggetti rappresentano il fondamento più immediato dell'attaccamento che
crea l'illusione della concezione dell'io. Contemplarli conduce
all'esame diretto della concezione dell'io e fornisce i mezzi mediante i
quali ogni generazione -- sia gli uomini sia le donne che applicano le
istruzioni del precettore per entrare nella comunità monastica -- può
veramente trascendere la concezione dell'io. Però, inizialmente
continuate a essere illusi, privi della visione profonda e perciò
incapaci di penetrare la concezione dell'io e di vedere la verità delle
cose così come sono. Non riuscite a vedere la Verità perché avete ancora
un attaccamento saldo e irriducibile. È questo attaccamento che sostiene
l'illusione.

Il Buddha insegnò a trascendere l'illusione. Il modo per trascenderla è
la chiara comprensione del corpo per quello che esso è. Con penetrante
visione profonda dovete vedere che la vera natura sia del vostro corpo
sia di quello degli altri è essenzialmente la stessa. Non vi è alcuna
fondamentale differenza tra i corpi delle persone. Il corpo è solo il
corpo. Non è sostanziale, non è un sé, non è vostro o loro. Questa
chiara visione profonda nella vera natura del corpo è chiamata
\emph{kāyānupassanā}. Un corpo esiste, voi lo classificate e gli date un
nome. Poi vi aggrappate e vi attaccate a esso pensando che sia il vostro
corpo o il corpo di lui o quello di lei. Vi attaccate alla concezione
che il corpo è permanente e che è una cosa pulita e gradevole. Questo
attaccamento va in profondità nella mente. È così che la mente si
attacca al corpo.

Concezione dell'io significa che siete ancora preda del dubbio e
dell'incertezza in relazione al corpo. La vostra visione profonda non ha
del tutto penetrato l'illusione che vede il corpo come un sé. Per tutto
il tempo che l'illusione resta, voi dite che il corpo è un sé o
\emph{atta}, e interpretate tutta la vostra esperienza come se ci fosse
un'entità solida e durevole che voi chiamate ``io''. Siete così
completamente attaccati al modo convenzionale di considerare il corpo
come un sé, che pare non vi sia alcuna possibilità di andare al di là di
esso. Però, la chiara comprensione in accordo con la verità del modo in
cui sono le cose significa che voi vedete il corpo per quello che è: il
corpo è solo il corpo. Con visione profonda, vedete il corpo per quello
che è, e questa saggezza neutralizza l'illusione del senso del sé.
Questa visione profonda che vede il corpo per quello che è conduce alla
distruzione dell'attaccamento (\emph{upādāna}) per mezzo del graduale
sradicamento dell'illusione mediante il lasciar andare.

Praticate la contemplazione del corpo per quello che è, fino a quando è
del tutto naturale che diciate a voi stessi: «~Oh, il corpo è solo il
corpo. È tutto qui.~» Quando questo modo di pensare s'è consolidato,
appena dite a voi stessi che ``è tutto qui'', la mente lascia andare.
C'è il lasciar andare dell'attaccamento al corpo. C'è la visione
profonda che vede il corpo solo come corpo. Alimentando questo senso di
distacco, vedendo continuamente il corpo solo come corpo, tutti i dubbi
e le incertezze vengono gradualmente sradicati. Quando investigate il
corpo, quanto più chiaramente lo vedete solo come corpo invece che come
persona, ``essere'', un ``io'' o un ``loro'', tanto più rilevante sarà
l'effetto sulla mente, un effetto che si tradurrà in una simultanea
rimozione del dubbio e dell'incertezza. Il cieco attaccamento a regole e
pratiche ritualistiche (\emph{sīlabbata-parāmāsa}), che nella mente si
manifesta come un brancolare, come un'ottusa percezione indotta dalla
mancanza di chiarezza in relazione al reale scopo della pratica, viene
anch'esso simultaneamente abbandonato, perché esso sorge congiuntamente
alla stessa concezione dell'io.

Potreste dire che le tre catene del dubbio, del cieco attaccamento a
rituali e pratiche e della concezione dell'io sono inseparabili e
perfino simili una all'altra. Quando vedete questa relazione con
chiarezza, allorché sorge una delle tre catene -- ad esempio il dubbio
-- voi siete in grado di lasciarla andare mediante la coltivazione della
visione profonda, e nello stesso tempo vengono abbandonate in modo
automatico le altre due catene. Si estinguono insieme.
Contemporaneamente lasciate andare quella concezione dell'io e quel
cieco attaccamento che inducono ad annaspare in modo confuso tra i
differenti modi di praticare. Le vedete come facenti parte del
complessivo attaccamento al senso del sé, che deve essere abbandonato.
Dovete ripetutamente indagare il corpo e frammentarlo nelle parti che lo
compongono. Quando vedete ogni parte per quello che veramente è,
gradualmente si corrode la percezione del corpo in quanto entità
compatta o io. Dovete sforzarvi continuamente in questa investigazione,
non dovete interromperla.

Un ulteriore aspetto dello sviluppo mentale che conduce a una più chiara
ed intensa visione profonda consiste nel meditare su un oggetto per
acquietare la mente. La mente calma è la mente che è salda e stabile nel
\emph{samādhi}. Può trattarsi di \emph{khanika samādhi} (concentrazione
momentanea), di \emph{upacāra samādhi} (concentrazione di accesso) o di
\emph{appanā samādhi} (assorbimento meditativo). Il livello di
concentrazione è determinato dal progressivo affinamento della coscienza
allorché addestrate la mente a mantenere la consapevolezza su un oggetto
di meditazione.

Nel \emph{khanika samādhi} (concentrazione momentanea) la mente si
unifica per un breve lasso di tempo. Si acquieta nel \emph{samādhi}, ma
essendosi unificata solo momentaneamente, subito si ritrae da quello
stato di quiete. Durante la meditazione, quando la concentrazione
diventa più raffinata, molte caratteristiche analoghe della mente serena
sono sperimentate a ogni livello, così che ognuno di essi è descritto
come un livello di \emph{samādhi}, sia che si tratti di \emph{khanika},
\emph{upacāra} o \emph{appanā}. A ogni livello la mente è calma, ma la
profondità del \emph{samādhi} varia e la natura dello stato di serenità
mentale sperimentato è differente. Su di un livello la mente è ancora
soggetta al movimento e può vagare, ma si muove all'interno dei confini
dello stato di concentrazione. Non viene catturata in attività che la
conducono ad agitazione e distrazione. La vostra consapevolezza può
seguire un oggetto mentale salutare per un po', prima di tornare a
stabilizzarsi sul punto di quiete nel quale rimane per un certo lasso di
tempo. Potete paragonare l'esperienza del \emph{khanika samādhi} con
un'attività fisica, ad esempio fare una camminata. Potreste camminare
per un po' prima di fermarvi a riposare, e dopo esservi riposati
ricominciare a camminare fino a quando arriva il momento di un'altra
sosta. Sebbene interrompiate periodicamente il cammino per smettere di
camminare e riposarvi, ogni volta che restate del tutto immobili è solo
un'immobilità temporanea del corpo. Dopo breve tempo dovete ricominciare
a muovervi per riprendere il cammino. Questo avviene all'interno della
mente allorché essa sperimenta tale livello di concentrazione.

Se praticate la meditazione focalizzandovi su un oggetto per calmare la
mente, e raggiungere un livello di quiete nel quale la mente è stabile
nel \emph{samādhi} ma vi è ancora una qualche attività mentale, si
tratta di \emph{upacāra samādhi}. Nell'\emph{upacāra samādhi} la mente
può ancora muoversi. Questo movimento si verifica entro certi limiti, la
mente non va oltre. I confini all'interno dei quali la mente può
muoversi sono determinati dalla saldezza e dalla stabilità della
concentrazione. Ciò che sperimentate è l'alternanza di uno stato di
calma e una certa qual attività mentale. La mente è calma per un po' di
tempo e attiva per il tempo rimanente. All'interno di quell'attività
persiste ancora un certo livello di calma e di concentrazione, ma la
mente non è completamente calma o immobile. Sta ancora pensando e
vagando un po'. È come quando vi muovete in casa vostra. Vagate
all'interno dei confini della vostra concentrazione, senza perdere la
consapevolezza e senza allontanarvi dall'oggetto di meditazione. Il
movimento della mente resta nei confini di stati mentali salutari
(\emph{kusala}). Non viene catturata da alcuna proliferazione mentale
legata a stati mentali nocivi (\emph{akusala}).\footnote{\emph{Akusala}:
  Non salutare, nocivo, maldestro, non meritorio.} Tutti i pensieri
rimangono salutari. Quando la mente è calma, di momento in momento
sperimenta necessariamente stati mentali salutari. Durante il tempo in
cui è concentrata, la mente sperimenta solo stati mentali salutari e
periodicamente si stabilizza e diventa completamente immobile,
unificandosi con il suo oggetto di meditazione. La mente sperimenta così
un po' di movimento, girando attorno al suo oggetto di meditazione. Può
ancora vagare. Può muoversi all'interno dei confini fissati dal livello
di concentrazione, ma da questo movimento non sorge alcun pericolo
perché la mente è calma in \emph{samādhi}. È così che lo sviluppo della
mente procede nel corso della pratica.

Nell'\emph{appanā samādhi} la mente si calma e si acquieta a un livello
in cui essa è raffinata e abile al più alto grado. Anche se sperimentate
interferenze sensoriali che provengono dall'esterno, ad esempio dei
suoni e delle sensazioni fisiche, esse rimangono esterne e non sono in
grado di disturbare la mente. Potete sentire un suono, ma non vi
distrarrà dalla vostra concentrazione. C'è il sentire il suono, ma è
come se non sperimentaste nulla. C'è consapevolezza dell'interferenza,
ma è come se non ve ne accorgeste. Questo avviene perché lasciate
andare. La mente lascia andare automaticamente. La concentrazione è
talmente profonda e stabile che lasciate andare l'attaccamento
all'interferenza dei sensi in modo del tutto naturale. La mente può
rimanere a lungo assorta in questo stato. Dopo essere stata all'interno
di questo stato per un appropriato lasso di tempo, se ne ritrae. A
volte, quando uscite da un livello di concentrazione così profondo, può
apparire l'immagine mentale di un qualche aspetto del vostro corpo. Può
essere un'immagine mentale che mostra un aspetto affiorato alla
consapevolezza della natura non attraente del corpo. Quando la mente
esce da uno stato raffinato, l'immagine del corpo sembra emergere ed
espandersi dall'interno della mente. A questo punto ogni aspetto del
corpo potrebbe sorgere come immagine mentale e colmare l'occhio della
mente.

Immagini che sopraggiungono in questo modo sono estremamente chiare e
inconfondibili. Dovete avere autenticamente sperimentato una
tranquillità davvero profonda affinché esse possano sorgere. Le vedete
in modo assolutamente chiaro, anche se i vostri occhi sono chiusi. Se li
aprite non riuscite a vederle, ma con gli occhi chiusi e la mente
nell'assorbimento del \emph{samādhi} potete vedere immagini di questo
tipo con la stessa chiarezza con cui le vedreste a occhi spalancati.
Potete sperimentare perfino un periodo ininterrotto di consapevolezza,
durante il quale la mente mette di volta in volta a fuoco immagini che
esprimono la natura non attraente del corpo. L'apparire di queste
immagini in una mente calma può essere la base per la visione profonda
della natura impermanente del corpo, e anche della sua natura non
attraente, immonda e sgradevole, oppure della completa mancanza di
qualsiasi sé o essenza all'interno di esso.

Quando sorgono questi tipi di speciale conoscenza, essi forniscono il
fondamento per un'abile investigazione e per lo sviluppo della visione
profonda. Portate questo tipo di visione profonda proprio nel vostro
cuore. Più lo fate, più ciò diventa la causa per far sorgere da sé la
conoscenza prodotta dalla visione profonda. A volte, quando rivolgete la
vostra riflessione sull'argomento dell'\emph{asubha},\footnote{\emph{Asubha}:
  Non bello, da intendersi come repulsivo, ripugnante e sporco.} nella
mente possono manifestarsi in modo automatico varie immagini di
differenti aspetti non attraenti del corpo. Queste immagini sono più
chiare di qualsiasi altra che potreste cercare di evocare mediante la
vostra immaginazione, e conducono a una visione profonda molto più
penetrante di quella che è possibile raggiungere mediante l'ordinario
pensiero discorsivo.

Questo genere di chiara visione profonda produce un impatto così forte
che l'attività mentale viene indotta a fermarsi, e subito dopo si
sperimenta una profonda sensazione di distacco. Tutto questo è così
chiaro e penetrante poiché si origina in una mente completamente serena.
Investigare quando si è in uno stato di serenità conduce a una visione
profonda sempre più chiara, e la mente diventa sempre più serena man
mano che l'assorbimento contemplativo aumenta. Più chiara e decisiva è
la visione profonda, più la mente penetra all'interno con la sua
investigazione, costantemente supportata dalla calma del \emph{samādhi}.
La pratica del \emph{kammaṭṭhāna} implica tutto questo. Investigare
continuamente in questo modo aiuta a lasciar andare in continuazione e,
infine, distrugge l'attaccamento alla concezione dell'io. Conduce al
termine di tutti i dubbi e di tutte le incertezze restanti su
quell'ammasso di carne che chiamiamo corpo e a lasciar andare il cieco
attaccamento a regole e pratiche ritualistiche.

Perfino in occasione di gravi malattie, di febbri tropicali e di vari
problemi di salute che solitamente hanno un forte impatto e scuotono il
corpo, il vostro \emph{samādhi} e la vostra visione profonda restano
stabili e imperturbabili. La vostra comprensione e la vostra visione
profonda vi consentono di distinguere con chiarezza tra mente e corpo.
La mente è un fenomeno, il corpo un altro. Quando vedete corpo e mente
come del tutto e indiscutibilmente separati l'uno dall'altra, ciò
significa che la pratica della visione profonda vi ha condotti a un
punto nel quale la vostra mente vede con certezza la vera natura del
corpo. Vedere il modo in cui il corpo veramente è, con chiarezza e senza
dubbi dall'interno della calma del \emph{samādhi}, conduce la mente a
sperimentare una forte sensazione di stanchezza e di allontanamento
(\emph{nibbidā}).

Questo allontanamento proviene da un senso di disincanto e di distacco
che sorge come naturale risultato del vedere le cose così come sono. Non
è un allontanamento che proviene dagli ordinari umori mondani quali la
paura e la repulsione, o da altri stati mentali non salutari come
l'invidia e l'avversione. Non proviene dalla stessa radice
dell'attaccamento, come quegli stati mentali contaminati. Questo
allontanamento reca in sé una qualità spirituale ed ha un effetto
differente sulla mente, se paragonato ai normali umori di noia e
stanchezza sperimentati dagli ordinari esseri umani non illuminati
(\emph{puthujjana}). Di solito, quando gli esseri umani ordinari non
illuminati sono stanchi ed esasperati, sono preda dell'avversione e del
rigetto, e cercano di evitare le situazioni. L'esperienza della visione
profonda non è la stessa cosa.

La sensazione di stanchezza del mondo che cresce con la visione profonda
conduce ovviamente al distacco, all'allontanamento e all'indifferenza,
le quali provengono tutte quante dall'investigazione e dalla
comprensione della verità sul modo in cui sono le cose. È libera
dall'attaccamento all'idea dell'io che cerca di controllare e di forzare
le cose affinché queste seguano i desideri. Si lascia invece andare,
accettando le cose così come sono. La chiarezza della visione profonda è
talmente forte che non si sperimenta più alcun senso dell'io che lotta
contro il fluire dei desideri o sopporta a causa dell'attaccamento. Le
tre catene della concezione dell'io, del dubbio e del cieco attaccamento
a regole e pratiche ritualistiche, che di norma soggiacciono al modo di
vedere il mondo, non possono più ingannarvi o indurvi a fare alcun grave
errore nella pratica. Proprio questo è l'inizio del Sentiero, la prima
chiara visione profonda all'interno della Verità ultima, e ciò spiana la
via per ulteriore visione profonda. Potreste descrivere tutto questo
come una penetrazione nelle Quattro Nobili Verità.

Le Quattro Nobili Verità devono essere realizzate mediante la visione
profonda. Ogni monaco e ogni monaca, chiunque le abbia comprese, ha
sperimentato questa visione profonda nella verità del modo in cui sono
le cose. Conoscete la sofferenza, conoscete la causa della sofferenza,
conoscete la cessazione della sofferenza e conoscete il Sentiero che
conduce alla cessazione della sofferenza. La comprensione di ogni Nobile
Verità emerge nello stesso luogo, all'interno della mente. Giungono
insieme e si armonizzano come fattori del Nobile Ottuplice Sentiero, e
il Buddha insegnò che devono essere comprese all'interno della mente.
Quando i fattori del Sentiero convergono al centro della mente,
eliminano ogni dubbio e ogni incertezza che ancora avete sul modo di
praticare.

Durante la pratica è normale che si sperimentino le varie condizioni
della mente. Sperimentate costantemente il desiderio di fare questo o
quello, oppure di andare in vari luoghi, come pure i differenti stati
mentali del dolore, della frustrazione o anche l'indulgere alla ricerca
del piacere. Sono tutti frutti del kamma passato. Tutto il
kamma risultante si gonfia dentro la mente e viene fuori.
Ovviamente, è il prodotto delle azioni passate. Sapere che tutta questa
roba viene dal passato non vi consente di fare qualcosa di nuovo o di
particolare. Osservate e riflettete sul sorgere e sul cessare delle
condizioni mentali. Ciò che non è già sorto, non è ancora sorto. Questa
parola, ``sorgere'', si riferisce a \emph{upādāna}, al saldo aggrapparsi
e attaccarsi della mente. Per lungo tempo la vostra mente è stata
esposta alla brama e alle contaminazioni ed è stata condizionata da
esse, e le condizioni e caratteristiche mentali che sperimentate ne sono
i riflessi. Dopo aver sviluppato la visione profonda, la vostra mente
non segue più questi vecchi modelli abituali, forgiati dalle
contaminazioni. Avviene una separazione tra la mente e questi modi
contaminati di pensare e di reagire. La mente si separa dalle
contaminazioni.

Potete paragonarlo con ciò che avviene quando si versano insieme olio e
acqua all'interno di una bottiglia. I due liquidi hanno una loro densità
molto diversa, e per questo non importa se li conservate nella stessa
bottiglia oppure in due bottiglie separate, perché le loro differenti
densità impediscono ai liquidi di mescolarsi, uno non riesce a penetrare
nell'altro. L'olio non si mescola con l'acqua e viceversa. Restano
separati in due diverse parti della bottiglia. Potete paragonare la
bottiglia al mondo, e questi due differenti liquidi presenti nella
bottiglia e messi lì -- costretti a stare all'interno di essa -- a voi
che vivete nel mondo con la visione profonda che separa la vostra mente
dalle contaminazioni. Potete dire che state vivendo nel mondo e
seguendone le convenzioni, ma senza attaccarvi a esso. Quando dovete
andare da qualche parte dite che andate, quando state tornando dite che
state tornando, qualsiasi cosa facciate utilizzate le convenzioni e il
linguaggio del mondo, ma avviene come per i due liquidi: sono nella
stessa bottiglia ma non si mescolano. Vivete nel mondo, ma nello stesso
tempo siete separati da esso. Il Buddha conobbe la Verità da sé. Egli
era \emph{lokavidū}, il Conoscitore del mondo.

Che cosa sono le basi dei sensi (\emph{āyatana})? Sono costituite dagli
occhi, dagli orecchi, dal naso, dalla lingua, dal corpo e dalla mente.
Gli orecchi sentono i suoni. Il naso svolge la funzione di sentire i
vari odori, sia fragranti che pungenti. La lingua ha la funzione di
sentire i sapori, sia dolci che aspri, intensi o salati che siano. Il
corpo percepisce il caldo e il freddo, la morbidezza e la durezza. La
mente riceve gli oggetti mentali che sorgono nel modo in cui ha sempre
fatto. Le basi dei sensi funzionano proprio come prima. Sperimentate
l'impatto sensoriale nel modo in cui l'avete sempre fatto. Non
corrisponde al vero che dopo l'esperienza della visione profonda il
vostro naso non può più sperimentare alcun odore o che la vostra lingua,
che prima era in grado di percepire i sapori, non può più assaporare
nulla, o che il corpo sia incapace di qualsiasi sensazione.

La vostra abilità di sperimentare il mondo per mezzo dei sensi rimane
intatta, è proprio la stessa che avevate prima di praticare la visione
profonda, ma la reazione della mente all'impatto sensoriale consiste nel
considerarlo per ``quello che è''. La mente non s'attacca a percezioni
fisse e non estrae nulla dall'esperienza degli oggetti dei sensi. Lascia
andare. La mente sa che si tratta del lasciar andare. Quando ottenete la
visione profonda nella vera natura del Dhamma, ne risulta naturalmente
il lasciar andare. C'è consapevolezza, seguita dall'abbandono
dell'attaccamento. C'è comprensione e poi il lasciar andare. Con la
visione profonda deponete le cose. La conoscenza della visione profonda
non conduce all'aggrapparsi, all'attaccamento, e la sofferenza non
aumenta. Non è questo ciò che avviene: la vera visione profonda nel
Dhamma ha come risultato il lasciar andare. Sapete che l'attaccamento è
la causa della sofferenza, e perciò lo abbandonate. Quando avete la
visione profonda la mente lascia andare. Depone tutto quello a cui in
precedenza si aggrappava.

Un altro modo di descrivere tutto questo è dire che nella vostra pratica
non tendete più a trafficare e brancolare. Non andate più ciecamente a
tentoni, e non vi attaccate più a forme, suoni, odori, sapori,
sensazioni corporee o a oggetti mentali. L'esperienza degli oggetti dei
sensi per mezzo degli occhi, degli orecchi, del naso, della lingua, del
corpo e della mente non stimola più i soliti vecchi movimenti abituali
della mente, che in precedenza mirava a essere coinvolta dagli oggetti
dei sensi o ad aggiungere all'esperienza ulteriore proliferazione
mentale. La mente non crea cose attorno al contatto con gli oggetti dei
sensi. Appena avviene il contatto, lasciate automaticamente andare. La
mente scarta l'esperienza. Questo significa che se siete attratti da
qualcosa, sperimentate l'attrazione nella mente ma non vi attaccate, non
vi aggrappate saldamente a essa. Se avete una reazione di avversione, vi
è semplicemente l'esperienza dell'avversione che sorge nella mente e
nulla di più. Non sorge alcun senso dell'io che si attacca e attribuisce
significato e importanza all'avversione. In altre parole, la mente sa
come lasciar andare, sa come mettere le cose da parte. Perché è in grado
di lasciar andare e di deporre le cose? Perché la presenza della visione
profonda fa sì che riusciate a capire i risultati dannosi che provengono
dall'attaccamento a tutti quegli stati mentali.

Quando vedete le forme la mente resta indisturbata. Quando sentite dei
suoni resta indisturbata. La mente non prende posizione pro e contro gli
oggetti che sperimenta. Lo stesso avviene con i contatti sensoriali che
avvengono per mezzo degli occhi, degli orecchi, del naso, della lingua,
del corpo o della mente. Qualsiasi pensiero sorga nella mente non è in
grado di disturbarvi. Siete capaci di lasciar andare. Potete percepire
una cosa come desiderabile, ma non vi attaccate a quella percezione e
non le attribuite alcuna importanza particolare. Diviene solo una
condizione della mente da osservare senza attaccamento. Questo è ciò che
il Buddha descrisse come sperimentare gli oggetti dei sensi per ``quello
che sono''. Le basi dei sensi sono ancora in funzione e fanno esperienza
degli oggetti dei sensi, ma senza quel processo dell'attaccamento che
stimola nella mente un andirivieni di pensieri. Non c'è quel
condizionamento della mente che si attiva con un senso dell'io che si
muove da qui a là o da là a qui. Il contatto sensoriale avviene com'è
normale nelle sei basi, ma la mente non prende posizione, non resta
coinvolta nelle condizioni dell'attrazione o dell'avversione. Capite
come si lascia andare. C'è consapevolezza del contatto sensoriale
seguito dal lasciar andare. Lasciate andare con consapevolezza e
sostenete la consapevolezza dopo aver lasciato andare. Così funziona il
processo della visione profonda. Ogni angolo e aspetto della mente e
della sua esperienza diventano con naturalezza parte della pratica.

L'addestramento agisce sulla mente in questo modo. È assolutamente ovvio
che la mente si modifichi e che non sia più la stessa di prima. Non si
comporta più nella maniera in cui eravate abituati. Non partite più
dalla vostra esperienza per creare un io. Ad esempio, se sperimentate la
morte di vostra madre, di vostro padre o di chiunque altro vi sia stato
vicino, e la vostra mente resta stabile nella pratica della calma e
della visione profonda ed è in grado di riflettere con abilità su quel
che è successo, non si genera sofferenza. Invece di farvi prendere dal
panico o di sentirvi sconvolti per la notizia della morte di quella
persona, c'è solo una sensazione di tristezza e di disincanto che
proviene dalla saggia riflessione. Siete consapevoli dell'esperienza e
poi lasciate andare. C'è conoscenza, e poi mettete la cosa da parte.
Lasciate andare senza procurarvi alcuna ulteriore sofferenza. Questo
avviene perché conoscete con chiarezza ciò che fa sorgere la sofferenza.
Quando incontrate la sofferenza, siete consapevoli di quella sofferenza.
Non appena cominciate a sperimentare sofferenza, automaticamente vi
ponete la domanda: da dove proviene? La sofferenza ha una causa, che è
l'aggrapparsi, l'attaccamento che ancora resta nella mente. Perciò
dovete lasciar andare l'attaccamento. Tutta la sofferenza proviene da
una causa. Dopo aver generato la causa, la abbandonate. La abbandonate
con saggezza. La lasciate andare tramite la visione profonda, ciò che
significa saggezza. Non potete lasciar andare tramite l'illusione. Così
stanno le cose.

L'investigazione e lo sviluppo della visione profonda nel Dhamma fa
sorgere questa profonda pace nella mente. Quando avete ottenuto una
visione profonda così chiara e penetrante, essa è sempre sostenuta sia
che stiate praticando la meditazione da seduti a occhi chiusi sia che
stiate facendo qualcos'altro a occhi aperti. Quali che siano le
circostanze in cui vi trovate, in meditazione formale o no, la chiara
visione profonda resta. Quando avete un'incrollabile consapevolezza
della mente nella mente, non vi dimenticate di voi stessi. In piedi,
camminando, seduti o distesi, l'interna presenza mentale rende
impossibile una perdita della consapevolezza. Si tratta di uno stato di
presenza mentale che vi impedisce di dimenticarvi di voi stessi. La
consapevolezza è diventata così forte che si sostiene da sé fino al
punto che è naturale per la mente essere in questo modo. Questi sono i
risultati dell'addestramento e della coltivazione della mente ed è qui
che andate al di là del dubbio. Non avete dubbi sul futuro, non avete
dubbi sul passato e, di conseguenza, non avete necessità di dubitare
nemmeno sul presente. Siete ancora consapevoli che vi sono cose chiamate
passato, presente e futuro, ma non vi interessano né vi preoccupano.

Perché non vi interessano più? Tutte le cose avvenute in passato sono
già successe. Il passato è già trascorso. Tutto quel che sorge nel
presente è il risultato di cause che stanno nel passato. Per fare un
esempio ovvio, si può dire che se ora non avete fame è perché avete già
mangiato in precedenza. La mancanza di fame nel presente è il risultato
di azioni compiute nel passato. Se avete conoscenza della vostra
esperienza nel presente, potete conoscere il passato. Aver consumato un
pasto è la causa che proviene dal passato, il cui risultato è sentirsi a
proprio agio o pieni di energie nel presente, e questa è la causa che vi
fa poi essere attivi e vi consente di lavorare. Perciò il presente
fornisce cause che avranno risultati in futuro. Il passato, il presente
e il futuro possono perciò essere visti come una cosa sola. Il Buddha la
chiamò \emph{eko Dhamma}, l'unitarietà del Dhamma. Non si tratta di
molte cose diverse: è tutto qui. Quando vedete il presente, vedete il
futuro. Comprendendo il presente, capite il passato. Passato, presente e
futuro costituiscono una catena ininterrotta di cause ed effetti e,
perciò, fluiscono costantemente l'uno dall'altro. Ci sono cause nel
passato che producono risultati nel presente, e questi ultimi stanno già
producendo cause per il futuro. Questo processo di causa ed effetto si
applica anche alla pratica. Sperimentate i frutti per aver addestrato la
mente al \emph{samādhi} e alla visione profonda, e necessariamente l'uno
e l'altra rendono la mente più saggia e abile.

La mente trascende del tutto il dubbio. Non siete più incerti né fate
congetture su alcunché. L'assenza di dubbio significa che non annaspate
né sentite il bisogno di capire quale debba essere il vostro modo di
praticare. Il risultato è che vivete e agite in consonanza con la
natura. Vivete nel mondo nella maniera più naturale possibile. Ciò
significa vivere nel mondo serenamente. Siete capaci di trovare serenità
anche laddove non c'è pace. Siete del tutto in grado di vivere nel
mondo. Siete in grado di vivere nel mondo senza farvi alcun problema. In
quanto praticanti del Dhamma, dovete imparare a fare così. Non perdetevi
nelle percezioni e non attaccatevi a esse pensando che le cose siano in
questo modo o in quell'altro. Non attaccatevi, non date eccessiva
importanza ad alcuna percezione, trasformandola in un'illusione.

Tutte le volte che la mente s'infiamma, investigate e contemplatene la
causa. Quando non creerete alcuna sofferenza a voi stessi partendo dalle
cose, sarete a vostro agio. Quando non ci sono problemi che causano
agitazione mentale, restate equanimi. Ossia continuate a praticare
normalmente con un'equanimità sostenuta dalla consapevolezza e da una
presenza mentale a tutto tondo. Conservate un senso di autocontrollo e
di equilibrio. Se sorge una qualsiasi cosa che prevale sulla mente,
immediatamente la accogliete per investigarla e contemplarla. Se in quel
momento vi è chiara visione profonda, la penetrate con saggezza e
prevenite la creazione di qualsiasi sofferenza. Se non c'è ancora chiara
visione profonda, lasciate momentaneamente andare per mezzo della
pratica della meditazione \emph{samatha} e non consentite alla mente di
attaccarsi. In futuro, prima o poi la vostra visione profonda sarà
certamente forte abbastanza per penetrare le cose, perché prima o poi la
svilupperete a sufficienza per comprendere tutto ciò che ancora causa
attaccamento e sofferenza.

In definitiva, la mente deve fare un grande sforzo per lottare con le
reazioni che sperimentate sia agli stimoli prodotti da ogni genere di
oggetto dei sensi sia agli stati mentali, e per superarle. Deve lavorare
sodo con ogni oggetto con il quale entra in contatto. Tutte e sei le
basi interne dei sensi con i loro oggetti esterni confluiscono nella
mente. Focalizzando la consapevolezza solo sulla mente, guadagnate
comprensione e visione profonda in relazione agli occhi, agli orecchi,
al naso, alla lingua, al corpo, alla mente e a tutti i loro oggetti. La
mente è già lì. Per questo motivo è importante investigare proprio il
centro della mente. Quanto più vi spingete a investigare la mente
stessa, tanto più chiara e intensa sarà la visione profonda che
emergerà. È una cosa che sottolineo quando insegno, perché comprendere
questo punto è di cruciale importanza per la pratica. Di solito, quando
sperimentate un contatto sensoriale, dai differenti oggetti deriva un
impatto, e la mente attende solo di reagire con attrazione o avversione.
Questo è quel che succede alla mente non illuminata. È pronta per
restare catturata nel buon umore a causa di un certo tipo di stimolo o
nel cattivo umore a causa di un altro.

Nel nostro caso, invece, esaminiamo la mente con ferma e incrollabile
attenzione. Quando fate esperienza dei vari oggetti per mezzo dei sensi,
non nutrite la proliferazione mentale. Non restate catturati da una gran
quantità di pensieri contaminati. State già praticando la meditazione
\emph{vipassanā} e fate affidamento sulla saggezza della visione
profonda per investigare tutti gli oggetti dei sensi. La meditazione
\emph{vipassanā} sviluppa la saggezza. Addestrandovi con i differenti
oggetti della meditazione \emph{samatha} -- che si tratti della
recitazione di parole come \emph{Buddho}, \emph{Dhammo}, \emph{Saṅgho},
o della pratica della consapevolezza del respiro -- il risultato è che
la mente sperimenta la calma e la stabilità del \emph{samādhi}. Nella
meditazione \emph{samatha} si mette a fuoco la consapevolezza su un solo
oggetto e si lascia temporaneamente andare tutto il resto.

La meditazione \emph{vipassanā} è simile perché, quando si entra in
contatto con gli oggetti dei sensi, si utilizza la riflessione ``non ci
credo''. Praticando la \emph{vipassanā} non consentite a nessun oggetto
dei sensi di ingannarvi. Siete consapevoli di ogni oggetto non appena
esso converge nella mente e -- che sia sperimentato con gli occhi, con
gli orecchi, con il naso, con la lingua, con il corpo o con la mente --
utilizzate questa riflessione, ``non ci credo'', quasi come un oggetto
verbale di meditazione da ripetere in continuazione. Ogni oggetto
diventa immediatamente fonte di visione profonda. Utilizzate la mente,
che è in stabile \emph{samādhi}, per investigare la natura impermanente
di ciascun oggetto. Ogni volta che si verifica un contatto con i sensi,
richiamate la riflessione: «~Non è sicuro.~» Oppure: «~Questo è
impermanente.~» Se siete catturati dall'illusione e credete nell'oggetto
sperimentato, soffrite, perché tutti questi dhamma (fenomeni)
sono non-sé (\emph{anattā}). Se vi attaccate a qualcosa che è non-sé e
lo percepite erroneamente come sé, esso diventa automaticamente causa di
dolore e di afflizione. Questo avviene perché vi attaccate a percezioni
sbagliate.

Esaminate ripetutamente la Verità, in continuazione, fino a che
comprendete con chiarezza che tutti questi oggetti dei sensi sono privi
di qualsiasi vera essenza. Non appartengono ad alcun sé. Perché dovreste
allora fraintendere, e attaccarvi a essi come se fossero un ``io''
oppure a un ``io'' appartenessero? È qui che dovete ulteriormente
sforzarvi, riflettere in continuazione sulla Verità. Le cose non sono
davvero voi, e non vi appartengono. Perché continuate a fraintenderle,
come se fossero un sé? Nessuno di questi oggetti dei sensi può essere
considerato in senso assoluto come se fosse voi stessi. Perché allora
riescono a ingannarvi e a farsi considerare come un sé? In verità, non è
in alcun modo possibile che sia così. Tutti gli oggetti dei sensi sono
impermanenti. Perché li vedete come permanenti? È incredibile come
riescano a ingannarvi. Il corpo è intrinsecamente non attraente. Com'è
possibile che vi attacchiate all'opinione che sia qualcosa di attraente?
Queste verità supreme -- la natura non attraente del corpo e l'assenza
di un sé in tutte le formazioni -- diverranno ovvie con
l'investigazione, e alla fine vedrete che questa cosa che chiamiamo
mondo è in realtà un'illusione generata da questi errati modi di vedere.

Quando utilizzate la meditazione di visione profonda per investigare le
Tre Caratteristiche\footnote{Tre Caratteristiche (\emph{tilakkhaṇa}): Le
  qualità di tutti i fenomeni; impermanenza (\emph{anicca}), carattere
  insoddisfacente (\emph{dukkha}) e non-sé (\emph{anatta}).} e penetrate
la vera natura dei fenomeni, non è necessario fare alcunché di speciale.
Non c'è bisogno di andare agli estremi. Non rendetevi le cose difficili.
Focalizzate in modo diretto la vostra consapevolezza, come se foste
seduti ad accogliere degli ospiti che entrano in una sala d'attesa.
Nella vostra sala d'attesa c'è una sola sedia, così che i vari ospiti
che giungono nella stanza per incontrarvi non possono sedersi perché voi
state già occupando l'unica sedia disponibile. Se un visitatore entra
nella stanza, voi sapete subito chi è. Perfino se due, tre o più
visitatori entrano contemporaneamente nella stanza, voi sapete
immediatamente chi sono, perché non hanno alcun luogo in cui sedersi.
Voi occupate l'unica sedia disponibile e, così, ogni visitatore che
entra vi è noto e non può fermarsi a lungo.

Potete osservare tutti i visitatori mentre voi siete a vostro agio, ma
loro non possono sedersi da nessuna parte. Fissate la vostra
consapevolezza sull'investigazione delle Tre Caratteristiche
dell'impermanenza, della sofferenza e del non-sé, e mantenete
l'attenzione su questa contemplazione senza consentire alla mente di
andare altrove. La visione profonda nella transitorietà, nel carattere
insoddisfacente e nella natura priva di un sé di tutti fenomeni cresce
costantemente e diventa più chiara e inclusiva. La comprensione si fa
più profonda. Una tale chiarezza della visione profonda conduce a una
serenità che penetra più a fondo nel cuore di qualsiasi altra
tranquillità che potreste sperimentare durante la pratica di
\emph{samatha}. È la chiarezza e la completezza di questa visione
profonda nel modo in cui sono le cose che ha l'effetto di purificare la
mente. È la saggezza che sorge quale risultato di un'intensa,
cristallina e chiara visione profonda ad agire come agente di
purificazione.

Per mezzo di ripetuti esami e contemplazioni della Verità, col
trascorrere del tempo i vostri modi di vedere cambiano e quel che in
precedenza avete erroneamente percepito come attraente perde
gradualmente il suo fascino man mano che affiora la verità sulla sua
natura non attraente. Investigate i fenomeni per vedere se hanno una
natura davvero permanente, oppure transitoria. All'inizio vi limitate a
ripetere l'insegnamento dell'impermanenza dei fenomeni condizionati, ma
in seguito vedete effettivamente la Verità con chiarezza grazie alla
vostra stessa investigazione. La Verità attende di essere trovata
proprio nel punto in cui investigate. Questa è la sedia sulla quale
attendete di accogliere i visitatori. Non potreste andare in nessun
altro posto per sviluppare la visione profonda. Dovete restare seduti
proprio qui: su quell'unica sedia presente nella stanza.

Quando i visitatori entrano nella sala d'accoglienza, è facile osservare
il loro aspetto e il modo in cui si comportano, perché non possono
sedersi. Dovete inevitabilmente conoscerli tutti. In altre parole, si
giunge a una chiara e distinta comprensione della natura impermanente,
insoddisfacente e priva di sé di tutti questi fenomeni, e questa visione
profonda diviene così indiscutibile e stabile nella vostra mente da
porre fine a qualsiasi restante incertezza sulla vera natura delle cose.
Sapete per certo che non è possibile alcun altro modo di vedere
l'esperienza. Questa è la realizzazione del Dhamma al livello più
profondo. Infine, la vostra meditazione implica il sostegno della
conoscenza, cui segue un continuo lasciar andare man mano che
sperimentate gli oggetti dei sensi tramite gli occhi, gli orecchi, il
naso, la lingua, il corpo e la mente. Coinvolge solo questo, e non c'è
bisogno di fare altro.

È importante sforzarsi ripetutamente per sviluppare la visione profonda
mediante l'investigazione delle Tre Caratteristiche. Tutto può diventare
una causa per il sorgere della saggezza, ed essa è ciò che distrugge
completamente ogni forma di contaminazione e di attaccamento. Questo è
il frutto della meditazione \emph{vipassanā}. Non pensiate che tutto
quel che fate provenga dalla visione profonda. A volte vi comportate
seguendo i vostri desideri. Se state ancora praticando seguendo i vostri
desideri, allora vi impegnerete solo nei giorni in cui vi sentite pieni
di energia e ispirati, e non farete meditazione nei giorni in cui vi
sentite pigri. Questo si chiama praticare sotto l'influsso delle
contaminazioni. Significa che non avete alcun reale potere sulla vostra
mente e che seguite solo i vostri desideri. Quando la vostra mente è
allineata con il Dhamma, non c'è nessuno che è diligente e nessuno che è
pigro. Dipende dal modo in cui la mente è addestrata. La pratica della
visione profonda continua a fluire in modo automatico, indipendentemente
dalla pigrizia o dalla diligenza. È uno stato che si sostiene da sé, il
cui carburante è la sua stessa energia. Quando la mente ha queste
caratteristiche, significa che non dovete più essere colui che svolge la
pratica. Potete dire che è come se aveste finito tutto il lavoro che
avete fatto e che l'unica cosa che resta da fare è lasciare le cose a se
stesse, e sorvegliare la mente. Non c'è più bisogno di essere qualcuno
che fa qualcosa. C'è ancora attività mentale -- sperimentate piacevoli e
spiacevoli contatti con i sensi in accordo con le accumulazioni dei
vostri kamma -- ma la considerate per ``quello che sono'' e,
nello stesso tempo, c'è sempre il lasciar andare l'attaccamento alla
concezione dell'io.

A questo punto non state creando alcun senso del sé e, perciò, non state
creando alcuna sofferenza. Alla fine tutti gli oggetti dei sensi e gli
stati mentali che sperimentate nella mente hanno lo stesso valore.
Qualsiasi fenomeno mentale o fisico esaminiate appare uguale a tutto il
resto, hanno tutti le stesse qualità intrinseche. Tutti i fenomeni
divengono un'unica, stessa cosa. La vostra saggezza deve svilupparsi
fino a questo punto affinché nella mente tutte le incertezze giungano al
termine. Quando iniziate a meditare, è come se tutto quel che sapete
fare è dubitare e indagare le cose. La mente ondeggia e vacilla in
continuazione. Trascorrete tutto il tempo tra pensieri agitati e
proliferazioni mentali sulle cose. Avete dubbi su tutto. Perché? A causa
dell'impazienza. Volete conoscere tutte le risposte, e subito. Volete
ottenere in fretta la visione profonda, senza che sia necessario fare
nulla. Volete conoscere la verità sul modo in cui sono le cose, ma nella
mente quel desiderio è così forte da essere più potente della visione
profonda che desiderate. Per questa ragione la pratica deve svilupparsi
per tappe. Dovete fare un passo alla volta. In primo luogo c'è bisogno
di persistere nello sforzo. Avete anche bisogno del continuo supporto
delle vostre buone azioni del passato e di sviluppare le Dieci
Perfezioni spirituali (\emph{pāramī}).

Continuate a suscitare energia nell'addestramento della mente. Non
restate intrappolati nel desiderare risultati veloci. Allorché i frutti
della visione profonda tarderanno ad arrivare, ciò vi condurrà solo alla
delusione e alla frustrazione. Pensare in questo modo non vi aiuterà. È
giusto attendersi di sperimentare una qualche condizione permanente,
quando si prova ancora piacere o dolore? Non importa cosa la mente vi
vomiti addosso. Quando siete sopraffatti dal piacere e dal dolore per lo
stimolo dovuto al contatto tra la mente e i vari oggetti dei sensi, non
potete avere alcuna idea di quale livello la pratica abbia raggiunto.
Però, in breve tempo tali stati mentali perdono potere sulla mente. In
verità, l'impatto può essere di beneficio, perché vi rammenta di
esaminare la vostra esperienza. Si arriva a conoscere quali reazioni
richiamano alla mente gli oggetti dei sensi, i pensieri e le percezioni
che sperimentate. Lo sapete, sia nel caso in cui conducono la mente
verso l'agitazione e la sofferenza sia quando la fanno muovere poco o
nulla. Alcuni meditanti vogliono solo avere la visione profonda sul modo
in cui la mente è influenzata dagli oggetti piacevoli. Vogliono
investigare solo gli stati mentali positivi. In questo modo non
otterranno la vera visione profonda. Non sono molto intelligenti.
Davvero, dovete esaminare anche cosa avviene quando sperimentate un
impatto spiacevole con gli oggetti dei sensi. Dovete conoscere quel che
fanno alla mente. Così dovete addestrarvi.

È pure importante comprendere che quando è in questione la pratica
stessa, non c'è bisogno di frugare tra le esperienze del passato e il
cumulo di memorie disponibili nelle fonti esterne, perché quel che conta
è la vostra stessa esperienza. Il solo modo per porre davvero fine ai
vostri dubbi e alle vostre congetture è la pratica, finché raggiungete
il punto nel quale vedete da voi stessi con chiarezza i risultati.
Questa è la cosa più importante. Imparare da vari maestri è un
preliminare essenziale. È un valido supporto allorché dall'ascolto degli
insegnamenti vi spostate a imparare dalla vostra stessa esperienza.
Dovete contemplare gli insegnamenti che ricevete alla luce della vostra
pratica, fino a quando ottenete una vostra propria comprensione. Se già
avete alcune qualità spirituali e virtù accumulate in passato, la
pratica sarà più lineare. In genere i consigli degli altri possono farvi
risparmiare tempo, aiutandovi a evitare errori e ad andare dritti al
cuore della pratica. Se cercate di praticare da soli senza alcuna guida,
seguirete un sentiero più lento e con più deviazioni. Se cercate di
scoprire il corretto modo di praticare completamente da soli, avrete la
tendenza a sprecare del tempo e a percorrere la strada più lunga. Questa
è la verità.

Alla fine, la pratica del Dhamma è il modo più sicuro per far appassire
e svanire tutti i dubbi e tutte le esitazioni. Man mano che continuate a
sforzarvi e addestrarvi per andare controcorrente rispetto alle vostre
contaminazioni, i dubbi avvizziranno e moriranno. Se ci pensate, avete
già ottenuto parecchio dai vostri sforzi nella pratica. Avete fatto
progressi, ma non è ancora abbastanza per farvi sentire del tutto
soddisfatti. Se guardate attentamente e riflettete sulla vostra vita, da
quando siete nati passando per la vostra giovinezza fino a oggi, potete
capire quante cose avete sperimentato del mondo attraverso la vostra
mente. In passato non vi stavate addestrando nella virtù, nella
concentrazione e nella saggezza, ed è facile vedere fino a che punto le
contaminazioni si fossero impossessate di voi. Quando vi voltate
indietro a guardare tutto ciò di cui avete avuto esperienza per mezzo
dei sensi, risulta ovvio che avete sperimentato in innumerevoli
occasioni la verità del modo in cui sono le cose. Contemplare quello che
vi è successo durante la vita, aiuta a illuminare la mente, a
consentirle di vedere che le contaminazioni non la soverchiano
completamente e con la stessa densità di prima.

Ogni tanto bisogna che vi incoraggiate in questo modo. Porta via un po'
di pesantezza. Ovviamente, non è cosa saggia solo lodarsi e
incoraggiarsi. Nell'addestrare la mente, di tanto in tanto dovete
rimproverarvi. Talvolta dovete forzarvi a fare cose che non volete fare,
ma non spingete sempre in modo eccessivo la vostra mente, fino al
limite. Quando vi addestrate nella meditazione, è normale che il corpo
-- che è un fenomeno condizionato -- sia soggetto alla tensione, al
dolore e a numerosi problemi allorché le condizioni hanno un impatto su
di esso. È del tutto normale che il corpo sia così. Più vi addestrate
nella meditazione seduta, più diventate abili in essa e, ovviamente, più
a lungo riuscite a stare seduti. Inizialmente ci riuscivate solo per
cinque minuti prima di dovervi alzare. Però, man mano che praticate di
più, il tempo durante il quale potete sedere comodamente cresce da dieci
a venti minuti, a mezz'ora, finché, alla fine, riuscite a stare seduti
per un'ora intera senza alzarvi. Gli altri vi guardano e vi lodano
perché siete in grado di stare seduti così a lungo, ma voi potreste
avere la sensazione di non riuscire ancora a sedere a lungo. Questo è il
modo in cui il desiderio di ottenere risultati può influire su di voi
durante la meditazione.

Un altro aspetto importante dell'addestramento è sostenere uniformemente
la pratica della consapevolezza in tutte le quattro posture: in piedi,
camminando, sedendo e stando distesi. Fate attenzione a non pensare --
sbagliando -- di praticare davvero solo quando sedete nella postura
della meditazione formale. Non consideratela come l'unica postura per
coltivare la consapevolezza. È un errore. È molto probabile perfino che
la calma e la visione profonda possano sorgere non durante una seduta di
meditazione formale. Anche se state seduti in meditazione per molte ore
in un solo giorno, dovete addestrarvi alla consapevolezza costantemente,
quando passate da una postura all'altra, e sviluppare una continua
presenza mentale. Tutte le volte che perdete la consapevolezza, cercate
di ristabilirla appena possibile e mantenetela con tutta la continuità
che potete. Questa è la maniera di ottenere celeri progressi. La visione
profonda arriva velocemente. Così si diventa saggi. Saggi per quanto
concerne la comprensione degli oggetti dei sensi e il modo in cui essi
esercitano un influsso sulla mente. Utilizzate questa saggezza per
comprendere i vostri stati mentali e per addestrare la mente a lasciar
andare. Così dovreste intendere la coltivazione della mente. Anche se
siete distesi per dormire, dovete fissare l'attenzione sulle
inspirazioni e sulle espirazioni fino a quando vi addormentate, e
continuate in questo modo appena vi svegliate. Così c'è solo un breve
periodo, quello durante il quale dormite profondamente, privo della
pratica della presenza mentale. Dovete impiegare tutta la vostra energia
nell'addestramento.

Quando avete sviluppato la consapevolezza, più vi addestrate più la
mente sperimenta uno stato di veglia, fino a che vi sembra di non
dormire affatto. È solo il corpo a dormire, la mente resta consapevole.
La mente resta sveglia e vigile anche quando il corpo dorme. Restate
sempre con la conoscenza. Appena vi svegliate, la consapevolezza è lì
fin dal primo momento in cui la mente abbandona il sonno e
immediatamente assume un oggetto di meditazione. Siete attenti e vigili.
Dormire è davvero una funzione corporea. Comporta il riposo del corpo.
Il corpo si prende il riposo di cui ha bisogno, ma è ancora presente la
conoscenza che vigila sulla mente. La consapevolezza è sostenuta durante
tutto il giorno e durante tutta la notte. Così, anche se siete distesi e
vi mettete a dormire, è come se la mente non dormisse. Però non vi
sentite spossati né avvertite la necessità di dormire di più. Restate
allerta e attenti. È per questa ragione che difficilmente si sogna
quando si pratica davvero. Se sognate, è nella forma di un \emph{supina
nimitta}, un sogno insolitamente chiaro e vivido che assume un qualche
significato particolare. In genere, ovviamente, sognate molto poco.
Quando vigilate sulla mente è come se non ci fossero cause per la
proliferazione mentale che è il propellente dei sogni. Restate in una
condizione nella quale non siete catturati dall'illusione. Sostenete la
consapevolezza, essa è presente in profondità nella mente. La mente è in
uno stato di vigilanza, acuta e pronta. La presenza di un'ininterrotta
consapevolezza rende la mente capace di investigare in modo lineare e
senza sforzo, e la mente di momento in momento tiene il passo con tutto
quello che in essa sorge.

Dovete coltivare la mente fino a quando diventa totalmente fluida e
abile nel mantenere la consapevolezza e nell'investigare i fenomeni.
Tutte le volte che la mente raggiunge uno stato di calma, addestratela a
esaminare il vostro corpo e quello degli altri finché avete visione
profonda sufficiente per vedere le comuni caratteristiche dei corpi.
Procedete con l'investigazione fino a quando vedete che tutti i corpi
hanno essenzialmente la stessa natura e provengono dagli stessi elementi
materiali. Dovete continuare a osservare e a contemplare. Di notte,
prima di andare a dormire, utilizzate la consapevolezza per spaziare su
tutto il corpo e ripetete la contemplazione appena vi svegliate al
mattino. In questo modo non avrete incubi, non parlerete durante i sogni
né sarete coinvolti in molti di essi. Dormite e vi svegliate
tranquillamente senza che nulla vi dia fastidio. Sostenete lo stato di
conoscenza durante il sonno e appena vi svegliate. Quando vi svegliate
con consapevolezza, la mente è luminosa, chiara e non viene disturbata
dalla sonnolenza. Quando vi svegliate la mente è radiosa, perché è
libera dal torpore e dagli umori condizionati dalle contaminazioni.

Vi ho offerto dei dettagli sullo sviluppo della mente durante la
pratica. Di solito non pensereste che sia possibile per la mente essere
davvero serena mentre si sta dormendo, appena ci si sveglia oppure in
altre situazioni nelle quali ci si attenderebbe che la consapevolezza
sia debole. Ad esempio, è possibile stare seduti in un bagno di sudore
dopo aver camminato nel bel mezzo di una tempesta, ma poiché abbiamo
coltivato il \emph{samādhi} e abbiamo imparato a contemplare, la mente
non viene toccata da umori contaminati ed è ancora in grado di
sperimentare la pace e la chiara visione profonda nel modo in cui ve
l'ho descritta.

L'ultimo insegnamento che il Buddha impartì alla comunità monastica fu
un'esortazione a non essere catturati dalla distrazione. Disse che la
distrazione è la via che conduce alla morte. Per favore comprendetelo, e
prendetelo a cuore nel modo più sincero che potete. Addestratevi a
pensare con saggezza. Avvaletevi della saggezza per guidare le vostre
parole. Qualsiasi cosa facciate, avvaletevi della guida della saggezza.

