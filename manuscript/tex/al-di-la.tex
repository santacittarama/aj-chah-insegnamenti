\chapter{Al di là}

\begin{openingQuote}
  \centering

  Versione abbreviata di un discorso tenuto al Wat Nong Pah Pong nel 1978 per il
  Chief Privy Councillor della Thailandia, Sanya~Dharmasakti.
\end{openingQuote}

È molto importante praticare il Dhamma. Se non pratichiamo, tutta la
nostra conoscenza è solo una conoscenza superficiale, è solo un
involucro esterno. È come avere un frutto senza averlo ancora mangiato.
Anche se quel frutto lo teniamo in mano, non ne traiamo alcun beneficio.
Solo per mezzo dell'effettivo atto di mangiarlo potremo conoscerne il
sapore. Il Buddha non elogiò chi crede agli altri, elogiò chi conosce se
stesso. Proprio come succede con quel frutto: se l'abbiamo già
assaggiato, non abbiamo bisogno di chiedere ad altri se è dolce o aspro.
I nostri problemi sono finiti. Perché sono finiti? Perché vediamo
secondo verità. Chi ha compreso il Dhamma è come chi ha compreso la
dolcezza o l'asprezza del frutto. È proprio lì che tutti i dubbi sono
finiti.

Quando parliamo del Dhamma, anche se si potrebbe dire molto, tutto si
può ridurre a quattro cose. Bisogna solo conoscere la sofferenza,
conoscere la causa della sofferenza, conoscere la fine della sofferenza
e conoscere il Sentiero della pratica che conduce alla fine della
sofferenza. Questo è quanto. Tutto quello che abbiamo sperimentato sul
Sentiero della pratica si riduce a queste quattro cose. Quando le
conosciamo, i nostri problemi sono finiti.

Dove sono nate queste cose? Sono nate proprio nel corpo e nella mente,
in nessun altro posto. Allora perché l'insegnamento del Buddha è così
dettagliato e vasto? Per spiegare queste cose in modo più sottile, per
aiutarci a vederle. Quando Siddhattha Gotama\footnote{Siddhattha Gotama:
  Il nome proprio del Buddha storico; nei testi canonici più antichi si
  menziona il Buddha solo con il nome di Gotama.} nacque nel mondo,
prima di vedere il Dhamma era una persona ordinaria proprio come noi.
Quando apprese ciò che doveva conoscere, ossia la verità della
sofferenza, la causa e la fine di essa, e il Sentiero che conduce alla
fine della sofferenza, comprese il Dhamma e divenne un Buddha
perfettamente illuminato.

Dopo aver compreso il Dhamma, ovunque sediamo conosciamo il Dhamma e
ovunque ci troviamo ascoltiamo l'insegnamento del Buddha. Quando
comprendiamo il Dhamma, il Buddha è nella nostra mente, il Dhamma è
nella nostra mente e la pratica che conduce alla saggezza è proprio
nella nostra mente. Avere il Buddha, il Dhamma e il Saṅgha nella nostra
mente significa che quando le nostre azioni sono buone o cattive,
conosciamo chiaramente da noi stessi la loro vera natura.

Così il Buddha abbandonò le opinioni del mondo, la lode e il biasimo.
Quando la gente lo lodava o lo biasimava, Egli lo accettava solo per
quello che era. Tali due cose sono semplicemente condizioni mondane e,
perciò, egli non ne era turbato. Perché? Perché conosceva la sofferenza.
Sapeva che se avesse creduto in quella lode o in quel biasimo, ciò gli
avrebbe causato sofferenza.

Quando la sofferenza sorge, ci agita e ci sentiamo a disagio. Qual è la
causa della sofferenza? È che non conosciamo la Verità, questa è la
causa. Quando la causa è presente, sorge allora la sofferenza. Una volta
che è sorta, non sappiamo come fermarla. Più cerchiamo di fermarla, più
ne arriva. Diciamo: «~non mi criticare~», «~non mi biasimare.~» Cercando
di fermarla in questo modo, la sofferenza aumenta davvero, non si ferma.

Per questa ragione il Buddha ci insegnò che percorrere la via che
conduce alla fine della sofferenza significa far sorgere la realtà del
Dhamma nella nostra mente. Diventiamo persone che testimoniano il Dhamma
in se stessi. Se qualcuno dice che siamo buoni, non ci perdiamo in
questo; se dicono che non siamo buoni, non ci dimentichiamo di noi
stessi. Così possiamo essere liberi. ``Bene'' e ``male'' sono solo
dhamma mondani,\footnote{\emph{Dhamma} mondani: Le otto
  condizioni mondane di guadagno e perdita, lode e biasimo, felicità e
  sofferenza, fama e discredito.} sono solo stati mentali. Se li
seguiamo, la nostra mente diventa il mondo, brancoliamo solo
nell'oscurità senza conoscere la via d'uscita.

Se è così, allora non abbiamo ancora imparato a essere padroni di noi
stessi. Cerchiamo di sconfiggere gli altri ma, così facendo,
sconfiggiamo solo noi stessi. Se invece siamo padroni di noi stessi,
abbiamo allora la padronanza di tutto, di ogni stato mentale, della
vista, dei suoni, degli odori, dei sapori e delle sensazioni tattili.
Ora sto parlando di cose esteriori, esse è così che sono, ma l'esterno
si riflette anche interiormente. Alcune persone conoscono solo
l'esterno, non conoscono l'interno. Come quando diciamo di «~vedere il
corpo nel corpo.~» Vedere l'esteriorità del corpo non basta, dobbiamo
conoscere il corpo dentro il corpo. Poi, dopo aver investigato la mente,
dovremmo conoscere la mente dentro la mente.

Perché dovremmo investigare il corpo? Che cos'è questo ``corpo nel
corpo''? Quando diciamo di conoscere la mente, che cos'è questa
``mente''? Se non conosciamo la mente, non conosciamo le cose dentro la
mente. Questo significa non conoscere la sofferenza, non conoscere la
causa, non conoscere la fine e non conoscere il Sentiero che conduce
alla fine della sofferenza. Le cose che dovrebbero aiutarci a estinguere
la sofferenza non ci aiutano, perché siamo distratti dalle cose che la
rendono più gravosa. È proprio come se ci prudesse la testa e ci
grattassimo una gamba! Se è la testa a pruderci, allora è ovvio che non
proveremo molto sollievo. Allo stesso modo, quando sorge la sofferenza
non sappiamo come affrontarla, non conosciamo la pratica che conduce
alla fine della sofferenza.

Prendiamo come esempio questo corpo, proprio il corpo che ognuno di noi
ha portato qui con sé. Se vediamo solo la forma del corpo non v'è modo
di fuggire dalla sofferenza. Perché no? Perché ancora non vediamo
l'interno del corpo, vediamo solo l'esterno. Lo vediamo solo come
qualcosa di bello, qualcosa di sostanziale. Il Buddha disse che vedere
solo questo non basta. Con i nostri occhi vediamo l'esterno: può
riuscirci pure un bambino, anche gli animali possono vederlo, non è
difficile. L'esterno del corpo si vede facilmente, ma dopo averlo visto
ci restiamo invischiati, non ne conosciamo la verità. Dopo averlo visto
lo afferriamo, ed esso ci morde!

È per questo motivo che dovremmo investigare il corpo nel corpo.
Qualsiasi cosa ci sia nel corpo -- forza! -- guardiamola. Se guardiamo
solo l'esterno, non si capisce. Guardiamo i capelli, le unghie e così
via, e vediamo solo belle cose che ci seducono. Perciò il Buddha ci
insegnò a vedere l'interno del corpo, a vedere il corpo nel corpo. Che
cosa c'è nel corpo? Guardateci dentro con attenzione! Troveremo molte
sorprese, perché non le abbiamo mai viste, anche se sono dentro di noi.
Ovunque si vada le portiamo con noi. Le portiamo con noi quando siamo
seduti in un'automobile, ma non le conosciamo affatto!

È come se andassimo a trovare alcuni parenti a casa loro e ricevessimo
un dono. Lo prendiamo e lo mettiamo nella nostra borsa, poi ce ne
andiamo senza averlo aperto per vedere che cosa c'è dentro. Quando
infine lo apriamo, è pieno di serpenti velenosi! Così è il nostro corpo.
Se vediamo solo l'involucro, diciamo che va bene, che è bello. Ci
dimentichiamo di noi stessi. Dimentichiamo l'impermanenza, la
sofferenza e il non-sé. Se ci guardiamo dentro, questo corpo è davvero
repellente.

Quando osserveremo in accordo con la realtà, senza cercare di addolcire
le cose, vedremo che tutto è davvero penoso e stancante. Sorgerà
distacco. Questa sensazione di ``disinteresse'' non significa provare
avversione nei riguardi del mondo o di qualcosa; mettiamo solo ordine
nella nostra mente, e la nostra mente lascia andare. Vediamo le cose
come non sostanziali e non affidabili, le vediamo così come sono. Per
quanto vogliamo che siano in un certo modo, esse senza curarsene si
limitano ad andare per la loro strada. Se ridiamo o piangiamo, sono
semplicemente così come sono. Le cose che sono instabili sono instabili,
le cose che non sono belle sono non belle.

Per questa ragione il Buddha disse che quando vediamo delle cose, quando
sperimentiamo suoni, sapori, odori, sensazioni tattili o stati mentali,
dovremmo lasciarli liberi di andare. Quando l'orecchio sente dei suoni,
lasciateli andare. Quando il naso percepisce un odore, lasciatelo
andare, abbandonatelo al naso! Quando sorgono sensazioni tattili,
lasciate andare il piacere o il dispiacere che ne consegue, lasciatelo
tornare nel luogo in cui è nato. Lo stesso vale per gli stati mentali.
Tutte queste cose lasciatele andare per la loro strada. Questo è
conoscere. Che si tratti di felicità o infelicità, è lo stesso. Questo è
fare meditazione.

Fare meditazione significa rendere la mente serena per consentire alla
saggezza di sorgere. Ciò richiede che si pratichi con il corpo e con la
mente per vedere e conoscere sia le impressioni sensoriali legate a
forma, suono, sapore, odore e tatto sia gli stati mentali. In poche
parole, è solo questione di felicità e infelicità. La felicità è solo
una sensazione mentale piacevole, l'infelicità è solo una sensazione
mentale spiacevole. Il Buddha insegnò a separare la felicità e
l'infelicità dalla mente. La mente è ciò che conosce. La
sensazione\footnote{\emph{Vedanā} in pāli; si veda il \emph{Glossario}, p. \pageref{glossary-vedana}.}
di piacere o dispiacere è la caratteristica della felicità o
dell'infelicità. Quando la mente indulge a queste cose, diciamo che si
aggrappa o che ritiene tale felicità o infelicità degne di essere
trattenute. Questo aggrapparsi è un'azione della mente; felicità o
infelicità sono sensazioni.

Affermiamo che il Buddha ci ha detto di separare la mente dalla
sensazione, ma Egli non voleva dire che, alla lettera, le dovremmo
mettere in due posti diversi. Voleva dire che la mente deve conoscere la
felicità e conoscere l'infelicità. Ad esempio, quando sediamo in
\emph{samādhi}\footnote{\emph{Samādhi:} Concentrazione, unificazione
  della mente, stabilità mentale.} e la pace riempie la mente, la
felicità arriva ma non ci raggiunge, anche l'infelicità arriva ma non ci
raggiunge. È così che si separa la sensazione dalla mente. Lo possiamo
paragonare all'olio e all'acqua in una bottiglia. Non si uniscono.
Potete anche cercare di mescolarli, ma l'olio rimane olio e l'acqua
rimane acqua, perché la loro densità è diversa.

Lo stato naturale della mente non è la felicità né l'infelicità. Quando
la sensazione entra nella mente, ecco che nasce la felicità o
l'infelicità. Se abbiamo presenza mentale, conosciamo la sensazione
piacevole come sensazione piacevole. La mente che conosce non l'afferra.
La felicità è lì, ma è fuori della mente, non sotterrata nella mente. La
mente la conosce con semplice chiarezza.

Se separiamo l'infelicità dalla mente, questo forse significa che non
c'è sofferenza, che non la sperimentiamo? Sì, la sperimentiamo, ma
conosciamo la mente come mente e la sensazione come sensazione. Non ci
aggrappiamo a quella sensazione e non ce la portiamo appresso. Il Buddha
separò queste cose mediante la conoscenza. Egli soffrì? Conobbe lo stato
della sofferenza ma non si aggrappò a esso; per questo diciamo che egli
interruppe la sofferenza. Ci fu anche felicità, ma egli quella felicità
la conobbe; se non è conosciuta, è come un veleno. Egli non s'identificò
con essa. La felicità non esisteva nella sua mente, era lì grazie alla
conoscenza. Così, noi diciamo che Egli separò felicità e infelicità
dalla sua mente.

Quando affermiamo che il Buddha e gli Esseri Illuminati uccisero le
contaminazioni, non è che le uccisero realmente. Se avessero ucciso
tutte le contaminazioni, allora noi probabilmente non ne avremmo alcuna.
Non uccisero le contaminazioni: quando le conobbero per quello che sono,
le lasciarono andare. Qualche sciocco le afferrerà, ma gli Esseri
Illuminati conobbero le contaminazioni nelle loro menti come un veleno
e, così, le spazzarono via. Spazzarono via le cose che causavano loro
sofferenza, non le uccisero. Chi non lo sa, considererà un bene alcune
cose, come la felicità, e poi le afferrerà, ma il Buddha le conosceva e
si limitò a spazzarle via.

Quando la sensazione sorge in noi, indulgiamo a essa: la mente porta con
sé felicità e infelicità. Nei fatti si tratta di due cose diverse. Le
attività della mente, sensazione piacevole, sensazione spiacevole e così
via, sono impressioni mentali, sono il mondo. Se la mente lo sa, può
lavorare nello stesso modo con la felicità e con l'infelicità. Perché?
Perché sa la verità su queste cose. Chi non le conosce, le vede come se
avessero un valore diverso, ma chi le conosce le considera uguali. Se vi
aggrappate alla felicità, in seguito essa sarà il luogo di nascita
dell'infelicità, perché la felicità è instabile, cambia in
continuazione. Quando la felicità scompare, sorge l'infelicità.

Il Buddha lo sapeva, perché sia la felicità sia l'infelicità sono
insoddisfacenti, hanno lo stesso valore. Quando la felicità sorgeva, la
lasciava andare. La sua era retta pratica, Egli vedeva che entrambe
queste cose hanno uguali vantaggi e svantaggi. Sono sottoposte alla
Legge del Dhamma, cioè sono instabili e insoddisfacenti. Dopo essere
nate, muoiono. Quando Egli lo comprese, sorse la Retta Visione\footnote{Retta
  Visione (\emph{Sammā-diṭṭhi}): La Retta Visione è il primo fattore del
  Nobile Ottuplice Sentiero.} e il retto modo di praticare divenne
chiaro. Non importa quale tipo di sensazione o di pensiero sorgesse
nella sua mente, Egli semplicemente sapeva che faceva parte
dell'altalena continua di felicità e infelicità. Non si aggrappava a
esse.

Quando il Buddha aveva da poco conseguito l'Illuminazione, tenne un
sermone sull'indulgenza al piacere e sull'indulgenza al dolore.
«~Monaci! Indulgere al piacere è la via del lassismo, indulgere al
dolore è la via della tensione.~» Queste erano le due cose che avevano
ostacolato la sua pratica fino al giorno in cui divenne l'Illuminato,
perché prima non le aveva lasciate andare. Quando le conobbe, le lasciò
andare e fu così in grado di pronunciare il suo primo sermone.

Per questo diciamo che un meditante non dovrebbe percorrere la via della
felicità o dell'infelicità, bensì conoscerle. Conoscendo la verità della
sofferenza, conoscerà la causa della sofferenza, la fine della
sofferenza e il Sentiero che conduce alla fine della sofferenza. E la
via d'uscita dalla sofferenza è la stessa meditazione. Detto in modo
semplice, dobbiamo avere consapevolezza. Consapevolezza è conoscere,
presenza mentale. Proprio ora, che cosa stiamo pensando, che cosa stiamo
facendo? Che cosa abbiamo con noi, proprio ora? Se osserviamo in questo
modo, siamo consapevoli di come stiamo vivendo. Praticando in questo
modo, può sorgere la saggezza. Riflettiamo e investighiamo sempre, in
tutte le posture. Quando sorge un'impressione mentale che ci piace, la
conosciamo per quello che è, non riteniamo che sia nulla di sostanziale.
È solo felicità. Quando sorge l'infelicità, sappiamo che essa è
indulgenza al dolore, che non è il Sentiero di un meditante.

Quando diciamo di separare la mente dalla sensazione, è questo che
intendiamo. Se siamo abili non ci aggrappiamo, lasciamo che le cose
siano. Diventiamo ``Colui che Conosce''.\footnote{Colui che Conosce. La
  qualità della presenza mentale, quella facoltà della mente che, se
  rettamente coltivata, conduce alla Liberazione.} Mente e sensazione
sono come acqua e olio; sono nella stessa bottiglia, ma non si
mescolano. Conosciamo la sensazione come sensazione e la mente come
mente perfino se siamo ammalati o addolorati. Conosciamo gli stati
dolorosi e quelli piacevoli, ma non ci identifichiamo con essi.
Dimoriamo solo nella pace, in quella pace che è al di là del piacere e
del dolore.

È così che dovreste pensare, perché altrimenti, visto non c'è un sé
permanente, non c'è rifugio. Dovete vivere in questo modo, senza
felicità e senza infelicità. Dovete stare solo con il conoscere, senza
portarvi dietro le cose. Fino a che non conseguiremo l'Illuminazione,
tutto questo può suonare strano, ma non importa: impostiamo il nostro
obiettivo in questa direzione. La mente è la mente. Essa incontra
felicità e infelicità e noi le vediamo solo in quanto tali, non c'è
nulla di più in esse. Sono separate, non mescolate. Se fossero mescolate
non potremmo conoscerle. È come vivere in una casa; la casa e i suoi
occupanti sono collegati, ma separati. Se la nostra casa è in pericolo,
siamo angosciati perché dobbiamo proteggerla, ma se la casa va in fiamme
dobbiamo uscire. Quando sorgono sensazioni dolorose usciamo, proprio
come usciremmo da quella casa. Se è piena di fuoco e noi lo sappiamo,
usciamo di corsa. Sono cose separate; la casa è una cosa, l'occupante
un'altra.

Diciamo che separiamo la mente dalle sensazioni in questo modo, ma nei
fatti sono già separate per natura. La nostra comprensione consiste
semplicemente nel conoscere questa differenza naturale e in accordo con
la realtà. Quando diciamo che non sono separate è perché ci stiamo
aggrappando a esse per ignoranza della verità.

È per questo che il Buddha ci disse di meditare. Questa pratica di
meditazione è davvero importante. Conoscere solo per mezzo
dell'intelletto non basta. La conoscenza che sorge dalla pratica che
conduce a una mente serena e la conoscenza che proviene dallo studio
sono davvero lontane l'una dall'altra. La conoscenza che proviene dallo
studio non è vera conoscenza della nostra mente. La mente cerca di
aggrapparsi e di trattenere questa conoscenza. Perché cerchiamo di
trattenerla? Solo per perderla! E poi, quando è perduta, piangiamo.

Se davvero conosciamo, c'è il lasciar andare, lasciare che le cose
siano. Sappiamo come stanno le cose e non ci dimentichiamo di noi
stessi. Se avviene che ci ammaliamo, non ci perdiamo in questo. Alcuni
pensano: «~Quest'anno sono stato sempre malato, non sono affatto
riuscito a meditare.~» Queste sono davvero le parole di un folle. Se
pensiamo in questo modo le cose diventano difficili. Il Buddha non ci
insegnò in questo modo. Disse che proprio allora è il caso di meditare.
Quando siamo malati o quasi morenti, è allora che possiamo davvero
conoscere e vedere la realtà.

Altri dicono di non aver avuto la possibilità di meditare perché troppo
indaffarati. A volte vengono a trovarmi degli insegnanti scolastici.
Dicono di avere molte responsabilità e di non avere perciò tempo per la
meditazione. Io chiedo: «~Mentre state insegnando avete il tempo di
respirare?~» «~Sì~», rispondono. «~Allora come potete avere il tempo di
respirare se il lavoro è così frenetico e caotico? Da questo punto di
vista siete molto lontani dal Dhamma.~»

In verità questa pratica riguarda solo la mente e le sue sensazioni. Non
è una cosa da rincorrere o per cui sforzarsi. Il respiro continua mentre
si lavora. È la natura a occuparsi dei processi naturali, tutto quello
che dobbiamo fare è cercare di essere consapevoli. Continuate solo a
provare a entrare dentro di voi per vedere con chiarezza. Questa è
meditazione. Se abbiamo presenza mentale, quale che sia il lavoro che
stiamo svolgendo, esso sarà il giusto strumento per consentirci di
conoscere giusto e sbagliato in continuazione. C'è una gran quantità di
tempo per meditare: è che non comprendiamo appieno la pratica, questo è
tutto. Respiriamo mentre dormiamo, respiriamo mentre mangiamo. Lo
facciamo o no? Perché non abbiamo tempo di meditare? Respiriamo ovunque
siamo. Se pensiamo in questo modo, la nostra vita ha allora lo stesso
valore del nostro respiro. Abbiamo tempo ovunque siamo.

I pensieri, di qualsiasi genere essi siano, sono condizioni mentali, non
condizioni del corpo, e perciò abbiamo bisogno solo di avere presenza
mentale. In questo modo conosceremo sempre giusto e sbagliato. Stando in
piedi, camminando, seduti e distesi, c'è una gran quantità di tempo. È
solo che non sappiamo come usarlo proficuamente. Per favore, prendete in
considerazione quel che vi dico.

Non possiamo fuggire dalla sensazione, dobbiamo conoscerla. La
sensazione è solo sensazione, la felicità è solo felicità, l'infelicità
è solo infelicità. Sono semplicemente così. Perché dovremmo aggrapparci
a esse? Se la mente è abile, basta solo ascoltare questo per essere in
grado di separare la sensazione dalla mente. Se investighiamo in questo
modo in continuazione la mente si libererà. Non si tratta, però, di una
fuga indotta dall'ignoranza. La mente lascia andare, ma conosce. Non
lascia andare per stupidità o perché non vuole che le cose siano nel
modo in cui sono. Lascia andare perché conosce in accordo con la Verità.
Questo significa vedere la natura, la realtà che è tutt'intorno a noi.

Quando sappiamo questo, siamo abili con la mente, siamo abili con le
impressioni mentali. Quando siamo abili con le impressioni mentali siamo
abili con il mondo. Questo significa essere un ``Conoscitore del
mondo''. Il Buddha conobbe chiaramente il mondo con tutte le sue
difficoltà. Il mondo può confonderci moltissimo. Com'è che il Buddha fu
in grado di conoscerlo? Ora dovremmo capire che il Dhamma insegnato dal
Buddha non è al di là delle nostre capacità. Dovremmo avere presenza
mentale e consapevolezza di noi stessi in tutte le posture e, quando è
il momento di sederci in meditazione, lo facciamo.

Sediamo in meditazione per instaurare la serenità e coltivare l'energia
mentale. Non lo facciamo per divertirci con qualcosa di speciale. La
meditazione di visione profonda è la stessa pratica del \emph{samādhi}.
Alcuni dicono: «~Ora ci sediamo in \emph{samādhi}, poi faremo
meditazione di visione profonda.~» Non dividetele in questo modo! La
tranquillità è il fondamento che fa sorgere la saggezza; la saggezza è
il frutto della tranquillità. Dire che ora facciamo meditazione di
tranquillità e poi faremo meditazione di visione profonda \ldots{} ma farlo è
impossibile! Possono essere separate solo a parole. È proprio come un
coltello, che ha la lama da un lato e il retro della lama dall'altro.
Non potete dividerle. Se prendete un lato, li prendete entrambi. Allo
stesso modo è la tranquillità a far sorgere la saggezza.

La moralità è il padre e la madre del Dhamma. All'inizio dobbiamo avere
moralità. Moralità è pace. Questo significa che non si commettono
cattive azioni con il corpo o con la parola. Quando non facciamo cose
sbagliate, non ci agitiamo, e quando non ci agitiamo la pace e il
raccoglimento sorgono nella mente. Per questo diciamo che moralità,
concentrazione e saggezza sono il Sentiero verso l'Illuminazione
percorso da tutti gli Esseri Nobili. Sono tutte quante una sola cosa.
Moralità è concentrazione, concentrazione è moralità. Concentrazione è
saggezza, saggezza è concentrazione. È come un mango. Quando è un fiore,
lo chiamiamo fiore. Quando diventa un frutto, lo chiamiamo mango. Quando
matura, lo chiamiamo mango maturo. Il~tutto è un mango, però cambia
continuamente. Il grande mango cresce dal mango piccolo, il piccolo
mango diventa un grande mango. Li potete considerare frutti differenti
oppure uno solo. Moralità, concentrazione e saggezza sono in relazione
in questo modo. Alla fine tutto è un sentiero che conduce
all'Illuminazione.

Il mango, dal momento in cui all'inizio appare come fiore, semplicemente
cresce in direzione della maturazione. È sufficiente. Dovremmo vedere le
cose in questo modo. Comunque gli altri lo chiamino, non importa. Appena
nasce cresce verso la vecchiaia, e poi verso quale direzione? È così che
dovremmo contemplare. Alcuni non vorrebbero invecchiare. Quando
invecchiano si deprimono. Queste persone non dovrebbero mangiare mango
maturi! Perché vogliamo che i mango siano maturi? Se non maturano
presto, li facciamo maturare artificialmente, vero? Quando però
diventiamo anziani siamo pieni di rimpianti. Alcuni piangono: temono
d'invecchiare o di morire. Se è così, non dovrebbero mangiare mango
maturi, meglio che mangino solo i fiori! Se riusciamo a vedere le cose
in questo modo, possiamo vedere il Dhamma. Tutto diventa chiaro, siamo
in pace. Dovete solo decidervi a praticare in questo modo.

Oggi il Presidente del Privy Council è venuto con i suoi collaboratori
per ascoltare il Dhamma. Dovreste prendere ciò che ho detto e
contemplarlo. Se qualcosa non è giusto, vi prego di scusarmi. Sapere se
è giusto o sbagliato dipende però dalla vostra pratica e dal vedere da
voi stessi. Se è sbagliato, sbarazzatevene. Se è giusto, prendetelo e
usatelo. In realtà, però, noi pratichiamo per lasciar andare sia quel
che è giusto sia quel che è sbagliato. Alla fine, ci sbarazziamo di
tutto. Se è giusto, sbarazzatevene. Se è sbagliato, sbarazzatevene! Di
solito, se è giusto ci aggrappiamo alla rettitudine, se è sbagliato lo
riteniamo sbagliato e poi seguono discussioni. Il Dhamma è però il posto
in cui non c'è nulla, proprio nulla.

