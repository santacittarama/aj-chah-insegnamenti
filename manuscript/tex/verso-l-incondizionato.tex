\chapter{Verso l'incondizionato}

\begin{openingQuote}
  \centering

  Discorso offerto durante una notte d'osservanza lunare (uposatha) al
  Wat Pah Pong nel 1976.
\end{openingQuote}

Oggi è l'\emph{uposatha},\footnote{\emph{uposatha.} Giorno di osservanza
  lunare, corrispondente alle fasi lunari.} il giorno in cui, secondo la
tradizione buddhista, ci riuniamo per osservare i precetti e per
ascoltare il Dhamma. Ascoltare il Dhamma serve in primo luogo per capire
un po' le cose non ancora comprese, per chiarirle e, in secondo luogo,
per afferrare ancor meglio le cose già comprese. Dobbiamo affidarci ai
discorsi di Dhamma per migliorare la nostra comprensione, e l'ascolto
rappresenta l'elemento essenziale. Prestate un'attenzione particolare al
discorso di oggi. Prima di tutto, raddrizzate la vostra postura per
renderla adatta all'ascolto. Non siate però troppo tesi. Ora, tutto quel
che vi resta da fare è rendere ferma la mente, stabilizzarla nel
\emph{samādhi}. La mente è l'elemento importante. La mente è ciò che
percepisce il bene e il male, quel che è giusto e quel che è sbagliato.
Se manchiamo di \emph{sati} anche per un solo minuto, siamo folli per un
minuto. Se mancheremo di \emph{sati} per mezz'ora, saremo folli per
mezz'ora. Più \emph{sati} manca alla nostra mente, più siamo folli. Ecco
perché è molto importante prestare attenzione quando ascoltiamo il
Dhamma.

Tutte le creature di questo mondo sono afflitte dalla sofferenza. C'è
solo la sofferenza che disturba la mente. Lo scopo dello studio del
Dhamma è di distruggere del tutto questa sofferenza. Se la sofferenza
sorge, è perché non la conosciamo davvero. Non importa quanto si cerchi
di controllarla con la forza di volontà o mediante la ricchezza e i
possedimenti, controllarla è impossibile. Se non comprendiamo a fondo la
sofferenza e ciò che la causa, non conta quanto si cerchi di
``barattarla'' con le nostre azioni, pensieri e ricchezze mondane, non
c'è modo di riuscire a vincerla. Solo con la chiara conoscenza e la
consapevolezza, con la comprensione della Verità, la sofferenza può
scomparire. E questo vale non solo per chi vive senza una fissa dimora,
i monaci e i novizi, ma anche per chi vive in famiglia. Per chiunque
conosca la Verità delle cose, la sofferenza cessa automaticamente.

Bene e male, in quanto condizioni, sono verità costanti. Dhamma
significa ciò che è costante, che si mantiene da sé. L'agitazione
sostiene l'agitazione, la serenità sostiene la serenità. Bene e male
sostengono le loro rispettive condizioni. Come l'acqua bollente, che
conserva il suo calore, non cambia per nessuno. È bollente sia che la
beva un giovane sia che la beva un anziano. È bollente indipendentemente
dalla nazionalità delle persone. Perciò il Dhamma è definito come ciò
che conserva la propria condizione. Nella nostra pratica dobbiamo
conoscere caldo e freddo, giusto e sbagliato, bene e male. Se conosciamo
il male, ad esempio, non creeremo le cause per il male, e il male non
sorgerà. I praticanti di Dhamma dovrebbero conoscere la fonte dei vari
dhamma. Estinguendo la causa del calore, il calore non sorgerà.
Per il male è la stessa cosa: sorge da una causa. Se pratichiamo il
Dhamma fino a quando conosciamo il Dhamma, conosceremo la fonte delle
cose, la causa di esse. Se estinguiamo la causa del male, pure il male è
estinto, non abbiamo bisogno di correre dietro al male per estinguerlo.

Questa è la pratica del Dhamma. Però, molti il Dhamma lo studiano, lo
imparano e perfino lo praticano, ma non sono ancora con il Dhamma e non
hanno ancora spento la causa del male e dell'agitazione all'interno del
loro cuore. Se la causa del calore è ancora presente, non possiamo
evitare in alcun modo che lì sorga il calore. Allo stesso modo, se la
causa della confusione è dentro la nostra mente, non possiamo evitare in
alcun modo che lì ci sia confusione, perché essa sorge da questa fonte.
Finché la fonte non è estinta, la confusione sorgerà ancora.

Tutte le volte che compiamo buone azioni, nella mente sorge la bontà.
Sorge dalla sua causa. Questo si chiama \emph{kusala}.\footnote{\emph{kusala.}
  Salutare, abile, buono, meritorio.} Se comprendiamo le cause in questo
modo, quelle cause possiamo crearle e i risultati seguiranno in modo
naturale. La gente, però, di solito non crea le cause giuste. Vogliono
davvero la bontà, e tuttavia non lavorano per farla nascere. Ottengono
solo cattivi risultati, avviluppano la mente nella sofferenza.
Oggigiorno tutto ciò che la gente vuole è il denaro. Pensano che se
riusciranno ad averne abbastanza ogni cosa andrà bene, e così utilizzano
tutto il loro tempo per cercare denaro, non cercano la bontà. È come
volere la carne senza avere il sale per conservarla. Potete solo
lasciare la carne in casa a imputridire. Quelli che vogliono il denaro
dovrebbero sapere non solo come procurarselo, ma anche come custodirlo.
Se volete la carne, non potete pensare di acquistarla e di lasciarla in
giro per casa. Imputridirà e basta. Questo modo di pensare è errato. Il
risultato del modo errato di pensare è l'agitazione e la confusione. Il
Buddha insegnò il Dhamma affinché le persone lo mettessero in pratica,
perché lo conoscessero, lo comprendessero e fossero tutt'uno con esso,
per far diventare Dhamma la loro mente. Quando la mente è Dhamma,
raggiunge felicità e appagamento. L'inquietudine del \emph{saṃsāra} è in
questo mondo, e anche la cessazione della sofferenza è in questo mondo.

La pratica del Dhamma serve per far trascendere la sofferenza alla
mente. Il corpo non può trascendere la sofferenza: è nato e, perciò,
deve sperimentare dolore e malattia, invecchiamento e morte. Solo la
mente può trascendere l'attaccamento, l'aggrapparsi. Tutti gli
insegnamenti del Buddha, che possiamo chiamare
\emph{pariyatti},\footnote{\emph{pariyatti.} Comprensione teorica del Dhamma.}
sono mezzi abili, utili a questo scopo. Il Buddha ci
parlò ad esempio di \emph{upādinnaka}-\emph{saṅkhāra} e di
\emph{anupādinnaka}-\emph{saṅkhāra}, di fenomeni condizionati dotati di
mente e di fenomeni condizionati privi di mente. Cose come gli alberi,
le montagne, i fiumi e così via vengono di solito definite fenomeni
condizionati privi di mente: le cose inanimate. Gli animali, gli esseri
umani e così via sono definiti fenomeni condizionati dotati di mente: le
cose animate. Per la maggior parte degli studenti di Dhamma queste
definizioni sono scontate, ma se considerate questo argomento a fondo,
se considerate che la mente umana è catturata da quello che vede, dai
suoni, dagli odori, dai sapori, dalle sensazioni e dagli stati mentali,
potrete notare che in realtà non c'è nulla che sia privo di mente.
Quando nella mente c'è brama, tutto è dotato di mente.

Studiando il Dhamma senza praticarlo, saremo ignari del suo significato
più profondo. Potremmo ad esempio pensare che i pilastri di questa sala,
i tavoli, le panche e tutte le cose inanimate siano ``prive di mente''.
Guardiamo solo un lato delle cose. Provate però a prendere un martello e
a fracassare una di queste cose e vedrete se sono dotate o prive di
mente! È la nostra stessa mente che si attacca ai tavoli, alle sedie e a
tutte le cose che possediamo, che è partecipe di esse. Fa male perfino
quando si rompe una piccola tazza, perché la mente ``è partecipe'' di
quella tazza. La mente è partecipe di tutto quello che percepiamo come
nostro: gli alberi, le montagne, qualsiasi cosa. Se non è la loro mente,
allora è quella di qualcun altro. Sono tutti ``fenomeni condizionati
dotati di mente'', non ``privi di mente''. Altrettanto vale per il
nostro corpo. Normalmente diremmo che il corpo è ``dotato di mente''. La
mente che è partecipe del corpo non è nient'altro che \emph{upādāna:}
impossessarsi del corpo e attaccarsi a esso come se fosse ``io'' e
``mio''. Proprio come un cieco non può immaginare i colori -- non
importa dove guardi, i colori non può vederli -- altrettanto avviene per
la mente bloccata dalla brama e ostruita dall'illusione. Tutti gli
oggetti della consapevolezza diventano dotati di mente. Per la mente
contaminata dalla brama e ostruita dall'illusione, tutto diventa dotato
di mente. I tavoli, le sedie, gli animali e qualsiasi altra cosa. Se
pensiamo che esista un sé intrinseco, la mente si attacca a ogni cosa.
Tutto quel che è natura diventa partecipe della mente, c'è sempre
attaccamento, l'afferrarsi a qualcosa.

Il Buddha parlò di \emph{saṅkhata-dhamma} e di \emph{asaṅkhata-dhamma:}
cose condizionate e cose incondizionate. Quelle condizionate sono
innumerevoli: materiali e immateriali, grandi e piccole. Se la nostra
mente è sotto l'influsso dell'illusione, prolifererà a proposito di
queste cose, suddividendole in buone e cattive, corte e lunghe,
grossolane e raffinate. Perché la mente prolifera in questo modo? Perché
non conosce la realtà convenzionale (\emph{sammutisacca}), non vede il
Dhamma. Non vedendo il Dhamma, la mente è colma di attaccamento. Finché
è pressata dall'attaccamento, non c'è via d'uscita, c'è confusione,
nascita, vecchiaia, malattia e morte persino nei processi del pensiero.
Questo tipo di mente è chiamata \emph{saṅkhata-dhamma} (mente
condizionata). L'\emph{asaṅkhata-dhamma}, l'incondizionato, si
riferisce alla mente che ha visto il Dhamma, la verità dei cinque
\emph{khandhā} così come sono: transitori, imperfetti e non sostanziali.
Tutte le idee di ``io'' e ``essi'', ``mio'' e ``loro'', appartengono
alla realtà convenzionale. Sono davvero tutti fenomeni condizionati.
Quando conosciamo la verità dei fenomeni condizionati e determinati, che
non sono noi stessi né ci appartengono, li lasciamo andare. Quando
lasciamo andare i fenomeni condizionati conseguiamo il Dhamma, entriamo
nel Dhamma e lo realizziamo. Quando conseguiamo il Dhamma abbiamo chiara
conoscenza. Che cosa conosciamo? Sappiamo che ci sono solo fenomeni
condizionati e determinazioni, non c'è essere, non c'è ``noi'' né
``essi''. Questa è conoscenza del modo in cui sono le cose.

Con questa comprensione la mente trascende le cose. Il corpo può
invecchiare, ammalarsi e morire, ma la mente trascende queste
situazioni. Quando la mente trascende le condizioni, conosce
l'incondizionato. Diventa l'incondizionato, uno stato che non contiene
più fattori condizionanti. La mente non è più condizionata dalle
preoccupazioni del mondo, le condizioni non la contaminano più. Piacere
e dolore non influiscono più su di essa. Nulla riesce più a influire
sulla mente o a modificarla, essa è al sicuro, sfugge a tutte le
costruzioni. Vedendo la vera natura dei fenomeni condizionati e
determinati, la mente diventa libera. Questa mente liberata è chiamata
``incondizionato'', ciò che si trova al di là della possibilità che su
di essa possano essere esercitati influssi.

Se la mente non conosce davvero i fenomeni condizionati e determinati, è
mossa da essi. Quando incontra buono, cattivo, piacere o dolore,
prolifera su queste cose. Perché prolifera? Perché c'è ancora una causa.
Qual è la causa? La causa è pensare che il corpo sia l'io o appartenga
all'io, che le sensazioni siano l'io o appartengano all'io, che la
percezione sia l'io o appartenga all'io, che il pensiero concettuale sia
l'io o appartenga all'io, che la coscienza sia l'io o appartenga all'io.
La tendenza a concepire le cose in termini di sé è la fonte della
felicità, della sofferenza, della nascita, della vecchiaia, della
malattia e della morte. Questa è la mente mondana, che gira in tondo e
cambia a comando delle condizioni mondane. Questa è la mente
condizionata. Se abbiamo un colpo di fortuna, la nostra mente ne risulta
condizionata. Ciò influisce sulla nostra mente e la conduce a una
sensazione di piacere, ma quando tale sensazione scompare è condizionata
a provare sofferenza. La mente diventa schiava dei fenomeni
condizionati, schiava del desiderio. Non importa cosa il mondo le
presenti, si muove di conseguenza. La mente non ha rifugio, non è ancora
certa di se stessa, non è ancora libera. Manca ancora di una base
stabile. La mente non conosce ancora la verità dei fenomeni
condizionati. Così è la mente condizionata.

Tutti voi che state ascoltando il Dhamma, rifletteteci un po' sopra.
Anche un bambino può farvi arrabbiare, vero? Pure un bambino può
ingannarvi. Può ingannarvi fino a farvi piangere o ridere, può
trascinarvi in qualsiasi genere di situazioni. Persino gli anziani sono
abbindolati da queste cose. La mente di una persona illusa, che non
conosce la verità dei fenomeni condizionati, reagisce sempre assumendo
le innumerevoli forme dell'amore, dell'odio, del piacere e del dolore.
Essi danno forma alla nostra mente in questo modo perché ne siamo
schiavi. Siamo schiavi di \emph{taṇhā}, della brama. La brama impartisce
ordini, e noi ci limitiamo a obbedire.

Sento la gente che si lamenta: «~Oh, sono così infelice. Notte e giorno
devo andare nei campi, non ho tempo per stare a casa. A mezzogiorno devo
lavorare sotto il sole cocente, non all'ombra. Non importa quanto freddo
faccia, non posso restare a casa, devo andare a lavorare. Sono così
angustiato.~» Allora io chiedo: «~Perché non vai via di casa e ti fai
monaco?~» Loro però rispondono:~«~Non posso, ho delle responsabilità.~»
\emph{Taṇhā} li fa desistere. A volte, quando stanno arando, hanno una
tale urgenza di urinare che sentono di scoppiare e arrivano fino al
punto di doverla fare mentre arano, come i bufali! Ecco quanto la brama
li incatena. Chiedo: «~Come va? Non hai avuto tempo per venire in
monastero?~» E loro dicono: «~Oh, sono così indaffarato.~» Non so come
facciano a rimanere così bloccati e coinvolti! Sono solo fenomeni
condizionati, arzigogoli. Il Buddha insegnò a vedere le apparenze come
tali, a vedere i fenomeni condizionati così come sono. Questo è vedere
il Dhamma, vedere le cose come sono in realtà. Se davvero queste cose le
vedete, dovete gettarle via, lasciarle andare.

Non importa cosa possiate ottenere, tutto è privo di reale sostanza.
Inizialmente può sembrare una buona cosa, ma alla fine andrà male. Vi
farà amare e odiare, vi farà ridere e piangere, vi tirerà da tutte le
parti. Perché è così? Perché la mente non è sviluppata. I fenomeni
condizionati diventano fattori che condizionano la mente stessa, la
rendono grande e piccola, felice e triste. Ai tempi dei nostri antenati,
quando uno moriva si invitavano i monaci e si recitavano le
rammemorazioni sull'impermanenza:

\begin{quote}

\emph{Aniccā vata saṅkhāra}

Tutti i fenomeni condizionati sono impermanenti

\emph{Uppāda-vaya-dhammino}

Soggetti a sorgere e a scomparire

\emph{Uppajjitvā nirujjhanti}

Dopo essere sorti, cessano

\emph{Tesaṃ vūpasamo sukho}

\end{quote}

Il loro placarsi è beatitudine.

Tutti i fenomeni condizionati sono impermanenti. Il corpo e la mente
sono entrambi impermanenti. Sono impermanenti perché non rimangono fissi
e immutabili. Tutte le cose che nascono devono necessariamente cambiare,
sono transitorie, soprattutto il nostro corpo. Che cos'è che non cambia
all'interno del corpo? I capelli, le unghie, i denti, la pelle sono
ancora com'erano prima? La condizione del corpo cambia in continuazione,
per questo è impermanente. Il corpo è stabile? La mente è stabile?
Pensateci. Quante volte c'è il sorgere e il cessare anche in un solo
giorno? Sia il corpo sia la mente sorgono e cessano di continuo, i
fenomeni condizionati sono in uno stato di costante agitazione.

La ragione per cui non vedete queste cose in accordo con la Verità, è
perché continuate a credere a ciò che è falso. È come essere guidati da
un cieco. Come potete viaggiare sicuri? Un cieco come potrebbe guidarvi
in un luogo sicuro, se non riesce a vedere? Vi ritroverete tra gli
alberi della foresta e nella boscaglia. Allo stesso modo, la nostra
mente è illusa dai fenomeni condizionati, genera sofferenza mentre è
alla ricerca della felicità, crea complicazioni mentre cerca il
benessere. Una mente così conduce solo a difficoltà e sofferenza.
Vogliamo davvero sbarazzarci della sofferenza e delle difficoltà, e
invece creiamo proprio quelle cose. Tutto quel che possiamo fare è
lamentarci. Creiamo cause non salutari, e la ragione per cui lo facciamo
è perché non conosciamo la verità delle apparenze e dei fenomeni
condizionati.

I fenomeni condizionati sono impermanenti, sia quelli dotati di mente
sia quelli privi di mente. In pratica, i fenomeni condizionati privi di
mente non esistono. Che cos'è che è privo di mente? Perfino il vostro
gabinetto, che voi pensate sia privo di mente: provate a vedere cosa
succede se qualcuno ve lo fracassa con una mazza! Probabilmente costui
dovrà vedersela con le ``autorità''. Tutto è dotato di mente, perfino le
feci e l'urina. Solo per le persone che vedono con chiarezza il modo in
cui le cose sono esistono fenomeni condizionati privi di mente.

Le apparenze pervengono all'esistenza grazie alle definizioni. Perché
dobbiamo definirle? Perché intrinsecamente non esistono. Ad esempio,
supponiamo che qualcuno voglia marcare un confine. Prenderà un pezzo di
legno o una pietra e li posizionerà sul terreno, chiamandoli termini di
confine. In realtà non si tratta di un termine di confine. Non c'è alcun
marcatore, ecco perché dovete definirlo, dovete farlo affinché esista.
Allo stesso modo noi ``definiamo'' città, persone, bestiame, tutto!
Perché li definiamo? Perché originariamente non esistono. Anche concetti
come ``monaco'' e ``laico'' sono ``definizioni''. Definiamo queste cose
per farle esistere, poiché intrinsecamente qui non ci sono. È come un
piatto vuoto. Ci si può mettere dentro tutto quello che vi pare proprio
perché è vuoto. Questa è la natura della realtà convenzionale. Uomini e
donne sono solo concetti, definizioni, come tutto quello che ci
circonda.

Se conoscessimo con chiarezza la verità delle definizioni, sapremmo che
non ci sono esseri, perché ``essere'' è una definizione. Comprendendo
che queste cose sono solo definizioni, potrete avere pace. Però, se
credete che la persona, l'essere, ``mio'', ``loro'' e così via siano
qualità intrinseche, allora dovrete ridere e piangere. Queste sono
proliferazioni di fattori condizionanti. Se noi riteniamo che queste
cose siano nostre, ci sarà sempre sofferenza. Questa è
\emph{micchā-ditti}, errata visione. I nomi non hanno realtà intrinseca,
sono verità provvisorie. Riceviamo un nome solo dopo essere nati, non è
così? Oppure quando siete nati un nome lo avevate già? I nomi vengono
dopo, giusto? Perché dobbiamo ricorrere a questi nomi? Perché
intrinsecamente non ci sono. Queste definizioni dovremmo comprenderle
con chiarezza. Bene, male, alto, basso, nero e bianco sono tutte
definizioni. Ci perdiamo tutti nelle definizioni. Questa è la ragione
per cui nelle cerimonie funebri i monaci cantano \emph{Aniccā vata
saṅkhāra} \ldots{} I fenomeni condizionati sono impermanenti, sorgono e poi
svaniscono. Questa è la verità. Che cos'è che, dopo essere sorto, non
cessa? Il buon umore sorge e poi cessa. Avete mai visto qualcuno
piangere per tre o quattro anni? Al massimo potreste aver visto persone
piangere per una notte intera, ma poi le lacrime si asciugano. Dopo
essere sorte, cessano.

\emph{Tesaṃ vūpasamo sukho.} Comprendere i \emph{saṅkhāra} -- le
proliferazioni -- significa soggiogarli e questa è la più grande
felicità. Vero merito è pervenire all'acquietamento delle
proliferazioni, all'acquietamento dell'``essere'', all'acquietamento
dell'individualità, del fardello del sé. Trascendendo queste cose si
vede l'incondizionato. Quel che avviene non importa, la mente non
prolifera. Non c'è nulla che possa far perdere alla mente il suo
naturale equilibrio. Che altro potreste volere? Questa è la conclusione,
il traguardo.

Il Buddha insegnò il modo in cui sono le cose. Il nostro fare offerte,
ascoltare i discorsi di Dhamma e così via, servono a cercare e a
realizzare tutto questo. Se lo comprendiamo, non dobbiamo andare a
studiare \emph{vipassanā}, avverrà da sé. Sia \emph{samatha} sia
\emph{vipassanā} esistono in quanto definizioni, proprio come tutto il
resto. La mente che conosce, che è al di là di queste cose, è il culmine
della pratica. La nostra pratica, la nostra ricerca, serve a trascendere
la sofferenza. Quando è stato troncato l'attaccamento, sono troncati gli
stati dell'esistenza. Quando gli stati dell'esistenza vengono troncati,
non c'è più nascita né morte. Se le cose stanno andando bene la mente
non si rallegra, se stanno andando male non si affligge. La mente non
viene trascinata da tutte le parti dalle tribolazioni del mondo, e così
la pratica è giunta al termine. Questo è il principio basilare, per
questo il Buddha impartì l'Insegnamento. Il Buddha insegnò il Dhamma
affinché lo usassimo nella nostra vita. L'insegnamento \emph{Tesaṃ
vūpasamo sukho} vale pure per quando si muore. Noi, però, non
soggioghiamo questi fenomeni condizionati, ce li portiamo solo dietro,
come se i monaci ci stessero dicendo di fare così. Li portiamo con noi e
ci piangiamo sopra. Questo significa perdersi nei fenomeni condizionati.
È qui che si trovano il paradiso, l'inferno e il Nibbāna.

Praticare il Dhamma serve a trascendere la sofferenza nella mente. Se
conosciamo la Verità delle cose così come ve l'ho spiegata, conosceremo
automaticamente le Quattro Nobili Verità, la sofferenza, la causa della
sofferenza, la cessazione della sofferenza e il Sentiero che conduce
alla cessazione della sofferenza.

Di solito la gente è ignorante a proposito delle definizioni, pensa che
esistano tutte quante di per sé. I libri semplificano le cose quando ci
dicono che alberi, montagne e fiumi sono fenomeni condizionati privi di
mente. È un insegnamento solo superficiale, non tiene conto della
sofferenza, come se essa nel mondo non ci fosse. Questo è solo
l'involucro del Dhamma. Se dovessimo spiegare le cose nei termini della
Verità ultima, vedremmo che è la gente con i suoi attaccamenti a
escogitare tutto questo. Come potete dire che le cose non hanno il
potere di plasmare gli eventi, che esse non sono dotate di mente, quando
la gente picchia i propri figli anche per un minuscolo ago? Un solo
piatto o una sola tazza, un'asse di legno: la mente esiste in tutte
queste cose. Osservate solo cosa succede se qualcuno ne fracassa una, e
lo scoprirete. Tutto è in grado di esercitare un influsso su di noi in
questo modo. La nostra pratica consiste nel conoscere tutto ciò appieno,
esaminare quelle cose che sono condizionate, quelle che sono
incondizionate, quelle dotate di mente e quelle prive di mente.

È una parte dell'``insegnamento esteriore'', così una volta il Buddha
fece riferimento a tutto questo. Una volta dimorava in una foresta,
prese una manciata di foglie e chiese: «~\emph{Bhikkhu}, sono di più le
foglie che tengo nella mano o quelle sparse al suolo nella foresta?~» I
\emph{bhikkhu} risposero: «~Le foglie che il Beato tiene in mano sono
poche, quelle sparse al suolo nella foresta sono ben di più.~» «~Allo
stesso modo, \emph{bhikkhu}, l'insegnamento complessivo del Buddha è
vasto, ma non è questo l'essenziale, non tutte le cose sono direttamente
correlate alla via d'uscita dalla sofferenza. Ci sono numerosissimi
aspetti nell'insegnamento, ma quello che il \emph{Tathāgata}\footnote{\emph{Tathāgata.}
  Letteralmente, ``così andato'', ``così venuto''.} vuole che facciate è
che trascendiate la sofferenza, che indaghiate le cose e abbandoniate
l'aggrapparsi e l'attaccamento alla forma, alla sensazione, alla
percezione, alla volizione e alla coscienza.\footnote{I cinque
  \emph{khandhā}.} Smettete di attaccarvi a queste cose e trascenderete
la sofferenza. Questi insegnamenti sono come le foglie nella mano del
Buddha. Non ne sono necessari molti, solo pochi sono già sufficienti.
Per quanto concerne i restanti insegnamenti, non dovete preoccuparvi. È
come il nostro pianeta, vasto e abbondante di vegetazione, terra,
montagne e foreste. Non mancano rocce e sassi, ma tutte queste pietre
non valgono quanto un solo gioiello. Così è il Dhamma del Buddha, non ce
n'è bisogno di molto.~»

Perciò, sia che stiate parlando del Dhamma o che lo stiate ascoltando,
dovreste conoscerlo. Non avete bisogno di chiedervi dove sia il Dhamma,
è proprio qui. Non importa dove andiate a studiare il Dhamma, è proprio
nella mente. È la mente che si aggrappa, che fa congetture, che
trascende e che lascia andare. Tutto questo studio esteriore in realtà
riguarda la mente. Non importa che studiate il \emph{Tipiṭaka},
l'\emph{Abhidhamma}\footnote{\emph{Abhidhamma.} Terza parte del Canone
  in pāli, composta di trattati analitici basati su elenchi di categorie
  estratte dai discorsi del Buddha.} o qualsiasi altra cosa. Basta non
dimenticare da dove vengono queste cose.

Per quanto concerne la pratica, le uniche cose di cui avete bisogno per
cominciare sono l'onestà e l'integrità, non dovete preoccuparvi troppo
di altro. Nessuno di voi laici ha studiato il \emph{Tipiṭaka}, ma siete
capaci di provare avidità, collera e illusione, vero? Dove le avete
imparate queste cose? Avete bisogno di leggere il \emph{Tipiṭaka} per
provare avidità, odio e illusione? Queste cose sono già nella vostra
mente, non avete bisogno di studiare i libri per provarle. Gli
insegnamenti servono a indagare e ad abbandonare queste cose. Lasciate
che la conoscenza si diffonda dentro di voi e praticherete
correttamente. Se volete vedere un treno, basta andare alla stazione
centrale, non c'è bisogno di viaggiare per le linee del nord, per le
linee del sud, per quelle dell'est e per quelle dell'ovest, e vedere
tutti i treni. Se volete vedere i treni, ognuno di essi, fareste meglio
ad aspettare nella grande stazione centrale, è lì che vanno tutti a
finire.

Alcuni mi dicono: «~Voglio praticare, ma non so come fare. Non sono in
grado di studiare le Scritture, sto diventando vecchio, la mia memoria
non è buona.~» Guardate solamente qui, nella ``stazione centrale''. Qui
sorge l'avidità, qui sorge la collera, qui sorge l'illusione. Basta
sedersi qui per osservare tutte queste cose che sorgono. Praticate
proprio qui, perché è proprio qui che siete bloccati. È proprio qui che
sorgono le definizioni, che sorgono le convenzioni, ed è proprio qui che
sorgerà il Dhamma.

Per questa ragione la pratica del Dhamma non conosce distinzioni basate
sul ceto sociale o sulla razza. Tutto ciò che chiede è che ci si guardi
dentro, che si veda e si comprenda. Inizialmente addestriamo il corpo e
le parole a essere privi di macchie, ossia a \emph{sīla}. Alcuni pensano
che per avere \emph{sīla} si debbano memorizzare frasi in pāli e
cantarle per tutto il giorno e per tutta la notte, ma in verità tutto
quel che si deve fare è rendere irreprensibili il corpo e le parole,
questo è \emph{sīla}. Non è così difficile da capire, è come cucinare.
Mettete un po' di questo e un po' di quello nella giusta misura, finché
il cibo diventa squisito! Non dovete aggiungere nient'altro per renderlo
squisito, già lo è, bastano solo gli ingredienti giusti. Alla stessa
maniera, fare in modo che le nostre azioni e le nostre parole siano
corrette ci darà \emph{sīla}.

La pratica del Dhamma può essere svolta ovunque. In passato ho viaggiato
dappertutto alla ricerca di un maestro, perché non sapevo come
praticare. Avevo sempre paura di praticare in modo errato. Andai in
continuazione da una montagna all'altra, da un posto all'altro, finché
mi fermai e ci pensai su. Ora capisco. Prima dovevo essere piuttosto
stupido. Sono andato ovunque alla ricerca di posti in cui praticare
meditazione, ma non capivo che il posto era già lì, nel mio cuore. Tutta
la meditazione che volete è proprio lì, dentro di voi. È per questo che
il Buddha disse \emph{paccattaṃ vetitabbo viññūhi:}\footnote{\emph{paccattaṃ.}
  Da sperimentare individualmente e personalmente (\emph{veditabba}) da
  parte dei saggi (\emph{viññūhi}).} il saggio deve conoscere da sé.
Avevo pronunciato già in precedenza queste parole, ma non ne conoscevo
ancora il significato. In questa ricerca ho viaggiato ovunque, ero
pronto a morire di stanchezza. Però ho trovato quel che cercavo solo
quando mi sono fermato, solo allora, e l'ho trovato dentro di me. Per
questo ora posso parlarvene.

Perciò, per la realizzazione di \emph{sīla} praticate solo come vi ho
spiegato. Non dubitate della pratica. Anche se alcuni dicono che non si
può praticare a casa, che ci sono troppi ostacoli, allora anche mangiare
e bere potrebbero essere degli ostacoli. Se queste cose rappresentano un
ostacolo per la pratica, allora non mangiate! Calpestare una spina è una
bella cosa? Non è meglio non calpestarla? La pratica del Dhamma reca
beneficio a tutti, ricchi e poveri. Ovviamente più praticate più
conoscerete la Verità. Alcuni dicono che non riescono a praticare da
laici, che l'ambiente è troppo affollato. Se vivete in un posto
affollato, allora guardate nell'affollamento, rendetelo aperto e vasto.
Se la mente è illusa dall'affollamento, addestratela a conoscere la
verità dell'affollamento. Più trascurate la pratica, più trascurate di
andare in monastero e di ascoltare gli insegnamenti, più la vostra mente
affonderà nella palude, come una rana che finisce in un buco. Qualcuno
arriva con un arpione e la rana è spacciata, non ha scampo. Tutto quel
che può fare è allungare e offrire il collo. Fate attenzione a non
chiudervi in un angolo, qualcuno potrebbe arrivare e arpionarvi. A casa,
assillati dai vostri figli e nipoti, le cose potrebbero andare peggio
che per la rana! Non sapete come staccarvi da queste cose. Quando
sopraggiungeranno la vecchiaia, la malattia e la morte, che farete?
Questo è l'arpione che sta per catturarvi. Dove potrete mai fuggire?

Questa è la difficile situazione in cui si trova la nostra mente.
Riassorbita da figli, parenti e possedimenti, non sa come lasciar andare
tutte queste cose. Senza la moralità e la comprensione, per voi non c'è
scampo. Siete sempre catturati quando la sensazione, la percezione, la
volizione e la coscienza producono sofferenza. Perché questa sofferenza
è lì? Se non investigate, non saprete. Se sorge la felicità, siete
semplicemente catturati dalla felicità, vi crogiolate in essa. Non vi
chiedete: «~Da dove viene questa felicità?~» Modificate la vostra
comprensione. Potete praticare ovunque, perché la mente è con voi
ovunque. Quando sedete, se avete buoni pensieri potete esserne
consapevoli. Anche se avete cattivi pensieri potete esserne consapevoli.
Queste cose sono con voi. Quando siete distesi, se avete buoni o cattivi
pensieri, anche in quel caso potete conoscerli, perché il luogo in cui
praticare è la mente. Alcuni pensano che si debba andare in monastero
tutti i giorni. Non è necessario, basta guardare la mente. Se sapete
dov'è che si pratica, siete al sicuro.

L'insegnamento del Buddha dice di osservare noi stessi, non di correre
dietro a manie e superstizioni. Per questo Egli disse:

\begin{quote}

\emph{Sīlena sugatiṃ yanti}

La rettitudine morale conduce al benessere

\emph{Sīlena bhogasampadā}

La rettitudine morale conduce alla ricchezza

\emph{Sīlena nibbutiṃ yanti}

La rettitudine morale conduce al Nibbāna

\emph{Tasmā sīlaṃ visodhaye}

\end{quote}

Perciò, osservate con purezza i vostri precetti.

\emph{Sīla} si riferisce alle nostre azioni. Le buone azioni recano
buoni risultati, le cattive azioni recano cattivi risultati. Non
attendetevi che gli déi facciamo le cose per voi o che gli angeli e gli
spiriti divini vi proteggano, oppure che dei giorni propizi vi aiutino.
Queste cose non sono vere, non credeteci. Se credete in esse,
soffrirete. Aspetterete sempre il giorno giusto, il mese giusto, l'anno
giusto, gli angeli e le divinità protettrici \ldots{} soffrirete. Guardate
nelle vostre azioni e nelle vostre parole, nel vostro kamma.
Facendo del bene erediterete bontà, facendo del male erediterete
malvagità. Se comprendete che bene e male, giusto e sbagliato stanno
dentro di voi, allora non dovrete andare a cercare queste cose da
qualche altra parte. Cercate queste cose laddove sorgono. Se perdete una
cosa qui, è qui che dovete cercare. Anche se all'inizio non la trovate,
continuate a cercare dove vi è caduta. Di solito, però, la perdiamo qui
e cerchiamo là. Quando potremo mai trovarla? Buone e cattive azioni
stanno con voi. Siete destinati a vederle prima o poi, continuate solo a
guardare proprio lì.

Tutti gli esseri vivono a seconda del loro kamma. Che cos'è il
kamma? La gente è troppo credulona. Se si commettono cattive
azioni, dicono che Yāma, il re della morte e degli inferi, le scriverà
tutte in un libro. Quando si andrà lì, egli tirerà fuori i suoi appunti
e cercherà il vostro nome. Avete tutti paura di Yāma nell'aldilà, ma non
conoscete quello Yāma che si trova nella vostra mente. Se commettete
cattive azioni, anche se vi nascondete e fate tutto da soli, questo Yāma
scriverà tutto. Probabilmente tra voi che siete seduti qui sono molti
quelli che hanno fatto brutte cose in segreto, senza che nessuno li
abbia visti. Voi stessi le vedete, però, o no? Questo Yāma vede tutto.
Riuscite a capirlo da soli? Tutti voi, pensateci per un po' \ldots{} Yāma ha
scritto tutto, non è vero? Non c'è possibilità di fuga. Sia che lo
facciate da soli o in gruppo, nei campi o in qualsiasi altro posto.

Fra voi c'è qualcuno che ha rubato qualcosa? Forse alcuni di noi sono
stati ladri in passato. Anche se non rubate cose di altre persone, forse
rubate ancora le vostre. Io stesso ho questa tendenza, ecco perché
suppongo che fra voi qualcuno faccia lo stesso. Forse avete segretamente
fatto cattive azioni in passato, senza che nessun altro lo sappia. Però,
anche se non lo raccontate a nessuno, voi lo sapete. Questo è quello
Yāma che vi sorveglia e scrive tutto. Ovunque andiate, scrive tutto nel
suo libro. Le nostre intenzioni noi le conosciamo. Quando commettiamo
cattive azioni, la malvagità è lì. Se fate buone azioni, la bontà è lì.
Non c'è luogo in cui possiate nascondervi. Anche se gli altri non vi
vedono, voi siete costretti a vedere voi stessi. Anche se andate in un
nascondiglio remoto, lì troverete ancora voi stessi. Non c'è modo di
commettere cattive azioni e farla franca. Allo stesso modo, perché non
dovreste vedere la vostra purezza? Vedete tutto -- la serenità,
l'agitazione, la liberazione o la schiavitù -- vedete tutto da voi
stessi.

Nella religione buddhista dovete essere consapevoli di tutte le vostre
azioni. Noi non facciamo come i brāhmaṇi, che entrano nelle vostre case
e dicono: «~Che tu possa stare bene ed essere forte, che tu possa vivere
a lungo.~» Il Buddha non parlava in questa maniera. Come si fa a guarire
una malattia con le sole parole? Il modo che il Buddha ha di curare un
malato consiste nel dire: «~Cos'è successo prima che ti ammalassi?~Cosa
ti ha fatto ammalare?~» Voi allora gli dite che cosa è successo. «~Ah, è
così, vero?~Prendi questa medicina e provala.~» Se non è la medicina
giusta, ne prova un'altra. Se è adatta alla malattia, allora è quella
giusta. Questo è il modo scientificamente giusto. Invece i brāhmaṇi vi
mettono un braccialetto attorno al polso e dicono: «~Ecco, guarisci, sii
forte, e quando uscirò di qui ti alzerai, farai un pasto abbondante e
starai bene.~» Non importa quanto abbiate pagato, la vostra malattia non
guarirà, perché il loro modo di curare non ha basi scientifiche. Però,
questo è quel che la gente vuole credere.

Il Buddha non voleva che confidassimo troppo in queste cose, voleva che
praticassimo con la ragione. Il buddhismo è esistito per migliaia di
anni, e la maggior parte delle persone ha continuato a praticare
seguendo l'insegnamento dei propri maestri, indipendentemente dal fatto
che fosse giusto o sbagliato. È una cosa sciocca. Seguivano solo
l'esempio dei loro predecessori. Il Buddha non incoraggiava questo modo
di fare. Voleva che facessimo le cose ragionevolmente. Una volta, mentre
stava insegnando ai monaci, chiese al venerabile Sāriputta: «~Sāriputta,
credi a questo insegnamento?~» Il venerabile Sāriputta rispose: «~Non ci
credo ancora.~» Il Buddha lodò la sua risposta: «~Molto bene, Sāriputta.
Un saggio non crede troppo in fretta. Guarda dentro le cose, nelle loro
cause e nelle loro condizioni, e comprende la loro vera natura prima di
credere o di non credere.~» Però, la maggior parte degli insegnanti
oggigiorno direbbe: «~Cosa?! Non mi credi? Fuori di qui!~» La maggior
parte della gente ha timore dei propri insegnanti. Qualsiasi cosa i loro
insegnanti facciano, li segue ciecamente. Il Buddha insegnò ad aderire
alla Verità. Ascoltate gli insegnamenti e considerateli con
intelligenza, indagateli. Fate lo stesso con i miei discorsi di Dhamma:
esaminateli. Quel che dico è giusto? Guardateci dentro davvero, e
guardate dentro voi stessi.

Per questa ragione è stato detto che dobbiamo custodire la mente.
Chiunque custodisca la propria mente si libererà dalle catene di Māra. È
solo questa mente che va ad afferrare le cose, conosce le cose, vede le
cose, sperimenta felicità e sofferenza: proprio e solo questa mente.
Quando conosceremo appieno la verità delle definizioni e dei fenomeni
condizionati, ci sbarazzeremo della sofferenza in modo naturale.

Tutte le cose sono solo così come sono. Di per sé non causano
sofferenza: proprio come una spina, una spina davvero appuntita. Vi fa
soffrire? No, è solo una spina, non dà fastidio a nessuno. Se la
calpesterete, allora sì che soffrirete. Perché c'è sofferenza? Perché
avete messo il piede sulla spina. La spina fa solo il suo lavoro, non fa
del male a nessuno. Se ci metterete il piede sopra, vi farà soffrire. È
a causa di noi stessi che c'è il dolore. Forma, sensazione, percezione,
volizione, coscienza: tutte le cose di questo mondo sono semplicemente
così come sono. Siamo noi che attacchiamo briga. E se noi le colpiamo,
esse ci colpiranno a loro volta. Se le lasciamo stare, non daranno
problemi a nessuno. Solo uno spavaldo ubriacone le infastidirà. Tutti i
fenomeni condizionati esistono in accordo con la loro natura. Per questo
il Buddha disse: \emph{Tesaṃ vūpasamo sukho}, il loro placarsi è
beatitudine. Se soggioghiamo i fenomeni condizionati, vedendo le
definizioni e i fenomeni condizionati quali in realtà sono -- e non come
``io'' né come ``mio'', non come ``noi'' né come ``loro'' -- quando
comprendiamo che queste opinioni sono semplicemente \emph{sakkāya
diṭṭhi},\footnote{\emph{sakkāya-diṭṭhi.} Convinzione che induce
  l'identificazione con il sé, con l'io.} i fenomeni condizionati
vengono liberati dall'illusione del sé.

Se pensate ``sono buono'', ``sono cattivo'', ``sono grande'', ``sono il
migliore'', state pensando in modo errato. Se vedete tutti questi
pensieri come mere definizioni e come fenomeni condizionati, quando gli
altri dicono ``buono'' o ``cattivo'' potete lasciare che queste
definizioni restino con loro. Per tutto il tempo che le considerate come
``io'' e ``tu'', sarà come avere tre vespai: appena dite qualcosa le
vespe vi raggiungono ronzando per pungervi. I tre vespai sono
\emph{sakkāya-diṭṭhi}, \emph{vicikicchā}\footnote{\emph{vicikicchā.} Il
  dubbio.} e \emph{sīlabbata-parāmāsa}.\footnote{\emph{sīlabbata-parāmāsa.}
  L'attaccamento ai riti e cerimonie/osservanze.}

Appena guardate nella vera natura delle definizioni e dei fenomeni
condizionati, l'orgoglio non può vincere. I padri degli altri sono
proprio come nostro padre, le loro madri sono proprio come le nostre
madri, i loro figli sono proprio come i nostri figli. Vediamo la
felicità e la sofferenza degli altri esseri proprio come le nostre.
Siamo tutti nella stessa barca. Se vediamo le cose in questo modo,
possiamo vedere faccia a faccia il Buddha del futuro, non è poi così
difficile. Tutto sarà liscio e levigato come la pelle di un tamburo. Se
volete stare ad attendere di incontrare Phra Sri Ariya Metteya, il
Buddha del futuro, allora non praticate. Forse vivrete abbastanza a
lungo per vederlo. Lui, però, non è così folle da scegliere persone di
questo genere come discepoli! La maggior parte della gente dubita e
basta. Se non avete più dubbi a proposito del sé, allora non importa
cosa la gente possa dire di voi, non ve ne preoccupate, perché la vostra
mente ha lasciato andare, è in pace. I fenomeni condizionati sono stati
soggiogati. Attaccarsi alle forme della pratica, quel maestro è cattivo,
quel posto non va bene, questo è giusto, quello è sbagliato \ldots{} No. Non
c'è più nessuna di queste cose. Tutti questi modi di pensare sono
annullati. Arrivate faccia a faccia con il Buddha del futuro. Coloro che
si limitano ad alzare le mani e pregare non ci arriveranno mai.

La pratica è così. Se avessi parlato di più, avrei parlato di più della
stessa cosa. Un altro discorso sarebbe stato uguale a questo. Vi ho
condotti fino a questo punto, ora pensateci su. Vi ho portati fino al
Sentiero, chiunque stia per percorrerlo sappia che è lì per voi. Chi non
vuole, può fermarsi. Il Buddha vi accompagna solo all'inizio del
Sentiero. \emph{Akkhātāro Tathāgata:} il \emph{Tathāgata} indica solo la
strada. La mia pratica mi ha insegnato solo questo. Io posso portarvi
all'inizio del Sentiero. Chiunque voglia tornare indietro può tornare
indietro, chiunque voglia percorrerlo può percorrerlo. Ora dipende da
voi.

