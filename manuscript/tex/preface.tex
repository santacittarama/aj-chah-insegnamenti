\chapter{Prefazione}

Gli insegnamenti del venerabile Ajahn Chah tradotti e resi fruibili in
questa edizione sono diretti e chiari. Mi dà molta gioia sapere che una
tale saggezza stia per essere ampiamente diffusa.

Ho avuto la grande fortuna di vivere con Ajahn Chah o di stargli vicino
tra il 1967 e il 1977, gli anni centrali del suo insegnamento. Dopo aver
ricevuto l'ordinazione nel maggio del 1967 nella Thailandia del
nord-est, nella provincia di Nong Khai, il mio precettore mi inviò al
Wat Nong Pah Pong per la formazione. Fu durante il mio primo ritiro
delle piogge (\emph{vassa}) in quel monastero, mentre vivevo sotto la
guida di Ajahn Chah, che davvero crebbero la mia fede e la mia fiducia
nei riguardi di questo modo di praticare. Durante quei dieci anni ho
avuto l'opportunità di studiare e di comprendere la relazione tra il
Dhamma e il Vinaya (la disciplina monastica), di sviluppare una visione
profonda della vacuità e della forma, e di riconoscere la sofferenza
causata dall'ignoranza dei miei attaccamenti ai fenomeni condizionati.

L'approccio di Ajahn Chah all'insegnamento e alla formazione è semplice
e pratico. È uno strumento perfetto per rendere chiare le illusioni del
sé, le presunzioni culturali e sociali nonché i processi del nostro
pensiero. Ora i suoi insegnamenti, originariamente registrati, sono
tradotti, disponibili e facilmente raggiungibili. Sono perciò grato sia
a chi ha lavorato alla traduzione e alla compilazione sia agli sponsor
che hanno reso possibile la distribuzione gratuita.

L'insegnamento del Buddha è un grande dono, ancor più necessario oggi
per far fronte ai problemi delle società contemporanee. Che questa
raccolta di insegnamenti possa essere di beneficio a molte persone.

\bigskip

{\raggedleft
  Luang Por Sumedho,\\
  novembre 2010
\par}

